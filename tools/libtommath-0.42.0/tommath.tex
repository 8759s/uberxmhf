\documentclass[b5paper]{book}
\usepackage{hyperref}
\usepackage{makeidx}
\usepackage{amssymb}
\usepackage{color}
\usepackage{alltt}
\usepackage{graphicx}
\usepackage{layout}
\def\union{\cup}
\def\intersect{\cap}
\def\getsrandom{\stackrel{\rm R}{\gets}}
\def\cross{\times}
\def\cat{\hspace{0.5em} \| \hspace{0.5em}}
\def\catn{$\|$}
\def\divides{\hspace{0.3em} | \hspace{0.3em}}
\def\nequiv{\not\equiv}
\def\approx{\raisebox{0.2ex}{\mbox{\small $\sim$}}}
\def\lcm{{\rm lcm}}
\def\gcd{{\rm gcd}}
\def\log{{\rm log}}
\def\ord{{\rm ord}}
\def\abs{{\mathit abs}}
\def\rep{{\mathit rep}}
\def\mod{{\mathit\ mod\ }}
\renewcommand{\pmod}[1]{\ ({\rm mod\ }{#1})}
\newcommand{\floor}[1]{\left\lfloor{#1}\right\rfloor}
\newcommand{\ceil}[1]{\left\lceil{#1}\right\rceil}
\def\Or{{\rm\ or\ }}
\def\And{{\rm\ and\ }}
\def\iff{\hspace{1em}\Longleftrightarrow\hspace{1em}}
\def\implies{\Rightarrow}
\def\undefined{{\rm ``undefined"}}
\def\Proof{\vspace{1ex}\noindent {\bf Proof:}\hspace{1em}}
\let\oldphi\phi
\def\phi{\varphi}
\def\Pr{{\rm Pr}}
\newcommand{\str}[1]{{\mathbf{#1}}}
\def\F{{\mathbb F}}
\def\N{{\mathbb N}}
\def\Z{{\mathbb Z}}
\def\R{{\mathbb R}}
\def\C{{\mathbb C}}
\def\Q{{\mathbb Q}}
\definecolor{DGray}{gray}{0.5}
\newcommand{\emailaddr}[1]{\mbox{$<${#1}$>$}}
\def\twiddle{\raisebox{0.3ex}{\mbox{\tiny $\sim$}}}
\def\gap{\vspace{0.5ex}}
\makeindex
\begin{document}
\frontmatter
\pagestyle{empty}
\title{Multi--Precision Math}
\author{\mbox{
%\begin{small}
\begin{tabular}{c}
Tom St Denis \\
Algonquin College \\
\\
Mads Rasmussen \\
Open Communications Security \\
\\
Greg Rose \\
QUALCOMM Australia \\
\end{tabular}
%\end{small}
}
}
\maketitle
This text has been placed in the public domain.  This text corresponds to the v0.39 release of the 
LibTomMath project.

\begin{alltt}
Tom St Denis
111 Banning Rd
Ottawa, Ontario
K2L 1C3
Canada

Phone: 1-613-836-3160
Email: tomstdenis@gmail.com
\end{alltt}

This text is formatted to the international B5 paper size of 176mm wide by 250mm tall using the \LaTeX{} 
{\em book} macro package and the Perl {\em booker} package.

\tableofcontents
\listoffigures
\chapter*{Prefaces}
When I tell people about my LibTom projects and that I release them as public domain they are often puzzled.  
They ask why I did it and especially why I continue to work on them for free.  The best I can explain it is ``Because I can.''  
Which seems odd and perhaps too terse for adult conversation. I often qualify it with ``I am able, I am willing.'' which 
perhaps explains it better.  I am the first to admit there is not anything that special with what I have done.  Perhaps
others can see that too and then we would have a society to be proud of.  My LibTom projects are what I am doing to give 
back to society in the form of tools and knowledge that can help others in their endeavours.

I started writing this book because it was the most logical task to further my goal of open academia.  The LibTomMath source
code itself was written to be easy to follow and learn from.  There are times, however, where pure C source code does not
explain the algorithms properly.  Hence this book.  The book literally starts with the foundation of the library and works
itself outwards to the more complicated algorithms.  The use of both pseudo--code and verbatim source code provides a duality
of ``theory'' and ``practice'' that the computer science students of the world shall appreciate.  I never deviate too far
from relatively straightforward algebra and I hope that this book can be a valuable learning asset.

This book and indeed much of the LibTom projects would not exist in their current form if it was not for a plethora
of kind people donating their time, resources and kind words to help support my work.  Writing a text of significant
length (along with the source code) is a tiresome and lengthy process.  Currently the LibTom project is four years old,
comprises of literally thousands of users and over 100,000 lines of source code, TeX and other material.  People like Mads and Greg 
were there at the beginning to encourage me to work well.  It is amazing how timely validation from others can boost morale to 
continue the project. Definitely my parents were there for me by providing room and board during the many months of work in 2003.  

To my many friends whom I have met through the years I thank you for the good times and the words of encouragement.  I hope I
honour your kind gestures with this project.

Open Source.  Open Academia.  Open Minds.

\begin{flushright} Tom St Denis \end{flushright}

\newpage
I found the opportunity to work with Tom appealing for several reasons, not only could I broaden my own horizons, but also 
contribute to educate others facing the problem of having to handle big number mathematical calculations.

This book is Tom's child and he has been caring and fostering the project ever since the beginning with a clear mind of 
how he wanted the project to turn out. I have helped by proofreading the text and we have had several discussions about 
the layout and language used.

I hold a masters degree in cryptography from the University of Southern Denmark and have always been interested in the 
practical aspects of cryptography. 

Having worked in the security consultancy business for several years in S\~{a}o Paulo, Brazil, I have been in touch with a 
great deal of work in which multiple precision mathematics was needed. Understanding the possibilities for speeding up 
multiple precision calculations is often very important since we deal with outdated machine architecture where modular 
reductions, for example, become painfully slow.

This text is for people who stop and wonder when first examining algorithms such as RSA for the first time and asks 
themselves, ``You tell me this is only secure for large numbers, fine; but how do you implement these numbers?''

\begin{flushright}
Mads Rasmussen

S\~{a}o Paulo - SP

Brazil
\end{flushright}

\newpage
It's all because I broke my leg. That just happened to be at about the same time that Tom asked for someone to review the section of the book about 
Karatsuba multiplication. I was laid up, alone and immobile, and thought ``Why not?'' I vaguely knew what Karatsuba multiplication was, but not 
really, so I thought I could help, learn, and stop myself from watching daytime cable TV, all at once.

At the time of writing this, I've still not met Tom or Mads in meatspace. I've been following Tom's progress since his first splash on the 
sci.crypt Usenet news group. I watched him go from a clueless newbie, to the cryptographic equivalent of a reformed smoker, to a real
contributor to the field, over a period of about two years. I've been impressed with his obvious intelligence, and astounded by his productivity. 
Of course, he's young enough to be my own child, so he doesn't have my problems with staying awake.

When I reviewed that single section of the book, in its very earliest form, I was very pleasantly surprised. So I decided to collaborate more fully, 
and at least review all of it, and perhaps write some bits too. There's still a long way to go with it, and I have watched a number of close 
friends go through the mill of publication, so I think that the way to go is longer than Tom thinks it is. Nevertheless, it's a good effort, 
and I'm pleased to be involved with it.

\begin{flushright}
Greg Rose, Sydney, Australia, June 2003. 
\end{flushright}

\mainmatter
\pagestyle{headings}
\chapter{Introduction}
\section{Multiple Precision Arithmetic}

\subsection{What is Multiple Precision Arithmetic?}
When we think of long-hand arithmetic such as addition or multiplication we rarely consider the fact that we instinctively
raise or lower the precision of the numbers we are dealing with.  For example, in decimal we almost immediate can 
reason that $7$ times $6$ is $42$.  However, $42$ has two digits of precision as opposed to one digit we started with.  
Further multiplications of say $3$ result in a larger precision result $126$.  In these few examples we have multiple 
precisions for the numbers we are working with.  Despite the various levels of precision a single subset\footnote{With the occasional optimization.}
 of algorithms can be designed to accomodate them.  

By way of comparison a fixed or single precision operation would lose precision on various operations.  For example, in
the decimal system with fixed precision $6 \cdot 7 = 2$.

Essentially at the heart of computer based multiple precision arithmetic are the same long-hand algorithms taught in
schools to manually add, subtract, multiply and divide.  

\subsection{The Need for Multiple Precision Arithmetic}
The most prevalent need for multiple precision arithmetic, often referred to as ``bignum'' math, is within the implementation
of public-key cryptography algorithms.   Algorithms such as RSA \cite{RSAREF} and Diffie-Hellman \cite{DHREF} require 
integers of significant magnitude to resist known cryptanalytic attacks.  For example, at the time of this writing a 
typical RSA modulus would be at least greater than $10^{309}$.  However, modern programming languages such as ISO C \cite{ISOC} and 
Java \cite{JAVA} only provide instrinsic support for integers which are relatively small and single precision.

\begin{figure}[!here]
\begin{center}
\begin{tabular}{|r|c|}
\hline \textbf{Data Type} & \textbf{Range} \\
\hline char  & $-128 \ldots 127$ \\
\hline short & $-32768 \ldots 32767$ \\
\hline long  & $-2147483648 \ldots 2147483647$ \\
\hline long long & $-9223372036854775808 \ldots 9223372036854775807$ \\
\hline
\end{tabular}
\end{center}
\caption{Typical Data Types for the C Programming Language}
\label{fig:ISOC}
\end{figure}

The largest data type guaranteed to be provided by the ISO C programming 
language\footnote{As per the ISO C standard.  However, each compiler vendor is allowed to augment the precision as they 
see fit.}  can only represent values up to $10^{19}$ as shown in figure \ref{fig:ISOC}. On its own the C language is 
insufficient to accomodate the magnitude required for the problem at hand.  An RSA modulus of magnitude $10^{19}$ could be 
trivially factored\footnote{A Pollard-Rho factoring would take only $2^{16}$ time.} on the average desktop computer, 
rendering any protocol based on the algorithm insecure.  Multiple precision algorithms solve this very problem by 
extending the range of representable integers while using single precision data types.

Most advancements in fast multiple precision arithmetic stem from the need for faster and more efficient cryptographic 
primitives.  Faster modular reduction and exponentiation algorithms such as Barrett's algorithm, which have appeared in 
various cryptographic journals, can render algorithms such as RSA and Diffie-Hellman more efficient.  In fact, several 
major companies such as RSA Security, Certicom and Entrust have built entire product lines on the implementation and 
deployment of efficient algorithms.

However, cryptography is not the only field of study that can benefit from fast multiple precision integer routines.  
Another auxiliary use of multiple precision integers is high precision floating point data types.  
The basic IEEE \cite{IEEE} standard floating point type is made up of an integer mantissa $q$, an exponent $e$ and a sign bit $s$.  
Numbers are given in the form $n = q \cdot b^e \cdot -1^s$ where $b = 2$ is the most common base for IEEE.  Since IEEE 
floating point is meant to be implemented in hardware the precision of the mantissa is often fairly small 
(\textit{23, 48 and 64 bits}).  The mantissa is merely an integer and a multiple precision integer could be used to create
a mantissa of much larger precision than hardware alone can efficiently support.  This approach could be useful where 
scientific applications must minimize the total output error over long calculations.

Yet another use for large integers is within arithmetic on polynomials of large characteristic (i.e. $GF(p)[x]$ for large $p$).
In fact the library discussed within this text has already been used to form a polynomial basis library\footnote{See \url{http://poly.libtomcrypt.org} for more details.}.

\subsection{Benefits of Multiple Precision Arithmetic}
\index{precision}
The benefit of multiple precision representations over single or fixed precision representations is that 
no precision is lost while representing the result of an operation which requires excess precision.  For example, 
the product of two $n$-bit integers requires at least $2n$ bits of precision to be represented faithfully.  A multiple 
precision algorithm would augment the precision of the destination to accomodate the result while a single precision system 
would truncate excess bits to maintain a fixed level of precision.

It is possible to implement algorithms which require large integers with fixed precision algorithms.  For example, elliptic
curve cryptography (\textit{ECC}) is often implemented on smartcards by fixing the precision of the integers to the maximum 
size the system will ever need.  Such an approach can lead to vastly simpler algorithms which can accomodate the 
integers required even if the host platform cannot natively accomodate them\footnote{For example, the average smartcard 
processor has an 8 bit accumulator.}.  However, as efficient as such an approach may be, the resulting source code is not
normally very flexible.  It cannot, at runtime, accomodate inputs of higher magnitude than the designer anticipated.

Multiple precision algorithms have the most overhead of any style of arithmetic.  For the the most part the 
overhead can be kept to a minimum with careful planning, but overall, it is not well suited for most memory starved
platforms.  However, multiple precision algorithms do offer the most flexibility in terms of the magnitude of the 
inputs.  That is, the same algorithms based on multiple precision integers can accomodate any reasonable size input 
without the designer's explicit forethought.  This leads to lower cost of ownership for the code as it only has to 
be written and tested once.

\section{Purpose of This Text}
The purpose of this text is to instruct the reader regarding how to implement efficient multiple precision algorithms.  
That is to not only explain a limited subset of the core theory behind the algorithms but also the various ``house keeping'' 
elements that are neglected by authors of other texts on the subject.  Several well reknowned texts \cite{TAOCPV2,HAC} 
give considerably detailed explanations of the theoretical aspects of algorithms and often very little information 
regarding the practical implementation aspects.  

In most cases how an algorithm is explained and how it is actually implemented are two very different concepts.  For 
example, the Handbook of Applied Cryptography (\textit{HAC}), algorithm 14.7 on page 594, gives a relatively simple 
algorithm for performing multiple precision integer addition.  However, the description lacks any discussion concerning 
the fact that the two integer inputs may be of differing magnitudes.  As a result the implementation is not as simple
as the text would lead people to believe.  Similarly the division routine (\textit{algorithm 14.20, pp. 598}) does not 
discuss how to handle sign or handle the dividend's decreasing magnitude in the main loop (\textit{step \#3}).

Both texts also do not discuss several key optimal algorithms required such as ``Comba'' and Karatsuba multipliers 
and fast modular inversion, which we consider practical oversights.  These optimal algorithms are vital to achieve 
any form of useful performance in non-trivial applications.  

To solve this problem the focus of this text is on the practical aspects of implementing a multiple precision integer
package.  As a case study the ``LibTomMath''\footnote{Available at \url{http://math.libtomcrypt.com}} package is used 
to demonstrate algorithms with real implementations\footnote{In the ISO C programming language.} that have been field 
tested and work very well.  The LibTomMath library is freely available on the Internet for all uses and this text 
discusses a very large portion of the inner workings of the library.

The algorithms that are presented will always include at least one ``pseudo-code'' description followed 
by the actual C source code that implements the algorithm.  The pseudo-code can be used to implement the same 
algorithm in other programming languages as the reader sees fit.  

This text shall also serve as a walkthrough of the creation of multiple precision algorithms from scratch.  Showing
the reader how the algorithms fit together as well as where to start on various taskings.  

\section{Discussion and Notation}
\subsection{Notation}
A multiple precision integer of $n$-digits shall be denoted as $x = (x_{n-1}, \ldots, x_1, x_0)_{ \beta }$ and represent
the integer $x \equiv \sum_{i=0}^{n-1} x_i\beta^i$.  The elements of the array $x$ are said to be the radix $\beta$ digits 
of the integer.  For example, $x = (1,2,3)_{10}$ would represent the integer 
$1\cdot 10^2 + 2\cdot10^1 + 3\cdot10^0 = 123$.  

\index{mp\_int}
The term ``mp\_int'' shall refer to a composite structure which contains the digits of the integer it represents, as well 
as auxilary data required to manipulate the data.  These additional members are discussed further in section 
\ref{sec:MPINT}.  For the purposes of this text a ``multiple precision integer'' and an ``mp\_int'' are assumed to be 
synonymous.  When an algorithm is specified to accept an mp\_int variable it is assumed the various auxliary data members 
are present as well.  An expression of the type \textit{variablename.item} implies that it should evaluate to the 
member named ``item'' of the variable.  For example, a string of characters may have a member ``length'' which would 
evaluate to the number of characters in the string.  If the string $a$ equals ``hello'' then it follows that 
$a.length = 5$.  

For certain discussions more generic algorithms are presented to help the reader understand the final algorithm used
to solve a given problem.  When an algorithm is described as accepting an integer input it is assumed the input is 
a plain integer with no additional multiple-precision members.  That is, algorithms that use integers as opposed to 
mp\_ints as inputs do not concern themselves with the housekeeping operations required such as memory management.  These 
algorithms will be used to establish the relevant theory which will subsequently be used to describe a multiple
precision algorithm to solve the same problem.  

\subsection{Precision Notation}
The variable $\beta$ represents the radix of a single digit of a multiple precision integer and 
must be of the form $q^p$ for $q, p \in \Z^+$.  A single precision variable must be able to represent integers in 
the range $0 \le x < q \beta$ while a double precision variable must be able to represent integers in the range 
$0 \le x < q \beta^2$.  The extra radix-$q$ factor allows additions and subtractions to proceed without truncation of the 
carry.  Since all modern computers are binary, it is assumed that $q$ is two.

\index{mp\_digit} \index{mp\_word}
Within the source code that will be presented for each algorithm, the data type \textbf{mp\_digit} will represent 
a single precision integer type, while, the data type \textbf{mp\_word} will represent a double precision integer type.  In 
several algorithms (notably the Comba routines) temporary results will be stored in arrays of double precision mp\_words.  
For the purposes of this text $x_j$ will refer to the $j$'th digit of a single precision array and $\hat x_j$ will refer to 
the $j$'th digit of a double precision array.  Whenever an expression is to be assigned to a double precision
variable it is assumed that all single precision variables are promoted to double precision during the evaluation.  
Expressions that are assigned to a single precision variable are truncated to fit within the precision of a single
precision data type.

For example, if $\beta = 10^2$ a single precision data type may represent a value in the 
range $0 \le x < 10^3$, while a double precision data type may represent a value in the range $0 \le x < 10^5$.  Let
$a = 23$ and $b = 49$ represent two single precision variables.  The single precision product shall be written
as $c \leftarrow a \cdot b$ while the double precision product shall be written as $\hat c \leftarrow a \cdot b$.
In this particular case, $\hat c = 1127$ and $c = 127$.  The most significant digit of the product would not fit 
in a single precision data type and as a result $c \ne \hat c$.  

\subsection{Algorithm Inputs and Outputs}
Within the algorithm descriptions all variables are assumed to be scalars of either single or double precision
as indicated.  The only exception to this rule is when variables have been indicated to be of type mp\_int.  This 
distinction is important as scalars are often used as array indicies and various other counters.  

\subsection{Mathematical Expressions}
The $\lfloor \mbox{ } \rfloor$ brackets imply an expression truncated to an integer not greater than the expression 
itself.  For example, $\lfloor 5.7 \rfloor = 5$.  Similarly the $\lceil \mbox{ } \rceil$ brackets imply an expression
rounded to an integer not less than the expression itself.  For example, $\lceil 5.1 \rceil = 6$.  Typically when 
the $/$ division symbol is used the intention is to perform an integer division with truncation.  For example, 
$5/2 = 2$ which will often be written as $\lfloor 5/2 \rfloor = 2$ for clarity.  When an expression is written as a 
fraction a real value division is implied, for example ${5 \over 2} = 2.5$.  

The norm of a multiple precision integer, for example $\vert \vert x \vert \vert$, will be used to represent the number of digits in the representation
of the integer.  For example, $\vert \vert 123 \vert \vert = 3$ and $\vert \vert 79452 \vert \vert = 5$.  

\subsection{Work Effort}
\index{big-Oh}
To measure the efficiency of the specified algorithms, a modified big-Oh notation is used.  In this system all 
single precision operations are considered to have the same cost\footnote{Except where explicitly noted.}.  
That is a single precision addition, multiplication and division are assumed to take the same time to 
complete.  While this is generally not true in practice, it will simplify the discussions considerably.

Some algorithms have slight advantages over others which is why some constants will not be removed in 
the notation.  For example, a normal baseline multiplication (section \ref{sec:basemult}) requires $O(n^2)$ work while a 
baseline squaring (section \ref{sec:basesquare}) requires $O({{n^2 + n}\over 2})$ work.  In standard big-Oh notation these 
would both be said to be equivalent to $O(n^2)$.  However, 
in the context of the this text this is not the case as the magnitude of the inputs will typically be rather small.  As a 
result small constant factors in the work effort will make an observable difference in algorithm efficiency.

All of the algorithms presented in this text have a polynomial time work level.  That is, of the form 
$O(n^k)$ for $n, k \in \Z^{+}$.  This will help make useful comparisons in terms of the speed of the algorithms and how 
various optimizations will help pay off in the long run.

\section{Exercises}
Within the more advanced chapters a section will be set aside to give the reader some challenging exercises related to
the discussion at hand.  These exercises are not designed to be prize winning problems, but instead to be thought 
provoking.  Wherever possible the problems are forward minded, stating problems that will be answered in subsequent 
chapters.  The reader is encouraged to finish the exercises as they appear to get a better understanding of the 
subject material.  

That being said, the problems are designed to affirm knowledge of a particular subject matter.  Students in particular
are encouraged to verify they can answer the problems correctly before moving on.

Similar to the exercises of \cite[pp. ix]{TAOCPV2} these exercises are given a scoring system based on the difficulty of
the problem.  However, unlike \cite{TAOCPV2} the problems do not get nearly as hard.  The scoring of these 
exercises ranges from one (the easiest) to five (the hardest).  The following table sumarizes the 
scoring system used.

\begin{figure}[here]
\begin{center}
\begin{small}
\begin{tabular}{|c|l|}
\hline $\left [ 1 \right ]$ & An easy problem that should only take the reader a manner of \\
                            & minutes to solve.  Usually does not involve much computer time \\
                            & to solve. \\
\hline $\left [ 2 \right ]$ & An easy problem that involves a marginal amount of computer \\
                     & time usage.  Usually requires a program to be written to \\
                     & solve the problem. \\
\hline $\left [ 3 \right ]$ & A moderately hard problem that requires a non-trivial amount \\
                     & of work.  Usually involves trivial research and development of \\
                     & new theory from the perspective of a student. \\
\hline $\left [ 4 \right ]$ & A moderately hard problem that involves a non-trivial amount \\
                     & of work and research, the solution to which will demonstrate \\
                     & a higher mastery of the subject matter. \\
\hline $\left [ 5 \right ]$ & A hard problem that involves concepts that are difficult for a \\
                     & novice to solve.  Solutions to these problems will demonstrate a \\
                     & complete mastery of the given subject. \\
\hline
\end{tabular}
\end{small}
\end{center}
\caption{Exercise Scoring System}
\end{figure}

Problems at the first level are meant to be simple questions that the reader can answer quickly without programming a solution or
devising new theory.  These problems are quick tests to see if the material is understood.  Problems at the second level 
are also designed to be easy but will require a program or algorithm to be implemented to arrive at the answer.  These
two levels are essentially entry level questions.  

Problems at the third level are meant to be a bit more difficult than the first two levels.  The answer is often 
fairly obvious but arriving at an exacting solution requires some thought and skill.  These problems will almost always 
involve devising a new algorithm or implementing a variation of another algorithm previously presented.  Readers who can
answer these questions will feel comfortable with the concepts behind the topic at hand.

Problems at the fourth level are meant to be similar to those of the level three questions except they will require 
additional research to be completed.  The reader will most likely not know the answer right away, nor will the text provide 
the exact details of the answer until a subsequent chapter.  

Problems at the fifth level are meant to be the hardest 
problems relative to all the other problems in the chapter.  People who can correctly answer fifth level problems have a 
mastery of the subject matter at hand.

Often problems will be tied together.  The purpose of this is to start a chain of thought that will be discussed in future chapters.  The reader
is encouraged to answer the follow-up problems and try to draw the relevance of problems.

\section{Introduction to LibTomMath}

\subsection{What is LibTomMath?}
LibTomMath is a free and open source multiple precision integer library written entirely in portable ISO C.  By portable it 
is meant that the library does not contain any code that is computer platform dependent or otherwise problematic to use on 
any given platform.  

The library has been successfully tested under numerous operating systems including Unix\footnote{All of these
trademarks belong to their respective rightful owners.}, MacOS, Windows, Linux, PalmOS and on standalone hardware such 
as the Gameboy Advance.  The library is designed to contain enough functionality to be able to develop applications such 
as public key cryptosystems and still maintain a relatively small footprint.

\subsection{Goals of LibTomMath}

Libraries which obtain the most efficiency are rarely written in a high level programming language such as C.  However, 
even though this library is written entirely in ISO C, considerable care has been taken to optimize the algorithm implementations within the 
library.  Specifically the code has been written to work well with the GNU C Compiler (\textit{GCC}) on both x86 and ARM 
processors.  Wherever possible, highly efficient algorithms, such as Karatsuba multiplication, sliding window 
exponentiation and Montgomery reduction have been provided to make the library more efficient.  

Even with the nearly optimal and specialized algorithms that have been included the Application Programing Interface 
(\textit{API}) has been kept as simple as possible.  Often generic place holder routines will make use of specialized 
algorithms automatically without the developer's specific attention.  One such example is the generic multiplication 
algorithm \textbf{mp\_mul()} which will automatically use Toom--Cook, Karatsuba, Comba or baseline multiplication 
based on the magnitude of the inputs and the configuration of the library.  

Making LibTomMath as efficient as possible is not the only goal of the LibTomMath project.  Ideally the library should 
be source compatible with another popular library which makes it more attractive for developers to use.  In this case the
MPI library was used as a API template for all the basic functions.  MPI was chosen because it is another library that fits 
in the same niche as LibTomMath.  Even though LibTomMath uses MPI as the template for the function names and argument 
passing conventions, it has been written from scratch by Tom St Denis.

The project is also meant to act as a learning tool for students, the logic being that no easy-to-follow ``bignum'' 
library exists which can be used to teach computer science students how to perform fast and reliable multiple precision 
integer arithmetic.  To this end the source code has been given quite a few comments and algorithm discussion points.  

\section{Choice of LibTomMath}
LibTomMath was chosen as the case study of this text not only because the author of both projects is one and the same but
for more worthy reasons.  Other libraries such as GMP \cite{GMP}, MPI \cite{MPI}, LIP \cite{LIP} and OpenSSL 
\cite{OPENSSL} have multiple precision integer arithmetic routines but would not be ideal for this text for 
reasons that will be explained in the following sub-sections.

\subsection{Code Base}
The LibTomMath code base is all portable ISO C source code.  This means that there are no platform dependent conditional
segments of code littered throughout the source.  This clean and uncluttered approach to the library means that a
developer can more readily discern the true intent of a given section of source code without trying to keep track of
what conditional code will be used.

The code base of LibTomMath is well organized.  Each function is in its own separate source code file 
which allows the reader to find a given function very quickly.  On average there are $76$ lines of code per source
file which makes the source very easily to follow.  By comparison MPI and LIP are single file projects making code tracing
very hard.  GMP has many conditional code segments which also hinder tracing.  

When compiled with GCC for the x86 processor and optimized for speed the entire library is approximately $100$KiB\footnote{The notation ``KiB'' means $2^{10}$ octets, similarly ``MiB'' means $2^{20}$ octets.}
 which is fairly small compared to GMP (over $250$KiB).  LibTomMath is slightly larger than MPI (which compiles to about 
$50$KiB) but LibTomMath is also much faster and more complete than MPI.

\subsection{API Simplicity}
LibTomMath is designed after the MPI library and shares the API design.  Quite often programs that use MPI will build 
with LibTomMath without change. The function names correlate directly to the action they perform.  Almost all of the 
functions share the same parameter passing convention.  The learning curve is fairly shallow with the API provided 
which is an extremely valuable benefit for the student and developer alike.  

The LIP library is an example of a library with an API that is awkward to work with.  LIP uses function names that are often ``compressed'' to 
illegible short hand.  LibTomMath does not share this characteristic.  

The GMP library also does not return error codes.  Instead it uses a POSIX.1 \cite{POSIX1} signal system where errors
are signaled to the host application.  This happens to be the fastest approach but definitely not the most versatile.  In
effect a math error (i.e. invalid input, heap error, etc) can cause a program to stop functioning which is definitely 
undersireable in many situations.

\subsection{Optimizations}
While LibTomMath is certainly not the fastest library (GMP often beats LibTomMath by a factor of two) it does
feature a set of optimal algorithms for tasks such as modular reduction, exponentiation, multiplication and squaring.  GMP 
and LIP also feature such optimizations while MPI only uses baseline algorithms with no optimizations.  GMP lacks a few
of the additional modular reduction optimizations that LibTomMath features\footnote{At the time of this writing GMP
only had Barrett and Montgomery modular reduction algorithms.}.  

LibTomMath is almost always an order of magnitude faster than the MPI library at computationally expensive tasks such as modular
exponentiation.  In the grand scheme of ``bignum'' libraries LibTomMath is faster than the average library and usually  
slower than the best libraries such as GMP and OpenSSL by only a small factor.

\subsection{Portability and Stability}
LibTomMath will build ``out of the box'' on any platform equipped with a modern version of the GNU C Compiler 
(\textit{GCC}).  This means that without changes the library will build without configuration or setting up any 
variables.  LIP and MPI will build ``out of the box'' as well but have numerous known bugs.  Most notably the author of 
MPI has recently stopped working on his library and LIP has long since been discontinued.  

GMP requires a configuration script to run and will not build out of the box.   GMP and LibTomMath are still in active
development and are very stable across a variety of platforms.

\subsection{Choice}
LibTomMath is a relatively compact, well documented, highly optimized and portable library which seems only natural for
the case study of this text.  Various source files from the LibTomMath project will be included within the text.  However, 
the reader is encouraged to download their own copy of the library to actually be able to work with the library.  

\chapter{Getting Started}
\section{Library Basics}
The trick to writing any useful library of source code is to build a solid foundation and work outwards from it.  First, 
a problem along with allowable solution parameters should be identified and analyzed.  In this particular case the 
inability to accomodate multiple precision integers is the problem.  Futhermore, the solution must be written
as portable source code that is reasonably efficient across several different computer platforms.

After a foundation is formed the remainder of the library can be designed and implemented in a hierarchical fashion.  
That is, to implement the lowest level dependencies first and work towards the most abstract functions last.  For example, 
before implementing a modular exponentiation algorithm one would implement a modular reduction algorithm.
By building outwards from a base foundation instead of using a parallel design methodology the resulting project is 
highly modular.  Being highly modular is a desirable property of any project as it often means the resulting product
has a small footprint and updates are easy to perform.  

Usually when I start a project I will begin with the header files.  I define the data types I think I will need and 
prototype the initial functions that are not dependent on other functions (within the library).  After I 
implement these base functions I prototype more dependent functions and implement them.   The process repeats until
I implement all of the functions I require.  For example, in the case of LibTomMath I implemented functions such as 
mp\_init() well before I implemented mp\_mul() and even further before I implemented mp\_exptmod().  As an example as to 
why this design works note that the Karatsuba and Toom-Cook multipliers were written \textit{after} the 
dependent function mp\_exptmod() was written.  Adding the new multiplication algorithms did not require changes to the 
mp\_exptmod() function itself and lowered the total cost of ownership (\textit{so to speak}) and of development 
for new algorithms.  This methodology allows new algorithms to be tested in a complete framework with relative ease.

\begin{center}
\begin{figure}[here]
\includegraphics{pics/design_process.ps}
\caption{Design Flow of the First Few Original LibTomMath Functions.}
\label{pic:design_process}
\end{figure}
\end{center}

Only after the majority of the functions were in place did I pursue a less hierarchical approach to auditing and optimizing
the source code.  For example, one day I may audit the multipliers and the next day the polynomial basis functions.  

It only makes sense to begin the text with the preliminary data types and support algorithms required as well.  
This chapter discusses the core algorithms of the library which are the dependents for every other algorithm.

\section{What is a Multiple Precision Integer?}
Recall that most programming languages, in particular ISO C \cite{ISOC}, only have fixed precision data types that on their own cannot 
be used to represent values larger than their precision will allow. The purpose of multiple precision algorithms is 
to use fixed precision data types to create and manipulate multiple precision integers which may represent values 
that are very large.  

As a well known analogy, school children are taught how to form numbers larger than nine by prepending more radix ten digits.  In the decimal system
the largest single digit value is $9$.  However, by concatenating digits together larger numbers may be represented.  Newly prepended digits 
(\textit{to the left}) are said to be in a different power of ten column.  That is, the number $123$ can be described as having a $1$ in the hundreds 
column, $2$ in the tens column and $3$ in the ones column.  Or more formally $123 = 1 \cdot 10^2 + 2 \cdot 10^1 + 3 \cdot 10^0$.  Computer based 
multiple precision arithmetic is essentially the same concept.  Larger integers are represented by adjoining fixed 
precision computer words with the exception that a different radix is used.

What most people probably do not think about explicitly are the various other attributes that describe a multiple precision 
integer.  For example, the integer $154_{10}$ has two immediately obvious properties.  First, the integer is positive, 
that is the sign of this particular integer is positive as opposed to negative.  Second, the integer has three digits in 
its representation.  There is an additional property that the integer posesses that does not concern pencil-and-paper 
arithmetic.  The third property is how many digits placeholders are available to hold the integer.  

The human analogy of this third property is ensuring there is enough space on the paper to write the integer.  For example,
if one starts writing a large number too far to the right on a piece of paper they will have to erase it and move left.  
Similarly, computer algorithms must maintain strict control over memory usage to ensure that the digits of an integer
will not exceed the allowed boundaries.  These three properties make up what is known as a multiple precision 
integer or mp\_int for short.  

\subsection{The mp\_int Structure}
\label{sec:MPINT}
The mp\_int structure is the ISO C based manifestation of what represents a multiple precision integer.  The ISO C standard does not provide for 
any such data type but it does provide for making composite data types known as structures.  The following is the structure definition 
used within LibTomMath.

\index{mp\_int}
\begin{figure}[here]
\begin{center}
\begin{small}
%\begin{verbatim}
\begin{tabular}{|l|}
\hline
typedef struct \{ \\
\hspace{3mm}int used, alloc, sign;\\
\hspace{3mm}mp\_digit *dp;\\
\} \textbf{mp\_int}; \\
\hline
\end{tabular}
%\end{verbatim}
\end{small}
\caption{The mp\_int Structure}
\label{fig:mpint}
\end{center}
\end{figure}

The mp\_int structure (fig. \ref{fig:mpint}) can be broken down as follows.

\begin{enumerate}
\item The \textbf{used} parameter denotes how many digits of the array \textbf{dp} contain the digits used to represent
a given integer.  The \textbf{used} count must be positive (or zero) and may not exceed the \textbf{alloc} count.  

\item The \textbf{alloc} parameter denotes how 
many digits are available in the array to use by functions before it has to increase in size.  When the \textbf{used} count 
of a result would exceed the \textbf{alloc} count all of the algorithms will automatically increase the size of the 
array to accommodate the precision of the result.  

\item The pointer \textbf{dp} points to a dynamically allocated array of digits that represent the given multiple 
precision integer.  It is padded with $(\textbf{alloc} - \textbf{used})$ zero digits.  The array is maintained in a least 
significant digit order.  As a pencil and paper analogy the array is organized such that the right most digits are stored
first starting at the location indexed by zero\footnote{In C all arrays begin at zero.} in the array.  For example, 
if \textbf{dp} contains $\lbrace a, b, c, \ldots \rbrace$ where \textbf{dp}$_0 = a$, \textbf{dp}$_1 = b$, \textbf{dp}$_2 = c$, $\ldots$ then 
it would represent the integer $a + b\beta + c\beta^2 + \ldots$  

\index{MP\_ZPOS} \index{MP\_NEG}
\item The \textbf{sign} parameter denotes the sign as either zero/positive (\textbf{MP\_ZPOS}) or negative (\textbf{MP\_NEG}).  
\end{enumerate}

\subsubsection{Valid mp\_int Structures}
Several rules are placed on the state of an mp\_int structure and are assumed to be followed for reasons of efficiency.  
The only exceptions are when the structure is passed to initialization functions such as mp\_init() and mp\_init\_copy().

\begin{enumerate}
\item The value of \textbf{alloc} may not be less than one.  That is \textbf{dp} always points to a previously allocated
array of digits.
\item The value of \textbf{used} may not exceed \textbf{alloc} and must be greater than or equal to zero.
\item The value of \textbf{used} implies the digit at index $(used - 1)$ of the \textbf{dp} array is non-zero.  That is, 
leading zero digits in the most significant positions must be trimmed.
   \begin{enumerate}
   \item Digits in the \textbf{dp} array at and above the \textbf{used} location must be zero.
   \end{enumerate}
\item The value of \textbf{sign} must be \textbf{MP\_ZPOS} if \textbf{used} is zero; 
this represents the mp\_int value of zero.
\end{enumerate}

\section{Argument Passing}
A convention of argument passing must be adopted early on in the development of any library.  Making the function 
prototypes consistent will help eliminate many headaches in the future as the library grows to significant complexity.  
In LibTomMath the multiple precision integer functions accept parameters from left to right as pointers to mp\_int 
structures.  That means that the source (input) operands are placed on the left and the destination (output) on the right.   
Consider the following examples.

\begin{verbatim}
   mp_mul(&a, &b, &c);   /* c = a * b */
   mp_add(&a, &b, &a);   /* a = a + b */
   mp_sqr(&a, &b);       /* b = a * a */
\end{verbatim}

The left to right order is a fairly natural way to implement the functions since it lets the developer read aloud the
functions and make sense of them.  For example, the first function would read ``multiply a and b and store in c''.

Certain libraries (\textit{LIP by Lenstra for instance}) accept parameters the other way around, to mimic the order
of assignment expressions.  That is, the destination (output) is on the left and arguments (inputs) are on the right.  In 
truth, it is entirely a matter of preference.  In the case of LibTomMath the convention from the MPI library has been 
adopted.  

Another very useful design consideration, provided for in LibTomMath, is whether to allow argument sources to also be a 
destination.  For example, the second example (\textit{mp\_add}) adds $a$ to $b$ and stores in $a$.  This is an important 
feature to implement since it allows the calling functions to cut down on the number of variables it must maintain.  
However, to implement this feature specific care has to be given to ensure the destination is not modified before the 
source is fully read.

\section{Return Values}
A well implemented application, no matter what its purpose, should trap as many runtime errors as possible and return them 
to the caller.  By catching runtime errors a library can be guaranteed to prevent undefined behaviour.  However, the end 
developer can still manage to cause a library to crash.  For example, by passing an invalid pointer an application may
fault by dereferencing memory not owned by the application.

In the case of LibTomMath the only errors that are checked for are related to inappropriate inputs (division by zero for 
instance) and memory allocation errors.  It will not check that the mp\_int passed to any function is valid nor 
will it check pointers for validity.  Any function that can cause a runtime error will return an error code as an 
\textbf{int} data type with one of the following values (fig \ref{fig:errcodes}).

\index{MP\_OKAY} \index{MP\_VAL} \index{MP\_MEM}
\begin{figure}[here]
\begin{center}
\begin{tabular}{|l|l|}
\hline \textbf{Value} & \textbf{Meaning} \\
\hline \textbf{MP\_OKAY} & The function was successful \\
\hline \textbf{MP\_VAL}  & One of the input value(s) was invalid \\
\hline \textbf{MP\_MEM}  & The function ran out of heap memory \\
\hline
\end{tabular}
\end{center}
\caption{LibTomMath Error Codes}
\label{fig:errcodes}
\end{figure}

When an error is detected within a function it should free any memory it allocated, often during the initialization of
temporary mp\_ints, and return as soon as possible.  The goal is to leave the system in the same state it was when the 
function was called.  Error checking with this style of API is fairly simple.

\begin{verbatim}
   int err;
   if ((err = mp_add(&a, &b, &c)) != MP_OKAY) {
      printf("Error: %s\n", mp_error_to_string(err));
      exit(EXIT_FAILURE);
   }
\end{verbatim}

The GMP \cite{GMP} library uses C style \textit{signals} to flag errors which is of questionable use.  Not all errors are fatal 
and it was not deemed ideal by the author of LibTomMath to force developers to have signal handlers for such cases.

\section{Initialization and Clearing}
The logical starting point when actually writing multiple precision integer functions is the initialization and 
clearing of the mp\_int structures.  These two algorithms will be used by the majority of the higher level algorithms.

Given the basic mp\_int structure an initialization routine must first allocate memory to hold the digits of
the integer.  Often it is optimal to allocate a sufficiently large pre-set number of digits even though
the initial integer will represent zero.  If only a single digit were allocated quite a few subsequent re-allocations
would occur when operations are performed on the integers.  There is a tradeoff between how many default digits to allocate
and how many re-allocations are tolerable.  Obviously allocating an excessive amount of digits initially will waste 
memory and become unmanageable.  

If the memory for the digits has been successfully allocated then the rest of the members of the structure must
be initialized.  Since the initial state of an mp\_int is to represent the zero integer, the allocated digits must be set
to zero.  The \textbf{used} count set to zero and \textbf{sign} set to \textbf{MP\_ZPOS}.

\subsection{Initializing an mp\_int}
An mp\_int is said to be initialized if it is set to a valid, preferably default, state such that all of the members of the
structure are set to valid values.  The mp\_init algorithm will perform such an action.

\index{mp\_init}
\begin{figure}[here]
\begin{center}
\begin{tabular}{l}
\hline Algorithm \textbf{mp\_init}. \\
\textbf{Input}.   An mp\_int $a$ \\
\textbf{Output}.  Allocate memory and initialize $a$ to a known valid mp\_int state.  \\
\hline \\
1.  Allocate memory for \textbf{MP\_PREC} digits. \\
2.  If the allocation failed return(\textit{MP\_MEM}) \\
3.  for $n$ from $0$ to $MP\_PREC - 1$ do  \\
\hspace{3mm}3.1  $a_n \leftarrow 0$\\
4.  $a.sign \leftarrow MP\_ZPOS$\\
5.  $a.used \leftarrow 0$\\
6.  $a.alloc \leftarrow MP\_PREC$\\
7.  Return(\textit{MP\_OKAY})\\
\hline
\end{tabular}
\end{center}
\caption{Algorithm mp\_init}
\end{figure}

\textbf{Algorithm mp\_init.}
The purpose of this function is to initialize an mp\_int structure so that the rest of the library can properly
manipulte it.  It is assumed that the input may not have had any of its members previously initialized which is certainly
a valid assumption if the input resides on the stack.  

Before any of the members such as \textbf{sign}, \textbf{used} or \textbf{alloc} are initialized the memory for
the digits is allocated.  If this fails the function returns before setting any of the other members.  The \textbf{MP\_PREC} 
name represents a constant\footnote{Defined in the ``tommath.h'' header file within LibTomMath.} 
used to dictate the minimum precision of newly initialized mp\_int integers.  Ideally, it is at least equal to the smallest
precision number you'll be working with.

Allocating a block of digits at first instead of a single digit has the benefit of lowering the number of usually slow
heap operations later functions will have to perform in the future.  If \textbf{MP\_PREC} is set correctly the slack 
memory and the number of heap operations will be trivial.

Once the allocation has been made the digits have to be set to zero as well as the \textbf{used}, \textbf{sign} and
\textbf{alloc} members initialized.  This ensures that the mp\_int will always represent the default state of zero regardless
of the original condition of the input.

\textbf{Remark.}
This function introduces the idiosyncrasy that all iterative loops, commonly initiated with the ``for'' keyword, iterate incrementally
when the ``to'' keyword is placed between two expressions.  For example, ``for $a$ from $b$ to $c$ do'' means that
a subsequent expression (or body of expressions) are to be evaluated upto $c - b$ times so long as $b \le c$.  In each
iteration the variable $a$ is substituted for a new integer that lies inclusively between $b$ and $c$.  If $b > c$ occured
the loop would not iterate.  By contrast if the ``downto'' keyword were used in place of ``to'' the loop would iterate 
decrementally.

\vspace{+3mm}\begin{small}
\hspace{-5.1mm}{\bf File}: bn\_mp\_init.c
\vspace{-3mm}
\begin{alltt}
\end{alltt}
\end{small}

One immediate observation of this initializtion function is that it does not return a pointer to a mp\_int structure.  It 
is assumed that the caller has already allocated memory for the mp\_int structure, typically on the application stack.  The 
call to mp\_init() is used only to initialize the members of the structure to a known default state.  

Here we see (line 24) the memory allocation is performed first.  This allows us to exit cleanly and quickly
if there is an error.  If the allocation fails the routine will return \textbf{MP\_MEM} to the caller to indicate there
was a memory error.  The function XMALLOC is what actually allocates the memory.  Technically XMALLOC is not a function
but a macro defined in ``tommath.h``.  By default, XMALLOC will evaluate to malloc() which is the C library's built--in
memory allocation routine.

In order to assure the mp\_int is in a known state the digits must be set to zero.  On most platforms this could have been
accomplished by using calloc() instead of malloc().  However,  to correctly initialize a integer type to a given value in a 
portable fashion you have to actually assign the value.  The for loop (line 30) performs this required
operation.

After the memory has been successfully initialized the remainder of the members are initialized 
(lines 34 through 35) to their respective default states.  At this point the algorithm has succeeded and
a success code is returned to the calling function.  If this function returns \textbf{MP\_OKAY} it is safe to assume the 
mp\_int structure has been properly initialized and is safe to use with other functions within the library.  

\subsection{Clearing an mp\_int}
When an mp\_int is no longer required by the application, the memory that has been allocated for its digits must be 
returned to the application's memory pool with the mp\_clear algorithm.

\begin{figure}[here]
\begin{center}
\begin{tabular}{l}
\hline Algorithm \textbf{mp\_clear}. \\
\textbf{Input}.   An mp\_int $a$ \\
\textbf{Output}.  The memory for $a$ shall be deallocated.  \\
\hline \\
1.  If $a$ has been previously freed then return(\textit{MP\_OKAY}). \\
2.  for $n$ from 0 to $a.used - 1$ do \\
\hspace{3mm}2.1  $a_n \leftarrow 0$ \\
3.  Free the memory allocated for the digits of $a$. \\
4.  $a.used \leftarrow 0$ \\
5.  $a.alloc \leftarrow 0$ \\
6.  $a.sign \leftarrow MP\_ZPOS$ \\
7.  Return(\textit{MP\_OKAY}). \\
\hline
\end{tabular}
\end{center}
\caption{Algorithm mp\_clear}
\end{figure}

\textbf{Algorithm mp\_clear.}
This algorithm accomplishes two goals.  First, it clears the digits and the other mp\_int members.  This ensures that 
if a developer accidentally re-uses a cleared structure it is less likely to cause problems.  The second goal
is to free the allocated memory.

The logic behind the algorithm is extended by marking cleared mp\_int structures so that subsequent calls to this
algorithm will not try to free the memory multiple times.  Cleared mp\_ints are detectable by having a pre-defined invalid 
digit pointer \textbf{dp} setting.

Once an mp\_int has been cleared the mp\_int structure is no longer in a valid state for any other algorithm
with the exception of algorithms mp\_init, mp\_init\_copy, mp\_init\_size and mp\_clear.

\vspace{+3mm}\begin{small}
\hspace{-5.1mm}{\bf File}: bn\_mp\_clear.c
\vspace{-3mm}
\begin{alltt}
\end{alltt}
\end{small}

The algorithm only operates on the mp\_int if it hasn't been previously cleared.  The if statement (line 25)
checks to see if the \textbf{dp} member is not \textbf{NULL}.  If the mp\_int is a valid mp\_int then \textbf{dp} cannot be
\textbf{NULL} in which case the if statement will evaluate to true.

The digits of the mp\_int are cleared by the for loop (line 27) which assigns a zero to every digit.  Similar to mp\_init()
the digits are assigned zero instead of using block memory operations (such as memset()) since this is more portable.  

The digits are deallocated off the heap via the XFREE macro.  Similar to XMALLOC the XFREE macro actually evaluates to
a standard C library function.  In this case the free() function.  Since free() only deallocates the memory the pointer
still has to be reset to \textbf{NULL} manually (line 35).  

Now that the digits have been cleared and deallocated the other members are set to their final values (lines 36 and 37).

\section{Maintenance Algorithms}

The previous sections describes how to initialize and clear an mp\_int structure.  To further support operations
that are to be performed on mp\_int structures (such as addition and multiplication) the dependent algorithms must be
able to augment the precision of an mp\_int and 
initialize mp\_ints with differing initial conditions.  

These algorithms complete the set of low level algorithms required to work with mp\_int structures in the higher level
algorithms such as addition, multiplication and modular exponentiation.

\subsection{Augmenting an mp\_int's Precision}
When storing a value in an mp\_int structure, a sufficient number of digits must be available to accomodate the entire 
result of an operation without loss of precision.  Quite often the size of the array given by the \textbf{alloc} member 
is large enough to simply increase the \textbf{used} digit count.  However, when the size of the array is too small it 
must be re-sized appropriately to accomodate the result.  The mp\_grow algorithm will provide this functionality.

\newpage\begin{figure}[here]
\begin{center}
\begin{tabular}{l}
\hline Algorithm \textbf{mp\_grow}. \\
\textbf{Input}.   An mp\_int $a$ and an integer $b$. \\
\textbf{Output}.  $a$ is expanded to accomodate $b$ digits. \\
\hline \\
1.  if $a.alloc \ge b$ then return(\textit{MP\_OKAY}) \\
2.  $u \leftarrow b\mbox{ (mod }MP\_PREC\mbox{)}$ \\
3.  $v \leftarrow b + 2 \cdot MP\_PREC - u$ \\
4.  Re-allocate the array of digits $a$ to size $v$ \\
5.  If the allocation failed then return(\textit{MP\_MEM}). \\
6.  for n from a.alloc to $v - 1$ do  \\
\hspace{+3mm}6.1  $a_n \leftarrow 0$ \\
7.  $a.alloc \leftarrow v$ \\
8.  Return(\textit{MP\_OKAY}) \\
\hline
\end{tabular}
\end{center}
\caption{Algorithm mp\_grow}
\end{figure}

\textbf{Algorithm mp\_grow.}
It is ideal to prevent re-allocations from being performed if they are not required (step one).  This is useful to 
prevent mp\_ints from growing excessively in code that erroneously calls mp\_grow.  

The requested digit count is padded up to next multiple of \textbf{MP\_PREC} plus an additional \textbf{MP\_PREC} (steps two and three).  
This helps prevent many trivial reallocations that would grow an mp\_int by trivially small values.  

It is assumed that the reallocation (step four) leaves the lower $a.alloc$ digits of the mp\_int intact.  This is much 
akin to how the \textit{realloc} function from the standard C library works.  Since the newly allocated digits are 
assumed to contain undefined values they are initially set to zero.

\vspace{+3mm}\begin{small}
\hspace{-5.1mm}{\bf File}: bn\_mp\_grow.c
\vspace{-3mm}
\begin{alltt}
\end{alltt}
\end{small}

A quick optimization is to first determine if a memory re-allocation is required at all.  The if statement (line 24) checks
if the \textbf{alloc} member of the mp\_int is smaller than the requested digit count.  If the count is not larger than \textbf{alloc}
the function skips the re-allocation part thus saving time.

When a re-allocation is performed it is turned into an optimal request to save time in the future.  The requested digit count is
padded upwards to 2nd multiple of \textbf{MP\_PREC} larger than \textbf{alloc} (line 25).  The XREALLOC function is used
to re-allocate the memory.  As per the other functions XREALLOC is actually a macro which evaluates to realloc by default.  The realloc
function leaves the base of the allocation intact which means the first \textbf{alloc} digits of the mp\_int are the same as before
the re-allocation.  All	that is left is to clear the newly allocated digits and return.

Note that the re-allocation result is actually stored in a temporary pointer $tmp$.  This is to allow this function to return
an error with a valid pointer.  Earlier releases of the library stored the result of XREALLOC into the mp\_int $a$.  That would
result in a memory leak if XREALLOC ever failed.  

\subsection{Initializing Variable Precision mp\_ints}
Occasionally the number of digits required will be known in advance of an initialization, based on, for example, the size 
of input mp\_ints to a given algorithm.  The purpose of algorithm mp\_init\_size is similar to mp\_init except that it 
will allocate \textit{at least} a specified number of digits.  

\begin{figure}[here]
\begin{small}
\begin{center}
\begin{tabular}{l}
\hline Algorithm \textbf{mp\_init\_size}. \\
\textbf{Input}.   An mp\_int $a$ and the requested number of digits $b$. \\
\textbf{Output}.  $a$ is initialized to hold at least $b$ digits. \\
\hline \\
1.  $u \leftarrow b \mbox{ (mod }MP\_PREC\mbox{)}$ \\
2.  $v \leftarrow b + 2 \cdot MP\_PREC - u$ \\
3.  Allocate $v$ digits. \\
4.  for $n$ from $0$ to $v - 1$ do \\
\hspace{3mm}4.1  $a_n \leftarrow 0$ \\
5.  $a.sign \leftarrow MP\_ZPOS$\\
6.  $a.used \leftarrow 0$\\
7.  $a.alloc \leftarrow v$\\
8.  Return(\textit{MP\_OKAY})\\
\hline
\end{tabular}
\end{center}
\end{small}
\caption{Algorithm mp\_init\_size}
\end{figure}

\textbf{Algorithm mp\_init\_size.}
This algorithm will initialize an mp\_int structure $a$ like algorithm mp\_init with the exception that the number of 
digits allocated can be controlled by the second input argument $b$.  The input size is padded upwards so it is a 
multiple of \textbf{MP\_PREC} plus an additional \textbf{MP\_PREC} digits.  This padding is used to prevent trivial 
allocations from becoming a bottleneck in the rest of the algorithms.

Like algorithm mp\_init, the mp\_int structure is initialized to a default state representing the integer zero.  This 
particular algorithm is useful if it is known ahead of time the approximate size of the input.  If the approximation is
correct no further memory re-allocations are required to work with the mp\_int.

\vspace{+3mm}\begin{small}
\hspace{-5.1mm}{\bf File}: bn\_mp\_init\_size.c
\vspace{-3mm}
\begin{alltt}
\end{alltt}
\end{small}

The number of digits $b$ requested is padded (line 24) by first augmenting it to the next multiple of 
\textbf{MP\_PREC} and then adding \textbf{MP\_PREC} to the result.  If the memory can be successfully allocated the 
mp\_int is placed in a default state representing the integer zero.  Otherwise, the error code \textbf{MP\_MEM} will be 
returned (line 29).  

The digits are allocated and set to zero at the same time with the calloc() function (line @25,XCALLOC@).  The 
\textbf{used} count is set to zero, the \textbf{alloc} count set to the padded digit count and the \textbf{sign} flag set 
to \textbf{MP\_ZPOS} to achieve a default valid mp\_int state (lines 33, 34 and 35).  If the function 
returns succesfully then it is correct to assume that the mp\_int structure is in a valid state for the remainder of the 
functions to work with.

\subsection{Multiple Integer Initializations and Clearings}
Occasionally a function will require a series of mp\_int data types to be made available simultaneously.  
The purpose of algorithm mp\_init\_multi is to initialize a variable length array of mp\_int structures in a single
statement.  It is essentially a shortcut to multiple initializations.

\newpage\begin{figure}[here]
\begin{center}
\begin{tabular}{l}
\hline Algorithm \textbf{mp\_init\_multi}. \\
\textbf{Input}.   Variable length array $V_k$ of mp\_int variables of length $k$. \\
\textbf{Output}.  The array is initialized such that each mp\_int of $V_k$ is ready to use. \\
\hline \\
1.  for $n$ from 0 to $k - 1$ do \\
\hspace{+3mm}1.1.  Initialize the mp\_int $V_n$ (\textit{mp\_init}) \\
\hspace{+3mm}1.2.  If initialization failed then do \\
\hspace{+6mm}1.2.1.  for $j$ from $0$ to $n$ do \\
\hspace{+9mm}1.2.1.1.  Free the mp\_int $V_j$ (\textit{mp\_clear}) \\
\hspace{+6mm}1.2.2.   Return(\textit{MP\_MEM}) \\
2.  Return(\textit{MP\_OKAY}) \\
\hline
\end{tabular}
\end{center}
\caption{Algorithm mp\_init\_multi}
\end{figure}

\textbf{Algorithm mp\_init\_multi.}
The algorithm will initialize the array of mp\_int variables one at a time.  If a runtime error has been detected 
(\textit{step 1.2}) all of the previously initialized variables are cleared.  The goal is an ``all or nothing'' 
initialization which allows for quick recovery from runtime errors.

\vspace{+3mm}\begin{small}
\hspace{-5.1mm}{\bf File}: bn\_mp\_init\_multi.c
\vspace{-3mm}
\begin{alltt}
\end{alltt}
\end{small}

This function intializes a variable length list of mp\_int structure pointers.  However, instead of having the mp\_int
structures in an actual C array they are simply passed as arguments to the function.  This function makes use of the 
``...'' argument syntax of the C programming language.  The list is terminated with a final \textbf{NULL} argument 
appended on the right.  

The function uses the ``stdarg.h'' \textit{va} functions to step portably through the arguments to the function.  A count
$n$ of succesfully initialized mp\_int structures is maintained (line 48) such that if a failure does occur,
the algorithm can backtrack and free the previously initialized structures (lines 28 to 47).  


\subsection{Clamping Excess Digits}
When a function anticipates a result will be $n$ digits it is simpler to assume this is true within the body of 
the function instead of checking during the computation.  For example, a multiplication of a $i$ digit number by a 
$j$ digit produces a result of at most $i + j$ digits.  It is entirely possible that the result is $i + j - 1$ 
though, with no final carry into the last position.  However, suppose the destination had to be first expanded 
(\textit{via mp\_grow}) to accomodate $i + j - 1$ digits than further expanded to accomodate the final carry.  
That would be a considerable waste of time since heap operations are relatively slow.

The ideal solution is to always assume the result is $i + j$ and fix up the \textbf{used} count after the function
terminates.  This way a single heap operation (\textit{at most}) is required.  However, if the result was not checked
there would be an excess high order zero digit.  

For example, suppose the product of two integers was $x_n = (0x_{n-1}x_{n-2}...x_0)_{\beta}$.  The leading zero digit 
will not contribute to the precision of the result.  In fact, through subsequent operations more leading zero digits would
accumulate to the point the size of the integer would be prohibitive.  As a result even though the precision is very 
low the representation is excessively large.  

The mp\_clamp algorithm is designed to solve this very problem.  It will trim high-order zeros by decrementing the 
\textbf{used} count until a non-zero most significant digit is found.  Also in this system, zero is considered to be a 
positive number which means that if the \textbf{used} count is decremented to zero, the sign must be set to 
\textbf{MP\_ZPOS}.

\begin{figure}[here]
\begin{center}
\begin{tabular}{l}
\hline Algorithm \textbf{mp\_clamp}. \\
\textbf{Input}.   An mp\_int $a$ \\
\textbf{Output}.  Any excess leading zero digits of $a$ are removed \\
\hline \\
1.  while $a.used > 0$ and $a_{a.used - 1} = 0$ do \\
\hspace{+3mm}1.1  $a.used \leftarrow a.used - 1$ \\
2.  if $a.used = 0$ then do \\
\hspace{+3mm}2.1  $a.sign \leftarrow MP\_ZPOS$ \\
\hline \\
\end{tabular}
\end{center}
\caption{Algorithm mp\_clamp}
\end{figure}

\textbf{Algorithm mp\_clamp.}
As can be expected this algorithm is very simple.  The loop on step one is expected to iterate only once or twice at
the most.  For example, this will happen in cases where there is not a carry to fill the last position.  Step two fixes the sign for 
when all of the digits are zero to ensure that the mp\_int is valid at all times.

\vspace{+3mm}\begin{small}
\hspace{-5.1mm}{\bf File}: bn\_mp\_clamp.c
\vspace{-3mm}
\begin{alltt}
\end{alltt}
\end{small}

Note on line 28 how to test for the \textbf{used} count is made on the left of the \&\& operator.  In the C programming
language the terms to \&\& are evaluated left to right with a boolean short-circuit if any condition fails.  This is 
important since if the \textbf{used} is zero the test on the right would fetch below the array.  That is obviously 
undesirable.  The parenthesis on line 31 is used to make sure the \textbf{used} count is decremented and not
the pointer ``a''.  

\section*{Exercises}
\begin{tabular}{cl}
$\left [ 1 \right ]$ & Discuss the relevance of the \textbf{used} member of the mp\_int structure. \\
                     & \\
$\left [ 1 \right ]$ & Discuss the consequences of not using padding when performing allocations.  \\
                     & \\
$\left [ 2 \right ]$ & Estimate an ideal value for \textbf{MP\_PREC} when performing 1024-bit RSA \\
                     & encryption when $\beta = 2^{28}$.  \\
                     & \\
$\left [ 1 \right ]$ & Discuss the relevance of the algorithm mp\_clamp.  What does it prevent? \\
                     & \\
$\left [ 1 \right ]$ & Give an example of when the algorithm  mp\_init\_copy might be useful. \\
                     & \\
\end{tabular}


%%%
% CHAPTER FOUR
%%%

\chapter{Basic Operations}

\section{Introduction}
In the previous chapter a series of low level algorithms were established that dealt with initializing and maintaining
mp\_int structures.  This chapter will discuss another set of seemingly non-algebraic algorithms which will form the low 
level basis of the entire library.  While these algorithm are relatively trivial it is important to understand how they
work before proceeding since these algorithms will be used almost intrinsically in the following chapters.

The algorithms in this chapter deal primarily with more ``programmer'' related tasks such as creating copies of
mp\_int structures, assigning small values to mp\_int structures and comparisons of the values mp\_int structures
represent.   

\section{Assigning Values to mp\_int Structures}
\subsection{Copying an mp\_int}
Assigning the value that a given mp\_int structure represents to another mp\_int structure shall be known as making
a copy for the purposes of this text.  The copy of the mp\_int will be a separate entity that represents the same
value as the mp\_int it was copied from.  The mp\_copy algorithm provides this functionality. 

\newpage\begin{figure}[here]
\begin{center}
\begin{tabular}{l}
\hline Algorithm \textbf{mp\_copy}. \\
\textbf{Input}.  An mp\_int $a$ and $b$. \\
\textbf{Output}.  Store a copy of $a$ in $b$. \\
\hline \\
1.  If $b.alloc < a.used$ then grow $b$ to $a.used$ digits.  (\textit{mp\_grow}) \\
2.  for $n$ from 0 to $a.used - 1$ do \\
\hspace{3mm}2.1  $b_{n} \leftarrow a_{n}$ \\
3.  for $n$ from $a.used$ to $b.used - 1$ do \\
\hspace{3mm}3.1  $b_{n} \leftarrow 0$ \\
4.  $b.used \leftarrow a.used$ \\
5.  $b.sign \leftarrow a.sign$ \\
6.  return(\textit{MP\_OKAY}) \\
\hline
\end{tabular}
\end{center}
\caption{Algorithm mp\_copy}
\end{figure}

\textbf{Algorithm mp\_copy.}
This algorithm copies the mp\_int $a$ such that upon succesful termination of the algorithm the mp\_int $b$ will
represent the same integer as the mp\_int $a$.  The mp\_int $b$ shall be a complete and distinct copy of the 
mp\_int $a$ meaing that the mp\_int $a$ can be modified and it shall not affect the value of the mp\_int $b$.

If $b$ does not have enough room for the digits of $a$ it must first have its precision augmented via the mp\_grow 
algorithm.  The digits of $a$ are copied over the digits of $b$ and any excess digits of $b$ are set to zero (step two
and three).  The \textbf{used} and \textbf{sign} members of $a$ are finally copied over the respective members of
$b$.

\textbf{Remark.}  This algorithm also introduces a new idiosyncrasy that will be used throughout the rest of the
text.  The error return codes of other algorithms are not explicitly checked in the pseudo-code presented.  For example, in 
step one of the mp\_copy algorithm the return of mp\_grow is not explicitly checked to ensure it succeeded.  Text space is 
limited so it is assumed that if a algorithm fails it will clear all temporarily allocated mp\_ints and return
the error code itself.  However, the C code presented will demonstrate all of the error handling logic required to 
implement the pseudo-code.

\vspace{+3mm}\begin{small}
\hspace{-5.1mm}{\bf File}: bn\_mp\_copy.c
\vspace{-3mm}
\begin{alltt}
\end{alltt}
\end{small}

Occasionally a dependent algorithm may copy an mp\_int effectively into itself such as when the input and output
mp\_int structures passed to a function are one and the same.  For this case it is optimal to return immediately without 
copying digits (line 25).  

The mp\_int $b$ must have enough digits to accomodate the used digits of the mp\_int $a$.  If $b.alloc$ is less than
$a.used$ the algorithm mp\_grow is used to augment the precision of $b$ (lines 30 to 33).  In order to
simplify the inner loop that copies the digits from $a$ to $b$, two aliases $tmpa$ and $tmpb$ point directly at the digits
of the mp\_ints $a$ and $b$ respectively.  These aliases (lines 43 and 46) allow the compiler to access the digits without first dereferencing the
mp\_int pointers and then subsequently the pointer to the digits.  

After the aliases are established the digits from $a$ are copied into $b$ (lines 49 to 51) and then the excess 
digits of $b$ are set to zero (lines 54 to 56).  Both ``for'' loops make use of the pointer aliases and in 
fact the alias for $b$ is carried through into the second ``for'' loop to clear the excess digits.  This optimization 
allows the alias to stay in a machine register fairly easy between the two loops.

\textbf{Remarks.}  The use of pointer aliases is an implementation methodology first introduced in this function that will
be used considerably in other functions.  Technically, a pointer alias is simply a short hand alias used to lower the 
number of pointer dereferencing operations required to access data.  For example, a for loop may resemble

\begin{alltt}
for (x = 0; x < 100; x++) \{
    a->num[4]->dp[x] = 0;
\}
\end{alltt}

This could be re-written using aliases as 

\begin{alltt}
mp_digit *tmpa;
a = a->num[4]->dp;
for (x = 0; x < 100; x++) \{
    *a++ = 0;
\}
\end{alltt}

In this case an alias is used to access the 
array of digits within an mp\_int structure directly.  It may seem that a pointer alias is strictly not required 
as a compiler may optimize out the redundant pointer operations.  However, there are two dominant reasons to use aliases.

The first reason is that most compilers will not effectively optimize pointer arithmetic.  For example, some optimizations 
may work for the Microsoft Visual C++ compiler (MSVC) and not for the GNU C Compiler (GCC).  Also some optimizations may 
work for GCC and not MSVC.  As such it is ideal to find a common ground for as many compilers as possible.  Pointer 
aliases optimize the code considerably before the compiler even reads the source code which means the end compiled code 
stands a better chance of being faster.

The second reason is that pointer aliases often can make an algorithm simpler to read.  Consider the first ``for'' 
loop of the function mp\_copy() re-written to not use pointer aliases.

\begin{alltt}
    /* copy all the digits */
    for (n = 0; n < a->used; n++) \{
      b->dp[n] = a->dp[n];
    \}
\end{alltt}

Whether this code is harder to read depends strongly on the individual.  However, it is quantifiably slightly more 
complicated as there are four variables within the statement instead of just two.

\subsubsection{Nested Statements}
Another commonly used technique in the source routines is that certain sections of code are nested.  This is used in
particular with the pointer aliases to highlight code phases.  For example, a Comba multiplier (discussed in chapter six)
will typically have three different phases.  First the temporaries are initialized, then the columns calculated and 
finally the carries are propagated.  In this example the middle column production phase will typically be nested as it
uses temporary variables and aliases the most.

The nesting also simplies the source code as variables that are nested are only valid for their scope.  As a result
the various temporary variables required do not propagate into other sections of code.


\subsection{Creating a Clone}
Another common operation is to make a local temporary copy of an mp\_int argument.  To initialize an mp\_int 
and then copy another existing mp\_int into the newly intialized mp\_int will be known as creating a clone.  This is 
useful within functions that need to modify an argument but do not wish to actually modify the original copy.  The 
mp\_init\_copy algorithm has been designed to help perform this task.

\begin{figure}[here]
\begin{center}
\begin{tabular}{l}
\hline Algorithm \textbf{mp\_init\_copy}. \\
\textbf{Input}.   An mp\_int $a$ and $b$\\
\textbf{Output}.  $a$ is initialized to be a copy of $b$. \\
\hline \\
1.  Init $a$.  (\textit{mp\_init}) \\
2.  Copy $b$ to $a$.  (\textit{mp\_copy}) \\
3.  Return the status of the copy operation. \\
\hline
\end{tabular}
\end{center}
\caption{Algorithm mp\_init\_copy}
\end{figure}

\textbf{Algorithm mp\_init\_copy.}
This algorithm will initialize an mp\_int variable and copy another previously initialized mp\_int variable into it.  As 
such this algorithm will perform two operations in one step.  

\vspace{+3mm}\begin{small}
\hspace{-5.1mm}{\bf File}: bn\_mp\_init\_copy.c
\vspace{-3mm}
\begin{alltt}
\end{alltt}
\end{small}

This will initialize \textbf{a} and make it a verbatim copy of the contents of \textbf{b}.  Note that 
\textbf{a} will have its own memory allocated which means that \textbf{b} may be cleared after the call
and \textbf{a} will be left intact.  

\section{Zeroing an Integer}
Reseting an mp\_int to the default state is a common step in many algorithms.  The mp\_zero algorithm will be the algorithm used to
perform this task.

\begin{figure}[here]
\begin{center}
\begin{tabular}{l}
\hline Algorithm \textbf{mp\_zero}. \\
\textbf{Input}.   An mp\_int $a$ \\
\textbf{Output}.  Zero the contents of $a$ \\
\hline \\
1.  $a.used \leftarrow 0$ \\
2.  $a.sign \leftarrow$ MP\_ZPOS \\
3.  for $n$ from 0 to $a.alloc - 1$ do \\
\hspace{3mm}3.1  $a_n \leftarrow 0$ \\
\hline
\end{tabular}
\end{center}
\caption{Algorithm mp\_zero}
\end{figure}

\textbf{Algorithm mp\_zero.}
This algorithm simply resets a mp\_int to the default state.  

\vspace{+3mm}\begin{small}
\hspace{-5.1mm}{\bf File}: bn\_mp\_zero.c
\vspace{-3mm}
\begin{alltt}
\end{alltt}
\end{small}

After the function is completed, all of the digits are zeroed, the \textbf{used} count is zeroed and the 
\textbf{sign} variable is set to \textbf{MP\_ZPOS}.

\section{Sign Manipulation}
\subsection{Absolute Value}
With the mp\_int representation of an integer, calculating the absolute value is trivial.  The mp\_abs algorithm will compute
the absolute value of an mp\_int.

\begin{figure}[here]
\begin{center}
\begin{tabular}{l}
\hline Algorithm \textbf{mp\_abs}. \\
\textbf{Input}.   An mp\_int $a$ \\
\textbf{Output}.  Computes $b = \vert a \vert$ \\
\hline \\
1.  Copy $a$ to $b$.  (\textit{mp\_copy}) \\
2.  If the copy failed return(\textit{MP\_MEM}). \\
3.  $b.sign \leftarrow MP\_ZPOS$ \\
4.  Return(\textit{MP\_OKAY}) \\
\hline
\end{tabular}
\end{center}
\caption{Algorithm mp\_abs}
\end{figure}

\textbf{Algorithm mp\_abs.}
This algorithm computes the absolute of an mp\_int input.  First it copies $a$ over $b$.  This is an example of an
algorithm where the check in mp\_copy that determines if the source and destination are equal proves useful.  This allows,
for instance, the developer to pass the same mp\_int as the source and destination to this function without addition 
logic to handle it.

\vspace{+3mm}\begin{small}
\hspace{-5.1mm}{\bf File}: bn\_mp\_abs.c
\vspace{-3mm}
\begin{alltt}
\end{alltt}
\end{small}

This fairly trivial algorithm first eliminates non--required duplications (line 28) and then sets the
\textbf{sign} flag to \textbf{MP\_ZPOS}.

\subsection{Integer Negation}
With the mp\_int representation of an integer, calculating the negation is also trivial.  The mp\_neg algorithm will compute
the negative of an mp\_int input.

\begin{figure}[here]
\begin{center}
\begin{tabular}{l}
\hline Algorithm \textbf{mp\_neg}. \\
\textbf{Input}.   An mp\_int $a$ \\
\textbf{Output}.  Computes $b = -a$ \\
\hline \\
1.  Copy $a$ to $b$.  (\textit{mp\_copy}) \\
2.  If the copy failed return(\textit{MP\_MEM}). \\
3.  If $a.used = 0$ then return(\textit{MP\_OKAY}). \\
4.  If $a.sign = MP\_ZPOS$ then do \\
\hspace{3mm}4.1  $b.sign = MP\_NEG$. \\
5.  else do \\
\hspace{3mm}5.1  $b.sign = MP\_ZPOS$. \\
6.  Return(\textit{MP\_OKAY}) \\
\hline
\end{tabular}
\end{center}
\caption{Algorithm mp\_neg}
\end{figure}

\textbf{Algorithm mp\_neg.}
This algorithm computes the negation of an input.  First it copies $a$ over $b$.  If $a$ has no used digits then
the algorithm returns immediately.  Otherwise it flips the sign flag and stores the result in $b$.  Note that if 
$a$ had no digits then it must be positive by definition.  Had step three been omitted then the algorithm would return
zero as negative.

\vspace{+3mm}\begin{small}
\hspace{-5.1mm}{\bf File}: bn\_mp\_neg.c
\vspace{-3mm}
\begin{alltt}
\end{alltt}
\end{small}

Like mp\_abs() this function avoids non--required duplications (line 22) and then sets the sign.  We
have to make sure that only non--zero values get a \textbf{sign} of \textbf{MP\_NEG}.  If the mp\_int is zero
than the \textbf{sign} is hard--coded to \textbf{MP\_ZPOS}.

\section{Small Constants}
\subsection{Setting Small Constants}
Often a mp\_int must be set to a relatively small value such as $1$ or $2$.  For these cases the mp\_set algorithm is useful.

\newpage\begin{figure}[here]
\begin{center}
\begin{tabular}{l}
\hline Algorithm \textbf{mp\_set}. \\
\textbf{Input}.   An mp\_int $a$ and a digit $b$ \\
\textbf{Output}.  Make $a$ equivalent to $b$ \\
\hline \\
1.  Zero $a$ (\textit{mp\_zero}). \\
2.  $a_0 \leftarrow b \mbox{ (mod }\beta\mbox{)}$ \\
3.  $a.used \leftarrow  \left \lbrace \begin{array}{ll}
                              1 &  \mbox{if }a_0 > 0 \\
                              0 &  \mbox{if }a_0 = 0 
                              \end{array} \right .$ \\
\hline                              
\end{tabular}
\end{center}
\caption{Algorithm mp\_set}
\end{figure}

\textbf{Algorithm mp\_set.}
This algorithm sets a mp\_int to a small single digit value.  Step number 1 ensures that the integer is reset to the default state.  The
single digit is set (\textit{modulo $\beta$}) and the \textbf{used} count is adjusted accordingly.

\vspace{+3mm}\begin{small}
\hspace{-5.1mm}{\bf File}: bn\_mp\_set.c
\vspace{-3mm}
\begin{alltt}
\end{alltt}
\end{small}

First we zero (line 21) the mp\_int to make sure that the other members are initialized for a 
small positive constant.  mp\_zero() ensures that the \textbf{sign} is positive and the \textbf{used} count
is zero.  Next we set the digit and reduce it modulo $\beta$ (line 22).  After this step we have to 
check if the resulting digit is zero or not.  If it is not then we set the \textbf{used} count to one, otherwise
to zero.

We can quickly reduce modulo $\beta$ since it is of the form $2^k$ and a quick binary AND operation with 
$2^k - 1$ will perform the same operation.

One important limitation of this function is that it will only set one digit.  The size of a digit is not fixed, meaning source that uses 
this function should take that into account.  Only trivially small constants can be set using this function.

\subsection{Setting Large Constants}
To overcome the limitations of the mp\_set algorithm the mp\_set\_int algorithm is ideal.  It accepts a ``long''
data type as input and will always treat it as a 32-bit integer.

\begin{figure}[here]
\begin{center}
\begin{tabular}{l}
\hline Algorithm \textbf{mp\_set\_int}. \\
\textbf{Input}.   An mp\_int $a$ and a ``long'' integer $b$ \\
\textbf{Output}.  Make $a$ equivalent to $b$ \\
\hline \\
1.  Zero $a$ (\textit{mp\_zero}) \\
2.  for $n$ from 0 to 7 do \\
\hspace{3mm}2.1  $a \leftarrow a \cdot 16$ (\textit{mp\_mul2d}) \\
\hspace{3mm}2.2  $u \leftarrow \lfloor b / 2^{4(7 - n)} \rfloor \mbox{ (mod }16\mbox{)}$\\
\hspace{3mm}2.3  $a_0 \leftarrow a_0 + u$ \\
\hspace{3mm}2.4  $a.used \leftarrow a.used + 1$ \\
3.  Clamp excess used digits (\textit{mp\_clamp}) \\
\hline
\end{tabular}
\end{center}
\caption{Algorithm mp\_set\_int}
\end{figure}

\textbf{Algorithm mp\_set\_int.}
The algorithm performs eight iterations of a simple loop where in each iteration four bits from the source are added to the 
mp\_int.  Step 2.1 will multiply the current result by sixteen making room for four more bits in the less significant positions.  In step 2.2 the
next four bits from the source are extracted and are added to the mp\_int. The \textbf{used} digit count is 
incremented to reflect the addition.  The \textbf{used} digit counter is incremented since if any of the leading digits were zero the mp\_int would have
zero digits used and the newly added four bits would be ignored.

Excess zero digits are trimmed in steps 2.1 and 3 by using higher level algorithms mp\_mul2d and mp\_clamp.

\vspace{+3mm}\begin{small}
\hspace{-5.1mm}{\bf File}: bn\_mp\_set\_int.c
\vspace{-3mm}
\begin{alltt}
\end{alltt}
\end{small}

This function sets four bits of the number at a time to handle all practical \textbf{DIGIT\_BIT} sizes.  The weird
addition on line 39 ensures that the newly added in bits are added to the number of digits.  While it may not 
seem obvious as to why the digit counter does not grow exceedingly large it is because of the shift on line 28 
as well as the  call to mp\_clamp() on line 41.  Both functions will clamp excess leading digits which keeps 
the number of used digits low.

\section{Comparisons}
\subsection{Unsigned Comparisions}
Comparing a multiple precision integer is performed with the exact same algorithm used to compare two decimal numbers.  For example,
to compare $1,234$ to $1,264$ the digits are extracted by their positions.  That is we compare $1 \cdot 10^3 + 2 \cdot 10^2 + 3 \cdot 10^1 + 4 \cdot 10^0$
to $1 \cdot 10^3 + 2 \cdot 10^2 + 6 \cdot 10^1 + 4 \cdot 10^0$ by comparing single digits at a time starting with the highest magnitude 
positions.  If any leading digit of one integer is greater than a digit in the same position of another integer then obviously it must be greater.  

The first comparision routine that will be developed is the unsigned magnitude compare which will perform a comparison based on the digits of two
mp\_int variables alone.  It will ignore the sign of the two inputs.  Such a function is useful when an absolute comparison is required or if the 
signs are known to agree in advance.

To facilitate working with the results of the comparison functions three constants are required.  

\begin{figure}[here]
\begin{center}
\begin{tabular}{|r|l|}
\hline \textbf{Constant} & \textbf{Meaning} \\
\hline \textbf{MP\_GT} & Greater Than \\
\hline \textbf{MP\_EQ} & Equal To \\
\hline \textbf{MP\_LT} & Less Than \\
\hline
\end{tabular}
\end{center}
\caption{Comparison Return Codes}
\end{figure}

\begin{figure}[here]
\begin{center}
\begin{tabular}{l}
\hline Algorithm \textbf{mp\_cmp\_mag}. \\
\textbf{Input}.   Two mp\_ints $a$ and $b$.  \\
\textbf{Output}.  Unsigned comparison results ($a$ to the left of $b$). \\
\hline \\
1.  If $a.used > b.used$ then return(\textit{MP\_GT}) \\
2.  If $a.used < b.used$ then return(\textit{MP\_LT}) \\
3.  for n from $a.used - 1$ to 0 do \\
\hspace{+3mm}3.1  if $a_n > b_n$ then return(\textit{MP\_GT}) \\
\hspace{+3mm}3.2  if $a_n < b_n$ then return(\textit{MP\_LT}) \\
4.  Return(\textit{MP\_EQ}) \\
\hline
\end{tabular}
\end{center}
\caption{Algorithm mp\_cmp\_mag}
\end{figure}

\textbf{Algorithm mp\_cmp\_mag.}
By saying ``$a$ to the left of $b$'' it is meant that the comparison is with respect to $a$, that is if $a$ is greater than $b$ it will return
\textbf{MP\_GT} and similar with respect to when $a = b$ and $a < b$.  The first two steps compare the number of digits used in both $a$ and $b$.  
Obviously if the digit counts differ there would be an imaginary zero digit in the smaller number where the leading digit of the larger number is.  
If both have the same number of digits than the actual digits themselves must be compared starting at the leading digit.  

By step three both inputs must have the same number of digits so its safe to start from either $a.used - 1$ or $b.used - 1$ and count down to
the zero'th digit.  If after all of the digits have been compared, no difference is found, the algorithm returns \textbf{MP\_EQ}.

\vspace{+3mm}\begin{small}
\hspace{-5.1mm}{\bf File}: bn\_mp\_cmp\_mag.c
\vspace{-3mm}
\begin{alltt}
\end{alltt}
\end{small}

The two if statements (lines 25 and 29) compare the number of digits in the two inputs.  These two are 
performed before all of the digits are compared since it is a very cheap test to perform and can potentially save 
considerable time.  The implementation given is also not valid without those two statements.  $b.alloc$ may be 
smaller than $a.used$, meaning that undefined values will be read from $b$ past the end of the array of digits.



\subsection{Signed Comparisons}
Comparing with sign considerations is also fairly critical in several routines (\textit{division for example}).  Based on an unsigned magnitude 
comparison a trivial signed comparison algorithm can be written.

\begin{figure}[here]
\begin{center}
\begin{tabular}{l}
\hline Algorithm \textbf{mp\_cmp}. \\
\textbf{Input}.   Two mp\_ints $a$ and $b$ \\
\textbf{Output}.  Signed Comparison Results ($a$ to the left of $b$) \\
\hline \\
1.  if $a.sign = MP\_NEG$ and $b.sign = MP\_ZPOS$ then return(\textit{MP\_LT}) \\
2.  if $a.sign = MP\_ZPOS$ and $b.sign = MP\_NEG$ then return(\textit{MP\_GT}) \\
3.  if $a.sign = MP\_NEG$ then \\
\hspace{+3mm}3.1  Return the unsigned comparison of $b$ and $a$ (\textit{mp\_cmp\_mag}) \\
4   Otherwise \\
\hspace{+3mm}4.1  Return the unsigned comparison of $a$ and $b$ \\
\hline
\end{tabular}
\end{center}
\caption{Algorithm mp\_cmp}
\end{figure}

\textbf{Algorithm mp\_cmp.}
The first two steps compare the signs of the two inputs.  If the signs do not agree then it can return right away with the appropriate 
comparison code.  When the signs are equal the digits of the inputs must be compared to determine the correct result.  In step 
three the unsigned comparision flips the order of the arguments since they are both negative.  For instance, if $-a > -b$ then 
$\vert a \vert < \vert b \vert$.  Step number four will compare the two when they are both positive.

\vspace{+3mm}\begin{small}
\hspace{-5.1mm}{\bf File}: bn\_mp\_cmp.c
\vspace{-3mm}
\begin{alltt}
\end{alltt}
\end{small}

The two if statements (lines 23 and 24) perform the initial sign comparison.  If the signs are not the equal then which ever
has the positive sign is larger.   The inputs are compared (line 32) based on magnitudes.  If the signs were both 
negative then the unsigned comparison is performed in the opposite direction (line 34).  Otherwise, the signs are assumed to 
be both positive and a forward direction unsigned comparison is performed.

\section*{Exercises}
\begin{tabular}{cl}
$\left [ 2 \right ]$ & Modify algorithm mp\_set\_int to accept as input a variable length array of bits. \\
                     & \\
$\left [ 3 \right ]$ & Give the probability that algorithm mp\_cmp\_mag will have to compare $k$ digits  \\
                     & of two random digits (of equal magnitude) before a difference is found. \\
                     & \\
$\left [ 1 \right ]$ & Suggest a simple method to speed up the implementation of mp\_cmp\_mag based  \\
                     & on the observations made in the previous problem. \\
                     &
\end{tabular}

\chapter{Basic Arithmetic}
\section{Introduction}
At this point algorithms for initialization, clearing, zeroing, copying, comparing and setting small constants have been 
established.  The next logical set of algorithms to develop are addition, subtraction and digit shifting algorithms.  These 
algorithms make use of the lower level algorithms and are the cruicial building block for the multiplication algorithms.  It is very important 
that these algorithms are highly optimized.  On their own they are simple $O(n)$ algorithms but they can be called from higher level algorithms 
which easily places them at $O(n^2)$ or even $O(n^3)$ work levels.  

All of the algorithms within this chapter make use of the logical bit shift operations denoted by $<<$ and $>>$ for left and right 
logical shifts respectively.  A logical shift is analogous to sliding the decimal point of radix-10 representations.  For example, the real 
number $0.9345$ is equivalent to $93.45\%$ which is found by sliding the the decimal two places to the right (\textit{multiplying by $\beta^2 = 10^2$}).  
Algebraically a binary logical shift is equivalent to a division or multiplication by a power of two.  
For example, $a << k = a \cdot 2^k$ while $a >> k = \lfloor a/2^k \rfloor$.

One significant difference between a logical shift and the way decimals are shifted is that digits below the zero'th position are removed
from the number.  For example, consider $1101_2 >> 1$ using decimal notation this would produce $110.1_2$.  However, with a logical shift the 
result is $110_2$.  

\section{Addition and Subtraction}
In common twos complement fixed precision arithmetic negative numbers are easily represented by subtraction from the modulus.  For example, with 32-bit integers
$a - b\mbox{ (mod }2^{32}\mbox{)}$ is the same as $a + (2^{32} - b) \mbox{ (mod }2^{32}\mbox{)}$  since $2^{32} \equiv 0 \mbox{ (mod }2^{32}\mbox{)}$.  
As a result subtraction can be performed with a trivial series of logical operations and an addition.

However, in multiple precision arithmetic negative numbers are not represented in the same way.  Instead a sign flag is used to keep track of the
sign of the integer.  As a result signed addition and subtraction are actually implemented as conditional usage of lower level addition or 
subtraction algorithms with the sign fixed up appropriately.

The lower level algorithms will add or subtract integers without regard to the sign flag.  That is they will add or subtract the magnitude of
the integers respectively.

\subsection{Low Level Addition}
An unsigned addition of multiple precision integers is performed with the same long-hand algorithm used to add decimal numbers.  That is to add the 
trailing digits first and propagate the resulting carry upwards.  Since this is a lower level algorithm the name will have a ``s\_'' prefix.  
Historically that convention stems from the MPI library where ``s\_'' stood for static functions that were hidden from the developer entirely.

\newpage
\begin{figure}[!here]
\begin{center}
\begin{small}
\begin{tabular}{l}
\hline Algorithm \textbf{s\_mp\_add}. \\
\textbf{Input}.   Two mp\_ints $a$ and $b$ \\
\textbf{Output}.  The unsigned addition $c = \vert a \vert + \vert b \vert$. \\
\hline \\
1.  if $a.used > b.used$ then \\
\hspace{+3mm}1.1  $min \leftarrow b.used$ \\
\hspace{+3mm}1.2  $max \leftarrow a.used$ \\
\hspace{+3mm}1.3  $x   \leftarrow a$ \\
2.  else  \\
\hspace{+3mm}2.1  $min \leftarrow a.used$ \\
\hspace{+3mm}2.2  $max \leftarrow b.used$ \\
\hspace{+3mm}2.3  $x   \leftarrow b$ \\
3.  If $c.alloc < max + 1$ then grow $c$ to hold at least $max + 1$ digits (\textit{mp\_grow}) \\
4.  $oldused \leftarrow c.used$ \\
5.  $c.used \leftarrow max + 1$ \\
6.  $u \leftarrow 0$ \\
7.  for $n$ from $0$ to $min - 1$ do \\
\hspace{+3mm}7.1  $c_n \leftarrow a_n + b_n + u$ \\
\hspace{+3mm}7.2  $u \leftarrow c_n >> lg(\beta)$ \\
\hspace{+3mm}7.3  $c_n \leftarrow c_n \mbox{ (mod }\beta\mbox{)}$ \\
8.  if $min \ne max$ then do \\
\hspace{+3mm}8.1  for $n$ from $min$ to $max - 1$ do \\
\hspace{+6mm}8.1.1  $c_n \leftarrow x_n + u$ \\
\hspace{+6mm}8.1.2  $u \leftarrow c_n >> lg(\beta)$ \\
\hspace{+6mm}8.1.3  $c_n \leftarrow c_n \mbox{ (mod }\beta\mbox{)}$ \\
9.  $c_{max} \leftarrow u$ \\
10.  if $olduse > max$ then \\
\hspace{+3mm}10.1  for $n$ from $max + 1$ to $oldused - 1$ do \\
\hspace{+6mm}10.1.1  $c_n \leftarrow 0$ \\
11.  Clamp excess digits in $c$.  (\textit{mp\_clamp}) \\
12.  Return(\textit{MP\_OKAY}) \\
\hline
\end{tabular}
\end{small}
\end{center}
\caption{Algorithm s\_mp\_add}
\end{figure}

\textbf{Algorithm s\_mp\_add.}
This algorithm is loosely based on algorithm 14.7 of HAC \cite[pp. 594]{HAC} but has been extended to allow the inputs to have different magnitudes.  
Coincidentally the description of algorithm A in Knuth \cite[pp. 266]{TAOCPV2} shares the same deficiency as the algorithm from \cite{HAC}.  Even the 
MIX pseudo  machine code presented by Knuth \cite[pp. 266-267]{TAOCPV2} is incapable of handling inputs which are of different magnitudes.

The first thing that has to be accomplished is to sort out which of the two inputs is the largest.  The addition logic
will simply add all of the smallest input to the largest input and store that first part of the result in the
destination.  Then it will apply a simpler addition loop to excess digits of the larger input.

The first two steps will handle sorting the inputs such that $min$ and $max$ hold the digit counts of the two 
inputs.  The variable $x$ will be an mp\_int alias for the largest input or the second input $b$ if they have the
same number of digits.  After the inputs are sorted the destination $c$ is grown as required to accomodate the sum 
of the two inputs.  The original \textbf{used} count of $c$ is copied and set to the new used count.  

At this point the first addition loop will go through as many digit positions that both inputs have.  The carry
variable $\mu$ is set to zero outside the loop.  Inside the loop an ``addition'' step requires three statements to produce
one digit of the summand.  First
two digits from $a$ and $b$ are added together along with the carry $\mu$.  The carry of this step is extracted and stored
in $\mu$ and finally the digit of the result $c_n$ is truncated within the range $0 \le c_n < \beta$.

Now all of the digit positions that both inputs have in common have been exhausted.  If $min \ne max$ then $x$ is an alias
for one of the inputs that has more digits.  A simplified addition loop is then used to essentially copy the remaining digits
and the carry to the destination.

The final carry is stored in $c_{max}$ and digits above $max$ upto $oldused$ are zeroed which completes the addition.


\vspace{+3mm}\begin{small}
\hspace{-5.1mm}{\bf File}: bn\_s\_mp\_add.c
\vspace{-3mm}
\begin{alltt}
\end{alltt}
\end{small}

We first sort (lines 28 to 36) the inputs based on magnitude and determine the $min$ and $max$ variables.
Note that $x$ is a pointer to an mp\_int assigned to the largest input, in effect it is a local alias.  Next we
grow the destination (38 to 42) ensure that it can accomodate the result of the addition. 

Similar to the implementation of mp\_copy this function uses the braced code and local aliases coding style.  The three aliases that are on 
lines 56, 59 and 62 represent the two inputs and destination variables respectively.  These aliases are used to ensure the
compiler does not have to dereference $a$, $b$ or $c$ (respectively) to access the digits of the respective mp\_int.

The initial carry $u$ will be cleared (line 65), note that $u$ is of type mp\_digit which ensures type 
compatibility within the implementation.  The initial addition (line 66 to 75) adds digits from
both inputs until the smallest input runs out of digits.  Similarly the conditional addition loop
(line 81 to 90) adds the remaining digits from the larger of the two inputs.  The addition is finished 
with the final carry being stored in $tmpc$ (line 94).  Note the ``++'' operator within the same expression.
After line 94, $tmpc$ will point to the $c.used$'th digit of the mp\_int $c$.  This is useful
for the next loop (line 97 to 99) which set any old upper digits to zero.

\subsection{Low Level Subtraction}
The low level unsigned subtraction algorithm is very similar to the low level unsigned addition algorithm.  The principle difference is that the
unsigned subtraction algorithm requires the result to be positive.  That is when computing $a - b$ the condition $\vert a \vert \ge \vert b\vert$ must 
be met for this algorithm to function properly.  Keep in mind this low level algorithm is not meant to be used in higher level algorithms directly.  
This algorithm as will be shown can be used to create functional signed addition and subtraction algorithms.


For this algorithm a new variable is required to make the description simpler.  Recall from section 1.3.1 that a mp\_digit must be able to represent
the range $0 \le x < 2\beta$ for the algorithms to work correctly.  However, it is allowable that a mp\_digit represent a larger range of values.  For 
this algorithm we will assume that the variable $\gamma$ represents the number of bits available in a 
mp\_digit (\textit{this implies $2^{\gamma} > \beta$}).  

For example, the default for LibTomMath is to use a ``unsigned long'' for the mp\_digit ``type'' while $\beta = 2^{28}$.  In ISO C an ``unsigned long''
data type must be able to represent $0 \le x < 2^{32}$ meaning that in this case $\gamma \ge 32$.

\newpage\begin{figure}[!here]
\begin{center}
\begin{small}
\begin{tabular}{l}
\hline Algorithm \textbf{s\_mp\_sub}. \\
\textbf{Input}.   Two mp\_ints $a$ and $b$ ($\vert a \vert \ge \vert b \vert$) \\
\textbf{Output}.  The unsigned subtraction $c = \vert a \vert - \vert b \vert$. \\
\hline \\
1.  $min \leftarrow b.used$ \\
2.  $max \leftarrow a.used$ \\
3.  If $c.alloc < max$ then grow $c$ to hold at least $max$ digits.  (\textit{mp\_grow}) \\
4.  $oldused \leftarrow c.used$ \\ 
5.  $c.used \leftarrow max$ \\
6.  $u \leftarrow 0$ \\
7.  for $n$ from $0$ to $min - 1$ do \\
\hspace{3mm}7.1  $c_n \leftarrow a_n - b_n - u$ \\
\hspace{3mm}7.2  $u   \leftarrow c_n >> (\gamma - 1)$ \\
\hspace{3mm}7.3  $c_n \leftarrow c_n \mbox{ (mod }\beta\mbox{)}$ \\
8.  if $min < max$ then do \\
\hspace{3mm}8.1  for $n$ from $min$ to $max - 1$ do \\
\hspace{6mm}8.1.1  $c_n \leftarrow a_n - u$ \\
\hspace{6mm}8.1.2  $u   \leftarrow c_n >> (\gamma - 1)$ \\
\hspace{6mm}8.1.3  $c_n \leftarrow c_n \mbox{ (mod }\beta\mbox{)}$ \\
9. if $oldused > max$ then do \\
\hspace{3mm}9.1  for $n$ from $max$ to $oldused - 1$ do \\
\hspace{6mm}9.1.1  $c_n \leftarrow 0$ \\
10. Clamp excess digits of $c$.  (\textit{mp\_clamp}). \\
11. Return(\textit{MP\_OKAY}). \\
\hline
\end{tabular}
\end{small}
\end{center}
\caption{Algorithm s\_mp\_sub}
\end{figure}

\textbf{Algorithm s\_mp\_sub.}
This algorithm performs the unsigned subtraction of two mp\_int variables under the restriction that the result must be positive.  That is when
passing variables $a$ and $b$ the condition that $\vert a \vert \ge \vert b \vert$ must be met for the algorithm to function correctly.  This
algorithm is loosely based on algorithm 14.9 \cite[pp. 595]{HAC} and is similar to algorithm S in \cite[pp. 267]{TAOCPV2} as well.  As was the case
of the algorithm s\_mp\_add both other references lack discussion concerning various practical details such as when the inputs differ in magnitude.

The initial sorting of the inputs is trivial in this algorithm since $a$ is guaranteed to have at least the same magnitude of $b$.  Steps 1 and 2 
set the $min$ and $max$ variables.  Unlike the addition routine there is guaranteed to be no carry which means that the final result can be at 
most $max$ digits in length as opposed to $max + 1$.  Similar to the addition algorithm the \textbf{used} count of $c$ is copied locally and 
set to the maximal count for the operation.

The subtraction loop that begins on step seven is essentially the same as the addition loop of algorithm s\_mp\_add except single precision 
subtraction is used instead.  Note the use of the $\gamma$ variable to extract the carry (\textit{also known as the borrow}) within the subtraction 
loops.  Under the assumption that two's complement single precision arithmetic is used this will successfully extract the desired carry.  

For example, consider subtracting $0101_2$ from $0100_2$ where $\gamma = 4$ and $\beta = 2$.  The least significant bit will force a carry upwards to 
the third bit which will be set to zero after the borrow.  After the very first bit has been subtracted $4 - 1 \equiv 0011_2$ will remain,  When the 
third bit of $0101_2$ is subtracted from the result it will cause another carry.  In this case though the carry will be forced to propagate all the 
way to the most significant bit.  

Recall that $\beta < 2^{\gamma}$.  This means that if a carry does occur just before the $lg(\beta)$'th bit it will propagate all the way to the most 
significant bit.  Thus, the high order bits of the mp\_digit that are not part of the actual digit will either be all zero, or all one. All that
is needed is a single zero or one bit for the carry.  Therefore a single logical shift right by $\gamma - 1$ positions is sufficient to extract the 
carry.  This method of carry extraction may seem awkward but the reason for it becomes apparent when the implementation is discussed.  

If $b$ has a smaller magnitude than $a$ then step 9 will force the carry and copy operation to propagate through the larger input $a$ into $c$.  Step
10 will ensure that any leading digits of $c$ above the $max$'th position are zeroed.

\vspace{+3mm}\begin{small}
\hspace{-5.1mm}{\bf File}: bn\_s\_mp\_sub.c
\vspace{-3mm}
\begin{alltt}
\end{alltt}
\end{small}

Like low level addition we ``sort'' the inputs.  Except in this case the sorting is hardcoded 
(lines 25 and 26).  In reality the $min$ and $max$ variables are only aliases and are only 
used to make the source code easier to read.  Again the pointer alias optimization is used 
within this algorithm.  The aliases $tmpa$, $tmpb$ and $tmpc$ are initialized
(lines 42, 43 and 44) for $a$, $b$ and $c$ respectively.

The first subtraction loop (lines 47 through 61) subtract digits from both inputs until the smaller of
the two inputs has been exhausted.  As remarked earlier there is an implementation reason for using the ``awkward'' 
method of extracting the carry (line 57).  The traditional method for extracting the carry would be to shift 
by $lg(\beta)$ positions and logically AND the least significant bit.  The AND operation is required because all of 
the bits above the $\lg(\beta)$'th bit will be set to one after a carry occurs from subtraction.  This carry 
extraction requires two relatively cheap operations to extract the carry.  The other method is to simply shift the 
most significant bit to the least significant bit thus extracting the carry with a single cheap operation.  This 
optimization only works on twos compliment machines which is a safe assumption to make.

If $a$ has a larger magnitude than $b$ an additional loop (lines 64 through 73) is required to propagate 
the carry through $a$ and copy the result to $c$.  

\subsection{High Level Addition}
Now that both lower level addition and subtraction algorithms have been established an effective high level signed addition algorithm can be
established.  This high level addition algorithm will be what other algorithms and developers will use to perform addition of mp\_int data 
types.  

Recall from section 5.2 that an mp\_int represents an integer with an unsigned mantissa (\textit{the array of digits}) and a \textbf{sign} 
flag.  A high level addition is actually performed as a series of eight separate cases which can be optimized down to three unique cases.

\begin{figure}[!here]
\begin{center}
\begin{tabular}{l}
\hline Algorithm \textbf{mp\_add}. \\
\textbf{Input}.   Two mp\_ints $a$ and $b$  \\
\textbf{Output}.  The signed addition $c = a + b$. \\
\hline \\
1.  if $a.sign = b.sign$ then do \\
\hspace{3mm}1.1  $c.sign \leftarrow a.sign$  \\
\hspace{3mm}1.2  $c \leftarrow \vert a \vert + \vert b \vert$ (\textit{s\_mp\_add})\\
2.  else do \\
\hspace{3mm}2.1  if $\vert a \vert < \vert b \vert$ then do (\textit{mp\_cmp\_mag})  \\
\hspace{6mm}2.1.1  $c.sign \leftarrow b.sign$ \\
\hspace{6mm}2.1.2  $c \leftarrow \vert b \vert - \vert a \vert$ (\textit{s\_mp\_sub}) \\
\hspace{3mm}2.2  else do \\
\hspace{6mm}2.2.1  $c.sign \leftarrow a.sign$ \\
\hspace{6mm}2.2.2  $c \leftarrow \vert a \vert - \vert b \vert$ \\
3.  Return(\textit{MP\_OKAY}). \\
\hline
\end{tabular}
\end{center}
\caption{Algorithm mp\_add}
\end{figure}

\textbf{Algorithm mp\_add.}
This algorithm performs the signed addition of two mp\_int variables.  There is no reference algorithm to draw upon from 
either \cite{TAOCPV2} or \cite{HAC} since they both only provide unsigned operations.  The algorithm is fairly 
straightforward but restricted since subtraction can only produce positive results.

\begin{figure}[here]
\begin{small}
\begin{center}
\begin{tabular}{|c|c|c|c|c|}
\hline \textbf{Sign of $a$} & \textbf{Sign of $b$} & \textbf{$\vert a \vert > \vert b \vert $} & \textbf{Unsigned Operation} & \textbf{Result Sign Flag} \\
\hline $+$ & $+$ & Yes & $c = a + b$ & $a.sign$ \\
\hline $+$ & $+$ & No  & $c = a + b$ & $a.sign$ \\
\hline $-$ & $-$ & Yes & $c = a + b$ & $a.sign$ \\
\hline $-$ & $-$ & No  & $c = a + b$ & $a.sign$ \\
\hline &&&&\\

\hline $+$ & $-$ & No  & $c = b - a$ & $b.sign$ \\
\hline $-$ & $+$ & No  & $c = b - a$ & $b.sign$ \\

\hline &&&&\\

\hline $+$ & $-$ & Yes & $c = a - b$ & $a.sign$ \\
\hline $-$ & $+$ & Yes & $c = a - b$ & $a.sign$ \\

\hline
\end{tabular}
\end{center}
\end{small}
\caption{Addition Guide Chart}
\label{fig:AddChart}
\end{figure}

Figure~\ref{fig:AddChart} lists all of the eight possible input combinations and is sorted to show that only three 
specific cases need to be handled.  The return code of the unsigned operations at step 1.2, 2.1.2 and 2.2.2 are 
forwarded to step three to check for errors.  This simplifies the description of the algorithm considerably and best 
follows how the implementation actually was achieved.

Also note how the \textbf{sign} is set before the unsigned addition or subtraction is performed.  Recall from the descriptions of algorithms
s\_mp\_add and s\_mp\_sub that the mp\_clamp function is used at the end to trim excess digits.  The mp\_clamp algorithm will set the \textbf{sign}
to \textbf{MP\_ZPOS} when the \textbf{used} digit count reaches zero.

For example, consider performing $-a + a$ with algorithm mp\_add.  By the description of the algorithm the sign is set to \textbf{MP\_NEG} which would
produce a result of $-0$.  However, since the sign is set first then the unsigned addition is performed the subsequent usage of algorithm mp\_clamp 
within algorithm s\_mp\_add will force $-0$ to become $0$.  

\vspace{+3mm}\begin{small}
\hspace{-5.1mm}{\bf File}: bn\_mp\_add.c
\vspace{-3mm}
\begin{alltt}
\end{alltt}
\end{small}

The source code follows the algorithm fairly closely.  The most notable new source code addition is the usage of the $res$ integer variable which
is used to pass result of the unsigned operations forward.  Unlike in the algorithm, the variable $res$ is merely returned as is without
explicitly checking it and returning the constant \textbf{MP\_OKAY}.  The observation is this algorithm will succeed or fail only if the lower
level functions do so.  Returning their return code is sufficient.

\subsection{High Level Subtraction}
The high level signed subtraction algorithm is essentially the same as the high level signed addition algorithm.  

\newpage\begin{figure}[!here]
\begin{center}
\begin{tabular}{l}
\hline Algorithm \textbf{mp\_sub}. \\
\textbf{Input}.   Two mp\_ints $a$ and $b$  \\
\textbf{Output}.  The signed subtraction $c = a - b$. \\
\hline \\
1.  if $a.sign \ne b.sign$ then do \\
\hspace{3mm}1.1  $c.sign \leftarrow a.sign$ \\
\hspace{3mm}1.2  $c \leftarrow \vert a \vert + \vert b \vert$ (\textit{s\_mp\_add}) \\
2.  else do \\
\hspace{3mm}2.1  if $\vert a \vert \ge \vert b \vert$ then do (\textit{mp\_cmp\_mag}) \\
\hspace{6mm}2.1.1  $c.sign \leftarrow a.sign$ \\
\hspace{6mm}2.1.2  $c \leftarrow \vert a \vert  - \vert b \vert$ (\textit{s\_mp\_sub}) \\
\hspace{3mm}2.2  else do \\
\hspace{6mm}2.2.1  $c.sign \leftarrow  \left \lbrace \begin{array}{ll}
                              MP\_ZPOS &  \mbox{if }a.sign = MP\_NEG \\
                              MP\_NEG  &  \mbox{otherwise} \\
                              \end{array} \right .$ \\
\hspace{6mm}2.2.2  $c \leftarrow \vert b \vert  - \vert a \vert$ \\
3.  Return(\textit{MP\_OKAY}). \\
\hline
\end{tabular}
\end{center}
\caption{Algorithm mp\_sub}
\end{figure}

\textbf{Algorithm mp\_sub.}
This algorithm performs the signed subtraction of two inputs.  Similar to algorithm mp\_add there is no reference in either \cite{TAOCPV2} or 
\cite{HAC}.  Also this algorithm is restricted by algorithm s\_mp\_sub.  Chart \ref{fig:SubChart} lists the eight possible inputs and
the operations required.

\begin{figure}[!here]
\begin{small}
\begin{center}
\begin{tabular}{|c|c|c|c|c|}
\hline \textbf{Sign of $a$} & \textbf{Sign of $b$} & \textbf{$\vert a \vert \ge \vert b \vert $} & \textbf{Unsigned Operation} & \textbf{Result Sign Flag} \\
\hline $+$ & $-$ & Yes & $c = a + b$ & $a.sign$ \\
\hline $+$ & $-$ & No  & $c = a + b$ & $a.sign$ \\
\hline $-$ & $+$ & Yes & $c = a + b$ & $a.sign$ \\
\hline $-$ & $+$ & No  & $c = a + b$ & $a.sign$ \\
\hline &&&& \\
\hline $+$ & $+$ & Yes & $c = a - b$ & $a.sign$ \\
\hline $-$ & $-$ & Yes & $c = a - b$ & $a.sign$ \\
\hline &&&& \\
\hline $+$ & $+$ & No  & $c = b - a$ & $\mbox{opposite of }a.sign$ \\
\hline $-$ & $-$ & No  & $c = b - a$ & $\mbox{opposite of }a.sign$ \\
\hline
\end{tabular}
\end{center}
\end{small}
\caption{Subtraction Guide Chart}
\label{fig:SubChart}
\end{figure}

Similar to the case of algorithm mp\_add the \textbf{sign} is set first before the unsigned addition or subtraction.  That is to prevent the 
algorithm from producing $-a - -a = -0$ as a result.  

\vspace{+3mm}\begin{small}
\hspace{-5.1mm}{\bf File}: bn\_mp\_sub.c
\vspace{-3mm}
\begin{alltt}
\end{alltt}
\end{small}

Much like the implementation of algorithm mp\_add the variable $res$ is used to catch the return code of the unsigned addition or subtraction operations
and forward it to the end of the function.  On line 39 the ``not equal to'' \textbf{MP\_LT} expression is used to emulate a 
``greater than or equal to'' comparison.  

\section{Bit and Digit Shifting}
It is quite common to think of a multiple precision integer as a polynomial in $x$, that is $y = f(\beta)$ where $f(x) = \sum_{i=0}^{n-1} a_i x^i$.  
This notation arises within discussion of Montgomery and Diminished Radix Reduction as well as Karatsuba multiplication and squaring.  

In order to facilitate operations on polynomials in $x$ as above a series of simple ``digit'' algorithms have to be established.  That is to shift
the digits left or right as well to shift individual bits of the digits left and right.  It is important to note that not all ``shift'' operations
are on radix-$\beta$ digits.  

\subsection{Multiplication by Two}

In a binary system where the radix is a power of two multiplication by two not only arises often in other algorithms it is a fairly efficient 
operation to perform.  A single precision logical shift left is sufficient to multiply a single digit by two.  

\newpage\begin{figure}[!here]
\begin{small}
\begin{center}
\begin{tabular}{l}
\hline Algorithm \textbf{mp\_mul\_2}. \\
\textbf{Input}.   One mp\_int $a$ \\
\textbf{Output}.  $b = 2a$. \\
\hline \\
1.  If $b.alloc < a.used + 1$ then grow $b$ to hold $a.used + 1$ digits.  (\textit{mp\_grow}) \\
2.  $oldused \leftarrow b.used$ \\
3.  $b.used \leftarrow a.used$ \\
4.  $r \leftarrow 0$ \\
5.  for $n$ from 0 to $a.used - 1$ do \\
\hspace{3mm}5.1  $rr \leftarrow a_n >> (lg(\beta) - 1)$ \\
\hspace{3mm}5.2  $b_n \leftarrow (a_n << 1) + r \mbox{ (mod }\beta\mbox{)}$ \\
\hspace{3mm}5.3  $r \leftarrow rr$ \\
6.  If $r \ne 0$ then do \\
\hspace{3mm}6.1  $b_{n + 1} \leftarrow r$ \\
\hspace{3mm}6.2  $b.used \leftarrow b.used + 1$ \\
7.  If $b.used < oldused - 1$ then do \\
\hspace{3mm}7.1  for $n$ from $b.used$ to $oldused - 1$ do \\
\hspace{6mm}7.1.1  $b_n \leftarrow 0$ \\
8.  $b.sign \leftarrow a.sign$ \\
9.  Return(\textit{MP\_OKAY}).\\
\hline
\end{tabular}
\end{center}
\end{small}
\caption{Algorithm mp\_mul\_2}
\end{figure}

\textbf{Algorithm mp\_mul\_2.}
This algorithm will quickly multiply a mp\_int by two provided $\beta$ is a power of two.  Neither \cite{TAOCPV2} nor \cite{HAC} describe such 
an algorithm despite the fact it arises often in other algorithms.  The algorithm is setup much like the lower level algorithm s\_mp\_add since 
it is for all intents and purposes equivalent to the operation $b = \vert a \vert + \vert a \vert$.  

Step 1 and 2 grow the input as required to accomodate the maximum number of \textbf{used} digits in the result.  The initial \textbf{used} count
is set to $a.used$ at step 4.  Only if there is a final carry will the \textbf{used} count require adjustment.

Step 6 is an optimization implementation of the addition loop for this specific case.  That is since the two values being added together 
are the same there is no need to perform two reads from the digits of $a$.  Step 6.1 performs a single precision shift on the current digit $a_n$ to
obtain what will be the carry for the next iteration.  Step 6.2 calculates the $n$'th digit of the result as single precision shift of $a_n$ plus
the previous carry.  Recall from section 4.1 that $a_n << 1$ is equivalent to $a_n \cdot 2$.  An iteration of the addition loop is finished with 
forwarding the carry to the next iteration.

Step 7 takes care of any final carry by setting the $a.used$'th digit of the result to the carry and augmenting the \textbf{used} count of $b$.  
Step 8 clears any leading digits of $b$ in case it originally had a larger magnitude than $a$.

\vspace{+3mm}\begin{small}
\hspace{-5.1mm}{\bf File}: bn\_mp\_mul\_2.c
\vspace{-3mm}
\begin{alltt}
\end{alltt}
\end{small}

This implementation is essentially an optimized implementation of s\_mp\_add for the case of doubling an input.  The only noteworthy difference
is the use of the logical shift operator on line 52 to perform a single precision doubling.  

\subsection{Division by Two}
A division by two can just as easily be accomplished with a logical shift right as multiplication by two can be with a logical shift left.

\newpage\begin{figure}[!here]
\begin{small}
\begin{center}
\begin{tabular}{l}
\hline Algorithm \textbf{mp\_div\_2}. \\
\textbf{Input}.   One mp\_int $a$ \\
\textbf{Output}.  $b = a/2$. \\
\hline \\
1.  If $b.alloc < a.used$ then grow $b$ to hold $a.used$ digits.  (\textit{mp\_grow}) \\
2.  If the reallocation failed return(\textit{MP\_MEM}). \\
3.  $oldused \leftarrow b.used$ \\
4.  $b.used \leftarrow a.used$ \\
5.  $r \leftarrow 0$ \\
6.  for $n$ from $b.used - 1$ to $0$ do \\
\hspace{3mm}6.1  $rr \leftarrow a_n \mbox{ (mod }2\mbox{)}$\\
\hspace{3mm}6.2  $b_n \leftarrow (a_n >> 1) + (r << (lg(\beta) - 1)) \mbox{ (mod }\beta\mbox{)}$ \\
\hspace{3mm}6.3  $r \leftarrow rr$ \\
7.  If $b.used < oldused - 1$ then do \\
\hspace{3mm}7.1  for $n$ from $b.used$ to $oldused - 1$ do \\
\hspace{6mm}7.1.1  $b_n \leftarrow 0$ \\
8.  $b.sign \leftarrow a.sign$ \\
9.  Clamp excess digits of $b$.  (\textit{mp\_clamp}) \\
10.  Return(\textit{MP\_OKAY}).\\
\hline
\end{tabular}
\end{center}
\end{small}
\caption{Algorithm mp\_div\_2}
\end{figure}

\textbf{Algorithm mp\_div\_2.}
This algorithm will divide an mp\_int by two using logical shifts to the right.  Like mp\_mul\_2 it uses a modified low level addition
core as the basis of the algorithm.  Unlike mp\_mul\_2 the shift operations work from the leading digit to the trailing digit.  The algorithm
could be written to work from the trailing digit to the leading digit however, it would have to stop one short of $a.used - 1$ digits to prevent
reading past the end of the array of digits.

Essentially the loop at step 6 is similar to that of mp\_mul\_2 except the logical shifts go in the opposite direction and the carry is at the 
least significant bit not the most significant bit.  

\vspace{+3mm}\begin{small}
\hspace{-5.1mm}{\bf File}: bn\_mp\_div\_2.c
\vspace{-3mm}
\begin{alltt}
\end{alltt}
\end{small}

\section{Polynomial Basis Operations}
Recall from section 4.3 that any integer can be represented as a polynomial in $x$ as $y = f(\beta)$.  Such a representation is also known as
the polynomial basis \cite[pp. 48]{ROSE}. Given such a notation a multiplication or division by $x$ amounts to shifting whole digits a single 
place.  The need for such operations arises in several other higher level algorithms such as Barrett and Montgomery reduction, integer
division and Karatsuba multiplication.  

Converting from an array of digits to polynomial basis is very simple.  Consider the integer $y \equiv (a_2, a_1, a_0)_{\beta}$ and recall that
$y = \sum_{i=0}^{2} a_i \beta^i$.  Simply replace $\beta$ with $x$ and the expression is in polynomial basis.  For example, $f(x) = 8x + 9$ is the
polynomial basis representation for $89$ using radix ten.  That is, $f(10) = 8(10) + 9 = 89$.  

\subsection{Multiplication by $x$}

Given a polynomial in $x$ such as $f(x) = a_n x^n + a_{n-1} x^{n-1} + ... + a_0$ multiplying by $x$ amounts to shifting the coefficients up one 
degree.  In this case $f(x) \cdot x = a_n x^{n+1} + a_{n-1} x^n + ... + a_0 x$.  From a scalar basis point of view multiplying by $x$ is equivalent to
multiplying by the integer $\beta$.  

\newpage\begin{figure}[!here]
\begin{small}
\begin{center}
\begin{tabular}{l}
\hline Algorithm \textbf{mp\_lshd}. \\
\textbf{Input}.   One mp\_int $a$ and an integer $b$ \\
\textbf{Output}.  $a \leftarrow a \cdot \beta^b$ (equivalent to multiplication by $x^b$). \\
\hline \\
1.  If $b \le 0$ then return(\textit{MP\_OKAY}). \\
2.  If $a.alloc < a.used + b$ then grow $a$ to at least $a.used + b$ digits.  (\textit{mp\_grow}). \\
3.  If the reallocation failed return(\textit{MP\_MEM}). \\
4.  $a.used \leftarrow a.used + b$ \\
5.  $i \leftarrow a.used - 1$ \\
6.  $j \leftarrow a.used - 1 - b$ \\
7.  for $n$ from $a.used - 1$ to $b$ do \\
\hspace{3mm}7.1  $a_{i} \leftarrow a_{j}$ \\
\hspace{3mm}7.2  $i \leftarrow i - 1$ \\
\hspace{3mm}7.3  $j \leftarrow j - 1$ \\
8.  for $n$ from 0 to $b - 1$ do \\
\hspace{3mm}8.1  $a_n \leftarrow 0$ \\
9.  Return(\textit{MP\_OKAY}). \\
\hline
\end{tabular}
\end{center}
\end{small}
\caption{Algorithm mp\_lshd}
\end{figure}

\textbf{Algorithm mp\_lshd.}
This algorithm multiplies an mp\_int by the $b$'th power of $x$.  This is equivalent to multiplying by $\beta^b$.  The algorithm differs 
from the other algorithms presented so far as it performs the operation in place instead storing the result in a separate location.  The
motivation behind this change is due to the way this function is typically used.  Algorithms such as mp\_add store the result in an optionally
different third mp\_int because the original inputs are often still required.  Algorithm mp\_lshd (\textit{and similarly algorithm mp\_rshd}) is
typically used on values where the original value is no longer required.  The algorithm will return success immediately if 
$b \le 0$ since the rest of algorithm is only valid when $b > 0$.  

First the destination $a$ is grown as required to accomodate the result.  The counters $i$ and $j$ are used to form a \textit{sliding window} over
the digits of $a$ of length $b$.  The head of the sliding window is at $i$ (\textit{the leading digit}) and the tail at $j$ (\textit{the trailing digit}).  
The loop on step 7 copies the digit from the tail to the head.  In each iteration the window is moved down one digit.   The last loop on 
step 8 sets the lower $b$ digits to zero.

\newpage
\begin{center}
\begin{figure}[here]
\includegraphics{pics/sliding_window.ps}
\caption{Sliding Window Movement}
\label{pic:sliding_window}
\end{figure}
\end{center}

\vspace{+3mm}\begin{small}
\hspace{-5.1mm}{\bf File}: bn\_mp\_lshd.c
\vspace{-3mm}
\begin{alltt}
\end{alltt}
\end{small}

The if statement (line 24) ensures that the $b$ variable is greater than zero since we do not interpret negative
shift counts properly.  The \textbf{used} count is incremented by $b$ before the copy loop begins.  This elminates 
the need for an additional variable in the for loop.  The variable $top$ (line 42) is an alias
for the leading digit while $bottom$ (line 45) is an alias for the trailing edge.  The aliases form a 
window of exactly $b$ digits over the input.  

\subsection{Division by $x$}

Division by powers of $x$ is easily achieved by shifting the digits right and removing any that will end up to the right of the zero'th digit.  

\newpage\begin{figure}[!here]
\begin{small}
\begin{center}
\begin{tabular}{l}
\hline Algorithm \textbf{mp\_rshd}. \\
\textbf{Input}.   One mp\_int $a$ and an integer $b$ \\
\textbf{Output}.  $a \leftarrow a / \beta^b$ (Divide by $x^b$). \\
\hline \\
1.  If $b \le 0$ then return. \\
2.  If $a.used \le b$ then do \\
\hspace{3mm}2.1  Zero $a$.  (\textit{mp\_zero}). \\
\hspace{3mm}2.2  Return. \\
3.  $i \leftarrow 0$ \\
4.  $j \leftarrow b$ \\
5.  for $n$ from 0 to $a.used - b - 1$ do \\
\hspace{3mm}5.1  $a_i \leftarrow a_j$ \\
\hspace{3mm}5.2  $i \leftarrow i + 1$ \\
\hspace{3mm}5.3  $j \leftarrow j + 1$ \\
6.  for $n$ from $a.used - b$ to $a.used - 1$ do \\
\hspace{3mm}6.1  $a_n \leftarrow 0$ \\
7.  $a.used \leftarrow a.used - b$ \\
8.  Return. \\
\hline
\end{tabular}
\end{center}
\end{small}
\caption{Algorithm mp\_rshd}
\end{figure}

\textbf{Algorithm mp\_rshd.}
This algorithm divides the input in place by the $b$'th power of $x$.  It is analogous to dividing by a $\beta^b$ but much quicker since
it does not require single precision division.  This algorithm does not actually return an error code as it cannot fail.  

If the input $b$ is less than one the algorithm quickly returns without performing any work.  If the \textbf{used} count is less than or equal
to the shift count $b$ then it will simply zero the input and return.

After the trivial cases of inputs have been handled the sliding window is setup.  Much like the case of algorithm mp\_lshd a sliding window that
is $b$ digits wide is used to copy the digits.  Unlike mp\_lshd the window slides in the opposite direction from the trailing to the leading digit.  
Also the digits are copied from the leading to the trailing edge.

Once the window copy is complete the upper digits must be zeroed and the \textbf{used} count decremented.

\vspace{+3mm}\begin{small}
\hspace{-5.1mm}{\bf File}: bn\_mp\_rshd.c
\vspace{-3mm}
\begin{alltt}
\end{alltt}
\end{small}

The only noteworthy element of this routine is the lack of a return type since it cannot fail.  Like mp\_lshd() we
form a sliding window except we copy in the other direction.  After the window (line 60) we then zero
the upper digits of the input to make sure the result is correct.

\section{Powers of Two}

Now that algorithms for moving single bits as well as whole digits exist algorithms for moving the ``in between'' distances are required.  For 
example, to quickly multiply by $2^k$ for any $k$ without using a full multiplier algorithm would prove useful.  Instead of performing single
shifts $k$ times to achieve a multiplication by $2^{\pm k}$ a mixture of whole digit shifting and partial digit shifting is employed.  

\subsection{Multiplication by Power of Two}

\newpage\begin{figure}[!here]
\begin{small}
\begin{center}
\begin{tabular}{l}
\hline Algorithm \textbf{mp\_mul\_2d}. \\
\textbf{Input}.   One mp\_int $a$ and an integer $b$ \\
\textbf{Output}.  $c \leftarrow a \cdot 2^b$. \\
\hline \\
1.  $c \leftarrow a$.  (\textit{mp\_copy}) \\
2.  If $c.alloc < c.used + \lfloor b / lg(\beta) \rfloor + 2$ then grow $c$ accordingly. \\
3.  If the reallocation failed return(\textit{MP\_MEM}). \\
4.  If $b \ge lg(\beta)$ then \\
\hspace{3mm}4.1  $c \leftarrow c \cdot \beta^{\lfloor b / lg(\beta) \rfloor}$ (\textit{mp\_lshd}). \\
\hspace{3mm}4.2  If step 4.1 failed return(\textit{MP\_MEM}). \\
5.  $d \leftarrow b \mbox{ (mod }lg(\beta)\mbox{)}$ \\
6.  If $d \ne 0$ then do \\
\hspace{3mm}6.1  $mask \leftarrow 2^d$ \\
\hspace{3mm}6.2  $r \leftarrow 0$ \\
\hspace{3mm}6.3  for $n$ from $0$ to $c.used - 1$ do \\
\hspace{6mm}6.3.1  $rr \leftarrow c_n >> (lg(\beta) - d) \mbox{ (mod }mask\mbox{)}$ \\
\hspace{6mm}6.3.2  $c_n \leftarrow (c_n << d) + r \mbox{ (mod }\beta\mbox{)}$ \\
\hspace{6mm}6.3.3  $r \leftarrow rr$ \\
\hspace{3mm}6.4  If $r > 0$ then do \\
\hspace{6mm}6.4.1  $c_{c.used} \leftarrow r$ \\
\hspace{6mm}6.4.2  $c.used \leftarrow c.used + 1$ \\
7.  Return(\textit{MP\_OKAY}). \\
\hline
\end{tabular}
\end{center}
\end{small}
\caption{Algorithm mp\_mul\_2d}
\end{figure}

\textbf{Algorithm mp\_mul\_2d.}
This algorithm multiplies $a$ by $2^b$ and stores the result in $c$.  The algorithm uses algorithm mp\_lshd and a derivative of algorithm mp\_mul\_2 to
quickly compute the product.

First the algorithm will multiply $a$ by $x^{\lfloor b / lg(\beta) \rfloor}$ which will ensure that the remainder multiplicand is less than 
$\beta$.  For example, if $b = 37$ and $\beta = 2^{28}$ then this step will multiply by $x$ leaving a multiplication by $2^{37 - 28} = 2^{9}$ 
left.

After the digits have been shifted appropriately at most $lg(\beta) - 1$ shifts are left to perform.  Step 5 calculates the number of remaining shifts 
required.  If it is non-zero a modified shift loop is used to calculate the remaining product.  
Essentially the loop is a generic version of algorithm mp\_mul\_2 designed to handle any shift count in the range $1 \le x < lg(\beta)$.  The $mask$
variable is used to extract the upper $d$ bits to form the carry for the next iteration.  

This algorithm is loosely measured as a $O(2n)$ algorithm which means that if the input is $n$-digits that it takes $2n$ ``time'' to 
complete.  It is possible to optimize this algorithm down to a $O(n)$ algorithm at a cost of making the algorithm slightly harder to follow.

\vspace{+3mm}\begin{small}
\hspace{-5.1mm}{\bf File}: bn\_mp\_mul\_2d.c
\vspace{-3mm}
\begin{alltt}
\end{alltt}
\end{small}

The shifting is performed in--place which means the first step (line 25) is to copy the input to the 
destination.  We avoid calling mp\_copy() by making sure the mp\_ints are different.  The destination then
has to be grown (line 32) to accomodate the result.

If the shift count $b$ is larger than $lg(\beta)$ then a call to mp\_lshd() is used to handle all of the multiples 
of $lg(\beta)$.  Leaving only a remaining shift of $lg(\beta) - 1$ or fewer bits left.  Inside the actual shift 
loop (lines 46 to 76) we make use of pre--computed values $shift$ and $mask$.   These are used to
extract the carry bit(s) to pass into the next iteration of the loop.  The $r$ and $rr$ variables form a 
chain between consecutive iterations to propagate the carry.  

\subsection{Division by Power of Two}

\newpage\begin{figure}[!here]
\begin{small}
\begin{center}
\begin{tabular}{l}
\hline Algorithm \textbf{mp\_div\_2d}. \\
\textbf{Input}.   One mp\_int $a$ and an integer $b$ \\
\textbf{Output}.  $c \leftarrow \lfloor a / 2^b \rfloor, d \leftarrow a \mbox{ (mod }2^b\mbox{)}$. \\
\hline \\
1.  If $b \le 0$ then do \\
\hspace{3mm}1.1  $c \leftarrow a$ (\textit{mp\_copy}) \\
\hspace{3mm}1.2  $d \leftarrow 0$ (\textit{mp\_zero}) \\
\hspace{3mm}1.3  Return(\textit{MP\_OKAY}). \\
2.  $c \leftarrow a$ \\
3.  $d \leftarrow a \mbox{ (mod }2^b\mbox{)}$ (\textit{mp\_mod\_2d}) \\
4.  If $b \ge lg(\beta)$ then do \\
\hspace{3mm}4.1  $c \leftarrow \lfloor c/\beta^{\lfloor b/lg(\beta) \rfloor} \rfloor$ (\textit{mp\_rshd}). \\
5.  $k \leftarrow b \mbox{ (mod }lg(\beta)\mbox{)}$ \\
6.  If $k \ne 0$ then do \\
\hspace{3mm}6.1  $mask \leftarrow 2^k$ \\
\hspace{3mm}6.2  $r \leftarrow 0$ \\
\hspace{3mm}6.3  for $n$ from $c.used - 1$ to $0$ do \\
\hspace{6mm}6.3.1  $rr \leftarrow c_n \mbox{ (mod }mask\mbox{)}$ \\
\hspace{6mm}6.3.2  $c_n \leftarrow (c_n >> k) + (r << (lg(\beta) - k))$ \\
\hspace{6mm}6.3.3  $r \leftarrow rr$ \\
7.  Clamp excess digits of $c$.  (\textit{mp\_clamp}) \\
8.  Return(\textit{MP\_OKAY}). \\
\hline
\end{tabular}
\end{center}
\end{small}
\caption{Algorithm mp\_div\_2d}
\end{figure}

\textbf{Algorithm mp\_div\_2d.}
This algorithm will divide an input $a$ by $2^b$ and produce the quotient and remainder.  The algorithm is designed much like algorithm 
mp\_mul\_2d by first using whole digit shifts then single precision shifts.  This algorithm will also produce the remainder of the division
by using algorithm mp\_mod\_2d.

\vspace{+3mm}\begin{small}
\hspace{-5.1mm}{\bf File}: bn\_mp\_div\_2d.c
\vspace{-3mm}
\begin{alltt}
\end{alltt}
\end{small}

The implementation of algorithm mp\_div\_2d is slightly different than the algorithm specifies.  The remainder $d$ may be optionally 
ignored by passing \textbf{NULL} as the pointer to the mp\_int variable.    The temporary mp\_int variable $t$ is used to hold the 
result of the remainder operation until the end.  This allows $d$ and $a$ to represent the same mp\_int without modifying $a$ before
the quotient is obtained.

The remainder of the source code is essentially the same as the source code for mp\_mul\_2d.  The only significant difference is
the direction of the shifts.

\subsection{Remainder of Division by Power of Two}

The last algorithm in the series of polynomial basis power of two algorithms is calculating the remainder of division by $2^b$.  This
algorithm benefits from the fact that in twos complement arithmetic $a \mbox{ (mod }2^b\mbox{)}$ is the same as $a$ AND $2^b - 1$.  

\begin{figure}[!here]
\begin{small}
\begin{center}
\begin{tabular}{l}
\hline Algorithm \textbf{mp\_mod\_2d}. \\
\textbf{Input}.   One mp\_int $a$ and an integer $b$ \\
\textbf{Output}.  $c \leftarrow a \mbox{ (mod }2^b\mbox{)}$. \\
\hline \\
1.  If $b \le 0$ then do \\
\hspace{3mm}1.1  $c \leftarrow 0$ (\textit{mp\_zero}) \\
\hspace{3mm}1.2  Return(\textit{MP\_OKAY}). \\
2.  If $b > a.used \cdot lg(\beta)$ then do \\
\hspace{3mm}2.1  $c \leftarrow a$ (\textit{mp\_copy}) \\
\hspace{3mm}2.2  Return the result of step 2.1. \\
3.  $c \leftarrow a$ \\
4.  If step 3 failed return(\textit{MP\_MEM}). \\
5.  for $n$ from $\lceil b / lg(\beta) \rceil$ to $c.used$ do \\
\hspace{3mm}5.1  $c_n \leftarrow 0$ \\
6.  $k \leftarrow b \mbox{ (mod }lg(\beta)\mbox{)}$ \\
7.  $c_{\lfloor b / lg(\beta) \rfloor} \leftarrow c_{\lfloor b / lg(\beta) \rfloor} \mbox{ (mod }2^{k}\mbox{)}$. \\
8.  Clamp excess digits of $c$.  (\textit{mp\_clamp}) \\
9.  Return(\textit{MP\_OKAY}). \\
\hline
\end{tabular}
\end{center}
\end{small}
\caption{Algorithm mp\_mod\_2d}
\end{figure}

\textbf{Algorithm mp\_mod\_2d.}
This algorithm will quickly calculate the value of $a \mbox{ (mod }2^b\mbox{)}$.  First if $b$ is less than or equal to zero the 
result is set to zero.  If $b$ is greater than the number of bits in $a$ then it simply copies $a$ to $c$ and returns.  Otherwise, $a$ 
is copied to $b$, leading digits are removed and the remaining leading digit is trimed to the exact bit count.

\vspace{+3mm}\begin{small}
\hspace{-5.1mm}{\bf File}: bn\_mp\_mod\_2d.c
\vspace{-3mm}
\begin{alltt}
\end{alltt}
\end{small}

We first avoid cases of $b \le 0$ by simply mp\_zero()'ing the destination in such cases.  Next if $2^b$ is larger
than the input we just mp\_copy() the input and return right away.  After this point we know we must actually
perform some work to produce the remainder.

Recalling that reducing modulo $2^k$ and a binary ``and'' with $2^k - 1$ are numerically equivalent we can quickly reduce 
the number.  First we zero any digits above the last digit in $2^b$ (line 42).  Next we reduce the 
leading digit of both (line 46) and then mp\_clamp().

\section*{Exercises}
\begin{tabular}{cl}
$\left [ 3 \right ] $ & Devise an algorithm that performs $a \cdot 2^b$ for generic values of $b$ \\
                      & in $O(n)$ time. \\
                      &\\
$\left [ 3 \right ] $ & Devise an efficient algorithm to multiply by small low hamming  \\
                      & weight values such as $3$, $5$ and $9$.  Extend it to handle all values \\
                      & upto $64$ with a hamming weight less than three. \\
                      &\\
$\left [ 2 \right ] $ & Modify the preceding algorithm to handle values of the form \\
                      & $2^k - 1$ as well. \\
                      &\\
$\left [ 3 \right ] $ & Using only algorithms mp\_mul\_2, mp\_div\_2 and mp\_add create an \\
                      & algorithm to multiply two integers in roughly $O(2n^2)$ time for \\
                      & any $n$-bit input.  Note that the time of addition is ignored in the \\
                      & calculation.  \\
                      & \\
$\left [ 5 \right ] $ & Improve the previous algorithm to have a working time of at most \\
                      & $O \left (2^{(k-1)}n + \left ({2n^2 \over k} \right ) \right )$ for an appropriate choice of $k$.  Again ignore \\
                      & the cost of addition. \\
                      & \\
$\left [ 2 \right ] $ & Devise a chart to find optimal values of $k$ for the previous problem \\
                      & for $n = 64 \ldots 1024$ in steps of $64$. \\
                      & \\
$\left [ 2 \right ] $ & Using only algorithms mp\_abs and mp\_sub devise another method for \\
                      & calculating the result of a signed comparison. \\
                      &
\end{tabular}

\chapter{Multiplication and Squaring}
\section{The Multipliers}
For most number theoretic problems including certain public key cryptographic algorithms, the ``multipliers'' form the most important subset of 
algorithms of any multiple precision integer package.  The set of multiplier algorithms include integer multiplication, squaring and modular reduction 
where in each of the algorithms single precision multiplication is the dominant operation performed.  This chapter will discuss integer multiplication 
and squaring, leaving modular reductions for the subsequent chapter.  

The importance of the multiplier algorithms is for the most part driven by the fact that certain popular public key algorithms are based on modular 
exponentiation, that is computing $d \equiv a^b \mbox{ (mod }c\mbox{)}$ for some arbitrary choice of $a$, $b$, $c$ and $d$.  During a modular
exponentiation the majority\footnote{Roughly speaking a modular exponentiation will spend about 40\% of the time performing modular reductions, 
35\% of the time performing squaring and 25\% of the time performing multiplications.} of the processor time is spent performing single precision 
multiplications.

For centuries general purpose multiplication has required a lengthly $O(n^2)$ process, whereby each digit of one multiplicand has to be multiplied 
against every digit of the other multiplicand.  Traditional long-hand multiplication is based on this process;  while the techniques can differ the 
overall algorithm used is essentially the same.  Only ``recently'' have faster algorithms been studied.  First Karatsuba multiplication was discovered in 
1962.  This algorithm can multiply two numbers with considerably fewer single precision multiplications when compared to the long-hand approach.  
This technique led to the discovery of polynomial basis algorithms (\textit{good reference?}) and subquently Fourier Transform based solutions.  

\section{Multiplication}
\subsection{The Baseline Multiplication}
\label{sec:basemult}
\index{baseline multiplication}
Computing the product of two integers in software can be achieved using a trivial adaptation of the standard $O(n^2)$ long-hand multiplication
algorithm that school children are taught.  The algorithm is considered an $O(n^2)$ algorithm since for two $n$-digit inputs $n^2$ single precision 
multiplications are required.  More specifically for a $m$ and $n$ digit input $m \cdot n$ single precision multiplications are required.  To 
simplify most discussions, it will be assumed that the inputs have comparable number of digits.  

The ``baseline multiplication'' algorithm is designed to act as the ``catch-all'' algorithm, only to be used when the faster algorithms cannot be 
used.  This algorithm does not use any particularly interesting optimizations and should ideally be avoided if possible.    One important 
facet of this algorithm, is that it has been modified to only produce a certain amount of output digits as resolution.  The importance of this 
modification will become evident during the discussion of Barrett modular reduction.  Recall that for a $n$ and $m$ digit input the product 
will be at most $n + m$ digits.  Therefore, this algorithm can be reduced to a full multiplier by having it produce $n + m$ digits of the product.  

Recall from sub-section 4.2.2 the definition of $\gamma$ as the number of bits in the type \textbf{mp\_digit}.  We shall now extend the variable set to 
include $\alpha$ which shall represent the number of bits in the type \textbf{mp\_word}.  This implies that $2^{\alpha} > 2 \cdot \beta^2$.  The 
constant $\delta = 2^{\alpha - 2lg(\beta)}$ will represent the maximal weight of any column in a product (\textit{see sub-section 5.2.2 for more information}).

\newpage\begin{figure}[!here]
\begin{small}
\begin{center}
\begin{tabular}{l}
\hline Algorithm \textbf{s\_mp\_mul\_digs}. \\
\textbf{Input}.   mp\_int $a$, mp\_int $b$ and an integer $digs$ \\
\textbf{Output}.  $c \leftarrow \vert a \vert \cdot \vert b \vert \mbox{ (mod }\beta^{digs}\mbox{)}$. \\
\hline \\
1.  If min$(a.used, b.used) < \delta$ then do \\
\hspace{3mm}1.1  Calculate $c = \vert a \vert \cdot \vert b \vert$ by the Comba method (\textit{see algorithm~\ref{fig:COMBAMULT}}).  \\
\hspace{3mm}1.2  Return the result of step 1.1 \\
\\
Allocate and initialize a temporary mp\_int. \\
2.  Init $t$ to be of size $digs$ \\
3.  If step 2 failed return(\textit{MP\_MEM}). \\
4.  $t.used \leftarrow digs$ \\
\\
Compute the product. \\
5.  for $ix$ from $0$ to $a.used - 1$ do \\
\hspace{3mm}5.1  $u \leftarrow 0$ \\
\hspace{3mm}5.2  $pb \leftarrow \mbox{min}(b.used, digs - ix)$ \\
\hspace{3mm}5.3  If $pb < 1$ then goto step 6. \\
\hspace{3mm}5.4  for $iy$ from $0$ to $pb - 1$ do \\
\hspace{6mm}5.4.1  $\hat r \leftarrow t_{iy + ix} + a_{ix} \cdot b_{iy} + u$ \\
\hspace{6mm}5.4.2  $t_{iy + ix} \leftarrow \hat r \mbox{ (mod }\beta\mbox{)}$ \\
\hspace{6mm}5.4.3  $u \leftarrow \lfloor \hat r / \beta \rfloor$ \\
\hspace{3mm}5.5  if $ix + pb < digs$ then do \\
\hspace{6mm}5.5.1  $t_{ix + pb} \leftarrow u$ \\
6.  Clamp excess digits of $t$. \\
7.  Swap $c$ with $t$ \\
8.  Clear $t$ \\
9.  Return(\textit{MP\_OKAY}). \\
\hline
\end{tabular}
\end{center}
\end{small}
\caption{Algorithm s\_mp\_mul\_digs}
\end{figure}

\textbf{Algorithm s\_mp\_mul\_digs.}
This algorithm computes the unsigned product of two inputs $a$ and $b$, limited to an output precision of $digs$ digits.  While it may seem
a bit awkward to modify the function from its simple $O(n^2)$ description, the usefulness of partial multipliers will arise in a subsequent 
algorithm.  The algorithm is loosely based on algorithm 14.12 from \cite[pp. 595]{HAC} and is similar to Algorithm M of Knuth \cite[pp. 268]{TAOCPV2}.  
Algorithm s\_mp\_mul\_digs differs from these cited references since it can produce a variable output precision regardless of the precision of the 
inputs.

The first thing this algorithm checks for is whether a Comba multiplier can be used instead.   If the minimum digit count of either
input is less than $\delta$, then the Comba method may be used instead.    After the Comba method is ruled out, the baseline algorithm begins.  A 
temporary mp\_int variable $t$ is used to hold the intermediate result of the product.  This allows the algorithm to be used to 
compute products when either $a = c$ or $b = c$ without overwriting the inputs.  

All of step 5 is the infamous $O(n^2)$ multiplication loop slightly modified to only produce upto $digs$ digits of output.  The $pb$ variable
is given the count of digits to read from $b$ inside the nested loop.  If $pb \le 1$ then no more output digits can be produced and the algorithm
will exit the loop.  The best way to think of the loops are as a series of $pb \times 1$ multiplications.    That is, in each pass of the 
innermost loop $a_{ix}$ is multiplied against $b$ and the result is added (\textit{with an appropriate shift}) to $t$.  

For example, consider multiplying $576$ by $241$.  That is equivalent to computing $10^0(1)(576) + 10^1(4)(576) + 10^2(2)(576)$ which is best
visualized in the following table.

\begin{figure}[here]
\begin{center}
\begin{tabular}{|c|c|c|c|c|c|l|}
\hline   &&          & 5 & 7 & 6 & \\
\hline   $\times$&&  & 2 & 4 & 1 & \\
\hline &&&&&&\\
  &&          & 5 & 7 & 6 & $10^0(1)(576)$ \\
  &2 &   3    & 6 & 1 & 6 & $10^1(4)(576) + 10^0(1)(576)$ \\
  1 & 3 & 8 & 8 & 1 & 6 &   $10^2(2)(576) + 10^1(4)(576) + 10^0(1)(576)$ \\
\hline  
\end{tabular}
\end{center}
\caption{Long-Hand Multiplication Diagram}
\end{figure}

Each row of the product is added to the result after being shifted to the left (\textit{multiplied by a power of the radix}) by the appropriate 
count.  That is in pass $ix$ of the inner loop the product is added starting at the $ix$'th digit of the reult.

Step 5.4.1 introduces the hat symbol (\textit{e.g. $\hat r$}) which represents a double precision variable.  The multiplication on that step
is assumed to be a double wide output single precision multiplication.  That is, two single precision variables are multiplied to produce a
double precision result.  The step is somewhat optimized from a long-hand multiplication algorithm because the carry from the addition in step
5.4.1 is propagated through the nested loop.  If the carry was not propagated immediately it would overflow the single precision digit 
$t_{ix+iy}$ and the result would be lost.  

At step 5.5 the nested loop is finished and any carry that was left over should be forwarded.  The carry does not have to be added to the $ix+pb$'th
digit since that digit is assumed to be zero at this point.  However, if $ix + pb \ge digs$ the carry is not set as it would make the result
exceed the precision requested.

\vspace{+3mm}\begin{small}
\hspace{-5.1mm}{\bf File}: bn\_s\_mp\_mul\_digs.c
\vspace{-3mm}
\begin{alltt}
\end{alltt}
\end{small}

First we determine (line 31) if the Comba method can be used first since it's faster.  The conditions for 
sing the Comba routine are that min$(a.used, b.used) < \delta$ and the number of digits of output is less than 
\textbf{MP\_WARRAY}.  This new constant is used to control the stack usage in the Comba routines.  By default it is 
set to $\delta$ but can be reduced when memory is at a premium.

If we cannot use the Comba method we proceed to setup the baseline routine.  We allocate the the destination mp\_int
$t$ (line 37) to the exact size of the output to avoid further re--allocations.  At this point we now 
begin the $O(n^2)$ loop.

This implementation of multiplication has the caveat that it can be trimmed to only produce a variable number of
digits as output.  In each iteration of the outer loop the $pb$ variable is set (line 49) to the maximum 
number of inner loop iterations.  

Inside the inner loop we calculate $\hat r$ as the mp\_word product of the two mp\_digits and the addition of the
carry from the previous iteration.  A particularly important observation is that most modern optimizing 
C compilers (GCC for instance) can recognize that a $N \times N \rightarrow 2N$ multiplication is all that 
is required for the product.  In x86 terms for example, this means using the MUL instruction.

Each digit of the product is stored in turn (line 69) and the carry propagated (line 72) to the 
next iteration.

\subsection{Faster Multiplication by the ``Comba'' Method}

One of the huge drawbacks of the ``baseline'' algorithms is that at the $O(n^2)$ level the carry must be 
computed and propagated upwards.  This makes the nested loop very sequential and hard to unroll and implement 
in parallel.  The ``Comba'' \cite{COMBA} method is named after little known (\textit{in cryptographic venues}) Paul G. 
Comba who described a method of implementing fast multipliers that do not require nested carry fixup operations.  As an 
interesting aside it seems that Paul Barrett describes a similar technique in his 1986 paper \cite{BARRETT} written 
five years before.

At the heart of the Comba technique is once again the long-hand algorithm.  Except in this case a slight 
twist is placed on how the columns of the result are produced.  In the standard long-hand algorithm rows of products 
are produced then added together to form the final result.  In the baseline algorithm the columns are added together 
after each iteration to get the result instantaneously.  

In the Comba algorithm the columns of the result are produced entirely independently of each other.  That is at 
the $O(n^2)$ level a simple multiplication and addition step is performed.  The carries of the columns are propagated 
after the nested loop to reduce the amount of work requiored. Succintly the first step of the algorithm is to compute 
the product vector $\vec x$ as follows. 

\begin{equation}
\vec x_n = \sum_{i+j = n} a_ib_j, \forall n \in \lbrace 0, 1, 2, \ldots, i + j \rbrace
\end{equation}

Where $\vec x_n$ is the $n'th$ column of the output vector.  Consider the following example which computes the vector $\vec x$ for the multiplication
of $576$ and $241$.  

\newpage\begin{figure}[here]
\begin{small}
\begin{center}
\begin{tabular}{|c|c|c|c|c|c|}
  \hline &          & 5 & 7 & 6 & First Input\\
  \hline $\times$ & & 2 & 4 & 1 & Second Input\\
\hline            &                        & $1 \cdot 5 = 5$   & $1 \cdot 7 = 7$   & $1 \cdot 6 = 6$ & First pass \\
                  &  $4 \cdot 5 = 20$      & $4 \cdot 7+5=33$  & $4 \cdot 6+7=31$  & 6               & Second pass \\
   $2 \cdot 5 = 10$ &  $2 \cdot 7 + 20 = 34$ & $2 \cdot 6+33=45$ & 31                & 6             & Third pass \\
\hline 10 & 34 & 45 & 31 & 6 & Final Result \\   
\hline   
\end{tabular}
\end{center}
\end{small}
\caption{Comba Multiplication Diagram}
\end{figure}

At this point the vector $x = \left < 10, 34, 45, 31, 6 \right >$ is the result of the first step of the Comba multipler.  
Now the columns must be fixed by propagating the carry upwards.  The resultant vector will have one extra dimension over the input vector which is
congruent to adding a leading zero digit.

\begin{figure}[!here]
\begin{small}
\begin{center}
\begin{tabular}{l}
\hline Algorithm \textbf{Comba Fixup}. \\
\textbf{Input}.   Vector $\vec x$ of dimension $k$ \\
\textbf{Output}.  Vector $\vec x$ such that the carries have been propagated. \\
\hline \\
1.  for $n$ from $0$ to $k - 1$ do \\
\hspace{3mm}1.1 $\vec x_{n+1} \leftarrow \vec x_{n+1} + \lfloor \vec x_{n}/\beta \rfloor$ \\
\hspace{3mm}1.2 $\vec x_{n} \leftarrow \vec x_{n} \mbox{ (mod }\beta\mbox{)}$ \\
2.  Return($\vec x$). \\
\hline
\end{tabular}
\end{center}
\end{small}
\caption{Algorithm Comba Fixup}
\end{figure}

With that algorithm and $k = 5$ and $\beta = 10$ the following vector is produced $\vec x= \left < 1, 3, 8, 8, 1, 6 \right >$.  In this case 
$241 \cdot 576$ is in fact $138816$ and the procedure succeeded.  If the algorithm is correct and as will be demonstrated shortly more
efficient than the baseline algorithm why not simply always use this algorithm?

\subsubsection{Column Weight.}
At the nested $O(n^2)$ level the Comba method adds the product of two single precision variables to each column of the output 
independently.  A serious obstacle is if the carry is lost, due to lack of precision before the algorithm has a chance to fix
the carries.  For example, in the multiplication of two three-digit numbers the third column of output will be the sum of
three single precision multiplications.  If the precision of the accumulator for the output digits is less then $3 \cdot (\beta - 1)^2$ then
an overflow can occur and the carry information will be lost.  For any $m$ and $n$ digit inputs the maximum weight of any column is 
min$(m, n)$ which is fairly obvious.

The maximum number of terms in any column of a product is known as the ``column weight'' and strictly governs when the algorithm can be used.  Recall
from earlier that a double precision type has $\alpha$ bits of resolution and a single precision digit has $lg(\beta)$ bits of precision.  Given these
two quantities we must not violate the following

\begin{equation}
k \cdot \left (\beta - 1 \right )^2 < 2^{\alpha}
\end{equation}

Which reduces to 

\begin{equation}
k \cdot \left ( \beta^2 - 2\beta + 1 \right ) < 2^{\alpha}
\end{equation}

Let $\rho = lg(\beta)$ represent the number of bits in a single precision digit.  By further re-arrangement of the equation the final solution is
found.

\begin{equation}
k  < {{2^{\alpha}} \over {\left (2^{2\rho} - 2^{\rho + 1} + 1 \right )}}
\end{equation}

The defaults for LibTomMath are $\beta = 2^{28}$ and $\alpha = 2^{64}$ which means that $k$ is bounded by $k < 257$.  In this configuration 
the smaller input may not have more than $256$ digits if the Comba method is to be used.  This is quite satisfactory for most applications since 
$256$ digits would allow for numbers in the range of $0 \le x < 2^{7168}$ which, is much larger than most public key cryptographic algorithms require.  

\newpage\begin{figure}[!here]
\begin{small}
\begin{center}
\begin{tabular}{l}
\hline Algorithm \textbf{fast\_s\_mp\_mul\_digs}. \\
\textbf{Input}.   mp\_int $a$, mp\_int $b$ and an integer $digs$ \\
\textbf{Output}.  $c \leftarrow \vert a \vert \cdot \vert b \vert \mbox{ (mod }\beta^{digs}\mbox{)}$. \\
\hline \\
Place an array of \textbf{MP\_WARRAY} single precision digits named $W$ on the stack. \\
1.  If $c.alloc < digs$ then grow $c$ to $digs$ digits. (\textit{mp\_grow}) \\
2.  If step 1 failed return(\textit{MP\_MEM}).\\
\\
3.  $pa \leftarrow \mbox{MIN}(digs, a.used + b.used)$ \\
\\
4.  $\_ \hat W \leftarrow 0$ \\
5.  for $ix$ from 0 to $pa - 1$ do \\
\hspace{3mm}5.1  $ty \leftarrow \mbox{MIN}(b.used - 1, ix)$ \\
\hspace{3mm}5.2  $tx \leftarrow ix - ty$ \\
\hspace{3mm}5.3  $iy \leftarrow \mbox{MIN}(a.used - tx, ty + 1)$ \\
\hspace{3mm}5.4  for $iz$ from 0 to $iy - 1$ do \\
\hspace{6mm}5.4.1  $\_ \hat W \leftarrow \_ \hat W + a_{tx+iy}b_{ty-iy}$ \\
\hspace{3mm}5.5  $W_{ix} \leftarrow \_ \hat W (\mbox{mod }\beta)$\\
\hspace{3mm}5.6  $\_ \hat W \leftarrow \lfloor \_ \hat W / \beta \rfloor$ \\
\\
6.  $oldused \leftarrow c.used$ \\
7.  $c.used \leftarrow digs$ \\
8.  for $ix$ from $0$ to $pa$ do \\
\hspace{3mm}8.1  $c_{ix} \leftarrow W_{ix}$ \\
9.  for $ix$ from $pa + 1$ to $oldused - 1$ do \\
\hspace{3mm}9.1 $c_{ix} \leftarrow 0$ \\
\\
10.  Clamp $c$. \\
11.  Return MP\_OKAY. \\
\hline
\end{tabular}
\end{center}
\end{small}
\caption{Algorithm fast\_s\_mp\_mul\_digs}
\label{fig:COMBAMULT}
\end{figure}

\textbf{Algorithm fast\_s\_mp\_mul\_digs.}
This algorithm performs the unsigned multiplication of $a$ and $b$ using the Comba method limited to $digs$ digits of precision.

The outer loop of this algorithm is more complicated than that of the baseline multiplier.  This is because on the inside of the 
loop we want to produce one column per pass.  This allows the accumulator $\_ \hat W$ to be placed in CPU registers and
reduce the memory bandwidth to two \textbf{mp\_digit} reads per iteration.

The $ty$ variable is set to the minimum count of $ix$ or the number of digits in $b$.  That way if $a$ has more digits than
$b$ this will be limited to $b.used - 1$.  The $tx$ variable is set to the to the distance past $b.used$ the variable
$ix$ is.  This is used for the immediately subsequent statement where we find $iy$.  

The variable $iy$ is the minimum digits we can read from either $a$ or $b$ before running out.  Computing one column at a time
means we have to scan one integer upwards and the other downwards.  $a$ starts at $tx$ and $b$ starts at $ty$.  In each
pass we are producing the $ix$'th output column and we note that $tx + ty = ix$.  As we move $tx$ upwards we have to 
move $ty$ downards so the equality remains valid.  The $iy$ variable is the number of iterations until 
$tx \ge a.used$ or $ty < 0$ occurs.

After every inner pass we store the lower half of the accumulator into $W_{ix}$ and then propagate the carry of the accumulator
into the next round by dividing $\_ \hat W$ by $\beta$.

To measure the benefits of the Comba method over the baseline method consider the number of operations that are required.  If the 
cost in terms of time of a multiply and addition is $p$ and the cost of a carry propagation is $q$ then a baseline multiplication would require 
$O \left ((p + q)n^2 \right )$ time to multiply two $n$-digit numbers.  The Comba method requires only $O(pn^2 + qn)$ time, however in practice, 
the speed increase is actually much more.  With $O(n)$ space the algorithm can be reduced to $O(pn + qn)$ time by implementing the $n$ multiply
and addition operations in the nested loop in parallel.  

\vspace{+3mm}\begin{small}
\hspace{-5.1mm}{\bf File}: bn\_fast\_s\_mp\_mul\_digs.c
\vspace{-3mm}
\begin{alltt}
\end{alltt}
\end{small}

As per the pseudo--code we first calculate $pa$ (line 48) as the number of digits to output.  Next we begin the outer loop
to produce the individual columns of the product.  We use the two aliases $tmpx$ and $tmpy$ (lines 62, 63) to point
inside the two multiplicands quickly.  

The inner loop (lines 71 to 74) of this implementation is where the tradeoff come into play.  Originally this comba 
implementation was ``row--major'' which means it adds to each of the columns in each pass.  After the outer loop it would then fix 
the carries.  This was very fast except it had an annoying drawback.  You had to read a mp\_word and two mp\_digits and write 
one mp\_word per iteration.  On processors such as the Athlon XP and P4 this did not matter much since the cache bandwidth 
is very high and it can keep the ALU fed with data.  It did, however, matter on older and embedded cpus where cache is often 
slower and also often doesn't exist.  This new algorithm only performs two reads per iteration under the assumption that the 
compiler has aliased $\_ \hat W$ to a CPU register.

After the inner loop we store the current accumulator in $W$ and shift $\_ \hat W$ (lines 77, 80) to forward it as 
a carry for the next pass.  After the outer loop we use the final carry (line 77) as the last digit of the product.  

\subsection{Polynomial Basis Multiplication}
To break the $O(n^2)$ barrier in multiplication requires a completely different look at integer multiplication.  In the following algorithms
the use of polynomial basis representation for two integers $a$ and $b$ as $f(x) = \sum_{i=0}^{n} a_i x^i$ and  
$g(x) = \sum_{i=0}^{n} b_i x^i$ respectively, is required.  In this system both $f(x)$ and $g(x)$ have $n + 1$ terms and are of the $n$'th degree.
 
The product $a \cdot b \equiv f(x)g(x)$ is the polynomial $W(x) = \sum_{i=0}^{2n} w_i x^i$.  The coefficients $w_i$ will
directly yield the desired product when $\beta$ is substituted for $x$.  The direct solution to solve for the $2n + 1$ coefficients
requires $O(n^2)$ time and would in practice be slower than the Comba technique.

However, numerical analysis theory indicates that only $2n + 1$ distinct points in $W(x)$ are required to determine the values of the $2n + 1$ unknown 
coefficients.   This means by finding $\zeta_y = W(y)$ for $2n + 1$ small values of $y$ the coefficients of $W(x)$ can be found with 
Gaussian elimination.  This technique is also occasionally refered to as the \textit{interpolation technique} (\textit{references please...}) since in 
effect an interpolation based on $2n + 1$ points will yield a polynomial equivalent to $W(x)$.  

The coefficients of the polynomial $W(x)$ are unknown which makes finding $W(y)$ for any value of $y$ impossible.  However, since 
$W(x) = f(x)g(x)$ the equivalent $\zeta_y = f(y) g(y)$ can be used in its place.  The benefit of this technique stems from the 
fact that $f(y)$ and $g(y)$ are much smaller than either $a$ or $b$ respectively.  As a result finding the $2n + 1$ relations required 
by multiplying $f(y)g(y)$ involves multiplying integers that are much smaller than either of the inputs.

When picking points to gather relations there are always three obvious points to choose, $y = 0, 1$ and $ \infty$.  The $\zeta_0$ term
is simply the product $W(0) = w_0 = a_0 \cdot b_0$.  The $\zeta_1$ term is the product 
$W(1) = \left (\sum_{i = 0}^{n} a_i \right ) \left (\sum_{i = 0}^{n} b_i \right )$.  The third point $\zeta_{\infty}$ is less obvious but rather
simple to explain.  The $2n + 1$'th coefficient of $W(x)$ is numerically equivalent to the most significant column in an integer multiplication.  
The point at $\infty$ is used symbolically to represent the most significant column, that is $W(\infty) = w_{2n} = a_nb_n$.  Note that the 
points at $y = 0$ and $\infty$ yield the coefficients $w_0$ and $w_{2n}$ directly.

If more points are required they should be of small values and powers of two such as $2^q$ and the related \textit{mirror points} 
$\left (2^q \right )^{2n}  \cdot \zeta_{2^{-q}}$ for small values of $q$.  The term ``mirror point'' stems from the fact that 
$\left (2^q \right )^{2n}  \cdot \zeta_{2^{-q}}$ can be calculated in the exact opposite fashion as $\zeta_{2^q}$.  For
example, when $n = 2$ and $q = 1$ then following two equations are equivalent to the point $\zeta_{2}$ and its mirror.

\begin{eqnarray}
\zeta_{2}                  = f(2)g(2) = (4a_2 + 2a_1 + a_0)(4b_2 + 2b_1 + b_0) \nonumber \\
16 \cdot \zeta_{1 \over 2} = 4f({1\over 2}) \cdot 4g({1 \over 2}) = (a_2 + 2a_1 + 4a_0)(b_2 + 2b_1 + 4b_0)
\end{eqnarray}

Using such points will allow the values of $f(y)$ and $g(y)$ to be independently calculated using only left shifts.  For example, when $n = 2$ the
polynomial $f(2^q)$ is equal to $2^q((2^qa_2) + a_1) + a_0$.  This technique of polynomial representation is known as Horner's method.  

As a general rule of the algorithm when the inputs are split into $n$ parts each there are $2n - 1$ multiplications.  Each multiplication is of 
multiplicands that have $n$ times fewer digits than the inputs.  The asymptotic running time of this algorithm is 
$O \left ( k^{lg_n(2n - 1)} \right )$ for $k$ digit inputs (\textit{assuming they have the same number of digits}).  Figure~\ref{fig:exponent}
summarizes the exponents for various values of $n$.

\begin{figure}
\begin{center}
\begin{tabular}{|c|c|c|}
\hline \textbf{Split into $n$ Parts} & \textbf{Exponent}  & \textbf{Notes}\\
\hline $2$ & $1.584962501$ & This is Karatsuba Multiplication. \\
\hline $3$ & $1.464973520$ & This is Toom-Cook Multiplication. \\
\hline $4$ & $1.403677461$ &\\
\hline $5$ & $1.365212389$ &\\
\hline $10$ & $1.278753601$ &\\
\hline $100$ & $1.149426538$ &\\
\hline $1000$ & $1.100270931$ &\\
\hline $10000$ & $1.075252070$ &\\
\hline
\end{tabular}
\end{center}
\caption{Asymptotic Running Time of Polynomial Basis Multiplication}
\label{fig:exponent}
\end{figure}

At first it may seem like a good idea to choose $n = 1000$ since the exponent is approximately $1.1$.  However, the overhead
of solving for the 2001 terms of $W(x)$ will certainly consume any savings the algorithm could offer for all but exceedingly large
numbers.  

\subsubsection{Cutoff Point}
The polynomial basis multiplication algorithms all require fewer single precision multiplications than a straight Comba approach.  However, 
the algorithms incur an overhead (\textit{at the $O(n)$ work level}) since they require a system of equations to be solved.  This makes the
polynomial basis approach more costly to use with small inputs.

Let $m$ represent the number of digits in the multiplicands (\textit{assume both multiplicands have the same number of digits}).  There exists a 
point $y$ such that when $m < y$ the polynomial basis algorithms are more costly than Comba, when $m = y$ they are roughly the same cost and 
when $m > y$ the Comba methods are slower than the polynomial basis algorithms.  

The exact location of $y$ depends on several key architectural elements of the computer platform in question.

\begin{enumerate}
\item  The ratio of clock cycles for single precision multiplication versus other simpler operations such as addition, shifting, etc.  For example
on the AMD Athlon the ratio is roughly $17 : 1$ while on the Intel P4 it is $29 : 1$.  The higher the ratio in favour of multiplication the lower
the cutoff point $y$ will be.  

\item  The complexity of the linear system of equations (\textit{for the coefficients of $W(x)$}) is.  Generally speaking as the number of splits
grows the complexity grows substantially.  Ideally solving the system will only involve addition, subtraction and shifting of integers.  This
directly reflects on the ratio previous mentioned.

\item  To a lesser extent memory bandwidth and function call overheads.  Provided the values are in the processor cache this is less of an
influence over the cutoff point.

\end{enumerate}

A clean cutoff point separation occurs when a point $y$ is found such that all of the cutoff point conditions are met.  For example, if the point
is too low then there will be values of $m$ such that $m > y$ and the Comba method is still faster.  Finding the cutoff points is fairly simple when
a high resolution timer is available.  

\subsection{Karatsuba Multiplication}
Karatsuba \cite{KARA} multiplication when originally proposed in 1962 was among the first set of algorithms to break the $O(n^2)$ barrier for
general purpose multiplication.  Given two polynomial basis representations $f(x) = ax + b$ and $g(x) = cx + d$, Karatsuba proved with 
light algebra \cite{KARAP} that the following polynomial is equivalent to multiplication of the two integers the polynomials represent.

\begin{equation}
f(x) \cdot g(x) = acx^2 + ((a + b)(c + d) - (ac + bd))x + bd
\end{equation}

Using the observation that $ac$ and $bd$ could be re-used only three half sized multiplications would be required to produce the product.  Applying
this algorithm recursively, the work factor becomes $O(n^{lg(3)})$ which is substantially better than the work factor $O(n^2)$ of the Comba technique.  It turns 
out what Karatsuba did not know or at least did not publish was that this is simply polynomial basis multiplication with the points 
$\zeta_0$, $\zeta_{\infty}$ and $\zeta_{1}$.  Consider the resultant system of equations.

\begin{center}
\begin{tabular}{rcrcrcrc}
$\zeta_{0}$ &      $=$ &  &  &  & & $w_0$ \\
$\zeta_{1}$ &      $=$ & $w_2$ & $+$ & $w_1$ & $+$ & $w_0$ \\
$\zeta_{\infty}$ & $=$ & $w_2$ &  & &  & \\
\end{tabular}
\end{center}

By adding the first and last equation to the equation in the middle the term $w_1$ can be isolated and all three coefficients solved for.  The simplicity
of this system of equations has made Karatsuba fairly popular.  In fact the cutoff point is often fairly low\footnote{With LibTomMath 0.18 it is 70 and 109 digits for the Intel P4 and AMD Athlon respectively.}
making it an ideal algorithm to speed up certain public key cryptosystems such as RSA and Diffie-Hellman.  

\newpage\begin{figure}[!here]
\begin{small}
\begin{center}
\begin{tabular}{l}
\hline Algorithm \textbf{mp\_karatsuba\_mul}. \\
\textbf{Input}.   mp\_int $a$ and mp\_int $b$ \\
\textbf{Output}.  $c \leftarrow \vert a \vert \cdot \vert b \vert$ \\
\hline \\
1.  Init the following mp\_int variables: $x0$, $x1$, $y0$, $y1$, $t1$, $x0y0$, $x1y1$.\\
2.  If step 2 failed then return(\textit{MP\_MEM}). \\
\\
Split the input.  e.g. $a = x1 \cdot \beta^B + x0$ \\
3.  $B \leftarrow \mbox{min}(a.used, b.used)/2$ \\
4.  $x0 \leftarrow a \mbox{ (mod }\beta^B\mbox{)}$ (\textit{mp\_mod\_2d}) \\
5.  $y0 \leftarrow b \mbox{ (mod }\beta^B\mbox{)}$ \\
6.  $x1 \leftarrow \lfloor a / \beta^B \rfloor$ (\textit{mp\_rshd}) \\
7.  $y1 \leftarrow \lfloor b / \beta^B \rfloor$ \\
\\
Calculate the three products. \\
8.  $x0y0 \leftarrow x0 \cdot y0$ (\textit{mp\_mul}) \\
9.  $x1y1 \leftarrow x1 \cdot y1$ \\
10.  $t1 \leftarrow x1 + x0$ (\textit{mp\_add}) \\
11.  $x0 \leftarrow y1 + y0$ \\
12.  $t1 \leftarrow t1 \cdot x0$ \\
\\
Calculate the middle term. \\
13.  $x0 \leftarrow x0y0 + x1y1$ \\
14.  $t1 \leftarrow t1 - x0$ (\textit{s\_mp\_sub}) \\
\\
Calculate the final product. \\
15.  $t1 \leftarrow t1 \cdot \beta^B$ (\textit{mp\_lshd}) \\
16.  $x1y1 \leftarrow x1y1 \cdot \beta^{2B}$ \\
17.  $t1 \leftarrow x0y0 + t1$ \\
18.  $c \leftarrow t1 + x1y1$ \\
19.  Clear all of the temporary variables. \\
20.  Return(\textit{MP\_OKAY}).\\
\hline 
\end{tabular}
\end{center}
\end{small}
\caption{Algorithm mp\_karatsuba\_mul}
\end{figure}

\textbf{Algorithm mp\_karatsuba\_mul.}
This algorithm computes the unsigned product of two inputs using the Karatsuba multiplication algorithm.  It is loosely based on the description
from Knuth \cite[pp. 294-295]{TAOCPV2}.  

\index{radix point}
In order to split the two inputs into their respective halves, a suitable \textit{radix point} must be chosen.  The radix point chosen must
be used for both of the inputs meaning that it must be smaller than the smallest input.  Step 3 chooses the radix point $B$ as half of the 
smallest input \textbf{used} count.  After the radix point is chosen the inputs are split into lower and upper halves.  Step 4 and 5 
compute the lower halves.  Step 6 and 7 computer the upper halves.  

After the halves have been computed the three intermediate half-size products must be computed.  Step 8 and 9 compute the trivial products
$x0 \cdot y0$ and $x1 \cdot y1$.  The mp\_int $x0$ is used as a temporary variable after $x1 + x0$ has been computed.  By using $x0$ instead
of an additional temporary variable, the algorithm can avoid an addition memory allocation operation.

The remaining steps 13 through 18 compute the Karatsuba polynomial through a variety of digit shifting and addition operations.

\vspace{+3mm}\begin{small}
\hspace{-5.1mm}{\bf File}: bn\_mp\_karatsuba\_mul.c
\vspace{-3mm}
\begin{alltt}
\end{alltt}
\end{small}

The new coding element in this routine, not  seen in previous routines, is the usage of goto statements.  The conventional
wisdom is that goto statements should be avoided.  This is generally true, however when every single function call can fail, it makes sense
to handle error recovery with a single piece of code.  Lines 62 to 76 handle initializing all of the temporary variables 
required.  Note how each of the if statements goes to a different label in case of failure.  This allows the routine to correctly free only
the temporaries that have been successfully allocated so far.

The temporary variables are all initialized using the mp\_init\_size routine since they are expected to be large.  This saves the 
additional reallocation that would have been necessary.  Also $x0$, $x1$, $y0$ and $y1$ have to be able to hold at least their respective
number of digits for the next section of code.

The first algebraic portion of the algorithm is to split the two inputs into their halves.  However, instead of using mp\_mod\_2d and mp\_rshd
to extract the halves, the respective code has been placed inline within the body of the function.  To initialize the halves, the \textbf{used} and 
\textbf{sign} members are copied first.  The first for loop on line 96 copies the lower halves.  Since they are both the same magnitude it 
is simpler to calculate both lower halves in a single loop.  The for loop on lines 102 and 107 calculate the upper halves $x1$ and 
$y1$ respectively.

By inlining the calculation of the halves, the Karatsuba multiplier has a slightly lower overhead and can be used for smaller magnitude inputs.

When line 151 is reached, the algorithm has completed succesfully.  The ``error status'' variable $err$ is set to \textbf{MP\_OKAY} so that
the same code that handles errors can be used to clear the temporary variables and return.  

\subsection{Toom-Cook $3$-Way Multiplication}
Toom-Cook $3$-Way \cite{TOOM} multiplication is essentially the polynomial basis algorithm for $n = 2$ except that the points  are 
chosen such that $\zeta$ is easy to compute and the resulting system of equations easy to reduce.  Here, the points $\zeta_{0}$, 
$16 \cdot \zeta_{1 \over 2}$, $\zeta_1$, $\zeta_2$ and $\zeta_{\infty}$ make up the five required points to solve for the coefficients 
of the $W(x)$.

With the five relations that Toom-Cook specifies, the following system of equations is formed.

\begin{center}
\begin{tabular}{rcrcrcrcrcr}
$\zeta_0$                    & $=$ & $0w_4$ & $+$ & $0w_3$ & $+$ & $0w_2$ & $+$ & $0w_1$ & $+$ & $1w_0$  \\
$16 \cdot \zeta_{1 \over 2}$ & $=$ & $1w_4$ & $+$ & $2w_3$ & $+$ & $4w_2$ & $+$ & $8w_1$ & $+$ & $16w_0$  \\
$\zeta_1$                    & $=$ & $1w_4$ & $+$ & $1w_3$ & $+$ & $1w_2$ & $+$ & $1w_1$ & $+$ & $1w_0$  \\
$\zeta_2$                    & $=$ & $16w_4$ & $+$ & $8w_3$ & $+$ & $4w_2$ & $+$ & $2w_1$ & $+$ & $1w_0$  \\
$\zeta_{\infty}$             & $=$ & $1w_4$ & $+$ & $0w_3$ & $+$ & $0w_2$ & $+$ & $0w_1$ & $+$ & $0w_0$  \\
\end{tabular}
\end{center}

A trivial solution to this matrix requires $12$ subtractions, two multiplications by a small power of two, two divisions by a small power
of two, two divisions by three and one multiplication by three.  All of these $19$ sub-operations require less than quadratic time, meaning that
the algorithm can be faster than a baseline multiplication.  However, the greater complexity of this algorithm places the cutoff point
(\textbf{TOOM\_MUL\_CUTOFF}) where Toom-Cook becomes more efficient much higher than the Karatsuba cutoff point.  

\begin{figure}[!here]
\begin{small}
\begin{center}
\begin{tabular}{l}
\hline Algorithm \textbf{mp\_toom\_mul}. \\
\textbf{Input}.   mp\_int $a$ and mp\_int $b$ \\
\textbf{Output}.  $c \leftarrow  a  \cdot  b $ \\
\hline \\
Split $a$ and $b$ into three pieces.  E.g. $a = a_2 \beta^{2k} + a_1 \beta^{k} + a_0$ \\
1.  $k \leftarrow \lfloor \mbox{min}(a.used, b.used) / 3 \rfloor$ \\
2.  $a_0 \leftarrow a \mbox{ (mod }\beta^{k}\mbox{)}$ \\
3.  $a_1 \leftarrow \lfloor a / \beta^k \rfloor$, $a_1 \leftarrow a_1 \mbox{ (mod }\beta^{k}\mbox{)}$ \\
4.  $a_2 \leftarrow \lfloor a / \beta^{2k} \rfloor$, $a_2 \leftarrow a_2 \mbox{ (mod }\beta^{k}\mbox{)}$ \\
5.  $b_0 \leftarrow a \mbox{ (mod }\beta^{k}\mbox{)}$ \\
6.  $b_1 \leftarrow \lfloor a / \beta^k \rfloor$, $b_1 \leftarrow b_1 \mbox{ (mod }\beta^{k}\mbox{)}$ \\
7.  $b_2 \leftarrow \lfloor a / \beta^{2k} \rfloor$, $b_2 \leftarrow b_2 \mbox{ (mod }\beta^{k}\mbox{)}$ \\
\\
Find the five equations for $w_0, w_1, ..., w_4$. \\
8.  $w_0 \leftarrow a_0 \cdot b_0$ \\
9.  $w_4 \leftarrow a_2 \cdot b_2$ \\
10. $tmp_1 \leftarrow 2 \cdot a_0$, $tmp_1 \leftarrow a_1 + tmp_1$, $tmp_1 \leftarrow 2 \cdot tmp_1$, $tmp_1 \leftarrow tmp_1 + a_2$ \\
11. $tmp_2 \leftarrow 2 \cdot b_0$, $tmp_2 \leftarrow b_1 + tmp_2$, $tmp_2 \leftarrow 2 \cdot tmp_2$, $tmp_2 \leftarrow tmp_2 + b_2$ \\
12. $w_1 \leftarrow tmp_1 \cdot tmp_2$ \\
13. $tmp_1 \leftarrow 2 \cdot a_2$, $tmp_1 \leftarrow a_1 + tmp_1$, $tmp_1 \leftarrow 2 \cdot tmp_1$, $tmp_1 \leftarrow tmp_1 + a_0$ \\
14. $tmp_2 \leftarrow 2 \cdot b_2$, $tmp_2 \leftarrow b_1 + tmp_2$, $tmp_2 \leftarrow 2 \cdot tmp_2$, $tmp_2 \leftarrow tmp_2 + b_0$ \\
15. $w_3 \leftarrow tmp_1 \cdot tmp_2$ \\
16. $tmp_1 \leftarrow a_0 + a_1$, $tmp_1 \leftarrow tmp_1 + a_2$, $tmp_2 \leftarrow b_0 + b_1$, $tmp_2 \leftarrow tmp_2 + b_2$ \\
17. $w_2 \leftarrow tmp_1 \cdot tmp_2$ \\
\\
Continued on the next page.\\
\hline
\end{tabular}
\end{center}
\end{small}
\caption{Algorithm mp\_toom\_mul}
\end{figure}

\newpage\begin{figure}[!here]
\begin{small}
\begin{center}
\begin{tabular}{l}
\hline Algorithm \textbf{mp\_toom\_mul} (continued). \\
\textbf{Input}.   mp\_int $a$ and mp\_int $b$ \\
\textbf{Output}.  $c \leftarrow a \cdot  b $ \\
\hline \\
Now solve the system of equations. \\
18. $w_1 \leftarrow w_4 - w_1$, $w_3 \leftarrow w_3 - w_0$ \\
19. $w_1 \leftarrow \lfloor w_1 / 2 \rfloor$, $w_3 \leftarrow \lfloor w_3 / 2 \rfloor$ \\
20. $w_2 \leftarrow w_2 - w_0$, $w_2 \leftarrow w_2 - w_4$ \\
21. $w_1 \leftarrow w_1 - w_2$, $w_3 \leftarrow w_3 - w_2$ \\
22. $tmp_1 \leftarrow 8 \cdot w_0$, $w_1 \leftarrow w_1 - tmp_1$, $tmp_1 \leftarrow 8 \cdot w_4$, $w_3 \leftarrow w_3 - tmp_1$ \\
23. $w_2 \leftarrow 3 \cdot w_2$, $w_2 \leftarrow w_2 - w_1$, $w_2 \leftarrow w_2 - w_3$ \\
24. $w_1 \leftarrow w_1 - w_2$, $w_3 \leftarrow w_3 - w_2$ \\
25. $w_1 \leftarrow \lfloor w_1 / 3 \rfloor, w_3 \leftarrow \lfloor w_3 / 3 \rfloor$ \\
\\
Now substitute $\beta^k$ for $x$ by shifting $w_0, w_1, ..., w_4$. \\
26. for $n$ from $1$ to $4$ do \\
\hspace{3mm}26.1  $w_n \leftarrow w_n \cdot \beta^{nk}$ \\
27. $c \leftarrow w_0 + w_1$, $c \leftarrow c + w_2$, $c \leftarrow c + w_3$, $c \leftarrow c + w_4$ \\
28. Return(\textit{MP\_OKAY}) \\
\hline
\end{tabular}
\end{center}
\end{small}
\caption{Algorithm mp\_toom\_mul (continued)}
\end{figure}

\textbf{Algorithm mp\_toom\_mul.}
This algorithm computes the product of two mp\_int variables $a$ and $b$ using the Toom-Cook approach.  Compared to the Karatsuba multiplication, this 
algorithm has a lower asymptotic running time of approximately $O(n^{1.464})$ but at an obvious cost in overhead.  In this
description, several statements have been compounded to save space.  The intention is that the statements are executed from left to right across
any given step.

The two inputs $a$ and $b$ are first split into three $k$-digit integers $a_0, a_1, a_2$ and $b_0, b_1, b_2$ respectively.  From these smaller
integers the coefficients of the polynomial basis representations $f(x)$ and $g(x)$ are known and can be used to find the relations required.

The first two relations $w_0$ and $w_4$ are the points $\zeta_{0}$ and $\zeta_{\infty}$ respectively.  The relation $w_1, w_2$ and $w_3$ correspond
to the points $16 \cdot \zeta_{1 \over 2}, \zeta_{2}$ and $\zeta_{1}$ respectively.  These are found using logical shifts to independently find
$f(y)$ and $g(y)$ which significantly speeds up the algorithm.

After the five relations $w_0, w_1, \ldots, w_4$ have been computed, the system they represent must be solved in order for the unknown coefficients 
$w_1, w_2$ and $w_3$ to be isolated.  The steps 18 through 25 perform the system reduction required as previously described.  Each step of
the reduction represents the comparable matrix operation that would be performed had this been performed by pencil.  For example, step 18 indicates
that row $1$ must be subtracted from row $4$ and simultaneously row $0$ subtracted from row $3$.  

Once the coeffients have been isolated, the polynomial $W(x) = \sum_{i=0}^{2n} w_i x^i$ is known.  By substituting $\beta^{k}$ for $x$, the integer 
result $a \cdot b$ is produced.

\vspace{+3mm}\begin{small}
\hspace{-5.1mm}{\bf File}: bn\_mp\_toom\_mul.c
\vspace{-3mm}
\begin{alltt}
\end{alltt}
\end{small}

The first obvious thing to note is that this algorithm is complicated.  The complexity is worth it if you are multiplying very 
large numbers.  For example, a 10,000 digit multiplication takes approximaly 99,282,205 fewer single precision multiplications with
Toom--Cook than a Comba or baseline approach (this is a savings of more than 99$\%$).  For most ``crypto'' sized numbers this
algorithm is not practical as Karatsuba has a much lower cutoff point.

First we split $a$ and $b$ into three roughly equal portions.  This has been accomplished (lines 41 to 70) with 
combinations of mp\_rshd() and mp\_mod\_2d() function calls.  At this point $a = a2 \cdot \beta^2 + a1 \cdot \beta + a0$ and similiarly
for $b$.  

Next we compute the five points $w0, w1, w2, w3$ and $w4$.  Recall that $w0$ and $w4$ can be computed directly from the portions so
we get those out of the way first (lines 73 and 78).  Next we compute $w1, w2$ and $w3$ using Horners method.

After this point we solve for the actual values of $w1, w2$ and $w3$ by reducing the $5 \times 5$ system which is relatively
straight forward.  

\subsection{Signed Multiplication}
Now that algorithms to handle multiplications of every useful dimensions have been developed, a rather simple finishing touch is required.  So far all
of the multiplication algorithms have been unsigned multiplications which leaves only a signed multiplication algorithm to be established.  

\begin{figure}[!here]
\begin{small}
\begin{center}
\begin{tabular}{l}
\hline Algorithm \textbf{mp\_mul}. \\
\textbf{Input}.   mp\_int $a$ and mp\_int $b$ \\
\textbf{Output}.  $c \leftarrow a \cdot b$ \\
\hline \\
1.  If $a.sign = b.sign$ then \\
\hspace{3mm}1.1  $sign = MP\_ZPOS$ \\
2.  else \\
\hspace{3mm}2.1  $sign = MP\_ZNEG$ \\
3.  If min$(a.used, b.used) \ge TOOM\_MUL\_CUTOFF$ then  \\
\hspace{3mm}3.1  $c \leftarrow a \cdot b$ using algorithm mp\_toom\_mul \\
4.  else if min$(a.used, b.used) \ge KARATSUBA\_MUL\_CUTOFF$ then \\
\hspace{3mm}4.1  $c \leftarrow a \cdot b$ using algorithm mp\_karatsuba\_mul \\
5.  else \\
\hspace{3mm}5.1  $digs \leftarrow a.used + b.used + 1$ \\
\hspace{3mm}5.2  If $digs < MP\_ARRAY$ and min$(a.used, b.used) \le \delta$ then \\
\hspace{6mm}5.2.1  $c \leftarrow a \cdot b \mbox{ (mod }\beta^{digs}\mbox{)}$ using algorithm fast\_s\_mp\_mul\_digs.  \\
\hspace{3mm}5.3  else \\
\hspace{6mm}5.3.1  $c \leftarrow a \cdot b \mbox{ (mod }\beta^{digs}\mbox{)}$ using algorithm s\_mp\_mul\_digs.  \\
6.  $c.sign \leftarrow sign$ \\
7.  Return the result of the unsigned multiplication performed. \\
\hline
\end{tabular}
\end{center}
\end{small}
\caption{Algorithm mp\_mul}
\end{figure}

\textbf{Algorithm mp\_mul.}
This algorithm performs the signed multiplication of two inputs.  It will make use of any of the three unsigned multiplication algorithms 
available when the input is of appropriate size.  The \textbf{sign} of the result is not set until the end of the algorithm since algorithm
s\_mp\_mul\_digs will clear it.  

\vspace{+3mm}\begin{small}
\hspace{-5.1mm}{\bf File}: bn\_mp\_mul.c
\vspace{-3mm}
\begin{alltt}
\end{alltt}
\end{small}

The implementation is rather simplistic and is not particularly noteworthy.  Line 22 computes the sign of the result using the ``?'' 
operator from the C programming language.  Line 48 computes $\delta$ using the fact that $1 << k$ is equal to $2^k$.  

\section{Squaring}
\label{sec:basesquare}

Squaring is a special case of multiplication where both multiplicands are equal.  At first it may seem like there is no significant optimization
available but in fact there is.  Consider the multiplication of $576$ against $241$.  In total there will be nine single precision multiplications
performed which are $1\cdot 6$, $1 \cdot 7$, $1 \cdot 5$, $4 \cdot 6$, $4 \cdot 7$, $4 \cdot 5$, $2 \cdot  6$, $2 \cdot 7$ and $2 \cdot 5$.  Now consider 
the multiplication of $123$ against $123$.  The nine products are $3 \cdot 3$, $3 \cdot 2$, $3 \cdot 1$, $2 \cdot 3$, $2 \cdot 2$, $2 \cdot 1$, 
$1 \cdot 3$, $1 \cdot 2$ and $1 \cdot 1$.  On closer inspection some of the products are equivalent.  For example, $3 \cdot 2 = 2 \cdot 3$ 
and $3 \cdot 1 = 1 \cdot 3$. 

For any $n$-digit input, there are ${{\left (n^2 + n \right)}\over 2}$ possible unique single precision multiplications required compared to the $n^2$
required for multiplication.  The following diagram gives an example of the operations required.

\begin{figure}[here]
\begin{center}
\begin{tabular}{ccccc|c}
&&1&2&3&\\
$\times$ &&1&2&3&\\
\hline && $3 \cdot 1$ & $3 \cdot 2$ & $3 \cdot 3$ & Row 0\\
       & $2 \cdot 1$  & $2 \cdot 2$ & $2 \cdot 3$ && Row 1 \\
         $1 \cdot 1$  & $1 \cdot 2$ & $1 \cdot 3$ &&& Row 2 \\
\end{tabular}
\end{center}
\caption{Squaring Optimization Diagram}
\end{figure}

Starting from zero and numbering the columns from right to left a very simple pattern becomes obvious.  For the purposes of this discussion let $x$
represent the number being squared.  The first observation is that in row $k$ the $2k$'th column of the product has a $\left (x_k \right)^2$ term in it.  

The second observation is that every column $j$ in row $k$ where $j \ne 2k$ is part of a double product.  Every non-square term of a column will
appear twice hence the name ``double product''.  Every odd column is made up entirely of double products.  In fact every column is made up of double 
products and at most one square (\textit{see the exercise section}).  

The third and final observation is that for row $k$ the first unique non-square term, that is, one that hasn't already appeared in an earlier row, 
occurs at column $2k + 1$.  For example, on row $1$ of the previous squaring, column one is part of the double product with column one from row zero. 
Column two of row one is a square and column three is the first unique column.

\subsection{The Baseline Squaring Algorithm}
The baseline squaring algorithm is meant to be a catch-all squaring algorithm.  It will handle any of the input sizes that the faster routines
will not handle.  

\begin{figure}[!here]
\begin{small}
\begin{center}
\begin{tabular}{l}
\hline Algorithm \textbf{s\_mp\_sqr}. \\
\textbf{Input}.   mp\_int $a$ \\
\textbf{Output}.  $b \leftarrow a^2$ \\
\hline \\
1.  Init a temporary mp\_int of at least $2 \cdot a.used +1$ digits.  (\textit{mp\_init\_size}) \\
2.  If step 1 failed return(\textit{MP\_MEM}) \\
3.  $t.used \leftarrow 2 \cdot a.used + 1$ \\
4.  For $ix$ from 0 to $a.used - 1$ do \\
\hspace{3mm}Calculate the square. \\
\hspace{3mm}4.1  $\hat r \leftarrow t_{2ix} + \left (a_{ix} \right )^2$ \\
\hspace{3mm}4.2  $t_{2ix} \leftarrow \hat r \mbox{ (mod }\beta\mbox{)}$ \\
\hspace{3mm}Calculate the double products after the square. \\
\hspace{3mm}4.3  $u \leftarrow \lfloor \hat r / \beta \rfloor$ \\
\hspace{3mm}4.4  For $iy$ from $ix + 1$ to $a.used - 1$ do \\
\hspace{6mm}4.4.1  $\hat r \leftarrow 2 \cdot a_{ix}a_{iy} + t_{ix + iy} + u$ \\
\hspace{6mm}4.4.2  $t_{ix + iy} \leftarrow \hat r \mbox{ (mod }\beta\mbox{)}$ \\
\hspace{6mm}4.4.3  $u \leftarrow \lfloor \hat r / \beta \rfloor$ \\
\hspace{3mm}Set the last carry. \\
\hspace{3mm}4.5  While $u > 0$ do \\
\hspace{6mm}4.5.1  $iy \leftarrow iy + 1$ \\
\hspace{6mm}4.5.2  $\hat r \leftarrow t_{ix + iy} + u$ \\
\hspace{6mm}4.5.3  $t_{ix + iy} \leftarrow \hat r \mbox{ (mod }\beta\mbox{)}$ \\
\hspace{6mm}4.5.4  $u \leftarrow \lfloor \hat r / \beta \rfloor$ \\
5.  Clamp excess digits of $t$.  (\textit{mp\_clamp}) \\
6.  Exchange $b$ and $t$. \\
7.  Clear $t$ (\textit{mp\_clear}) \\
8.  Return(\textit{MP\_OKAY}) \\
\hline
\end{tabular}
\end{center}
\end{small}
\caption{Algorithm s\_mp\_sqr}
\end{figure}

\textbf{Algorithm s\_mp\_sqr.}
This algorithm computes the square of an input using the three observations on squaring.  It is based fairly faithfully on  algorithm 14.16 of HAC
\cite[pp.596-597]{HAC}.  Similar to algorithm s\_mp\_mul\_digs, a temporary mp\_int is allocated to hold the result of the squaring.  This allows the 
destination mp\_int to be the same as the source mp\_int.

The outer loop of this algorithm begins on step 4. It is best to think of the outer loop as walking down the rows of the partial results, while
the inner loop computes the columns of the partial result.  Step 4.1 and 4.2 compute the square term for each row, and step 4.3 and 4.4 propagate
the carry and compute the double products.  

The requirement that a mp\_word be able to represent the range $0 \le x < 2 \beta^2$ arises from this
very algorithm.  The product $a_{ix}a_{iy}$ will lie in the range $0 \le x \le \beta^2 - 2\beta + 1$ which is obviously less than $\beta^2$ meaning that
when it is multiplied by two, it can be properly represented by a mp\_word.

Similar to algorithm s\_mp\_mul\_digs, after every pass of the inner loop, the destination is correctly set to the sum of all of the partial 
results calculated so far.  This involves expensive carry propagation which will be eliminated in the next algorithm.  

\vspace{+3mm}\begin{small}
\hspace{-5.1mm}{\bf File}: bn\_s\_mp\_sqr.c
\vspace{-3mm}
\begin{alltt}
\end{alltt}
\end{small}

Inside the outer loop (line 34) the square term is calculated on line 37.  The carry (line 44) has been
extracted from the mp\_word accumulator using a right shift.  Aliases for $a_{ix}$ and $t_{ix+iy}$ are initialized 
(lines 47 and 50) to simplify the inner loop.  The doubling is performed using two
additions (line 59) since it is usually faster than shifting, if not at least as fast.  

The important observation is that the inner loop does not begin at $iy = 0$ like for multiplication.  As such the inner loops
get progressively shorter as the algorithm proceeds.  This is what leads to the savings compared to using a multiplication to
square a number. 

\subsection{Faster Squaring by the ``Comba'' Method}
A major drawback to the baseline method is the requirement for single precision shifting inside the $O(n^2)$ nested loop.  Squaring has an additional
drawback that it must double the product inside the inner loop as well.  As for multiplication, the Comba technique can be used to eliminate these
performance hazards.

The first obvious solution is to make an array of mp\_words which will hold all of the columns.  This will indeed eliminate all of the carry
propagation operations from the inner loop.  However, the inner product must still be doubled $O(n^2)$ times.  The solution stems from the simple fact
that $2a + 2b + 2c = 2(a + b + c)$.  That is the sum of all of the double products is equal to double the sum of all the products.  For example,
$ab + ba + ac + ca = 2ab + 2ac = 2(ab + ac)$.  

However, we cannot simply double all of the columns, since the squares appear only once per row.  The most practical solution is to have two 
mp\_word arrays.  One array will hold the squares and the other array will hold the double products.  With both arrays the doubling and 
carry propagation can be moved to a $O(n)$ work level outside the $O(n^2)$ level.  In this case, we have an even simpler solution in mind.

\newpage\begin{figure}[!here]
\begin{small}
\begin{center}
\begin{tabular}{l}
\hline Algorithm \textbf{fast\_s\_mp\_sqr}. \\
\textbf{Input}.   mp\_int $a$ \\
\textbf{Output}.  $b \leftarrow a^2$ \\
\hline \\
Place an array of \textbf{MP\_WARRAY} mp\_digits named $W$ on the stack. \\
1.  If $b.alloc < 2a.used + 1$ then grow $b$ to $2a.used + 1$ digits.  (\textit{mp\_grow}). \\
2.  If step 1 failed return(\textit{MP\_MEM}). \\
\\
3.  $pa \leftarrow 2 \cdot a.used$ \\
4.  $\hat W1 \leftarrow 0$ \\
5.  for $ix$ from $0$ to $pa - 1$ do \\
\hspace{3mm}5.1  $\_ \hat W \leftarrow 0$ \\
\hspace{3mm}5.2  $ty \leftarrow \mbox{MIN}(a.used - 1, ix)$ \\
\hspace{3mm}5.3  $tx \leftarrow ix - ty$ \\
\hspace{3mm}5.4  $iy \leftarrow \mbox{MIN}(a.used - tx, ty + 1)$ \\
\hspace{3mm}5.5  $iy \leftarrow \mbox{MIN}(iy, \lfloor \left (ty - tx + 1 \right )/2 \rfloor)$ \\
\hspace{3mm}5.6  for $iz$ from $0$ to $iz - 1$ do \\
\hspace{6mm}5.6.1  $\_ \hat W \leftarrow \_ \hat W + a_{tx + iz}a_{ty - iz}$ \\
\hspace{3mm}5.7  $\_ \hat W \leftarrow 2 \cdot \_ \hat W  + \hat W1$ \\
\hspace{3mm}5.8  if $ix$ is even then \\
\hspace{6mm}5.8.1  $\_ \hat W \leftarrow \_ \hat W + \left ( a_{\lfloor ix/2 \rfloor}\right )^2$ \\
\hspace{3mm}5.9  $W_{ix} \leftarrow \_ \hat W (\mbox{mod }\beta)$ \\
\hspace{3mm}5.10  $\hat W1 \leftarrow \lfloor \_ \hat W / \beta \rfloor$ \\
\\
6.  $oldused \leftarrow b.used$ \\
7.  $b.used \leftarrow 2 \cdot a.used$ \\
8.  for $ix$ from $0$ to $pa - 1$ do \\
\hspace{3mm}8.1  $b_{ix} \leftarrow W_{ix}$ \\
9.  for $ix$ from $pa$ to $oldused - 1$ do \\
\hspace{3mm}9.1  $b_{ix} \leftarrow 0$ \\
10.  Clamp excess digits from $b$.  (\textit{mp\_clamp}) \\
11.  Return(\textit{MP\_OKAY}). \\ 
\hline
\end{tabular}
\end{center}
\end{small}
\caption{Algorithm fast\_s\_mp\_sqr}
\end{figure}

\textbf{Algorithm fast\_s\_mp\_sqr.}
This algorithm computes the square of an input using the Comba technique.  It is designed to be a replacement for algorithm 
s\_mp\_sqr when the number of input digits is less than \textbf{MP\_WARRAY} and less than $\delta \over 2$.  
This algorithm is very similar to the Comba multiplier except with a few key differences we shall make note of.

First, we have an accumulator and carry variables $\_ \hat W$ and $\hat W1$ respectively.  This is because the inner loop
products are to be doubled.  If we had added the previous carry in we would be doubling too much.  Next we perform an
addition MIN condition on $iy$ (step 5.5) to prevent overlapping digits.  For example, $a_3 \cdot a_5$ is equal
$a_5 \cdot a_3$.  Whereas in the multiplication case we would have $5 < a.used$ and $3 \ge 0$ is maintained since we double the sum
of the products just outside the inner loop we have to avoid doing this.  This is also a good thing since we perform
fewer multiplications and the routine ends up being faster.

Finally the last difference is the addition of the ``square'' term outside the inner loop (step 5.8).  We add in the square
only to even outputs and it is the square of the term at the $\lfloor ix / 2 \rfloor$ position.

\vspace{+3mm}\begin{small}
\hspace{-5.1mm}{\bf File}: bn\_fast\_s\_mp\_sqr.c
\vspace{-3mm}
\begin{alltt}
\end{alltt}
\end{small}

This implementation is essentially a copy of Comba multiplication with the appropriate changes added to make it faster for 
the special case of squaring.  

\subsection{Polynomial Basis Squaring}
The same algorithm that performs optimal polynomial basis multiplication can be used to perform polynomial basis squaring.  The minor exception
is that $\zeta_y = f(y)g(y)$ is actually equivalent to $\zeta_y = f(y)^2$ since $f(y) = g(y)$.  Instead of performing $2n + 1$
multiplications to find the $\zeta$ relations, squaring operations are performed instead.  

\subsection{Karatsuba Squaring}
Let $f(x) = ax + b$ represent the polynomial basis representation of a number to square.  
Let $h(x) = \left ( f(x) \right )^2$ represent the square of the polynomial.  The Karatsuba equation can be modified to square a 
number with the following equation.

\begin{equation}
h(x) = a^2x^2 + \left ((a + b)^2 - (a^2 + b^2) \right )x + b^2
\end{equation}

Upon closer inspection this equation only requires the calculation of three half-sized squares: $a^2$, $b^2$ and $(a + b)^2$.  As in 
Karatsuba multiplication, this algorithm can be applied recursively on the input and will achieve an asymptotic running time of 
$O \left ( n^{lg(3)} \right )$.

If the asymptotic times of Karatsuba squaring and multiplication are the same, why not simply use the multiplication algorithm 
instead?  The answer to this arises from the cutoff point for squaring.  As in multiplication there exists a cutoff point, at which the 
time required for a Comba based squaring and a Karatsuba based squaring meet.  Due to the overhead inherent in the Karatsuba method, the cutoff 
point is fairly high.  For example, on an AMD Athlon XP processor with $\beta = 2^{28}$, the cutoff point is around 127 digits.  

Consider squaring a 200 digit number with this technique.  It will be split into two 100 digit halves which are subsequently squared.  
The 100 digit halves will not be squared using Karatsuba, but instead using the faster Comba based squaring algorithm.  If Karatsuba multiplication
were used instead, the 100 digit numbers would be squared with a slower Comba based multiplication.  

\newpage\begin{figure}[!here]
\begin{small}
\begin{center}
\begin{tabular}{l}
\hline Algorithm \textbf{mp\_karatsuba\_sqr}. \\
\textbf{Input}.   mp\_int $a$ \\
\textbf{Output}.  $b \leftarrow a^2$ \\
\hline \\
1.  Initialize the following temporary mp\_ints:  $x0$, $x1$, $t1$, $t2$, $x0x0$ and $x1x1$. \\
2.  If any of the initializations on step 1 failed return(\textit{MP\_MEM}). \\
\\
Split the input.  e.g. $a = x1\beta^B + x0$ \\
3.  $B \leftarrow \lfloor a.used / 2 \rfloor$ \\
4.  $x0 \leftarrow a \mbox{ (mod }\beta^B\mbox{)}$ (\textit{mp\_mod\_2d}) \\
5.  $x1 \leftarrow \lfloor a / \beta^B \rfloor$ (\textit{mp\_lshd}) \\
\\
Calculate the three squares. \\
6.  $x0x0 \leftarrow x0^2$ (\textit{mp\_sqr}) \\
7.  $x1x1 \leftarrow x1^2$ \\
8.  $t1 \leftarrow x1 + x0$ (\textit{s\_mp\_add}) \\
9.  $t1 \leftarrow t1^2$ \\
\\
Compute the middle term. \\
10.  $t2 \leftarrow x0x0 + x1x1$ (\textit{s\_mp\_add}) \\
11.  $t1 \leftarrow t1 - t2$ \\
\\
Compute final product. \\
12.  $t1 \leftarrow t1\beta^B$ (\textit{mp\_lshd}) \\
13.  $x1x1 \leftarrow x1x1\beta^{2B}$ \\
14.  $t1 \leftarrow t1 + x0x0$ \\
15.  $b \leftarrow t1 + x1x1$ \\
16.  Return(\textit{MP\_OKAY}). \\
\hline
\end{tabular}
\end{center}
\end{small}
\caption{Algorithm mp\_karatsuba\_sqr}
\end{figure}

\textbf{Algorithm mp\_karatsuba\_sqr.}
This algorithm computes the square of an input $a$ using the Karatsuba technique.  This algorithm is very similar to the Karatsuba based
multiplication algorithm with the exception that the three half-size multiplications have been replaced with three half-size squarings.

The radix point for squaring is simply placed exactly in the middle of the digits when the input has an odd number of digits, otherwise it is
placed just below the middle.  Step 3, 4 and 5 compute the two halves required using $B$
as the radix point.  The first two squares in steps 6 and 7 are rather straightforward while the last square is of a more compact form.

By expanding $\left (x1 + x0 \right )^2$, the $x1^2$ and $x0^2$ terms in the middle disappear, that is $(x0 - x1)^2 - (x1^2 + x0^2)  = 2 \cdot x0 \cdot x1$.
Now if $5n$ single precision additions and a squaring of $n$-digits is faster than multiplying two $n$-digit numbers and doubling then
this method is faster.  Assuming no further recursions occur, the difference can be estimated with the following inequality.

Let $p$ represent the cost of a single precision addition and $q$ the cost of a single precision multiplication both in terms of time\footnote{Or
machine clock cycles.}. 

\begin{equation}
5pn +{{q(n^2 + n)} \over 2} \le pn + qn^2
\end{equation}

For example, on an AMD Athlon XP processor $p = {1 \over 3}$ and $q = 6$.  This implies that the following inequality should hold.
\begin{center}
\begin{tabular}{rcl}
${5n \over 3} + 3n^2 + 3n$     & $<$ & ${n \over 3} + 6n^2$ \\
${5 \over 3} + 3n + 3$     & $<$ & ${1 \over 3} + 6n$ \\
${13 \over 9}$     & $<$ & $n$ \\
\end{tabular}
\end{center}

This results in a cutoff point around $n = 2$.  As a consequence it is actually faster to compute the middle term the ``long way'' on processors
where multiplication is substantially slower\footnote{On the Athlon there is a 1:17 ratio between clock cycles for addition and multiplication.  On
the Intel P4 processor this ratio is 1:29 making this method even more beneficial.  The only common exception is the ARMv4 processor which has a
ratio of 1:7.  } than simpler operations such as addition.  

\vspace{+3mm}\begin{small}
\hspace{-5.1mm}{\bf File}: bn\_mp\_karatsuba\_sqr.c
\vspace{-3mm}
\begin{alltt}
\end{alltt}
\end{small}

This implementation is largely based on the implementation of algorithm mp\_karatsuba\_mul.  It uses the same inline style to copy and 
shift the input into the two halves.  The loop from line 54 to line 70 has been modified since only one input exists.  The \textbf{used}
count of both $x0$ and $x1$ is fixed up and $x0$ is clamped before the calculations begin.  At this point $x1$ and $x0$ are valid equivalents
to the respective halves as if mp\_rshd and mp\_mod\_2d had been used.  

By inlining the copy and shift operations the cutoff point for Karatsuba multiplication can be lowered.  On the Athlon the cutoff point
is exactly at the point where Comba squaring can no longer be used (\textit{128 digits}).  On slower processors such as the Intel P4
it is actually below the Comba limit (\textit{at 110 digits}).

This routine uses the same error trap coding style as mp\_karatsuba\_sqr.  As the temporary variables are initialized errors are 
redirected to the error trap higher up.  If the algorithm completes without error the error code is set to \textbf{MP\_OKAY} and 
mp\_clears are executed normally.

\subsection{Toom-Cook Squaring}
The Toom-Cook squaring algorithm mp\_toom\_sqr is heavily based on the algorithm mp\_toom\_mul with the exception that squarings are used
instead of multiplication to find the five relations.  The reader is encouraged to read the description of the latter algorithm and try to 
derive their own Toom-Cook squaring algorithm.  

\subsection{High Level Squaring}
\newpage\begin{figure}[!here]
\begin{small}
\begin{center}
\begin{tabular}{l}
\hline Algorithm \textbf{mp\_sqr}. \\
\textbf{Input}.   mp\_int $a$ \\
\textbf{Output}.  $b \leftarrow a^2$ \\
\hline \\
1.  If $a.used \ge TOOM\_SQR\_CUTOFF$ then  \\
\hspace{3mm}1.1  $b \leftarrow a^2$ using algorithm mp\_toom\_sqr \\
2.  else if $a.used \ge KARATSUBA\_SQR\_CUTOFF$ then \\
\hspace{3mm}2.1  $b \leftarrow a^2$ using algorithm mp\_karatsuba\_sqr \\
3.  else \\
\hspace{3mm}3.1  $digs \leftarrow a.used + b.used + 1$ \\
\hspace{3mm}3.2  If $digs < MP\_ARRAY$ and $a.used \le \delta$ then \\
\hspace{6mm}3.2.1  $b \leftarrow a^2$ using algorithm fast\_s\_mp\_sqr.  \\
\hspace{3mm}3.3  else \\
\hspace{6mm}3.3.1  $b \leftarrow a^2$ using algorithm s\_mp\_sqr.  \\
4.  $b.sign \leftarrow MP\_ZPOS$ \\
5.  Return the result of the unsigned squaring performed. \\
\hline
\end{tabular}
\end{center}
\end{small}
\caption{Algorithm mp\_sqr}
\end{figure}

\textbf{Algorithm mp\_sqr.}
This algorithm computes the square of the input using one of four different algorithms.  If the input is very large and has at least
\textbf{TOOM\_SQR\_CUTOFF} or \textbf{KARATSUBA\_SQR\_CUTOFF} digits then either the Toom-Cook or the Karatsuba Squaring algorithm is used.  If
neither of the polynomial basis algorithms should be used then either the Comba or baseline algorithm is used.  

\vspace{+3mm}\begin{small}
\hspace{-5.1mm}{\bf File}: bn\_mp\_sqr.c
\vspace{-3mm}
\begin{alltt}
\end{alltt}
\end{small}

\section*{Exercises}
\begin{tabular}{cl}
$\left [ 3 \right ] $ & Devise an efficient algorithm for selection of the radix point to handle inputs \\
                      & that have different number of digits in Karatsuba multiplication. \\
                      & \\
$\left [ 2 \right ] $ & In section 5.3 the fact that every column of a squaring is made up \\
                      & of double products and at most one square is stated.  Prove this statement. \\
                      & \\                      
$\left [ 3 \right ] $ & Prove the equation for Karatsuba squaring. \\
                      & \\
$\left [ 1 \right ] $ & Prove that Karatsuba squaring requires $O \left (n^{lg(3)} \right )$ time. \\
                      & \\ 
$\left [ 2 \right ] $ & Determine the minimal ratio between addition and multiplication clock cycles \\
                      & required for equation $6.7$ to be true.  \\
                      & \\
$\left [ 3 \right ] $ & Implement a threaded version of Comba multiplication (and squaring) where you \\
                      & compute subsets of the columns in each thread.  Determine a cutoff point where \\
                      & it is effective and add the logic to mp\_mul() and mp\_sqr(). \\
                      &\\
$\left [ 4 \right ] $ & Same as the previous but also modify the Karatsuba and Toom-Cook.  You must \\
                      & increase the throughput of mp\_exptmod() for random odd moduli in the range \\
                      & $512 \ldots 4096$ bits significantly ($> 2x$) to complete this challenge. \\
                      & \\
\end{tabular}

\chapter{Modular Reduction}
\section{Basics of Modular Reduction}
\index{modular residue}
Modular reduction is an operation that arises quite often within public key cryptography algorithms and various number theoretic algorithms, 
such as factoring.  Modular reduction algorithms are the third class of algorithms of the ``multipliers'' set.  A number $a$ is said to be \textit{reduced}
modulo another number $b$ by finding the remainder of the division $a/b$.  Full integer division with remainder is a topic to be covered 
in~\ref{sec:division}.

Modular reduction is equivalent to solving for $r$ in the following equation.  $a = bq + r$ where $q = \lfloor a/b \rfloor$.  The result 
$r$ is said to be ``congruent to $a$ modulo $b$'' which is also written as $r \equiv a \mbox{ (mod }b\mbox{)}$.  In other vernacular $r$ is known as the 
``modular residue'' which leads to ``quadratic residue''\footnote{That's fancy talk for $b \equiv a^2 \mbox{ (mod }p\mbox{)}$.} and
other forms of residues.  

Modular reductions are normally used to create either finite groups, rings or fields.  The most common usage for performance driven modular reductions 
is in modular exponentiation algorithms.  That is to compute $d = a^b \mbox{ (mod }c\mbox{)}$ as fast as possible.  This operation is used in the 
RSA and Diffie-Hellman public key algorithms, for example.  Modular multiplication and squaring also appears as a fundamental operation in 
elliptic curve cryptographic algorithms.  As will be discussed in the subsequent chapter there exist fast algorithms for computing modular 
exponentiations without having to perform (\textit{in this example}) $b - 1$ multiplications.  These algorithms will produce partial results in the 
range $0 \le x < c^2$ which can be taken advantage of to create several efficient algorithms.   They have also been used to create redundancy check 
algorithms known as CRCs, error correction codes such as Reed-Solomon and solve a variety of number theoeretic problems.  

\section{The Barrett Reduction}
The Barrett reduction algorithm \cite{BARRETT} was inspired by fast division algorithms which multiply by the reciprocal to emulate
division.  Barretts observation was that the residue $c$ of $a$ modulo $b$ is equal to 

\begin{equation}
c = a - b \cdot \lfloor a/b \rfloor
\end{equation}

Since algorithms such as modular exponentiation would be using the same modulus extensively, typical DSP\footnote{It is worth noting that Barrett's paper 
targeted the DSP56K processor.}  intuition would indicate the next step would be to replace $a/b$ by a multiplication by the reciprocal.  However, 
DSP intuition on its own will not work as these numbers are considerably larger than the precision of common DSP floating point data types.  
It would take another common optimization to optimize the algorithm.

\subsection{Fixed Point Arithmetic}
The trick used to optimize the above equation is based on a technique of emulating floating point data types with fixed precision integers.  Fixed
point arithmetic would become very popular as it greatly optimize the ``3d-shooter'' genre of games in the mid 1990s when floating point units were 
fairly slow if not unavailable.   The idea behind fixed point arithmetic is to take a normal $k$-bit integer data type and break it into $p$-bit 
integer and a $q$-bit fraction part (\textit{where $p+q = k$}).  

In this system a $k$-bit integer $n$ would actually represent $n/2^q$.  For example, with $q = 4$ the integer $n = 37$ would actually represent the
value $2.3125$.  To multiply two fixed point numbers the integers are multiplied using traditional arithmetic and subsequently normalized by 
moving the implied decimal point back to where it should be.  For example, with $q = 4$ to multiply the integers $9$ and $5$ they must be converted 
to fixed point first by multiplying by $2^q$.  Let $a = 9(2^q)$ represent the fixed point representation of $9$ and $b = 5(2^q)$ represent the 
fixed point representation of $5$.  The product $ab$ is equal to $45(2^{2q})$ which when normalized by dividing by $2^q$ produces $45(2^q)$.  

This technique became popular since a normal integer multiplication and logical shift right are the only required operations to perform a multiplication
of two fixed point numbers.  Using fixed point arithmetic, division can be easily approximated by multiplying by the reciprocal.  If $2^q$ is 
equivalent to one than $2^q/b$ is equivalent to the fixed point approximation of $1/b$ using real arithmetic.  Using this fact dividing an integer 
$a$ by another integer $b$ can be achieved with the following expression.

\begin{equation}
\lfloor a / b \rfloor \mbox{ }\approx\mbox{ } \lfloor (a \cdot \lfloor 2^q / b \rfloor)/2^q \rfloor
\end{equation}

The precision of the division is proportional to the value of $q$.  If the divisor $b$ is used frequently as is the case with 
modular exponentiation pre-computing $2^q/b$ will allow a division to be performed with a multiplication and a right shift.  Both operations
are considerably faster than division on most processors.  

Consider dividing $19$ by $5$.  The correct result is $\lfloor 19/5 \rfloor = 3$.  With $q = 3$ the reciprocal is $\lfloor 2^q/5 \rfloor = 1$ which
leads to a product of $19$ which when divided by $2^q$ produces $2$.  However, with $q = 4$ the reciprocal is $\lfloor 2^q/5 \rfloor = 3$ and
the result of the emulated division is $\lfloor 3 \cdot 19 / 2^q \rfloor = 3$ which is correct.  The value of $2^q$ must be close to or ideally
larger than the dividend.  In effect if $a$ is the dividend then $q$ should allow $0 \le \lfloor a/2^q \rfloor \le 1$ in order for this approach
to work correctly.  Plugging this form of divison into the original equation the following modular residue equation arises.

\begin{equation}
c = a - b \cdot \lfloor (a \cdot \lfloor 2^q / b \rfloor)/2^q \rfloor
\end{equation}

Using the notation from \cite{BARRETT} the value of $\lfloor 2^q / b \rfloor$ will be represented by the $\mu$ symbol.  Using the $\mu$
variable also helps re-inforce the idea that it is meant to be computed once and re-used.

\begin{equation}
c = a - b \cdot \lfloor (a \cdot \mu)/2^q \rfloor
\end{equation}

Provided that $2^q \ge a$ this algorithm will produce a quotient that is either exactly correct or off by a value of one.  In the context of Barrett
reduction the value of $a$ is bound by $0 \le a \le (b - 1)^2$ meaning that $2^q \ge b^2$ is sufficient to ensure the reciprocal will have enough
precision.  

Let $n$ represent the number of digits in $b$.  This algorithm requires approximately $2n^2$ single precision multiplications to produce the quotient and 
another $n^2$ single precision multiplications to find the residue.  In total $3n^2$ single precision multiplications are required to 
reduce the number.  

For example, if $b = 1179677$ and $q = 41$ ($2^q > b^2$), then the reciprocal $\mu$ is equal to $\lfloor 2^q / b \rfloor = 1864089$.  Consider reducing
$a = 180388626447$ modulo $b$ using the above reduction equation.  The quotient using the new formula is $\lfloor (a \cdot \mu) / 2^q \rfloor = 152913$.
By subtracting $152913b$ from $a$ the correct residue $a \equiv 677346 \mbox{ (mod }b\mbox{)}$ is found.

\subsection{Choosing a Radix Point}
Using the fixed point representation a modular reduction can be performed with $3n^2$ single precision multiplications.  If that were the best
that could be achieved a full division\footnote{A division requires approximately $O(2cn^2)$ single precision multiplications for a small value of $c$.  
See~\ref{sec:division} for further details.} might as well be used in its place.  The key to optimizing the reduction is to reduce the precision of
the initial multiplication that finds the quotient.  

Let $a$ represent the number of which the residue is sought.  Let $b$ represent the modulus used to find the residue.  Let $m$ represent
the number of digits in $b$.  For the purposes of this discussion we will assume that the number of digits in $a$ is $2m$, which is generally true if 
two $m$-digit numbers have been multiplied.  Dividing $a$ by $b$ is the same as dividing a $2m$ digit integer by a $m$ digit integer.  Digits below the 
$m - 1$'th digit of $a$ will contribute at most a value of $1$ to the quotient because $\beta^k < b$ for any $0 \le k \le m - 1$.  Another way to
express this is by re-writing $a$ as two parts.  If $a' \equiv a \mbox{ (mod }b^m\mbox{)}$ and $a'' = a - a'$ then 
${a \over b} \equiv {{a' + a''} \over b}$ which is equivalent to ${a' \over b} + {a'' \over b}$.  Since $a'$ is bound to be less than $b$ the quotient
is bound by $0 \le {a' \over b} < 1$.

Since the digits of $a'$ do not contribute much to the quotient the observation is that they might as well be zero.  However, if the digits 
``might as well be zero'' they might as well not be there in the first place.  Let $q_0 = \lfloor a/\beta^{m-1} \rfloor$ represent the input
with the irrelevant digits trimmed.  Now the modular reduction is trimmed to the almost equivalent equation

\begin{equation}
c = a - b \cdot \lfloor (q_0 \cdot \mu) / \beta^{m+1} \rfloor
\end{equation}

Note that the original divisor $2^q$ has been replaced with $\beta^{m+1}$ where in this case $q$ is a multiple of $lg(\beta)$. Also note that the 
exponent on the divisor when added to the amount $q_0$ was shifted by equals $2m$.  If the optimization had not been performed the divisor 
would have the exponent $2m$ so in the end the exponents do ``add up''. Using the above equation the quotient 
$\lfloor (q_0 \cdot \mu) / \beta^{m+1} \rfloor$ can be off from the true quotient by at most two.  The original fixed point quotient can be off
by as much as one (\textit{provided the radix point is chosen suitably}) and now that the lower irrelevent digits have been trimmed the quotient
can be off by an additional value of one for a total of at most two.  This implies that 
$0 \le a - b \cdot \lfloor (q_0 \cdot \mu) / \beta^{m+1} \rfloor < 3b$.  By first subtracting $b$ times the quotient and then conditionally subtracting 
$b$ once or twice the residue is found.

The quotient is now found using $(m + 1)(m) = m^2 + m$ single precision multiplications and the residue with an additional $m^2$ single
precision multiplications, ignoring the subtractions required.  In total $2m^2 + m$ single precision multiplications are required to find the residue.  
This is considerably faster than the original attempt.

For example, let $\beta = 10$ represent the radix of the digits.  Let $b = 9999$ represent the modulus which implies $m = 4$. Let $a = 99929878$ 
represent the value of which the residue is desired.  In this case $q = 8$ since $10^7 < 9999^2$ meaning that $\mu = \lfloor \beta^{q}/b \rfloor = 10001$.  
With the new observation the multiplicand for the quotient is equal to $q_0 = \lfloor a / \beta^{m - 1} \rfloor = 99929$.  The quotient is then 
$\lfloor (q_0 \cdot \mu) / \beta^{m+1} \rfloor = 9993$.  Subtracting $9993b$ from $a$ and the correct residue $a \equiv 9871 \mbox{ (mod }b\mbox{)}$ 
is found.  

\subsection{Trimming the Quotient}
So far the reduction algorithm has been optimized from $3m^2$ single precision multiplications down to $2m^2 + m$ single precision multiplications.  As 
it stands now the algorithm is already fairly fast compared to a full integer division algorithm.  However, there is still room for
optimization.  

After the first multiplication inside the quotient ($q_0 \cdot \mu$) the value is shifted right by $m + 1$ places effectively nullifying the lower
half of the product.  It would be nice to be able to remove those digits from the product to effectively cut down the number of single precision 
multiplications.  If the number of digits in the modulus $m$ is far less than $\beta$ a full product is not required for the algorithm to work properly.  
In fact the lower $m - 2$ digits will not affect the upper half of the product at all and do not need to be computed.  

The value of $\mu$ is a $m$-digit number and $q_0$ is a $m + 1$ digit number.  Using a full multiplier $(m + 1)(m) = m^2 + m$ single precision
multiplications would be required.  Using a multiplier that will only produce digits at and above the $m - 1$'th digit reduces the number
of single precision multiplications to ${m^2 + m} \over 2$ single precision multiplications.  

\subsection{Trimming the Residue}
After the quotient has been calculated it is used to reduce the input.  As previously noted the algorithm is not exact and it can be off by a small
multiple of the modulus, that is $0 \le a - b \cdot \lfloor (q_0 \cdot \mu) / \beta^{m+1} \rfloor < 3b$.  If $b$ is $m$ digits than the 
result of reduction equation is a value of at most $m + 1$ digits (\textit{provided $3 < \beta$}) implying that the upper $m - 1$ digits are
implicitly zero.  

The next optimization arises from this very fact.  Instead of computing $b \cdot \lfloor (q_0 \cdot \mu) / \beta^{m+1} \rfloor$ using a full
$O(m^2)$ multiplication algorithm only the lower $m+1$ digits of the product have to be computed.  Similarly the value of $a$ can
be reduced modulo $\beta^{m+1}$ before the multiple of $b$ is subtracted which simplifes the subtraction as well.  A multiplication that produces 
only the lower $m+1$ digits requires ${m^2 + 3m - 2} \over 2$ single precision multiplications.  

With both optimizations in place the algorithm is the algorithm Barrett proposed.  It requires $m^2 + 2m - 1$ single precision multiplications which
is considerably faster than the straightforward $3m^2$ method.  

\subsection{The Barrett Algorithm}
\newpage\begin{figure}[!here]
\begin{small}
\begin{center}
\begin{tabular}{l}
\hline Algorithm \textbf{mp\_reduce}. \\
\textbf{Input}.   mp\_int $a$, mp\_int $b$ and $\mu = \lfloor \beta^{2m}/b \rfloor, m = \lceil lg_{\beta}(b) \rceil, (0 \le a < b^2, b > 1)$ \\
\textbf{Output}.  $a \mbox{ (mod }b\mbox{)}$ \\
\hline \\
Let $m$ represent the number of digits in $b$.  \\
1.  Make a copy of $a$ and store it in $q$.  (\textit{mp\_init\_copy}) \\
2.  $q \leftarrow \lfloor q / \beta^{m - 1} \rfloor$ (\textit{mp\_rshd}) \\
\\
Produce the quotient. \\
3.  $q \leftarrow q \cdot \mu$  (\textit{note: only produce digits at or above $m-1$}) \\
4.  $q \leftarrow \lfloor q / \beta^{m + 1} \rfloor$ \\
\\
Subtract the multiple of modulus from the input. \\
5.  $a \leftarrow a \mbox{ (mod }\beta^{m+1}\mbox{)}$ (\textit{mp\_mod\_2d}) \\
6.  $q \leftarrow q \cdot b \mbox{ (mod }\beta^{m+1}\mbox{)}$ (\textit{s\_mp\_mul\_digs}) \\
7.  $a \leftarrow a - q$ (\textit{mp\_sub}) \\
\\
Add $\beta^{m+1}$ if a carry occured. \\
8.  If $a < 0$ then (\textit{mp\_cmp\_d}) \\
\hspace{3mm}8.1  $q \leftarrow 1$ (\textit{mp\_set}) \\
\hspace{3mm}8.2  $q \leftarrow q \cdot \beta^{m+1}$ (\textit{mp\_lshd}) \\
\hspace{3mm}8.3  $a \leftarrow a + q$ \\
\\
Now subtract the modulus if the residue is too large (e.g. quotient too small). \\
9.  While $a \ge b$ do (\textit{mp\_cmp}) \\
\hspace{3mm}9.1  $c \leftarrow a - b$ \\
10.  Clear $q$. \\
11.  Return(\textit{MP\_OKAY}) \\
\hline
\end{tabular}
\end{center}
\end{small}
\caption{Algorithm mp\_reduce}
\end{figure}

\textbf{Algorithm mp\_reduce.}
This algorithm will reduce the input $a$ modulo $b$ in place using the Barrett algorithm.  It is loosely based on algorithm 14.42 of HAC
\cite[pp.  602]{HAC} which is based on the paper from Paul Barrett \cite{BARRETT}.  The algorithm has several restrictions and assumptions which must 
be adhered to for the algorithm to work.

First the modulus $b$ is assumed to be positive and greater than one.  If the modulus were less than or equal to one than subtracting
a multiple of it would either accomplish nothing or actually enlarge the input.  The input $a$ must be in the range $0 \le a < b^2$ in order
for the quotient to have enough precision.  If $a$ is the product of two numbers that were already reduced modulo $b$, this will not be a problem.
Technically the algorithm will still work if $a \ge b^2$ but it will take much longer to finish.  The value of $\mu$ is passed as an argument to this 
algorithm and is assumed to be calculated and stored before the algorithm is used.  

Recall that the multiplication for the quotient on step 3 must only produce digits at or above the $m-1$'th position.  An algorithm called 
$s\_mp\_mul\_high\_digs$ which has not been presented is used to accomplish this task.  The algorithm is based on $s\_mp\_mul\_digs$ except that
instead of stopping at a given level of precision it starts at a given level of precision.  This optimal algorithm can only be used if the number
of digits in $b$ is very much smaller than $\beta$.  

While it is known that 
$a \ge b \cdot \lfloor (q_0 \cdot \mu) / \beta^{m+1} \rfloor$ only the lower $m+1$ digits are being used to compute the residue, so an implied 
``borrow'' from the higher digits might leave a negative result.  After the multiple of the modulus has been subtracted from $a$ the residue must be 
fixed up in case it is negative.  The invariant $\beta^{m+1}$ must be added to the residue to make it positive again.  

The while loop at step 9 will subtract $b$ until the residue is less than $b$.  If the algorithm is performed correctly this step is 
performed at most twice, and on average once. However, if $a \ge b^2$ than it will iterate substantially more times than it should.

\vspace{+3mm}\begin{small}
\hspace{-5.1mm}{\bf File}: bn\_mp\_reduce.c
\vspace{-3mm}
\begin{alltt}
\end{alltt}
\end{small}

The first multiplication that determines the quotient can be performed by only producing the digits from $m - 1$ and up.  This essentially halves
the number of single precision multiplications required.  However, the optimization is only safe if $\beta$ is much larger than the number of digits
in the modulus.  In the source code this is evaluated on lines 36 to 44 where algorithm s\_mp\_mul\_high\_digs is used when it is
safe to do so.  

\subsection{The Barrett Setup Algorithm}
In order to use algorithm mp\_reduce the value of $\mu$ must be calculated in advance.  Ideally this value should be computed once and stored for
future use so that the Barrett algorithm can be used without delay.  

\newpage\begin{figure}[!here]
\begin{small}
\begin{center}
\begin{tabular}{l}
\hline Algorithm \textbf{mp\_reduce\_setup}. \\
\textbf{Input}.   mp\_int $a$ ($a > 1$)  \\
\textbf{Output}.  $\mu \leftarrow \lfloor \beta^{2m}/a \rfloor$ \\
\hline \\
1.  $\mu \leftarrow 2^{2 \cdot lg(\beta) \cdot  m}$ (\textit{mp\_2expt}) \\
2.  $\mu \leftarrow \lfloor \mu / b \rfloor$ (\textit{mp\_div}) \\
3.  Return(\textit{MP\_OKAY}) \\
\hline
\end{tabular}
\end{center}
\end{small}
\caption{Algorithm mp\_reduce\_setup}
\end{figure}

\textbf{Algorithm mp\_reduce\_setup.}
This algorithm computes the reciprocal $\mu$ required for Barrett reduction.  First $\beta^{2m}$ is calculated as $2^{2 \cdot lg(\beta) \cdot  m}$ which
is equivalent and much faster.  The final value is computed by taking the integer quotient of $\lfloor \mu / b \rfloor$.

\vspace{+3mm}\begin{small}
\hspace{-5.1mm}{\bf File}: bn\_mp\_reduce\_setup.c
\vspace{-3mm}
\begin{alltt}
\end{alltt}
\end{small}

This simple routine calculates the reciprocal $\mu$ required by Barrett reduction.  Note the extended usage of algorithm mp\_div where the variable
which would received the remainder is passed as NULL.  As will be discussed in~\ref{sec:division} the division routine allows both the quotient and the 
remainder to be passed as NULL meaning to ignore the value.  

\section{The Montgomery Reduction}
Montgomery reduction\footnote{Thanks to Niels Ferguson for his insightful explanation of the algorithm.} \cite{MONT} is by far the most interesting 
form of reduction in common use.  It computes a modular residue which is not actually equal to the residue of the input yet instead equal to a 
residue times a constant.  However, as perplexing as this may sound the algorithm is relatively simple and very efficient.  

Throughout this entire section the variable $n$ will represent the modulus used to form the residue.  As will be discussed shortly the value of
$n$ must be odd.  The variable $x$ will represent the quantity of which the residue is sought.  Similar to the Barrett algorithm the input
is restricted to $0 \le x < n^2$.  To begin the description some simple number theory facts must be established.

\textbf{Fact 1.}  Adding $n$ to $x$ does not change the residue since in effect it adds one to the quotient $\lfloor x / n \rfloor$.  Another way
to explain this is that $n$ is (\textit{or multiples of $n$ are}) congruent to zero modulo $n$.  Adding zero will not change the value of the residue.  

\textbf{Fact 2.}  If $x$ is even then performing a division by two in $\Z$ is congruent to $x \cdot 2^{-1} \mbox{ (mod }n\mbox{)}$.  Actually
this is an application of the fact that if $x$ is evenly divisible by any $k \in \Z$ then division in $\Z$ will be congruent to 
multiplication by $k^{-1}$ modulo $n$.  

From these two simple facts the following simple algorithm can be derived.

\newpage\begin{figure}[!here]
\begin{small}
\begin{center}
\begin{tabular}{l}
\hline Algorithm \textbf{Montgomery Reduction}. \\
\textbf{Input}.   Integer $x$, $n$ and $k$ \\
\textbf{Output}.  $2^{-k}x \mbox{ (mod }n\mbox{)}$ \\
\hline \\
1.  for $t$ from $1$ to $k$ do \\
\hspace{3mm}1.1  If $x$ is odd then \\
\hspace{6mm}1.1.1  $x \leftarrow x + n$ \\
\hspace{3mm}1.2  $x \leftarrow x/2$ \\
2.  Return $x$. \\
\hline
\end{tabular}
\end{center}
\end{small}
\caption{Algorithm Montgomery Reduction}
\end{figure}

The algorithm reduces the input one bit at a time using the two congruencies stated previously.  Inside the loop $n$, which is odd, is
added to $x$ if $x$ is odd.  This forces $x$ to be even which allows the division by two in $\Z$ to be congruent to a modular division by two.  Since
$x$ is assumed to be initially much larger than $n$ the addition of $n$ will contribute an insignificant magnitude to $x$.  Let $r$ represent the 
final result of the Montgomery algorithm.  If $k > lg(n)$ and $0 \le x < n^2$ then the final result is limited to 
$0 \le r < \lfloor x/2^k \rfloor + n$.  As a result at most a single subtraction is required to get the residue desired.

\begin{figure}[here]
\begin{small}
\begin{center}
\begin{tabular}{|c|l|}
\hline \textbf{Step number ($t$)} & \textbf{Result ($x$)} \\
\hline $1$ & $x + n = 5812$, $x/2 = 2906$ \\
\hline $2$ & $x/2 = 1453$ \\
\hline $3$ & $x + n = 1710$, $x/2 = 855$ \\
\hline $4$ & $x + n = 1112$, $x/2 = 556$ \\
\hline $5$ & $x/2 = 278$ \\
\hline $6$ & $x/2 = 139$ \\
\hline $7$ & $x + n = 396$, $x/2 = 198$ \\
\hline $8$ & $x/2 = 99$ \\
\hline $9$ & $x + n = 356$, $x/2 = 178$ \\
\hline
\end{tabular}
\end{center}
\end{small}
\caption{Example of Montgomery Reduction (I)}
\label{fig:MONT1}
\end{figure}

Consider the example in figure~\ref{fig:MONT1} which reduces $x = 5555$ modulo $n = 257$ when $k = 9$ (note $\beta^k = 512$ which is larger than $n$).  The result of 
the algorithm $r = 178$ is congruent to the value of $2^{-9} \cdot 5555 \mbox{ (mod }257\mbox{)}$.  When $r$ is multiplied by $2^9$ modulo $257$ the correct residue 
$r \equiv 158$ is produced.  

Let $k = \lfloor lg(n) \rfloor + 1$ represent the number of bits in $n$.  The current algorithm requires $2k^2$ single precision shifts
and $k^2$ single precision additions.  At this rate the algorithm is most certainly slower than Barrett reduction and not terribly useful.  
Fortunately there exists an alternative representation of the algorithm.

\begin{figure}[!here]
\begin{small}
\begin{center}
\begin{tabular}{l}
\hline Algorithm \textbf{Montgomery Reduction} (modified I). \\
\textbf{Input}.   Integer $x$, $n$ and $k$ ($2^k > n$) \\
\textbf{Output}.  $2^{-k}x \mbox{ (mod }n\mbox{)}$ \\
\hline \\
1.  for $t$ from $1$ to $k$ do \\
\hspace{3mm}1.1  If the $t$'th bit of $x$ is one then \\
\hspace{6mm}1.1.1  $x \leftarrow x + 2^tn$ \\
2.  Return $x/2^k$. \\
\hline
\end{tabular}
\end{center}
\end{small}
\caption{Algorithm Montgomery Reduction (modified I)}
\end{figure}

This algorithm is equivalent since $2^tn$ is a multiple of $n$ and the lower $k$ bits of $x$ are zero by step 2.  The number of single
precision shifts has now been reduced from $2k^2$ to $k^2 + k$ which is only a small improvement.

\begin{figure}[here]
\begin{small}
\begin{center}
\begin{tabular}{|c|l|r|}
\hline \textbf{Step number ($t$)} & \textbf{Result ($x$)} & \textbf{Result ($x$) in Binary} \\
\hline -- & $5555$ & $1010110110011$ \\
\hline $1$ & $x + 2^{0}n = 5812$ &  $1011010110100$ \\
\hline $2$ & $5812$ & $1011010110100$ \\
\hline $3$ & $x + 2^{2}n = 6840$ & $1101010111000$ \\
\hline $4$ & $x + 2^{3}n = 8896$ & $10001011000000$ \\
\hline $5$ & $8896$ & $10001011000000$ \\
\hline $6$ & $8896$ & $10001011000000$ \\
\hline $7$ & $x + 2^{6}n = 25344$ & $110001100000000$ \\
\hline $8$ & $25344$ & $110001100000000$ \\
\hline $9$ & $x + 2^{7}n = 91136$ & $10110010000000000$ \\
\hline -- & $x/2^k = 178$ & \\
\hline
\end{tabular}
\end{center}
\end{small}
\caption{Example of Montgomery Reduction (II)}
\label{fig:MONT2}
\end{figure}

Figure~\ref{fig:MONT2} demonstrates the modified algorithm reducing $x = 5555$ modulo $n = 257$ with $k = 9$. 
With this algorithm a single shift right at the end is the only right shift required to reduce the input instead of $k$ right shifts inside the 
loop.  Note that for the iterations $t = 2, 5, 6$ and $8$ where the result $x$ is not changed.  In those iterations the $t$'th bit of $x$ is 
zero and the appropriate multiple of $n$ does not need to be added to force the $t$'th bit of the result to zero.  

\subsection{Digit Based Montgomery Reduction}
Instead of computing the reduction on a bit-by-bit basis it is actually much faster to compute it on digit-by-digit basis.  Consider the
previous algorithm re-written to compute the Montgomery reduction in this new fashion.

\begin{figure}[!here]
\begin{small}
\begin{center}
\begin{tabular}{l}
\hline Algorithm \textbf{Montgomery Reduction} (modified II). \\
\textbf{Input}.   Integer $x$, $n$ and $k$ ($\beta^k > n$) \\
\textbf{Output}.  $\beta^{-k}x \mbox{ (mod }n\mbox{)}$ \\
\hline \\
1.  for $t$ from $0$ to $k - 1$ do \\
\hspace{3mm}1.1  $x \leftarrow x + \mu n \beta^t$ \\
2.  Return $x/\beta^k$. \\
\hline
\end{tabular}
\end{center}
\end{small}
\caption{Algorithm Montgomery Reduction (modified II)}
\end{figure}

The value $\mu n \beta^t$ is a multiple of the modulus $n$ meaning that it will not change the residue.  If the first digit of 
the value $\mu n \beta^t$ equals the negative (modulo $\beta$) of the $t$'th digit of $x$ then the addition will result in a zero digit.  This
problem breaks down to solving the following congruency.  

\begin{center}
\begin{tabular}{rcl}
$x_t + \mu n_0$ & $\equiv$ & $0 \mbox{ (mod }\beta\mbox{)}$ \\
$\mu n_0$ & $\equiv$ & $-x_t \mbox{ (mod }\beta\mbox{)}$ \\
$\mu$ & $\equiv$ & $-x_t/n_0 \mbox{ (mod }\beta\mbox{)}$ \\
\end{tabular}
\end{center}

In each iteration of the loop on step 1 a new value of $\mu$ must be calculated.  The value of $-1/n_0 \mbox{ (mod }\beta\mbox{)}$ is used 
extensively in this algorithm and should be precomputed.  Let $\rho$ represent the negative of the modular inverse of $n_0$ modulo $\beta$.  

For example, let $\beta = 10$ represent the radix.  Let $n = 17$ represent the modulus which implies $k = 2$ and $\rho \equiv 7$.  Let $x = 33$ 
represent the value to reduce.

\newpage\begin{figure}
\begin{center}
\begin{tabular}{|c|c|c|}
\hline \textbf{Step ($t$)} & \textbf{Value of $x$} & \textbf{Value of $\mu$} \\
\hline --                 & $33$ & --\\
\hline $0$                 & $33 + \mu n = 50$ & $1$ \\
\hline $1$                 & $50 + \mu n \beta = 900$ & $5$ \\
\hline
\end{tabular}
\end{center}
\caption{Example of Montgomery Reduction}
\end{figure}

The final result $900$ is then divided by $\beta^k$ to produce the final result $9$.  The first observation is that $9 \nequiv x \mbox{ (mod }n\mbox{)}$ 
which implies the result is not the modular residue of $x$ modulo $n$.  However, recall that the residue is actually multiplied by $\beta^{-k}$ in
the algorithm.  To get the true residue the value must be multiplied by $\beta^k$.  In this case $\beta^k \equiv 15 \mbox{ (mod }n\mbox{)}$ and
the correct residue is $9 \cdot 15 \equiv 16 \mbox{ (mod }n\mbox{)}$.  

\subsection{Baseline Montgomery Reduction}
The baseline Montgomery reduction algorithm will produce the residue for any size input.  It is designed to be a catch-all algororithm for 
Montgomery reductions.  

\newpage\begin{figure}[!here]
\begin{small}
\begin{center}
\begin{tabular}{l}
\hline Algorithm \textbf{mp\_montgomery\_reduce}. \\
\textbf{Input}.   mp\_int $x$, mp\_int $n$ and a digit $\rho \equiv -1/n_0 \mbox{ (mod }n\mbox{)}$. \\
\hspace{11.5mm}($0 \le x < n^2, n > 1, (n, \beta) = 1, \beta^k > n$) \\
\textbf{Output}.  $\beta^{-k}x \mbox{ (mod }n\mbox{)}$ \\
\hline \\
1.  $digs \leftarrow 2n.used + 1$ \\
2.  If $digs < MP\_ARRAY$ and $m.used < \delta$ then \\
\hspace{3mm}2.1  Use algorithm fast\_mp\_montgomery\_reduce instead. \\
\\
Setup $x$ for the reduction. \\
3.  If $x.alloc < digs$ then grow $x$ to $digs$ digits. \\
4.  $x.used \leftarrow digs$ \\
\\
Eliminate the lower $k$ digits. \\
5.  For $ix$ from $0$ to $k - 1$ do \\
\hspace{3mm}5.1  $\mu \leftarrow x_{ix} \cdot \rho \mbox{ (mod }\beta\mbox{)}$ \\
\hspace{3mm}5.2  $u \leftarrow 0$ \\
\hspace{3mm}5.3  For $iy$ from $0$ to $k - 1$ do \\
\hspace{6mm}5.3.1  $\hat r \leftarrow \mu n_{iy} + x_{ix + iy} + u$ \\
\hspace{6mm}5.3.2  $x_{ix + iy} \leftarrow \hat r \mbox{ (mod }\beta\mbox{)}$ \\
\hspace{6mm}5.3.3  $u \leftarrow \lfloor \hat r / \beta \rfloor$ \\
\hspace{3mm}5.4  While $u > 0$ do \\
\hspace{6mm}5.4.1  $iy \leftarrow iy + 1$ \\
\hspace{6mm}5.4.2  $x_{ix + iy} \leftarrow x_{ix + iy} + u$ \\
\hspace{6mm}5.4.3  $u \leftarrow \lfloor x_{ix+iy} / \beta \rfloor$ \\
\hspace{6mm}5.4.4  $x_{ix + iy} \leftarrow x_{ix+iy} \mbox{ (mod }\beta\mbox{)}$ \\
\\
Divide by $\beta^k$ and fix up as required. \\
6.  $x \leftarrow \lfloor x / \beta^k \rfloor$ \\
7.  If $x \ge n$ then \\
\hspace{3mm}7.1  $x \leftarrow x - n$ \\
8.  Return(\textit{MP\_OKAY}). \\
\hline
\end{tabular}
\end{center}
\end{small}
\caption{Algorithm mp\_montgomery\_reduce}
\end{figure}

\textbf{Algorithm mp\_montgomery\_reduce.}
This algorithm reduces the input $x$ modulo $n$ in place using the Montgomery reduction algorithm.  The algorithm is loosely based
on algorithm 14.32 of \cite[pp.601]{HAC} except it merges the multiplication of $\mu n \beta^t$ with the addition in the inner loop.  The
restrictions on this algorithm are fairly easy to adapt to.  First $0 \le x < n^2$ bounds the input to numbers in the same range as 
for the Barrett algorithm.  Additionally if $n > 1$ and $n$ is odd there will exist a modular inverse $\rho$.  $\rho$ must be calculated in
advance of this algorithm.  Finally the variable $k$ is fixed and a pseudonym for $n.used$.  

Step 2 decides whether a faster Montgomery algorithm can be used.  It is based on the Comba technique meaning that there are limits on
the size of the input.  This algorithm is discussed in sub-section 6.3.3.

Step 5 is the main reduction loop of the algorithm.  The value of $\mu$ is calculated once per iteration in the outer loop.  The inner loop
calculates $x + \mu n \beta^{ix}$ by multiplying $\mu n$ and adding the result to $x$ shifted by $ix$ digits.  Both the addition and
multiplication are performed in the same loop to save time and memory.  Step 5.4 will handle any additional carries that escape the inner loop.

Using a quick inspection this algorithm requires $n$ single precision multiplications for the outer loop and $n^2$ single precision multiplications 
in the inner loop.  In total $n^2 + n$ single precision multiplications which compares favourably to Barrett at $n^2 + 2n - 1$ single precision
multiplications.  

\vspace{+3mm}\begin{small}
\hspace{-5.1mm}{\bf File}: bn\_mp\_montgomery\_reduce.c
\vspace{-3mm}
\begin{alltt}
\end{alltt}
\end{small}

This is the baseline implementation of the Montgomery reduction algorithm.  Lines 31 to 36 determine if the Comba based
routine can be used instead.  Line 47 computes the value of $\mu$ for that particular iteration of the outer loop.  

The multiplication $\mu n \beta^{ix}$ is performed in one step in the inner loop.  The alias $tmpx$ refers to the $ix$'th digit of $x$ and
the alias $tmpn$ refers to the modulus $n$.  

\subsection{Faster ``Comba'' Montgomery Reduction}

The Montgomery reduction requires fewer single precision multiplications than a Barrett reduction, however it is much slower due to the serial
nature of the inner loop.  The Barrett reduction algorithm requires two slightly modified multipliers which can be implemented with the Comba
technique.  The Montgomery reduction algorithm cannot directly use the Comba technique to any significant advantage since the inner loop calculates
a $k \times 1$ product $k$ times. 

The biggest obstacle is that at the $ix$'th iteration of the outer loop the value of $x_{ix}$ is required to calculate $\mu$.  This means the 
carries from $0$ to $ix - 1$ must have been propagated upwards to form a valid $ix$'th digit.  The solution as it turns out is very simple.  
Perform a Comba like multiplier and inside the outer loop just after the inner loop fix up the $ix + 1$'th digit by forwarding the carry.  

With this change in place the Montgomery reduction algorithm can be performed with a Comba style multiplication loop which substantially increases
the speed of the algorithm.  

\newpage\begin{figure}[!here]
\begin{small}
\begin{center}
\begin{tabular}{l}
\hline Algorithm \textbf{fast\_mp\_montgomery\_reduce}. \\
\textbf{Input}.   mp\_int $x$, mp\_int $n$ and a digit $\rho \equiv -1/n_0 \mbox{ (mod }n\mbox{)}$. \\
\hspace{11.5mm}($0 \le x < n^2, n > 1, (n, \beta) = 1, \beta^k > n$) \\
\textbf{Output}.  $\beta^{-k}x \mbox{ (mod }n\mbox{)}$ \\
\hline \\
Place an array of \textbf{MP\_WARRAY} mp\_word variables called $\hat W$ on the stack. \\
1.  if $x.alloc < n.used + 1$ then grow $x$ to $n.used + 1$ digits. \\
Copy the digits of $x$ into the array $\hat W$ \\
2.  For $ix$ from $0$ to $x.used - 1$ do \\
\hspace{3mm}2.1  $\hat W_{ix} \leftarrow x_{ix}$ \\
3.  For $ix$ from $x.used$ to $2n.used - 1$ do \\
\hspace{3mm}3.1  $\hat W_{ix} \leftarrow 0$ \\
Elimiate the lower $k$ digits. \\
4.  for $ix$ from $0$ to $n.used - 1$ do \\
\hspace{3mm}4.1  $\mu \leftarrow \hat W_{ix} \cdot \rho \mbox{ (mod }\beta\mbox{)}$ \\
\hspace{3mm}4.2  For $iy$ from $0$ to $n.used - 1$ do \\
\hspace{6mm}4.2.1  $\hat W_{iy + ix} \leftarrow \hat W_{iy + ix} + \mu \cdot n_{iy}$ \\
\hspace{3mm}4.3  $\hat W_{ix + 1} \leftarrow \hat W_{ix + 1} + \lfloor \hat W_{ix} / \beta \rfloor$ \\
Propagate carries upwards. \\
5.  for $ix$ from $n.used$ to $2n.used + 1$ do \\
\hspace{3mm}5.1  $\hat W_{ix + 1} \leftarrow \hat W_{ix + 1} + \lfloor \hat W_{ix} / \beta \rfloor$ \\
Shift right and reduce modulo $\beta$ simultaneously. \\
6.  for $ix$ from $0$ to $n.used + 1$ do \\
\hspace{3mm}6.1  $x_{ix} \leftarrow \hat W_{ix + n.used} \mbox{ (mod }\beta\mbox{)}$ \\
Zero excess digits and fixup $x$. \\
7.  if $x.used > n.used + 1$ then do \\
\hspace{3mm}7.1  for $ix$ from $n.used + 1$ to $x.used - 1$ do \\
\hspace{6mm}7.1.1  $x_{ix} \leftarrow 0$ \\
8.  $x.used \leftarrow n.used + 1$ \\
9.  Clamp excessive digits of $x$. \\
10.  If $x \ge n$ then \\
\hspace{3mm}10.1  $x \leftarrow x - n$ \\
11.  Return(\textit{MP\_OKAY}). \\
\hline
\end{tabular}
\end{center}
\end{small}
\caption{Algorithm fast\_mp\_montgomery\_reduce}
\end{figure}

\textbf{Algorithm fast\_mp\_montgomery\_reduce.}
This algorithm will compute the Montgomery reduction of $x$ modulo $n$ using the Comba technique.  It is on most computer platforms significantly
faster than algorithm mp\_montgomery\_reduce and algorithm mp\_reduce (\textit{Barrett reduction}).  The algorithm has the same restrictions
on the input as the baseline reduction algorithm.  An additional two restrictions are imposed on this algorithm.  The number of digits $k$ in the 
the modulus $n$ must not violate $MP\_WARRAY > 2k +1$ and $n < \delta$.   When $\beta = 2^{28}$ this algorithm can be used to reduce modulo
a modulus of at most $3,556$ bits in length.  

As in the other Comba reduction algorithms there is a $\hat W$ array which stores the columns of the product.  It is initially filled with the
contents of $x$ with the excess digits zeroed.  The reduction loop is very similar the to the baseline loop at heart.  The multiplication on step
4.1 can be single precision only since $ab \mbox{ (mod }\beta\mbox{)} \equiv (a \mbox{ mod }\beta)(b \mbox{ mod }\beta)$.  Some multipliers such
as those on the ARM processors take a variable length time to complete depending on the number of bytes of result it must produce.  By performing
a single precision multiplication instead half the amount of time is spent.

Also note that digit $\hat W_{ix}$ must have the carry from the $ix - 1$'th digit propagated upwards in order for this to work.  That is what step
4.3 will do.  In effect over the $n.used$ iterations of the outer loop the $n.used$'th lower columns all have the their carries propagated forwards.  Note
how the upper bits of those same words are not reduced modulo $\beta$.  This is because those values will be discarded shortly and there is no
point.

Step 5 will propagate the remainder of the carries upwards.  On step 6 the columns are reduced modulo $\beta$ and shifted simultaneously as they are
stored in the destination $x$.  

\vspace{+3mm}\begin{small}
\hspace{-5.1mm}{\bf File}: bn\_fast\_mp\_montgomery\_reduce.c
\vspace{-3mm}
\begin{alltt}
\end{alltt}
\end{small}

The $\hat W$ array is first filled with digits of $x$ on line 48 then the rest of the digits are zeroed on line 55.  Both loops share
the same alias variables to make the code easier to read.  

The value of $\mu$ is calculated in an interesting fashion.  First the value $\hat W_{ix}$ is reduced modulo $\beta$ and cast to a mp\_digit.  This
forces the compiler to use a single precision multiplication and prevents any concerns about loss of precision.   Line 110 fixes the carry 
for the next iteration of the loop by propagating the carry from $\hat W_{ix}$ to $\hat W_{ix+1}$.

The for loop on line 109 propagates the rest of the carries upwards through the columns.  The for loop on line 126 reduces the columns
modulo $\beta$ and shifts them $k$ places at the same time.  The alias $\_ \hat W$ actually refers to the array $\hat W$ starting at the $n.used$'th
digit, that is $\_ \hat W_{t} = \hat W_{n.used + t}$.  

\subsection{Montgomery Setup}
To calculate the variable $\rho$ a relatively simple algorithm will be required.  

\begin{figure}[!here]
\begin{small}
\begin{center}
\begin{tabular}{l}
\hline Algorithm \textbf{mp\_montgomery\_setup}. \\
\textbf{Input}.   mp\_int $n$ ($n > 1$ and $(n, 2) = 1$) \\
\textbf{Output}.  $\rho \equiv -1/n_0 \mbox{ (mod }\beta\mbox{)}$ \\
\hline \\
1.  $b \leftarrow n_0$ \\
2.  If $b$ is even return(\textit{MP\_VAL}) \\
3.  $x \leftarrow (((b + 2) \mbox{ AND } 4) << 1) + b$ \\
4.  for $k$ from 0 to $\lceil lg(lg(\beta)) \rceil - 2$ do \\
\hspace{3mm}4.1  $x \leftarrow x \cdot (2 - bx)$ \\
5.  $\rho \leftarrow \beta - x \mbox{ (mod }\beta\mbox{)}$ \\
6.  Return(\textit{MP\_OKAY}). \\
\hline
\end{tabular}
\end{center}
\end{small}
\caption{Algorithm mp\_montgomery\_setup} 
\end{figure}

\textbf{Algorithm mp\_montgomery\_setup.}
This algorithm will calculate the value of $\rho$ required within the Montgomery reduction algorithms.  It uses a very interesting trick 
to calculate $1/n_0$ when $\beta$ is a power of two.  

\vspace{+3mm}\begin{small}
\hspace{-5.1mm}{\bf File}: bn\_mp\_montgomery\_setup.c
\vspace{-3mm}
\begin{alltt}
\end{alltt}
\end{small}

This source code computes the value of $\rho$ required to perform Montgomery reduction.  It has been modified to avoid performing excess
multiplications when $\beta$ is not the default 28-bits.  

\section{The Diminished Radix Algorithm}
The Diminished Radix method of modular reduction \cite{DRMET} is a fairly clever technique which can be more efficient than either the Barrett
or Montgomery methods for certain forms of moduli.  The technique is based on the following simple congruence.

\begin{equation}
(x \mbox{ mod } n) + k \lfloor x / n \rfloor \equiv x \mbox{ (mod }(n - k)\mbox{)}
\end{equation}

This observation was used in the MMB \cite{MMB} block cipher to create a diffusion primitive.  It used the fact that if $n = 2^{31}$ and $k=1$ that 
then a x86 multiplier could produce the 62-bit product and use  the ``shrd'' instruction to perform a double-precision right shift.  The proof
of the above equation is very simple.  First write $x$ in the product form.

\begin{equation}
x = qn + r
\end{equation}

Now reduce both sides modulo $(n - k)$.

\begin{equation}
x \equiv qk + r  \mbox{ (mod }(n-k)\mbox{)}
\end{equation}

The variable $n$ reduces modulo $n - k$ to $k$.  By putting $q = \lfloor x/n \rfloor$ and $r = x \mbox{ mod } n$ 
into the equation the original congruence is reproduced, thus concluding the proof.  The following algorithm is based on this observation.

\begin{figure}[!here]
\begin{small}
\begin{center}
\begin{tabular}{l}
\hline Algorithm \textbf{Diminished Radix Reduction}. \\
\textbf{Input}.   Integer $x$, $n$, $k$ \\
\textbf{Output}.  $x \mbox{ mod } (n - k)$ \\
\hline \\
1.  $q \leftarrow \lfloor x / n \rfloor$ \\
2.  $q \leftarrow k \cdot q$ \\
3.  $x \leftarrow x \mbox{ (mod }n\mbox{)}$ \\
4.  $x \leftarrow x + q$ \\
5.  If $x \ge (n - k)$ then \\
\hspace{3mm}5.1  $x \leftarrow x - (n - k)$ \\
\hspace{3mm}5.2  Goto step 1. \\
6.  Return $x$ \\
\hline
\end{tabular}
\end{center}
\end{small}
\caption{Algorithm Diminished Radix Reduction}
\label{fig:DR}
\end{figure}

This algorithm will reduce $x$ modulo $n - k$ and return the residue.  If $0 \le x < (n - k)^2$ then the algorithm will loop almost always
once or twice and occasionally three times.  For simplicity sake the value of $x$ is bounded by the following simple polynomial.

\begin{equation} 
0 \le x < n^2 + k^2 - 2nk
\end{equation}

The true bound is  $0 \le x < (n - k - 1)^2$ but this has quite a few more terms.  The value of $q$ after step 1 is bounded by the following.

\begin{equation}
q < n - 2k - k^2/n
\end{equation}

Since $k^2$ is going to be considerably smaller than $n$ that term will always be zero.  The value of $x$ after step 3 is bounded trivially as
$0 \le x < n$.  By step four the sum $x + q$ is bounded by 

\begin{equation}
0 \le q + x < (k + 1)n - 2k^2 - 1
\end{equation}

With a second pass $q$ will be loosely bounded by $0 \le q < k^2$ after step 2 while $x$ will still be loosely bounded by $0 \le x < n$ after step 3.  After the second pass it is highly unlike that the
sum in step 4 will exceed $n - k$.  In practice fewer than three passes of the algorithm are required to reduce virtually every input in the 
range $0 \le x < (n - k - 1)^2$.  

\begin{figure}
\begin{small}
\begin{center}
\begin{tabular}{|l|}
\hline
$x = 123456789, n = 256, k = 3$ \\
\hline $q \leftarrow \lfloor x/n \rfloor = 482253$ \\
$q \leftarrow q*k = 1446759$ \\
$x \leftarrow x \mbox{ mod } n = 21$ \\
$x \leftarrow x + q = 1446780$ \\
$x \leftarrow x - (n - k) = 1446527$ \\
\hline 
$q \leftarrow \lfloor x/n \rfloor = 5650$ \\
$q \leftarrow q*k = 16950$ \\
$x \leftarrow x \mbox{ mod } n = 127$ \\
$x \leftarrow x + q = 17077$ \\
$x \leftarrow x - (n - k) = 16824$ \\
\hline 
$q \leftarrow \lfloor x/n \rfloor = 65$ \\
$q \leftarrow q*k = 195$ \\
$x \leftarrow x \mbox{ mod } n = 184$ \\
$x \leftarrow x + q = 379$ \\
$x \leftarrow x - (n - k) = 126$ \\
\hline
\end{tabular}
\end{center}
\end{small}
\caption{Example Diminished Radix Reduction}
\label{fig:EXDR}
\end{figure}

Figure~\ref{fig:EXDR} demonstrates the reduction of $x = 123456789$ modulo $n - k = 253$ when $n = 256$ and $k = 3$.  Note that even while $x$
is considerably larger than $(n - k - 1)^2 = 63504$ the algorithm still converges on the modular residue exceedingly fast.  In this case only
three passes were required to find the residue $x \equiv 126$.


\subsection{Choice of Moduli}
On the surface this algorithm looks like a very expensive algorithm.  It requires a couple of subtractions followed by multiplication and other
modular reductions.  The usefulness of this algorithm becomes exceedingly clear when an appropriate modulus is chosen.

Division in general is a very expensive operation to perform.  The one exception is when the division is by a power of the radix of representation used.  
Division by ten for example is simple for pencil and paper mathematics since it amounts to shifting the decimal place to the right.  Similarly division 
by two (\textit{or powers of two}) is very simple for binary computers to perform.  It would therefore seem logical to choose $n$ of the form $2^p$ 
which would imply that $\lfloor x / n \rfloor$ is a simple shift of $x$ right $p$ bits.  

However, there is one operation related to division of power of twos that is even faster than this.  If $n = \beta^p$ then the division may be 
performed by moving whole digits to the right $p$ places.  In practice division by $\beta^p$ is much faster than division by $2^p$ for any $p$.  
Also with the choice of $n = \beta^p$ reducing $x$ modulo $n$ merely requires zeroing the digits above the $p-1$'th digit of $x$.  

Throughout the next section the term ``restricted modulus'' will refer to a modulus of the form $\beta^p - k$ whereas the term ``unrestricted
modulus'' will refer to a modulus of the form $2^p - k$.  The word ``restricted'' in this case refers to the fact that it is based on the 
$2^p$ logic except $p$ must be a multiple of $lg(\beta)$.  

\subsection{Choice of $k$}
Now that division and reduction (\textit{step 1 and 3 of figure~\ref{fig:DR}}) have been optimized to simple digit operations the multiplication by $k$
in step 2 is the most expensive operation.  Fortunately the choice of $k$ is not terribly limited.  For all intents and purposes it might
as well be a single digit.  The smaller the value of $k$ is the faster the algorithm will be.  

\subsection{Restricted Diminished Radix Reduction}
The restricted Diminished Radix algorithm can quickly reduce an input modulo a modulus of the form $n = \beta^p - k$.  This algorithm can reduce 
an input $x$ within the range $0 \le x < n^2$ using only a couple passes of the algorithm demonstrated in figure~\ref{fig:DR}.  The implementation
of this algorithm has been optimized to avoid additional overhead associated with a division by $\beta^p$, the multiplication by $k$ or the addition 
of $x$ and $q$.  The resulting algorithm is very efficient and can lead to substantial improvements over Barrett and Montgomery reduction when modular 
exponentiations are performed.

\newpage\begin{figure}[!here]
\begin{small}
\begin{center}
\begin{tabular}{l}
\hline Algorithm \textbf{mp\_dr\_reduce}. \\
\textbf{Input}.   mp\_int $x$, $n$ and a mp\_digit $k = \beta - n_0$ \\
\hspace{11.5mm}($0 \le x < n^2$, $n > 1$, $0 < k < \beta$) \\
\textbf{Output}.  $x \mbox{ mod } n$ \\
\hline \\
1.  $m \leftarrow n.used$ \\
2.  If $x.alloc < 2m$ then grow $x$ to $2m$ digits. \\
3.  $\mu \leftarrow 0$ \\
4.  for $i$ from $0$ to $m - 1$ do \\
\hspace{3mm}4.1  $\hat r \leftarrow k \cdot x_{m+i} + x_{i} + \mu$ \\
\hspace{3mm}4.2  $x_{i} \leftarrow \hat r \mbox{ (mod }\beta\mbox{)}$ \\
\hspace{3mm}4.3  $\mu \leftarrow \lfloor \hat r / \beta \rfloor$ \\
5.  $x_{m} \leftarrow \mu$ \\
6.  for $i$ from $m + 1$ to $x.used - 1$ do \\
\hspace{3mm}6.1  $x_{i} \leftarrow 0$ \\
7.  Clamp excess digits of $x$. \\
8.  If $x \ge n$ then \\
\hspace{3mm}8.1  $x \leftarrow x - n$ \\
\hspace{3mm}8.2  Goto step 3. \\
9.  Return(\textit{MP\_OKAY}). \\
\hline
\end{tabular}
\end{center}
\end{small}
\caption{Algorithm mp\_dr\_reduce}
\end{figure}

\textbf{Algorithm mp\_dr\_reduce.}
This algorithm will perform the Dimished Radix reduction of $x$ modulo $n$.  It has similar restrictions to that of the Barrett reduction
with the addition that $n$ must be of the form $n = \beta^m - k$ where $0 < k <\beta$.  

This algorithm essentially implements the pseudo-code in figure~\ref{fig:DR} except with a slight optimization.  The division by $\beta^m$, multiplication by $k$
and addition of $x \mbox{ mod }\beta^m$ are all performed simultaneously inside the loop on step 4.  The division by $\beta^m$ is emulated by accessing
the term at the $m+i$'th position which is subsequently multiplied by $k$ and added to the term at the $i$'th position.  After the loop the $m$'th
digit is set to the carry and the upper digits are zeroed.  Steps 5 and 6 emulate the reduction modulo $\beta^m$ that should have happend to 
$x$ before the addition of the multiple of the upper half.  

At step 8 if $x$ is still larger than $n$ another pass of the algorithm is required.  First $n$ is subtracted from $x$ and then the algorithm resumes
at step 3.  

\vspace{+3mm}\begin{small}
\hspace{-5.1mm}{\bf File}: bn\_mp\_dr\_reduce.c
\vspace{-3mm}
\begin{alltt}
\end{alltt}
\end{small}

The first step is to grow $x$ as required to $2m$ digits since the reduction is performed in place on $x$.  The label on line 52 is where
the algorithm will resume if further reduction passes are required.  In theory it could be placed at the top of the function however, the size of
the modulus and question of whether $x$ is large enough are invariant after the first pass meaning that it would be a waste of time.  

The aliases $tmpx1$ and $tmpx2$ refer to the digits of $x$ where the latter is offset by $m$ digits.  By reading digits from $x$ offset by $m$ digits
a division by $\beta^m$ can be simulated virtually for free.  The loop on line 64 performs the bulk of the work (\textit{corresponds to step 4 of algorithm 7.11})
in this algorithm.

By line 67 the pointer $tmpx1$ points to the $m$'th digit of $x$ which is where the final carry will be placed.  Similarly by line 74 the 
same pointer will point to the $m+1$'th digit where the zeroes will be placed.  

Since the algorithm is only valid if both $x$ and $n$ are greater than zero an unsigned comparison suffices to determine if another pass is required.  
With the same logic at line 81 the value of $x$ is known to be greater than or equal to $n$ meaning that an unsigned subtraction can be used
as well.  Since the destination of the subtraction is the larger of the inputs the call to algorithm s\_mp\_sub cannot fail and the return code
does not need to be checked.

\subsubsection{Setup}
To setup the restricted Diminished Radix algorithm the value $k = \beta - n_0$ is required.  This algorithm is not really complicated but provided for
completeness.

\begin{figure}[!here]
\begin{small}
\begin{center}
\begin{tabular}{l}
\hline Algorithm \textbf{mp\_dr\_setup}. \\
\textbf{Input}.   mp\_int $n$ \\
\textbf{Output}.  $k = \beta - n_0$ \\
\hline \\
1.  $k \leftarrow \beta - n_0$ \\
\hline
\end{tabular}
\end{center}
\end{small}
\caption{Algorithm mp\_dr\_setup}
\end{figure}

\vspace{+3mm}\begin{small}
\hspace{-5.1mm}{\bf File}: bn\_mp\_dr\_setup.c
\vspace{-3mm}
\begin{alltt}
\end{alltt}
\end{small}

\subsubsection{Modulus Detection}
Another algorithm which will be useful is the ability to detect a restricted Diminished Radix modulus.  An integer is said to be
of restricted Diminished Radix form if all of the digits are equal to $\beta - 1$ except the trailing digit which may be any value.

\begin{figure}[!here]
\begin{small}
\begin{center}
\begin{tabular}{l}
\hline Algorithm \textbf{mp\_dr\_is\_modulus}. \\
\textbf{Input}.   mp\_int $n$ \\
\textbf{Output}.  $1$ if $n$ is in D.R form, $0$ otherwise \\
\hline
1.  If $n.used < 2$ then return($0$). \\
2.  for $ix$ from $1$ to $n.used - 1$ do \\
\hspace{3mm}2.1  If $n_{ix} \ne \beta - 1$ return($0$). \\
3.  Return($1$). \\
\hline
\end{tabular}
\end{center}
\end{small}
\caption{Algorithm mp\_dr\_is\_modulus}
\end{figure}

\textbf{Algorithm mp\_dr\_is\_modulus.}
This algorithm determines if a value is in Diminished Radix form.  Step 1 rejects obvious cases where fewer than two digits are
in the mp\_int.  Step 2 tests all but the first digit to see if they are equal to $\beta - 1$.  If the algorithm manages to get to
step 3 then $n$ must be of Diminished Radix form.

\vspace{+3mm}\begin{small}
\hspace{-5.1mm}{\bf File}: bn\_mp\_dr\_is\_modulus.c
\vspace{-3mm}
\begin{alltt}
\end{alltt}
\end{small}

\subsection{Unrestricted Diminished Radix Reduction}
The unrestricted Diminished Radix algorithm allows modular reductions to be performed when the modulus is of the form $2^p - k$.  This algorithm
is a straightforward adaptation of algorithm~\ref{fig:DR}.

In general the restricted Diminished Radix reduction algorithm is much faster since it has considerably lower overhead.  However, this new
algorithm is much faster than either Montgomery or Barrett reduction when the moduli are of the appropriate form.

\begin{figure}[!here]
\begin{small}
\begin{center}
\begin{tabular}{l}
\hline Algorithm \textbf{mp\_reduce\_2k}. \\
\textbf{Input}.   mp\_int $a$ and $n$.  mp\_digit $k$  \\
\hspace{11.5mm}($a \ge 0$, $n > 1$, $0 < k < \beta$, $n + k$ is a power of two) \\
\textbf{Output}.  $a \mbox{ (mod }n\mbox{)}$ \\
\hline
1.  $p \leftarrow \lceil lg(n) \rceil$  (\textit{mp\_count\_bits}) \\
2.  While $a \ge n$ do \\
\hspace{3mm}2.1  $q \leftarrow \lfloor a / 2^p \rfloor$ (\textit{mp\_div\_2d}) \\
\hspace{3mm}2.2  $a \leftarrow a \mbox{ (mod }2^p\mbox{)}$ (\textit{mp\_mod\_2d}) \\
\hspace{3mm}2.3  $q \leftarrow q \cdot k$ (\textit{mp\_mul\_d}) \\
\hspace{3mm}2.4  $a \leftarrow a - q$ (\textit{s\_mp\_sub}) \\
\hspace{3mm}2.5  If $a \ge n$ then do \\
\hspace{6mm}2.5.1  $a \leftarrow a - n$ \\
3.  Return(\textit{MP\_OKAY}). \\
\hline
\end{tabular}
\end{center}
\end{small}
\caption{Algorithm mp\_reduce\_2k}
\end{figure}

\textbf{Algorithm mp\_reduce\_2k.}
This algorithm quickly reduces an input $a$ modulo an unrestricted Diminished Radix modulus $n$.  Division by $2^p$ is emulated with a right
shift which makes the algorithm fairly inexpensive to use.  

\vspace{+3mm}\begin{small}
\hspace{-5.1mm}{\bf File}: bn\_mp\_reduce\_2k.c
\vspace{-3mm}
\begin{alltt}
\end{alltt}
\end{small}

The algorithm mp\_count\_bits calculates the number of bits in an mp\_int which is used to find the initial value of $p$.  The call to mp\_div\_2d
on line 31 calculates both the quotient $q$ and the remainder $a$ required.  By doing both in a single function call the code size
is kept fairly small.  The multiplication by $k$ is only performed if $k > 1$. This allows reductions modulo $2^p - 1$ to be performed without
any multiplications.  

The unsigned s\_mp\_add, mp\_cmp\_mag and s\_mp\_sub are used in place of their full sign counterparts since the inputs are only valid if they are 
positive.  By using the unsigned versions the overhead is kept to a minimum.  

\subsubsection{Unrestricted Setup}
To setup this reduction algorithm the value of $k = 2^p - n$ is required.  

\begin{figure}[!here]
\begin{small}
\begin{center}
\begin{tabular}{l}
\hline Algorithm \textbf{mp\_reduce\_2k\_setup}. \\
\textbf{Input}.   mp\_int $n$   \\
\textbf{Output}.  $k = 2^p - n$ \\
\hline
1.  $p \leftarrow \lceil lg(n) \rceil$  (\textit{mp\_count\_bits}) \\
2.  $x \leftarrow 2^p$ (\textit{mp\_2expt}) \\
3.  $x \leftarrow x - n$ (\textit{mp\_sub}) \\
4.  $k \leftarrow x_0$ \\
5.  Return(\textit{MP\_OKAY}). \\
\hline
\end{tabular}
\end{center}
\end{small}
\caption{Algorithm mp\_reduce\_2k\_setup}
\end{figure}

\textbf{Algorithm mp\_reduce\_2k\_setup.}
This algorithm computes the value of $k$ required for the algorithm mp\_reduce\_2k.  By making a temporary variable $x$ equal to $2^p$ a subtraction
is sufficient to solve for $k$.  Alternatively if $n$ has more than one digit the value of $k$ is simply $\beta - n_0$.  

\vspace{+3mm}\begin{small}
\hspace{-5.1mm}{\bf File}: bn\_mp\_reduce\_2k\_setup.c
\vspace{-3mm}
\begin{alltt}
\end{alltt}
\end{small}

\subsubsection{Unrestricted Detection}
An integer $n$ is a valid unrestricted Diminished Radix modulus if either of the following are true.

\begin{enumerate}
\item  The number has only one digit.
\item  The number has more than one digit and every bit from the $\beta$'th to the most significant is one.
\end{enumerate}

If either condition is true than there is a power of two $2^p$ such that $0 < 2^p - n < \beta$.   If the input is only
one digit than it will always be of the correct form.  Otherwise all of the bits above the first digit must be one.  This arises from the fact
that there will be value of $k$ that when added to the modulus causes a carry in the first digit which propagates all the way to the most
significant bit.  The resulting sum will be a power of two.

\begin{figure}[!here]
\begin{small}
\begin{center}
\begin{tabular}{l}
\hline Algorithm \textbf{mp\_reduce\_is\_2k}. \\
\textbf{Input}.   mp\_int $n$   \\
\textbf{Output}.  $1$ if of proper form, $0$ otherwise \\
\hline
1.  If $n.used = 0$ then return($0$). \\
2.  If $n.used = 1$ then return($1$). \\
3.  $p \leftarrow \lceil lg(n) \rceil$  (\textit{mp\_count\_bits}) \\
4.  for $x$ from $lg(\beta)$ to $p$ do \\
\hspace{3mm}4.1  If the ($x \mbox{ mod }lg(\beta)$)'th bit of the $\lfloor x / lg(\beta) \rfloor$ of $n$ is zero then return($0$). \\
5.  Return($1$). \\
\hline
\end{tabular}
\end{center}
\end{small}
\caption{Algorithm mp\_reduce\_is\_2k}
\end{figure}

\textbf{Algorithm mp\_reduce\_is\_2k.}
This algorithm quickly determines if a modulus is of the form required for algorithm mp\_reduce\_2k to function properly.  

\vspace{+3mm}\begin{small}
\hspace{-5.1mm}{\bf File}: bn\_mp\_reduce\_is\_2k.c
\vspace{-3mm}
\begin{alltt}
\end{alltt}
\end{small}



\section{Algorithm Comparison}
So far three very different algorithms for modular reduction have been discussed.  Each of the algorithms have their own strengths and weaknesses
that makes having such a selection very useful.  The following table sumarizes the three algorithms along with comparisons of work factors.  Since
all three algorithms have the restriction that $0 \le x < n^2$ and $n > 1$ those limitations are not included in the table.  

\begin{center}
\begin{small}
\begin{tabular}{|c|c|c|c|c|c|}
\hline \textbf{Method} & \textbf{Work Required} & \textbf{Limitations} & \textbf{$m = 8$} & \textbf{$m = 32$} & \textbf{$m = 64$} \\
\hline Barrett    & $m^2 + 2m - 1$ & None              & $79$ & $1087$ & $4223$ \\
\hline Montgomery & $m^2 + m$      & $n$ must be odd   & $72$ & $1056$ & $4160$ \\
\hline D.R.       & $2m$           & $n = \beta^m - k$ & $16$ & $64$   & $128$  \\
\hline
\end{tabular}
\end{small}
\end{center}

In theory Montgomery and Barrett reductions would require roughly the same amount of time to complete.  However, in practice since Montgomery
reduction can be written as a single function with the Comba technique it is much faster.  Barrett reduction suffers from the overhead of
calling the half precision multipliers, addition and division by $\beta$ algorithms.

For almost every cryptographic algorithm Montgomery reduction is the algorithm of choice.  The one set of algorithms where Diminished Radix reduction truly
shines are based on the discrete logarithm problem such as Diffie-Hellman \cite{DH} and ElGamal \cite{ELGAMAL}.  In these algorithms
primes of the form $\beta^m - k$ can be found and shared amongst users.  These primes will allow the Diminished Radix algorithm to be used in
modular exponentiation to greatly speed up the operation.



\section*{Exercises}
\begin{tabular}{cl}
$\left [ 3 \right ]$ & Prove that the ``trick'' in algorithm mp\_montgomery\_setup actually \\
                     & calculates the correct value of $\rho$. \\
                     & \\
$\left [ 2 \right ]$ & Devise an algorithm to reduce modulo $n + k$ for small $k$ quickly.  \\
                     & \\
$\left [ 4 \right ]$ & Prove that the pseudo-code algorithm ``Diminished Radix Reduction'' \\
                     & (\textit{figure~\ref{fig:DR}}) terminates.  Also prove the probability that it will \\
                     & terminate within $1 \le k \le 10$ iterations. \\
                     & \\
\end{tabular}                     


\chapter{Exponentiation}
Exponentiation is the operation of raising one variable to the power of another, for example, $a^b$.  A variant of exponentiation, computed
in a finite field or ring, is called modular exponentiation.  This latter style of operation is typically used in public key 
cryptosystems such as RSA and Diffie-Hellman.  The ability to quickly compute modular exponentiations is of great benefit to any
such cryptosystem and many methods have been sought to speed it up.

\section{Exponentiation Basics}
A trivial algorithm would simply multiply $a$ against itself $b - 1$ times to compute the exponentiation desired.  However, as $b$ grows in size
the number of multiplications becomes prohibitive.  Imagine what would happen if $b$ $\approx$ $2^{1024}$ as is the case when computing an RSA signature
with a $1024$-bit key.  Such a calculation could never be completed as it would take simply far too long.

Fortunately there is a very simple algorithm based on the laws of exponents.  Recall that $lg_a(a^b) = b$ and that $lg_a(a^ba^c) = b + c$ which
are two trivial relationships between the base and the exponent.  Let $b_i$ represent the $i$'th bit of $b$ starting from the least 
significant bit.  If $b$ is a $k$-bit integer than the following equation is true.

\begin{equation}
a^b = \prod_{i=0}^{k-1} a^{2^i \cdot b_i}
\end{equation}

By taking the base $a$ logarithm of both sides of the equation the following equation is the result.

\begin{equation}
b = \sum_{i=0}^{k-1}2^i \cdot b_i
\end{equation}

The term $a^{2^i}$ can be found from the $i - 1$'th term by squaring the term since $\left ( a^{2^i} \right )^2$ is equal to
$a^{2^{i+1}}$.  This observation forms the basis of essentially all fast exponentiation algorithms.  It requires $k$ squarings and on average
$k \over 2$ multiplications to compute the result.  This is indeed quite an improvement over simply multiplying by $a$ a total of $b-1$ times.

While this current method is a considerable speed up there are further improvements to be made.  For example, the $a^{2^i}$ term does not need to 
be computed in an auxilary variable.  Consider the following equivalent algorithm.

\begin{figure}[!here]
\begin{small}
\begin{center}
\begin{tabular}{l}
\hline Algorithm \textbf{Left to Right Exponentiation}. \\
\textbf{Input}.   Integer $a$, $b$ and $k$ \\
\textbf{Output}.  $c = a^b$ \\
\hline \\
1.  $c \leftarrow 1$ \\
2.  for $i$ from $k - 1$ to $0$ do \\
\hspace{3mm}2.1  $c \leftarrow c^2$ \\
\hspace{3mm}2.2  $c \leftarrow c \cdot a^{b_i}$ \\
3.  Return $c$. \\
\hline
\end{tabular}
\end{center}
\end{small}
\caption{Left to Right Exponentiation}
\label{fig:LTOR}
\end{figure}

This algorithm starts from the most significant bit and works towards the least significant bit.  When the $i$'th bit of $b$ is set $a$ is
multiplied against the current product.  In each iteration the product is squared which doubles the exponent of the individual terms of the
product.  

For example, let $b = 101100_2 \equiv 44_{10}$.  The following chart demonstrates the actions of the algorithm.

\newpage\begin{figure}
\begin{center}
\begin{tabular}{|c|c|}
\hline \textbf{Value of $i$} & \textbf{Value of $c$} \\
\hline - & $1$ \\
\hline $5$ & $a$ \\
\hline $4$ & $a^2$ \\
\hline $3$ & $a^4 \cdot a$ \\
\hline $2$ & $a^8 \cdot a^2 \cdot a$ \\
\hline $1$ & $a^{16} \cdot a^4 \cdot a^2$ \\
\hline $0$ & $a^{32} \cdot a^8 \cdot a^4$ \\
\hline
\end{tabular}
\end{center}
\caption{Example of Left to Right Exponentiation}
\end{figure}

When the product $a^{32} \cdot a^8 \cdot a^4$ is simplified it is equal $a^{44}$ which is the desired exponentiation.  This particular algorithm is 
called ``Left to Right'' because it reads the exponent in that order.  All of the exponentiation algorithms that will be presented are of this nature.  

\subsection{Single Digit Exponentiation}
The first algorithm in the series of exponentiation algorithms will be an unbounded algorithm where the exponent is a single digit.  It is intended 
to be used when a small power of an input is required (\textit{e.g. $a^5$}).  It is faster than simply multiplying $b - 1$ times for all values of 
$b$ that are greater than three.  

\newpage\begin{figure}[!here]
\begin{small}
\begin{center}
\begin{tabular}{l}
\hline Algorithm \textbf{mp\_expt\_d}. \\
\textbf{Input}.   mp\_int $a$ and mp\_digit $b$ \\
\textbf{Output}.  $c = a^b$ \\
\hline \\
1.  $g \leftarrow a$ (\textit{mp\_init\_copy}) \\
2.  $c \leftarrow 1$ (\textit{mp\_set}) \\
3.  for $x$ from 1 to $lg(\beta)$ do \\
\hspace{3mm}3.1  $c \leftarrow c^2$ (\textit{mp\_sqr}) \\
\hspace{3mm}3.2  If $b$ AND $2^{lg(\beta) - 1} \ne 0$ then \\
\hspace{6mm}3.2.1  $c \leftarrow c \cdot g$ (\textit{mp\_mul}) \\
\hspace{3mm}3.3  $b \leftarrow b << 1$ \\
4.  Clear $g$. \\
5.  Return(\textit{MP\_OKAY}). \\
\hline
\end{tabular}
\end{center}
\end{small}
\caption{Algorithm mp\_expt\_d}
\end{figure}

\textbf{Algorithm mp\_expt\_d.}
This algorithm computes the value of $a$ raised to the power of a single digit $b$.  It uses the left to right exponentiation algorithm to
quickly compute the exponentiation.  It is loosely based on algorithm 14.79 of HAC \cite[pp. 615]{HAC} with the difference that the 
exponent is a fixed width.  

A copy of $a$ is made first to allow destination variable $c$ be the same as the source variable $a$.  The result is set to the initial value of 
$1$ in the subsequent step.

Inside the loop the exponent is read from the most significant bit first down to the least significant bit.  First $c$ is invariably squared
on step 3.1.  In the following step if the most significant bit of $b$ is one the copy of $a$ is multiplied against $c$.  The value
of $b$ is shifted left one bit to make the next bit down from the most signficant bit the new most significant bit.  In effect each
iteration of the loop moves the bits of the exponent $b$ upwards to the most significant location.

\vspace{+3mm}\begin{small}
\hspace{-5.1mm}{\bf File}: bn\_mp\_expt\_d.c
\vspace{-3mm}
\begin{alltt}
\end{alltt}
\end{small}

Line 29 sets the initial value of the result to $1$.  Next the loop on line 31 steps through each bit of the exponent starting from
the most significant down towards the least significant. The invariant squaring operation placed on line 33 is performed first.  After 
the squaring the result $c$ is multiplied by the base $g$ if and only if the most significant bit of the exponent is set.  The shift on line
47 moves all of the bits of the exponent upwards towards the most significant location.  

\section{$k$-ary Exponentiation}
When calculating an exponentiation the most time consuming bottleneck is the multiplications which are in general a small factor
slower than squaring.  Recall from the previous algorithm that $b_{i}$ refers to the $i$'th bit of the exponent $b$.  Suppose instead it referred to
the $i$'th $k$-bit digit of the exponent of $b$.  For $k = 1$ the definitions are synonymous and for $k > 1$ algorithm~\ref{fig:KARY}
computes the same exponentiation.  A group of $k$ bits from the exponent is called a \textit{window}.  That is it is a small window on only a
portion of the entire exponent.  Consider the following modification to the basic left to right exponentiation algorithm.

\begin{figure}[!here]
\begin{small}
\begin{center}
\begin{tabular}{l}
\hline Algorithm \textbf{$k$-ary Exponentiation}. \\
\textbf{Input}.   Integer $a$, $b$, $k$ and $t$ \\
\textbf{Output}.  $c = a^b$ \\
\hline \\
1.  $c \leftarrow 1$ \\
2.  for $i$ from $t - 1$ to $0$ do \\
\hspace{3mm}2.1  $c \leftarrow c^{2^k} $ \\
\hspace{3mm}2.2  Extract the $i$'th $k$-bit word from $b$ and store it in $g$. \\
\hspace{3mm}2.3  $c \leftarrow c \cdot a^g$ \\
3.  Return $c$. \\
\hline
\end{tabular}
\end{center}
\end{small}
\caption{$k$-ary Exponentiation}
\label{fig:KARY}
\end{figure}

The squaring on step 2.1 can be calculated by squaring the value $c$ successively $k$ times.  If the values of $a^g$ for $0 < g < 2^k$ have been
precomputed this algorithm requires only $t$ multiplications and $tk$ squarings.  The table can be generated with $2^{k - 1} - 1$ squarings and
$2^{k - 1} + 1$ multiplications.  This algorithm assumes that the number of bits in the exponent is evenly divisible by $k$.  
However, when it is not the remaining $0 < x \le k - 1$ bits can be handled with algorithm~\ref{fig:LTOR}.

Suppose $k = 4$ and $t = 100$.  This modified algorithm will require $109$ multiplications and $408$ squarings to compute the exponentiation.  The
original algorithm would on average have required $200$ multiplications and $400$ squrings to compute the same value.  The total number of squarings
has increased slightly but the number of multiplications has nearly halved.

\subsection{Optimal Values of $k$}
An optimal value of $k$ will minimize $2^{k} + \lceil n / k \rceil + n - 1$ for a fixed number of bits in the exponent $n$.  The simplest
approach is to brute force search amongst the values $k = 2, 3, \ldots, 8$ for the lowest result.  Table~\ref{fig:OPTK} lists optimal values of $k$
for various exponent sizes and compares the number of multiplication and squarings required against algorithm~\ref{fig:LTOR}.  

\begin{figure}[here]
\begin{center}
\begin{small}
\begin{tabular}{|c|c|c|c|c|c|}
\hline \textbf{Exponent (bits)} & \textbf{Optimal $k$} & \textbf{Work at $k$} & \textbf{Work with ~\ref{fig:LTOR}} \\
\hline $16$ & $2$ & $27$ & $24$ \\
\hline $32$ & $3$ & $49$ & $48$ \\
\hline $64$ & $3$ & $92$ & $96$ \\
\hline $128$ & $4$ & $175$ & $192$ \\
\hline $256$ & $4$ & $335$ & $384$ \\
\hline $512$ & $5$ & $645$ & $768$ \\
\hline $1024$ & $6$ & $1257$ & $1536$ \\
\hline $2048$ & $6$ & $2452$ & $3072$ \\
\hline $4096$ & $7$ & $4808$ & $6144$ \\
\hline
\end{tabular}
\end{small}
\end{center}
\caption{Optimal Values of $k$ for $k$-ary Exponentiation}
\label{fig:OPTK}
\end{figure}

\subsection{Sliding-Window Exponentiation}
A simple modification to the previous algorithm is only generate the upper half of the table in the range $2^{k-1} \le g < 2^k$.  Essentially
this is a table for all values of $g$ where the most significant bit of $g$ is a one.  However, in order for this to be allowed in the 
algorithm values of $g$ in the range $0 \le g < 2^{k-1}$ must be avoided.  

Table~\ref{fig:OPTK2} lists optimal values of $k$ for various exponent sizes and compares the work required against algorithm~\ref{fig:KARY}.  

\begin{figure}[here]
\begin{center}
\begin{small}
\begin{tabular}{|c|c|c|c|c|c|}
\hline \textbf{Exponent (bits)} & \textbf{Optimal $k$} & \textbf{Work at $k$} & \textbf{Work with ~\ref{fig:KARY}} \\
\hline $16$ & $3$ & $24$ & $27$ \\
\hline $32$ & $3$ & $45$ & $49$ \\
\hline $64$ & $4$ & $87$ & $92$ \\
\hline $128$ & $4$ & $167$ & $175$ \\
\hline $256$ & $5$ & $322$ & $335$ \\
\hline $512$ & $6$ & $628$ & $645$ \\
\hline $1024$ & $6$ & $1225$ & $1257$ \\
\hline $2048$ & $7$ & $2403$ & $2452$ \\
\hline $4096$ & $8$ & $4735$ & $4808$ \\
\hline
\end{tabular}
\end{small}
\end{center}
\caption{Optimal Values of $k$ for Sliding Window Exponentiation}
\label{fig:OPTK2}
\end{figure}

\newpage\begin{figure}[!here]
\begin{small}
\begin{center}
\begin{tabular}{l}
\hline Algorithm \textbf{Sliding Window $k$-ary Exponentiation}. \\
\textbf{Input}.   Integer $a$, $b$, $k$ and $t$ \\
\textbf{Output}.  $c = a^b$ \\
\hline \\
1.  $c \leftarrow 1$ \\
2.  for $i$ from $t - 1$ to $0$ do \\
\hspace{3mm}2.1  If the $i$'th bit of $b$ is a zero then \\
\hspace{6mm}2.1.1   $c \leftarrow c^2$ \\
\hspace{3mm}2.2  else do \\
\hspace{6mm}2.2.1  $c \leftarrow c^{2^k}$ \\
\hspace{6mm}2.2.2  Extract the $k$ bits from $(b_{i}b_{i-1}\ldots b_{i-(k-1)})$ and store it in $g$. \\
\hspace{6mm}2.2.3  $c \leftarrow c \cdot a^g$ \\
\hspace{6mm}2.2.4  $i \leftarrow i - k$ \\
3.  Return $c$. \\
\hline
\end{tabular}
\end{center}
\end{small}
\caption{Sliding Window $k$-ary Exponentiation}
\end{figure}

Similar to the previous algorithm this algorithm must have a special handler when fewer than $k$ bits are left in the exponent.  While this
algorithm requires the same number of squarings it can potentially have fewer multiplications.  The pre-computed table $a^g$ is also half
the size as the previous table.  

Consider the exponent $b = 111101011001000_2 \equiv 31432_{10}$ with $k = 3$ using both algorithms.  The first algorithm will divide the exponent up as 
the following five $3$-bit words $b \equiv \left ( 111, 101, 011, 001, 000 \right )_{2}$.  The second algorithm will break the 
exponent as $b \equiv \left ( 111, 101, 0, 110, 0, 100, 0 \right )_{2}$.  The single digit $0$ in the second representation are where
a single squaring took place instead of a squaring and multiplication.  In total the first method requires $10$ multiplications and $18$ 
squarings.  The second method requires $8$ multiplications and $18$ squarings.  

In general the sliding window method is never slower than the generic $k$-ary method and often it is slightly faster.  

\section{Modular Exponentiation}

Modular exponentiation is essentially computing the power of a base within a finite field or ring.  For example, computing 
$d \equiv a^b \mbox{ (mod }c\mbox{)}$ is a modular exponentiation.  Instead of first computing $a^b$ and then reducing it 
modulo $c$ the intermediate result is reduced modulo $c$ after every squaring or multiplication operation.  

This guarantees that any intermediate result is bounded by $0 \le d \le c^2 - 2c + 1$ and can be reduced modulo $c$ quickly using
one of the algorithms presented in chapter six.  

Before the actual modular exponentiation algorithm can be written a wrapper algorithm must be written first.  This algorithm
will allow the exponent $b$ to be negative which is computed as $c \equiv \left (1 / a \right )^{\vert b \vert} \mbox{(mod }d\mbox{)}$. The
value of $(1/a) \mbox{ mod }c$ is computed using the modular inverse (\textit{see \ref{sec;modinv}}).  If no inverse exists the algorithm
terminates with an error.  

\begin{figure}[!here]
\begin{small}
\begin{center}
\begin{tabular}{l}
\hline Algorithm \textbf{mp\_exptmod}. \\
\textbf{Input}.   mp\_int $a$, $b$ and $c$ \\
\textbf{Output}.  $y \equiv g^x \mbox{ (mod }p\mbox{)}$ \\
\hline \\
1.  If $c.sign = MP\_NEG$ return(\textit{MP\_VAL}). \\
2.  If $b.sign = MP\_NEG$ then \\
\hspace{3mm}2.1  $g' \leftarrow g^{-1} \mbox{ (mod }c\mbox{)}$ \\
\hspace{3mm}2.2  $x' \leftarrow \vert x \vert$ \\
\hspace{3mm}2.3  Compute $d \equiv g'^{x'} \mbox{ (mod }c\mbox{)}$ via recursion. \\
3.  if $p$ is odd \textbf{OR} $p$ is a D.R. modulus then \\
\hspace{3mm}3.1  Compute $y \equiv g^{x} \mbox{ (mod }p\mbox{)}$ via algorithm mp\_exptmod\_fast. \\
4.  else \\
\hspace{3mm}4.1  Compute $y \equiv g^{x} \mbox{ (mod }p\mbox{)}$ via algorithm s\_mp\_exptmod. \\
\hline
\end{tabular}
\end{center}
\end{small}
\caption{Algorithm mp\_exptmod}
\end{figure}

\textbf{Algorithm mp\_exptmod.}
The first algorithm which actually performs modular exponentiation is algorithm s\_mp\_exptmod.  It is a sliding window $k$-ary algorithm 
which uses Barrett reduction to reduce the product modulo $p$.  The second algorithm mp\_exptmod\_fast performs the same operation 
except it uses either Montgomery or Diminished Radix reduction.  The two latter reduction algorithms are clumped in the same exponentiation
algorithm since their arguments are essentially the same (\textit{two mp\_ints and one mp\_digit}).  

\vspace{+3mm}\begin{small}
\hspace{-5.1mm}{\bf File}: bn\_mp\_exptmod.c
\vspace{-3mm}
\begin{alltt}
\end{alltt}
\end{small}

In order to keep the algorithms in a known state the first step on line 29 is to reject any negative modulus as input.  If the exponent is
negative the algorithm tries to perform a modular exponentiation with the modular inverse of the base $G$.  The temporary variable $tmpG$ is assigned
the modular inverse of $G$ and $tmpX$ is assigned the absolute value of $X$.  The algorithm will recuse with these new values with a positive
exponent.

If the exponent is positive the algorithm resumes the exponentiation.  Line 77 determines if the modulus is of the restricted Diminished Radix 
form.  If it is not line 70 attempts to determine if it is of a unrestricted Diminished Radix form.  The integer $dr$ will take on one
of three values.

\begin{enumerate}
\item $dr = 0$ means that the modulus is not of either restricted or unrestricted Diminished Radix form.
\item $dr = 1$ means that the modulus is of restricted Diminished Radix form.
\item $dr = 2$ means that the modulus is of unrestricted Diminished Radix form.
\end{enumerate}

Line 69 determines if the fast modular exponentiation algorithm can be used.  It is allowed if $dr \ne 0$ or if the modulus is odd.  Otherwise,
the slower s\_mp\_exptmod algorithm is used which uses Barrett reduction.  

\subsection{Barrett Modular Exponentiation}

\newpage\begin{figure}[!here]
\begin{small}
\begin{center}
\begin{tabular}{l}
\hline Algorithm \textbf{s\_mp\_exptmod}. \\
\textbf{Input}.   mp\_int $a$, $b$ and $c$ \\
\textbf{Output}.  $y \equiv g^x \mbox{ (mod }p\mbox{)}$ \\
\hline \\
1.  $k \leftarrow lg(x)$ \\
2.  $winsize \leftarrow  \left \lbrace \begin{array}{ll}
                              2 &  \mbox{if }k \le 7 \\
                              3 &  \mbox{if }7 < k \le 36 \\
                              4 &  \mbox{if }36 < k \le 140 \\
                              5 &  \mbox{if }140 < k \le 450 \\
                              6 &  \mbox{if }450 < k \le 1303 \\
                              7 &  \mbox{if }1303 < k \le 3529 \\
                              8 &  \mbox{if }3529 < k \\
                              \end{array} \right .$ \\
3.  Initialize $2^{winsize}$ mp\_ints in an array named $M$ and one mp\_int named $\mu$ \\
4.  Calculate the $\mu$ required for Barrett Reduction (\textit{mp\_reduce\_setup}). \\
5.  $M_1 \leftarrow g \mbox{ (mod }p\mbox{)}$ \\
\\
Setup the table of small powers of $g$.  First find $g^{2^{winsize}}$ and then all multiples of it. \\
6.  $k \leftarrow 2^{winsize - 1}$ \\
7.  $M_{k} \leftarrow M_1$ \\
8.  for $ix$ from 0 to $winsize - 2$ do \\
\hspace{3mm}8.1  $M_k \leftarrow \left ( M_k \right )^2$ (\textit{mp\_sqr})  \\
\hspace{3mm}8.2  $M_k \leftarrow M_k \mbox{ (mod }p\mbox{)}$ (\textit{mp\_reduce}) \\
9.  for $ix$ from $2^{winsize - 1} + 1$ to $2^{winsize} - 1$ do \\
\hspace{3mm}9.1  $M_{ix} \leftarrow M_{ix - 1} \cdot M_{1}$ (\textit{mp\_mul}) \\
\hspace{3mm}9.2  $M_{ix} \leftarrow M_{ix} \mbox{ (mod }p\mbox{)}$ (\textit{mp\_reduce}) \\
10.  $res \leftarrow 1$ \\
\\
Start Sliding Window. \\
11.  $mode \leftarrow 0, bitcnt \leftarrow 1, buf \leftarrow 0, digidx \leftarrow x.used - 1, bitcpy \leftarrow 0, bitbuf \leftarrow 0$ \\
12.  Loop \\
\hspace{3mm}12.1  $bitcnt \leftarrow bitcnt - 1$ \\
\hspace{3mm}12.2  If $bitcnt = 0$ then do \\
\hspace{6mm}12.2.1  If $digidx = -1$ goto step 13. \\
\hspace{6mm}12.2.2  $buf \leftarrow x_{digidx}$ \\
\hspace{6mm}12.2.3  $digidx \leftarrow digidx - 1$ \\
\hspace{6mm}12.2.4  $bitcnt \leftarrow lg(\beta)$ \\
Continued on next page. \\
\hline
\end{tabular}
\end{center}
\end{small}
\caption{Algorithm s\_mp\_exptmod}
\end{figure}

\newpage\begin{figure}[!here]
\begin{small}
\begin{center}
\begin{tabular}{l}
\hline Algorithm \textbf{s\_mp\_exptmod} (\textit{continued}). \\
\textbf{Input}.   mp\_int $a$, $b$ and $c$ \\
\textbf{Output}.  $y \equiv g^x \mbox{ (mod }p\mbox{)}$ \\
\hline \\
\hspace{3mm}12.3  $y \leftarrow (buf >> (lg(\beta) - 1))$ AND $1$ \\
\hspace{3mm}12.4  $buf \leftarrow buf << 1$ \\
\hspace{3mm}12.5  if $mode = 0$ and $y = 0$ then goto step 12. \\
\hspace{3mm}12.6  if $mode = 1$ and $y = 0$ then do \\
\hspace{6mm}12.6.1  $res \leftarrow res^2$ \\
\hspace{6mm}12.6.2  $res \leftarrow res \mbox{ (mod }p\mbox{)}$ \\
\hspace{6mm}12.6.3  Goto step 12. \\
\hspace{3mm}12.7  $bitcpy \leftarrow bitcpy + 1$ \\
\hspace{3mm}12.8  $bitbuf \leftarrow bitbuf + (y << (winsize - bitcpy))$ \\
\hspace{3mm}12.9  $mode \leftarrow 2$ \\
\hspace{3mm}12.10  If $bitcpy = winsize$ then do \\
\hspace{6mm}Window is full so perform the squarings and single multiplication. \\
\hspace{6mm}12.10.1  for $ix$ from $0$ to $winsize -1$ do \\
\hspace{9mm}12.10.1.1  $res \leftarrow res^2$ \\
\hspace{9mm}12.10.1.2  $res \leftarrow res \mbox{ (mod }p\mbox{)}$ \\
\hspace{6mm}12.10.2  $res \leftarrow res \cdot M_{bitbuf}$ \\
\hspace{6mm}12.10.3  $res \leftarrow res \mbox{ (mod }p\mbox{)}$ \\
\hspace{6mm}Reset the window. \\
\hspace{6mm}12.10.4  $bitcpy \leftarrow 0, bitbuf \leftarrow 0, mode \leftarrow 1$ \\
\\
No more windows left.  Check for residual bits of exponent. \\
13.  If $mode = 2$ and $bitcpy > 0$ then do \\
\hspace{3mm}13.1  for $ix$ form $0$ to $bitcpy - 1$ do \\
\hspace{6mm}13.1.1  $res \leftarrow res^2$ \\
\hspace{6mm}13.1.2  $res \leftarrow res \mbox{ (mod }p\mbox{)}$ \\
\hspace{6mm}13.1.3  $bitbuf \leftarrow bitbuf << 1$ \\
\hspace{6mm}13.1.4  If $bitbuf$ AND $2^{winsize} \ne 0$ then do \\
\hspace{9mm}13.1.4.1  $res \leftarrow res \cdot M_{1}$ \\
\hspace{9mm}13.1.4.2  $res \leftarrow res \mbox{ (mod }p\mbox{)}$ \\
14.  $y \leftarrow res$ \\
15.  Clear $res$, $mu$ and the $M$ array. \\
16.  Return(\textit{MP\_OKAY}). \\
\hline
\end{tabular}
\end{center}
\end{small}
\caption{Algorithm s\_mp\_exptmod (continued)}
\end{figure}

\textbf{Algorithm s\_mp\_exptmod.}
This algorithm computes the $x$'th power of $g$ modulo $p$ and stores the result in $y$.  It takes advantage of the Barrett reduction
algorithm to keep the product small throughout the algorithm.

The first two steps determine the optimal window size based on the number of bits in the exponent.  The larger the exponent the 
larger the window size becomes.  After a window size $winsize$ has been chosen an array of $2^{winsize}$ mp\_int variables is allocated.  This
table will hold the values of $g^x \mbox{ (mod }p\mbox{)}$ for $2^{winsize - 1} \le x < 2^{winsize}$.  

After the table is allocated the first power of $g$ is found.  Since $g \ge p$ is allowed it must be first reduced modulo $p$ to make
the rest of the algorithm more efficient.  The first element of the table at $2^{winsize - 1}$ is found by squaring $M_1$ successively $winsize - 2$
times.  The rest of the table elements are found by multiplying the previous element by $M_1$ modulo $p$.

Now that the table is available the sliding window may begin.  The following list describes the functions of all the variables in the window.
\begin{enumerate}
\item The variable $mode$ dictates how the bits of the exponent are interpreted.  
\begin{enumerate}
   \item When $mode = 0$ the bits are ignored since no non-zero bit of the exponent has been seen yet.  For example, if the exponent were simply 
         $1$ then there would be $lg(\beta) - 1$ zero bits before the first non-zero bit.  In this case bits are ignored until a non-zero bit is found.  
   \item When $mode = 1$ a non-zero bit has been seen before and a new $winsize$-bit window has not been formed yet.  In this mode leading $0$ bits 
         are read and a single squaring is performed.  If a non-zero bit is read a new window is created.  
   \item When $mode = 2$ the algorithm is in the middle of forming a window and new bits are appended to the window from the most significant bit
         downwards.
\end{enumerate}
\item The variable $bitcnt$ indicates how many bits are left in the current digit of the exponent left to be read.  When it reaches zero a new digit
      is fetched from the exponent.
\item The variable $buf$ holds the currently read digit of the exponent. 
\item The variable $digidx$ is an index into the exponents digits.  It starts at the leading digit $x.used - 1$ and moves towards the trailing digit.
\item The variable $bitcpy$ indicates how many bits are in the currently formed window.  When it reaches $winsize$ the window is flushed and
      the appropriate operations performed.
\item The variable $bitbuf$ holds the current bits of the window being formed.  
\end{enumerate}

All of step 12 is the window processing loop.  It will iterate while there are digits available form the exponent to read.  The first step
inside this loop is to extract a new digit if no more bits are available in the current digit.  If there are no bits left a new digit is
read and if there are no digits left than the loop terminates.  

After a digit is made available step 12.3 will extract the most significant bit of the current digit and move all other bits in the digit
upwards.  In effect the digit is read from most significant bit to least significant bit and since the digits are read from leading to 
trailing edges the entire exponent is read from most significant bit to least significant bit.

At step 12.5 if the $mode$ and currently extracted bit $y$ are both zero the bit is ignored and the next bit is read.  This prevents the 
algorithm from having to perform trivial squaring and reduction operations before the first non-zero bit is read.  Step 12.6 and 12.7-10 handle
the two cases of $mode = 1$ and $mode = 2$ respectively.  

\begin{center}
\begin{figure}[here]
\includegraphics{pics/expt_state.ps}
\caption{Sliding Window State Diagram}
\label{pic:expt_state}
\end{figure}
\end{center}

By step 13 there are no more digits left in the exponent.  However, there may be partial bits in the window left.  If $mode = 2$ then 
a Left-to-Right algorithm is used to process the remaining few bits.  

\vspace{+3mm}\begin{small}
\hspace{-5.1mm}{\bf File}: bn\_s\_mp\_exptmod.c
\vspace{-3mm}
\begin{alltt}
\end{alltt}
\end{small}

Lines 32 through 46 determine the optimal window size based on the length of the exponent in bits.  The window divisions are sorted
from smallest to greatest so that in each \textbf{if} statement only one condition must be tested.  For example, by the \textbf{if} statement 
on line 38 the value of $x$ is already known to be greater than $140$.  

The conditional piece of code beginning on line 48 allows the window size to be restricted to five bits.  This logic is used to ensure
the table of precomputed powers of $G$ remains relatively small.  

The for loop on line 61 initializes the $M$ array while lines 72 and 77 through 86 initialize the reduction
function that will be used for this modulus.

-- More later.

\section{Quick Power of Two}
Calculating $b = 2^a$ can be performed much quicker than with any of the previous algorithms.  Recall that a logical shift left $m << k$ is
equivalent to $m \cdot 2^k$.  By this logic when $m = 1$ a quick power of two can be achieved.

\begin{figure}[!here]
\begin{small}
\begin{center}
\begin{tabular}{l}
\hline Algorithm \textbf{mp\_2expt}. \\
\textbf{Input}.   integer $b$ \\
\textbf{Output}.  $a \leftarrow 2^b$ \\
\hline \\
1.  $a \leftarrow 0$ \\
2.  If $a.alloc < \lfloor b / lg(\beta) \rfloor + 1$ then grow $a$ appropriately. \\
3.  $a.used \leftarrow \lfloor b / lg(\beta) \rfloor + 1$ \\
4.  $a_{\lfloor b / lg(\beta) \rfloor} \leftarrow 1 << (b \mbox{ mod } lg(\beta))$ \\
5.  Return(\textit{MP\_OKAY}). \\
\hline
\end{tabular}
\end{center}
\end{small}
\caption{Algorithm mp\_2expt}
\end{figure}

\textbf{Algorithm mp\_2expt.}

\vspace{+3mm}\begin{small}
\hspace{-5.1mm}{\bf File}: bn\_mp\_2expt.c
\vspace{-3mm}
\begin{alltt}
\end{alltt}
\end{small}

\chapter{Higher Level Algorithms}

This chapter discusses the various higher level algorithms that are required to complete a well rounded multiple precision integer package.  These
routines are less performance oriented than the algorithms of chapters five, six and seven but are no less important.  

The first section describes a method of integer division with remainder that is universally well known.  It provides the signed division logic
for the package.  The subsequent section discusses a set of algorithms which allow a single digit to be the 2nd operand for a variety of operations.  
These algorithms serve mostly to simplify other algorithms where small constants are required.  The last two sections discuss how to manipulate 
various representations of integers.  For example, converting from an mp\_int to a string of character.

\section{Integer Division with Remainder}
\label{sec:division}

Integer division aside from modular exponentiation is the most intensive algorithm to compute.  Like addition, subtraction and multiplication
the basis of this algorithm is the long-hand division algorithm taught to school children.  Throughout this discussion several common variables
will be used.  Let $x$ represent the divisor and $y$ represent the dividend.  Let $q$ represent the integer quotient $\lfloor y / x \rfloor$ and 
let $r$ represent the remainder $r = y - x \lfloor y / x \rfloor$.  The following simple algorithm will be used to start the discussion.

\newpage\begin{figure}[!here]
\begin{small}
\begin{center}
\begin{tabular}{l}
\hline Algorithm \textbf{Radix-$\beta$ Integer Division}. \\
\textbf{Input}.   integer $x$ and $y$ \\
\textbf{Output}.  $q = \lfloor y/x\rfloor, r = y - xq$ \\
\hline \\
1.  $q \leftarrow 0$ \\
2.  $n \leftarrow \vert \vert y \vert \vert - \vert \vert x \vert \vert$ \\
3.  for $t$ from $n$ down to $0$ do \\
\hspace{3mm}3.1  Maximize $k$ such that $kx\beta^t$ is less than or equal to $y$ and $(k + 1)x\beta^t$ is greater. \\
\hspace{3mm}3.2  $q \leftarrow q + k\beta^t$ \\
\hspace{3mm}3.3  $y \leftarrow y - kx\beta^t$ \\
4.  $r \leftarrow y$ \\
5.  Return($q, r$) \\
\hline
\end{tabular}
\end{center}
\end{small}
\caption{Algorithm Radix-$\beta$ Integer Division}
\label{fig:raddiv}
\end{figure}

As children we are taught this very simple algorithm for the case of $\beta = 10$.  Almost instinctively several optimizations are taught for which
their reason of existing are never explained.  For this example let $y = 5471$ represent the dividend and $x = 23$ represent the divisor.

To find the first digit of the quotient the value of $k$ must be maximized such that $kx\beta^t$ is less than or equal to $y$ and 
simultaneously $(k + 1)x\beta^t$ is greater than $y$.  Implicitly $k$ is the maximum value the $t$'th digit of the quotient may have.  The habitual method
used to find the maximum is to ``eyeball'' the two numbers, typically only the leading digits and quickly estimate a quotient.  By only using leading
digits a much simpler division may be used to form an educated guess at what the value must be.  In this case $k = \lfloor 54/23\rfloor = 2$ quickly 
arises as a possible  solution.  Indeed $2x\beta^2 = 4600$ is less than $y = 5471$ and simultaneously $(k + 1)x\beta^2 = 6900$ is larger than $y$.  
As a  result $k\beta^2$ is added to the quotient which now equals $q = 200$ and $4600$ is subtracted from $y$ to give a remainder of $y = 841$.

Again this process is repeated to produce the quotient digit $k = 3$ which makes the quotient $q = 200 + 3\beta = 230$ and the remainder 
$y = 841 - 3x\beta = 181$.  Finally the last iteration of the loop produces $k = 7$ which leads to the quotient $q = 230 + 7 = 237$ and the
remainder $y = 181 - 7x = 20$.  The final quotient and remainder found are $q = 237$ and $r = y = 20$ which are indeed correct since 
$237 \cdot 23 + 20 = 5471$ is true.  

\subsection{Quotient Estimation}
\label{sec:divest}
As alluded to earlier the quotient digit $k$ can be estimated from only the leading digits of both the divisor and dividend.  When $p$ leading
digits are used from both the divisor and dividend to form an estimation the accuracy of the estimation rises as $p$ grows.  Technically
speaking the estimation is based on assuming the lower $\vert \vert y \vert \vert - p$ and $\vert \vert x \vert \vert - p$ lower digits of the
dividend and divisor are zero.  

The value of the estimation may off by a few values in either direction and in general is fairly correct.  A simplification \cite[pp. 271]{TAOCPV2}
of the estimation technique is to use $t + 1$ digits of the dividend and $t$ digits of the divisor, in particularly when $t = 1$.  The estimate 
using this technique is never too small.  For the following proof let $t = \vert \vert y \vert \vert - 1$ and $s = \vert \vert x \vert \vert - 1$ 
represent the most significant digits of the dividend and divisor respectively.

\textbf{Proof.}\textit{  The quotient $\hat k = \lfloor (y_t\beta + y_{t-1}) / x_s \rfloor$ is greater than or equal to 
$k = \lfloor y / (x \cdot \beta^{\vert \vert y \vert \vert - \vert \vert x \vert \vert - 1}) \rfloor$. }
The first obvious case is when $\hat k = \beta - 1$ in which case the proof is concluded since the real quotient cannot be larger.  For all other 
cases $\hat k = \lfloor (y_t\beta + y_{t-1}) / x_s \rfloor$ and $\hat k x_s \ge y_t\beta + y_{t-1} - x_s + 1$.  The latter portion of the inequalility
$-x_s + 1$ arises from the fact that a truncated integer division will give the same quotient for at most $x_s - 1$ values.  Next a series of 
inequalities will prove the hypothesis.

\begin{equation}
y - \hat k x \le y - \hat k x_s\beta^s
\end{equation}

This is trivially true since $x \ge x_s\beta^s$.  Next we replace $\hat kx_s\beta^s$ by the previous inequality for $\hat kx_s$.  

\begin{equation}
y - \hat k x \le y_t\beta^t + \ldots + y_0 - (y_t\beta^t + y_{t-1}\beta^{t-1} - x_s\beta^t + \beta^s)
\end{equation}

By simplifying the previous inequality the following inequality is formed.

\begin{equation}
y - \hat k x \le y_{t-2}\beta^{t-2} + \ldots + y_0 + x_s\beta^s - \beta^s
\end{equation}

Subsequently,

\begin{equation}
y_{t-2}\beta^{t-2} + \ldots +  y_0  + x_s\beta^s - \beta^s < x_s\beta^s \le x
\end{equation}

Which proves that $y - \hat kx \le x$ and by consequence $\hat k \ge k$ which concludes the proof.  \textbf{QED}


\subsection{Normalized Integers}
For the purposes of division a normalized input is when the divisors leading digit $x_n$ is greater than or equal to $\beta / 2$.  By multiplying both
$x$ and $y$ by $j = \lfloor (\beta / 2) / x_n \rfloor$ the quotient remains unchanged and the remainder is simply $j$ times the original
remainder.  The purpose of normalization is to ensure the leading digit of the divisor is sufficiently large such that the estimated quotient will
lie in the domain of a single digit.  Consider the maximum dividend $(\beta - 1) \cdot \beta + (\beta - 1)$ and the minimum divisor $\beta / 2$.  

\begin{equation} 
{{\beta^2 - 1} \over { \beta / 2}} \le 2\beta - {2 \over \beta} 
\end{equation}

At most the quotient approaches $2\beta$, however, in practice this will not occur since that would imply the previous quotient digit was too small.  

\subsection{Radix-$\beta$ Division with Remainder}
\newpage\begin{figure}[!here]
\begin{small}
\begin{center}
\begin{tabular}{l}
\hline Algorithm \textbf{mp\_div}. \\
\textbf{Input}.   mp\_int $a, b$ \\
\textbf{Output}.  $c = \lfloor a/b \rfloor$, $d = a - bc$ \\
\hline \\
1.  If $b = 0$ return(\textit{MP\_VAL}). \\
2.  If $\vert a \vert < \vert b \vert$ then do \\
\hspace{3mm}2.1  $d \leftarrow a$ \\
\hspace{3mm}2.2  $c \leftarrow 0$ \\
\hspace{3mm}2.3  Return(\textit{MP\_OKAY}). \\
\\
Setup the quotient to receive the digits. \\
3.  Grow $q$ to $a.used + 2$ digits. \\
4.  $q \leftarrow 0$ \\
5.  $x \leftarrow \vert a \vert , y \leftarrow \vert b \vert$ \\
6.  $sign \leftarrow  \left \lbrace \begin{array}{ll}
                              MP\_ZPOS &  \mbox{if }a.sign = b.sign \\
                              MP\_NEG  &  \mbox{otherwise} \\
                              \end{array} \right .$ \\
\\
Normalize the inputs such that the leading digit of $y$ is greater than or equal to $\beta / 2$. \\
7.  $norm \leftarrow (lg(\beta) - 1) - (\lceil lg(y) \rceil \mbox{ (mod }lg(\beta)\mbox{)})$ \\
8.  $x \leftarrow x \cdot 2^{norm}, y \leftarrow y \cdot 2^{norm}$ \\
\\
Find the leading digit of the quotient. \\
9.  $n \leftarrow x.used - 1, t \leftarrow y.used - 1$ \\
10.  $y \leftarrow y \cdot \beta^{n - t}$ \\
11.  While ($x \ge y$) do \\
\hspace{3mm}11.1  $q_{n - t} \leftarrow q_{n - t} + 1$ \\
\hspace{3mm}11.2  $x \leftarrow x - y$ \\
12.  $y \leftarrow \lfloor y / \beta^{n-t} \rfloor$ \\
\\
Continued on the next page. \\
\hline
\end{tabular}
\end{center}
\end{small}
\caption{Algorithm mp\_div}
\end{figure}

\newpage\begin{figure}[!here]
\begin{small}
\begin{center}
\begin{tabular}{l}
\hline Algorithm \textbf{mp\_div} (continued). \\
\textbf{Input}.   mp\_int $a, b$ \\
\textbf{Output}.  $c = \lfloor a/b \rfloor$, $d = a - bc$ \\
\hline \\
Now find the remainder fo the digits. \\
13.  for $i$ from $n$ down to $(t + 1)$ do \\
\hspace{3mm}13.1  If $i > x.used$ then jump to the next iteration of this loop. \\
\hspace{3mm}13.2  If $x_{i} = y_{t}$ then \\
\hspace{6mm}13.2.1  $q_{i - t - 1} \leftarrow \beta - 1$ \\
\hspace{3mm}13.3  else \\
\hspace{6mm}13.3.1  $\hat r \leftarrow x_{i} \cdot \beta + x_{i - 1}$ \\
\hspace{6mm}13.3.2  $\hat r \leftarrow \lfloor \hat r / y_{t} \rfloor$ \\
\hspace{6mm}13.3.3  $q_{i - t - 1} \leftarrow \hat r$ \\
\hspace{3mm}13.4  $q_{i - t - 1} \leftarrow q_{i - t - 1} + 1$ \\
\\
Fixup quotient estimation. \\
\hspace{3mm}13.5  Loop \\
\hspace{6mm}13.5.1  $q_{i - t - 1} \leftarrow q_{i - t - 1} - 1$ \\
\hspace{6mm}13.5.2  t$1 \leftarrow 0$ \\
\hspace{6mm}13.5.3  t$1_0 \leftarrow y_{t - 1}, $ t$1_1 \leftarrow y_t,$ t$1.used \leftarrow 2$ \\
\hspace{6mm}13.5.4  $t1 \leftarrow t1 \cdot q_{i - t - 1}$ \\
\hspace{6mm}13.5.5  t$2_0 \leftarrow x_{i - 2}, $ t$2_1 \leftarrow x_{i - 1}, $ t$2_2 \leftarrow x_i, $ t$2.used \leftarrow 3$ \\
\hspace{6mm}13.5.6  If $\vert t1 \vert > \vert t2 \vert$ then goto step 13.5. \\
\hspace{3mm}13.6  t$1 \leftarrow y \cdot q_{i - t - 1}$ \\
\hspace{3mm}13.7  t$1 \leftarrow $ t$1 \cdot \beta^{i - t - 1}$ \\
\hspace{3mm}13.8  $x \leftarrow x - $ t$1$ \\
\hspace{3mm}13.9  If $x.sign = MP\_NEG$ then \\
\hspace{6mm}13.10  t$1 \leftarrow y$ \\
\hspace{6mm}13.11  t$1 \leftarrow $ t$1 \cdot \beta^{i - t - 1}$ \\
\hspace{6mm}13.12  $x \leftarrow x + $ t$1$ \\
\hspace{6mm}13.13  $q_{i - t - 1} \leftarrow q_{i - t - 1} - 1$ \\
\\
Finalize the result. \\
14.  Clamp excess digits of $q$ \\
15.  $c \leftarrow q, c.sign \leftarrow sign$ \\
16.  $x.sign \leftarrow a.sign$ \\
17.  $d \leftarrow \lfloor x / 2^{norm} \rfloor$ \\
18.  Return(\textit{MP\_OKAY}). \\
\hline
\end{tabular}
\end{center}
\end{small}
\caption{Algorithm mp\_div (continued)}
\end{figure}
\textbf{Algorithm mp\_div.}
This algorithm will calculate quotient and remainder from an integer division given a dividend and divisor.  The algorithm is a signed
division and will produce a fully qualified quotient and remainder.

First the divisor $b$ must be non-zero which is enforced in step one.  If the divisor is larger than the dividend than the quotient is implicitly 
zero and the remainder is the dividend.  

After the first two trivial cases of inputs are handled the variable $q$ is setup to receive the digits of the quotient.  Two unsigned copies of the
divisor $y$ and dividend $x$ are made as well.  The core of the division algorithm is an unsigned division and will only work if the values are
positive.  Now the two values $x$ and $y$ must be normalized such that the leading digit of $y$ is greater than or equal to $\beta / 2$.  
This is performed by shifting both to the left by enough bits to get the desired normalization.  

At this point the division algorithm can begin producing digits of the quotient.  Recall that maximum value of the estimation used is 
$2\beta - {2 \over \beta}$ which means that a digit of the quotient must be first produced by another means.  In this case $y$ is shifted
to the left (\textit{step ten}) so that it has the same number of digits as $x$.  The loop on step eleven will subtract multiples of the 
shifted copy of $y$ until $x$ is smaller.  Since the leading digit of $y$ is greater than or equal to $\beta/2$ this loop will iterate at most two
times to produce the desired leading digit of the quotient.  

Now the remainder of the digits can be produced.  The equation $\hat q = \lfloor {{x_i \beta + x_{i-1}}\over y_t} \rfloor$ is used to fairly
accurately approximate the true quotient digit.  The estimation can in theory produce an estimation as high as $2\beta - {2 \over \beta}$ but by
induction the upper quotient digit is correct (\textit{as established on step eleven}) and the estimate must be less than $\beta$.  

Recall from section~\ref{sec:divest} that the estimation is never too low but may be too high.  The next step of the estimation process is
to refine the estimation.  The loop on step 13.5 uses $x_i\beta^2 + x_{i-1}\beta + x_{i-2}$ and $q_{i - t - 1}(y_t\beta + y_{t-1})$ as a higher
order approximation to adjust the quotient digit.

After both phases of estimation the quotient digit may still be off by a value of one\footnote{This is similar to the error introduced
by optimizing Barrett reduction.}.  Steps 13.6 and 13.7 subtract the multiple of the divisor from the dividend (\textit{Similar to step 3.3 of
algorithm~\ref{fig:raddiv}} and then subsequently add a multiple of the divisor if the quotient was too large.  

Now that the quotient has been determine finializing the result is a matter of clamping the quotient, fixing the sizes and de-normalizing the 
remainder.  An important aspect of this algorithm seemingly overlooked in other descriptions such as that of Algorithm 14.20 HAC \cite[pp. 598]{HAC}
is that when the estimations are being made (\textit{inside the loop on step 13.5}) that the digits $y_{t-1}$, $x_{i-2}$ and $x_{i-1}$ may lie 
outside their respective boundaries.  For example, if $t = 0$ or $i \le 1$ then the digits would be undefined.  In those cases the digits should
respectively be replaced with a zero.  

\vspace{+3mm}\begin{small}
\hspace{-5.1mm}{\bf File}: bn\_mp\_div.c
\vspace{-3mm}
\begin{alltt}
\end{alltt}
\end{small}

The implementation of this algorithm differs slightly from the pseudo code presented previously.  In this algorithm either of the quotient $c$ or
remainder $d$ may be passed as a \textbf{NULL} pointer which indicates their value is not desired.  For example, the C code to call the division
algorithm with only the quotient is 

\begin{verbatim}
mp_div(&a, &b, &c, NULL);  /* c = [a/b] */
\end{verbatim}

Lines 109 and 113 handle the two trivial cases of inputs which are division by zero and dividend smaller than the divisor 
respectively.  After the two trivial cases all of the temporary variables are initialized.  Line 148 determines the sign of 
the quotient and line 148 ensures that both $x$ and $y$ are positive.  

The number of bits in the leading digit is calculated on line 151.  Implictly an mp\_int with $r$ digits will require $lg(\beta)(r-1) + k$ bits
of precision which when reduced modulo $lg(\beta)$ produces the value of $k$.  In this case $k$ is the number of bits in the leading digit which is
exactly what is required.  For the algorithm to operate $k$ must equal $lg(\beta) - 1$ and when it does not the inputs must be normalized by shifting
them to the left by $lg(\beta) - 1 - k$ bits.

Throughout the variables $n$ and $t$ will represent the highest digit of $x$ and $y$ respectively.  These are first used to produce the 
leading digit of the quotient.  The loop beginning on line 184 will produce the remainder of the quotient digits.

The conditional ``continue'' on line 187 is used to prevent the algorithm from reading past the leading edge of $x$ which can occur when the
algorithm eliminates multiple non-zero digits in a single iteration.  This ensures that $x_i$ is always non-zero since by definition the digits
above the $i$'th position $x$ must be zero in order for the quotient to be precise\footnote{Precise as far as integer division is concerned.}.  

Lines 214, 216 and 223 through 225 manually construct the high accuracy estimations by setting the digits of the two mp\_int 
variables directly.  

\section{Single Digit Helpers}

This section briefly describes a series of single digit helper algorithms which come in handy when working with small constants.  All of 
the helper functions assume the single digit input is positive and will treat them as such.

\subsection{Single Digit Addition and Subtraction}

Both addition and subtraction are performed by ``cheating'' and using mp\_set followed by the higher level addition or subtraction 
algorithms.   As a result these algorithms are subtantially simpler with a slight cost in performance.

\newpage\begin{figure}[!here]
\begin{small}
\begin{center}
\begin{tabular}{l}
\hline Algorithm \textbf{mp\_add\_d}. \\
\textbf{Input}.   mp\_int $a$ and a mp\_digit $b$ \\
\textbf{Output}.  $c = a + b$ \\
\hline \\
1.  $t \leftarrow b$ (\textit{mp\_set}) \\
2.  $c \leftarrow a + t$ \\
3.  Return(\textit{MP\_OKAY}) \\
\hline
\end{tabular}
\end{center}
\end{small}
\caption{Algorithm mp\_add\_d}
\end{figure}

\textbf{Algorithm mp\_add\_d.}
This algorithm initiates a temporary mp\_int with the value of the single digit and uses algorithm mp\_add to add the two values together.

\vspace{+3mm}\begin{small}
\hspace{-5.1mm}{\bf File}: bn\_mp\_add\_d.c
\vspace{-3mm}
\begin{alltt}
\end{alltt}
\end{small}

Clever use of the letter 't'.

\subsubsection{Subtraction}
The single digit subtraction algorithm mp\_sub\_d is essentially the same except it uses mp\_sub to subtract the digit from the mp\_int.

\subsection{Single Digit Multiplication}
Single digit multiplication arises enough in division and radix conversion that it ought to be implement as a special case of the baseline
multiplication algorithm.  Essentially this algorithm is a modified version of algorithm s\_mp\_mul\_digs where one of the multiplicands
only has one digit.

\begin{figure}[!here]
\begin{small}
\begin{center}
\begin{tabular}{l}
\hline Algorithm \textbf{mp\_mul\_d}. \\
\textbf{Input}.   mp\_int $a$ and a mp\_digit $b$ \\
\textbf{Output}.  $c = ab$ \\
\hline \\
1.  $pa \leftarrow a.used$ \\
2.  Grow $c$ to at least $pa + 1$ digits. \\
3.  $oldused \leftarrow c.used$ \\
4.  $c.used \leftarrow pa + 1$ \\
5.  $c.sign \leftarrow a.sign$ \\
6.  $\mu \leftarrow 0$ \\
7.  for $ix$ from $0$ to $pa - 1$ do \\
\hspace{3mm}7.1  $\hat r \leftarrow \mu + a_{ix}b$ \\
\hspace{3mm}7.2  $c_{ix} \leftarrow \hat r \mbox{ (mod }\beta\mbox{)}$ \\
\hspace{3mm}7.3  $\mu \leftarrow \lfloor \hat r / \beta \rfloor$ \\
8.  $c_{pa} \leftarrow \mu$ \\
9.  for $ix$ from $pa + 1$ to $oldused$ do \\
\hspace{3mm}9.1  $c_{ix} \leftarrow 0$ \\
10.  Clamp excess digits of $c$. \\
11.  Return(\textit{MP\_OKAY}). \\
\hline
\end{tabular}
\end{center}
\end{small}
\caption{Algorithm mp\_mul\_d}
\end{figure}
\textbf{Algorithm mp\_mul\_d.}
This algorithm quickly multiplies an mp\_int by a small single digit value.  It is specially tailored to the job and has a minimal of overhead.  
Unlike the full multiplication algorithms this algorithm does not require any significnat temporary storage or memory allocations.  

\vspace{+3mm}\begin{small}
\hspace{-5.1mm}{\bf File}: bn\_mp\_mul\_d.c
\vspace{-3mm}
\begin{alltt}
\end{alltt}
\end{small}

In this implementation the destination $c$ may point to the same mp\_int as the source $a$ since the result is written after the digit is 
read from the source.  This function uses pointer aliases $tmpa$ and $tmpc$ for the digits of $a$ and $c$ respectively.  

\subsection{Single Digit Division}
Like the single digit multiplication algorithm, single digit division is also a fairly common algorithm used in radix conversion.  Since the
divisor is only a single digit a specialized variant of the division algorithm can be used to compute the quotient.  

\newpage\begin{figure}[!here]
\begin{small}
\begin{center}
\begin{tabular}{l}
\hline Algorithm \textbf{mp\_div\_d}. \\
\textbf{Input}.   mp\_int $a$ and a mp\_digit $b$ \\
\textbf{Output}.  $c = \lfloor a / b \rfloor, d = a - cb$ \\
\hline \\
1.  If $b = 0$ then return(\textit{MP\_VAL}).\\
2.  If $b = 3$ then use algorithm mp\_div\_3 instead. \\
3.  Init $q$ to $a.used$ digits.  \\
4.  $q.used \leftarrow a.used$ \\
5.  $q.sign \leftarrow a.sign$ \\
6.  $\hat w \leftarrow 0$ \\
7.  for $ix$ from $a.used - 1$ down to $0$ do \\
\hspace{3mm}7.1  $\hat w \leftarrow \hat w \beta + a_{ix}$ \\
\hspace{3mm}7.2  If $\hat w \ge b$ then \\
\hspace{6mm}7.2.1  $t \leftarrow \lfloor \hat w / b \rfloor$ \\
\hspace{6mm}7.2.2  $\hat w \leftarrow \hat w \mbox{ (mod }b\mbox{)}$ \\
\hspace{3mm}7.3  else\\
\hspace{6mm}7.3.1  $t \leftarrow 0$ \\
\hspace{3mm}7.4  $q_{ix} \leftarrow t$ \\
8.  $d \leftarrow \hat w$ \\
9.  Clamp excess digits of $q$. \\
10.  $c \leftarrow q$ \\
11.  Return(\textit{MP\_OKAY}). \\
\hline
\end{tabular}
\end{center}
\end{small}
\caption{Algorithm mp\_div\_d}
\end{figure}
\textbf{Algorithm mp\_div\_d.}
This algorithm divides the mp\_int $a$ by the single mp\_digit $b$ using an optimized approach.  Essentially in every iteration of the
algorithm another digit of the dividend is reduced and another digit of quotient produced.  Provided $b < \beta$ the value of $\hat w$
after step 7.1 will be limited such that $0 \le \lfloor \hat w / b \rfloor < \beta$.  

If the divisor $b$ is equal to three a variant of this algorithm is used which is called mp\_div\_3.  It replaces the division by three with
a multiplication by $\lfloor \beta / 3 \rfloor$ and the appropriate shift and residual fixup.  In essence it is much like the Barrett reduction
from chapter seven.  

\vspace{+3mm}\begin{small}
\hspace{-5.1mm}{\bf File}: bn\_mp\_div\_d.c
\vspace{-3mm}
\begin{alltt}
\end{alltt}
\end{small}

Like the implementation of algorithm mp\_div this algorithm allows either of the quotient or remainder to be passed as a \textbf{NULL} pointer to
indicate the respective value is not required.  This allows a trivial single digit modular reduction algorithm, mp\_mod\_d to be created.

The division and remainder on lines 44 and @45,%@ can be replaced often by a single division on most processors.  For example, the 32-bit x86 based 
processors can divide a 64-bit quantity by a 32-bit quantity and produce the quotient and remainder simultaneously.  Unfortunately the GCC 
compiler does not recognize that optimization and will actually produce two function calls to find the quotient and remainder respectively.  

\subsection{Single Digit Root Extraction}

Finding the $n$'th root of an integer is fairly easy as far as numerical analysis is concerned.  Algorithms such as the Newton-Raphson approximation 
(\ref{eqn:newton}) series will converge very quickly to a root for any continuous function $f(x)$.  

\begin{equation}
x_{i+1} = x_i - {f(x_i) \over f'(x_i)}
\label{eqn:newton}
\end{equation}

In this case the $n$'th root is desired and $f(x) = x^n - a$ where $a$ is the integer of which the root is desired.  The derivative of $f(x)$ is 
simply $f'(x) = nx^{n - 1}$.  Of particular importance is that this algorithm will be used over the integers not over the a more continuous domain
such as the real numbers.  As a result the root found can be above the true root by few and must be manually adjusted.  Ideally at the end of the 
algorithm the $n$'th root $b$ of an integer $a$ is desired such that $b^n \le a$.  

\newpage\begin{figure}[!here]
\begin{small}
\begin{center}
\begin{tabular}{l}
\hline Algorithm \textbf{mp\_n\_root}. \\
\textbf{Input}.   mp\_int $a$ and a mp\_digit $b$ \\
\textbf{Output}.  $c^b \le a$ \\
\hline \\
1.  If $b$ is even and $a.sign = MP\_NEG$ return(\textit{MP\_VAL}). \\
2.  $sign \leftarrow a.sign$ \\
3.  $a.sign \leftarrow MP\_ZPOS$ \\
4.  t$2 \leftarrow 2$ \\
5.  Loop \\
\hspace{3mm}5.1  t$1 \leftarrow $ t$2$ \\
\hspace{3mm}5.2  t$3 \leftarrow $ t$1^{b - 1}$ \\
\hspace{3mm}5.3  t$2 \leftarrow $ t$3 $ $\cdot$ t$1$ \\
\hspace{3mm}5.4  t$2 \leftarrow $ t$2 - a$ \\
\hspace{3mm}5.5  t$3 \leftarrow $ t$3 \cdot b$ \\
\hspace{3mm}5.6  t$3 \leftarrow \lfloor $t$2 / $t$3 \rfloor$ \\
\hspace{3mm}5.7  t$2 \leftarrow $ t$1 - $ t$3$ \\
\hspace{3mm}5.8  If t$1 \ne $ t$2$ then goto step 5.  \\
6.  Loop \\
\hspace{3mm}6.1  t$2 \leftarrow $ t$1^b$ \\
\hspace{3mm}6.2  If t$2 > a$ then \\
\hspace{6mm}6.2.1  t$1 \leftarrow $ t$1 - 1$ \\
\hspace{6mm}6.2.2  Goto step 6. \\
7.  $a.sign \leftarrow sign$ \\
8.  $c \leftarrow $ t$1$ \\
9.  $c.sign \leftarrow sign$  \\
10.  Return(\textit{MP\_OKAY}).  \\
\hline
\end{tabular}
\end{center}
\end{small}
\caption{Algorithm mp\_n\_root}
\end{figure}
\textbf{Algorithm mp\_n\_root.}
This algorithm finds the integer $n$'th root of an input using the Newton-Raphson approach.  It is partially optimized based on the observation
that the numerator of ${f(x) \over f'(x)}$ can be derived from a partial denominator.  That is at first the denominator is calculated by finding
$x^{b - 1}$.  This value can then be multiplied by $x$ and have $a$ subtracted from it to find the numerator.  This saves a total of $b - 1$ 
multiplications by t$1$ inside the loop.  

The initial value of the approximation is t$2 = 2$ which allows the algorithm to start with very small values and quickly converge on the
root.  Ideally this algorithm is meant to find the $n$'th root of an input where $n$ is bounded by $2 \le n \le 5$.  

\vspace{+3mm}\begin{small}
\hspace{-5.1mm}{\bf File}: bn\_mp\_n\_root.c
\vspace{-3mm}
\begin{alltt}
\end{alltt}
\end{small}

\section{Random Number Generation}

Random numbers come up in a variety of activities from public key cryptography to simple simulations and various randomized algorithms.  Pollard-Rho 
factoring for example, can make use of random values as starting points to find factors of a composite integer.  In this case the algorithm presented
is solely for simulations and not intended for cryptographic use.  

\newpage\begin{figure}[!here]
\begin{small}
\begin{center}
\begin{tabular}{l}
\hline Algorithm \textbf{mp\_rand}. \\
\textbf{Input}.   An integer $b$ \\
\textbf{Output}.  A pseudo-random number of $b$ digits \\
\hline \\
1.  $a \leftarrow 0$ \\
2.  If $b \le 0$ return(\textit{MP\_OKAY}) \\
3.  Pick a non-zero random digit $d$. \\
4.  $a \leftarrow a + d$ \\
5.  for $ix$ from 1 to $d - 1$ do \\
\hspace{3mm}5.1  $a \leftarrow a \cdot \beta$ \\
\hspace{3mm}5.2  Pick a random digit $d$. \\
\hspace{3mm}5.3  $a \leftarrow a + d$ \\
6.  Return(\textit{MP\_OKAY}). \\
\hline
\end{tabular}
\end{center}
\end{small}
\caption{Algorithm mp\_rand}
\end{figure}
\textbf{Algorithm mp\_rand.}
This algorithm produces a pseudo-random integer of $b$ digits.  By ensuring that the first digit is non-zero the algorithm also guarantees that the
final result has at least $b$ digits.  It relies heavily on a third-part random number generator which should ideally generate uniformly all of
the integers from $0$ to $\beta - 1$.  

\vspace{+3mm}\begin{small}
\hspace{-5.1mm}{\bf File}: bn\_mp\_rand.c
\vspace{-3mm}
\begin{alltt}
\end{alltt}
\end{small}

\section{Formatted Representations}
The ability to emit a radix-$n$ textual representation of an integer is useful for interacting with human parties.  For example, the ability to
be given a string of characters such as ``114585'' and turn it into the radix-$\beta$ equivalent would make it easier to enter numbers
into a program.

\subsection{Reading Radix-n Input}
For the purposes of this text we will assume that a simple lower ASCII map (\ref{fig:ASC}) is used for the values of from $0$ to $63$ to 
printable characters.  For example, when the character ``N'' is read it represents the integer $23$.  The first $16$ characters of the
map are for the common representations up to hexadecimal.  After that they match the ``base64'' encoding scheme which are suitable chosen
such that they are printable.  While outputting as base64 may not be too helpful for human operators it does allow communication via non binary
mediums.

\newpage\begin{figure}[here]
\begin{center}
\begin{tabular}{cc|cc|cc|cc}
\hline \textbf{Value} & \textbf{Char} & \textbf{Value} & \textbf{Char} & \textbf{Value} & \textbf{Char} &  \textbf{Value} & \textbf{Char} \\
\hline 
0 & 0 & 1 & 1 & 2 & 2 & 3 & 3 \\
4 & 4 & 5 & 5 & 6 & 6 & 7 & 7 \\
8 & 8 & 9 & 9 & 10 & A & 11 & B \\
12 & C & 13 & D & 14 & E & 15 & F \\
16 & G & 17 & H & 18 & I & 19 & J \\
20 & K & 21 & L & 22 & M & 23 & N \\
24 & O & 25 & P & 26 & Q & 27 & R \\
28 & S & 29 & T & 30 & U & 31 & V \\
32 & W & 33 & X & 34 & Y & 35 & Z \\
36 & a & 37 & b & 38 & c & 39 & d \\
40 & e & 41 & f & 42 & g & 43 & h \\
44 & i & 45 & j & 46 & k & 47 & l \\
48 & m & 49 & n & 50 & o & 51 & p \\
52 & q & 53 & r & 54 & s & 55 & t \\
56 & u & 57 & v & 58 & w & 59 & x \\
60 & y & 61 & z & 62 & $+$ & 63 & $/$ \\
\hline
\end{tabular}
\end{center}
\caption{Lower ASCII Map}
\label{fig:ASC}
\end{figure}

\newpage\begin{figure}[!here]
\begin{small}
\begin{center}
\begin{tabular}{l}
\hline Algorithm \textbf{mp\_read\_radix}. \\
\textbf{Input}.   A string $str$ of length $sn$ and radix $r$. \\
\textbf{Output}.  The radix-$\beta$ equivalent mp\_int. \\
\hline \\
1.  If $r < 2$ or $r > 64$ return(\textit{MP\_VAL}). \\
2.  $ix \leftarrow 0$ \\
3.  If $str_0 =$ ``-'' then do \\
\hspace{3mm}3.1  $ix \leftarrow ix + 1$ \\
\hspace{3mm}3.2  $sign \leftarrow MP\_NEG$ \\
4.  else \\
\hspace{3mm}4.1  $sign \leftarrow MP\_ZPOS$ \\
5.  $a \leftarrow 0$ \\
6.  for $iy$ from $ix$ to $sn - 1$ do \\
\hspace{3mm}6.1  Let $y$ denote the position in the map of $str_{iy}$. \\
\hspace{3mm}6.2  If $str_{iy}$ is not in the map or $y \ge r$ then goto step 7. \\
\hspace{3mm}6.3  $a \leftarrow a \cdot r$ \\
\hspace{3mm}6.4  $a \leftarrow a + y$ \\
7.  If $a \ne 0$ then $a.sign \leftarrow sign$ \\
8.  Return(\textit{MP\_OKAY}). \\
\hline
\end{tabular}
\end{center}
\end{small}
\caption{Algorithm mp\_read\_radix}
\end{figure}
\textbf{Algorithm mp\_read\_radix.}
This algorithm will read an ASCII string and produce the radix-$\beta$ mp\_int representation of the same integer.  A minus symbol ``-'' may precede the 
string  to indicate the value is negative, otherwise it is assumed to be positive.  The algorithm will read up to $sn$ characters from the input
and will stop when it reads a character it cannot map the algorithm stops reading characters from the string.  This allows numbers to be embedded
as part of larger input without any significant problem.

\vspace{+3mm}\begin{small}
\hspace{-5.1mm}{\bf File}: bn\_mp\_read\_radix.c
\vspace{-3mm}
\begin{alltt}
\end{alltt}
\end{small}

\subsection{Generating Radix-$n$ Output}
Generating radix-$n$ output is fairly trivial with a division and remainder algorithm.  

\newpage\begin{figure}[!here]
\begin{small}
\begin{center}
\begin{tabular}{l}
\hline Algorithm \textbf{mp\_toradix}. \\
\textbf{Input}.   A mp\_int $a$ and an integer $r$\\
\textbf{Output}.  The radix-$r$ representation of $a$ \\
\hline \\
1.  If $r < 2$ or $r > 64$ return(\textit{MP\_VAL}). \\
2.  If $a = 0$ then $str = $ ``$0$'' and return(\textit{MP\_OKAY}).  \\
3.  $t \leftarrow a$ \\
4.  $str \leftarrow$ ``'' \\
5.  if $t.sign = MP\_NEG$ then \\
\hspace{3mm}5.1  $str \leftarrow str + $ ``-'' \\
\hspace{3mm}5.2  $t.sign = MP\_ZPOS$ \\
6.  While ($t \ne 0$) do \\
\hspace{3mm}6.1  $d \leftarrow t \mbox{ (mod }r\mbox{)}$ \\
\hspace{3mm}6.2  $t \leftarrow \lfloor t / r \rfloor$ \\
\hspace{3mm}6.3  Look up $d$ in the map and store the equivalent character in $y$. \\
\hspace{3mm}6.4  $str \leftarrow str + y$ \\
7.  If $str_0 = $``$-$'' then \\
\hspace{3mm}7.1  Reverse the digits $str_1, str_2, \ldots str_n$. \\
8.  Otherwise \\
\hspace{3mm}8.1  Reverse the digits $str_0, str_1, \ldots str_n$. \\
9.  Return(\textit{MP\_OKAY}).\\
\hline
\end{tabular}
\end{center}
\end{small}
\caption{Algorithm mp\_toradix}
\end{figure}
\textbf{Algorithm mp\_toradix.}
This algorithm computes the radix-$r$ representation of an mp\_int $a$.  The ``digits'' of the representation are extracted by reducing 
successive powers of $\lfloor a / r^k \rfloor$ the input modulo $r$ until $r^k > a$.  Note that instead of actually dividing by $r^k$ in
each iteration the quotient $\lfloor a / r \rfloor$ is saved for the next iteration.  As a result a series of trivial $n \times 1$ divisions
are required instead of a series of $n \times k$ divisions.  One design flaw of this approach is that the digits are produced in the reverse order 
(see~\ref{fig:mpradix}).  To remedy this flaw the digits must be swapped or simply ``reversed''.

\begin{figure}
\begin{center}
\begin{tabular}{|c|c|c|}
\hline \textbf{Value of $a$} & \textbf{Value of $d$} & \textbf{Value of $str$} \\
\hline $1234$ & -- & -- \\
\hline $123$  & $4$ & ``4'' \\
\hline $12$   & $3$ & ``43'' \\
\hline $1$    & $2$ & ``432'' \\
\hline $0$    & $1$ & ``4321'' \\
\hline
\end{tabular}
\end{center}
\caption{Example of Algorithm mp\_toradix.}
\label{fig:mpradix}
\end{figure}

\vspace{+3mm}\begin{small}
\hspace{-5.1mm}{\bf File}: bn\_mp\_toradix.c
\vspace{-3mm}
\begin{alltt}
\end{alltt}
\end{small}

\chapter{Number Theoretic Algorithms}
This chapter discusses several fundamental number theoretic algorithms such as the greatest common divisor, least common multiple and Jacobi 
symbol computation.  These algorithms arise as essential components in several key cryptographic algorithms such as the RSA public key algorithm and
various Sieve based factoring algorithms.

\section{Greatest Common Divisor}
The greatest common divisor of two integers $a$ and $b$, often denoted as $(a, b)$ is the largest integer $k$ that is a proper divisor of
both $a$ and $b$.  That is, $k$ is the largest integer such that $0 \equiv a \mbox{ (mod }k\mbox{)}$ and $0 \equiv b \mbox{ (mod }k\mbox{)}$ occur
simultaneously.

The most common approach (cite) is to reduce one input modulo another.  That is if $a$ and $b$ are divisible by some integer $k$ and if $qa + r = b$ then
$r$ is also divisible by $k$.  The reduction pattern follows $\left < a , b \right > \rightarrow \left < b, a \mbox{ mod } b \right >$.  

\newpage\begin{figure}[!here]
\begin{small}
\begin{center}
\begin{tabular}{l}
\hline Algorithm \textbf{Greatest Common Divisor (I)}. \\
\textbf{Input}.   Two positive integers $a$ and $b$ greater than zero. \\
\textbf{Output}.  The greatest common divisor $(a, b)$.  \\
\hline \\
1.  While ($b > 0$) do \\
\hspace{3mm}1.1  $r \leftarrow a \mbox{ (mod }b\mbox{)}$ \\
\hspace{3mm}1.2  $a \leftarrow b$ \\
\hspace{3mm}1.3  $b \leftarrow r$ \\
2.  Return($a$). \\
\hline
\end{tabular}
\end{center}
\end{small}
\caption{Algorithm Greatest Common Divisor (I)}
\label{fig:gcd1}
\end{figure}

This algorithm will quickly converge on the greatest common divisor since the residue $r$ tends diminish rapidly.  However, divisions are
relatively expensive operations to perform and should ideally be avoided.  There is another approach based on a similar relationship of 
greatest common divisors.  The faster approach is based on the observation that if $k$ divides both $a$ and $b$ it will also divide $a - b$.  
In particular, we would like $a - b$ to decrease in magnitude which implies that $b \ge a$.  

\begin{figure}[!here]
\begin{small}
\begin{center}
\begin{tabular}{l}
\hline Algorithm \textbf{Greatest Common Divisor (II)}. \\
\textbf{Input}.   Two positive integers $a$ and $b$ greater than zero. \\
\textbf{Output}.  The greatest common divisor $(a, b)$.  \\
\hline \\
1.  While ($b > 0$) do \\
\hspace{3mm}1.1  Swap $a$ and $b$ such that $a$ is the smallest of the two. \\
\hspace{3mm}1.2  $b \leftarrow b - a$ \\
2.  Return($a$). \\
\hline
\end{tabular}
\end{center}
\end{small}
\caption{Algorithm Greatest Common Divisor (II)}
\label{fig:gcd2}
\end{figure}

\textbf{Proof} \textit{Algorithm~\ref{fig:gcd2} will return the greatest common divisor of $a$ and $b$.}
The algorithm in figure~\ref{fig:gcd2} will eventually terminate since $b \ge a$ the subtraction in step 1.2 will be a value less than $b$.  In other
words in every iteration that tuple $\left < a, b \right >$ decrease in magnitude until eventually $a = b$.  Since both $a$ and $b$ are always 
divisible by the greatest common divisor (\textit{until the last iteration}) and in the last iteration of the algorithm $b = 0$, therefore, in the 
second to last iteration of the algorithm $b = a$ and clearly $(a, a) = a$ which concludes the proof.  \textbf{QED}.

As a matter of practicality algorithm \ref{fig:gcd1} decreases far too slowly to be useful.  Specially if $b$ is much larger than $a$ such that 
$b - a$ is still very much larger than $a$.  A simple addition to the algorithm is to divide $b - a$ by a power of some integer $p$ which does
not divide the greatest common divisor but will divide $b - a$.  In this case ${b - a} \over p$ is also an integer and still divisible by
the greatest common divisor.

However, instead of factoring $b - a$ to find a suitable value of $p$ the powers of $p$ can be removed from $a$ and $b$ that are in common first.  
Then inside the loop whenever $b - a$ is divisible by some power of $p$ it can be safely removed.  

\begin{figure}[!here]
\begin{small}
\begin{center}
\begin{tabular}{l}
\hline Algorithm \textbf{Greatest Common Divisor (III)}. \\
\textbf{Input}.   Two positive integers $a$ and $b$ greater than zero. \\
\textbf{Output}.  The greatest common divisor $(a, b)$.  \\
\hline \\
1.  $k \leftarrow 0$ \\
2.  While $a$ and $b$ are both divisible by $p$ do \\
\hspace{3mm}2.1  $a \leftarrow \lfloor a / p \rfloor$ \\
\hspace{3mm}2.2  $b \leftarrow \lfloor b / p \rfloor$ \\
\hspace{3mm}2.3  $k \leftarrow k + 1$ \\
3.  While $a$ is divisible by $p$ do \\
\hspace{3mm}3.1  $a \leftarrow \lfloor a / p \rfloor$ \\
4.  While $b$ is divisible by $p$ do \\
\hspace{3mm}4.1  $b \leftarrow \lfloor b / p \rfloor$ \\
5.  While ($b > 0$) do \\
\hspace{3mm}5.1  Swap $a$ and $b$ such that $a$ is the smallest of the two. \\
\hspace{3mm}5.2  $b \leftarrow b - a$ \\
\hspace{3mm}5.3  While $b$ is divisible by $p$ do \\
\hspace{6mm}5.3.1  $b \leftarrow \lfloor b / p \rfloor$ \\
6.  Return($a \cdot p^k$). \\
\hline
\end{tabular}
\end{center}
\end{small}
\caption{Algorithm Greatest Common Divisor (III)}
\label{fig:gcd3}
\end{figure}

This algorithm is based on the first except it removes powers of $p$ first and inside the main loop to ensure the tuple $\left < a, b \right >$ 
decreases more rapidly.  The first loop on step two removes powers of $p$ that are in common.  A count, $k$, is kept which will present a common
divisor of $p^k$.  After step two the remaining common divisor of $a$ and $b$ cannot be divisible by $p$.  This means that $p$ can be safely 
divided out of the difference $b - a$ so long as the division leaves no remainder.  

In particular the value of $p$ should be chosen such that the division on step 5.3.1 occur often.  It also helps that division by $p$ be easy
to compute.  The ideal choice of $p$ is two since division by two amounts to a right logical shift.  Another important observation is that by
step five both $a$ and $b$ are odd.  Therefore, the diffrence $b - a$ must be even which means that each iteration removes one bit from the 
largest of the pair.

\subsection{Complete Greatest Common Divisor}
The algorithms presented so far cannot handle inputs which are zero or negative.  The following algorithm can handle all input cases properly
and will produce the greatest common divisor.

\newpage\begin{figure}[!here]
\begin{small}
\begin{center}
\begin{tabular}{l}
\hline Algorithm \textbf{mp\_gcd}. \\
\textbf{Input}.   mp\_int $a$ and $b$ \\
\textbf{Output}.  The greatest common divisor $c = (a, b)$.  \\
\hline \\
1.  If $a = 0$ then \\
\hspace{3mm}1.1  $c \leftarrow \vert b \vert $ \\
\hspace{3mm}1.2  Return(\textit{MP\_OKAY}). \\
2.  If $b = 0$ then \\
\hspace{3mm}2.1  $c \leftarrow \vert a \vert $ \\
\hspace{3mm}2.2  Return(\textit{MP\_OKAY}). \\
3.  $u \leftarrow \vert a \vert, v \leftarrow \vert b \vert$ \\
4.  $k \leftarrow 0$ \\
5.  While $u.used > 0$ and $v.used > 0$ and $u_0 \equiv v_0 \equiv 0 \mbox{ (mod }2\mbox{)}$ \\
\hspace{3mm}5.1  $k \leftarrow k + 1$ \\
\hspace{3mm}5.2  $u \leftarrow \lfloor u / 2 \rfloor$ \\
\hspace{3mm}5.3  $v \leftarrow \lfloor v / 2 \rfloor$ \\
6.  While $u.used > 0$ and $u_0 \equiv 0 \mbox{ (mod }2\mbox{)}$ \\
\hspace{3mm}6.1  $u \leftarrow \lfloor u / 2 \rfloor$ \\
7.  While $v.used > 0$ and $v_0 \equiv 0 \mbox{ (mod }2\mbox{)}$ \\
\hspace{3mm}7.1  $v \leftarrow \lfloor v / 2 \rfloor$ \\
8.  While $v.used > 0$ \\
\hspace{3mm}8.1  If $\vert u \vert > \vert v \vert$ then \\
\hspace{6mm}8.1.1  Swap $u$ and $v$. \\
\hspace{3mm}8.2  $v \leftarrow \vert v \vert - \vert u \vert$ \\
\hspace{3mm}8.3  While $v.used > 0$ and $v_0 \equiv 0 \mbox{ (mod }2\mbox{)}$ \\
\hspace{6mm}8.3.1  $v \leftarrow \lfloor v / 2 \rfloor$ \\
9.  $c \leftarrow u \cdot 2^k$ \\
10.  Return(\textit{MP\_OKAY}). \\
\hline
\end{tabular}
\end{center}
\end{small}
\caption{Algorithm mp\_gcd}
\end{figure}
\textbf{Algorithm mp\_gcd.}
This algorithm will produce the greatest common divisor of two mp\_ints $a$ and $b$.  The algorithm was originally based on Algorithm B of
Knuth \cite[pp. 338]{TAOCPV2} but has been modified to be simpler to explain.  In theory it achieves the same asymptotic working time as
Algorithm B and in practice this appears to be true.  

The first two steps handle the cases where either one of or both inputs are zero.  If either input is zero the greatest common divisor is the 
largest input or zero if they are both zero.  If the inputs are not trivial than $u$ and $v$ are assigned the absolute values of 
$a$ and $b$ respectively and the algorithm will proceed to reduce the pair.

Step five will divide out any common factors of two and keep track of the count in the variable $k$.  After this step, two is no longer a
factor of the remaining greatest common divisor between $u$ and $v$ and can be safely evenly divided out of either whenever they are even.  Step 
six and seven ensure that the $u$ and $v$ respectively have no more factors of two.  At most only one of the while--loops will iterate since 
they cannot both be even.

By step eight both of $u$ and $v$ are odd which is required for the inner logic.  First the pair are swapped such that $v$ is equal to
or greater than $u$.  This ensures that the subtraction on step 8.2 will always produce a positive and even result.  Step 8.3 removes any
factors of two from the difference $u$ to ensure that in the next iteration of the loop both are once again odd.

After $v = 0$ occurs the variable $u$ has the greatest common divisor of the pair $\left < u, v \right >$ just after step six.  The result
must be adjusted by multiplying by the common factors of two ($2^k$) removed earlier.  

\vspace{+3mm}\begin{small}
\hspace{-5.1mm}{\bf File}: bn\_mp\_gcd.c
\vspace{-3mm}
\begin{alltt}
\end{alltt}
\end{small}

This function makes use of the macros mp\_iszero and mp\_iseven.  The former evaluates to $1$ if the input mp\_int is equivalent to the 
integer zero otherwise it evaluates to $0$.  The latter evaluates to $1$ if the input mp\_int represents a non-zero even integer otherwise
it evaluates to $0$.  Note that just because mp\_iseven may evaluate to $0$ does not mean the input is odd, it could also be zero.  The three 
trivial cases of inputs are handled on lines 24 through 30.  After those lines the inputs are assumed to be non-zero.

Lines 32 and 37 make local copies $u$ and $v$ of the inputs $a$ and $b$ respectively.  At this point the common factors of two 
must be divided out of the two inputs.  The block starting at line 44 removes common factors of two by first counting the number of trailing
zero bits in both.  The local integer $k$ is used to keep track of how many factors of $2$ are pulled out of both values.  It is assumed that 
the number of factors will not exceed the maximum value of a C ``int'' data type\footnote{Strictly speaking no array in C may have more than 
entries than are accessible by an ``int'' so this is not a limitation.}.  

At this point there are no more common factors of two in the two values.  The divisions by a power of two on lines 62 and 68 remove 
any independent factors of two such that both $u$ and $v$ are guaranteed to be an odd integer before hitting the main body of the algorithm.  The while loop
on line 73 performs the reduction of the pair until $v$ is equal to zero.  The unsigned comparison and subtraction algorithms are used in
place of the full signed routines since both values are guaranteed to be positive and the result of the subtraction is guaranteed to be non-negative.

\section{Least Common Multiple}
The least common multiple of a pair of integers is their product divided by their greatest common divisor.  For two integers $a$ and $b$ the
least common multiple is normally denoted as $[ a, b ]$ and numerically equivalent to ${ab} \over {(a, b)}$.  For example, if $a = 2 \cdot 2 \cdot 3 = 12$
and $b = 2 \cdot 3 \cdot 3 \cdot 7 = 126$ the least common multiple is ${126 \over {(12, 126)}} = {126 \over 6} = 21$.

The least common multiple arises often in coding theory as well as number theory.  If two functions have periods of $a$ and $b$ respectively they will
collide, that is be in synchronous states, after only $[ a, b ]$ iterations.  This is why, for example, random number generators based on 
Linear Feedback Shift Registers (LFSR) tend to use registers with periods which are co-prime (\textit{e.g. the greatest common divisor is one.}).  
Similarly in number theory if a composite $n$ has two prime factors $p$ and $q$ then maximal order of any unit of $\Z/n\Z$ will be $[ p - 1, q - 1] $.

\begin{figure}[!here]
\begin{small}
\begin{center}
\begin{tabular}{l}
\hline Algorithm \textbf{mp\_lcm}. \\
\textbf{Input}.   mp\_int $a$ and $b$ \\
\textbf{Output}.  The least common multiple $c = [a, b]$.  \\
\hline \\
1.  $c \leftarrow (a, b)$ \\
2.  $t \leftarrow a \cdot b$ \\
3.  $c \leftarrow \lfloor t / c \rfloor$ \\
4.  Return(\textit{MP\_OKAY}). \\
\hline
\end{tabular}
\end{center}
\end{small}
\caption{Algorithm mp\_lcm}
\end{figure}
\textbf{Algorithm mp\_lcm.}
This algorithm computes the least common multiple of two mp\_int inputs $a$ and $b$.  It computes the least common multiple directly by
dividing the product of the two inputs by their greatest common divisor.

\vspace{+3mm}\begin{small}
\hspace{-5.1mm}{\bf File}: bn\_mp\_lcm.c
\vspace{-3mm}
\begin{alltt}
\end{alltt}
\end{small}

\section{Jacobi Symbol Computation}
To explain the Jacobi Symbol we shall first discuss the Legendre function\footnote{Arrg.  What is the name of this?} off which the Jacobi symbol is 
defined.  The Legendre function computes whether or not an integer $a$ is a quadratic residue modulo an odd prime $p$.  Numerically it is
equivalent to equation \ref{eqn:legendre}.

\textit{-- Tom, don't be an ass, cite your source here...!}

\begin{equation}
a^{(p-1)/2} \equiv \begin{array}{rl}
                              -1 &  \mbox{if }a\mbox{ is a quadratic non-residue.} \\
                              0  &  \mbox{if }a\mbox{ divides }p\mbox{.} \\
                              1  &  \mbox{if }a\mbox{ is a quadratic residue}. 
                              \end{array} \mbox{ (mod }p\mbox{)}
\label{eqn:legendre}                              
\end{equation}

\textbf{Proof.} \textit{Equation \ref{eqn:legendre} correctly identifies the residue status of an integer $a$ modulo a prime $p$.}
An integer $a$ is a quadratic residue if the following equation has a solution.

\begin{equation}
x^2 \equiv a \mbox{ (mod }p\mbox{)}
\label{eqn:root}
\end{equation}

Consider the following equation.

\begin{equation}
0 \equiv x^{p-1} - 1 \equiv \left \lbrace \left (x^2 \right )^{(p-1)/2} - a^{(p-1)/2} \right \rbrace + \left ( a^{(p-1)/2} - 1 \right ) \mbox{ (mod }p\mbox{)}
\label{eqn:rooti}
\end{equation}

Whether equation \ref{eqn:root} has a solution or not equation \ref{eqn:rooti} is always true.  If $a^{(p-1)/2} - 1 \equiv 0 \mbox{ (mod }p\mbox{)}$
then the quantity in the braces must be zero.  By reduction,

\begin{eqnarray}
\left (x^2 \right )^{(p-1)/2} - a^{(p-1)/2} \equiv 0  \nonumber \\
\left (x^2 \right )^{(p-1)/2} \equiv a^{(p-1)/2} \nonumber \\
x^2 \equiv a \mbox{ (mod }p\mbox{)} 
\end{eqnarray}

As a result there must be a solution to the quadratic equation and in turn $a$ must be a quadratic residue.  If $a$ does not divide $p$ and $a$
is not a quadratic residue then the only other value $a^{(p-1)/2}$ may be congruent to is $-1$ since
\begin{equation}
0 \equiv a^{p - 1} - 1 \equiv (a^{(p-1)/2} + 1)(a^{(p-1)/2} - 1) \mbox{ (mod }p\mbox{)}
\end{equation}
One of the terms on the right hand side must be zero.  \textbf{QED}

\subsection{Jacobi Symbol}
The Jacobi symbol is a generalization of the Legendre function for any odd non prime moduli $p$ greater than 2.  If $p = \prod_{i=0}^n p_i$ then
the Jacobi symbol $\left ( { a \over p } \right )$ is equal to the following equation.

\begin{equation}
\left ( { a \over p } \right ) = \left ( { a \over p_0} \right ) \left ( { a \over p_1} \right ) \ldots \left ( { a \over p_n} \right )
\end{equation}

By inspection if $p$ is prime the Jacobi symbol is equivalent to the Legendre function.  The following facts\footnote{See HAC \cite[pp. 72-74]{HAC} for
further details.} will be used to derive an efficient Jacobi symbol algorithm.  Where $p$ is an odd integer greater than two and $a, b \in \Z$ the
following are true.  

\begin{enumerate}
\item $\left ( { a \over p} \right )$ equals $-1$, $0$ or $1$. 
\item $\left ( { ab \over p} \right ) = \left ( { a \over p} \right )\left ( { b \over p} \right )$.
\item If $a \equiv b$ then $\left ( { a \over p} \right ) = \left ( { b \over p} \right )$.
\item $\left ( { 2 \over p} \right )$ equals $1$ if $p \equiv 1$ or $7 \mbox{ (mod }8\mbox{)}$.  Otherwise, it equals $-1$.
\item $\left ( { a \over p} \right ) \equiv \left ( { p \over a} \right ) \cdot (-1)^{(p-1)(a-1)/4}$.  More specifically 
$\left ( { a \over p} \right ) = \left ( { p \over a} \right )$ if $p \equiv a \equiv 1 \mbox{ (mod }4\mbox{)}$.  
\end{enumerate}

Using these facts if $a = 2^k \cdot a'$ then

\begin{eqnarray}
\left ( { a \over p } \right ) = \left ( {{2^k} \over p } \right ) \left ( {a' \over p} \right ) \nonumber \\
                               = \left ( {2 \over p } \right )^k \left ( {a' \over p} \right ) 
\label{eqn:jacobi}
\end{eqnarray}

By fact five, 

\begin{equation}
\left ( { a \over p } \right ) = \left ( { p \over a } \right ) \cdot (-1)^{(p-1)(a-1)/4} 
\end{equation}

Subsequently by fact three since $p \equiv (p \mbox{ mod }a) \mbox{ (mod }a\mbox{)}$ then 

\begin{equation}
\left ( { a \over p } \right ) = \left ( { {p \mbox{ mod } a} \over a } \right ) \cdot (-1)^{(p-1)(a-1)/4} 
\end{equation}

By putting both observations into equation \ref{eqn:jacobi} the following simplified equation is formed.

\begin{equation}
\left ( { a \over p } \right ) = \left ( {2 \over p } \right )^k \left ( {{p\mbox{ mod }a'} \over a'} \right )  \cdot (-1)^{(p-1)(a'-1)/4} 
\end{equation}

The value of $\left ( {{p \mbox{ mod }a'} \over a'} \right )$ can be found by using the same equation recursively.  The value of 
$\left ( {2 \over p } \right )^k$ equals $1$ if $k$ is even otherwise it equals $\left ( {2 \over p } \right )$.  Using this approach the 
factors of $p$ do not have to be known.  Furthermore, if $(a, p) = 1$ then the algorithm will terminate when the recursion requests the 
Jacobi symbol computation of $\left ( {1 \over a'} \right )$ which is simply $1$.  

\newpage\begin{figure}[!here]
\begin{small}
\begin{center}
\begin{tabular}{l}
\hline Algorithm \textbf{mp\_jacobi}. \\
\textbf{Input}.   mp\_int $a$ and $p$, $a \ge 0$, $p \ge 3$, $p \equiv 1 \mbox{ (mod }2\mbox{)}$ \\
\textbf{Output}.  The Jacobi symbol $c = \left ( {a \over p } \right )$. \\
\hline \\
1.  If $a = 0$ then \\
\hspace{3mm}1.1  $c \leftarrow 0$ \\
\hspace{3mm}1.2  Return(\textit{MP\_OKAY}). \\
2.  If $a = 1$ then \\
\hspace{3mm}2.1  $c \leftarrow 1$ \\
\hspace{3mm}2.2  Return(\textit{MP\_OKAY}). \\
3.  $a' \leftarrow a$ \\
4.  $k \leftarrow 0$ \\
5.  While $a'.used > 0$ and $a'_0 \equiv 0 \mbox{ (mod }2\mbox{)}$ \\
\hspace{3mm}5.1  $k \leftarrow k + 1$ \\
\hspace{3mm}5.2  $a' \leftarrow \lfloor a' / 2 \rfloor$ \\
6.  If $k \equiv 0 \mbox{ (mod }2\mbox{)}$ then \\
\hspace{3mm}6.1  $s \leftarrow 1$ \\
7.  else \\
\hspace{3mm}7.1  $r \leftarrow p_0 \mbox{ (mod }8\mbox{)}$ \\
\hspace{3mm}7.2  If $r = 1$ or $r = 7$ then \\
\hspace{6mm}7.2.1  $s \leftarrow 1$ \\
\hspace{3mm}7.3  else \\
\hspace{6mm}7.3.1  $s \leftarrow -1$ \\
8.  If $p_0 \equiv a'_0 \equiv 3 \mbox{ (mod }4\mbox{)}$ then \\
\hspace{3mm}8.1  $s \leftarrow -s$ \\
9.  If $a' \ne 1$ then \\
\hspace{3mm}9.1  $p' \leftarrow p \mbox{ (mod }a'\mbox{)}$ \\
\hspace{3mm}9.2  $s \leftarrow s \cdot \mbox{mp\_jacobi}(p', a')$ \\
10.  $c \leftarrow s$ \\
11.  Return(\textit{MP\_OKAY}). \\
\hline
\end{tabular}
\end{center}
\end{small}
\caption{Algorithm mp\_jacobi}
\end{figure}
\textbf{Algorithm mp\_jacobi.}
This algorithm computes the Jacobi symbol for an arbitrary positive integer $a$ with respect to an odd integer $p$ greater than three.  The algorithm
is based on algorithm 2.149 of HAC \cite[pp. 73]{HAC}.  

Step numbers one and two handle the trivial cases of $a = 0$ and $a = 1$ respectively.  Step five determines the number of two factors in the
input $a$.  If $k$ is even than the term $\left ( { 2 \over p } \right )^k$ must always evaluate to one.  If $k$ is odd than the term evaluates to one 
if $p_0$ is congruent to one or seven modulo eight, otherwise it evaluates to $-1$. After the the $\left ( { 2 \over p } \right )^k$ term is handled 
the $(-1)^{(p-1)(a'-1)/4}$ is computed and multiplied against the current product $s$.  The latter term evaluates to one if both $p$ and $a'$ 
are congruent to one modulo four, otherwise it evaluates to negative one.

By step nine if $a'$ does not equal one a recursion is required.  Step 9.1 computes $p' \equiv p \mbox{ (mod }a'\mbox{)}$ and will recurse to compute
$\left ( {p' \over a'} \right )$ which is multiplied against the current Jacobi product.

\vspace{+3mm}\begin{small}
\hspace{-5.1mm}{\bf File}: bn\_mp\_jacobi.c
\vspace{-3mm}
\begin{alltt}
\end{alltt}
\end{small}

As a matter of practicality the variable $a'$ as per the pseudo-code is reprensented by the variable $a1$ since the $'$ symbol is not valid for a C 
variable name character. 

The two simple cases of $a = 0$ and $a = 1$ are handled at the very beginning to simplify the algorithm.  If the input is non-trivial the algorithm
has to proceed compute the Jacobi.  The variable $s$ is used to hold the current Jacobi product.  Note that $s$ is merely a C ``int'' data type since
the values it may obtain are merely $-1$, $0$ and $1$.  

After a local copy of $a$ is made all of the factors of two are divided out and the total stored in $k$.  Technically only the least significant
bit of $k$ is required, however, it makes the algorithm simpler to follow to perform an addition. In practice an exclusive-or and addition have the same 
processor requirements and neither is faster than the other.

Line 58 through 71 determines the value of $\left ( { 2 \over p } \right )^k$.  If the least significant bit of $k$ is zero than
$k$ is even and the value is one.  Otherwise, the value of $s$ depends on which residue class $p$ belongs to modulo eight.  The value of
$(-1)^{(p-1)(a'-1)/4}$ is compute and multiplied against $s$ on lines 71 through 74.  

Finally, if $a1$ does not equal one the algorithm must recurse and compute $\left ( {p' \over a'} \right )$.  

\textit{-- Comment about default $s$ and such...}

\section{Modular Inverse}
\label{sec:modinv}
The modular inverse of a number actually refers to the modular multiplicative inverse.  Essentially for any integer $a$ such that $(a, p) = 1$ there
exist another integer $b$ such that $ab \equiv 1 \mbox{ (mod }p\mbox{)}$.  The integer $b$ is called the multiplicative inverse of $a$ which is
denoted as $b = a^{-1}$.  Technically speaking modular inversion is a well defined operation for any finite ring or field not just for rings and 
fields of integers.  However, the former will be the matter of discussion.

The simplest approach is to compute the algebraic inverse of the input.  That is to compute $b \equiv a^{\Phi(p) - 1}$.  If $\Phi(p)$ is the 
order of the multiplicative subgroup modulo $p$ then $b$ must be the multiplicative inverse of $a$.  The proof of which is trivial.

\begin{equation}
ab \equiv a \left (a^{\Phi(p) - 1} \right ) \equiv a^{\Phi(p)} \equiv a^0 \equiv 1 \mbox{ (mod }p\mbox{)}
\end{equation}

However, as simple as this approach may be it has two serious flaws.  It requires that the value of $\Phi(p)$ be known which if $p$ is composite 
requires all of the prime factors.  This approach also is very slow as the size of $p$ grows.  

A simpler approach is based on the observation that solving for the multiplicative inverse is equivalent to solving the linear 
Diophantine\footnote{See LeVeque \cite[pp. 40-43]{LeVeque} for more information.} equation.

\begin{equation}
ab + pq = 1
\end{equation}

Where $a$, $b$, $p$ and $q$ are all integers.  If such a pair of integers $ \left < b, q \right >$ exist than $b$ is the multiplicative inverse of 
$a$ modulo $p$.  The extended Euclidean algorithm (Knuth \cite[pp. 342]{TAOCPV2}) can be used to solve such equations provided $(a, p) = 1$.  
However, instead of using that algorithm directly a variant known as the binary Extended Euclidean algorithm will be used in its place.  The
binary approach is very similar to the binary greatest common divisor algorithm except it will produce a full solution to the Diophantine 
equation.  

\subsection{General Case}
\newpage\begin{figure}[!here]
\begin{small}
\begin{center}
\begin{tabular}{l}
\hline Algorithm \textbf{mp\_invmod}. \\
\textbf{Input}.   mp\_int $a$ and $b$, $(a, b) = 1$, $p \ge 2$, $0 < a < p$.  \\
\textbf{Output}.  The modular inverse $c \equiv a^{-1} \mbox{ (mod }b\mbox{)}$. \\
\hline \\
1.  If $b \le 0$ then return(\textit{MP\_VAL}). \\
2.  If $b_0 \equiv 1 \mbox{ (mod }2\mbox{)}$ then use algorithm fast\_mp\_invmod. \\
3.  $x \leftarrow \vert a \vert, y \leftarrow b$ \\
4.  If $x_0 \equiv y_0  \equiv 0 \mbox{ (mod }2\mbox{)}$ then return(\textit{MP\_VAL}). \\
5.  $B \leftarrow 0, C \leftarrow 0, A \leftarrow 1, D \leftarrow 1$ \\
6.  While $u.used > 0$ and $u_0 \equiv 0 \mbox{ (mod }2\mbox{)}$ \\
\hspace{3mm}6.1  $u \leftarrow \lfloor u / 2 \rfloor$ \\
\hspace{3mm}6.2  If ($A.used > 0$ and $A_0 \equiv 1 \mbox{ (mod }2\mbox{)}$) or ($B.used > 0$ and $B_0 \equiv 1 \mbox{ (mod }2\mbox{)}$) then \\
\hspace{6mm}6.2.1  $A \leftarrow A + y$ \\
\hspace{6mm}6.2.2  $B \leftarrow B - x$ \\
\hspace{3mm}6.3  $A \leftarrow \lfloor A / 2 \rfloor$ \\
\hspace{3mm}6.4  $B \leftarrow \lfloor B / 2 \rfloor$ \\
7.  While $v.used > 0$ and $v_0 \equiv 0 \mbox{ (mod }2\mbox{)}$ \\
\hspace{3mm}7.1  $v \leftarrow \lfloor v / 2 \rfloor$ \\
\hspace{3mm}7.2  If ($C.used > 0$ and $C_0 \equiv 1 \mbox{ (mod }2\mbox{)}$) or ($D.used > 0$ and $D_0 \equiv 1 \mbox{ (mod }2\mbox{)}$) then \\
\hspace{6mm}7.2.1  $C \leftarrow C + y$ \\
\hspace{6mm}7.2.2  $D \leftarrow D - x$ \\
\hspace{3mm}7.3  $C \leftarrow \lfloor C / 2 \rfloor$ \\
\hspace{3mm}7.4  $D \leftarrow \lfloor D / 2 \rfloor$ \\
8.  If $u \ge v$ then \\
\hspace{3mm}8.1  $u \leftarrow u - v$ \\
\hspace{3mm}8.2  $A \leftarrow A - C$ \\
\hspace{3mm}8.3  $B \leftarrow B - D$ \\
9.  else \\
\hspace{3mm}9.1  $v \leftarrow v - u$ \\
\hspace{3mm}9.2  $C \leftarrow C - A$ \\
\hspace{3mm}9.3  $D \leftarrow D - B$ \\
10.  If $u \ne 0$ goto step 6. \\
11.  If $v \ne 1$ return(\textit{MP\_VAL}). \\
12.  While $C \le 0$ do \\
\hspace{3mm}12.1  $C \leftarrow C + b$ \\
13.  While $C \ge b$ do \\
\hspace{3mm}13.1  $C \leftarrow C - b$ \\
14.  $c \leftarrow C$ \\
15.  Return(\textit{MP\_OKAY}). \\
\hline
\end{tabular}
\end{center}
\end{small}
\end{figure}
\textbf{Algorithm mp\_invmod.}
This algorithm computes the modular multiplicative inverse of an integer $a$ modulo an integer $b$.  This algorithm is a variation of the 
extended binary Euclidean algorithm from HAC \cite[pp. 608]{HAC}.  It has been modified to only compute the modular inverse and not a complete
Diophantine solution.  

If $b \le 0$ than the modulus is invalid and MP\_VAL is returned.  Similarly if both $a$ and $b$ are even then there cannot be a multiplicative
inverse for $a$ and the error is reported.  

The astute reader will observe that steps seven through nine are very similar to the binary greatest common divisor algorithm mp\_gcd.  In this case
the other variables to the Diophantine equation are solved.  The algorithm terminates when $u = 0$ in which case the solution is

\begin{equation}
Ca + Db = v
\end{equation}

If $v$, the greatest common divisor of $a$ and $b$ is not equal to one then the algorithm will report an error as no inverse exists.  Otherwise, $C$
is the modular inverse of $a$.  The actual value of $C$ is congruent to, but not necessarily equal to, the ideal modular inverse which should lie 
within $1 \le a^{-1} < b$.  Step numbers twelve and thirteen adjust the inverse until it is in range.  If the original input $a$ is within $0 < a < p$ 
then only a couple of additions or subtractions will be required to adjust the inverse.

\vspace{+3mm}\begin{small}
\hspace{-5.1mm}{\bf File}: bn\_mp\_invmod.c
\vspace{-3mm}
\begin{alltt}
\end{alltt}
\end{small}

\subsubsection{Odd Moduli}

When the modulus $b$ is odd the variables $A$ and $C$ are fixed and are not required to compute the inverse.  In particular by attempting to solve
the Diophantine $Cb + Da = 1$ only $B$ and $D$ are required to find the inverse of $a$.  

The algorithm fast\_mp\_invmod is a direct adaptation of algorithm mp\_invmod with all all steps involving either $A$ or $C$ removed.  This 
optimization will halve the time required to compute the modular inverse.

\section{Primality Tests}

A non-zero integer $a$ is said to be prime if it is not divisible by any other integer excluding one and itself.  For example, $a = 7$ is prime 
since the integers $2 \ldots 6$ do not evenly divide $a$.  By contrast, $a = 6$ is not prime since $a = 6 = 2 \cdot 3$. 

Prime numbers arise in cryptography considerably as they allow finite fields to be formed.  The ability to determine whether an integer is prime or
not quickly has been a viable subject in cryptography and number theory for considerable time.  The algorithms that will be presented are all
probablistic algorithms in that when they report an integer is composite it must be composite.  However, when the algorithms report an integer is
prime the algorithm may be incorrect.  

As will be discussed it is possible to limit the probability of error so well that for practical purposes the probablity of error might as 
well be zero.  For the purposes of these discussions let $n$ represent the candidate integer of which the primality is in question.

\subsection{Trial Division}

Trial division means to attempt to evenly divide a candidate integer by small prime integers.  If the candidate can be evenly divided it obviously
cannot be prime.  By dividing by all primes $1 < p \le \sqrt{n}$ this test can actually prove whether an integer is prime.  However, such a test
would require a prohibitive amount of time as $n$ grows.

Instead of dividing by every prime, a smaller, more mangeable set of primes may be used instead.  By performing trial division with only a subset
of the primes less than $\sqrt{n} + 1$ the algorithm cannot prove if a candidate is prime.  However, often it can prove a candidate is not prime.

The benefit of this test is that trial division by small values is fairly efficient.  Specially compared to the other algorithms that will be
discussed shortly.  The probability that this approach correctly identifies a composite candidate when tested with all primes upto $q$ is given by
$1 - {1.12 \over ln(q)}$.  The graph (\ref{pic:primality}, will be added later) demonstrates the probability of success for the range 
$3 \le q \le 100$.  

At approximately $q = 30$ the gain of performing further tests diminishes fairly quickly.  At $q = 90$ further testing is generally not going to 
be of any practical use.  In the case of LibTomMath the default limit $q = 256$ was chosen since it is not too high and will eliminate 
approximately $80\%$ of all candidate integers.  The constant \textbf{PRIME\_SIZE} is equal to the number of primes in the test base.  The 
array \_\_prime\_tab is an array of the first \textbf{PRIME\_SIZE} prime numbers.  

\begin{figure}[!here]
\begin{small}
\begin{center}
\begin{tabular}{l}
\hline Algorithm \textbf{mp\_prime\_is\_divisible}. \\
\textbf{Input}.   mp\_int $a$ \\
\textbf{Output}.  $c = 1$ if $n$ is divisible by a small prime, otherwise $c = 0$.  \\
\hline \\
1.  for $ix$ from $0$ to $PRIME\_SIZE$ do \\
\hspace{3mm}1.1  $d \leftarrow n \mbox{ (mod }\_\_prime\_tab_{ix}\mbox{)}$ \\
\hspace{3mm}1.2  If $d = 0$ then \\
\hspace{6mm}1.2.1  $c \leftarrow 1$ \\
\hspace{6mm}1.2.2  Return(\textit{MP\_OKAY}). \\
2.  $c \leftarrow 0$ \\
3.  Return(\textit{MP\_OKAY}). \\
\hline
\end{tabular}
\end{center}
\end{small}
\caption{Algorithm mp\_prime\_is\_divisible}
\end{figure}
\textbf{Algorithm mp\_prime\_is\_divisible.}
This algorithm attempts to determine if a candidate integer $n$ is composite by performing trial divisions.  

\vspace{+3mm}\begin{small}
\hspace{-5.1mm}{\bf File}: bn\_mp\_prime\_is\_divisible.c
\vspace{-3mm}
\begin{alltt}
\end{alltt}
\end{small}

The algorithm defaults to a return of $0$ in case an error occurs.  The values in the prime table are all specified to be in the range of a 
mp\_digit.  The table \_\_prime\_tab is defined in the following file.

\vspace{+3mm}\begin{small}
\hspace{-5.1mm}{\bf File}: bn\_prime\_tab.c
\vspace{-3mm}
\begin{alltt}
\end{alltt}
\end{small}

Note that there are two possible tables.  When an mp\_digit is 7-bits long only the primes upto $127$ may be included, otherwise the primes
upto $1619$ are used.  Note that the value of \textbf{PRIME\_SIZE} is a constant dependent on the size of a mp\_digit. 

\subsection{The Fermat Test}
The Fermat test is probably one the oldest tests to have a non-trivial probability of success.  It is based on the fact that if $n$ is in 
fact prime then $a^{n} \equiv a \mbox{ (mod }n\mbox{)}$ for all $0 < a < n$.  The reason being that if $n$ is prime than the order of
the multiplicative sub group is $n - 1$.  Any base $a$ must have an order which divides $n - 1$ and as such $a^n$ is equivalent to 
$a^1 = a$.  

If $n$ is composite then any given base $a$ does not have to have a period which divides $n - 1$.  In which case 
it is possible that $a^n \nequiv a \mbox{ (mod }n\mbox{)}$.  However, this test is not absolute as it is possible that the order
of a base will divide $n - 1$ which would then be reported as prime.  Such a base yields what is known as a Fermat pseudo-prime.  Several 
integers known as Carmichael numbers will be a pseudo-prime to all valid bases.  Fortunately such numbers are extremely rare as $n$ grows
in size.

\begin{figure}[!here]
\begin{small}
\begin{center}
\begin{tabular}{l}
\hline Algorithm \textbf{mp\_prime\_fermat}. \\
\textbf{Input}.   mp\_int $a$ and $b$, $a \ge 2$, $0 < b < a$.  \\
\textbf{Output}.  $c = 1$ if $b^a \equiv b \mbox{ (mod }a\mbox{)}$, otherwise $c = 0$.  \\
\hline \\
1.  $t \leftarrow b^a \mbox{ (mod }a\mbox{)}$ \\
2.  If $t = b$ then \\
\hspace{3mm}2.1  $c = 1$ \\
3.  else \\
\hspace{3mm}3.1  $c = 0$ \\
4.  Return(\textit{MP\_OKAY}). \\
\hline
\end{tabular}
\end{center}
\end{small}
\caption{Algorithm mp\_prime\_fermat}
\end{figure}
\textbf{Algorithm mp\_prime\_fermat.}
This algorithm determines whether an mp\_int $a$ is a Fermat prime to the base $b$ or not.  It uses a single modular exponentiation to
determine the result.  

\vspace{+3mm}\begin{small}
\hspace{-5.1mm}{\bf File}: bn\_mp\_prime\_fermat.c
\vspace{-3mm}
\begin{alltt}
\end{alltt}
\end{small}

\subsection{The Miller-Rabin Test}
The Miller-Rabin (citation) test is another primality test which has tighter error bounds than the Fermat test specifically with sequentially chosen 
candidate  integers.  The algorithm is based on the observation that if $n - 1 = 2^kr$ and if $b^r \nequiv \pm 1$ then after upto $k - 1$ squarings the 
value must be equal to $-1$.  The squarings are stopped as soon as $-1$ is observed.  If the value of $1$ is observed first it means that
some value not congruent to $\pm 1$ when squared equals one which cannot occur if $n$ is prime.

\begin{figure}[!here]
\begin{small}
\begin{center}
\begin{tabular}{l}
\hline Algorithm \textbf{mp\_prime\_miller\_rabin}. \\
\textbf{Input}.   mp\_int $a$ and $b$, $a \ge 2$, $0 < b < a$.  \\
\textbf{Output}.  $c = 1$ if $a$ is a Miller-Rabin prime to the base $a$, otherwise $c = 0$.  \\
\hline
1.  $a' \leftarrow a - 1$ \\
2.  $r  \leftarrow n1$    \\
3.  $c \leftarrow 0, s  \leftarrow 0$ \\
4.  While $r.used > 0$ and $r_0 \equiv 0 \mbox{ (mod }2\mbox{)}$ \\
\hspace{3mm}4.1  $s \leftarrow s + 1$ \\
\hspace{3mm}4.2  $r \leftarrow \lfloor r / 2 \rfloor$ \\
5.  $y \leftarrow b^r \mbox{ (mod }a\mbox{)}$ \\
6.  If $y \nequiv \pm 1$ then \\
\hspace{3mm}6.1  $j \leftarrow 1$ \\
\hspace{3mm}6.2  While $j \le (s - 1)$ and $y \nequiv a'$ \\
\hspace{6mm}6.2.1  $y \leftarrow y^2 \mbox{ (mod }a\mbox{)}$ \\
\hspace{6mm}6.2.2  If $y = 1$ then goto step 8. \\
\hspace{6mm}6.2.3  $j \leftarrow j + 1$ \\
\hspace{3mm}6.3  If $y \nequiv a'$ goto step 8. \\
7.  $c \leftarrow 1$\\
8.  Return(\textit{MP\_OKAY}). \\
\hline
\end{tabular}
\end{center}
\end{small}
\caption{Algorithm mp\_prime\_miller\_rabin}
\end{figure}
\textbf{Algorithm mp\_prime\_miller\_rabin.}
This algorithm performs one trial round of the Miller-Rabin algorithm to the base $b$.  It will set $c = 1$ if the algorithm cannot determine
if $b$ is composite or $c = 0$ if $b$ is provably composite.  The values of $s$ and $r$ are computed such that $a' = a - 1 = 2^sr$.  

If the value $y \equiv b^r$ is congruent to $\pm 1$ then the algorithm cannot prove if $a$ is composite or not.  Otherwise, the algorithm will
square $y$ upto $s - 1$ times stopping only when $y \equiv -1$.  If $y^2 \equiv 1$ and $y \nequiv \pm 1$ then the algorithm can report that $a$
is provably composite.  If the algorithm performs $s - 1$ squarings and $y \nequiv -1$ then $a$ is provably composite.  If $a$ is not provably 
composite then it is \textit{probably} prime.

\vspace{+3mm}\begin{small}
\hspace{-5.1mm}{\bf File}: bn\_mp\_prime\_miller\_rabin.c
\vspace{-3mm}
\begin{alltt}
\end{alltt}
\end{small}




\backmatter
\appendix
\begin{thebibliography}{ABCDEF}
\bibitem[1]{TAOCPV2}
Donald Knuth, \textit{The Art of Computer Programming}, Third Edition, Volume Two, Seminumerical Algorithms, Addison-Wesley, 1998

\bibitem[2]{HAC}
A. Menezes, P. van Oorschot, S. Vanstone, \textit{Handbook of Applied Cryptography}, CRC Press, 1996

\bibitem[3]{ROSE}
Michael Rosing, \textit{Implementing Elliptic Curve Cryptography}, Manning Publications, 1999

\bibitem[4]{COMBA}
Paul G. Comba, \textit{Exponentiation Cryptosystems on the IBM PC}. IBM Systems Journal 29(4): 526-538 (1990)

\bibitem[5]{KARA}
A. Karatsuba, Doklay Akad. Nauk SSSR 145 (1962), pp.293-294

\bibitem[6]{KARAP}
Andre Weimerskirch and Christof Paar, \textit{Generalizations of the Karatsuba Algorithm for Polynomial Multiplication}, Submitted to Design, Codes and Cryptography, March 2002

\bibitem[7]{BARRETT}
Paul Barrett, \textit{Implementing the Rivest Shamir and Adleman Public Key Encryption Algorithm on a Standard Digital Signal Processor}, Advances in Cryptology, Crypto '86, Springer-Verlag.

\bibitem[8]{MONT}
P.L.Montgomery. \textit{Modular multiplication without trial division}. Mathematics of Computation, 44(170):519-521, April 1985.

\bibitem[9]{DRMET}
Chae Hoon Lim and Pil Joong Lee, \textit{Generating Efficient Primes for Discrete Log Cryptosystems}, POSTECH Information Research Laboratories

\bibitem[10]{MMB}
J. Daemen and R. Govaerts and J. Vandewalle, \textit{Block ciphers based on Modular Arithmetic}, State and {P}rogress in the {R}esearch of {C}ryptography, 1993, pp. 80-89

\bibitem[11]{RSAREF}
R.L. Rivest, A. Shamir, L. Adleman, \textit{A Method for Obtaining Digital Signatures and Public-Key Cryptosystems}

\bibitem[12]{DHREF}
Whitfield Diffie, Martin E. Hellman, \textit{New Directions in Cryptography}, IEEE Transactions on Information Theory, 1976

\bibitem[13]{IEEE}
IEEE Standard for Binary Floating-Point Arithmetic (ANSI/IEEE Std 754-1985)

\bibitem[14]{GMP}
GNU Multiple Precision (GMP), \url{http://www.swox.com/gmp/}

\bibitem[15]{MPI}
Multiple Precision Integer Library (MPI), Michael Fromberger, \url{http://thayer.dartmouth.edu/~sting/mpi/}

\bibitem[16]{OPENSSL}
OpenSSL Cryptographic Toolkit, \url{http://openssl.org}

\bibitem[17]{LIP}
Large Integer Package, \url{http://home.hetnet.nl/~ecstr/LIP.zip}

\bibitem[18]{ISOC}
JTC1/SC22/WG14, ISO/IEC 9899:1999, ``A draft rationale for the C99 standard.''

\bibitem[19]{JAVA}
The Sun Java Website, \url{http://java.sun.com/}

\end{thebibliography}

\documentclass[b5paper]{book}
\usepackage{hyperref}
\usepackage{makeidx}
\usepackage{amssymb}
\usepackage{color}
\usepackage{alltt}
\usepackage{graphicx}
\usepackage{layout}
\def\union{\cup}
\def\intersect{\cap}
\def\getsrandom{\stackrel{\rm R}{\gets}}
\def\cross{\times}
\def\cat{\hspace{0.5em} \| \hspace{0.5em}}
\def\catn{$\|$}
\def\divides{\hspace{0.3em} | \hspace{0.3em}}
\def\nequiv{\not\equiv}
\def\approx{\raisebox{0.2ex}{\mbox{\small $\sim$}}}
\def\lcm{{\rm lcm}}
\def\gcd{{\rm gcd}}
\def\log{{\rm log}}
\def\ord{{\rm ord}}
\def\abs{{\mathit abs}}
\def\rep{{\mathit rep}}
\def\mod{{\mathit\ mod\ }}
\renewcommand{\pmod}[1]{\ ({\rm mod\ }{#1})}
\newcommand{\floor}[1]{\left\lfloor{#1}\right\rfloor}
\newcommand{\ceil}[1]{\left\lceil{#1}\right\rceil}
\def\Or{{\rm\ or\ }}
\def\And{{\rm\ and\ }}
\def\iff{\hspace{1em}\Longleftrightarrow\hspace{1em}}
\def\implies{\Rightarrow}
\def\undefined{{\rm ``undefined"}}
\def\Proof{\vspace{1ex}\noindent {\bf Proof:}\hspace{1em}}
\let\oldphi\phi
\def\phi{\varphi}
\def\Pr{{\rm Pr}}
\newcommand{\str}[1]{{\mathbf{#1}}}
\def\F{{\mathbb F}}
\def\N{{\mathbb N}}
\def\Z{{\mathbb Z}}
\def\R{{\mathbb R}}
\def\C{{\mathbb C}}
\def\Q{{\mathbb Q}}
\definecolor{DGray}{gray}{0.5}
\newcommand{\emailaddr}[1]{\mbox{$<${#1}$>$}}
\def\twiddle{\raisebox{0.3ex}{\mbox{\tiny $\sim$}}}
\def\gap{\vspace{0.5ex}}
\makeindex
\begin{document}
\frontmatter
\pagestyle{empty}
\title{Multi--Precision Math}
\author{\mbox{
%\begin{small}
\begin{tabular}{c}
Tom St Denis \\
Algonquin College \\
\\
Mads Rasmussen \\
Open Communications Security \\
\\
Greg Rose \\
QUALCOMM Australia \\
\end{tabular}
%\end{small}
}
}
\maketitle
This text has been placed in the public domain.  This text corresponds to the v0.39 release of the 
LibTomMath project.

\begin{alltt}
Tom St Denis
111 Banning Rd
Ottawa, Ontario
K2L 1C3
Canada

Phone: 1-613-836-3160
Email: tomstdenis@gmail.com
\end{alltt}

This text is formatted to the international B5 paper size of 176mm wide by 250mm tall using the \LaTeX{} 
{\em book} macro package and the Perl {\em booker} package.

\tableofcontents
\listoffigures
\chapter*{Prefaces}
When I tell people about my LibTom projects and that I release them as public domain they are often puzzled.  
They ask why I did it and especially why I continue to work on them for free.  The best I can explain it is ``Because I can.''  
Which seems odd and perhaps too terse for adult conversation. I often qualify it with ``I am able, I am willing.'' which 
perhaps explains it better.  I am the first to admit there is not anything that special with what I have done.  Perhaps
others can see that too and then we would have a society to be proud of.  My LibTom projects are what I am doing to give 
back to society in the form of tools and knowledge that can help others in their endeavours.

I started writing this book because it was the most logical task to further my goal of open academia.  The LibTomMath source
code itself was written to be easy to follow and learn from.  There are times, however, where pure C source code does not
explain the algorithms properly.  Hence this book.  The book literally starts with the foundation of the library and works
itself outwards to the more complicated algorithms.  The use of both pseudo--code and verbatim source code provides a duality
of ``theory'' and ``practice'' that the computer science students of the world shall appreciate.  I never deviate too far
from relatively straightforward algebra and I hope that this book can be a valuable learning asset.

This book and indeed much of the LibTom projects would not exist in their current form if it was not for a plethora
of kind people donating their time, resources and kind words to help support my work.  Writing a text of significant
length (along with the source code) is a tiresome and lengthy process.  Currently the LibTom project is four years old,
comprises of literally thousands of users and over 100,000 lines of source code, TeX and other material.  People like Mads and Greg 
were there at the beginning to encourage me to work well.  It is amazing how timely validation from others can boost morale to 
continue the project. Definitely my parents were there for me by providing room and board during the many months of work in 2003.  

To my many friends whom I have met through the years I thank you for the good times and the words of encouragement.  I hope I
honour your kind gestures with this project.

Open Source.  Open Academia.  Open Minds.

\begin{flushright} Tom St Denis \end{flushright}

\newpage
I found the opportunity to work with Tom appealing for several reasons, not only could I broaden my own horizons, but also 
contribute to educate others facing the problem of having to handle big number mathematical calculations.

This book is Tom's child and he has been caring and fostering the project ever since the beginning with a clear mind of 
how he wanted the project to turn out. I have helped by proofreading the text and we have had several discussions about 
the layout and language used.

I hold a masters degree in cryptography from the University of Southern Denmark and have always been interested in the 
practical aspects of cryptography. 

Having worked in the security consultancy business for several years in S\~{a}o Paulo, Brazil, I have been in touch with a 
great deal of work in which multiple precision mathematics was needed. Understanding the possibilities for speeding up 
multiple precision calculations is often very important since we deal with outdated machine architecture where modular 
reductions, for example, become painfully slow.

This text is for people who stop and wonder when first examining algorithms such as RSA for the first time and asks 
themselves, ``You tell me this is only secure for large numbers, fine; but how do you implement these numbers?''

\begin{flushright}
Mads Rasmussen

S\~{a}o Paulo - SP

Brazil
\end{flushright}

\newpage
It's all because I broke my leg. That just happened to be at about the same time that Tom asked for someone to review the section of the book about 
Karatsuba multiplication. I was laid up, alone and immobile, and thought ``Why not?'' I vaguely knew what Karatsuba multiplication was, but not 
really, so I thought I could help, learn, and stop myself from watching daytime cable TV, all at once.

At the time of writing this, I've still not met Tom or Mads in meatspace. I've been following Tom's progress since his first splash on the 
sci.crypt Usenet news group. I watched him go from a clueless newbie, to the cryptographic equivalent of a reformed smoker, to a real
contributor to the field, over a period of about two years. I've been impressed with his obvious intelligence, and astounded by his productivity. 
Of course, he's young enough to be my own child, so he doesn't have my problems with staying awake.

When I reviewed that single section of the book, in its very earliest form, I was very pleasantly surprised. So I decided to collaborate more fully, 
and at least review all of it, and perhaps write some bits too. There's still a long way to go with it, and I have watched a number of close 
friends go through the mill of publication, so I think that the way to go is longer than Tom thinks it is. Nevertheless, it's a good effort, 
and I'm pleased to be involved with it.

\begin{flushright}
Greg Rose, Sydney, Australia, June 2003. 
\end{flushright}

\mainmatter
\pagestyle{headings}
\chapter{Introduction}
\section{Multiple Precision Arithmetic}

\subsection{What is Multiple Precision Arithmetic?}
When we think of long-hand arithmetic such as addition or multiplication we rarely consider the fact that we instinctively
raise or lower the precision of the numbers we are dealing with.  For example, in decimal we almost immediate can 
reason that $7$ times $6$ is $42$.  However, $42$ has two digits of precision as opposed to one digit we started with.  
Further multiplications of say $3$ result in a larger precision result $126$.  In these few examples we have multiple 
precisions for the numbers we are working with.  Despite the various levels of precision a single subset\footnote{With the occasional optimization.}
 of algorithms can be designed to accomodate them.  

By way of comparison a fixed or single precision operation would lose precision on various operations.  For example, in
the decimal system with fixed precision $6 \cdot 7 = 2$.

Essentially at the heart of computer based multiple precision arithmetic are the same long-hand algorithms taught in
schools to manually add, subtract, multiply and divide.  

\subsection{The Need for Multiple Precision Arithmetic}
The most prevalent need for multiple precision arithmetic, often referred to as ``bignum'' math, is within the implementation
of public-key cryptography algorithms.   Algorithms such as RSA \cite{RSAREF} and Diffie-Hellman \cite{DHREF} require 
integers of significant magnitude to resist known cryptanalytic attacks.  For example, at the time of this writing a 
typical RSA modulus would be at least greater than $10^{309}$.  However, modern programming languages such as ISO C \cite{ISOC} and 
Java \cite{JAVA} only provide instrinsic support for integers which are relatively small and single precision.

\begin{figure}[!here]
\begin{center}
\begin{tabular}{|r|c|}
\hline \textbf{Data Type} & \textbf{Range} \\
\hline char  & $-128 \ldots 127$ \\
\hline short & $-32768 \ldots 32767$ \\
\hline long  & $-2147483648 \ldots 2147483647$ \\
\hline long long & $-9223372036854775808 \ldots 9223372036854775807$ \\
\hline
\end{tabular}
\end{center}
\caption{Typical Data Types for the C Programming Language}
\label{fig:ISOC}
\end{figure}

The largest data type guaranteed to be provided by the ISO C programming 
language\footnote{As per the ISO C standard.  However, each compiler vendor is allowed to augment the precision as they 
see fit.}  can only represent values up to $10^{19}$ as shown in figure \ref{fig:ISOC}. On its own the C language is 
insufficient to accomodate the magnitude required for the problem at hand.  An RSA modulus of magnitude $10^{19}$ could be 
trivially factored\footnote{A Pollard-Rho factoring would take only $2^{16}$ time.} on the average desktop computer, 
rendering any protocol based on the algorithm insecure.  Multiple precision algorithms solve this very problem by 
extending the range of representable integers while using single precision data types.

Most advancements in fast multiple precision arithmetic stem from the need for faster and more efficient cryptographic 
primitives.  Faster modular reduction and exponentiation algorithms such as Barrett's algorithm, which have appeared in 
various cryptographic journals, can render algorithms such as RSA and Diffie-Hellman more efficient.  In fact, several 
major companies such as RSA Security, Certicom and Entrust have built entire product lines on the implementation and 
deployment of efficient algorithms.

However, cryptography is not the only field of study that can benefit from fast multiple precision integer routines.  
Another auxiliary use of multiple precision integers is high precision floating point data types.  
The basic IEEE \cite{IEEE} standard floating point type is made up of an integer mantissa $q$, an exponent $e$ and a sign bit $s$.  
Numbers are given in the form $n = q \cdot b^e \cdot -1^s$ where $b = 2$ is the most common base for IEEE.  Since IEEE 
floating point is meant to be implemented in hardware the precision of the mantissa is often fairly small 
(\textit{23, 48 and 64 bits}).  The mantissa is merely an integer and a multiple precision integer could be used to create
a mantissa of much larger precision than hardware alone can efficiently support.  This approach could be useful where 
scientific applications must minimize the total output error over long calculations.

Yet another use for large integers is within arithmetic on polynomials of large characteristic (i.e. $GF(p)[x]$ for large $p$).
In fact the library discussed within this text has already been used to form a polynomial basis library\footnote{See \url{http://poly.libtomcrypt.org} for more details.}.

\subsection{Benefits of Multiple Precision Arithmetic}
\index{precision}
The benefit of multiple precision representations over single or fixed precision representations is that 
no precision is lost while representing the result of an operation which requires excess precision.  For example, 
the product of two $n$-bit integers requires at least $2n$ bits of precision to be represented faithfully.  A multiple 
precision algorithm would augment the precision of the destination to accomodate the result while a single precision system 
would truncate excess bits to maintain a fixed level of precision.

It is possible to implement algorithms which require large integers with fixed precision algorithms.  For example, elliptic
curve cryptography (\textit{ECC}) is often implemented on smartcards by fixing the precision of the integers to the maximum 
size the system will ever need.  Such an approach can lead to vastly simpler algorithms which can accomodate the 
integers required even if the host platform cannot natively accomodate them\footnote{For example, the average smartcard 
processor has an 8 bit accumulator.}.  However, as efficient as such an approach may be, the resulting source code is not
normally very flexible.  It cannot, at runtime, accomodate inputs of higher magnitude than the designer anticipated.

Multiple precision algorithms have the most overhead of any style of arithmetic.  For the the most part the 
overhead can be kept to a minimum with careful planning, but overall, it is not well suited for most memory starved
platforms.  However, multiple precision algorithms do offer the most flexibility in terms of the magnitude of the 
inputs.  That is, the same algorithms based on multiple precision integers can accomodate any reasonable size input 
without the designer's explicit forethought.  This leads to lower cost of ownership for the code as it only has to 
be written and tested once.

\section{Purpose of This Text}
The purpose of this text is to instruct the reader regarding how to implement efficient multiple precision algorithms.  
That is to not only explain a limited subset of the core theory behind the algorithms but also the various ``house keeping'' 
elements that are neglected by authors of other texts on the subject.  Several well reknowned texts \cite{TAOCPV2,HAC} 
give considerably detailed explanations of the theoretical aspects of algorithms and often very little information 
regarding the practical implementation aspects.  

In most cases how an algorithm is explained and how it is actually implemented are two very different concepts.  For 
example, the Handbook of Applied Cryptography (\textit{HAC}), algorithm 14.7 on page 594, gives a relatively simple 
algorithm for performing multiple precision integer addition.  However, the description lacks any discussion concerning 
the fact that the two integer inputs may be of differing magnitudes.  As a result the implementation is not as simple
as the text would lead people to believe.  Similarly the division routine (\textit{algorithm 14.20, pp. 598}) does not 
discuss how to handle sign or handle the dividend's decreasing magnitude in the main loop (\textit{step \#3}).

Both texts also do not discuss several key optimal algorithms required such as ``Comba'' and Karatsuba multipliers 
and fast modular inversion, which we consider practical oversights.  These optimal algorithms are vital to achieve 
any form of useful performance in non-trivial applications.  

To solve this problem the focus of this text is on the practical aspects of implementing a multiple precision integer
package.  As a case study the ``LibTomMath''\footnote{Available at \url{http://math.libtomcrypt.com}} package is used 
to demonstrate algorithms with real implementations\footnote{In the ISO C programming language.} that have been field 
tested and work very well.  The LibTomMath library is freely available on the Internet for all uses and this text 
discusses a very large portion of the inner workings of the library.

The algorithms that are presented will always include at least one ``pseudo-code'' description followed 
by the actual C source code that implements the algorithm.  The pseudo-code can be used to implement the same 
algorithm in other programming languages as the reader sees fit.  

This text shall also serve as a walkthrough of the creation of multiple precision algorithms from scratch.  Showing
the reader how the algorithms fit together as well as where to start on various taskings.  

\section{Discussion and Notation}
\subsection{Notation}
A multiple precision integer of $n$-digits shall be denoted as $x = (x_{n-1}, \ldots, x_1, x_0)_{ \beta }$ and represent
the integer $x \equiv \sum_{i=0}^{n-1} x_i\beta^i$.  The elements of the array $x$ are said to be the radix $\beta$ digits 
of the integer.  For example, $x = (1,2,3)_{10}$ would represent the integer 
$1\cdot 10^2 + 2\cdot10^1 + 3\cdot10^0 = 123$.  

\index{mp\_int}
The term ``mp\_int'' shall refer to a composite structure which contains the digits of the integer it represents, as well 
as auxilary data required to manipulate the data.  These additional members are discussed further in section 
\ref{sec:MPINT}.  For the purposes of this text a ``multiple precision integer'' and an ``mp\_int'' are assumed to be 
synonymous.  When an algorithm is specified to accept an mp\_int variable it is assumed the various auxliary data members 
are present as well.  An expression of the type \textit{variablename.item} implies that it should evaluate to the 
member named ``item'' of the variable.  For example, a string of characters may have a member ``length'' which would 
evaluate to the number of characters in the string.  If the string $a$ equals ``hello'' then it follows that 
$a.length = 5$.  

For certain discussions more generic algorithms are presented to help the reader understand the final algorithm used
to solve a given problem.  When an algorithm is described as accepting an integer input it is assumed the input is 
a plain integer with no additional multiple-precision members.  That is, algorithms that use integers as opposed to 
mp\_ints as inputs do not concern themselves with the housekeeping operations required such as memory management.  These 
algorithms will be used to establish the relevant theory which will subsequently be used to describe a multiple
precision algorithm to solve the same problem.  

\subsection{Precision Notation}
The variable $\beta$ represents the radix of a single digit of a multiple precision integer and 
must be of the form $q^p$ for $q, p \in \Z^+$.  A single precision variable must be able to represent integers in 
the range $0 \le x < q \beta$ while a double precision variable must be able to represent integers in the range 
$0 \le x < q \beta^2$.  The extra radix-$q$ factor allows additions and subtractions to proceed without truncation of the 
carry.  Since all modern computers are binary, it is assumed that $q$ is two.

\index{mp\_digit} \index{mp\_word}
Within the source code that will be presented for each algorithm, the data type \textbf{mp\_digit} will represent 
a single precision integer type, while, the data type \textbf{mp\_word} will represent a double precision integer type.  In 
several algorithms (notably the Comba routines) temporary results will be stored in arrays of double precision mp\_words.  
For the purposes of this text $x_j$ will refer to the $j$'th digit of a single precision array and $\hat x_j$ will refer to 
the $j$'th digit of a double precision array.  Whenever an expression is to be assigned to a double precision
variable it is assumed that all single precision variables are promoted to double precision during the evaluation.  
Expressions that are assigned to a single precision variable are truncated to fit within the precision of a single
precision data type.

For example, if $\beta = 10^2$ a single precision data type may represent a value in the 
range $0 \le x < 10^3$, while a double precision data type may represent a value in the range $0 \le x < 10^5$.  Let
$a = 23$ and $b = 49$ represent two single precision variables.  The single precision product shall be written
as $c \leftarrow a \cdot b$ while the double precision product shall be written as $\hat c \leftarrow a \cdot b$.
In this particular case, $\hat c = 1127$ and $c = 127$.  The most significant digit of the product would not fit 
in a single precision data type and as a result $c \ne \hat c$.  

\subsection{Algorithm Inputs and Outputs}
Within the algorithm descriptions all variables are assumed to be scalars of either single or double precision
as indicated.  The only exception to this rule is when variables have been indicated to be of type mp\_int.  This 
distinction is important as scalars are often used as array indicies and various other counters.  

\subsection{Mathematical Expressions}
The $\lfloor \mbox{ } \rfloor$ brackets imply an expression truncated to an integer not greater than the expression 
itself.  For example, $\lfloor 5.7 \rfloor = 5$.  Similarly the $\lceil \mbox{ } \rceil$ brackets imply an expression
rounded to an integer not less than the expression itself.  For example, $\lceil 5.1 \rceil = 6$.  Typically when 
the $/$ division symbol is used the intention is to perform an integer division with truncation.  For example, 
$5/2 = 2$ which will often be written as $\lfloor 5/2 \rfloor = 2$ for clarity.  When an expression is written as a 
fraction a real value division is implied, for example ${5 \over 2} = 2.5$.  

The norm of a multiple precision integer, for example $\vert \vert x \vert \vert$, will be used to represent the number of digits in the representation
of the integer.  For example, $\vert \vert 123 \vert \vert = 3$ and $\vert \vert 79452 \vert \vert = 5$.  

\subsection{Work Effort}
\index{big-Oh}
To measure the efficiency of the specified algorithms, a modified big-Oh notation is used.  In this system all 
single precision operations are considered to have the same cost\footnote{Except where explicitly noted.}.  
That is a single precision addition, multiplication and division are assumed to take the same time to 
complete.  While this is generally not true in practice, it will simplify the discussions considerably.

Some algorithms have slight advantages over others which is why some constants will not be removed in 
the notation.  For example, a normal baseline multiplication (section \ref{sec:basemult}) requires $O(n^2)$ work while a 
baseline squaring (section \ref{sec:basesquare}) requires $O({{n^2 + n}\over 2})$ work.  In standard big-Oh notation these 
would both be said to be equivalent to $O(n^2)$.  However, 
in the context of the this text this is not the case as the magnitude of the inputs will typically be rather small.  As a 
result small constant factors in the work effort will make an observable difference in algorithm efficiency.

All of the algorithms presented in this text have a polynomial time work level.  That is, of the form 
$O(n^k)$ for $n, k \in \Z^{+}$.  This will help make useful comparisons in terms of the speed of the algorithms and how 
various optimizations will help pay off in the long run.

\section{Exercises}
Within the more advanced chapters a section will be set aside to give the reader some challenging exercises related to
the discussion at hand.  These exercises are not designed to be prize winning problems, but instead to be thought 
provoking.  Wherever possible the problems are forward minded, stating problems that will be answered in subsequent 
chapters.  The reader is encouraged to finish the exercises as they appear to get a better understanding of the 
subject material.  

That being said, the problems are designed to affirm knowledge of a particular subject matter.  Students in particular
are encouraged to verify they can answer the problems correctly before moving on.

Similar to the exercises of \cite[pp. ix]{TAOCPV2} these exercises are given a scoring system based on the difficulty of
the problem.  However, unlike \cite{TAOCPV2} the problems do not get nearly as hard.  The scoring of these 
exercises ranges from one (the easiest) to five (the hardest).  The following table sumarizes the 
scoring system used.

\begin{figure}[here]
\begin{center}
\begin{small}
\begin{tabular}{|c|l|}
\hline $\left [ 1 \right ]$ & An easy problem that should only take the reader a manner of \\
                            & minutes to solve.  Usually does not involve much computer time \\
                            & to solve. \\
\hline $\left [ 2 \right ]$ & An easy problem that involves a marginal amount of computer \\
                     & time usage.  Usually requires a program to be written to \\
                     & solve the problem. \\
\hline $\left [ 3 \right ]$ & A moderately hard problem that requires a non-trivial amount \\
                     & of work.  Usually involves trivial research and development of \\
                     & new theory from the perspective of a student. \\
\hline $\left [ 4 \right ]$ & A moderately hard problem that involves a non-trivial amount \\
                     & of work and research, the solution to which will demonstrate \\
                     & a higher mastery of the subject matter. \\
\hline $\left [ 5 \right ]$ & A hard problem that involves concepts that are difficult for a \\
                     & novice to solve.  Solutions to these problems will demonstrate a \\
                     & complete mastery of the given subject. \\
\hline
\end{tabular}
\end{small}
\end{center}
\caption{Exercise Scoring System}
\end{figure}

Problems at the first level are meant to be simple questions that the reader can answer quickly without programming a solution or
devising new theory.  These problems are quick tests to see if the material is understood.  Problems at the second level 
are also designed to be easy but will require a program or algorithm to be implemented to arrive at the answer.  These
two levels are essentially entry level questions.  

Problems at the third level are meant to be a bit more difficult than the first two levels.  The answer is often 
fairly obvious but arriving at an exacting solution requires some thought and skill.  These problems will almost always 
involve devising a new algorithm or implementing a variation of another algorithm previously presented.  Readers who can
answer these questions will feel comfortable with the concepts behind the topic at hand.

Problems at the fourth level are meant to be similar to those of the level three questions except they will require 
additional research to be completed.  The reader will most likely not know the answer right away, nor will the text provide 
the exact details of the answer until a subsequent chapter.  

Problems at the fifth level are meant to be the hardest 
problems relative to all the other problems in the chapter.  People who can correctly answer fifth level problems have a 
mastery of the subject matter at hand.

Often problems will be tied together.  The purpose of this is to start a chain of thought that will be discussed in future chapters.  The reader
is encouraged to answer the follow-up problems and try to draw the relevance of problems.

\section{Introduction to LibTomMath}

\subsection{What is LibTomMath?}
LibTomMath is a free and open source multiple precision integer library written entirely in portable ISO C.  By portable it 
is meant that the library does not contain any code that is computer platform dependent or otherwise problematic to use on 
any given platform.  

The library has been successfully tested under numerous operating systems including Unix\footnote{All of these
trademarks belong to their respective rightful owners.}, MacOS, Windows, Linux, PalmOS and on standalone hardware such 
as the Gameboy Advance.  The library is designed to contain enough functionality to be able to develop applications such 
as public key cryptosystems and still maintain a relatively small footprint.

\subsection{Goals of LibTomMath}

Libraries which obtain the most efficiency are rarely written in a high level programming language such as C.  However, 
even though this library is written entirely in ISO C, considerable care has been taken to optimize the algorithm implementations within the 
library.  Specifically the code has been written to work well with the GNU C Compiler (\textit{GCC}) on both x86 and ARM 
processors.  Wherever possible, highly efficient algorithms, such as Karatsuba multiplication, sliding window 
exponentiation and Montgomery reduction have been provided to make the library more efficient.  

Even with the nearly optimal and specialized algorithms that have been included the Application Programing Interface 
(\textit{API}) has been kept as simple as possible.  Often generic place holder routines will make use of specialized 
algorithms automatically without the developer's specific attention.  One such example is the generic multiplication 
algorithm \textbf{mp\_mul()} which will automatically use Toom--Cook, Karatsuba, Comba or baseline multiplication 
based on the magnitude of the inputs and the configuration of the library.  

Making LibTomMath as efficient as possible is not the only goal of the LibTomMath project.  Ideally the library should 
be source compatible with another popular library which makes it more attractive for developers to use.  In this case the
MPI library was used as a API template for all the basic functions.  MPI was chosen because it is another library that fits 
in the same niche as LibTomMath.  Even though LibTomMath uses MPI as the template for the function names and argument 
passing conventions, it has been written from scratch by Tom St Denis.

The project is also meant to act as a learning tool for students, the logic being that no easy-to-follow ``bignum'' 
library exists which can be used to teach computer science students how to perform fast and reliable multiple precision 
integer arithmetic.  To this end the source code has been given quite a few comments and algorithm discussion points.  

\section{Choice of LibTomMath}
LibTomMath was chosen as the case study of this text not only because the author of both projects is one and the same but
for more worthy reasons.  Other libraries such as GMP \cite{GMP}, MPI \cite{MPI}, LIP \cite{LIP} and OpenSSL 
\cite{OPENSSL} have multiple precision integer arithmetic routines but would not be ideal for this text for 
reasons that will be explained in the following sub-sections.

\subsection{Code Base}
The LibTomMath code base is all portable ISO C source code.  This means that there are no platform dependent conditional
segments of code littered throughout the source.  This clean and uncluttered approach to the library means that a
developer can more readily discern the true intent of a given section of source code without trying to keep track of
what conditional code will be used.

The code base of LibTomMath is well organized.  Each function is in its own separate source code file 
which allows the reader to find a given function very quickly.  On average there are $76$ lines of code per source
file which makes the source very easily to follow.  By comparison MPI and LIP are single file projects making code tracing
very hard.  GMP has many conditional code segments which also hinder tracing.  

When compiled with GCC for the x86 processor and optimized for speed the entire library is approximately $100$KiB\footnote{The notation ``KiB'' means $2^{10}$ octets, similarly ``MiB'' means $2^{20}$ octets.}
 which is fairly small compared to GMP (over $250$KiB).  LibTomMath is slightly larger than MPI (which compiles to about 
$50$KiB) but LibTomMath is also much faster and more complete than MPI.

\subsection{API Simplicity}
LibTomMath is designed after the MPI library and shares the API design.  Quite often programs that use MPI will build 
with LibTomMath without change. The function names correlate directly to the action they perform.  Almost all of the 
functions share the same parameter passing convention.  The learning curve is fairly shallow with the API provided 
which is an extremely valuable benefit for the student and developer alike.  

The LIP library is an example of a library with an API that is awkward to work with.  LIP uses function names that are often ``compressed'' to 
illegible short hand.  LibTomMath does not share this characteristic.  

The GMP library also does not return error codes.  Instead it uses a POSIX.1 \cite{POSIX1} signal system where errors
are signaled to the host application.  This happens to be the fastest approach but definitely not the most versatile.  In
effect a math error (i.e. invalid input, heap error, etc) can cause a program to stop functioning which is definitely 
undersireable in many situations.

\subsection{Optimizations}
While LibTomMath is certainly not the fastest library (GMP often beats LibTomMath by a factor of two) it does
feature a set of optimal algorithms for tasks such as modular reduction, exponentiation, multiplication and squaring.  GMP 
and LIP also feature such optimizations while MPI only uses baseline algorithms with no optimizations.  GMP lacks a few
of the additional modular reduction optimizations that LibTomMath features\footnote{At the time of this writing GMP
only had Barrett and Montgomery modular reduction algorithms.}.  

LibTomMath is almost always an order of magnitude faster than the MPI library at computationally expensive tasks such as modular
exponentiation.  In the grand scheme of ``bignum'' libraries LibTomMath is faster than the average library and usually  
slower than the best libraries such as GMP and OpenSSL by only a small factor.

\subsection{Portability and Stability}
LibTomMath will build ``out of the box'' on any platform equipped with a modern version of the GNU C Compiler 
(\textit{GCC}).  This means that without changes the library will build without configuration or setting up any 
variables.  LIP and MPI will build ``out of the box'' as well but have numerous known bugs.  Most notably the author of 
MPI has recently stopped working on his library and LIP has long since been discontinued.  

GMP requires a configuration script to run and will not build out of the box.   GMP and LibTomMath are still in active
development and are very stable across a variety of platforms.

\subsection{Choice}
LibTomMath is a relatively compact, well documented, highly optimized and portable library which seems only natural for
the case study of this text.  Various source files from the LibTomMath project will be included within the text.  However, 
the reader is encouraged to download their own copy of the library to actually be able to work with the library.  

\chapter{Getting Started}
\section{Library Basics}
The trick to writing any useful library of source code is to build a solid foundation and work outwards from it.  First, 
a problem along with allowable solution parameters should be identified and analyzed.  In this particular case the 
inability to accomodate multiple precision integers is the problem.  Futhermore, the solution must be written
as portable source code that is reasonably efficient across several different computer platforms.

After a foundation is formed the remainder of the library can be designed and implemented in a hierarchical fashion.  
That is, to implement the lowest level dependencies first and work towards the most abstract functions last.  For example, 
before implementing a modular exponentiation algorithm one would implement a modular reduction algorithm.
By building outwards from a base foundation instead of using a parallel design methodology the resulting project is 
highly modular.  Being highly modular is a desirable property of any project as it often means the resulting product
has a small footprint and updates are easy to perform.  

Usually when I start a project I will begin with the header files.  I define the data types I think I will need and 
prototype the initial functions that are not dependent on other functions (within the library).  After I 
implement these base functions I prototype more dependent functions and implement them.   The process repeats until
I implement all of the functions I require.  For example, in the case of LibTomMath I implemented functions such as 
mp\_init() well before I implemented mp\_mul() and even further before I implemented mp\_exptmod().  As an example as to 
why this design works note that the Karatsuba and Toom-Cook multipliers were written \textit{after} the 
dependent function mp\_exptmod() was written.  Adding the new multiplication algorithms did not require changes to the 
mp\_exptmod() function itself and lowered the total cost of ownership (\textit{so to speak}) and of development 
for new algorithms.  This methodology allows new algorithms to be tested in a complete framework with relative ease.

\begin{center}
\begin{figure}[here]
\includegraphics{pics/design_process.ps}
\caption{Design Flow of the First Few Original LibTomMath Functions.}
\label{pic:design_process}
\end{figure}
\end{center}

Only after the majority of the functions were in place did I pursue a less hierarchical approach to auditing and optimizing
the source code.  For example, one day I may audit the multipliers and the next day the polynomial basis functions.  

It only makes sense to begin the text with the preliminary data types and support algorithms required as well.  
This chapter discusses the core algorithms of the library which are the dependents for every other algorithm.

\section{What is a Multiple Precision Integer?}
Recall that most programming languages, in particular ISO C \cite{ISOC}, only have fixed precision data types that on their own cannot 
be used to represent values larger than their precision will allow. The purpose of multiple precision algorithms is 
to use fixed precision data types to create and manipulate multiple precision integers which may represent values 
that are very large.  

As a well known analogy, school children are taught how to form numbers larger than nine by prepending more radix ten digits.  In the decimal system
the largest single digit value is $9$.  However, by concatenating digits together larger numbers may be represented.  Newly prepended digits 
(\textit{to the left}) are said to be in a different power of ten column.  That is, the number $123$ can be described as having a $1$ in the hundreds 
column, $2$ in the tens column and $3$ in the ones column.  Or more formally $123 = 1 \cdot 10^2 + 2 \cdot 10^1 + 3 \cdot 10^0$.  Computer based 
multiple precision arithmetic is essentially the same concept.  Larger integers are represented by adjoining fixed 
precision computer words with the exception that a different radix is used.

What most people probably do not think about explicitly are the various other attributes that describe a multiple precision 
integer.  For example, the integer $154_{10}$ has two immediately obvious properties.  First, the integer is positive, 
that is the sign of this particular integer is positive as opposed to negative.  Second, the integer has three digits in 
its representation.  There is an additional property that the integer posesses that does not concern pencil-and-paper 
arithmetic.  The third property is how many digits placeholders are available to hold the integer.  

The human analogy of this third property is ensuring there is enough space on the paper to write the integer.  For example,
if one starts writing a large number too far to the right on a piece of paper they will have to erase it and move left.  
Similarly, computer algorithms must maintain strict control over memory usage to ensure that the digits of an integer
will not exceed the allowed boundaries.  These three properties make up what is known as a multiple precision 
integer or mp\_int for short.  

\subsection{The mp\_int Structure}
\label{sec:MPINT}
The mp\_int structure is the ISO C based manifestation of what represents a multiple precision integer.  The ISO C standard does not provide for 
any such data type but it does provide for making composite data types known as structures.  The following is the structure definition 
used within LibTomMath.

\index{mp\_int}
\begin{figure}[here]
\begin{center}
\begin{small}
%\begin{verbatim}
\begin{tabular}{|l|}
\hline
typedef struct \{ \\
\hspace{3mm}int used, alloc, sign;\\
\hspace{3mm}mp\_digit *dp;\\
\} \textbf{mp\_int}; \\
\hline
\end{tabular}
%\end{verbatim}
\end{small}
\caption{The mp\_int Structure}
\label{fig:mpint}
\end{center}
\end{figure}

The mp\_int structure (fig. \ref{fig:mpint}) can be broken down as follows.

\begin{enumerate}
\item The \textbf{used} parameter denotes how many digits of the array \textbf{dp} contain the digits used to represent
a given integer.  The \textbf{used} count must be positive (or zero) and may not exceed the \textbf{alloc} count.  

\item The \textbf{alloc} parameter denotes how 
many digits are available in the array to use by functions before it has to increase in size.  When the \textbf{used} count 
of a result would exceed the \textbf{alloc} count all of the algorithms will automatically increase the size of the 
array to accommodate the precision of the result.  

\item The pointer \textbf{dp} points to a dynamically allocated array of digits that represent the given multiple 
precision integer.  It is padded with $(\textbf{alloc} - \textbf{used})$ zero digits.  The array is maintained in a least 
significant digit order.  As a pencil and paper analogy the array is organized such that the right most digits are stored
first starting at the location indexed by zero\footnote{In C all arrays begin at zero.} in the array.  For example, 
if \textbf{dp} contains $\lbrace a, b, c, \ldots \rbrace$ where \textbf{dp}$_0 = a$, \textbf{dp}$_1 = b$, \textbf{dp}$_2 = c$, $\ldots$ then 
it would represent the integer $a + b\beta + c\beta^2 + \ldots$  

\index{MP\_ZPOS} \index{MP\_NEG}
\item The \textbf{sign} parameter denotes the sign as either zero/positive (\textbf{MP\_ZPOS}) or negative (\textbf{MP\_NEG}).  
\end{enumerate}

\subsubsection{Valid mp\_int Structures}
Several rules are placed on the state of an mp\_int structure and are assumed to be followed for reasons of efficiency.  
The only exceptions are when the structure is passed to initialization functions such as mp\_init() and mp\_init\_copy().

\begin{enumerate}
\item The value of \textbf{alloc} may not be less than one.  That is \textbf{dp} always points to a previously allocated
array of digits.
\item The value of \textbf{used} may not exceed \textbf{alloc} and must be greater than or equal to zero.
\item The value of \textbf{used} implies the digit at index $(used - 1)$ of the \textbf{dp} array is non-zero.  That is, 
leading zero digits in the most significant positions must be trimmed.
   \begin{enumerate}
   \item Digits in the \textbf{dp} array at and above the \textbf{used} location must be zero.
   \end{enumerate}
\item The value of \textbf{sign} must be \textbf{MP\_ZPOS} if \textbf{used} is zero; 
this represents the mp\_int value of zero.
\end{enumerate}

\section{Argument Passing}
A convention of argument passing must be adopted early on in the development of any library.  Making the function 
prototypes consistent will help eliminate many headaches in the future as the library grows to significant complexity.  
In LibTomMath the multiple precision integer functions accept parameters from left to right as pointers to mp\_int 
structures.  That means that the source (input) operands are placed on the left and the destination (output) on the right.   
Consider the following examples.

\begin{verbatim}
   mp_mul(&a, &b, &c);   /* c = a * b */
   mp_add(&a, &b, &a);   /* a = a + b */
   mp_sqr(&a, &b);       /* b = a * a */
\end{verbatim}

The left to right order is a fairly natural way to implement the functions since it lets the developer read aloud the
functions and make sense of them.  For example, the first function would read ``multiply a and b and store in c''.

Certain libraries (\textit{LIP by Lenstra for instance}) accept parameters the other way around, to mimic the order
of assignment expressions.  That is, the destination (output) is on the left and arguments (inputs) are on the right.  In 
truth, it is entirely a matter of preference.  In the case of LibTomMath the convention from the MPI library has been 
adopted.  

Another very useful design consideration, provided for in LibTomMath, is whether to allow argument sources to also be a 
destination.  For example, the second example (\textit{mp\_add}) adds $a$ to $b$ and stores in $a$.  This is an important 
feature to implement since it allows the calling functions to cut down on the number of variables it must maintain.  
However, to implement this feature specific care has to be given to ensure the destination is not modified before the 
source is fully read.

\section{Return Values}
A well implemented application, no matter what its purpose, should trap as many runtime errors as possible and return them 
to the caller.  By catching runtime errors a library can be guaranteed to prevent undefined behaviour.  However, the end 
developer can still manage to cause a library to crash.  For example, by passing an invalid pointer an application may
fault by dereferencing memory not owned by the application.

In the case of LibTomMath the only errors that are checked for are related to inappropriate inputs (division by zero for 
instance) and memory allocation errors.  It will not check that the mp\_int passed to any function is valid nor 
will it check pointers for validity.  Any function that can cause a runtime error will return an error code as an 
\textbf{int} data type with one of the following values (fig \ref{fig:errcodes}).

\index{MP\_OKAY} \index{MP\_VAL} \index{MP\_MEM}
\begin{figure}[here]
\begin{center}
\begin{tabular}{|l|l|}
\hline \textbf{Value} & \textbf{Meaning} \\
\hline \textbf{MP\_OKAY} & The function was successful \\
\hline \textbf{MP\_VAL}  & One of the input value(s) was invalid \\
\hline \textbf{MP\_MEM}  & The function ran out of heap memory \\
\hline
\end{tabular}
\end{center}
\caption{LibTomMath Error Codes}
\label{fig:errcodes}
\end{figure}

When an error is detected within a function it should free any memory it allocated, often during the initialization of
temporary mp\_ints, and return as soon as possible.  The goal is to leave the system in the same state it was when the 
function was called.  Error checking with this style of API is fairly simple.

\begin{verbatim}
   int err;
   if ((err = mp_add(&a, &b, &c)) != MP_OKAY) {
      printf("Error: %s\n", mp_error_to_string(err));
      exit(EXIT_FAILURE);
   }
\end{verbatim}

The GMP \cite{GMP} library uses C style \textit{signals} to flag errors which is of questionable use.  Not all errors are fatal 
and it was not deemed ideal by the author of LibTomMath to force developers to have signal handlers for such cases.

\section{Initialization and Clearing}
The logical starting point when actually writing multiple precision integer functions is the initialization and 
clearing of the mp\_int structures.  These two algorithms will be used by the majority of the higher level algorithms.

Given the basic mp\_int structure an initialization routine must first allocate memory to hold the digits of
the integer.  Often it is optimal to allocate a sufficiently large pre-set number of digits even though
the initial integer will represent zero.  If only a single digit were allocated quite a few subsequent re-allocations
would occur when operations are performed on the integers.  There is a tradeoff between how many default digits to allocate
and how many re-allocations are tolerable.  Obviously allocating an excessive amount of digits initially will waste 
memory and become unmanageable.  

If the memory for the digits has been successfully allocated then the rest of the members of the structure must
be initialized.  Since the initial state of an mp\_int is to represent the zero integer, the allocated digits must be set
to zero.  The \textbf{used} count set to zero and \textbf{sign} set to \textbf{MP\_ZPOS}.

\subsection{Initializing an mp\_int}
An mp\_int is said to be initialized if it is set to a valid, preferably default, state such that all of the members of the
structure are set to valid values.  The mp\_init algorithm will perform such an action.

\index{mp\_init}
\begin{figure}[here]
\begin{center}
\begin{tabular}{l}
\hline Algorithm \textbf{mp\_init}. \\
\textbf{Input}.   An mp\_int $a$ \\
\textbf{Output}.  Allocate memory and initialize $a$ to a known valid mp\_int state.  \\
\hline \\
1.  Allocate memory for \textbf{MP\_PREC} digits. \\
2.  If the allocation failed return(\textit{MP\_MEM}) \\
3.  for $n$ from $0$ to $MP\_PREC - 1$ do  \\
\hspace{3mm}3.1  $a_n \leftarrow 0$\\
4.  $a.sign \leftarrow MP\_ZPOS$\\
5.  $a.used \leftarrow 0$\\
6.  $a.alloc \leftarrow MP\_PREC$\\
7.  Return(\textit{MP\_OKAY})\\
\hline
\end{tabular}
\end{center}
\caption{Algorithm mp\_init}
\end{figure}

\textbf{Algorithm mp\_init.}
The purpose of this function is to initialize an mp\_int structure so that the rest of the library can properly
manipulte it.  It is assumed that the input may not have had any of its members previously initialized which is certainly
a valid assumption if the input resides on the stack.  

Before any of the members such as \textbf{sign}, \textbf{used} or \textbf{alloc} are initialized the memory for
the digits is allocated.  If this fails the function returns before setting any of the other members.  The \textbf{MP\_PREC} 
name represents a constant\footnote{Defined in the ``tommath.h'' header file within LibTomMath.} 
used to dictate the minimum precision of newly initialized mp\_int integers.  Ideally, it is at least equal to the smallest
precision number you'll be working with.

Allocating a block of digits at first instead of a single digit has the benefit of lowering the number of usually slow
heap operations later functions will have to perform in the future.  If \textbf{MP\_PREC} is set correctly the slack 
memory and the number of heap operations will be trivial.

Once the allocation has been made the digits have to be set to zero as well as the \textbf{used}, \textbf{sign} and
\textbf{alloc} members initialized.  This ensures that the mp\_int will always represent the default state of zero regardless
of the original condition of the input.

\textbf{Remark.}
This function introduces the idiosyncrasy that all iterative loops, commonly initiated with the ``for'' keyword, iterate incrementally
when the ``to'' keyword is placed between two expressions.  For example, ``for $a$ from $b$ to $c$ do'' means that
a subsequent expression (or body of expressions) are to be evaluated upto $c - b$ times so long as $b \le c$.  In each
iteration the variable $a$ is substituted for a new integer that lies inclusively between $b$ and $c$.  If $b > c$ occured
the loop would not iterate.  By contrast if the ``downto'' keyword were used in place of ``to'' the loop would iterate 
decrementally.

\vspace{+3mm}\begin{small}
\hspace{-5.1mm}{\bf File}: bn\_mp\_init.c
\vspace{-3mm}
\begin{alltt}
\end{alltt}
\end{small}

One immediate observation of this initializtion function is that it does not return a pointer to a mp\_int structure.  It 
is assumed that the caller has already allocated memory for the mp\_int structure, typically on the application stack.  The 
call to mp\_init() is used only to initialize the members of the structure to a known default state.  

Here we see (line 24) the memory allocation is performed first.  This allows us to exit cleanly and quickly
if there is an error.  If the allocation fails the routine will return \textbf{MP\_MEM} to the caller to indicate there
was a memory error.  The function XMALLOC is what actually allocates the memory.  Technically XMALLOC is not a function
but a macro defined in ``tommath.h``.  By default, XMALLOC will evaluate to malloc() which is the C library's built--in
memory allocation routine.

In order to assure the mp\_int is in a known state the digits must be set to zero.  On most platforms this could have been
accomplished by using calloc() instead of malloc().  However,  to correctly initialize a integer type to a given value in a 
portable fashion you have to actually assign the value.  The for loop (line 30) performs this required
operation.

After the memory has been successfully initialized the remainder of the members are initialized 
(lines 34 through 35) to their respective default states.  At this point the algorithm has succeeded and
a success code is returned to the calling function.  If this function returns \textbf{MP\_OKAY} it is safe to assume the 
mp\_int structure has been properly initialized and is safe to use with other functions within the library.  

\subsection{Clearing an mp\_int}
When an mp\_int is no longer required by the application, the memory that has been allocated for its digits must be 
returned to the application's memory pool with the mp\_clear algorithm.

\begin{figure}[here]
\begin{center}
\begin{tabular}{l}
\hline Algorithm \textbf{mp\_clear}. \\
\textbf{Input}.   An mp\_int $a$ \\
\textbf{Output}.  The memory for $a$ shall be deallocated.  \\
\hline \\
1.  If $a$ has been previously freed then return(\textit{MP\_OKAY}). \\
2.  for $n$ from 0 to $a.used - 1$ do \\
\hspace{3mm}2.1  $a_n \leftarrow 0$ \\
3.  Free the memory allocated for the digits of $a$. \\
4.  $a.used \leftarrow 0$ \\
5.  $a.alloc \leftarrow 0$ \\
6.  $a.sign \leftarrow MP\_ZPOS$ \\
7.  Return(\textit{MP\_OKAY}). \\
\hline
\end{tabular}
\end{center}
\caption{Algorithm mp\_clear}
\end{figure}

\textbf{Algorithm mp\_clear.}
This algorithm accomplishes two goals.  First, it clears the digits and the other mp\_int members.  This ensures that 
if a developer accidentally re-uses a cleared structure it is less likely to cause problems.  The second goal
is to free the allocated memory.

The logic behind the algorithm is extended by marking cleared mp\_int structures so that subsequent calls to this
algorithm will not try to free the memory multiple times.  Cleared mp\_ints are detectable by having a pre-defined invalid 
digit pointer \textbf{dp} setting.

Once an mp\_int has been cleared the mp\_int structure is no longer in a valid state for any other algorithm
with the exception of algorithms mp\_init, mp\_init\_copy, mp\_init\_size and mp\_clear.

\vspace{+3mm}\begin{small}
\hspace{-5.1mm}{\bf File}: bn\_mp\_clear.c
\vspace{-3mm}
\begin{alltt}
\end{alltt}
\end{small}

The algorithm only operates on the mp\_int if it hasn't been previously cleared.  The if statement (line 25)
checks to see if the \textbf{dp} member is not \textbf{NULL}.  If the mp\_int is a valid mp\_int then \textbf{dp} cannot be
\textbf{NULL} in which case the if statement will evaluate to true.

The digits of the mp\_int are cleared by the for loop (line 27) which assigns a zero to every digit.  Similar to mp\_init()
the digits are assigned zero instead of using block memory operations (such as memset()) since this is more portable.  

The digits are deallocated off the heap via the XFREE macro.  Similar to XMALLOC the XFREE macro actually evaluates to
a standard C library function.  In this case the free() function.  Since free() only deallocates the memory the pointer
still has to be reset to \textbf{NULL} manually (line 35).  

Now that the digits have been cleared and deallocated the other members are set to their final values (lines 36 and 37).

\section{Maintenance Algorithms}

The previous sections describes how to initialize and clear an mp\_int structure.  To further support operations
that are to be performed on mp\_int structures (such as addition and multiplication) the dependent algorithms must be
able to augment the precision of an mp\_int and 
initialize mp\_ints with differing initial conditions.  

These algorithms complete the set of low level algorithms required to work with mp\_int structures in the higher level
algorithms such as addition, multiplication and modular exponentiation.

\subsection{Augmenting an mp\_int's Precision}
When storing a value in an mp\_int structure, a sufficient number of digits must be available to accomodate the entire 
result of an operation without loss of precision.  Quite often the size of the array given by the \textbf{alloc} member 
is large enough to simply increase the \textbf{used} digit count.  However, when the size of the array is too small it 
must be re-sized appropriately to accomodate the result.  The mp\_grow algorithm will provide this functionality.

\newpage\begin{figure}[here]
\begin{center}
\begin{tabular}{l}
\hline Algorithm \textbf{mp\_grow}. \\
\textbf{Input}.   An mp\_int $a$ and an integer $b$. \\
\textbf{Output}.  $a$ is expanded to accomodate $b$ digits. \\
\hline \\
1.  if $a.alloc \ge b$ then return(\textit{MP\_OKAY}) \\
2.  $u \leftarrow b\mbox{ (mod }MP\_PREC\mbox{)}$ \\
3.  $v \leftarrow b + 2 \cdot MP\_PREC - u$ \\
4.  Re-allocate the array of digits $a$ to size $v$ \\
5.  If the allocation failed then return(\textit{MP\_MEM}). \\
6.  for n from a.alloc to $v - 1$ do  \\
\hspace{+3mm}6.1  $a_n \leftarrow 0$ \\
7.  $a.alloc \leftarrow v$ \\
8.  Return(\textit{MP\_OKAY}) \\
\hline
\end{tabular}
\end{center}
\caption{Algorithm mp\_grow}
\end{figure}

\textbf{Algorithm mp\_grow.}
It is ideal to prevent re-allocations from being performed if they are not required (step one).  This is useful to 
prevent mp\_ints from growing excessively in code that erroneously calls mp\_grow.  

The requested digit count is padded up to next multiple of \textbf{MP\_PREC} plus an additional \textbf{MP\_PREC} (steps two and three).  
This helps prevent many trivial reallocations that would grow an mp\_int by trivially small values.  

It is assumed that the reallocation (step four) leaves the lower $a.alloc$ digits of the mp\_int intact.  This is much 
akin to how the \textit{realloc} function from the standard C library works.  Since the newly allocated digits are 
assumed to contain undefined values they are initially set to zero.

\vspace{+3mm}\begin{small}
\hspace{-5.1mm}{\bf File}: bn\_mp\_grow.c
\vspace{-3mm}
\begin{alltt}
\end{alltt}
\end{small}

A quick optimization is to first determine if a memory re-allocation is required at all.  The if statement (line 24) checks
if the \textbf{alloc} member of the mp\_int is smaller than the requested digit count.  If the count is not larger than \textbf{alloc}
the function skips the re-allocation part thus saving time.

When a re-allocation is performed it is turned into an optimal request to save time in the future.  The requested digit count is
padded upwards to 2nd multiple of \textbf{MP\_PREC} larger than \textbf{alloc} (line 25).  The XREALLOC function is used
to re-allocate the memory.  As per the other functions XREALLOC is actually a macro which evaluates to realloc by default.  The realloc
function leaves the base of the allocation intact which means the first \textbf{alloc} digits of the mp\_int are the same as before
the re-allocation.  All	that is left is to clear the newly allocated digits and return.

Note that the re-allocation result is actually stored in a temporary pointer $tmp$.  This is to allow this function to return
an error with a valid pointer.  Earlier releases of the library stored the result of XREALLOC into the mp\_int $a$.  That would
result in a memory leak if XREALLOC ever failed.  

\subsection{Initializing Variable Precision mp\_ints}
Occasionally the number of digits required will be known in advance of an initialization, based on, for example, the size 
of input mp\_ints to a given algorithm.  The purpose of algorithm mp\_init\_size is similar to mp\_init except that it 
will allocate \textit{at least} a specified number of digits.  

\begin{figure}[here]
\begin{small}
\begin{center}
\begin{tabular}{l}
\hline Algorithm \textbf{mp\_init\_size}. \\
\textbf{Input}.   An mp\_int $a$ and the requested number of digits $b$. \\
\textbf{Output}.  $a$ is initialized to hold at least $b$ digits. \\
\hline \\
1.  $u \leftarrow b \mbox{ (mod }MP\_PREC\mbox{)}$ \\
2.  $v \leftarrow b + 2 \cdot MP\_PREC - u$ \\
3.  Allocate $v$ digits. \\
4.  for $n$ from $0$ to $v - 1$ do \\
\hspace{3mm}4.1  $a_n \leftarrow 0$ \\
5.  $a.sign \leftarrow MP\_ZPOS$\\
6.  $a.used \leftarrow 0$\\
7.  $a.alloc \leftarrow v$\\
8.  Return(\textit{MP\_OKAY})\\
\hline
\end{tabular}
\end{center}
\end{small}
\caption{Algorithm mp\_init\_size}
\end{figure}

\textbf{Algorithm mp\_init\_size.}
This algorithm will initialize an mp\_int structure $a$ like algorithm mp\_init with the exception that the number of 
digits allocated can be controlled by the second input argument $b$.  The input size is padded upwards so it is a 
multiple of \textbf{MP\_PREC} plus an additional \textbf{MP\_PREC} digits.  This padding is used to prevent trivial 
allocations from becoming a bottleneck in the rest of the algorithms.

Like algorithm mp\_init, the mp\_int structure is initialized to a default state representing the integer zero.  This 
particular algorithm is useful if it is known ahead of time the approximate size of the input.  If the approximation is
correct no further memory re-allocations are required to work with the mp\_int.

\vspace{+3mm}\begin{small}
\hspace{-5.1mm}{\bf File}: bn\_mp\_init\_size.c
\vspace{-3mm}
\begin{alltt}
\end{alltt}
\end{small}

The number of digits $b$ requested is padded (line 24) by first augmenting it to the next multiple of 
\textbf{MP\_PREC} and then adding \textbf{MP\_PREC} to the result.  If the memory can be successfully allocated the 
mp\_int is placed in a default state representing the integer zero.  Otherwise, the error code \textbf{MP\_MEM} will be 
returned (line 29).  

The digits are allocated and set to zero at the same time with the calloc() function (line @25,XCALLOC@).  The 
\textbf{used} count is set to zero, the \textbf{alloc} count set to the padded digit count and the \textbf{sign} flag set 
to \textbf{MP\_ZPOS} to achieve a default valid mp\_int state (lines 33, 34 and 35).  If the function 
returns succesfully then it is correct to assume that the mp\_int structure is in a valid state for the remainder of the 
functions to work with.

\subsection{Multiple Integer Initializations and Clearings}
Occasionally a function will require a series of mp\_int data types to be made available simultaneously.  
The purpose of algorithm mp\_init\_multi is to initialize a variable length array of mp\_int structures in a single
statement.  It is essentially a shortcut to multiple initializations.

\newpage\begin{figure}[here]
\begin{center}
\begin{tabular}{l}
\hline Algorithm \textbf{mp\_init\_multi}. \\
\textbf{Input}.   Variable length array $V_k$ of mp\_int variables of length $k$. \\
\textbf{Output}.  The array is initialized such that each mp\_int of $V_k$ is ready to use. \\
\hline \\
1.  for $n$ from 0 to $k - 1$ do \\
\hspace{+3mm}1.1.  Initialize the mp\_int $V_n$ (\textit{mp\_init}) \\
\hspace{+3mm}1.2.  If initialization failed then do \\
\hspace{+6mm}1.2.1.  for $j$ from $0$ to $n$ do \\
\hspace{+9mm}1.2.1.1.  Free the mp\_int $V_j$ (\textit{mp\_clear}) \\
\hspace{+6mm}1.2.2.   Return(\textit{MP\_MEM}) \\
2.  Return(\textit{MP\_OKAY}) \\
\hline
\end{tabular}
\end{center}
\caption{Algorithm mp\_init\_multi}
\end{figure}

\textbf{Algorithm mp\_init\_multi.}
The algorithm will initialize the array of mp\_int variables one at a time.  If a runtime error has been detected 
(\textit{step 1.2}) all of the previously initialized variables are cleared.  The goal is an ``all or nothing'' 
initialization which allows for quick recovery from runtime errors.

\vspace{+3mm}\begin{small}
\hspace{-5.1mm}{\bf File}: bn\_mp\_init\_multi.c
\vspace{-3mm}
\begin{alltt}
\end{alltt}
\end{small}

This function intializes a variable length list of mp\_int structure pointers.  However, instead of having the mp\_int
structures in an actual C array they are simply passed as arguments to the function.  This function makes use of the 
``...'' argument syntax of the C programming language.  The list is terminated with a final \textbf{NULL} argument 
appended on the right.  

The function uses the ``stdarg.h'' \textit{va} functions to step portably through the arguments to the function.  A count
$n$ of succesfully initialized mp\_int structures is maintained (line 48) such that if a failure does occur,
the algorithm can backtrack and free the previously initialized structures (lines 28 to 47).  


\subsection{Clamping Excess Digits}
When a function anticipates a result will be $n$ digits it is simpler to assume this is true within the body of 
the function instead of checking during the computation.  For example, a multiplication of a $i$ digit number by a 
$j$ digit produces a result of at most $i + j$ digits.  It is entirely possible that the result is $i + j - 1$ 
though, with no final carry into the last position.  However, suppose the destination had to be first expanded 
(\textit{via mp\_grow}) to accomodate $i + j - 1$ digits than further expanded to accomodate the final carry.  
That would be a considerable waste of time since heap operations are relatively slow.

The ideal solution is to always assume the result is $i + j$ and fix up the \textbf{used} count after the function
terminates.  This way a single heap operation (\textit{at most}) is required.  However, if the result was not checked
there would be an excess high order zero digit.  

For example, suppose the product of two integers was $x_n = (0x_{n-1}x_{n-2}...x_0)_{\beta}$.  The leading zero digit 
will not contribute to the precision of the result.  In fact, through subsequent operations more leading zero digits would
accumulate to the point the size of the integer would be prohibitive.  As a result even though the precision is very 
low the representation is excessively large.  

The mp\_clamp algorithm is designed to solve this very problem.  It will trim high-order zeros by decrementing the 
\textbf{used} count until a non-zero most significant digit is found.  Also in this system, zero is considered to be a 
positive number which means that if the \textbf{used} count is decremented to zero, the sign must be set to 
\textbf{MP\_ZPOS}.

\begin{figure}[here]
\begin{center}
\begin{tabular}{l}
\hline Algorithm \textbf{mp\_clamp}. \\
\textbf{Input}.   An mp\_int $a$ \\
\textbf{Output}.  Any excess leading zero digits of $a$ are removed \\
\hline \\
1.  while $a.used > 0$ and $a_{a.used - 1} = 0$ do \\
\hspace{+3mm}1.1  $a.used \leftarrow a.used - 1$ \\
2.  if $a.used = 0$ then do \\
\hspace{+3mm}2.1  $a.sign \leftarrow MP\_ZPOS$ \\
\hline \\
\end{tabular}
\end{center}
\caption{Algorithm mp\_clamp}
\end{figure}

\textbf{Algorithm mp\_clamp.}
As can be expected this algorithm is very simple.  The loop on step one is expected to iterate only once or twice at
the most.  For example, this will happen in cases where there is not a carry to fill the last position.  Step two fixes the sign for 
when all of the digits are zero to ensure that the mp\_int is valid at all times.

\vspace{+3mm}\begin{small}
\hspace{-5.1mm}{\bf File}: bn\_mp\_clamp.c
\vspace{-3mm}
\begin{alltt}
\end{alltt}
\end{small}

Note on line 28 how to test for the \textbf{used} count is made on the left of the \&\& operator.  In the C programming
language the terms to \&\& are evaluated left to right with a boolean short-circuit if any condition fails.  This is 
important since if the \textbf{used} is zero the test on the right would fetch below the array.  That is obviously 
undesirable.  The parenthesis on line 31 is used to make sure the \textbf{used} count is decremented and not
the pointer ``a''.  

\section*{Exercises}
\begin{tabular}{cl}
$\left [ 1 \right ]$ & Discuss the relevance of the \textbf{used} member of the mp\_int structure. \\
                     & \\
$\left [ 1 \right ]$ & Discuss the consequences of not using padding when performing allocations.  \\
                     & \\
$\left [ 2 \right ]$ & Estimate an ideal value for \textbf{MP\_PREC} when performing 1024-bit RSA \\
                     & encryption when $\beta = 2^{28}$.  \\
                     & \\
$\left [ 1 \right ]$ & Discuss the relevance of the algorithm mp\_clamp.  What does it prevent? \\
                     & \\
$\left [ 1 \right ]$ & Give an example of when the algorithm  mp\_init\_copy might be useful. \\
                     & \\
\end{tabular}


%%%
% CHAPTER FOUR
%%%

\chapter{Basic Operations}

\section{Introduction}
In the previous chapter a series of low level algorithms were established that dealt with initializing and maintaining
mp\_int structures.  This chapter will discuss another set of seemingly non-algebraic algorithms which will form the low 
level basis of the entire library.  While these algorithm are relatively trivial it is important to understand how they
work before proceeding since these algorithms will be used almost intrinsically in the following chapters.

The algorithms in this chapter deal primarily with more ``programmer'' related tasks such as creating copies of
mp\_int structures, assigning small values to mp\_int structures and comparisons of the values mp\_int structures
represent.   

\section{Assigning Values to mp\_int Structures}
\subsection{Copying an mp\_int}
Assigning the value that a given mp\_int structure represents to another mp\_int structure shall be known as making
a copy for the purposes of this text.  The copy of the mp\_int will be a separate entity that represents the same
value as the mp\_int it was copied from.  The mp\_copy algorithm provides this functionality. 

\newpage\begin{figure}[here]
\begin{center}
\begin{tabular}{l}
\hline Algorithm \textbf{mp\_copy}. \\
\textbf{Input}.  An mp\_int $a$ and $b$. \\
\textbf{Output}.  Store a copy of $a$ in $b$. \\
\hline \\
1.  If $b.alloc < a.used$ then grow $b$ to $a.used$ digits.  (\textit{mp\_grow}) \\
2.  for $n$ from 0 to $a.used - 1$ do \\
\hspace{3mm}2.1  $b_{n} \leftarrow a_{n}$ \\
3.  for $n$ from $a.used$ to $b.used - 1$ do \\
\hspace{3mm}3.1  $b_{n} \leftarrow 0$ \\
4.  $b.used \leftarrow a.used$ \\
5.  $b.sign \leftarrow a.sign$ \\
6.  return(\textit{MP\_OKAY}) \\
\hline
\end{tabular}
\end{center}
\caption{Algorithm mp\_copy}
\end{figure}

\textbf{Algorithm mp\_copy.}
This algorithm copies the mp\_int $a$ such that upon succesful termination of the algorithm the mp\_int $b$ will
represent the same integer as the mp\_int $a$.  The mp\_int $b$ shall be a complete and distinct copy of the 
mp\_int $a$ meaing that the mp\_int $a$ can be modified and it shall not affect the value of the mp\_int $b$.

If $b$ does not have enough room for the digits of $a$ it must first have its precision augmented via the mp\_grow 
algorithm.  The digits of $a$ are copied over the digits of $b$ and any excess digits of $b$ are set to zero (step two
and three).  The \textbf{used} and \textbf{sign} members of $a$ are finally copied over the respective members of
$b$.

\textbf{Remark.}  This algorithm also introduces a new idiosyncrasy that will be used throughout the rest of the
text.  The error return codes of other algorithms are not explicitly checked in the pseudo-code presented.  For example, in 
step one of the mp\_copy algorithm the return of mp\_grow is not explicitly checked to ensure it succeeded.  Text space is 
limited so it is assumed that if a algorithm fails it will clear all temporarily allocated mp\_ints and return
the error code itself.  However, the C code presented will demonstrate all of the error handling logic required to 
implement the pseudo-code.

\vspace{+3mm}\begin{small}
\hspace{-5.1mm}{\bf File}: bn\_mp\_copy.c
\vspace{-3mm}
\begin{alltt}
\end{alltt}
\end{small}

Occasionally a dependent algorithm may copy an mp\_int effectively into itself such as when the input and output
mp\_int structures passed to a function are one and the same.  For this case it is optimal to return immediately without 
copying digits (line 25).  

The mp\_int $b$ must have enough digits to accomodate the used digits of the mp\_int $a$.  If $b.alloc$ is less than
$a.used$ the algorithm mp\_grow is used to augment the precision of $b$ (lines 30 to 33).  In order to
simplify the inner loop that copies the digits from $a$ to $b$, two aliases $tmpa$ and $tmpb$ point directly at the digits
of the mp\_ints $a$ and $b$ respectively.  These aliases (lines 43 and 46) allow the compiler to access the digits without first dereferencing the
mp\_int pointers and then subsequently the pointer to the digits.  

After the aliases are established the digits from $a$ are copied into $b$ (lines 49 to 51) and then the excess 
digits of $b$ are set to zero (lines 54 to 56).  Both ``for'' loops make use of the pointer aliases and in 
fact the alias for $b$ is carried through into the second ``for'' loop to clear the excess digits.  This optimization 
allows the alias to stay in a machine register fairly easy between the two loops.

\textbf{Remarks.}  The use of pointer aliases is an implementation methodology first introduced in this function that will
be used considerably in other functions.  Technically, a pointer alias is simply a short hand alias used to lower the 
number of pointer dereferencing operations required to access data.  For example, a for loop may resemble

\begin{alltt}
for (x = 0; x < 100; x++) \{
    a->num[4]->dp[x] = 0;
\}
\end{alltt}

This could be re-written using aliases as 

\begin{alltt}
mp_digit *tmpa;
a = a->num[4]->dp;
for (x = 0; x < 100; x++) \{
    *a++ = 0;
\}
\end{alltt}

In this case an alias is used to access the 
array of digits within an mp\_int structure directly.  It may seem that a pointer alias is strictly not required 
as a compiler may optimize out the redundant pointer operations.  However, there are two dominant reasons to use aliases.

The first reason is that most compilers will not effectively optimize pointer arithmetic.  For example, some optimizations 
may work for the Microsoft Visual C++ compiler (MSVC) and not for the GNU C Compiler (GCC).  Also some optimizations may 
work for GCC and not MSVC.  As such it is ideal to find a common ground for as many compilers as possible.  Pointer 
aliases optimize the code considerably before the compiler even reads the source code which means the end compiled code 
stands a better chance of being faster.

The second reason is that pointer aliases often can make an algorithm simpler to read.  Consider the first ``for'' 
loop of the function mp\_copy() re-written to not use pointer aliases.

\begin{alltt}
    /* copy all the digits */
    for (n = 0; n < a->used; n++) \{
      b->dp[n] = a->dp[n];
    \}
\end{alltt}

Whether this code is harder to read depends strongly on the individual.  However, it is quantifiably slightly more 
complicated as there are four variables within the statement instead of just two.

\subsubsection{Nested Statements}
Another commonly used technique in the source routines is that certain sections of code are nested.  This is used in
particular with the pointer aliases to highlight code phases.  For example, a Comba multiplier (discussed in chapter six)
will typically have three different phases.  First the temporaries are initialized, then the columns calculated and 
finally the carries are propagated.  In this example the middle column production phase will typically be nested as it
uses temporary variables and aliases the most.

The nesting also simplies the source code as variables that are nested are only valid for their scope.  As a result
the various temporary variables required do not propagate into other sections of code.


\subsection{Creating a Clone}
Another common operation is to make a local temporary copy of an mp\_int argument.  To initialize an mp\_int 
and then copy another existing mp\_int into the newly intialized mp\_int will be known as creating a clone.  This is 
useful within functions that need to modify an argument but do not wish to actually modify the original copy.  The 
mp\_init\_copy algorithm has been designed to help perform this task.

\begin{figure}[here]
\begin{center}
\begin{tabular}{l}
\hline Algorithm \textbf{mp\_init\_copy}. \\
\textbf{Input}.   An mp\_int $a$ and $b$\\
\textbf{Output}.  $a$ is initialized to be a copy of $b$. \\
\hline \\
1.  Init $a$.  (\textit{mp\_init}) \\
2.  Copy $b$ to $a$.  (\textit{mp\_copy}) \\
3.  Return the status of the copy operation. \\
\hline
\end{tabular}
\end{center}
\caption{Algorithm mp\_init\_copy}
\end{figure}

\textbf{Algorithm mp\_init\_copy.}
This algorithm will initialize an mp\_int variable and copy another previously initialized mp\_int variable into it.  As 
such this algorithm will perform two operations in one step.  

\vspace{+3mm}\begin{small}
\hspace{-5.1mm}{\bf File}: bn\_mp\_init\_copy.c
\vspace{-3mm}
\begin{alltt}
\end{alltt}
\end{small}

This will initialize \textbf{a} and make it a verbatim copy of the contents of \textbf{b}.  Note that 
\textbf{a} will have its own memory allocated which means that \textbf{b} may be cleared after the call
and \textbf{a} will be left intact.  

\section{Zeroing an Integer}
Reseting an mp\_int to the default state is a common step in many algorithms.  The mp\_zero algorithm will be the algorithm used to
perform this task.

\begin{figure}[here]
\begin{center}
\begin{tabular}{l}
\hline Algorithm \textbf{mp\_zero}. \\
\textbf{Input}.   An mp\_int $a$ \\
\textbf{Output}.  Zero the contents of $a$ \\
\hline \\
1.  $a.used \leftarrow 0$ \\
2.  $a.sign \leftarrow$ MP\_ZPOS \\
3.  for $n$ from 0 to $a.alloc - 1$ do \\
\hspace{3mm}3.1  $a_n \leftarrow 0$ \\
\hline
\end{tabular}
\end{center}
\caption{Algorithm mp\_zero}
\end{figure}

\textbf{Algorithm mp\_zero.}
This algorithm simply resets a mp\_int to the default state.  

\vspace{+3mm}\begin{small}
\hspace{-5.1mm}{\bf File}: bn\_mp\_zero.c
\vspace{-3mm}
\begin{alltt}
\end{alltt}
\end{small}

After the function is completed, all of the digits are zeroed, the \textbf{used} count is zeroed and the 
\textbf{sign} variable is set to \textbf{MP\_ZPOS}.

\section{Sign Manipulation}
\subsection{Absolute Value}
With the mp\_int representation of an integer, calculating the absolute value is trivial.  The mp\_abs algorithm will compute
the absolute value of an mp\_int.

\begin{figure}[here]
\begin{center}
\begin{tabular}{l}
\hline Algorithm \textbf{mp\_abs}. \\
\textbf{Input}.   An mp\_int $a$ \\
\textbf{Output}.  Computes $b = \vert a \vert$ \\
\hline \\
1.  Copy $a$ to $b$.  (\textit{mp\_copy}) \\
2.  If the copy failed return(\textit{MP\_MEM}). \\
3.  $b.sign \leftarrow MP\_ZPOS$ \\
4.  Return(\textit{MP\_OKAY}) \\
\hline
\end{tabular}
\end{center}
\caption{Algorithm mp\_abs}
\end{figure}

\textbf{Algorithm mp\_abs.}
This algorithm computes the absolute of an mp\_int input.  First it copies $a$ over $b$.  This is an example of an
algorithm where the check in mp\_copy that determines if the source and destination are equal proves useful.  This allows,
for instance, the developer to pass the same mp\_int as the source and destination to this function without addition 
logic to handle it.

\vspace{+3mm}\begin{small}
\hspace{-5.1mm}{\bf File}: bn\_mp\_abs.c
\vspace{-3mm}
\begin{alltt}
\end{alltt}
\end{small}

This fairly trivial algorithm first eliminates non--required duplications (line 28) and then sets the
\textbf{sign} flag to \textbf{MP\_ZPOS}.

\subsection{Integer Negation}
With the mp\_int representation of an integer, calculating the negation is also trivial.  The mp\_neg algorithm will compute
the negative of an mp\_int input.

\begin{figure}[here]
\begin{center}
\begin{tabular}{l}
\hline Algorithm \textbf{mp\_neg}. \\
\textbf{Input}.   An mp\_int $a$ \\
\textbf{Output}.  Computes $b = -a$ \\
\hline \\
1.  Copy $a$ to $b$.  (\textit{mp\_copy}) \\
2.  If the copy failed return(\textit{MP\_MEM}). \\
3.  If $a.used = 0$ then return(\textit{MP\_OKAY}). \\
4.  If $a.sign = MP\_ZPOS$ then do \\
\hspace{3mm}4.1  $b.sign = MP\_NEG$. \\
5.  else do \\
\hspace{3mm}5.1  $b.sign = MP\_ZPOS$. \\
6.  Return(\textit{MP\_OKAY}) \\
\hline
\end{tabular}
\end{center}
\caption{Algorithm mp\_neg}
\end{figure}

\textbf{Algorithm mp\_neg.}
This algorithm computes the negation of an input.  First it copies $a$ over $b$.  If $a$ has no used digits then
the algorithm returns immediately.  Otherwise it flips the sign flag and stores the result in $b$.  Note that if 
$a$ had no digits then it must be positive by definition.  Had step three been omitted then the algorithm would return
zero as negative.

\vspace{+3mm}\begin{small}
\hspace{-5.1mm}{\bf File}: bn\_mp\_neg.c
\vspace{-3mm}
\begin{alltt}
\end{alltt}
\end{small}

Like mp\_abs() this function avoids non--required duplications (line 22) and then sets the sign.  We
have to make sure that only non--zero values get a \textbf{sign} of \textbf{MP\_NEG}.  If the mp\_int is zero
than the \textbf{sign} is hard--coded to \textbf{MP\_ZPOS}.

\section{Small Constants}
\subsection{Setting Small Constants}
Often a mp\_int must be set to a relatively small value such as $1$ or $2$.  For these cases the mp\_set algorithm is useful.

\newpage\begin{figure}[here]
\begin{center}
\begin{tabular}{l}
\hline Algorithm \textbf{mp\_set}. \\
\textbf{Input}.   An mp\_int $a$ and a digit $b$ \\
\textbf{Output}.  Make $a$ equivalent to $b$ \\
\hline \\
1.  Zero $a$ (\textit{mp\_zero}). \\
2.  $a_0 \leftarrow b \mbox{ (mod }\beta\mbox{)}$ \\
3.  $a.used \leftarrow  \left \lbrace \begin{array}{ll}
                              1 &  \mbox{if }a_0 > 0 \\
                              0 &  \mbox{if }a_0 = 0 
                              \end{array} \right .$ \\
\hline                              
\end{tabular}
\end{center}
\caption{Algorithm mp\_set}
\end{figure}

\textbf{Algorithm mp\_set.}
This algorithm sets a mp\_int to a small single digit value.  Step number 1 ensures that the integer is reset to the default state.  The
single digit is set (\textit{modulo $\beta$}) and the \textbf{used} count is adjusted accordingly.

\vspace{+3mm}\begin{small}
\hspace{-5.1mm}{\bf File}: bn\_mp\_set.c
\vspace{-3mm}
\begin{alltt}
\end{alltt}
\end{small}

First we zero (line 21) the mp\_int to make sure that the other members are initialized for a 
small positive constant.  mp\_zero() ensures that the \textbf{sign} is positive and the \textbf{used} count
is zero.  Next we set the digit and reduce it modulo $\beta$ (line 22).  After this step we have to 
check if the resulting digit is zero or not.  If it is not then we set the \textbf{used} count to one, otherwise
to zero.

We can quickly reduce modulo $\beta$ since it is of the form $2^k$ and a quick binary AND operation with 
$2^k - 1$ will perform the same operation.

One important limitation of this function is that it will only set one digit.  The size of a digit is not fixed, meaning source that uses 
this function should take that into account.  Only trivially small constants can be set using this function.

\subsection{Setting Large Constants}
To overcome the limitations of the mp\_set algorithm the mp\_set\_int algorithm is ideal.  It accepts a ``long''
data type as input and will always treat it as a 32-bit integer.

\begin{figure}[here]
\begin{center}
\begin{tabular}{l}
\hline Algorithm \textbf{mp\_set\_int}. \\
\textbf{Input}.   An mp\_int $a$ and a ``long'' integer $b$ \\
\textbf{Output}.  Make $a$ equivalent to $b$ \\
\hline \\
1.  Zero $a$ (\textit{mp\_zero}) \\
2.  for $n$ from 0 to 7 do \\
\hspace{3mm}2.1  $a \leftarrow a \cdot 16$ (\textit{mp\_mul2d}) \\
\hspace{3mm}2.2  $u \leftarrow \lfloor b / 2^{4(7 - n)} \rfloor \mbox{ (mod }16\mbox{)}$\\
\hspace{3mm}2.3  $a_0 \leftarrow a_0 + u$ \\
\hspace{3mm}2.4  $a.used \leftarrow a.used + 1$ \\
3.  Clamp excess used digits (\textit{mp\_clamp}) \\
\hline
\end{tabular}
\end{center}
\caption{Algorithm mp\_set\_int}
\end{figure}

\textbf{Algorithm mp\_set\_int.}
The algorithm performs eight iterations of a simple loop where in each iteration four bits from the source are added to the 
mp\_int.  Step 2.1 will multiply the current result by sixteen making room for four more bits in the less significant positions.  In step 2.2 the
next four bits from the source are extracted and are added to the mp\_int. The \textbf{used} digit count is 
incremented to reflect the addition.  The \textbf{used} digit counter is incremented since if any of the leading digits were zero the mp\_int would have
zero digits used and the newly added four bits would be ignored.

Excess zero digits are trimmed in steps 2.1 and 3 by using higher level algorithms mp\_mul2d and mp\_clamp.

\vspace{+3mm}\begin{small}
\hspace{-5.1mm}{\bf File}: bn\_mp\_set\_int.c
\vspace{-3mm}
\begin{alltt}
\end{alltt}
\end{small}

This function sets four bits of the number at a time to handle all practical \textbf{DIGIT\_BIT} sizes.  The weird
addition on line 39 ensures that the newly added in bits are added to the number of digits.  While it may not 
seem obvious as to why the digit counter does not grow exceedingly large it is because of the shift on line 28 
as well as the  call to mp\_clamp() on line 41.  Both functions will clamp excess leading digits which keeps 
the number of used digits low.

\section{Comparisons}
\subsection{Unsigned Comparisions}
Comparing a multiple precision integer is performed with the exact same algorithm used to compare two decimal numbers.  For example,
to compare $1,234$ to $1,264$ the digits are extracted by their positions.  That is we compare $1 \cdot 10^3 + 2 \cdot 10^2 + 3 \cdot 10^1 + 4 \cdot 10^0$
to $1 \cdot 10^3 + 2 \cdot 10^2 + 6 \cdot 10^1 + 4 \cdot 10^0$ by comparing single digits at a time starting with the highest magnitude 
positions.  If any leading digit of one integer is greater than a digit in the same position of another integer then obviously it must be greater.  

The first comparision routine that will be developed is the unsigned magnitude compare which will perform a comparison based on the digits of two
mp\_int variables alone.  It will ignore the sign of the two inputs.  Such a function is useful when an absolute comparison is required or if the 
signs are known to agree in advance.

To facilitate working with the results of the comparison functions three constants are required.  

\begin{figure}[here]
\begin{center}
\begin{tabular}{|r|l|}
\hline \textbf{Constant} & \textbf{Meaning} \\
\hline \textbf{MP\_GT} & Greater Than \\
\hline \textbf{MP\_EQ} & Equal To \\
\hline \textbf{MP\_LT} & Less Than \\
\hline
\end{tabular}
\end{center}
\caption{Comparison Return Codes}
\end{figure}

\begin{figure}[here]
\begin{center}
\begin{tabular}{l}
\hline Algorithm \textbf{mp\_cmp\_mag}. \\
\textbf{Input}.   Two mp\_ints $a$ and $b$.  \\
\textbf{Output}.  Unsigned comparison results ($a$ to the left of $b$). \\
\hline \\
1.  If $a.used > b.used$ then return(\textit{MP\_GT}) \\
2.  If $a.used < b.used$ then return(\textit{MP\_LT}) \\
3.  for n from $a.used - 1$ to 0 do \\
\hspace{+3mm}3.1  if $a_n > b_n$ then return(\textit{MP\_GT}) \\
\hspace{+3mm}3.2  if $a_n < b_n$ then return(\textit{MP\_LT}) \\
4.  Return(\textit{MP\_EQ}) \\
\hline
\end{tabular}
\end{center}
\caption{Algorithm mp\_cmp\_mag}
\end{figure}

\textbf{Algorithm mp\_cmp\_mag.}
By saying ``$a$ to the left of $b$'' it is meant that the comparison is with respect to $a$, that is if $a$ is greater than $b$ it will return
\textbf{MP\_GT} and similar with respect to when $a = b$ and $a < b$.  The first two steps compare the number of digits used in both $a$ and $b$.  
Obviously if the digit counts differ there would be an imaginary zero digit in the smaller number where the leading digit of the larger number is.  
If both have the same number of digits than the actual digits themselves must be compared starting at the leading digit.  

By step three both inputs must have the same number of digits so its safe to start from either $a.used - 1$ or $b.used - 1$ and count down to
the zero'th digit.  If after all of the digits have been compared, no difference is found, the algorithm returns \textbf{MP\_EQ}.

\vspace{+3mm}\begin{small}
\hspace{-5.1mm}{\bf File}: bn\_mp\_cmp\_mag.c
\vspace{-3mm}
\begin{alltt}
\end{alltt}
\end{small}

The two if statements (lines 25 and 29) compare the number of digits in the two inputs.  These two are 
performed before all of the digits are compared since it is a very cheap test to perform and can potentially save 
considerable time.  The implementation given is also not valid without those two statements.  $b.alloc$ may be 
smaller than $a.used$, meaning that undefined values will be read from $b$ past the end of the array of digits.



\subsection{Signed Comparisons}
Comparing with sign considerations is also fairly critical in several routines (\textit{division for example}).  Based on an unsigned magnitude 
comparison a trivial signed comparison algorithm can be written.

\begin{figure}[here]
\begin{center}
\begin{tabular}{l}
\hline Algorithm \textbf{mp\_cmp}. \\
\textbf{Input}.   Two mp\_ints $a$ and $b$ \\
\textbf{Output}.  Signed Comparison Results ($a$ to the left of $b$) \\
\hline \\
1.  if $a.sign = MP\_NEG$ and $b.sign = MP\_ZPOS$ then return(\textit{MP\_LT}) \\
2.  if $a.sign = MP\_ZPOS$ and $b.sign = MP\_NEG$ then return(\textit{MP\_GT}) \\
3.  if $a.sign = MP\_NEG$ then \\
\hspace{+3mm}3.1  Return the unsigned comparison of $b$ and $a$ (\textit{mp\_cmp\_mag}) \\
4   Otherwise \\
\hspace{+3mm}4.1  Return the unsigned comparison of $a$ and $b$ \\
\hline
\end{tabular}
\end{center}
\caption{Algorithm mp\_cmp}
\end{figure}

\textbf{Algorithm mp\_cmp.}
The first two steps compare the signs of the two inputs.  If the signs do not agree then it can return right away with the appropriate 
comparison code.  When the signs are equal the digits of the inputs must be compared to determine the correct result.  In step 
three the unsigned comparision flips the order of the arguments since they are both negative.  For instance, if $-a > -b$ then 
$\vert a \vert < \vert b \vert$.  Step number four will compare the two when they are both positive.

\vspace{+3mm}\begin{small}
\hspace{-5.1mm}{\bf File}: bn\_mp\_cmp.c
\vspace{-3mm}
\begin{alltt}
\end{alltt}
\end{small}

The two if statements (lines 23 and 24) perform the initial sign comparison.  If the signs are not the equal then which ever
has the positive sign is larger.   The inputs are compared (line 32) based on magnitudes.  If the signs were both 
negative then the unsigned comparison is performed in the opposite direction (line 34).  Otherwise, the signs are assumed to 
be both positive and a forward direction unsigned comparison is performed.

\section*{Exercises}
\begin{tabular}{cl}
$\left [ 2 \right ]$ & Modify algorithm mp\_set\_int to accept as input a variable length array of bits. \\
                     & \\
$\left [ 3 \right ]$ & Give the probability that algorithm mp\_cmp\_mag will have to compare $k$ digits  \\
                     & of two random digits (of equal magnitude) before a difference is found. \\
                     & \\
$\left [ 1 \right ]$ & Suggest a simple method to speed up the implementation of mp\_cmp\_mag based  \\
                     & on the observations made in the previous problem. \\
                     &
\end{tabular}

\chapter{Basic Arithmetic}
\section{Introduction}
At this point algorithms for initialization, clearing, zeroing, copying, comparing and setting small constants have been 
established.  The next logical set of algorithms to develop are addition, subtraction and digit shifting algorithms.  These 
algorithms make use of the lower level algorithms and are the cruicial building block for the multiplication algorithms.  It is very important 
that these algorithms are highly optimized.  On their own they are simple $O(n)$ algorithms but they can be called from higher level algorithms 
which easily places them at $O(n^2)$ or even $O(n^3)$ work levels.  

All of the algorithms within this chapter make use of the logical bit shift operations denoted by $<<$ and $>>$ for left and right 
logical shifts respectively.  A logical shift is analogous to sliding the decimal point of radix-10 representations.  For example, the real 
number $0.9345$ is equivalent to $93.45\%$ which is found by sliding the the decimal two places to the right (\textit{multiplying by $\beta^2 = 10^2$}).  
Algebraically a binary logical shift is equivalent to a division or multiplication by a power of two.  
For example, $a << k = a \cdot 2^k$ while $a >> k = \lfloor a/2^k \rfloor$.

One significant difference between a logical shift and the way decimals are shifted is that digits below the zero'th position are removed
from the number.  For example, consider $1101_2 >> 1$ using decimal notation this would produce $110.1_2$.  However, with a logical shift the 
result is $110_2$.  

\section{Addition and Subtraction}
In common twos complement fixed precision arithmetic negative numbers are easily represented by subtraction from the modulus.  For example, with 32-bit integers
$a - b\mbox{ (mod }2^{32}\mbox{)}$ is the same as $a + (2^{32} - b) \mbox{ (mod }2^{32}\mbox{)}$  since $2^{32} \equiv 0 \mbox{ (mod }2^{32}\mbox{)}$.  
As a result subtraction can be performed with a trivial series of logical operations and an addition.

However, in multiple precision arithmetic negative numbers are not represented in the same way.  Instead a sign flag is used to keep track of the
sign of the integer.  As a result signed addition and subtraction are actually implemented as conditional usage of lower level addition or 
subtraction algorithms with the sign fixed up appropriately.

The lower level algorithms will add or subtract integers without regard to the sign flag.  That is they will add or subtract the magnitude of
the integers respectively.

\subsection{Low Level Addition}
An unsigned addition of multiple precision integers is performed with the same long-hand algorithm used to add decimal numbers.  That is to add the 
trailing digits first and propagate the resulting carry upwards.  Since this is a lower level algorithm the name will have a ``s\_'' prefix.  
Historically that convention stems from the MPI library where ``s\_'' stood for static functions that were hidden from the developer entirely.

\newpage
\begin{figure}[!here]
\begin{center}
\begin{small}
\begin{tabular}{l}
\hline Algorithm \textbf{s\_mp\_add}. \\
\textbf{Input}.   Two mp\_ints $a$ and $b$ \\
\textbf{Output}.  The unsigned addition $c = \vert a \vert + \vert b \vert$. \\
\hline \\
1.  if $a.used > b.used$ then \\
\hspace{+3mm}1.1  $min \leftarrow b.used$ \\
\hspace{+3mm}1.2  $max \leftarrow a.used$ \\
\hspace{+3mm}1.3  $x   \leftarrow a$ \\
2.  else  \\
\hspace{+3mm}2.1  $min \leftarrow a.used$ \\
\hspace{+3mm}2.2  $max \leftarrow b.used$ \\
\hspace{+3mm}2.3  $x   \leftarrow b$ \\
3.  If $c.alloc < max + 1$ then grow $c$ to hold at least $max + 1$ digits (\textit{mp\_grow}) \\
4.  $oldused \leftarrow c.used$ \\
5.  $c.used \leftarrow max + 1$ \\
6.  $u \leftarrow 0$ \\
7.  for $n$ from $0$ to $min - 1$ do \\
\hspace{+3mm}7.1  $c_n \leftarrow a_n + b_n + u$ \\
\hspace{+3mm}7.2  $u \leftarrow c_n >> lg(\beta)$ \\
\hspace{+3mm}7.3  $c_n \leftarrow c_n \mbox{ (mod }\beta\mbox{)}$ \\
8.  if $min \ne max$ then do \\
\hspace{+3mm}8.1  for $n$ from $min$ to $max - 1$ do \\
\hspace{+6mm}8.1.1  $c_n \leftarrow x_n + u$ \\
\hspace{+6mm}8.1.2  $u \leftarrow c_n >> lg(\beta)$ \\
\hspace{+6mm}8.1.3  $c_n \leftarrow c_n \mbox{ (mod }\beta\mbox{)}$ \\
9.  $c_{max} \leftarrow u$ \\
10.  if $olduse > max$ then \\
\hspace{+3mm}10.1  for $n$ from $max + 1$ to $oldused - 1$ do \\
\hspace{+6mm}10.1.1  $c_n \leftarrow 0$ \\
11.  Clamp excess digits in $c$.  (\textit{mp\_clamp}) \\
12.  Return(\textit{MP\_OKAY}) \\
\hline
\end{tabular}
\end{small}
\end{center}
\caption{Algorithm s\_mp\_add}
\end{figure}

\textbf{Algorithm s\_mp\_add.}
This algorithm is loosely based on algorithm 14.7 of HAC \cite[pp. 594]{HAC} but has been extended to allow the inputs to have different magnitudes.  
Coincidentally the description of algorithm A in Knuth \cite[pp. 266]{TAOCPV2} shares the same deficiency as the algorithm from \cite{HAC}.  Even the 
MIX pseudo  machine code presented by Knuth \cite[pp. 266-267]{TAOCPV2} is incapable of handling inputs which are of different magnitudes.

The first thing that has to be accomplished is to sort out which of the two inputs is the largest.  The addition logic
will simply add all of the smallest input to the largest input and store that first part of the result in the
destination.  Then it will apply a simpler addition loop to excess digits of the larger input.

The first two steps will handle sorting the inputs such that $min$ and $max$ hold the digit counts of the two 
inputs.  The variable $x$ will be an mp\_int alias for the largest input or the second input $b$ if they have the
same number of digits.  After the inputs are sorted the destination $c$ is grown as required to accomodate the sum 
of the two inputs.  The original \textbf{used} count of $c$ is copied and set to the new used count.  

At this point the first addition loop will go through as many digit positions that both inputs have.  The carry
variable $\mu$ is set to zero outside the loop.  Inside the loop an ``addition'' step requires three statements to produce
one digit of the summand.  First
two digits from $a$ and $b$ are added together along with the carry $\mu$.  The carry of this step is extracted and stored
in $\mu$ and finally the digit of the result $c_n$ is truncated within the range $0 \le c_n < \beta$.

Now all of the digit positions that both inputs have in common have been exhausted.  If $min \ne max$ then $x$ is an alias
for one of the inputs that has more digits.  A simplified addition loop is then used to essentially copy the remaining digits
and the carry to the destination.

The final carry is stored in $c_{max}$ and digits above $max$ upto $oldused$ are zeroed which completes the addition.


\vspace{+3mm}\begin{small}
\hspace{-5.1mm}{\bf File}: bn\_s\_mp\_add.c
\vspace{-3mm}
\begin{alltt}
\end{alltt}
\end{small}

We first sort (lines 28 to 36) the inputs based on magnitude and determine the $min$ and $max$ variables.
Note that $x$ is a pointer to an mp\_int assigned to the largest input, in effect it is a local alias.  Next we
grow the destination (38 to 42) ensure that it can accomodate the result of the addition. 

Similar to the implementation of mp\_copy this function uses the braced code and local aliases coding style.  The three aliases that are on 
lines 56, 59 and 62 represent the two inputs and destination variables respectively.  These aliases are used to ensure the
compiler does not have to dereference $a$, $b$ or $c$ (respectively) to access the digits of the respective mp\_int.

The initial carry $u$ will be cleared (line 65), note that $u$ is of type mp\_digit which ensures type 
compatibility within the implementation.  The initial addition (line 66 to 75) adds digits from
both inputs until the smallest input runs out of digits.  Similarly the conditional addition loop
(line 81 to 90) adds the remaining digits from the larger of the two inputs.  The addition is finished 
with the final carry being stored in $tmpc$ (line 94).  Note the ``++'' operator within the same expression.
After line 94, $tmpc$ will point to the $c.used$'th digit of the mp\_int $c$.  This is useful
for the next loop (line 97 to 99) which set any old upper digits to zero.

\subsection{Low Level Subtraction}
The low level unsigned subtraction algorithm is very similar to the low level unsigned addition algorithm.  The principle difference is that the
unsigned subtraction algorithm requires the result to be positive.  That is when computing $a - b$ the condition $\vert a \vert \ge \vert b\vert$ must 
be met for this algorithm to function properly.  Keep in mind this low level algorithm is not meant to be used in higher level algorithms directly.  
This algorithm as will be shown can be used to create functional signed addition and subtraction algorithms.


For this algorithm a new variable is required to make the description simpler.  Recall from section 1.3.1 that a mp\_digit must be able to represent
the range $0 \le x < 2\beta$ for the algorithms to work correctly.  However, it is allowable that a mp\_digit represent a larger range of values.  For 
this algorithm we will assume that the variable $\gamma$ represents the number of bits available in a 
mp\_digit (\textit{this implies $2^{\gamma} > \beta$}).  

For example, the default for LibTomMath is to use a ``unsigned long'' for the mp\_digit ``type'' while $\beta = 2^{28}$.  In ISO C an ``unsigned long''
data type must be able to represent $0 \le x < 2^{32}$ meaning that in this case $\gamma \ge 32$.

\newpage\begin{figure}[!here]
\begin{center}
\begin{small}
\begin{tabular}{l}
\hline Algorithm \textbf{s\_mp\_sub}. \\
\textbf{Input}.   Two mp\_ints $a$ and $b$ ($\vert a \vert \ge \vert b \vert$) \\
\textbf{Output}.  The unsigned subtraction $c = \vert a \vert - \vert b \vert$. \\
\hline \\
1.  $min \leftarrow b.used$ \\
2.  $max \leftarrow a.used$ \\
3.  If $c.alloc < max$ then grow $c$ to hold at least $max$ digits.  (\textit{mp\_grow}) \\
4.  $oldused \leftarrow c.used$ \\ 
5.  $c.used \leftarrow max$ \\
6.  $u \leftarrow 0$ \\
7.  for $n$ from $0$ to $min - 1$ do \\
\hspace{3mm}7.1  $c_n \leftarrow a_n - b_n - u$ \\
\hspace{3mm}7.2  $u   \leftarrow c_n >> (\gamma - 1)$ \\
\hspace{3mm}7.3  $c_n \leftarrow c_n \mbox{ (mod }\beta\mbox{)}$ \\
8.  if $min < max$ then do \\
\hspace{3mm}8.1  for $n$ from $min$ to $max - 1$ do \\
\hspace{6mm}8.1.1  $c_n \leftarrow a_n - u$ \\
\hspace{6mm}8.1.2  $u   \leftarrow c_n >> (\gamma - 1)$ \\
\hspace{6mm}8.1.3  $c_n \leftarrow c_n \mbox{ (mod }\beta\mbox{)}$ \\
9. if $oldused > max$ then do \\
\hspace{3mm}9.1  for $n$ from $max$ to $oldused - 1$ do \\
\hspace{6mm}9.1.1  $c_n \leftarrow 0$ \\
10. Clamp excess digits of $c$.  (\textit{mp\_clamp}). \\
11. Return(\textit{MP\_OKAY}). \\
\hline
\end{tabular}
\end{small}
\end{center}
\caption{Algorithm s\_mp\_sub}
\end{figure}

\textbf{Algorithm s\_mp\_sub.}
This algorithm performs the unsigned subtraction of two mp\_int variables under the restriction that the result must be positive.  That is when
passing variables $a$ and $b$ the condition that $\vert a \vert \ge \vert b \vert$ must be met for the algorithm to function correctly.  This
algorithm is loosely based on algorithm 14.9 \cite[pp. 595]{HAC} and is similar to algorithm S in \cite[pp. 267]{TAOCPV2} as well.  As was the case
of the algorithm s\_mp\_add both other references lack discussion concerning various practical details such as when the inputs differ in magnitude.

The initial sorting of the inputs is trivial in this algorithm since $a$ is guaranteed to have at least the same magnitude of $b$.  Steps 1 and 2 
set the $min$ and $max$ variables.  Unlike the addition routine there is guaranteed to be no carry which means that the final result can be at 
most $max$ digits in length as opposed to $max + 1$.  Similar to the addition algorithm the \textbf{used} count of $c$ is copied locally and 
set to the maximal count for the operation.

The subtraction loop that begins on step seven is essentially the same as the addition loop of algorithm s\_mp\_add except single precision 
subtraction is used instead.  Note the use of the $\gamma$ variable to extract the carry (\textit{also known as the borrow}) within the subtraction 
loops.  Under the assumption that two's complement single precision arithmetic is used this will successfully extract the desired carry.  

For example, consider subtracting $0101_2$ from $0100_2$ where $\gamma = 4$ and $\beta = 2$.  The least significant bit will force a carry upwards to 
the third bit which will be set to zero after the borrow.  After the very first bit has been subtracted $4 - 1 \equiv 0011_2$ will remain,  When the 
third bit of $0101_2$ is subtracted from the result it will cause another carry.  In this case though the carry will be forced to propagate all the 
way to the most significant bit.  

Recall that $\beta < 2^{\gamma}$.  This means that if a carry does occur just before the $lg(\beta)$'th bit it will propagate all the way to the most 
significant bit.  Thus, the high order bits of the mp\_digit that are not part of the actual digit will either be all zero, or all one. All that
is needed is a single zero or one bit for the carry.  Therefore a single logical shift right by $\gamma - 1$ positions is sufficient to extract the 
carry.  This method of carry extraction may seem awkward but the reason for it becomes apparent when the implementation is discussed.  

If $b$ has a smaller magnitude than $a$ then step 9 will force the carry and copy operation to propagate through the larger input $a$ into $c$.  Step
10 will ensure that any leading digits of $c$ above the $max$'th position are zeroed.

\vspace{+3mm}\begin{small}
\hspace{-5.1mm}{\bf File}: bn\_s\_mp\_sub.c
\vspace{-3mm}
\begin{alltt}
\end{alltt}
\end{small}

Like low level addition we ``sort'' the inputs.  Except in this case the sorting is hardcoded 
(lines 25 and 26).  In reality the $min$ and $max$ variables are only aliases and are only 
used to make the source code easier to read.  Again the pointer alias optimization is used 
within this algorithm.  The aliases $tmpa$, $tmpb$ and $tmpc$ are initialized
(lines 42, 43 and 44) for $a$, $b$ and $c$ respectively.

The first subtraction loop (lines 47 through 61) subtract digits from both inputs until the smaller of
the two inputs has been exhausted.  As remarked earlier there is an implementation reason for using the ``awkward'' 
method of extracting the carry (line 57).  The traditional method for extracting the carry would be to shift 
by $lg(\beta)$ positions and logically AND the least significant bit.  The AND operation is required because all of 
the bits above the $\lg(\beta)$'th bit will be set to one after a carry occurs from subtraction.  This carry 
extraction requires two relatively cheap operations to extract the carry.  The other method is to simply shift the 
most significant bit to the least significant bit thus extracting the carry with a single cheap operation.  This 
optimization only works on twos compliment machines which is a safe assumption to make.

If $a$ has a larger magnitude than $b$ an additional loop (lines 64 through 73) is required to propagate 
the carry through $a$ and copy the result to $c$.  

\subsection{High Level Addition}
Now that both lower level addition and subtraction algorithms have been established an effective high level signed addition algorithm can be
established.  This high level addition algorithm will be what other algorithms and developers will use to perform addition of mp\_int data 
types.  

Recall from section 5.2 that an mp\_int represents an integer with an unsigned mantissa (\textit{the array of digits}) and a \textbf{sign} 
flag.  A high level addition is actually performed as a series of eight separate cases which can be optimized down to three unique cases.

\begin{figure}[!here]
\begin{center}
\begin{tabular}{l}
\hline Algorithm \textbf{mp\_add}. \\
\textbf{Input}.   Two mp\_ints $a$ and $b$  \\
\textbf{Output}.  The signed addition $c = a + b$. \\
\hline \\
1.  if $a.sign = b.sign$ then do \\
\hspace{3mm}1.1  $c.sign \leftarrow a.sign$  \\
\hspace{3mm}1.2  $c \leftarrow \vert a \vert + \vert b \vert$ (\textit{s\_mp\_add})\\
2.  else do \\
\hspace{3mm}2.1  if $\vert a \vert < \vert b \vert$ then do (\textit{mp\_cmp\_mag})  \\
\hspace{6mm}2.1.1  $c.sign \leftarrow b.sign$ \\
\hspace{6mm}2.1.2  $c \leftarrow \vert b \vert - \vert a \vert$ (\textit{s\_mp\_sub}) \\
\hspace{3mm}2.2  else do \\
\hspace{6mm}2.2.1  $c.sign \leftarrow a.sign$ \\
\hspace{6mm}2.2.2  $c \leftarrow \vert a \vert - \vert b \vert$ \\
3.  Return(\textit{MP\_OKAY}). \\
\hline
\end{tabular}
\end{center}
\caption{Algorithm mp\_add}
\end{figure}

\textbf{Algorithm mp\_add.}
This algorithm performs the signed addition of two mp\_int variables.  There is no reference algorithm to draw upon from 
either \cite{TAOCPV2} or \cite{HAC} since they both only provide unsigned operations.  The algorithm is fairly 
straightforward but restricted since subtraction can only produce positive results.

\begin{figure}[here]
\begin{small}
\begin{center}
\begin{tabular}{|c|c|c|c|c|}
\hline \textbf{Sign of $a$} & \textbf{Sign of $b$} & \textbf{$\vert a \vert > \vert b \vert $} & \textbf{Unsigned Operation} & \textbf{Result Sign Flag} \\
\hline $+$ & $+$ & Yes & $c = a + b$ & $a.sign$ \\
\hline $+$ & $+$ & No  & $c = a + b$ & $a.sign$ \\
\hline $-$ & $-$ & Yes & $c = a + b$ & $a.sign$ \\
\hline $-$ & $-$ & No  & $c = a + b$ & $a.sign$ \\
\hline &&&&\\

\hline $+$ & $-$ & No  & $c = b - a$ & $b.sign$ \\
\hline $-$ & $+$ & No  & $c = b - a$ & $b.sign$ \\

\hline &&&&\\

\hline $+$ & $-$ & Yes & $c = a - b$ & $a.sign$ \\
\hline $-$ & $+$ & Yes & $c = a - b$ & $a.sign$ \\

\hline
\end{tabular}
\end{center}
\end{small}
\caption{Addition Guide Chart}
\label{fig:AddChart}
\end{figure}

Figure~\ref{fig:AddChart} lists all of the eight possible input combinations and is sorted to show that only three 
specific cases need to be handled.  The return code of the unsigned operations at step 1.2, 2.1.2 and 2.2.2 are 
forwarded to step three to check for errors.  This simplifies the description of the algorithm considerably and best 
follows how the implementation actually was achieved.

Also note how the \textbf{sign} is set before the unsigned addition or subtraction is performed.  Recall from the descriptions of algorithms
s\_mp\_add and s\_mp\_sub that the mp\_clamp function is used at the end to trim excess digits.  The mp\_clamp algorithm will set the \textbf{sign}
to \textbf{MP\_ZPOS} when the \textbf{used} digit count reaches zero.

For example, consider performing $-a + a$ with algorithm mp\_add.  By the description of the algorithm the sign is set to \textbf{MP\_NEG} which would
produce a result of $-0$.  However, since the sign is set first then the unsigned addition is performed the subsequent usage of algorithm mp\_clamp 
within algorithm s\_mp\_add will force $-0$ to become $0$.  

\vspace{+3mm}\begin{small}
\hspace{-5.1mm}{\bf File}: bn\_mp\_add.c
\vspace{-3mm}
\begin{alltt}
\end{alltt}
\end{small}

The source code follows the algorithm fairly closely.  The most notable new source code addition is the usage of the $res$ integer variable which
is used to pass result of the unsigned operations forward.  Unlike in the algorithm, the variable $res$ is merely returned as is without
explicitly checking it and returning the constant \textbf{MP\_OKAY}.  The observation is this algorithm will succeed or fail only if the lower
level functions do so.  Returning their return code is sufficient.

\subsection{High Level Subtraction}
The high level signed subtraction algorithm is essentially the same as the high level signed addition algorithm.  

\newpage\begin{figure}[!here]
\begin{center}
\begin{tabular}{l}
\hline Algorithm \textbf{mp\_sub}. \\
\textbf{Input}.   Two mp\_ints $a$ and $b$  \\
\textbf{Output}.  The signed subtraction $c = a - b$. \\
\hline \\
1.  if $a.sign \ne b.sign$ then do \\
\hspace{3mm}1.1  $c.sign \leftarrow a.sign$ \\
\hspace{3mm}1.2  $c \leftarrow \vert a \vert + \vert b \vert$ (\textit{s\_mp\_add}) \\
2.  else do \\
\hspace{3mm}2.1  if $\vert a \vert \ge \vert b \vert$ then do (\textit{mp\_cmp\_mag}) \\
\hspace{6mm}2.1.1  $c.sign \leftarrow a.sign$ \\
\hspace{6mm}2.1.2  $c \leftarrow \vert a \vert  - \vert b \vert$ (\textit{s\_mp\_sub}) \\
\hspace{3mm}2.2  else do \\
\hspace{6mm}2.2.1  $c.sign \leftarrow  \left \lbrace \begin{array}{ll}
                              MP\_ZPOS &  \mbox{if }a.sign = MP\_NEG \\
                              MP\_NEG  &  \mbox{otherwise} \\
                              \end{array} \right .$ \\
\hspace{6mm}2.2.2  $c \leftarrow \vert b \vert  - \vert a \vert$ \\
3.  Return(\textit{MP\_OKAY}). \\
\hline
\end{tabular}
\end{center}
\caption{Algorithm mp\_sub}
\end{figure}

\textbf{Algorithm mp\_sub.}
This algorithm performs the signed subtraction of two inputs.  Similar to algorithm mp\_add there is no reference in either \cite{TAOCPV2} or 
\cite{HAC}.  Also this algorithm is restricted by algorithm s\_mp\_sub.  Chart \ref{fig:SubChart} lists the eight possible inputs and
the operations required.

\begin{figure}[!here]
\begin{small}
\begin{center}
\begin{tabular}{|c|c|c|c|c|}
\hline \textbf{Sign of $a$} & \textbf{Sign of $b$} & \textbf{$\vert a \vert \ge \vert b \vert $} & \textbf{Unsigned Operation} & \textbf{Result Sign Flag} \\
\hline $+$ & $-$ & Yes & $c = a + b$ & $a.sign$ \\
\hline $+$ & $-$ & No  & $c = a + b$ & $a.sign$ \\
\hline $-$ & $+$ & Yes & $c = a + b$ & $a.sign$ \\
\hline $-$ & $+$ & No  & $c = a + b$ & $a.sign$ \\
\hline &&&& \\
\hline $+$ & $+$ & Yes & $c = a - b$ & $a.sign$ \\
\hline $-$ & $-$ & Yes & $c = a - b$ & $a.sign$ \\
\hline &&&& \\
\hline $+$ & $+$ & No  & $c = b - a$ & $\mbox{opposite of }a.sign$ \\
\hline $-$ & $-$ & No  & $c = b - a$ & $\mbox{opposite of }a.sign$ \\
\hline
\end{tabular}
\end{center}
\end{small}
\caption{Subtraction Guide Chart}
\label{fig:SubChart}
\end{figure}

Similar to the case of algorithm mp\_add the \textbf{sign} is set first before the unsigned addition or subtraction.  That is to prevent the 
algorithm from producing $-a - -a = -0$ as a result.  

\vspace{+3mm}\begin{small}
\hspace{-5.1mm}{\bf File}: bn\_mp\_sub.c
\vspace{-3mm}
\begin{alltt}
\end{alltt}
\end{small}

Much like the implementation of algorithm mp\_add the variable $res$ is used to catch the return code of the unsigned addition or subtraction operations
and forward it to the end of the function.  On line 39 the ``not equal to'' \textbf{MP\_LT} expression is used to emulate a 
``greater than or equal to'' comparison.  

\section{Bit and Digit Shifting}
It is quite common to think of a multiple precision integer as a polynomial in $x$, that is $y = f(\beta)$ where $f(x) = \sum_{i=0}^{n-1} a_i x^i$.  
This notation arises within discussion of Montgomery and Diminished Radix Reduction as well as Karatsuba multiplication and squaring.  

In order to facilitate operations on polynomials in $x$ as above a series of simple ``digit'' algorithms have to be established.  That is to shift
the digits left or right as well to shift individual bits of the digits left and right.  It is important to note that not all ``shift'' operations
are on radix-$\beta$ digits.  

\subsection{Multiplication by Two}

In a binary system where the radix is a power of two multiplication by two not only arises often in other algorithms it is a fairly efficient 
operation to perform.  A single precision logical shift left is sufficient to multiply a single digit by two.  

\newpage\begin{figure}[!here]
\begin{small}
\begin{center}
\begin{tabular}{l}
\hline Algorithm \textbf{mp\_mul\_2}. \\
\textbf{Input}.   One mp\_int $a$ \\
\textbf{Output}.  $b = 2a$. \\
\hline \\
1.  If $b.alloc < a.used + 1$ then grow $b$ to hold $a.used + 1$ digits.  (\textit{mp\_grow}) \\
2.  $oldused \leftarrow b.used$ \\
3.  $b.used \leftarrow a.used$ \\
4.  $r \leftarrow 0$ \\
5.  for $n$ from 0 to $a.used - 1$ do \\
\hspace{3mm}5.1  $rr \leftarrow a_n >> (lg(\beta) - 1)$ \\
\hspace{3mm}5.2  $b_n \leftarrow (a_n << 1) + r \mbox{ (mod }\beta\mbox{)}$ \\
\hspace{3mm}5.3  $r \leftarrow rr$ \\
6.  If $r \ne 0$ then do \\
\hspace{3mm}6.1  $b_{n + 1} \leftarrow r$ \\
\hspace{3mm}6.2  $b.used \leftarrow b.used + 1$ \\
7.  If $b.used < oldused - 1$ then do \\
\hspace{3mm}7.1  for $n$ from $b.used$ to $oldused - 1$ do \\
\hspace{6mm}7.1.1  $b_n \leftarrow 0$ \\
8.  $b.sign \leftarrow a.sign$ \\
9.  Return(\textit{MP\_OKAY}).\\
\hline
\end{tabular}
\end{center}
\end{small}
\caption{Algorithm mp\_mul\_2}
\end{figure}

\textbf{Algorithm mp\_mul\_2.}
This algorithm will quickly multiply a mp\_int by two provided $\beta$ is a power of two.  Neither \cite{TAOCPV2} nor \cite{HAC} describe such 
an algorithm despite the fact it arises often in other algorithms.  The algorithm is setup much like the lower level algorithm s\_mp\_add since 
it is for all intents and purposes equivalent to the operation $b = \vert a \vert + \vert a \vert$.  

Step 1 and 2 grow the input as required to accomodate the maximum number of \textbf{used} digits in the result.  The initial \textbf{used} count
is set to $a.used$ at step 4.  Only if there is a final carry will the \textbf{used} count require adjustment.

Step 6 is an optimization implementation of the addition loop for this specific case.  That is since the two values being added together 
are the same there is no need to perform two reads from the digits of $a$.  Step 6.1 performs a single precision shift on the current digit $a_n$ to
obtain what will be the carry for the next iteration.  Step 6.2 calculates the $n$'th digit of the result as single precision shift of $a_n$ plus
the previous carry.  Recall from section 4.1 that $a_n << 1$ is equivalent to $a_n \cdot 2$.  An iteration of the addition loop is finished with 
forwarding the carry to the next iteration.

Step 7 takes care of any final carry by setting the $a.used$'th digit of the result to the carry and augmenting the \textbf{used} count of $b$.  
Step 8 clears any leading digits of $b$ in case it originally had a larger magnitude than $a$.

\vspace{+3mm}\begin{small}
\hspace{-5.1mm}{\bf File}: bn\_mp\_mul\_2.c
\vspace{-3mm}
\begin{alltt}
\end{alltt}
\end{small}

This implementation is essentially an optimized implementation of s\_mp\_add for the case of doubling an input.  The only noteworthy difference
is the use of the logical shift operator on line 52 to perform a single precision doubling.  

\subsection{Division by Two}
A division by two can just as easily be accomplished with a logical shift right as multiplication by two can be with a logical shift left.

\newpage\begin{figure}[!here]
\begin{small}
\begin{center}
\begin{tabular}{l}
\hline Algorithm \textbf{mp\_div\_2}. \\
\textbf{Input}.   One mp\_int $a$ \\
\textbf{Output}.  $b = a/2$. \\
\hline \\
1.  If $b.alloc < a.used$ then grow $b$ to hold $a.used$ digits.  (\textit{mp\_grow}) \\
2.  If the reallocation failed return(\textit{MP\_MEM}). \\
3.  $oldused \leftarrow b.used$ \\
4.  $b.used \leftarrow a.used$ \\
5.  $r \leftarrow 0$ \\
6.  for $n$ from $b.used - 1$ to $0$ do \\
\hspace{3mm}6.1  $rr \leftarrow a_n \mbox{ (mod }2\mbox{)}$\\
\hspace{3mm}6.2  $b_n \leftarrow (a_n >> 1) + (r << (lg(\beta) - 1)) \mbox{ (mod }\beta\mbox{)}$ \\
\hspace{3mm}6.3  $r \leftarrow rr$ \\
7.  If $b.used < oldused - 1$ then do \\
\hspace{3mm}7.1  for $n$ from $b.used$ to $oldused - 1$ do \\
\hspace{6mm}7.1.1  $b_n \leftarrow 0$ \\
8.  $b.sign \leftarrow a.sign$ \\
9.  Clamp excess digits of $b$.  (\textit{mp\_clamp}) \\
10.  Return(\textit{MP\_OKAY}).\\
\hline
\end{tabular}
\end{center}
\end{small}
\caption{Algorithm mp\_div\_2}
\end{figure}

\textbf{Algorithm mp\_div\_2.}
This algorithm will divide an mp\_int by two using logical shifts to the right.  Like mp\_mul\_2 it uses a modified low level addition
core as the basis of the algorithm.  Unlike mp\_mul\_2 the shift operations work from the leading digit to the trailing digit.  The algorithm
could be written to work from the trailing digit to the leading digit however, it would have to stop one short of $a.used - 1$ digits to prevent
reading past the end of the array of digits.

Essentially the loop at step 6 is similar to that of mp\_mul\_2 except the logical shifts go in the opposite direction and the carry is at the 
least significant bit not the most significant bit.  

\vspace{+3mm}\begin{small}
\hspace{-5.1mm}{\bf File}: bn\_mp\_div\_2.c
\vspace{-3mm}
\begin{alltt}
\end{alltt}
\end{small}

\section{Polynomial Basis Operations}
Recall from section 4.3 that any integer can be represented as a polynomial in $x$ as $y = f(\beta)$.  Such a representation is also known as
the polynomial basis \cite[pp. 48]{ROSE}. Given such a notation a multiplication or division by $x$ amounts to shifting whole digits a single 
place.  The need for such operations arises in several other higher level algorithms such as Barrett and Montgomery reduction, integer
division and Karatsuba multiplication.  

Converting from an array of digits to polynomial basis is very simple.  Consider the integer $y \equiv (a_2, a_1, a_0)_{\beta}$ and recall that
$y = \sum_{i=0}^{2} a_i \beta^i$.  Simply replace $\beta$ with $x$ and the expression is in polynomial basis.  For example, $f(x) = 8x + 9$ is the
polynomial basis representation for $89$ using radix ten.  That is, $f(10) = 8(10) + 9 = 89$.  

\subsection{Multiplication by $x$}

Given a polynomial in $x$ such as $f(x) = a_n x^n + a_{n-1} x^{n-1} + ... + a_0$ multiplying by $x$ amounts to shifting the coefficients up one 
degree.  In this case $f(x) \cdot x = a_n x^{n+1} + a_{n-1} x^n + ... + a_0 x$.  From a scalar basis point of view multiplying by $x$ is equivalent to
multiplying by the integer $\beta$.  

\newpage\begin{figure}[!here]
\begin{small}
\begin{center}
\begin{tabular}{l}
\hline Algorithm \textbf{mp\_lshd}. \\
\textbf{Input}.   One mp\_int $a$ and an integer $b$ \\
\textbf{Output}.  $a \leftarrow a \cdot \beta^b$ (equivalent to multiplication by $x^b$). \\
\hline \\
1.  If $b \le 0$ then return(\textit{MP\_OKAY}). \\
2.  If $a.alloc < a.used + b$ then grow $a$ to at least $a.used + b$ digits.  (\textit{mp\_grow}). \\
3.  If the reallocation failed return(\textit{MP\_MEM}). \\
4.  $a.used \leftarrow a.used + b$ \\
5.  $i \leftarrow a.used - 1$ \\
6.  $j \leftarrow a.used - 1 - b$ \\
7.  for $n$ from $a.used - 1$ to $b$ do \\
\hspace{3mm}7.1  $a_{i} \leftarrow a_{j}$ \\
\hspace{3mm}7.2  $i \leftarrow i - 1$ \\
\hspace{3mm}7.3  $j \leftarrow j - 1$ \\
8.  for $n$ from 0 to $b - 1$ do \\
\hspace{3mm}8.1  $a_n \leftarrow 0$ \\
9.  Return(\textit{MP\_OKAY}). \\
\hline
\end{tabular}
\end{center}
\end{small}
\caption{Algorithm mp\_lshd}
\end{figure}

\textbf{Algorithm mp\_lshd.}
This algorithm multiplies an mp\_int by the $b$'th power of $x$.  This is equivalent to multiplying by $\beta^b$.  The algorithm differs 
from the other algorithms presented so far as it performs the operation in place instead storing the result in a separate location.  The
motivation behind this change is due to the way this function is typically used.  Algorithms such as mp\_add store the result in an optionally
different third mp\_int because the original inputs are often still required.  Algorithm mp\_lshd (\textit{and similarly algorithm mp\_rshd}) is
typically used on values where the original value is no longer required.  The algorithm will return success immediately if 
$b \le 0$ since the rest of algorithm is only valid when $b > 0$.  

First the destination $a$ is grown as required to accomodate the result.  The counters $i$ and $j$ are used to form a \textit{sliding window} over
the digits of $a$ of length $b$.  The head of the sliding window is at $i$ (\textit{the leading digit}) and the tail at $j$ (\textit{the trailing digit}).  
The loop on step 7 copies the digit from the tail to the head.  In each iteration the window is moved down one digit.   The last loop on 
step 8 sets the lower $b$ digits to zero.

\newpage
\begin{center}
\begin{figure}[here]
\includegraphics{pics/sliding_window.ps}
\caption{Sliding Window Movement}
\label{pic:sliding_window}
\end{figure}
\end{center}

\vspace{+3mm}\begin{small}
\hspace{-5.1mm}{\bf File}: bn\_mp\_lshd.c
\vspace{-3mm}
\begin{alltt}
\end{alltt}
\end{small}

The if statement (line 24) ensures that the $b$ variable is greater than zero since we do not interpret negative
shift counts properly.  The \textbf{used} count is incremented by $b$ before the copy loop begins.  This elminates 
the need for an additional variable in the for loop.  The variable $top$ (line 42) is an alias
for the leading digit while $bottom$ (line 45) is an alias for the trailing edge.  The aliases form a 
window of exactly $b$ digits over the input.  

\subsection{Division by $x$}

Division by powers of $x$ is easily achieved by shifting the digits right and removing any that will end up to the right of the zero'th digit.  

\newpage\begin{figure}[!here]
\begin{small}
\begin{center}
\begin{tabular}{l}
\hline Algorithm \textbf{mp\_rshd}. \\
\textbf{Input}.   One mp\_int $a$ and an integer $b$ \\
\textbf{Output}.  $a \leftarrow a / \beta^b$ (Divide by $x^b$). \\
\hline \\
1.  If $b \le 0$ then return. \\
2.  If $a.used \le b$ then do \\
\hspace{3mm}2.1  Zero $a$.  (\textit{mp\_zero}). \\
\hspace{3mm}2.2  Return. \\
3.  $i \leftarrow 0$ \\
4.  $j \leftarrow b$ \\
5.  for $n$ from 0 to $a.used - b - 1$ do \\
\hspace{3mm}5.1  $a_i \leftarrow a_j$ \\
\hspace{3mm}5.2  $i \leftarrow i + 1$ \\
\hspace{3mm}5.3  $j \leftarrow j + 1$ \\
6.  for $n$ from $a.used - b$ to $a.used - 1$ do \\
\hspace{3mm}6.1  $a_n \leftarrow 0$ \\
7.  $a.used \leftarrow a.used - b$ \\
8.  Return. \\
\hline
\end{tabular}
\end{center}
\end{small}
\caption{Algorithm mp\_rshd}
\end{figure}

\textbf{Algorithm mp\_rshd.}
This algorithm divides the input in place by the $b$'th power of $x$.  It is analogous to dividing by a $\beta^b$ but much quicker since
it does not require single precision division.  This algorithm does not actually return an error code as it cannot fail.  

If the input $b$ is less than one the algorithm quickly returns without performing any work.  If the \textbf{used} count is less than or equal
to the shift count $b$ then it will simply zero the input and return.

After the trivial cases of inputs have been handled the sliding window is setup.  Much like the case of algorithm mp\_lshd a sliding window that
is $b$ digits wide is used to copy the digits.  Unlike mp\_lshd the window slides in the opposite direction from the trailing to the leading digit.  
Also the digits are copied from the leading to the trailing edge.

Once the window copy is complete the upper digits must be zeroed and the \textbf{used} count decremented.

\vspace{+3mm}\begin{small}
\hspace{-5.1mm}{\bf File}: bn\_mp\_rshd.c
\vspace{-3mm}
\begin{alltt}
\end{alltt}
\end{small}

The only noteworthy element of this routine is the lack of a return type since it cannot fail.  Like mp\_lshd() we
form a sliding window except we copy in the other direction.  After the window (line 60) we then zero
the upper digits of the input to make sure the result is correct.

\section{Powers of Two}

Now that algorithms for moving single bits as well as whole digits exist algorithms for moving the ``in between'' distances are required.  For 
example, to quickly multiply by $2^k$ for any $k$ without using a full multiplier algorithm would prove useful.  Instead of performing single
shifts $k$ times to achieve a multiplication by $2^{\pm k}$ a mixture of whole digit shifting and partial digit shifting is employed.  

\subsection{Multiplication by Power of Two}

\newpage\begin{figure}[!here]
\begin{small}
\begin{center}
\begin{tabular}{l}
\hline Algorithm \textbf{mp\_mul\_2d}. \\
\textbf{Input}.   One mp\_int $a$ and an integer $b$ \\
\textbf{Output}.  $c \leftarrow a \cdot 2^b$. \\
\hline \\
1.  $c \leftarrow a$.  (\textit{mp\_copy}) \\
2.  If $c.alloc < c.used + \lfloor b / lg(\beta) \rfloor + 2$ then grow $c$ accordingly. \\
3.  If the reallocation failed return(\textit{MP\_MEM}). \\
4.  If $b \ge lg(\beta)$ then \\
\hspace{3mm}4.1  $c \leftarrow c \cdot \beta^{\lfloor b / lg(\beta) \rfloor}$ (\textit{mp\_lshd}). \\
\hspace{3mm}4.2  If step 4.1 failed return(\textit{MP\_MEM}). \\
5.  $d \leftarrow b \mbox{ (mod }lg(\beta)\mbox{)}$ \\
6.  If $d \ne 0$ then do \\
\hspace{3mm}6.1  $mask \leftarrow 2^d$ \\
\hspace{3mm}6.2  $r \leftarrow 0$ \\
\hspace{3mm}6.3  for $n$ from $0$ to $c.used - 1$ do \\
\hspace{6mm}6.3.1  $rr \leftarrow c_n >> (lg(\beta) - d) \mbox{ (mod }mask\mbox{)}$ \\
\hspace{6mm}6.3.2  $c_n \leftarrow (c_n << d) + r \mbox{ (mod }\beta\mbox{)}$ \\
\hspace{6mm}6.3.3  $r \leftarrow rr$ \\
\hspace{3mm}6.4  If $r > 0$ then do \\
\hspace{6mm}6.4.1  $c_{c.used} \leftarrow r$ \\
\hspace{6mm}6.4.2  $c.used \leftarrow c.used + 1$ \\
7.  Return(\textit{MP\_OKAY}). \\
\hline
\end{tabular}
\end{center}
\end{small}
\caption{Algorithm mp\_mul\_2d}
\end{figure}

\textbf{Algorithm mp\_mul\_2d.}
This algorithm multiplies $a$ by $2^b$ and stores the result in $c$.  The algorithm uses algorithm mp\_lshd and a derivative of algorithm mp\_mul\_2 to
quickly compute the product.

First the algorithm will multiply $a$ by $x^{\lfloor b / lg(\beta) \rfloor}$ which will ensure that the remainder multiplicand is less than 
$\beta$.  For example, if $b = 37$ and $\beta = 2^{28}$ then this step will multiply by $x$ leaving a multiplication by $2^{37 - 28} = 2^{9}$ 
left.

After the digits have been shifted appropriately at most $lg(\beta) - 1$ shifts are left to perform.  Step 5 calculates the number of remaining shifts 
required.  If it is non-zero a modified shift loop is used to calculate the remaining product.  
Essentially the loop is a generic version of algorithm mp\_mul\_2 designed to handle any shift count in the range $1 \le x < lg(\beta)$.  The $mask$
variable is used to extract the upper $d$ bits to form the carry for the next iteration.  

This algorithm is loosely measured as a $O(2n)$ algorithm which means that if the input is $n$-digits that it takes $2n$ ``time'' to 
complete.  It is possible to optimize this algorithm down to a $O(n)$ algorithm at a cost of making the algorithm slightly harder to follow.

\vspace{+3mm}\begin{small}
\hspace{-5.1mm}{\bf File}: bn\_mp\_mul\_2d.c
\vspace{-3mm}
\begin{alltt}
\end{alltt}
\end{small}

The shifting is performed in--place which means the first step (line 25) is to copy the input to the 
destination.  We avoid calling mp\_copy() by making sure the mp\_ints are different.  The destination then
has to be grown (line 32) to accomodate the result.

If the shift count $b$ is larger than $lg(\beta)$ then a call to mp\_lshd() is used to handle all of the multiples 
of $lg(\beta)$.  Leaving only a remaining shift of $lg(\beta) - 1$ or fewer bits left.  Inside the actual shift 
loop (lines 46 to 76) we make use of pre--computed values $shift$ and $mask$.   These are used to
extract the carry bit(s) to pass into the next iteration of the loop.  The $r$ and $rr$ variables form a 
chain between consecutive iterations to propagate the carry.  

\subsection{Division by Power of Two}

\newpage\begin{figure}[!here]
\begin{small}
\begin{center}
\begin{tabular}{l}
\hline Algorithm \textbf{mp\_div\_2d}. \\
\textbf{Input}.   One mp\_int $a$ and an integer $b$ \\
\textbf{Output}.  $c \leftarrow \lfloor a / 2^b \rfloor, d \leftarrow a \mbox{ (mod }2^b\mbox{)}$. \\
\hline \\
1.  If $b \le 0$ then do \\
\hspace{3mm}1.1  $c \leftarrow a$ (\textit{mp\_copy}) \\
\hspace{3mm}1.2  $d \leftarrow 0$ (\textit{mp\_zero}) \\
\hspace{3mm}1.3  Return(\textit{MP\_OKAY}). \\
2.  $c \leftarrow a$ \\
3.  $d \leftarrow a \mbox{ (mod }2^b\mbox{)}$ (\textit{mp\_mod\_2d}) \\
4.  If $b \ge lg(\beta)$ then do \\
\hspace{3mm}4.1  $c \leftarrow \lfloor c/\beta^{\lfloor b/lg(\beta) \rfloor} \rfloor$ (\textit{mp\_rshd}). \\
5.  $k \leftarrow b \mbox{ (mod }lg(\beta)\mbox{)}$ \\
6.  If $k \ne 0$ then do \\
\hspace{3mm}6.1  $mask \leftarrow 2^k$ \\
\hspace{3mm}6.2  $r \leftarrow 0$ \\
\hspace{3mm}6.3  for $n$ from $c.used - 1$ to $0$ do \\
\hspace{6mm}6.3.1  $rr \leftarrow c_n \mbox{ (mod }mask\mbox{)}$ \\
\hspace{6mm}6.3.2  $c_n \leftarrow (c_n >> k) + (r << (lg(\beta) - k))$ \\
\hspace{6mm}6.3.3  $r \leftarrow rr$ \\
7.  Clamp excess digits of $c$.  (\textit{mp\_clamp}) \\
8.  Return(\textit{MP\_OKAY}). \\
\hline
\end{tabular}
\end{center}
\end{small}
\caption{Algorithm mp\_div\_2d}
\end{figure}

\textbf{Algorithm mp\_div\_2d.}
This algorithm will divide an input $a$ by $2^b$ and produce the quotient and remainder.  The algorithm is designed much like algorithm 
mp\_mul\_2d by first using whole digit shifts then single precision shifts.  This algorithm will also produce the remainder of the division
by using algorithm mp\_mod\_2d.

\vspace{+3mm}\begin{small}
\hspace{-5.1mm}{\bf File}: bn\_mp\_div\_2d.c
\vspace{-3mm}
\begin{alltt}
\end{alltt}
\end{small}

The implementation of algorithm mp\_div\_2d is slightly different than the algorithm specifies.  The remainder $d$ may be optionally 
ignored by passing \textbf{NULL} as the pointer to the mp\_int variable.    The temporary mp\_int variable $t$ is used to hold the 
result of the remainder operation until the end.  This allows $d$ and $a$ to represent the same mp\_int without modifying $a$ before
the quotient is obtained.

The remainder of the source code is essentially the same as the source code for mp\_mul\_2d.  The only significant difference is
the direction of the shifts.

\subsection{Remainder of Division by Power of Two}

The last algorithm in the series of polynomial basis power of two algorithms is calculating the remainder of division by $2^b$.  This
algorithm benefits from the fact that in twos complement arithmetic $a \mbox{ (mod }2^b\mbox{)}$ is the same as $a$ AND $2^b - 1$.  

\begin{figure}[!here]
\begin{small}
\begin{center}
\begin{tabular}{l}
\hline Algorithm \textbf{mp\_mod\_2d}. \\
\textbf{Input}.   One mp\_int $a$ and an integer $b$ \\
\textbf{Output}.  $c \leftarrow a \mbox{ (mod }2^b\mbox{)}$. \\
\hline \\
1.  If $b \le 0$ then do \\
\hspace{3mm}1.1  $c \leftarrow 0$ (\textit{mp\_zero}) \\
\hspace{3mm}1.2  Return(\textit{MP\_OKAY}). \\
2.  If $b > a.used \cdot lg(\beta)$ then do \\
\hspace{3mm}2.1  $c \leftarrow a$ (\textit{mp\_copy}) \\
\hspace{3mm}2.2  Return the result of step 2.1. \\
3.  $c \leftarrow a$ \\
4.  If step 3 failed return(\textit{MP\_MEM}). \\
5.  for $n$ from $\lceil b / lg(\beta) \rceil$ to $c.used$ do \\
\hspace{3mm}5.1  $c_n \leftarrow 0$ \\
6.  $k \leftarrow b \mbox{ (mod }lg(\beta)\mbox{)}$ \\
7.  $c_{\lfloor b / lg(\beta) \rfloor} \leftarrow c_{\lfloor b / lg(\beta) \rfloor} \mbox{ (mod }2^{k}\mbox{)}$. \\
8.  Clamp excess digits of $c$.  (\textit{mp\_clamp}) \\
9.  Return(\textit{MP\_OKAY}). \\
\hline
\end{tabular}
\end{center}
\end{small}
\caption{Algorithm mp\_mod\_2d}
\end{figure}

\textbf{Algorithm mp\_mod\_2d.}
This algorithm will quickly calculate the value of $a \mbox{ (mod }2^b\mbox{)}$.  First if $b$ is less than or equal to zero the 
result is set to zero.  If $b$ is greater than the number of bits in $a$ then it simply copies $a$ to $c$ and returns.  Otherwise, $a$ 
is copied to $b$, leading digits are removed and the remaining leading digit is trimed to the exact bit count.

\vspace{+3mm}\begin{small}
\hspace{-5.1mm}{\bf File}: bn\_mp\_mod\_2d.c
\vspace{-3mm}
\begin{alltt}
\end{alltt}
\end{small}

We first avoid cases of $b \le 0$ by simply mp\_zero()'ing the destination in such cases.  Next if $2^b$ is larger
than the input we just mp\_copy() the input and return right away.  After this point we know we must actually
perform some work to produce the remainder.

Recalling that reducing modulo $2^k$ and a binary ``and'' with $2^k - 1$ are numerically equivalent we can quickly reduce 
the number.  First we zero any digits above the last digit in $2^b$ (line 42).  Next we reduce the 
leading digit of both (line 46) and then mp\_clamp().

\section*{Exercises}
\begin{tabular}{cl}
$\left [ 3 \right ] $ & Devise an algorithm that performs $a \cdot 2^b$ for generic values of $b$ \\
                      & in $O(n)$ time. \\
                      &\\
$\left [ 3 \right ] $ & Devise an efficient algorithm to multiply by small low hamming  \\
                      & weight values such as $3$, $5$ and $9$.  Extend it to handle all values \\
                      & upto $64$ with a hamming weight less than three. \\
                      &\\
$\left [ 2 \right ] $ & Modify the preceding algorithm to handle values of the form \\
                      & $2^k - 1$ as well. \\
                      &\\
$\left [ 3 \right ] $ & Using only algorithms mp\_mul\_2, mp\_div\_2 and mp\_add create an \\
                      & algorithm to multiply two integers in roughly $O(2n^2)$ time for \\
                      & any $n$-bit input.  Note that the time of addition is ignored in the \\
                      & calculation.  \\
                      & \\
$\left [ 5 \right ] $ & Improve the previous algorithm to have a working time of at most \\
                      & $O \left (2^{(k-1)}n + \left ({2n^2 \over k} \right ) \right )$ for an appropriate choice of $k$.  Again ignore \\
                      & the cost of addition. \\
                      & \\
$\left [ 2 \right ] $ & Devise a chart to find optimal values of $k$ for the previous problem \\
                      & for $n = 64 \ldots 1024$ in steps of $64$. \\
                      & \\
$\left [ 2 \right ] $ & Using only algorithms mp\_abs and mp\_sub devise another method for \\
                      & calculating the result of a signed comparison. \\
                      &
\end{tabular}

\chapter{Multiplication and Squaring}
\section{The Multipliers}
For most number theoretic problems including certain public key cryptographic algorithms, the ``multipliers'' form the most important subset of 
algorithms of any multiple precision integer package.  The set of multiplier algorithms include integer multiplication, squaring and modular reduction 
where in each of the algorithms single precision multiplication is the dominant operation performed.  This chapter will discuss integer multiplication 
and squaring, leaving modular reductions for the subsequent chapter.  

The importance of the multiplier algorithms is for the most part driven by the fact that certain popular public key algorithms are based on modular 
exponentiation, that is computing $d \equiv a^b \mbox{ (mod }c\mbox{)}$ for some arbitrary choice of $a$, $b$, $c$ and $d$.  During a modular
exponentiation the majority\footnote{Roughly speaking a modular exponentiation will spend about 40\% of the time performing modular reductions, 
35\% of the time performing squaring and 25\% of the time performing multiplications.} of the processor time is spent performing single precision 
multiplications.

For centuries general purpose multiplication has required a lengthly $O(n^2)$ process, whereby each digit of one multiplicand has to be multiplied 
against every digit of the other multiplicand.  Traditional long-hand multiplication is based on this process;  while the techniques can differ the 
overall algorithm used is essentially the same.  Only ``recently'' have faster algorithms been studied.  First Karatsuba multiplication was discovered in 
1962.  This algorithm can multiply two numbers with considerably fewer single precision multiplications when compared to the long-hand approach.  
This technique led to the discovery of polynomial basis algorithms (\textit{good reference?}) and subquently Fourier Transform based solutions.  

\section{Multiplication}
\subsection{The Baseline Multiplication}
\label{sec:basemult}
\index{baseline multiplication}
Computing the product of two integers in software can be achieved using a trivial adaptation of the standard $O(n^2)$ long-hand multiplication
algorithm that school children are taught.  The algorithm is considered an $O(n^2)$ algorithm since for two $n$-digit inputs $n^2$ single precision 
multiplications are required.  More specifically for a $m$ and $n$ digit input $m \cdot n$ single precision multiplications are required.  To 
simplify most discussions, it will be assumed that the inputs have comparable number of digits.  

The ``baseline multiplication'' algorithm is designed to act as the ``catch-all'' algorithm, only to be used when the faster algorithms cannot be 
used.  This algorithm does not use any particularly interesting optimizations and should ideally be avoided if possible.    One important 
facet of this algorithm, is that it has been modified to only produce a certain amount of output digits as resolution.  The importance of this 
modification will become evident during the discussion of Barrett modular reduction.  Recall that for a $n$ and $m$ digit input the product 
will be at most $n + m$ digits.  Therefore, this algorithm can be reduced to a full multiplier by having it produce $n + m$ digits of the product.  

Recall from sub-section 4.2.2 the definition of $\gamma$ as the number of bits in the type \textbf{mp\_digit}.  We shall now extend the variable set to 
include $\alpha$ which shall represent the number of bits in the type \textbf{mp\_word}.  This implies that $2^{\alpha} > 2 \cdot \beta^2$.  The 
constant $\delta = 2^{\alpha - 2lg(\beta)}$ will represent the maximal weight of any column in a product (\textit{see sub-section 5.2.2 for more information}).

\newpage\begin{figure}[!here]
\begin{small}
\begin{center}
\begin{tabular}{l}
\hline Algorithm \textbf{s\_mp\_mul\_digs}. \\
\textbf{Input}.   mp\_int $a$, mp\_int $b$ and an integer $digs$ \\
\textbf{Output}.  $c \leftarrow \vert a \vert \cdot \vert b \vert \mbox{ (mod }\beta^{digs}\mbox{)}$. \\
\hline \\
1.  If min$(a.used, b.used) < \delta$ then do \\
\hspace{3mm}1.1  Calculate $c = \vert a \vert \cdot \vert b \vert$ by the Comba method (\textit{see algorithm~\ref{fig:COMBAMULT}}).  \\
\hspace{3mm}1.2  Return the result of step 1.1 \\
\\
Allocate and initialize a temporary mp\_int. \\
2.  Init $t$ to be of size $digs$ \\
3.  If step 2 failed return(\textit{MP\_MEM}). \\
4.  $t.used \leftarrow digs$ \\
\\
Compute the product. \\
5.  for $ix$ from $0$ to $a.used - 1$ do \\
\hspace{3mm}5.1  $u \leftarrow 0$ \\
\hspace{3mm}5.2  $pb \leftarrow \mbox{min}(b.used, digs - ix)$ \\
\hspace{3mm}5.3  If $pb < 1$ then goto step 6. \\
\hspace{3mm}5.4  for $iy$ from $0$ to $pb - 1$ do \\
\hspace{6mm}5.4.1  $\hat r \leftarrow t_{iy + ix} + a_{ix} \cdot b_{iy} + u$ \\
\hspace{6mm}5.4.2  $t_{iy + ix} \leftarrow \hat r \mbox{ (mod }\beta\mbox{)}$ \\
\hspace{6mm}5.4.3  $u \leftarrow \lfloor \hat r / \beta \rfloor$ \\
\hspace{3mm}5.5  if $ix + pb < digs$ then do \\
\hspace{6mm}5.5.1  $t_{ix + pb} \leftarrow u$ \\
6.  Clamp excess digits of $t$. \\
7.  Swap $c$ with $t$ \\
8.  Clear $t$ \\
9.  Return(\textit{MP\_OKAY}). \\
\hline
\end{tabular}
\end{center}
\end{small}
\caption{Algorithm s\_mp\_mul\_digs}
\end{figure}

\textbf{Algorithm s\_mp\_mul\_digs.}
This algorithm computes the unsigned product of two inputs $a$ and $b$, limited to an output precision of $digs$ digits.  While it may seem
a bit awkward to modify the function from its simple $O(n^2)$ description, the usefulness of partial multipliers will arise in a subsequent 
algorithm.  The algorithm is loosely based on algorithm 14.12 from \cite[pp. 595]{HAC} and is similar to Algorithm M of Knuth \cite[pp. 268]{TAOCPV2}.  
Algorithm s\_mp\_mul\_digs differs from these cited references since it can produce a variable output precision regardless of the precision of the 
inputs.

The first thing this algorithm checks for is whether a Comba multiplier can be used instead.   If the minimum digit count of either
input is less than $\delta$, then the Comba method may be used instead.    After the Comba method is ruled out, the baseline algorithm begins.  A 
temporary mp\_int variable $t$ is used to hold the intermediate result of the product.  This allows the algorithm to be used to 
compute products when either $a = c$ or $b = c$ without overwriting the inputs.  

All of step 5 is the infamous $O(n^2)$ multiplication loop slightly modified to only produce upto $digs$ digits of output.  The $pb$ variable
is given the count of digits to read from $b$ inside the nested loop.  If $pb \le 1$ then no more output digits can be produced and the algorithm
will exit the loop.  The best way to think of the loops are as a series of $pb \times 1$ multiplications.    That is, in each pass of the 
innermost loop $a_{ix}$ is multiplied against $b$ and the result is added (\textit{with an appropriate shift}) to $t$.  

For example, consider multiplying $576$ by $241$.  That is equivalent to computing $10^0(1)(576) + 10^1(4)(576) + 10^2(2)(576)$ which is best
visualized in the following table.

\begin{figure}[here]
\begin{center}
\begin{tabular}{|c|c|c|c|c|c|l|}
\hline   &&          & 5 & 7 & 6 & \\
\hline   $\times$&&  & 2 & 4 & 1 & \\
\hline &&&&&&\\
  &&          & 5 & 7 & 6 & $10^0(1)(576)$ \\
  &2 &   3    & 6 & 1 & 6 & $10^1(4)(576) + 10^0(1)(576)$ \\
  1 & 3 & 8 & 8 & 1 & 6 &   $10^2(2)(576) + 10^1(4)(576) + 10^0(1)(576)$ \\
\hline  
\end{tabular}
\end{center}
\caption{Long-Hand Multiplication Diagram}
\end{figure}

Each row of the product is added to the result after being shifted to the left (\textit{multiplied by a power of the radix}) by the appropriate 
count.  That is in pass $ix$ of the inner loop the product is added starting at the $ix$'th digit of the reult.

Step 5.4.1 introduces the hat symbol (\textit{e.g. $\hat r$}) which represents a double precision variable.  The multiplication on that step
is assumed to be a double wide output single precision multiplication.  That is, two single precision variables are multiplied to produce a
double precision result.  The step is somewhat optimized from a long-hand multiplication algorithm because the carry from the addition in step
5.4.1 is propagated through the nested loop.  If the carry was not propagated immediately it would overflow the single precision digit 
$t_{ix+iy}$ and the result would be lost.  

At step 5.5 the nested loop is finished and any carry that was left over should be forwarded.  The carry does not have to be added to the $ix+pb$'th
digit since that digit is assumed to be zero at this point.  However, if $ix + pb \ge digs$ the carry is not set as it would make the result
exceed the precision requested.

\vspace{+3mm}\begin{small}
\hspace{-5.1mm}{\bf File}: bn\_s\_mp\_mul\_digs.c
\vspace{-3mm}
\begin{alltt}
\end{alltt}
\end{small}

First we determine (line 31) if the Comba method can be used first since it's faster.  The conditions for 
sing the Comba routine are that min$(a.used, b.used) < \delta$ and the number of digits of output is less than 
\textbf{MP\_WARRAY}.  This new constant is used to control the stack usage in the Comba routines.  By default it is 
set to $\delta$ but can be reduced when memory is at a premium.

If we cannot use the Comba method we proceed to setup the baseline routine.  We allocate the the destination mp\_int
$t$ (line 37) to the exact size of the output to avoid further re--allocations.  At this point we now 
begin the $O(n^2)$ loop.

This implementation of multiplication has the caveat that it can be trimmed to only produce a variable number of
digits as output.  In each iteration of the outer loop the $pb$ variable is set (line 49) to the maximum 
number of inner loop iterations.  

Inside the inner loop we calculate $\hat r$ as the mp\_word product of the two mp\_digits and the addition of the
carry from the previous iteration.  A particularly important observation is that most modern optimizing 
C compilers (GCC for instance) can recognize that a $N \times N \rightarrow 2N$ multiplication is all that 
is required for the product.  In x86 terms for example, this means using the MUL instruction.

Each digit of the product is stored in turn (line 69) and the carry propagated (line 72) to the 
next iteration.

\subsection{Faster Multiplication by the ``Comba'' Method}

One of the huge drawbacks of the ``baseline'' algorithms is that at the $O(n^2)$ level the carry must be 
computed and propagated upwards.  This makes the nested loop very sequential and hard to unroll and implement 
in parallel.  The ``Comba'' \cite{COMBA} method is named after little known (\textit{in cryptographic venues}) Paul G. 
Comba who described a method of implementing fast multipliers that do not require nested carry fixup operations.  As an 
interesting aside it seems that Paul Barrett describes a similar technique in his 1986 paper \cite{BARRETT} written 
five years before.

At the heart of the Comba technique is once again the long-hand algorithm.  Except in this case a slight 
twist is placed on how the columns of the result are produced.  In the standard long-hand algorithm rows of products 
are produced then added together to form the final result.  In the baseline algorithm the columns are added together 
after each iteration to get the result instantaneously.  

In the Comba algorithm the columns of the result are produced entirely independently of each other.  That is at 
the $O(n^2)$ level a simple multiplication and addition step is performed.  The carries of the columns are propagated 
after the nested loop to reduce the amount of work requiored. Succintly the first step of the algorithm is to compute 
the product vector $\vec x$ as follows. 

\begin{equation}
\vec x_n = \sum_{i+j = n} a_ib_j, \forall n \in \lbrace 0, 1, 2, \ldots, i + j \rbrace
\end{equation}

Where $\vec x_n$ is the $n'th$ column of the output vector.  Consider the following example which computes the vector $\vec x$ for the multiplication
of $576$ and $241$.  

\newpage\begin{figure}[here]
\begin{small}
\begin{center}
\begin{tabular}{|c|c|c|c|c|c|}
  \hline &          & 5 & 7 & 6 & First Input\\
  \hline $\times$ & & 2 & 4 & 1 & Second Input\\
\hline            &                        & $1 \cdot 5 = 5$   & $1 \cdot 7 = 7$   & $1 \cdot 6 = 6$ & First pass \\
                  &  $4 \cdot 5 = 20$      & $4 \cdot 7+5=33$  & $4 \cdot 6+7=31$  & 6               & Second pass \\
   $2 \cdot 5 = 10$ &  $2 \cdot 7 + 20 = 34$ & $2 \cdot 6+33=45$ & 31                & 6             & Third pass \\
\hline 10 & 34 & 45 & 31 & 6 & Final Result \\   
\hline   
\end{tabular}
\end{center}
\end{small}
\caption{Comba Multiplication Diagram}
\end{figure}

At this point the vector $x = \left < 10, 34, 45, 31, 6 \right >$ is the result of the first step of the Comba multipler.  
Now the columns must be fixed by propagating the carry upwards.  The resultant vector will have one extra dimension over the input vector which is
congruent to adding a leading zero digit.

\begin{figure}[!here]
\begin{small}
\begin{center}
\begin{tabular}{l}
\hline Algorithm \textbf{Comba Fixup}. \\
\textbf{Input}.   Vector $\vec x$ of dimension $k$ \\
\textbf{Output}.  Vector $\vec x$ such that the carries have been propagated. \\
\hline \\
1.  for $n$ from $0$ to $k - 1$ do \\
\hspace{3mm}1.1 $\vec x_{n+1} \leftarrow \vec x_{n+1} + \lfloor \vec x_{n}/\beta \rfloor$ \\
\hspace{3mm}1.2 $\vec x_{n} \leftarrow \vec x_{n} \mbox{ (mod }\beta\mbox{)}$ \\
2.  Return($\vec x$). \\
\hline
\end{tabular}
\end{center}
\end{small}
\caption{Algorithm Comba Fixup}
\end{figure}

With that algorithm and $k = 5$ and $\beta = 10$ the following vector is produced $\vec x= \left < 1, 3, 8, 8, 1, 6 \right >$.  In this case 
$241 \cdot 576$ is in fact $138816$ and the procedure succeeded.  If the algorithm is correct and as will be demonstrated shortly more
efficient than the baseline algorithm why not simply always use this algorithm?

\subsubsection{Column Weight.}
At the nested $O(n^2)$ level the Comba method adds the product of two single precision variables to each column of the output 
independently.  A serious obstacle is if the carry is lost, due to lack of precision before the algorithm has a chance to fix
the carries.  For example, in the multiplication of two three-digit numbers the third column of output will be the sum of
three single precision multiplications.  If the precision of the accumulator for the output digits is less then $3 \cdot (\beta - 1)^2$ then
an overflow can occur and the carry information will be lost.  For any $m$ and $n$ digit inputs the maximum weight of any column is 
min$(m, n)$ which is fairly obvious.

The maximum number of terms in any column of a product is known as the ``column weight'' and strictly governs when the algorithm can be used.  Recall
from earlier that a double precision type has $\alpha$ bits of resolution and a single precision digit has $lg(\beta)$ bits of precision.  Given these
two quantities we must not violate the following

\begin{equation}
k \cdot \left (\beta - 1 \right )^2 < 2^{\alpha}
\end{equation}

Which reduces to 

\begin{equation}
k \cdot \left ( \beta^2 - 2\beta + 1 \right ) < 2^{\alpha}
\end{equation}

Let $\rho = lg(\beta)$ represent the number of bits in a single precision digit.  By further re-arrangement of the equation the final solution is
found.

\begin{equation}
k  < {{2^{\alpha}} \over {\left (2^{2\rho} - 2^{\rho + 1} + 1 \right )}}
\end{equation}

The defaults for LibTomMath are $\beta = 2^{28}$ and $\alpha = 2^{64}$ which means that $k$ is bounded by $k < 257$.  In this configuration 
the smaller input may not have more than $256$ digits if the Comba method is to be used.  This is quite satisfactory for most applications since 
$256$ digits would allow for numbers in the range of $0 \le x < 2^{7168}$ which, is much larger than most public key cryptographic algorithms require.  

\newpage\begin{figure}[!here]
\begin{small}
\begin{center}
\begin{tabular}{l}
\hline Algorithm \textbf{fast\_s\_mp\_mul\_digs}. \\
\textbf{Input}.   mp\_int $a$, mp\_int $b$ and an integer $digs$ \\
\textbf{Output}.  $c \leftarrow \vert a \vert \cdot \vert b \vert \mbox{ (mod }\beta^{digs}\mbox{)}$. \\
\hline \\
Place an array of \textbf{MP\_WARRAY} single precision digits named $W$ on the stack. \\
1.  If $c.alloc < digs$ then grow $c$ to $digs$ digits. (\textit{mp\_grow}) \\
2.  If step 1 failed return(\textit{MP\_MEM}).\\
\\
3.  $pa \leftarrow \mbox{MIN}(digs, a.used + b.used)$ \\
\\
4.  $\_ \hat W \leftarrow 0$ \\
5.  for $ix$ from 0 to $pa - 1$ do \\
\hspace{3mm}5.1  $ty \leftarrow \mbox{MIN}(b.used - 1, ix)$ \\
\hspace{3mm}5.2  $tx \leftarrow ix - ty$ \\
\hspace{3mm}5.3  $iy \leftarrow \mbox{MIN}(a.used - tx, ty + 1)$ \\
\hspace{3mm}5.4  for $iz$ from 0 to $iy - 1$ do \\
\hspace{6mm}5.4.1  $\_ \hat W \leftarrow \_ \hat W + a_{tx+iy}b_{ty-iy}$ \\
\hspace{3mm}5.5  $W_{ix} \leftarrow \_ \hat W (\mbox{mod }\beta)$\\
\hspace{3mm}5.6  $\_ \hat W \leftarrow \lfloor \_ \hat W / \beta \rfloor$ \\
\\
6.  $oldused \leftarrow c.used$ \\
7.  $c.used \leftarrow digs$ \\
8.  for $ix$ from $0$ to $pa$ do \\
\hspace{3mm}8.1  $c_{ix} \leftarrow W_{ix}$ \\
9.  for $ix$ from $pa + 1$ to $oldused - 1$ do \\
\hspace{3mm}9.1 $c_{ix} \leftarrow 0$ \\
\\
10.  Clamp $c$. \\
11.  Return MP\_OKAY. \\
\hline
\end{tabular}
\end{center}
\end{small}
\caption{Algorithm fast\_s\_mp\_mul\_digs}
\label{fig:COMBAMULT}
\end{figure}

\textbf{Algorithm fast\_s\_mp\_mul\_digs.}
This algorithm performs the unsigned multiplication of $a$ and $b$ using the Comba method limited to $digs$ digits of precision.

The outer loop of this algorithm is more complicated than that of the baseline multiplier.  This is because on the inside of the 
loop we want to produce one column per pass.  This allows the accumulator $\_ \hat W$ to be placed in CPU registers and
reduce the memory bandwidth to two \textbf{mp\_digit} reads per iteration.

The $ty$ variable is set to the minimum count of $ix$ or the number of digits in $b$.  That way if $a$ has more digits than
$b$ this will be limited to $b.used - 1$.  The $tx$ variable is set to the to the distance past $b.used$ the variable
$ix$ is.  This is used for the immediately subsequent statement where we find $iy$.  

The variable $iy$ is the minimum digits we can read from either $a$ or $b$ before running out.  Computing one column at a time
means we have to scan one integer upwards and the other downwards.  $a$ starts at $tx$ and $b$ starts at $ty$.  In each
pass we are producing the $ix$'th output column and we note that $tx + ty = ix$.  As we move $tx$ upwards we have to 
move $ty$ downards so the equality remains valid.  The $iy$ variable is the number of iterations until 
$tx \ge a.used$ or $ty < 0$ occurs.

After every inner pass we store the lower half of the accumulator into $W_{ix}$ and then propagate the carry of the accumulator
into the next round by dividing $\_ \hat W$ by $\beta$.

To measure the benefits of the Comba method over the baseline method consider the number of operations that are required.  If the 
cost in terms of time of a multiply and addition is $p$ and the cost of a carry propagation is $q$ then a baseline multiplication would require 
$O \left ((p + q)n^2 \right )$ time to multiply two $n$-digit numbers.  The Comba method requires only $O(pn^2 + qn)$ time, however in practice, 
the speed increase is actually much more.  With $O(n)$ space the algorithm can be reduced to $O(pn + qn)$ time by implementing the $n$ multiply
and addition operations in the nested loop in parallel.  

\vspace{+3mm}\begin{small}
\hspace{-5.1mm}{\bf File}: bn\_fast\_s\_mp\_mul\_digs.c
\vspace{-3mm}
\begin{alltt}
\end{alltt}
\end{small}

As per the pseudo--code we first calculate $pa$ (line 48) as the number of digits to output.  Next we begin the outer loop
to produce the individual columns of the product.  We use the two aliases $tmpx$ and $tmpy$ (lines 62, 63) to point
inside the two multiplicands quickly.  

The inner loop (lines 71 to 74) of this implementation is where the tradeoff come into play.  Originally this comba 
implementation was ``row--major'' which means it adds to each of the columns in each pass.  After the outer loop it would then fix 
the carries.  This was very fast except it had an annoying drawback.  You had to read a mp\_word and two mp\_digits and write 
one mp\_word per iteration.  On processors such as the Athlon XP and P4 this did not matter much since the cache bandwidth 
is very high and it can keep the ALU fed with data.  It did, however, matter on older and embedded cpus where cache is often 
slower and also often doesn't exist.  This new algorithm only performs two reads per iteration under the assumption that the 
compiler has aliased $\_ \hat W$ to a CPU register.

After the inner loop we store the current accumulator in $W$ and shift $\_ \hat W$ (lines 77, 80) to forward it as 
a carry for the next pass.  After the outer loop we use the final carry (line 77) as the last digit of the product.  

\subsection{Polynomial Basis Multiplication}
To break the $O(n^2)$ barrier in multiplication requires a completely different look at integer multiplication.  In the following algorithms
the use of polynomial basis representation for two integers $a$ and $b$ as $f(x) = \sum_{i=0}^{n} a_i x^i$ and  
$g(x) = \sum_{i=0}^{n} b_i x^i$ respectively, is required.  In this system both $f(x)$ and $g(x)$ have $n + 1$ terms and are of the $n$'th degree.
 
The product $a \cdot b \equiv f(x)g(x)$ is the polynomial $W(x) = \sum_{i=0}^{2n} w_i x^i$.  The coefficients $w_i$ will
directly yield the desired product when $\beta$ is substituted for $x$.  The direct solution to solve for the $2n + 1$ coefficients
requires $O(n^2)$ time and would in practice be slower than the Comba technique.

However, numerical analysis theory indicates that only $2n + 1$ distinct points in $W(x)$ are required to determine the values of the $2n + 1$ unknown 
coefficients.   This means by finding $\zeta_y = W(y)$ for $2n + 1$ small values of $y$ the coefficients of $W(x)$ can be found with 
Gaussian elimination.  This technique is also occasionally refered to as the \textit{interpolation technique} (\textit{references please...}) since in 
effect an interpolation based on $2n + 1$ points will yield a polynomial equivalent to $W(x)$.  

The coefficients of the polynomial $W(x)$ are unknown which makes finding $W(y)$ for any value of $y$ impossible.  However, since 
$W(x) = f(x)g(x)$ the equivalent $\zeta_y = f(y) g(y)$ can be used in its place.  The benefit of this technique stems from the 
fact that $f(y)$ and $g(y)$ are much smaller than either $a$ or $b$ respectively.  As a result finding the $2n + 1$ relations required 
by multiplying $f(y)g(y)$ involves multiplying integers that are much smaller than either of the inputs.

When picking points to gather relations there are always three obvious points to choose, $y = 0, 1$ and $ \infty$.  The $\zeta_0$ term
is simply the product $W(0) = w_0 = a_0 \cdot b_0$.  The $\zeta_1$ term is the product 
$W(1) = \left (\sum_{i = 0}^{n} a_i \right ) \left (\sum_{i = 0}^{n} b_i \right )$.  The third point $\zeta_{\infty}$ is less obvious but rather
simple to explain.  The $2n + 1$'th coefficient of $W(x)$ is numerically equivalent to the most significant column in an integer multiplication.  
The point at $\infty$ is used symbolically to represent the most significant column, that is $W(\infty) = w_{2n} = a_nb_n$.  Note that the 
points at $y = 0$ and $\infty$ yield the coefficients $w_0$ and $w_{2n}$ directly.

If more points are required they should be of small values and powers of two such as $2^q$ and the related \textit{mirror points} 
$\left (2^q \right )^{2n}  \cdot \zeta_{2^{-q}}$ for small values of $q$.  The term ``mirror point'' stems from the fact that 
$\left (2^q \right )^{2n}  \cdot \zeta_{2^{-q}}$ can be calculated in the exact opposite fashion as $\zeta_{2^q}$.  For
example, when $n = 2$ and $q = 1$ then following two equations are equivalent to the point $\zeta_{2}$ and its mirror.

\begin{eqnarray}
\zeta_{2}                  = f(2)g(2) = (4a_2 + 2a_1 + a_0)(4b_2 + 2b_1 + b_0) \nonumber \\
16 \cdot \zeta_{1 \over 2} = 4f({1\over 2}) \cdot 4g({1 \over 2}) = (a_2 + 2a_1 + 4a_0)(b_2 + 2b_1 + 4b_0)
\end{eqnarray}

Using such points will allow the values of $f(y)$ and $g(y)$ to be independently calculated using only left shifts.  For example, when $n = 2$ the
polynomial $f(2^q)$ is equal to $2^q((2^qa_2) + a_1) + a_0$.  This technique of polynomial representation is known as Horner's method.  

As a general rule of the algorithm when the inputs are split into $n$ parts each there are $2n - 1$ multiplications.  Each multiplication is of 
multiplicands that have $n$ times fewer digits than the inputs.  The asymptotic running time of this algorithm is 
$O \left ( k^{lg_n(2n - 1)} \right )$ for $k$ digit inputs (\textit{assuming they have the same number of digits}).  Figure~\ref{fig:exponent}
summarizes the exponents for various values of $n$.

\begin{figure}
\begin{center}
\begin{tabular}{|c|c|c|}
\hline \textbf{Split into $n$ Parts} & \textbf{Exponent}  & \textbf{Notes}\\
\hline $2$ & $1.584962501$ & This is Karatsuba Multiplication. \\
\hline $3$ & $1.464973520$ & This is Toom-Cook Multiplication. \\
\hline $4$ & $1.403677461$ &\\
\hline $5$ & $1.365212389$ &\\
\hline $10$ & $1.278753601$ &\\
\hline $100$ & $1.149426538$ &\\
\hline $1000$ & $1.100270931$ &\\
\hline $10000$ & $1.075252070$ &\\
\hline
\end{tabular}
\end{center}
\caption{Asymptotic Running Time of Polynomial Basis Multiplication}
\label{fig:exponent}
\end{figure}

At first it may seem like a good idea to choose $n = 1000$ since the exponent is approximately $1.1$.  However, the overhead
of solving for the 2001 terms of $W(x)$ will certainly consume any savings the algorithm could offer for all but exceedingly large
numbers.  

\subsubsection{Cutoff Point}
The polynomial basis multiplication algorithms all require fewer single precision multiplications than a straight Comba approach.  However, 
the algorithms incur an overhead (\textit{at the $O(n)$ work level}) since they require a system of equations to be solved.  This makes the
polynomial basis approach more costly to use with small inputs.

Let $m$ represent the number of digits in the multiplicands (\textit{assume both multiplicands have the same number of digits}).  There exists a 
point $y$ such that when $m < y$ the polynomial basis algorithms are more costly than Comba, when $m = y$ they are roughly the same cost and 
when $m > y$ the Comba methods are slower than the polynomial basis algorithms.  

The exact location of $y$ depends on several key architectural elements of the computer platform in question.

\begin{enumerate}
\item  The ratio of clock cycles for single precision multiplication versus other simpler operations such as addition, shifting, etc.  For example
on the AMD Athlon the ratio is roughly $17 : 1$ while on the Intel P4 it is $29 : 1$.  The higher the ratio in favour of multiplication the lower
the cutoff point $y$ will be.  

\item  The complexity of the linear system of equations (\textit{for the coefficients of $W(x)$}) is.  Generally speaking as the number of splits
grows the complexity grows substantially.  Ideally solving the system will only involve addition, subtraction and shifting of integers.  This
directly reflects on the ratio previous mentioned.

\item  To a lesser extent memory bandwidth and function call overheads.  Provided the values are in the processor cache this is less of an
influence over the cutoff point.

\end{enumerate}

A clean cutoff point separation occurs when a point $y$ is found such that all of the cutoff point conditions are met.  For example, if the point
is too low then there will be values of $m$ such that $m > y$ and the Comba method is still faster.  Finding the cutoff points is fairly simple when
a high resolution timer is available.  

\subsection{Karatsuba Multiplication}
Karatsuba \cite{KARA} multiplication when originally proposed in 1962 was among the first set of algorithms to break the $O(n^2)$ barrier for
general purpose multiplication.  Given two polynomial basis representations $f(x) = ax + b$ and $g(x) = cx + d$, Karatsuba proved with 
light algebra \cite{KARAP} that the following polynomial is equivalent to multiplication of the two integers the polynomials represent.

\begin{equation}
f(x) \cdot g(x) = acx^2 + ((a + b)(c + d) - (ac + bd))x + bd
\end{equation}

Using the observation that $ac$ and $bd$ could be re-used only three half sized multiplications would be required to produce the product.  Applying
this algorithm recursively, the work factor becomes $O(n^{lg(3)})$ which is substantially better than the work factor $O(n^2)$ of the Comba technique.  It turns 
out what Karatsuba did not know or at least did not publish was that this is simply polynomial basis multiplication with the points 
$\zeta_0$, $\zeta_{\infty}$ and $\zeta_{1}$.  Consider the resultant system of equations.

\begin{center}
\begin{tabular}{rcrcrcrc}
$\zeta_{0}$ &      $=$ &  &  &  & & $w_0$ \\
$\zeta_{1}$ &      $=$ & $w_2$ & $+$ & $w_1$ & $+$ & $w_0$ \\
$\zeta_{\infty}$ & $=$ & $w_2$ &  & &  & \\
\end{tabular}
\end{center}

By adding the first and last equation to the equation in the middle the term $w_1$ can be isolated and all three coefficients solved for.  The simplicity
of this system of equations has made Karatsuba fairly popular.  In fact the cutoff point is often fairly low\footnote{With LibTomMath 0.18 it is 70 and 109 digits for the Intel P4 and AMD Athlon respectively.}
making it an ideal algorithm to speed up certain public key cryptosystems such as RSA and Diffie-Hellman.  

\newpage\begin{figure}[!here]
\begin{small}
\begin{center}
\begin{tabular}{l}
\hline Algorithm \textbf{mp\_karatsuba\_mul}. \\
\textbf{Input}.   mp\_int $a$ and mp\_int $b$ \\
\textbf{Output}.  $c \leftarrow \vert a \vert \cdot \vert b \vert$ \\
\hline \\
1.  Init the following mp\_int variables: $x0$, $x1$, $y0$, $y1$, $t1$, $x0y0$, $x1y1$.\\
2.  If step 2 failed then return(\textit{MP\_MEM}). \\
\\
Split the input.  e.g. $a = x1 \cdot \beta^B + x0$ \\
3.  $B \leftarrow \mbox{min}(a.used, b.used)/2$ \\
4.  $x0 \leftarrow a \mbox{ (mod }\beta^B\mbox{)}$ (\textit{mp\_mod\_2d}) \\
5.  $y0 \leftarrow b \mbox{ (mod }\beta^B\mbox{)}$ \\
6.  $x1 \leftarrow \lfloor a / \beta^B \rfloor$ (\textit{mp\_rshd}) \\
7.  $y1 \leftarrow \lfloor b / \beta^B \rfloor$ \\
\\
Calculate the three products. \\
8.  $x0y0 \leftarrow x0 \cdot y0$ (\textit{mp\_mul}) \\
9.  $x1y1 \leftarrow x1 \cdot y1$ \\
10.  $t1 \leftarrow x1 + x0$ (\textit{mp\_add}) \\
11.  $x0 \leftarrow y1 + y0$ \\
12.  $t1 \leftarrow t1 \cdot x0$ \\
\\
Calculate the middle term. \\
13.  $x0 \leftarrow x0y0 + x1y1$ \\
14.  $t1 \leftarrow t1 - x0$ (\textit{s\_mp\_sub}) \\
\\
Calculate the final product. \\
15.  $t1 \leftarrow t1 \cdot \beta^B$ (\textit{mp\_lshd}) \\
16.  $x1y1 \leftarrow x1y1 \cdot \beta^{2B}$ \\
17.  $t1 \leftarrow x0y0 + t1$ \\
18.  $c \leftarrow t1 + x1y1$ \\
19.  Clear all of the temporary variables. \\
20.  Return(\textit{MP\_OKAY}).\\
\hline 
\end{tabular}
\end{center}
\end{small}
\caption{Algorithm mp\_karatsuba\_mul}
\end{figure}

\textbf{Algorithm mp\_karatsuba\_mul.}
This algorithm computes the unsigned product of two inputs using the Karatsuba multiplication algorithm.  It is loosely based on the description
from Knuth \cite[pp. 294-295]{TAOCPV2}.  

\index{radix point}
In order to split the two inputs into their respective halves, a suitable \textit{radix point} must be chosen.  The radix point chosen must
be used for both of the inputs meaning that it must be smaller than the smallest input.  Step 3 chooses the radix point $B$ as half of the 
smallest input \textbf{used} count.  After the radix point is chosen the inputs are split into lower and upper halves.  Step 4 and 5 
compute the lower halves.  Step 6 and 7 computer the upper halves.  

After the halves have been computed the three intermediate half-size products must be computed.  Step 8 and 9 compute the trivial products
$x0 \cdot y0$ and $x1 \cdot y1$.  The mp\_int $x0$ is used as a temporary variable after $x1 + x0$ has been computed.  By using $x0$ instead
of an additional temporary variable, the algorithm can avoid an addition memory allocation operation.

The remaining steps 13 through 18 compute the Karatsuba polynomial through a variety of digit shifting and addition operations.

\vspace{+3mm}\begin{small}
\hspace{-5.1mm}{\bf File}: bn\_mp\_karatsuba\_mul.c
\vspace{-3mm}
\begin{alltt}
\end{alltt}
\end{small}

The new coding element in this routine, not  seen in previous routines, is the usage of goto statements.  The conventional
wisdom is that goto statements should be avoided.  This is generally true, however when every single function call can fail, it makes sense
to handle error recovery with a single piece of code.  Lines 62 to 76 handle initializing all of the temporary variables 
required.  Note how each of the if statements goes to a different label in case of failure.  This allows the routine to correctly free only
the temporaries that have been successfully allocated so far.

The temporary variables are all initialized using the mp\_init\_size routine since they are expected to be large.  This saves the 
additional reallocation that would have been necessary.  Also $x0$, $x1$, $y0$ and $y1$ have to be able to hold at least their respective
number of digits for the next section of code.

The first algebraic portion of the algorithm is to split the two inputs into their halves.  However, instead of using mp\_mod\_2d and mp\_rshd
to extract the halves, the respective code has been placed inline within the body of the function.  To initialize the halves, the \textbf{used} and 
\textbf{sign} members are copied first.  The first for loop on line 96 copies the lower halves.  Since they are both the same magnitude it 
is simpler to calculate both lower halves in a single loop.  The for loop on lines 102 and 107 calculate the upper halves $x1$ and 
$y1$ respectively.

By inlining the calculation of the halves, the Karatsuba multiplier has a slightly lower overhead and can be used for smaller magnitude inputs.

When line 151 is reached, the algorithm has completed succesfully.  The ``error status'' variable $err$ is set to \textbf{MP\_OKAY} so that
the same code that handles errors can be used to clear the temporary variables and return.  

\subsection{Toom-Cook $3$-Way Multiplication}
Toom-Cook $3$-Way \cite{TOOM} multiplication is essentially the polynomial basis algorithm for $n = 2$ except that the points  are 
chosen such that $\zeta$ is easy to compute and the resulting system of equations easy to reduce.  Here, the points $\zeta_{0}$, 
$16 \cdot \zeta_{1 \over 2}$, $\zeta_1$, $\zeta_2$ and $\zeta_{\infty}$ make up the five required points to solve for the coefficients 
of the $W(x)$.

With the five relations that Toom-Cook specifies, the following system of equations is formed.

\begin{center}
\begin{tabular}{rcrcrcrcrcr}
$\zeta_0$                    & $=$ & $0w_4$ & $+$ & $0w_3$ & $+$ & $0w_2$ & $+$ & $0w_1$ & $+$ & $1w_0$  \\
$16 \cdot \zeta_{1 \over 2}$ & $=$ & $1w_4$ & $+$ & $2w_3$ & $+$ & $4w_2$ & $+$ & $8w_1$ & $+$ & $16w_0$  \\
$\zeta_1$                    & $=$ & $1w_4$ & $+$ & $1w_3$ & $+$ & $1w_2$ & $+$ & $1w_1$ & $+$ & $1w_0$  \\
$\zeta_2$                    & $=$ & $16w_4$ & $+$ & $8w_3$ & $+$ & $4w_2$ & $+$ & $2w_1$ & $+$ & $1w_0$  \\
$\zeta_{\infty}$             & $=$ & $1w_4$ & $+$ & $0w_3$ & $+$ & $0w_2$ & $+$ & $0w_1$ & $+$ & $0w_0$  \\
\end{tabular}
\end{center}

A trivial solution to this matrix requires $12$ subtractions, two multiplications by a small power of two, two divisions by a small power
of two, two divisions by three and one multiplication by three.  All of these $19$ sub-operations require less than quadratic time, meaning that
the algorithm can be faster than a baseline multiplication.  However, the greater complexity of this algorithm places the cutoff point
(\textbf{TOOM\_MUL\_CUTOFF}) where Toom-Cook becomes more efficient much higher than the Karatsuba cutoff point.  

\begin{figure}[!here]
\begin{small}
\begin{center}
\begin{tabular}{l}
\hline Algorithm \textbf{mp\_toom\_mul}. \\
\textbf{Input}.   mp\_int $a$ and mp\_int $b$ \\
\textbf{Output}.  $c \leftarrow  a  \cdot  b $ \\
\hline \\
Split $a$ and $b$ into three pieces.  E.g. $a = a_2 \beta^{2k} + a_1 \beta^{k} + a_0$ \\
1.  $k \leftarrow \lfloor \mbox{min}(a.used, b.used) / 3 \rfloor$ \\
2.  $a_0 \leftarrow a \mbox{ (mod }\beta^{k}\mbox{)}$ \\
3.  $a_1 \leftarrow \lfloor a / \beta^k \rfloor$, $a_1 \leftarrow a_1 \mbox{ (mod }\beta^{k}\mbox{)}$ \\
4.  $a_2 \leftarrow \lfloor a / \beta^{2k} \rfloor$, $a_2 \leftarrow a_2 \mbox{ (mod }\beta^{k}\mbox{)}$ \\
5.  $b_0 \leftarrow a \mbox{ (mod }\beta^{k}\mbox{)}$ \\
6.  $b_1 \leftarrow \lfloor a / \beta^k \rfloor$, $b_1 \leftarrow b_1 \mbox{ (mod }\beta^{k}\mbox{)}$ \\
7.  $b_2 \leftarrow \lfloor a / \beta^{2k} \rfloor$, $b_2 \leftarrow b_2 \mbox{ (mod }\beta^{k}\mbox{)}$ \\
\\
Find the five equations for $w_0, w_1, ..., w_4$. \\
8.  $w_0 \leftarrow a_0 \cdot b_0$ \\
9.  $w_4 \leftarrow a_2 \cdot b_2$ \\
10. $tmp_1 \leftarrow 2 \cdot a_0$, $tmp_1 \leftarrow a_1 + tmp_1$, $tmp_1 \leftarrow 2 \cdot tmp_1$, $tmp_1 \leftarrow tmp_1 + a_2$ \\
11. $tmp_2 \leftarrow 2 \cdot b_0$, $tmp_2 \leftarrow b_1 + tmp_2$, $tmp_2 \leftarrow 2 \cdot tmp_2$, $tmp_2 \leftarrow tmp_2 + b_2$ \\
12. $w_1 \leftarrow tmp_1 \cdot tmp_2$ \\
13. $tmp_1 \leftarrow 2 \cdot a_2$, $tmp_1 \leftarrow a_1 + tmp_1$, $tmp_1 \leftarrow 2 \cdot tmp_1$, $tmp_1 \leftarrow tmp_1 + a_0$ \\
14. $tmp_2 \leftarrow 2 \cdot b_2$, $tmp_2 \leftarrow b_1 + tmp_2$, $tmp_2 \leftarrow 2 \cdot tmp_2$, $tmp_2 \leftarrow tmp_2 + b_0$ \\
15. $w_3 \leftarrow tmp_1 \cdot tmp_2$ \\
16. $tmp_1 \leftarrow a_0 + a_1$, $tmp_1 \leftarrow tmp_1 + a_2$, $tmp_2 \leftarrow b_0 + b_1$, $tmp_2 \leftarrow tmp_2 + b_2$ \\
17. $w_2 \leftarrow tmp_1 \cdot tmp_2$ \\
\\
Continued on the next page.\\
\hline
\end{tabular}
\end{center}
\end{small}
\caption{Algorithm mp\_toom\_mul}
\end{figure}

\newpage\begin{figure}[!here]
\begin{small}
\begin{center}
\begin{tabular}{l}
\hline Algorithm \textbf{mp\_toom\_mul} (continued). \\
\textbf{Input}.   mp\_int $a$ and mp\_int $b$ \\
\textbf{Output}.  $c \leftarrow a \cdot  b $ \\
\hline \\
Now solve the system of equations. \\
18. $w_1 \leftarrow w_4 - w_1$, $w_3 \leftarrow w_3 - w_0$ \\
19. $w_1 \leftarrow \lfloor w_1 / 2 \rfloor$, $w_3 \leftarrow \lfloor w_3 / 2 \rfloor$ \\
20. $w_2 \leftarrow w_2 - w_0$, $w_2 \leftarrow w_2 - w_4$ \\
21. $w_1 \leftarrow w_1 - w_2$, $w_3 \leftarrow w_3 - w_2$ \\
22. $tmp_1 \leftarrow 8 \cdot w_0$, $w_1 \leftarrow w_1 - tmp_1$, $tmp_1 \leftarrow 8 \cdot w_4$, $w_3 \leftarrow w_3 - tmp_1$ \\
23. $w_2 \leftarrow 3 \cdot w_2$, $w_2 \leftarrow w_2 - w_1$, $w_2 \leftarrow w_2 - w_3$ \\
24. $w_1 \leftarrow w_1 - w_2$, $w_3 \leftarrow w_3 - w_2$ \\
25. $w_1 \leftarrow \lfloor w_1 / 3 \rfloor, w_3 \leftarrow \lfloor w_3 / 3 \rfloor$ \\
\\
Now substitute $\beta^k$ for $x$ by shifting $w_0, w_1, ..., w_4$. \\
26. for $n$ from $1$ to $4$ do \\
\hspace{3mm}26.1  $w_n \leftarrow w_n \cdot \beta^{nk}$ \\
27. $c \leftarrow w_0 + w_1$, $c \leftarrow c + w_2$, $c \leftarrow c + w_3$, $c \leftarrow c + w_4$ \\
28. Return(\textit{MP\_OKAY}) \\
\hline
\end{tabular}
\end{center}
\end{small}
\caption{Algorithm mp\_toom\_mul (continued)}
\end{figure}

\textbf{Algorithm mp\_toom\_mul.}
This algorithm computes the product of two mp\_int variables $a$ and $b$ using the Toom-Cook approach.  Compared to the Karatsuba multiplication, this 
algorithm has a lower asymptotic running time of approximately $O(n^{1.464})$ but at an obvious cost in overhead.  In this
description, several statements have been compounded to save space.  The intention is that the statements are executed from left to right across
any given step.

The two inputs $a$ and $b$ are first split into three $k$-digit integers $a_0, a_1, a_2$ and $b_0, b_1, b_2$ respectively.  From these smaller
integers the coefficients of the polynomial basis representations $f(x)$ and $g(x)$ are known and can be used to find the relations required.

The first two relations $w_0$ and $w_4$ are the points $\zeta_{0}$ and $\zeta_{\infty}$ respectively.  The relation $w_1, w_2$ and $w_3$ correspond
to the points $16 \cdot \zeta_{1 \over 2}, \zeta_{2}$ and $\zeta_{1}$ respectively.  These are found using logical shifts to independently find
$f(y)$ and $g(y)$ which significantly speeds up the algorithm.

After the five relations $w_0, w_1, \ldots, w_4$ have been computed, the system they represent must be solved in order for the unknown coefficients 
$w_1, w_2$ and $w_3$ to be isolated.  The steps 18 through 25 perform the system reduction required as previously described.  Each step of
the reduction represents the comparable matrix operation that would be performed had this been performed by pencil.  For example, step 18 indicates
that row $1$ must be subtracted from row $4$ and simultaneously row $0$ subtracted from row $3$.  

Once the coeffients have been isolated, the polynomial $W(x) = \sum_{i=0}^{2n} w_i x^i$ is known.  By substituting $\beta^{k}$ for $x$, the integer 
result $a \cdot b$ is produced.

\vspace{+3mm}\begin{small}
\hspace{-5.1mm}{\bf File}: bn\_mp\_toom\_mul.c
\vspace{-3mm}
\begin{alltt}
\end{alltt}
\end{small}

The first obvious thing to note is that this algorithm is complicated.  The complexity is worth it if you are multiplying very 
large numbers.  For example, a 10,000 digit multiplication takes approximaly 99,282,205 fewer single precision multiplications with
Toom--Cook than a Comba or baseline approach (this is a savings of more than 99$\%$).  For most ``crypto'' sized numbers this
algorithm is not practical as Karatsuba has a much lower cutoff point.

First we split $a$ and $b$ into three roughly equal portions.  This has been accomplished (lines 41 to 70) with 
combinations of mp\_rshd() and mp\_mod\_2d() function calls.  At this point $a = a2 \cdot \beta^2 + a1 \cdot \beta + a0$ and similiarly
for $b$.  

Next we compute the five points $w0, w1, w2, w3$ and $w4$.  Recall that $w0$ and $w4$ can be computed directly from the portions so
we get those out of the way first (lines 73 and 78).  Next we compute $w1, w2$ and $w3$ using Horners method.

After this point we solve for the actual values of $w1, w2$ and $w3$ by reducing the $5 \times 5$ system which is relatively
straight forward.  

\subsection{Signed Multiplication}
Now that algorithms to handle multiplications of every useful dimensions have been developed, a rather simple finishing touch is required.  So far all
of the multiplication algorithms have been unsigned multiplications which leaves only a signed multiplication algorithm to be established.  

\begin{figure}[!here]
\begin{small}
\begin{center}
\begin{tabular}{l}
\hline Algorithm \textbf{mp\_mul}. \\
\textbf{Input}.   mp\_int $a$ and mp\_int $b$ \\
\textbf{Output}.  $c \leftarrow a \cdot b$ \\
\hline \\
1.  If $a.sign = b.sign$ then \\
\hspace{3mm}1.1  $sign = MP\_ZPOS$ \\
2.  else \\
\hspace{3mm}2.1  $sign = MP\_ZNEG$ \\
3.  If min$(a.used, b.used) \ge TOOM\_MUL\_CUTOFF$ then  \\
\hspace{3mm}3.1  $c \leftarrow a \cdot b$ using algorithm mp\_toom\_mul \\
4.  else if min$(a.used, b.used) \ge KARATSUBA\_MUL\_CUTOFF$ then \\
\hspace{3mm}4.1  $c \leftarrow a \cdot b$ using algorithm mp\_karatsuba\_mul \\
5.  else \\
\hspace{3mm}5.1  $digs \leftarrow a.used + b.used + 1$ \\
\hspace{3mm}5.2  If $digs < MP\_ARRAY$ and min$(a.used, b.used) \le \delta$ then \\
\hspace{6mm}5.2.1  $c \leftarrow a \cdot b \mbox{ (mod }\beta^{digs}\mbox{)}$ using algorithm fast\_s\_mp\_mul\_digs.  \\
\hspace{3mm}5.3  else \\
\hspace{6mm}5.3.1  $c \leftarrow a \cdot b \mbox{ (mod }\beta^{digs}\mbox{)}$ using algorithm s\_mp\_mul\_digs.  \\
6.  $c.sign \leftarrow sign$ \\
7.  Return the result of the unsigned multiplication performed. \\
\hline
\end{tabular}
\end{center}
\end{small}
\caption{Algorithm mp\_mul}
\end{figure}

\textbf{Algorithm mp\_mul.}
This algorithm performs the signed multiplication of two inputs.  It will make use of any of the three unsigned multiplication algorithms 
available when the input is of appropriate size.  The \textbf{sign} of the result is not set until the end of the algorithm since algorithm
s\_mp\_mul\_digs will clear it.  

\vspace{+3mm}\begin{small}
\hspace{-5.1mm}{\bf File}: bn\_mp\_mul.c
\vspace{-3mm}
\begin{alltt}
\end{alltt}
\end{small}

The implementation is rather simplistic and is not particularly noteworthy.  Line 22 computes the sign of the result using the ``?'' 
operator from the C programming language.  Line 48 computes $\delta$ using the fact that $1 << k$ is equal to $2^k$.  

\section{Squaring}
\label{sec:basesquare}

Squaring is a special case of multiplication where both multiplicands are equal.  At first it may seem like there is no significant optimization
available but in fact there is.  Consider the multiplication of $576$ against $241$.  In total there will be nine single precision multiplications
performed which are $1\cdot 6$, $1 \cdot 7$, $1 \cdot 5$, $4 \cdot 6$, $4 \cdot 7$, $4 \cdot 5$, $2 \cdot  6$, $2 \cdot 7$ and $2 \cdot 5$.  Now consider 
the multiplication of $123$ against $123$.  The nine products are $3 \cdot 3$, $3 \cdot 2$, $3 \cdot 1$, $2 \cdot 3$, $2 \cdot 2$, $2 \cdot 1$, 
$1 \cdot 3$, $1 \cdot 2$ and $1 \cdot 1$.  On closer inspection some of the products are equivalent.  For example, $3 \cdot 2 = 2 \cdot 3$ 
and $3 \cdot 1 = 1 \cdot 3$. 

For any $n$-digit input, there are ${{\left (n^2 + n \right)}\over 2}$ possible unique single precision multiplications required compared to the $n^2$
required for multiplication.  The following diagram gives an example of the operations required.

\begin{figure}[here]
\begin{center}
\begin{tabular}{ccccc|c}
&&1&2&3&\\
$\times$ &&1&2&3&\\
\hline && $3 \cdot 1$ & $3 \cdot 2$ & $3 \cdot 3$ & Row 0\\
       & $2 \cdot 1$  & $2 \cdot 2$ & $2 \cdot 3$ && Row 1 \\
         $1 \cdot 1$  & $1 \cdot 2$ & $1 \cdot 3$ &&& Row 2 \\
\end{tabular}
\end{center}
\caption{Squaring Optimization Diagram}
\end{figure}

Starting from zero and numbering the columns from right to left a very simple pattern becomes obvious.  For the purposes of this discussion let $x$
represent the number being squared.  The first observation is that in row $k$ the $2k$'th column of the product has a $\left (x_k \right)^2$ term in it.  

The second observation is that every column $j$ in row $k$ where $j \ne 2k$ is part of a double product.  Every non-square term of a column will
appear twice hence the name ``double product''.  Every odd column is made up entirely of double products.  In fact every column is made up of double 
products and at most one square (\textit{see the exercise section}).  

The third and final observation is that for row $k$ the first unique non-square term, that is, one that hasn't already appeared in an earlier row, 
occurs at column $2k + 1$.  For example, on row $1$ of the previous squaring, column one is part of the double product with column one from row zero. 
Column two of row one is a square and column three is the first unique column.

\subsection{The Baseline Squaring Algorithm}
The baseline squaring algorithm is meant to be a catch-all squaring algorithm.  It will handle any of the input sizes that the faster routines
will not handle.  

\begin{figure}[!here]
\begin{small}
\begin{center}
\begin{tabular}{l}
\hline Algorithm \textbf{s\_mp\_sqr}. \\
\textbf{Input}.   mp\_int $a$ \\
\textbf{Output}.  $b \leftarrow a^2$ \\
\hline \\
1.  Init a temporary mp\_int of at least $2 \cdot a.used +1$ digits.  (\textit{mp\_init\_size}) \\
2.  If step 1 failed return(\textit{MP\_MEM}) \\
3.  $t.used \leftarrow 2 \cdot a.used + 1$ \\
4.  For $ix$ from 0 to $a.used - 1$ do \\
\hspace{3mm}Calculate the square. \\
\hspace{3mm}4.1  $\hat r \leftarrow t_{2ix} + \left (a_{ix} \right )^2$ \\
\hspace{3mm}4.2  $t_{2ix} \leftarrow \hat r \mbox{ (mod }\beta\mbox{)}$ \\
\hspace{3mm}Calculate the double products after the square. \\
\hspace{3mm}4.3  $u \leftarrow \lfloor \hat r / \beta \rfloor$ \\
\hspace{3mm}4.4  For $iy$ from $ix + 1$ to $a.used - 1$ do \\
\hspace{6mm}4.4.1  $\hat r \leftarrow 2 \cdot a_{ix}a_{iy} + t_{ix + iy} + u$ \\
\hspace{6mm}4.4.2  $t_{ix + iy} \leftarrow \hat r \mbox{ (mod }\beta\mbox{)}$ \\
\hspace{6mm}4.4.3  $u \leftarrow \lfloor \hat r / \beta \rfloor$ \\
\hspace{3mm}Set the last carry. \\
\hspace{3mm}4.5  While $u > 0$ do \\
\hspace{6mm}4.5.1  $iy \leftarrow iy + 1$ \\
\hspace{6mm}4.5.2  $\hat r \leftarrow t_{ix + iy} + u$ \\
\hspace{6mm}4.5.3  $t_{ix + iy} \leftarrow \hat r \mbox{ (mod }\beta\mbox{)}$ \\
\hspace{6mm}4.5.4  $u \leftarrow \lfloor \hat r / \beta \rfloor$ \\
5.  Clamp excess digits of $t$.  (\textit{mp\_clamp}) \\
6.  Exchange $b$ and $t$. \\
7.  Clear $t$ (\textit{mp\_clear}) \\
8.  Return(\textit{MP\_OKAY}) \\
\hline
\end{tabular}
\end{center}
\end{small}
\caption{Algorithm s\_mp\_sqr}
\end{figure}

\textbf{Algorithm s\_mp\_sqr.}
This algorithm computes the square of an input using the three observations on squaring.  It is based fairly faithfully on  algorithm 14.16 of HAC
\cite[pp.596-597]{HAC}.  Similar to algorithm s\_mp\_mul\_digs, a temporary mp\_int is allocated to hold the result of the squaring.  This allows the 
destination mp\_int to be the same as the source mp\_int.

The outer loop of this algorithm begins on step 4. It is best to think of the outer loop as walking down the rows of the partial results, while
the inner loop computes the columns of the partial result.  Step 4.1 and 4.2 compute the square term for each row, and step 4.3 and 4.4 propagate
the carry and compute the double products.  

The requirement that a mp\_word be able to represent the range $0 \le x < 2 \beta^2$ arises from this
very algorithm.  The product $a_{ix}a_{iy}$ will lie in the range $0 \le x \le \beta^2 - 2\beta + 1$ which is obviously less than $\beta^2$ meaning that
when it is multiplied by two, it can be properly represented by a mp\_word.

Similar to algorithm s\_mp\_mul\_digs, after every pass of the inner loop, the destination is correctly set to the sum of all of the partial 
results calculated so far.  This involves expensive carry propagation which will be eliminated in the next algorithm.  

\vspace{+3mm}\begin{small}
\hspace{-5.1mm}{\bf File}: bn\_s\_mp\_sqr.c
\vspace{-3mm}
\begin{alltt}
\end{alltt}
\end{small}

Inside the outer loop (line 34) the square term is calculated on line 37.  The carry (line 44) has been
extracted from the mp\_word accumulator using a right shift.  Aliases for $a_{ix}$ and $t_{ix+iy}$ are initialized 
(lines 47 and 50) to simplify the inner loop.  The doubling is performed using two
additions (line 59) since it is usually faster than shifting, if not at least as fast.  

The important observation is that the inner loop does not begin at $iy = 0$ like for multiplication.  As such the inner loops
get progressively shorter as the algorithm proceeds.  This is what leads to the savings compared to using a multiplication to
square a number. 

\subsection{Faster Squaring by the ``Comba'' Method}
A major drawback to the baseline method is the requirement for single precision shifting inside the $O(n^2)$ nested loop.  Squaring has an additional
drawback that it must double the product inside the inner loop as well.  As for multiplication, the Comba technique can be used to eliminate these
performance hazards.

The first obvious solution is to make an array of mp\_words which will hold all of the columns.  This will indeed eliminate all of the carry
propagation operations from the inner loop.  However, the inner product must still be doubled $O(n^2)$ times.  The solution stems from the simple fact
that $2a + 2b + 2c = 2(a + b + c)$.  That is the sum of all of the double products is equal to double the sum of all the products.  For example,
$ab + ba + ac + ca = 2ab + 2ac = 2(ab + ac)$.  

However, we cannot simply double all of the columns, since the squares appear only once per row.  The most practical solution is to have two 
mp\_word arrays.  One array will hold the squares and the other array will hold the double products.  With both arrays the doubling and 
carry propagation can be moved to a $O(n)$ work level outside the $O(n^2)$ level.  In this case, we have an even simpler solution in mind.

\newpage\begin{figure}[!here]
\begin{small}
\begin{center}
\begin{tabular}{l}
\hline Algorithm \textbf{fast\_s\_mp\_sqr}. \\
\textbf{Input}.   mp\_int $a$ \\
\textbf{Output}.  $b \leftarrow a^2$ \\
\hline \\
Place an array of \textbf{MP\_WARRAY} mp\_digits named $W$ on the stack. \\
1.  If $b.alloc < 2a.used + 1$ then grow $b$ to $2a.used + 1$ digits.  (\textit{mp\_grow}). \\
2.  If step 1 failed return(\textit{MP\_MEM}). \\
\\
3.  $pa \leftarrow 2 \cdot a.used$ \\
4.  $\hat W1 \leftarrow 0$ \\
5.  for $ix$ from $0$ to $pa - 1$ do \\
\hspace{3mm}5.1  $\_ \hat W \leftarrow 0$ \\
\hspace{3mm}5.2  $ty \leftarrow \mbox{MIN}(a.used - 1, ix)$ \\
\hspace{3mm}5.3  $tx \leftarrow ix - ty$ \\
\hspace{3mm}5.4  $iy \leftarrow \mbox{MIN}(a.used - tx, ty + 1)$ \\
\hspace{3mm}5.5  $iy \leftarrow \mbox{MIN}(iy, \lfloor \left (ty - tx + 1 \right )/2 \rfloor)$ \\
\hspace{3mm}5.6  for $iz$ from $0$ to $iz - 1$ do \\
\hspace{6mm}5.6.1  $\_ \hat W \leftarrow \_ \hat W + a_{tx + iz}a_{ty - iz}$ \\
\hspace{3mm}5.7  $\_ \hat W \leftarrow 2 \cdot \_ \hat W  + \hat W1$ \\
\hspace{3mm}5.8  if $ix$ is even then \\
\hspace{6mm}5.8.1  $\_ \hat W \leftarrow \_ \hat W + \left ( a_{\lfloor ix/2 \rfloor}\right )^2$ \\
\hspace{3mm}5.9  $W_{ix} \leftarrow \_ \hat W (\mbox{mod }\beta)$ \\
\hspace{3mm}5.10  $\hat W1 \leftarrow \lfloor \_ \hat W / \beta \rfloor$ \\
\\
6.  $oldused \leftarrow b.used$ \\
7.  $b.used \leftarrow 2 \cdot a.used$ \\
8.  for $ix$ from $0$ to $pa - 1$ do \\
\hspace{3mm}8.1  $b_{ix} \leftarrow W_{ix}$ \\
9.  for $ix$ from $pa$ to $oldused - 1$ do \\
\hspace{3mm}9.1  $b_{ix} \leftarrow 0$ \\
10.  Clamp excess digits from $b$.  (\textit{mp\_clamp}) \\
11.  Return(\textit{MP\_OKAY}). \\ 
\hline
\end{tabular}
\end{center}
\end{small}
\caption{Algorithm fast\_s\_mp\_sqr}
\end{figure}

\textbf{Algorithm fast\_s\_mp\_sqr.}
This algorithm computes the square of an input using the Comba technique.  It is designed to be a replacement for algorithm 
s\_mp\_sqr when the number of input digits is less than \textbf{MP\_WARRAY} and less than $\delta \over 2$.  
This algorithm is very similar to the Comba multiplier except with a few key differences we shall make note of.

First, we have an accumulator and carry variables $\_ \hat W$ and $\hat W1$ respectively.  This is because the inner loop
products are to be doubled.  If we had added the previous carry in we would be doubling too much.  Next we perform an
addition MIN condition on $iy$ (step 5.5) to prevent overlapping digits.  For example, $a_3 \cdot a_5$ is equal
$a_5 \cdot a_3$.  Whereas in the multiplication case we would have $5 < a.used$ and $3 \ge 0$ is maintained since we double the sum
of the products just outside the inner loop we have to avoid doing this.  This is also a good thing since we perform
fewer multiplications and the routine ends up being faster.

Finally the last difference is the addition of the ``square'' term outside the inner loop (step 5.8).  We add in the square
only to even outputs and it is the square of the term at the $\lfloor ix / 2 \rfloor$ position.

\vspace{+3mm}\begin{small}
\hspace{-5.1mm}{\bf File}: bn\_fast\_s\_mp\_sqr.c
\vspace{-3mm}
\begin{alltt}
\end{alltt}
\end{small}

This implementation is essentially a copy of Comba multiplication with the appropriate changes added to make it faster for 
the special case of squaring.  

\subsection{Polynomial Basis Squaring}
The same algorithm that performs optimal polynomial basis multiplication can be used to perform polynomial basis squaring.  The minor exception
is that $\zeta_y = f(y)g(y)$ is actually equivalent to $\zeta_y = f(y)^2$ since $f(y) = g(y)$.  Instead of performing $2n + 1$
multiplications to find the $\zeta$ relations, squaring operations are performed instead.  

\subsection{Karatsuba Squaring}
Let $f(x) = ax + b$ represent the polynomial basis representation of a number to square.  
Let $h(x) = \left ( f(x) \right )^2$ represent the square of the polynomial.  The Karatsuba equation can be modified to square a 
number with the following equation.

\begin{equation}
h(x) = a^2x^2 + \left ((a + b)^2 - (a^2 + b^2) \right )x + b^2
\end{equation}

Upon closer inspection this equation only requires the calculation of three half-sized squares: $a^2$, $b^2$ and $(a + b)^2$.  As in 
Karatsuba multiplication, this algorithm can be applied recursively on the input and will achieve an asymptotic running time of 
$O \left ( n^{lg(3)} \right )$.

If the asymptotic times of Karatsuba squaring and multiplication are the same, why not simply use the multiplication algorithm 
instead?  The answer to this arises from the cutoff point for squaring.  As in multiplication there exists a cutoff point, at which the 
time required for a Comba based squaring and a Karatsuba based squaring meet.  Due to the overhead inherent in the Karatsuba method, the cutoff 
point is fairly high.  For example, on an AMD Athlon XP processor with $\beta = 2^{28}$, the cutoff point is around 127 digits.  

Consider squaring a 200 digit number with this technique.  It will be split into two 100 digit halves which are subsequently squared.  
The 100 digit halves will not be squared using Karatsuba, but instead using the faster Comba based squaring algorithm.  If Karatsuba multiplication
were used instead, the 100 digit numbers would be squared with a slower Comba based multiplication.  

\newpage\begin{figure}[!here]
\begin{small}
\begin{center}
\begin{tabular}{l}
\hline Algorithm \textbf{mp\_karatsuba\_sqr}. \\
\textbf{Input}.   mp\_int $a$ \\
\textbf{Output}.  $b \leftarrow a^2$ \\
\hline \\
1.  Initialize the following temporary mp\_ints:  $x0$, $x1$, $t1$, $t2$, $x0x0$ and $x1x1$. \\
2.  If any of the initializations on step 1 failed return(\textit{MP\_MEM}). \\
\\
Split the input.  e.g. $a = x1\beta^B + x0$ \\
3.  $B \leftarrow \lfloor a.used / 2 \rfloor$ \\
4.  $x0 \leftarrow a \mbox{ (mod }\beta^B\mbox{)}$ (\textit{mp\_mod\_2d}) \\
5.  $x1 \leftarrow \lfloor a / \beta^B \rfloor$ (\textit{mp\_lshd}) \\
\\
Calculate the three squares. \\
6.  $x0x0 \leftarrow x0^2$ (\textit{mp\_sqr}) \\
7.  $x1x1 \leftarrow x1^2$ \\
8.  $t1 \leftarrow x1 + x0$ (\textit{s\_mp\_add}) \\
9.  $t1 \leftarrow t1^2$ \\
\\
Compute the middle term. \\
10.  $t2 \leftarrow x0x0 + x1x1$ (\textit{s\_mp\_add}) \\
11.  $t1 \leftarrow t1 - t2$ \\
\\
Compute final product. \\
12.  $t1 \leftarrow t1\beta^B$ (\textit{mp\_lshd}) \\
13.  $x1x1 \leftarrow x1x1\beta^{2B}$ \\
14.  $t1 \leftarrow t1 + x0x0$ \\
15.  $b \leftarrow t1 + x1x1$ \\
16.  Return(\textit{MP\_OKAY}). \\
\hline
\end{tabular}
\end{center}
\end{small}
\caption{Algorithm mp\_karatsuba\_sqr}
\end{figure}

\textbf{Algorithm mp\_karatsuba\_sqr.}
This algorithm computes the square of an input $a$ using the Karatsuba technique.  This algorithm is very similar to the Karatsuba based
multiplication algorithm with the exception that the three half-size multiplications have been replaced with three half-size squarings.

The radix point for squaring is simply placed exactly in the middle of the digits when the input has an odd number of digits, otherwise it is
placed just below the middle.  Step 3, 4 and 5 compute the two halves required using $B$
as the radix point.  The first two squares in steps 6 and 7 are rather straightforward while the last square is of a more compact form.

By expanding $\left (x1 + x0 \right )^2$, the $x1^2$ and $x0^2$ terms in the middle disappear, that is $(x0 - x1)^2 - (x1^2 + x0^2)  = 2 \cdot x0 \cdot x1$.
Now if $5n$ single precision additions and a squaring of $n$-digits is faster than multiplying two $n$-digit numbers and doubling then
this method is faster.  Assuming no further recursions occur, the difference can be estimated with the following inequality.

Let $p$ represent the cost of a single precision addition and $q$ the cost of a single precision multiplication both in terms of time\footnote{Or
machine clock cycles.}. 

\begin{equation}
5pn +{{q(n^2 + n)} \over 2} \le pn + qn^2
\end{equation}

For example, on an AMD Athlon XP processor $p = {1 \over 3}$ and $q = 6$.  This implies that the following inequality should hold.
\begin{center}
\begin{tabular}{rcl}
${5n \over 3} + 3n^2 + 3n$     & $<$ & ${n \over 3} + 6n^2$ \\
${5 \over 3} + 3n + 3$     & $<$ & ${1 \over 3} + 6n$ \\
${13 \over 9}$     & $<$ & $n$ \\
\end{tabular}
\end{center}

This results in a cutoff point around $n = 2$.  As a consequence it is actually faster to compute the middle term the ``long way'' on processors
where multiplication is substantially slower\footnote{On the Athlon there is a 1:17 ratio between clock cycles for addition and multiplication.  On
the Intel P4 processor this ratio is 1:29 making this method even more beneficial.  The only common exception is the ARMv4 processor which has a
ratio of 1:7.  } than simpler operations such as addition.  

\vspace{+3mm}\begin{small}
\hspace{-5.1mm}{\bf File}: bn\_mp\_karatsuba\_sqr.c
\vspace{-3mm}
\begin{alltt}
\end{alltt}
\end{small}

This implementation is largely based on the implementation of algorithm mp\_karatsuba\_mul.  It uses the same inline style to copy and 
shift the input into the two halves.  The loop from line 54 to line 70 has been modified since only one input exists.  The \textbf{used}
count of both $x0$ and $x1$ is fixed up and $x0$ is clamped before the calculations begin.  At this point $x1$ and $x0$ are valid equivalents
to the respective halves as if mp\_rshd and mp\_mod\_2d had been used.  

By inlining the copy and shift operations the cutoff point for Karatsuba multiplication can be lowered.  On the Athlon the cutoff point
is exactly at the point where Comba squaring can no longer be used (\textit{128 digits}).  On slower processors such as the Intel P4
it is actually below the Comba limit (\textit{at 110 digits}).

This routine uses the same error trap coding style as mp\_karatsuba\_sqr.  As the temporary variables are initialized errors are 
redirected to the error trap higher up.  If the algorithm completes without error the error code is set to \textbf{MP\_OKAY} and 
mp\_clears are executed normally.

\subsection{Toom-Cook Squaring}
The Toom-Cook squaring algorithm mp\_toom\_sqr is heavily based on the algorithm mp\_toom\_mul with the exception that squarings are used
instead of multiplication to find the five relations.  The reader is encouraged to read the description of the latter algorithm and try to 
derive their own Toom-Cook squaring algorithm.  

\subsection{High Level Squaring}
\newpage\begin{figure}[!here]
\begin{small}
\begin{center}
\begin{tabular}{l}
\hline Algorithm \textbf{mp\_sqr}. \\
\textbf{Input}.   mp\_int $a$ \\
\textbf{Output}.  $b \leftarrow a^2$ \\
\hline \\
1.  If $a.used \ge TOOM\_SQR\_CUTOFF$ then  \\
\hspace{3mm}1.1  $b \leftarrow a^2$ using algorithm mp\_toom\_sqr \\
2.  else if $a.used \ge KARATSUBA\_SQR\_CUTOFF$ then \\
\hspace{3mm}2.1  $b \leftarrow a^2$ using algorithm mp\_karatsuba\_sqr \\
3.  else \\
\hspace{3mm}3.1  $digs \leftarrow a.used + b.used + 1$ \\
\hspace{3mm}3.2  If $digs < MP\_ARRAY$ and $a.used \le \delta$ then \\
\hspace{6mm}3.2.1  $b \leftarrow a^2$ using algorithm fast\_s\_mp\_sqr.  \\
\hspace{3mm}3.3  else \\
\hspace{6mm}3.3.1  $b \leftarrow a^2$ using algorithm s\_mp\_sqr.  \\
4.  $b.sign \leftarrow MP\_ZPOS$ \\
5.  Return the result of the unsigned squaring performed. \\
\hline
\end{tabular}
\end{center}
\end{small}
\caption{Algorithm mp\_sqr}
\end{figure}

\textbf{Algorithm mp\_sqr.}
This algorithm computes the square of the input using one of four different algorithms.  If the input is very large and has at least
\textbf{TOOM\_SQR\_CUTOFF} or \textbf{KARATSUBA\_SQR\_CUTOFF} digits then either the Toom-Cook or the Karatsuba Squaring algorithm is used.  If
neither of the polynomial basis algorithms should be used then either the Comba or baseline algorithm is used.  

\vspace{+3mm}\begin{small}
\hspace{-5.1mm}{\bf File}: bn\_mp\_sqr.c
\vspace{-3mm}
\begin{alltt}
\end{alltt}
\end{small}

\section*{Exercises}
\begin{tabular}{cl}
$\left [ 3 \right ] $ & Devise an efficient algorithm for selection of the radix point to handle inputs \\
                      & that have different number of digits in Karatsuba multiplication. \\
                      & \\
$\left [ 2 \right ] $ & In section 5.3 the fact that every column of a squaring is made up \\
                      & of double products and at most one square is stated.  Prove this statement. \\
                      & \\                      
$\left [ 3 \right ] $ & Prove the equation for Karatsuba squaring. \\
                      & \\
$\left [ 1 \right ] $ & Prove that Karatsuba squaring requires $O \left (n^{lg(3)} \right )$ time. \\
                      & \\ 
$\left [ 2 \right ] $ & Determine the minimal ratio between addition and multiplication clock cycles \\
                      & required for equation $6.7$ to be true.  \\
                      & \\
$\left [ 3 \right ] $ & Implement a threaded version of Comba multiplication (and squaring) where you \\
                      & compute subsets of the columns in each thread.  Determine a cutoff point where \\
                      & it is effective and add the logic to mp\_mul() and mp\_sqr(). \\
                      &\\
$\left [ 4 \right ] $ & Same as the previous but also modify the Karatsuba and Toom-Cook.  You must \\
                      & increase the throughput of mp\_exptmod() for random odd moduli in the range \\
                      & $512 \ldots 4096$ bits significantly ($> 2x$) to complete this challenge. \\
                      & \\
\end{tabular}

\chapter{Modular Reduction}
\section{Basics of Modular Reduction}
\index{modular residue}
Modular reduction is an operation that arises quite often within public key cryptography algorithms and various number theoretic algorithms, 
such as factoring.  Modular reduction algorithms are the third class of algorithms of the ``multipliers'' set.  A number $a$ is said to be \textit{reduced}
modulo another number $b$ by finding the remainder of the division $a/b$.  Full integer division with remainder is a topic to be covered 
in~\ref{sec:division}.

Modular reduction is equivalent to solving for $r$ in the following equation.  $a = bq + r$ where $q = \lfloor a/b \rfloor$.  The result 
$r$ is said to be ``congruent to $a$ modulo $b$'' which is also written as $r \equiv a \mbox{ (mod }b\mbox{)}$.  In other vernacular $r$ is known as the 
``modular residue'' which leads to ``quadratic residue''\footnote{That's fancy talk for $b \equiv a^2 \mbox{ (mod }p\mbox{)}$.} and
other forms of residues.  

Modular reductions are normally used to create either finite groups, rings or fields.  The most common usage for performance driven modular reductions 
is in modular exponentiation algorithms.  That is to compute $d = a^b \mbox{ (mod }c\mbox{)}$ as fast as possible.  This operation is used in the 
RSA and Diffie-Hellman public key algorithms, for example.  Modular multiplication and squaring also appears as a fundamental operation in 
elliptic curve cryptographic algorithms.  As will be discussed in the subsequent chapter there exist fast algorithms for computing modular 
exponentiations without having to perform (\textit{in this example}) $b - 1$ multiplications.  These algorithms will produce partial results in the 
range $0 \le x < c^2$ which can be taken advantage of to create several efficient algorithms.   They have also been used to create redundancy check 
algorithms known as CRCs, error correction codes such as Reed-Solomon and solve a variety of number theoeretic problems.  

\section{The Barrett Reduction}
The Barrett reduction algorithm \cite{BARRETT} was inspired by fast division algorithms which multiply by the reciprocal to emulate
division.  Barretts observation was that the residue $c$ of $a$ modulo $b$ is equal to 

\begin{equation}
c = a - b \cdot \lfloor a/b \rfloor
\end{equation}

Since algorithms such as modular exponentiation would be using the same modulus extensively, typical DSP\footnote{It is worth noting that Barrett's paper 
targeted the DSP56K processor.}  intuition would indicate the next step would be to replace $a/b$ by a multiplication by the reciprocal.  However, 
DSP intuition on its own will not work as these numbers are considerably larger than the precision of common DSP floating point data types.  
It would take another common optimization to optimize the algorithm.

\subsection{Fixed Point Arithmetic}
The trick used to optimize the above equation is based on a technique of emulating floating point data types with fixed precision integers.  Fixed
point arithmetic would become very popular as it greatly optimize the ``3d-shooter'' genre of games in the mid 1990s when floating point units were 
fairly slow if not unavailable.   The idea behind fixed point arithmetic is to take a normal $k$-bit integer data type and break it into $p$-bit 
integer and a $q$-bit fraction part (\textit{where $p+q = k$}).  

In this system a $k$-bit integer $n$ would actually represent $n/2^q$.  For example, with $q = 4$ the integer $n = 37$ would actually represent the
value $2.3125$.  To multiply two fixed point numbers the integers are multiplied using traditional arithmetic and subsequently normalized by 
moving the implied decimal point back to where it should be.  For example, with $q = 4$ to multiply the integers $9$ and $5$ they must be converted 
to fixed point first by multiplying by $2^q$.  Let $a = 9(2^q)$ represent the fixed point representation of $9$ and $b = 5(2^q)$ represent the 
fixed point representation of $5$.  The product $ab$ is equal to $45(2^{2q})$ which when normalized by dividing by $2^q$ produces $45(2^q)$.  

This technique became popular since a normal integer multiplication and logical shift right are the only required operations to perform a multiplication
of two fixed point numbers.  Using fixed point arithmetic, division can be easily approximated by multiplying by the reciprocal.  If $2^q$ is 
equivalent to one than $2^q/b$ is equivalent to the fixed point approximation of $1/b$ using real arithmetic.  Using this fact dividing an integer 
$a$ by another integer $b$ can be achieved with the following expression.

\begin{equation}
\lfloor a / b \rfloor \mbox{ }\approx\mbox{ } \lfloor (a \cdot \lfloor 2^q / b \rfloor)/2^q \rfloor
\end{equation}

The precision of the division is proportional to the value of $q$.  If the divisor $b$ is used frequently as is the case with 
modular exponentiation pre-computing $2^q/b$ will allow a division to be performed with a multiplication and a right shift.  Both operations
are considerably faster than division on most processors.  

Consider dividing $19$ by $5$.  The correct result is $\lfloor 19/5 \rfloor = 3$.  With $q = 3$ the reciprocal is $\lfloor 2^q/5 \rfloor = 1$ which
leads to a product of $19$ which when divided by $2^q$ produces $2$.  However, with $q = 4$ the reciprocal is $\lfloor 2^q/5 \rfloor = 3$ and
the result of the emulated division is $\lfloor 3 \cdot 19 / 2^q \rfloor = 3$ which is correct.  The value of $2^q$ must be close to or ideally
larger than the dividend.  In effect if $a$ is the dividend then $q$ should allow $0 \le \lfloor a/2^q \rfloor \le 1$ in order for this approach
to work correctly.  Plugging this form of divison into the original equation the following modular residue equation arises.

\begin{equation}
c = a - b \cdot \lfloor (a \cdot \lfloor 2^q / b \rfloor)/2^q \rfloor
\end{equation}

Using the notation from \cite{BARRETT} the value of $\lfloor 2^q / b \rfloor$ will be represented by the $\mu$ symbol.  Using the $\mu$
variable also helps re-inforce the idea that it is meant to be computed once and re-used.

\begin{equation}
c = a - b \cdot \lfloor (a \cdot \mu)/2^q \rfloor
\end{equation}

Provided that $2^q \ge a$ this algorithm will produce a quotient that is either exactly correct or off by a value of one.  In the context of Barrett
reduction the value of $a$ is bound by $0 \le a \le (b - 1)^2$ meaning that $2^q \ge b^2$ is sufficient to ensure the reciprocal will have enough
precision.  

Let $n$ represent the number of digits in $b$.  This algorithm requires approximately $2n^2$ single precision multiplications to produce the quotient and 
another $n^2$ single precision multiplications to find the residue.  In total $3n^2$ single precision multiplications are required to 
reduce the number.  

For example, if $b = 1179677$ and $q = 41$ ($2^q > b^2$), then the reciprocal $\mu$ is equal to $\lfloor 2^q / b \rfloor = 1864089$.  Consider reducing
$a = 180388626447$ modulo $b$ using the above reduction equation.  The quotient using the new formula is $\lfloor (a \cdot \mu) / 2^q \rfloor = 152913$.
By subtracting $152913b$ from $a$ the correct residue $a \equiv 677346 \mbox{ (mod }b\mbox{)}$ is found.

\subsection{Choosing a Radix Point}
Using the fixed point representation a modular reduction can be performed with $3n^2$ single precision multiplications.  If that were the best
that could be achieved a full division\footnote{A division requires approximately $O(2cn^2)$ single precision multiplications for a small value of $c$.  
See~\ref{sec:division} for further details.} might as well be used in its place.  The key to optimizing the reduction is to reduce the precision of
the initial multiplication that finds the quotient.  

Let $a$ represent the number of which the residue is sought.  Let $b$ represent the modulus used to find the residue.  Let $m$ represent
the number of digits in $b$.  For the purposes of this discussion we will assume that the number of digits in $a$ is $2m$, which is generally true if 
two $m$-digit numbers have been multiplied.  Dividing $a$ by $b$ is the same as dividing a $2m$ digit integer by a $m$ digit integer.  Digits below the 
$m - 1$'th digit of $a$ will contribute at most a value of $1$ to the quotient because $\beta^k < b$ for any $0 \le k \le m - 1$.  Another way to
express this is by re-writing $a$ as two parts.  If $a' \equiv a \mbox{ (mod }b^m\mbox{)}$ and $a'' = a - a'$ then 
${a \over b} \equiv {{a' + a''} \over b}$ which is equivalent to ${a' \over b} + {a'' \over b}$.  Since $a'$ is bound to be less than $b$ the quotient
is bound by $0 \le {a' \over b} < 1$.

Since the digits of $a'$ do not contribute much to the quotient the observation is that they might as well be zero.  However, if the digits 
``might as well be zero'' they might as well not be there in the first place.  Let $q_0 = \lfloor a/\beta^{m-1} \rfloor$ represent the input
with the irrelevant digits trimmed.  Now the modular reduction is trimmed to the almost equivalent equation

\begin{equation}
c = a - b \cdot \lfloor (q_0 \cdot \mu) / \beta^{m+1} \rfloor
\end{equation}

Note that the original divisor $2^q$ has been replaced with $\beta^{m+1}$ where in this case $q$ is a multiple of $lg(\beta)$. Also note that the 
exponent on the divisor when added to the amount $q_0$ was shifted by equals $2m$.  If the optimization had not been performed the divisor 
would have the exponent $2m$ so in the end the exponents do ``add up''. Using the above equation the quotient 
$\lfloor (q_0 \cdot \mu) / \beta^{m+1} \rfloor$ can be off from the true quotient by at most two.  The original fixed point quotient can be off
by as much as one (\textit{provided the radix point is chosen suitably}) and now that the lower irrelevent digits have been trimmed the quotient
can be off by an additional value of one for a total of at most two.  This implies that 
$0 \le a - b \cdot \lfloor (q_0 \cdot \mu) / \beta^{m+1} \rfloor < 3b$.  By first subtracting $b$ times the quotient and then conditionally subtracting 
$b$ once or twice the residue is found.

The quotient is now found using $(m + 1)(m) = m^2 + m$ single precision multiplications and the residue with an additional $m^2$ single
precision multiplications, ignoring the subtractions required.  In total $2m^2 + m$ single precision multiplications are required to find the residue.  
This is considerably faster than the original attempt.

For example, let $\beta = 10$ represent the radix of the digits.  Let $b = 9999$ represent the modulus which implies $m = 4$. Let $a = 99929878$ 
represent the value of which the residue is desired.  In this case $q = 8$ since $10^7 < 9999^2$ meaning that $\mu = \lfloor \beta^{q}/b \rfloor = 10001$.  
With the new observation the multiplicand for the quotient is equal to $q_0 = \lfloor a / \beta^{m - 1} \rfloor = 99929$.  The quotient is then 
$\lfloor (q_0 \cdot \mu) / \beta^{m+1} \rfloor = 9993$.  Subtracting $9993b$ from $a$ and the correct residue $a \equiv 9871 \mbox{ (mod }b\mbox{)}$ 
is found.  

\subsection{Trimming the Quotient}
So far the reduction algorithm has been optimized from $3m^2$ single precision multiplications down to $2m^2 + m$ single precision multiplications.  As 
it stands now the algorithm is already fairly fast compared to a full integer division algorithm.  However, there is still room for
optimization.  

After the first multiplication inside the quotient ($q_0 \cdot \mu$) the value is shifted right by $m + 1$ places effectively nullifying the lower
half of the product.  It would be nice to be able to remove those digits from the product to effectively cut down the number of single precision 
multiplications.  If the number of digits in the modulus $m$ is far less than $\beta$ a full product is not required for the algorithm to work properly.  
In fact the lower $m - 2$ digits will not affect the upper half of the product at all and do not need to be computed.  

The value of $\mu$ is a $m$-digit number and $q_0$ is a $m + 1$ digit number.  Using a full multiplier $(m + 1)(m) = m^2 + m$ single precision
multiplications would be required.  Using a multiplier that will only produce digits at and above the $m - 1$'th digit reduces the number
of single precision multiplications to ${m^2 + m} \over 2$ single precision multiplications.  

\subsection{Trimming the Residue}
After the quotient has been calculated it is used to reduce the input.  As previously noted the algorithm is not exact and it can be off by a small
multiple of the modulus, that is $0 \le a - b \cdot \lfloor (q_0 \cdot \mu) / \beta^{m+1} \rfloor < 3b$.  If $b$ is $m$ digits than the 
result of reduction equation is a value of at most $m + 1$ digits (\textit{provided $3 < \beta$}) implying that the upper $m - 1$ digits are
implicitly zero.  

The next optimization arises from this very fact.  Instead of computing $b \cdot \lfloor (q_0 \cdot \mu) / \beta^{m+1} \rfloor$ using a full
$O(m^2)$ multiplication algorithm only the lower $m+1$ digits of the product have to be computed.  Similarly the value of $a$ can
be reduced modulo $\beta^{m+1}$ before the multiple of $b$ is subtracted which simplifes the subtraction as well.  A multiplication that produces 
only the lower $m+1$ digits requires ${m^2 + 3m - 2} \over 2$ single precision multiplications.  

With both optimizations in place the algorithm is the algorithm Barrett proposed.  It requires $m^2 + 2m - 1$ single precision multiplications which
is considerably faster than the straightforward $3m^2$ method.  

\subsection{The Barrett Algorithm}
\newpage\begin{figure}[!here]
\begin{small}
\begin{center}
\begin{tabular}{l}
\hline Algorithm \textbf{mp\_reduce}. \\
\textbf{Input}.   mp\_int $a$, mp\_int $b$ and $\mu = \lfloor \beta^{2m}/b \rfloor, m = \lceil lg_{\beta}(b) \rceil, (0 \le a < b^2, b > 1)$ \\
\textbf{Output}.  $a \mbox{ (mod }b\mbox{)}$ \\
\hline \\
Let $m$ represent the number of digits in $b$.  \\
1.  Make a copy of $a$ and store it in $q$.  (\textit{mp\_init\_copy}) \\
2.  $q \leftarrow \lfloor q / \beta^{m - 1} \rfloor$ (\textit{mp\_rshd}) \\
\\
Produce the quotient. \\
3.  $q \leftarrow q \cdot \mu$  (\textit{note: only produce digits at or above $m-1$}) \\
4.  $q \leftarrow \lfloor q / \beta^{m + 1} \rfloor$ \\
\\
Subtract the multiple of modulus from the input. \\
5.  $a \leftarrow a \mbox{ (mod }\beta^{m+1}\mbox{)}$ (\textit{mp\_mod\_2d}) \\
6.  $q \leftarrow q \cdot b \mbox{ (mod }\beta^{m+1}\mbox{)}$ (\textit{s\_mp\_mul\_digs}) \\
7.  $a \leftarrow a - q$ (\textit{mp\_sub}) \\
\\
Add $\beta^{m+1}$ if a carry occured. \\
8.  If $a < 0$ then (\textit{mp\_cmp\_d}) \\
\hspace{3mm}8.1  $q \leftarrow 1$ (\textit{mp\_set}) \\
\hspace{3mm}8.2  $q \leftarrow q \cdot \beta^{m+1}$ (\textit{mp\_lshd}) \\
\hspace{3mm}8.3  $a \leftarrow a + q$ \\
\\
Now subtract the modulus if the residue is too large (e.g. quotient too small). \\
9.  While $a \ge b$ do (\textit{mp\_cmp}) \\
\hspace{3mm}9.1  $c \leftarrow a - b$ \\
10.  Clear $q$. \\
11.  Return(\textit{MP\_OKAY}) \\
\hline
\end{tabular}
\end{center}
\end{small}
\caption{Algorithm mp\_reduce}
\end{figure}

\textbf{Algorithm mp\_reduce.}
This algorithm will reduce the input $a$ modulo $b$ in place using the Barrett algorithm.  It is loosely based on algorithm 14.42 of HAC
\cite[pp.  602]{HAC} which is based on the paper from Paul Barrett \cite{BARRETT}.  The algorithm has several restrictions and assumptions which must 
be adhered to for the algorithm to work.

First the modulus $b$ is assumed to be positive and greater than one.  If the modulus were less than or equal to one than subtracting
a multiple of it would either accomplish nothing or actually enlarge the input.  The input $a$ must be in the range $0 \le a < b^2$ in order
for the quotient to have enough precision.  If $a$ is the product of two numbers that were already reduced modulo $b$, this will not be a problem.
Technically the algorithm will still work if $a \ge b^2$ but it will take much longer to finish.  The value of $\mu$ is passed as an argument to this 
algorithm and is assumed to be calculated and stored before the algorithm is used.  

Recall that the multiplication for the quotient on step 3 must only produce digits at or above the $m-1$'th position.  An algorithm called 
$s\_mp\_mul\_high\_digs$ which has not been presented is used to accomplish this task.  The algorithm is based on $s\_mp\_mul\_digs$ except that
instead of stopping at a given level of precision it starts at a given level of precision.  This optimal algorithm can only be used if the number
of digits in $b$ is very much smaller than $\beta$.  

While it is known that 
$a \ge b \cdot \lfloor (q_0 \cdot \mu) / \beta^{m+1} \rfloor$ only the lower $m+1$ digits are being used to compute the residue, so an implied 
``borrow'' from the higher digits might leave a negative result.  After the multiple of the modulus has been subtracted from $a$ the residue must be 
fixed up in case it is negative.  The invariant $\beta^{m+1}$ must be added to the residue to make it positive again.  

The while loop at step 9 will subtract $b$ until the residue is less than $b$.  If the algorithm is performed correctly this step is 
performed at most twice, and on average once. However, if $a \ge b^2$ than it will iterate substantially more times than it should.

\vspace{+3mm}\begin{small}
\hspace{-5.1mm}{\bf File}: bn\_mp\_reduce.c
\vspace{-3mm}
\begin{alltt}
\end{alltt}
\end{small}

The first multiplication that determines the quotient can be performed by only producing the digits from $m - 1$ and up.  This essentially halves
the number of single precision multiplications required.  However, the optimization is only safe if $\beta$ is much larger than the number of digits
in the modulus.  In the source code this is evaluated on lines 36 to 44 where algorithm s\_mp\_mul\_high\_digs is used when it is
safe to do so.  

\subsection{The Barrett Setup Algorithm}
In order to use algorithm mp\_reduce the value of $\mu$ must be calculated in advance.  Ideally this value should be computed once and stored for
future use so that the Barrett algorithm can be used without delay.  

\newpage\begin{figure}[!here]
\begin{small}
\begin{center}
\begin{tabular}{l}
\hline Algorithm \textbf{mp\_reduce\_setup}. \\
\textbf{Input}.   mp\_int $a$ ($a > 1$)  \\
\textbf{Output}.  $\mu \leftarrow \lfloor \beta^{2m}/a \rfloor$ \\
\hline \\
1.  $\mu \leftarrow 2^{2 \cdot lg(\beta) \cdot  m}$ (\textit{mp\_2expt}) \\
2.  $\mu \leftarrow \lfloor \mu / b \rfloor$ (\textit{mp\_div}) \\
3.  Return(\textit{MP\_OKAY}) \\
\hline
\end{tabular}
\end{center}
\end{small}
\caption{Algorithm mp\_reduce\_setup}
\end{figure}

\textbf{Algorithm mp\_reduce\_setup.}
This algorithm computes the reciprocal $\mu$ required for Barrett reduction.  First $\beta^{2m}$ is calculated as $2^{2 \cdot lg(\beta) \cdot  m}$ which
is equivalent and much faster.  The final value is computed by taking the integer quotient of $\lfloor \mu / b \rfloor$.

\vspace{+3mm}\begin{small}
\hspace{-5.1mm}{\bf File}: bn\_mp\_reduce\_setup.c
\vspace{-3mm}
\begin{alltt}
\end{alltt}
\end{small}

This simple routine calculates the reciprocal $\mu$ required by Barrett reduction.  Note the extended usage of algorithm mp\_div where the variable
which would received the remainder is passed as NULL.  As will be discussed in~\ref{sec:division} the division routine allows both the quotient and the 
remainder to be passed as NULL meaning to ignore the value.  

\section{The Montgomery Reduction}
Montgomery reduction\footnote{Thanks to Niels Ferguson for his insightful explanation of the algorithm.} \cite{MONT} is by far the most interesting 
form of reduction in common use.  It computes a modular residue which is not actually equal to the residue of the input yet instead equal to a 
residue times a constant.  However, as perplexing as this may sound the algorithm is relatively simple and very efficient.  

Throughout this entire section the variable $n$ will represent the modulus used to form the residue.  As will be discussed shortly the value of
$n$ must be odd.  The variable $x$ will represent the quantity of which the residue is sought.  Similar to the Barrett algorithm the input
is restricted to $0 \le x < n^2$.  To begin the description some simple number theory facts must be established.

\textbf{Fact 1.}  Adding $n$ to $x$ does not change the residue since in effect it adds one to the quotient $\lfloor x / n \rfloor$.  Another way
to explain this is that $n$ is (\textit{or multiples of $n$ are}) congruent to zero modulo $n$.  Adding zero will not change the value of the residue.  

\textbf{Fact 2.}  If $x$ is even then performing a division by two in $\Z$ is congruent to $x \cdot 2^{-1} \mbox{ (mod }n\mbox{)}$.  Actually
this is an application of the fact that if $x$ is evenly divisible by any $k \in \Z$ then division in $\Z$ will be congruent to 
multiplication by $k^{-1}$ modulo $n$.  

From these two simple facts the following simple algorithm can be derived.

\newpage\begin{figure}[!here]
\begin{small}
\begin{center}
\begin{tabular}{l}
\hline Algorithm \textbf{Montgomery Reduction}. \\
\textbf{Input}.   Integer $x$, $n$ and $k$ \\
\textbf{Output}.  $2^{-k}x \mbox{ (mod }n\mbox{)}$ \\
\hline \\
1.  for $t$ from $1$ to $k$ do \\
\hspace{3mm}1.1  If $x$ is odd then \\
\hspace{6mm}1.1.1  $x \leftarrow x + n$ \\
\hspace{3mm}1.2  $x \leftarrow x/2$ \\
2.  Return $x$. \\
\hline
\end{tabular}
\end{center}
\end{small}
\caption{Algorithm Montgomery Reduction}
\end{figure}

The algorithm reduces the input one bit at a time using the two congruencies stated previously.  Inside the loop $n$, which is odd, is
added to $x$ if $x$ is odd.  This forces $x$ to be even which allows the division by two in $\Z$ to be congruent to a modular division by two.  Since
$x$ is assumed to be initially much larger than $n$ the addition of $n$ will contribute an insignificant magnitude to $x$.  Let $r$ represent the 
final result of the Montgomery algorithm.  If $k > lg(n)$ and $0 \le x < n^2$ then the final result is limited to 
$0 \le r < \lfloor x/2^k \rfloor + n$.  As a result at most a single subtraction is required to get the residue desired.

\begin{figure}[here]
\begin{small}
\begin{center}
\begin{tabular}{|c|l|}
\hline \textbf{Step number ($t$)} & \textbf{Result ($x$)} \\
\hline $1$ & $x + n = 5812$, $x/2 = 2906$ \\
\hline $2$ & $x/2 = 1453$ \\
\hline $3$ & $x + n = 1710$, $x/2 = 855$ \\
\hline $4$ & $x + n = 1112$, $x/2 = 556$ \\
\hline $5$ & $x/2 = 278$ \\
\hline $6$ & $x/2 = 139$ \\
\hline $7$ & $x + n = 396$, $x/2 = 198$ \\
\hline $8$ & $x/2 = 99$ \\
\hline $9$ & $x + n = 356$, $x/2 = 178$ \\
\hline
\end{tabular}
\end{center}
\end{small}
\caption{Example of Montgomery Reduction (I)}
\label{fig:MONT1}
\end{figure}

Consider the example in figure~\ref{fig:MONT1} which reduces $x = 5555$ modulo $n = 257$ when $k = 9$ (note $\beta^k = 512$ which is larger than $n$).  The result of 
the algorithm $r = 178$ is congruent to the value of $2^{-9} \cdot 5555 \mbox{ (mod }257\mbox{)}$.  When $r$ is multiplied by $2^9$ modulo $257$ the correct residue 
$r \equiv 158$ is produced.  

Let $k = \lfloor lg(n) \rfloor + 1$ represent the number of bits in $n$.  The current algorithm requires $2k^2$ single precision shifts
and $k^2$ single precision additions.  At this rate the algorithm is most certainly slower than Barrett reduction and not terribly useful.  
Fortunately there exists an alternative representation of the algorithm.

\begin{figure}[!here]
\begin{small}
\begin{center}
\begin{tabular}{l}
\hline Algorithm \textbf{Montgomery Reduction} (modified I). \\
\textbf{Input}.   Integer $x$, $n$ and $k$ ($2^k > n$) \\
\textbf{Output}.  $2^{-k}x \mbox{ (mod }n\mbox{)}$ \\
\hline \\
1.  for $t$ from $1$ to $k$ do \\
\hspace{3mm}1.1  If the $t$'th bit of $x$ is one then \\
\hspace{6mm}1.1.1  $x \leftarrow x + 2^tn$ \\
2.  Return $x/2^k$. \\
\hline
\end{tabular}
\end{center}
\end{small}
\caption{Algorithm Montgomery Reduction (modified I)}
\end{figure}

This algorithm is equivalent since $2^tn$ is a multiple of $n$ and the lower $k$ bits of $x$ are zero by step 2.  The number of single
precision shifts has now been reduced from $2k^2$ to $k^2 + k$ which is only a small improvement.

\begin{figure}[here]
\begin{small}
\begin{center}
\begin{tabular}{|c|l|r|}
\hline \textbf{Step number ($t$)} & \textbf{Result ($x$)} & \textbf{Result ($x$) in Binary} \\
\hline -- & $5555$ & $1010110110011$ \\
\hline $1$ & $x + 2^{0}n = 5812$ &  $1011010110100$ \\
\hline $2$ & $5812$ & $1011010110100$ \\
\hline $3$ & $x + 2^{2}n = 6840$ & $1101010111000$ \\
\hline $4$ & $x + 2^{3}n = 8896$ & $10001011000000$ \\
\hline $5$ & $8896$ & $10001011000000$ \\
\hline $6$ & $8896$ & $10001011000000$ \\
\hline $7$ & $x + 2^{6}n = 25344$ & $110001100000000$ \\
\hline $8$ & $25344$ & $110001100000000$ \\
\hline $9$ & $x + 2^{7}n = 91136$ & $10110010000000000$ \\
\hline -- & $x/2^k = 178$ & \\
\hline
\end{tabular}
\end{center}
\end{small}
\caption{Example of Montgomery Reduction (II)}
\label{fig:MONT2}
\end{figure}

Figure~\ref{fig:MONT2} demonstrates the modified algorithm reducing $x = 5555$ modulo $n = 257$ with $k = 9$. 
With this algorithm a single shift right at the end is the only right shift required to reduce the input instead of $k$ right shifts inside the 
loop.  Note that for the iterations $t = 2, 5, 6$ and $8$ where the result $x$ is not changed.  In those iterations the $t$'th bit of $x$ is 
zero and the appropriate multiple of $n$ does not need to be added to force the $t$'th bit of the result to zero.  

\subsection{Digit Based Montgomery Reduction}
Instead of computing the reduction on a bit-by-bit basis it is actually much faster to compute it on digit-by-digit basis.  Consider the
previous algorithm re-written to compute the Montgomery reduction in this new fashion.

\begin{figure}[!here]
\begin{small}
\begin{center}
\begin{tabular}{l}
\hline Algorithm \textbf{Montgomery Reduction} (modified II). \\
\textbf{Input}.   Integer $x$, $n$ and $k$ ($\beta^k > n$) \\
\textbf{Output}.  $\beta^{-k}x \mbox{ (mod }n\mbox{)}$ \\
\hline \\
1.  for $t$ from $0$ to $k - 1$ do \\
\hspace{3mm}1.1  $x \leftarrow x + \mu n \beta^t$ \\
2.  Return $x/\beta^k$. \\
\hline
\end{tabular}
\end{center}
\end{small}
\caption{Algorithm Montgomery Reduction (modified II)}
\end{figure}

The value $\mu n \beta^t$ is a multiple of the modulus $n$ meaning that it will not change the residue.  If the first digit of 
the value $\mu n \beta^t$ equals the negative (modulo $\beta$) of the $t$'th digit of $x$ then the addition will result in a zero digit.  This
problem breaks down to solving the following congruency.  

\begin{center}
\begin{tabular}{rcl}
$x_t + \mu n_0$ & $\equiv$ & $0 \mbox{ (mod }\beta\mbox{)}$ \\
$\mu n_0$ & $\equiv$ & $-x_t \mbox{ (mod }\beta\mbox{)}$ \\
$\mu$ & $\equiv$ & $-x_t/n_0 \mbox{ (mod }\beta\mbox{)}$ \\
\end{tabular}
\end{center}

In each iteration of the loop on step 1 a new value of $\mu$ must be calculated.  The value of $-1/n_0 \mbox{ (mod }\beta\mbox{)}$ is used 
extensively in this algorithm and should be precomputed.  Let $\rho$ represent the negative of the modular inverse of $n_0$ modulo $\beta$.  

For example, let $\beta = 10$ represent the radix.  Let $n = 17$ represent the modulus which implies $k = 2$ and $\rho \equiv 7$.  Let $x = 33$ 
represent the value to reduce.

\newpage\begin{figure}
\begin{center}
\begin{tabular}{|c|c|c|}
\hline \textbf{Step ($t$)} & \textbf{Value of $x$} & \textbf{Value of $\mu$} \\
\hline --                 & $33$ & --\\
\hline $0$                 & $33 + \mu n = 50$ & $1$ \\
\hline $1$                 & $50 + \mu n \beta = 900$ & $5$ \\
\hline
\end{tabular}
\end{center}
\caption{Example of Montgomery Reduction}
\end{figure}

The final result $900$ is then divided by $\beta^k$ to produce the final result $9$.  The first observation is that $9 \nequiv x \mbox{ (mod }n\mbox{)}$ 
which implies the result is not the modular residue of $x$ modulo $n$.  However, recall that the residue is actually multiplied by $\beta^{-k}$ in
the algorithm.  To get the true residue the value must be multiplied by $\beta^k$.  In this case $\beta^k \equiv 15 \mbox{ (mod }n\mbox{)}$ and
the correct residue is $9 \cdot 15 \equiv 16 \mbox{ (mod }n\mbox{)}$.  

\subsection{Baseline Montgomery Reduction}
The baseline Montgomery reduction algorithm will produce the residue for any size input.  It is designed to be a catch-all algororithm for 
Montgomery reductions.  

\newpage\begin{figure}[!here]
\begin{small}
\begin{center}
\begin{tabular}{l}
\hline Algorithm \textbf{mp\_montgomery\_reduce}. \\
\textbf{Input}.   mp\_int $x$, mp\_int $n$ and a digit $\rho \equiv -1/n_0 \mbox{ (mod }n\mbox{)}$. \\
\hspace{11.5mm}($0 \le x < n^2, n > 1, (n, \beta) = 1, \beta^k > n$) \\
\textbf{Output}.  $\beta^{-k}x \mbox{ (mod }n\mbox{)}$ \\
\hline \\
1.  $digs \leftarrow 2n.used + 1$ \\
2.  If $digs < MP\_ARRAY$ and $m.used < \delta$ then \\
\hspace{3mm}2.1  Use algorithm fast\_mp\_montgomery\_reduce instead. \\
\\
Setup $x$ for the reduction. \\
3.  If $x.alloc < digs$ then grow $x$ to $digs$ digits. \\
4.  $x.used \leftarrow digs$ \\
\\
Eliminate the lower $k$ digits. \\
5.  For $ix$ from $0$ to $k - 1$ do \\
\hspace{3mm}5.1  $\mu \leftarrow x_{ix} \cdot \rho \mbox{ (mod }\beta\mbox{)}$ \\
\hspace{3mm}5.2  $u \leftarrow 0$ \\
\hspace{3mm}5.3  For $iy$ from $0$ to $k - 1$ do \\
\hspace{6mm}5.3.1  $\hat r \leftarrow \mu n_{iy} + x_{ix + iy} + u$ \\
\hspace{6mm}5.3.2  $x_{ix + iy} \leftarrow \hat r \mbox{ (mod }\beta\mbox{)}$ \\
\hspace{6mm}5.3.3  $u \leftarrow \lfloor \hat r / \beta \rfloor$ \\
\hspace{3mm}5.4  While $u > 0$ do \\
\hspace{6mm}5.4.1  $iy \leftarrow iy + 1$ \\
\hspace{6mm}5.4.2  $x_{ix + iy} \leftarrow x_{ix + iy} + u$ \\
\hspace{6mm}5.4.3  $u \leftarrow \lfloor x_{ix+iy} / \beta \rfloor$ \\
\hspace{6mm}5.4.4  $x_{ix + iy} \leftarrow x_{ix+iy} \mbox{ (mod }\beta\mbox{)}$ \\
\\
Divide by $\beta^k$ and fix up as required. \\
6.  $x \leftarrow \lfloor x / \beta^k \rfloor$ \\
7.  If $x \ge n$ then \\
\hspace{3mm}7.1  $x \leftarrow x - n$ \\
8.  Return(\textit{MP\_OKAY}). \\
\hline
\end{tabular}
\end{center}
\end{small}
\caption{Algorithm mp\_montgomery\_reduce}
\end{figure}

\textbf{Algorithm mp\_montgomery\_reduce.}
This algorithm reduces the input $x$ modulo $n$ in place using the Montgomery reduction algorithm.  The algorithm is loosely based
on algorithm 14.32 of \cite[pp.601]{HAC} except it merges the multiplication of $\mu n \beta^t$ with the addition in the inner loop.  The
restrictions on this algorithm are fairly easy to adapt to.  First $0 \le x < n^2$ bounds the input to numbers in the same range as 
for the Barrett algorithm.  Additionally if $n > 1$ and $n$ is odd there will exist a modular inverse $\rho$.  $\rho$ must be calculated in
advance of this algorithm.  Finally the variable $k$ is fixed and a pseudonym for $n.used$.  

Step 2 decides whether a faster Montgomery algorithm can be used.  It is based on the Comba technique meaning that there are limits on
the size of the input.  This algorithm is discussed in sub-section 6.3.3.

Step 5 is the main reduction loop of the algorithm.  The value of $\mu$ is calculated once per iteration in the outer loop.  The inner loop
calculates $x + \mu n \beta^{ix}$ by multiplying $\mu n$ and adding the result to $x$ shifted by $ix$ digits.  Both the addition and
multiplication are performed in the same loop to save time and memory.  Step 5.4 will handle any additional carries that escape the inner loop.

Using a quick inspection this algorithm requires $n$ single precision multiplications for the outer loop and $n^2$ single precision multiplications 
in the inner loop.  In total $n^2 + n$ single precision multiplications which compares favourably to Barrett at $n^2 + 2n - 1$ single precision
multiplications.  

\vspace{+3mm}\begin{small}
\hspace{-5.1mm}{\bf File}: bn\_mp\_montgomery\_reduce.c
\vspace{-3mm}
\begin{alltt}
\end{alltt}
\end{small}

This is the baseline implementation of the Montgomery reduction algorithm.  Lines 31 to 36 determine if the Comba based
routine can be used instead.  Line 47 computes the value of $\mu$ for that particular iteration of the outer loop.  

The multiplication $\mu n \beta^{ix}$ is performed in one step in the inner loop.  The alias $tmpx$ refers to the $ix$'th digit of $x$ and
the alias $tmpn$ refers to the modulus $n$.  

\subsection{Faster ``Comba'' Montgomery Reduction}

The Montgomery reduction requires fewer single precision multiplications than a Barrett reduction, however it is much slower due to the serial
nature of the inner loop.  The Barrett reduction algorithm requires two slightly modified multipliers which can be implemented with the Comba
technique.  The Montgomery reduction algorithm cannot directly use the Comba technique to any significant advantage since the inner loop calculates
a $k \times 1$ product $k$ times. 

The biggest obstacle is that at the $ix$'th iteration of the outer loop the value of $x_{ix}$ is required to calculate $\mu$.  This means the 
carries from $0$ to $ix - 1$ must have been propagated upwards to form a valid $ix$'th digit.  The solution as it turns out is very simple.  
Perform a Comba like multiplier and inside the outer loop just after the inner loop fix up the $ix + 1$'th digit by forwarding the carry.  

With this change in place the Montgomery reduction algorithm can be performed with a Comba style multiplication loop which substantially increases
the speed of the algorithm.  

\newpage\begin{figure}[!here]
\begin{small}
\begin{center}
\begin{tabular}{l}
\hline Algorithm \textbf{fast\_mp\_montgomery\_reduce}. \\
\textbf{Input}.   mp\_int $x$, mp\_int $n$ and a digit $\rho \equiv -1/n_0 \mbox{ (mod }n\mbox{)}$. \\
\hspace{11.5mm}($0 \le x < n^2, n > 1, (n, \beta) = 1, \beta^k > n$) \\
\textbf{Output}.  $\beta^{-k}x \mbox{ (mod }n\mbox{)}$ \\
\hline \\
Place an array of \textbf{MP\_WARRAY} mp\_word variables called $\hat W$ on the stack. \\
1.  if $x.alloc < n.used + 1$ then grow $x$ to $n.used + 1$ digits. \\
Copy the digits of $x$ into the array $\hat W$ \\
2.  For $ix$ from $0$ to $x.used - 1$ do \\
\hspace{3mm}2.1  $\hat W_{ix} \leftarrow x_{ix}$ \\
3.  For $ix$ from $x.used$ to $2n.used - 1$ do \\
\hspace{3mm}3.1  $\hat W_{ix} \leftarrow 0$ \\
Elimiate the lower $k$ digits. \\
4.  for $ix$ from $0$ to $n.used - 1$ do \\
\hspace{3mm}4.1  $\mu \leftarrow \hat W_{ix} \cdot \rho \mbox{ (mod }\beta\mbox{)}$ \\
\hspace{3mm}4.2  For $iy$ from $0$ to $n.used - 1$ do \\
\hspace{6mm}4.2.1  $\hat W_{iy + ix} \leftarrow \hat W_{iy + ix} + \mu \cdot n_{iy}$ \\
\hspace{3mm}4.3  $\hat W_{ix + 1} \leftarrow \hat W_{ix + 1} + \lfloor \hat W_{ix} / \beta \rfloor$ \\
Propagate carries upwards. \\
5.  for $ix$ from $n.used$ to $2n.used + 1$ do \\
\hspace{3mm}5.1  $\hat W_{ix + 1} \leftarrow \hat W_{ix + 1} + \lfloor \hat W_{ix} / \beta \rfloor$ \\
Shift right and reduce modulo $\beta$ simultaneously. \\
6.  for $ix$ from $0$ to $n.used + 1$ do \\
\hspace{3mm}6.1  $x_{ix} \leftarrow \hat W_{ix + n.used} \mbox{ (mod }\beta\mbox{)}$ \\
Zero excess digits and fixup $x$. \\
7.  if $x.used > n.used + 1$ then do \\
\hspace{3mm}7.1  for $ix$ from $n.used + 1$ to $x.used - 1$ do \\
\hspace{6mm}7.1.1  $x_{ix} \leftarrow 0$ \\
8.  $x.used \leftarrow n.used + 1$ \\
9.  Clamp excessive digits of $x$. \\
10.  If $x \ge n$ then \\
\hspace{3mm}10.1  $x \leftarrow x - n$ \\
11.  Return(\textit{MP\_OKAY}). \\
\hline
\end{tabular}
\end{center}
\end{small}
\caption{Algorithm fast\_mp\_montgomery\_reduce}
\end{figure}

\textbf{Algorithm fast\_mp\_montgomery\_reduce.}
This algorithm will compute the Montgomery reduction of $x$ modulo $n$ using the Comba technique.  It is on most computer platforms significantly
faster than algorithm mp\_montgomery\_reduce and algorithm mp\_reduce (\textit{Barrett reduction}).  The algorithm has the same restrictions
on the input as the baseline reduction algorithm.  An additional two restrictions are imposed on this algorithm.  The number of digits $k$ in the 
the modulus $n$ must not violate $MP\_WARRAY > 2k +1$ and $n < \delta$.   When $\beta = 2^{28}$ this algorithm can be used to reduce modulo
a modulus of at most $3,556$ bits in length.  

As in the other Comba reduction algorithms there is a $\hat W$ array which stores the columns of the product.  It is initially filled with the
contents of $x$ with the excess digits zeroed.  The reduction loop is very similar the to the baseline loop at heart.  The multiplication on step
4.1 can be single precision only since $ab \mbox{ (mod }\beta\mbox{)} \equiv (a \mbox{ mod }\beta)(b \mbox{ mod }\beta)$.  Some multipliers such
as those on the ARM processors take a variable length time to complete depending on the number of bytes of result it must produce.  By performing
a single precision multiplication instead half the amount of time is spent.

Also note that digit $\hat W_{ix}$ must have the carry from the $ix - 1$'th digit propagated upwards in order for this to work.  That is what step
4.3 will do.  In effect over the $n.used$ iterations of the outer loop the $n.used$'th lower columns all have the their carries propagated forwards.  Note
how the upper bits of those same words are not reduced modulo $\beta$.  This is because those values will be discarded shortly and there is no
point.

Step 5 will propagate the remainder of the carries upwards.  On step 6 the columns are reduced modulo $\beta$ and shifted simultaneously as they are
stored in the destination $x$.  

\vspace{+3mm}\begin{small}
\hspace{-5.1mm}{\bf File}: bn\_fast\_mp\_montgomery\_reduce.c
\vspace{-3mm}
\begin{alltt}
\end{alltt}
\end{small}

The $\hat W$ array is first filled with digits of $x$ on line 48 then the rest of the digits are zeroed on line 55.  Both loops share
the same alias variables to make the code easier to read.  

The value of $\mu$ is calculated in an interesting fashion.  First the value $\hat W_{ix}$ is reduced modulo $\beta$ and cast to a mp\_digit.  This
forces the compiler to use a single precision multiplication and prevents any concerns about loss of precision.   Line 110 fixes the carry 
for the next iteration of the loop by propagating the carry from $\hat W_{ix}$ to $\hat W_{ix+1}$.

The for loop on line 109 propagates the rest of the carries upwards through the columns.  The for loop on line 126 reduces the columns
modulo $\beta$ and shifts them $k$ places at the same time.  The alias $\_ \hat W$ actually refers to the array $\hat W$ starting at the $n.used$'th
digit, that is $\_ \hat W_{t} = \hat W_{n.used + t}$.  

\subsection{Montgomery Setup}
To calculate the variable $\rho$ a relatively simple algorithm will be required.  

\begin{figure}[!here]
\begin{small}
\begin{center}
\begin{tabular}{l}
\hline Algorithm \textbf{mp\_montgomery\_setup}. \\
\textbf{Input}.   mp\_int $n$ ($n > 1$ and $(n, 2) = 1$) \\
\textbf{Output}.  $\rho \equiv -1/n_0 \mbox{ (mod }\beta\mbox{)}$ \\
\hline \\
1.  $b \leftarrow n_0$ \\
2.  If $b$ is even return(\textit{MP\_VAL}) \\
3.  $x \leftarrow (((b + 2) \mbox{ AND } 4) << 1) + b$ \\
4.  for $k$ from 0 to $\lceil lg(lg(\beta)) \rceil - 2$ do \\
\hspace{3mm}4.1  $x \leftarrow x \cdot (2 - bx)$ \\
5.  $\rho \leftarrow \beta - x \mbox{ (mod }\beta\mbox{)}$ \\
6.  Return(\textit{MP\_OKAY}). \\
\hline
\end{tabular}
\end{center}
\end{small}
\caption{Algorithm mp\_montgomery\_setup} 
\end{figure}

\textbf{Algorithm mp\_montgomery\_setup.}
This algorithm will calculate the value of $\rho$ required within the Montgomery reduction algorithms.  It uses a very interesting trick 
to calculate $1/n_0$ when $\beta$ is a power of two.  

\vspace{+3mm}\begin{small}
\hspace{-5.1mm}{\bf File}: bn\_mp\_montgomery\_setup.c
\vspace{-3mm}
\begin{alltt}
\end{alltt}
\end{small}

This source code computes the value of $\rho$ required to perform Montgomery reduction.  It has been modified to avoid performing excess
multiplications when $\beta$ is not the default 28-bits.  

\section{The Diminished Radix Algorithm}
The Diminished Radix method of modular reduction \cite{DRMET} is a fairly clever technique which can be more efficient than either the Barrett
or Montgomery methods for certain forms of moduli.  The technique is based on the following simple congruence.

\begin{equation}
(x \mbox{ mod } n) + k \lfloor x / n \rfloor \equiv x \mbox{ (mod }(n - k)\mbox{)}
\end{equation}

This observation was used in the MMB \cite{MMB} block cipher to create a diffusion primitive.  It used the fact that if $n = 2^{31}$ and $k=1$ that 
then a x86 multiplier could produce the 62-bit product and use  the ``shrd'' instruction to perform a double-precision right shift.  The proof
of the above equation is very simple.  First write $x$ in the product form.

\begin{equation}
x = qn + r
\end{equation}

Now reduce both sides modulo $(n - k)$.

\begin{equation}
x \equiv qk + r  \mbox{ (mod }(n-k)\mbox{)}
\end{equation}

The variable $n$ reduces modulo $n - k$ to $k$.  By putting $q = \lfloor x/n \rfloor$ and $r = x \mbox{ mod } n$ 
into the equation the original congruence is reproduced, thus concluding the proof.  The following algorithm is based on this observation.

\begin{figure}[!here]
\begin{small}
\begin{center}
\begin{tabular}{l}
\hline Algorithm \textbf{Diminished Radix Reduction}. \\
\textbf{Input}.   Integer $x$, $n$, $k$ \\
\textbf{Output}.  $x \mbox{ mod } (n - k)$ \\
\hline \\
1.  $q \leftarrow \lfloor x / n \rfloor$ \\
2.  $q \leftarrow k \cdot q$ \\
3.  $x \leftarrow x \mbox{ (mod }n\mbox{)}$ \\
4.  $x \leftarrow x + q$ \\
5.  If $x \ge (n - k)$ then \\
\hspace{3mm}5.1  $x \leftarrow x - (n - k)$ \\
\hspace{3mm}5.2  Goto step 1. \\
6.  Return $x$ \\
\hline
\end{tabular}
\end{center}
\end{small}
\caption{Algorithm Diminished Radix Reduction}
\label{fig:DR}
\end{figure}

This algorithm will reduce $x$ modulo $n - k$ and return the residue.  If $0 \le x < (n - k)^2$ then the algorithm will loop almost always
once or twice and occasionally three times.  For simplicity sake the value of $x$ is bounded by the following simple polynomial.

\begin{equation} 
0 \le x < n^2 + k^2 - 2nk
\end{equation}

The true bound is  $0 \le x < (n - k - 1)^2$ but this has quite a few more terms.  The value of $q$ after step 1 is bounded by the following.

\begin{equation}
q < n - 2k - k^2/n
\end{equation}

Since $k^2$ is going to be considerably smaller than $n$ that term will always be zero.  The value of $x$ after step 3 is bounded trivially as
$0 \le x < n$.  By step four the sum $x + q$ is bounded by 

\begin{equation}
0 \le q + x < (k + 1)n - 2k^2 - 1
\end{equation}

With a second pass $q$ will be loosely bounded by $0 \le q < k^2$ after step 2 while $x$ will still be loosely bounded by $0 \le x < n$ after step 3.  After the second pass it is highly unlike that the
sum in step 4 will exceed $n - k$.  In practice fewer than three passes of the algorithm are required to reduce virtually every input in the 
range $0 \le x < (n - k - 1)^2$.  

\begin{figure}
\begin{small}
\begin{center}
\begin{tabular}{|l|}
\hline
$x = 123456789, n = 256, k = 3$ \\
\hline $q \leftarrow \lfloor x/n \rfloor = 482253$ \\
$q \leftarrow q*k = 1446759$ \\
$x \leftarrow x \mbox{ mod } n = 21$ \\
$x \leftarrow x + q = 1446780$ \\
$x \leftarrow x - (n - k) = 1446527$ \\
\hline 
$q \leftarrow \lfloor x/n \rfloor = 5650$ \\
$q \leftarrow q*k = 16950$ \\
$x \leftarrow x \mbox{ mod } n = 127$ \\
$x \leftarrow x + q = 17077$ \\
$x \leftarrow x - (n - k) = 16824$ \\
\hline 
$q \leftarrow \lfloor x/n \rfloor = 65$ \\
$q \leftarrow q*k = 195$ \\
$x \leftarrow x \mbox{ mod } n = 184$ \\
$x \leftarrow x + q = 379$ \\
$x \leftarrow x - (n - k) = 126$ \\
\hline
\end{tabular}
\end{center}
\end{small}
\caption{Example Diminished Radix Reduction}
\label{fig:EXDR}
\end{figure}

Figure~\ref{fig:EXDR} demonstrates the reduction of $x = 123456789$ modulo $n - k = 253$ when $n = 256$ and $k = 3$.  Note that even while $x$
is considerably larger than $(n - k - 1)^2 = 63504$ the algorithm still converges on the modular residue exceedingly fast.  In this case only
three passes were required to find the residue $x \equiv 126$.


\subsection{Choice of Moduli}
On the surface this algorithm looks like a very expensive algorithm.  It requires a couple of subtractions followed by multiplication and other
modular reductions.  The usefulness of this algorithm becomes exceedingly clear when an appropriate modulus is chosen.

Division in general is a very expensive operation to perform.  The one exception is when the division is by a power of the radix of representation used.  
Division by ten for example is simple for pencil and paper mathematics since it amounts to shifting the decimal place to the right.  Similarly division 
by two (\textit{or powers of two}) is very simple for binary computers to perform.  It would therefore seem logical to choose $n$ of the form $2^p$ 
which would imply that $\lfloor x / n \rfloor$ is a simple shift of $x$ right $p$ bits.  

However, there is one operation related to division of power of twos that is even faster than this.  If $n = \beta^p$ then the division may be 
performed by moving whole digits to the right $p$ places.  In practice division by $\beta^p$ is much faster than division by $2^p$ for any $p$.  
Also with the choice of $n = \beta^p$ reducing $x$ modulo $n$ merely requires zeroing the digits above the $p-1$'th digit of $x$.  

Throughout the next section the term ``restricted modulus'' will refer to a modulus of the form $\beta^p - k$ whereas the term ``unrestricted
modulus'' will refer to a modulus of the form $2^p - k$.  The word ``restricted'' in this case refers to the fact that it is based on the 
$2^p$ logic except $p$ must be a multiple of $lg(\beta)$.  

\subsection{Choice of $k$}
Now that division and reduction (\textit{step 1 and 3 of figure~\ref{fig:DR}}) have been optimized to simple digit operations the multiplication by $k$
in step 2 is the most expensive operation.  Fortunately the choice of $k$ is not terribly limited.  For all intents and purposes it might
as well be a single digit.  The smaller the value of $k$ is the faster the algorithm will be.  

\subsection{Restricted Diminished Radix Reduction}
The restricted Diminished Radix algorithm can quickly reduce an input modulo a modulus of the form $n = \beta^p - k$.  This algorithm can reduce 
an input $x$ within the range $0 \le x < n^2$ using only a couple passes of the algorithm demonstrated in figure~\ref{fig:DR}.  The implementation
of this algorithm has been optimized to avoid additional overhead associated with a division by $\beta^p$, the multiplication by $k$ or the addition 
of $x$ and $q$.  The resulting algorithm is very efficient and can lead to substantial improvements over Barrett and Montgomery reduction when modular 
exponentiations are performed.

\newpage\begin{figure}[!here]
\begin{small}
\begin{center}
\begin{tabular}{l}
\hline Algorithm \textbf{mp\_dr\_reduce}. \\
\textbf{Input}.   mp\_int $x$, $n$ and a mp\_digit $k = \beta - n_0$ \\
\hspace{11.5mm}($0 \le x < n^2$, $n > 1$, $0 < k < \beta$) \\
\textbf{Output}.  $x \mbox{ mod } n$ \\
\hline \\
1.  $m \leftarrow n.used$ \\
2.  If $x.alloc < 2m$ then grow $x$ to $2m$ digits. \\
3.  $\mu \leftarrow 0$ \\
4.  for $i$ from $0$ to $m - 1$ do \\
\hspace{3mm}4.1  $\hat r \leftarrow k \cdot x_{m+i} + x_{i} + \mu$ \\
\hspace{3mm}4.2  $x_{i} \leftarrow \hat r \mbox{ (mod }\beta\mbox{)}$ \\
\hspace{3mm}4.3  $\mu \leftarrow \lfloor \hat r / \beta \rfloor$ \\
5.  $x_{m} \leftarrow \mu$ \\
6.  for $i$ from $m + 1$ to $x.used - 1$ do \\
\hspace{3mm}6.1  $x_{i} \leftarrow 0$ \\
7.  Clamp excess digits of $x$. \\
8.  If $x \ge n$ then \\
\hspace{3mm}8.1  $x \leftarrow x - n$ \\
\hspace{3mm}8.2  Goto step 3. \\
9.  Return(\textit{MP\_OKAY}). \\
\hline
\end{tabular}
\end{center}
\end{small}
\caption{Algorithm mp\_dr\_reduce}
\end{figure}

\textbf{Algorithm mp\_dr\_reduce.}
This algorithm will perform the Dimished Radix reduction of $x$ modulo $n$.  It has similar restrictions to that of the Barrett reduction
with the addition that $n$ must be of the form $n = \beta^m - k$ where $0 < k <\beta$.  

This algorithm essentially implements the pseudo-code in figure~\ref{fig:DR} except with a slight optimization.  The division by $\beta^m$, multiplication by $k$
and addition of $x \mbox{ mod }\beta^m$ are all performed simultaneously inside the loop on step 4.  The division by $\beta^m$ is emulated by accessing
the term at the $m+i$'th position which is subsequently multiplied by $k$ and added to the term at the $i$'th position.  After the loop the $m$'th
digit is set to the carry and the upper digits are zeroed.  Steps 5 and 6 emulate the reduction modulo $\beta^m$ that should have happend to 
$x$ before the addition of the multiple of the upper half.  

At step 8 if $x$ is still larger than $n$ another pass of the algorithm is required.  First $n$ is subtracted from $x$ and then the algorithm resumes
at step 3.  

\vspace{+3mm}\begin{small}
\hspace{-5.1mm}{\bf File}: bn\_mp\_dr\_reduce.c
\vspace{-3mm}
\begin{alltt}
\end{alltt}
\end{small}

The first step is to grow $x$ as required to $2m$ digits since the reduction is performed in place on $x$.  The label on line 52 is where
the algorithm will resume if further reduction passes are required.  In theory it could be placed at the top of the function however, the size of
the modulus and question of whether $x$ is large enough are invariant after the first pass meaning that it would be a waste of time.  

The aliases $tmpx1$ and $tmpx2$ refer to the digits of $x$ where the latter is offset by $m$ digits.  By reading digits from $x$ offset by $m$ digits
a division by $\beta^m$ can be simulated virtually for free.  The loop on line 64 performs the bulk of the work (\textit{corresponds to step 4 of algorithm 7.11})
in this algorithm.

By line 67 the pointer $tmpx1$ points to the $m$'th digit of $x$ which is where the final carry will be placed.  Similarly by line 74 the 
same pointer will point to the $m+1$'th digit where the zeroes will be placed.  

Since the algorithm is only valid if both $x$ and $n$ are greater than zero an unsigned comparison suffices to determine if another pass is required.  
With the same logic at line 81 the value of $x$ is known to be greater than or equal to $n$ meaning that an unsigned subtraction can be used
as well.  Since the destination of the subtraction is the larger of the inputs the call to algorithm s\_mp\_sub cannot fail and the return code
does not need to be checked.

\subsubsection{Setup}
To setup the restricted Diminished Radix algorithm the value $k = \beta - n_0$ is required.  This algorithm is not really complicated but provided for
completeness.

\begin{figure}[!here]
\begin{small}
\begin{center}
\begin{tabular}{l}
\hline Algorithm \textbf{mp\_dr\_setup}. \\
\textbf{Input}.   mp\_int $n$ \\
\textbf{Output}.  $k = \beta - n_0$ \\
\hline \\
1.  $k \leftarrow \beta - n_0$ \\
\hline
\end{tabular}
\end{center}
\end{small}
\caption{Algorithm mp\_dr\_setup}
\end{figure}

\vspace{+3mm}\begin{small}
\hspace{-5.1mm}{\bf File}: bn\_mp\_dr\_setup.c
\vspace{-3mm}
\begin{alltt}
\end{alltt}
\end{small}

\subsubsection{Modulus Detection}
Another algorithm which will be useful is the ability to detect a restricted Diminished Radix modulus.  An integer is said to be
of restricted Diminished Radix form if all of the digits are equal to $\beta - 1$ except the trailing digit which may be any value.

\begin{figure}[!here]
\begin{small}
\begin{center}
\begin{tabular}{l}
\hline Algorithm \textbf{mp\_dr\_is\_modulus}. \\
\textbf{Input}.   mp\_int $n$ \\
\textbf{Output}.  $1$ if $n$ is in D.R form, $0$ otherwise \\
\hline
1.  If $n.used < 2$ then return($0$). \\
2.  for $ix$ from $1$ to $n.used - 1$ do \\
\hspace{3mm}2.1  If $n_{ix} \ne \beta - 1$ return($0$). \\
3.  Return($1$). \\
\hline
\end{tabular}
\end{center}
\end{small}
\caption{Algorithm mp\_dr\_is\_modulus}
\end{figure}

\textbf{Algorithm mp\_dr\_is\_modulus.}
This algorithm determines if a value is in Diminished Radix form.  Step 1 rejects obvious cases where fewer than two digits are
in the mp\_int.  Step 2 tests all but the first digit to see if they are equal to $\beta - 1$.  If the algorithm manages to get to
step 3 then $n$ must be of Diminished Radix form.

\vspace{+3mm}\begin{small}
\hspace{-5.1mm}{\bf File}: bn\_mp\_dr\_is\_modulus.c
\vspace{-3mm}
\begin{alltt}
\end{alltt}
\end{small}

\subsection{Unrestricted Diminished Radix Reduction}
The unrestricted Diminished Radix algorithm allows modular reductions to be performed when the modulus is of the form $2^p - k$.  This algorithm
is a straightforward adaptation of algorithm~\ref{fig:DR}.

In general the restricted Diminished Radix reduction algorithm is much faster since it has considerably lower overhead.  However, this new
algorithm is much faster than either Montgomery or Barrett reduction when the moduli are of the appropriate form.

\begin{figure}[!here]
\begin{small}
\begin{center}
\begin{tabular}{l}
\hline Algorithm \textbf{mp\_reduce\_2k}. \\
\textbf{Input}.   mp\_int $a$ and $n$.  mp\_digit $k$  \\
\hspace{11.5mm}($a \ge 0$, $n > 1$, $0 < k < \beta$, $n + k$ is a power of two) \\
\textbf{Output}.  $a \mbox{ (mod }n\mbox{)}$ \\
\hline
1.  $p \leftarrow \lceil lg(n) \rceil$  (\textit{mp\_count\_bits}) \\
2.  While $a \ge n$ do \\
\hspace{3mm}2.1  $q \leftarrow \lfloor a / 2^p \rfloor$ (\textit{mp\_div\_2d}) \\
\hspace{3mm}2.2  $a \leftarrow a \mbox{ (mod }2^p\mbox{)}$ (\textit{mp\_mod\_2d}) \\
\hspace{3mm}2.3  $q \leftarrow q \cdot k$ (\textit{mp\_mul\_d}) \\
\hspace{3mm}2.4  $a \leftarrow a - q$ (\textit{s\_mp\_sub}) \\
\hspace{3mm}2.5  If $a \ge n$ then do \\
\hspace{6mm}2.5.1  $a \leftarrow a - n$ \\
3.  Return(\textit{MP\_OKAY}). \\
\hline
\end{tabular}
\end{center}
\end{small}
\caption{Algorithm mp\_reduce\_2k}
\end{figure}

\textbf{Algorithm mp\_reduce\_2k.}
This algorithm quickly reduces an input $a$ modulo an unrestricted Diminished Radix modulus $n$.  Division by $2^p$ is emulated with a right
shift which makes the algorithm fairly inexpensive to use.  

\vspace{+3mm}\begin{small}
\hspace{-5.1mm}{\bf File}: bn\_mp\_reduce\_2k.c
\vspace{-3mm}
\begin{alltt}
\end{alltt}
\end{small}

The algorithm mp\_count\_bits calculates the number of bits in an mp\_int which is used to find the initial value of $p$.  The call to mp\_div\_2d
on line 31 calculates both the quotient $q$ and the remainder $a$ required.  By doing both in a single function call the code size
is kept fairly small.  The multiplication by $k$ is only performed if $k > 1$. This allows reductions modulo $2^p - 1$ to be performed without
any multiplications.  

The unsigned s\_mp\_add, mp\_cmp\_mag and s\_mp\_sub are used in place of their full sign counterparts since the inputs are only valid if they are 
positive.  By using the unsigned versions the overhead is kept to a minimum.  

\subsubsection{Unrestricted Setup}
To setup this reduction algorithm the value of $k = 2^p - n$ is required.  

\begin{figure}[!here]
\begin{small}
\begin{center}
\begin{tabular}{l}
\hline Algorithm \textbf{mp\_reduce\_2k\_setup}. \\
\textbf{Input}.   mp\_int $n$   \\
\textbf{Output}.  $k = 2^p - n$ \\
\hline
1.  $p \leftarrow \lceil lg(n) \rceil$  (\textit{mp\_count\_bits}) \\
2.  $x \leftarrow 2^p$ (\textit{mp\_2expt}) \\
3.  $x \leftarrow x - n$ (\textit{mp\_sub}) \\
4.  $k \leftarrow x_0$ \\
5.  Return(\textit{MP\_OKAY}). \\
\hline
\end{tabular}
\end{center}
\end{small}
\caption{Algorithm mp\_reduce\_2k\_setup}
\end{figure}

\textbf{Algorithm mp\_reduce\_2k\_setup.}
This algorithm computes the value of $k$ required for the algorithm mp\_reduce\_2k.  By making a temporary variable $x$ equal to $2^p$ a subtraction
is sufficient to solve for $k$.  Alternatively if $n$ has more than one digit the value of $k$ is simply $\beta - n_0$.  

\vspace{+3mm}\begin{small}
\hspace{-5.1mm}{\bf File}: bn\_mp\_reduce\_2k\_setup.c
\vspace{-3mm}
\begin{alltt}
\end{alltt}
\end{small}

\subsubsection{Unrestricted Detection}
An integer $n$ is a valid unrestricted Diminished Radix modulus if either of the following are true.

\begin{enumerate}
\item  The number has only one digit.
\item  The number has more than one digit and every bit from the $\beta$'th to the most significant is one.
\end{enumerate}

If either condition is true than there is a power of two $2^p$ such that $0 < 2^p - n < \beta$.   If the input is only
one digit than it will always be of the correct form.  Otherwise all of the bits above the first digit must be one.  This arises from the fact
that there will be value of $k$ that when added to the modulus causes a carry in the first digit which propagates all the way to the most
significant bit.  The resulting sum will be a power of two.

\begin{figure}[!here]
\begin{small}
\begin{center}
\begin{tabular}{l}
\hline Algorithm \textbf{mp\_reduce\_is\_2k}. \\
\textbf{Input}.   mp\_int $n$   \\
\textbf{Output}.  $1$ if of proper form, $0$ otherwise \\
\hline
1.  If $n.used = 0$ then return($0$). \\
2.  If $n.used = 1$ then return($1$). \\
3.  $p \leftarrow \lceil lg(n) \rceil$  (\textit{mp\_count\_bits}) \\
4.  for $x$ from $lg(\beta)$ to $p$ do \\
\hspace{3mm}4.1  If the ($x \mbox{ mod }lg(\beta)$)'th bit of the $\lfloor x / lg(\beta) \rfloor$ of $n$ is zero then return($0$). \\
5.  Return($1$). \\
\hline
\end{tabular}
\end{center}
\end{small}
\caption{Algorithm mp\_reduce\_is\_2k}
\end{figure}

\textbf{Algorithm mp\_reduce\_is\_2k.}
This algorithm quickly determines if a modulus is of the form required for algorithm mp\_reduce\_2k to function properly.  

\vspace{+3mm}\begin{small}
\hspace{-5.1mm}{\bf File}: bn\_mp\_reduce\_is\_2k.c
\vspace{-3mm}
\begin{alltt}
\end{alltt}
\end{small}



\section{Algorithm Comparison}
So far three very different algorithms for modular reduction have been discussed.  Each of the algorithms have their own strengths and weaknesses
that makes having such a selection very useful.  The following table sumarizes the three algorithms along with comparisons of work factors.  Since
all three algorithms have the restriction that $0 \le x < n^2$ and $n > 1$ those limitations are not included in the table.  

\begin{center}
\begin{small}
\begin{tabular}{|c|c|c|c|c|c|}
\hline \textbf{Method} & \textbf{Work Required} & \textbf{Limitations} & \textbf{$m = 8$} & \textbf{$m = 32$} & \textbf{$m = 64$} \\
\hline Barrett    & $m^2 + 2m - 1$ & None              & $79$ & $1087$ & $4223$ \\
\hline Montgomery & $m^2 + m$      & $n$ must be odd   & $72$ & $1056$ & $4160$ \\
\hline D.R.       & $2m$           & $n = \beta^m - k$ & $16$ & $64$   & $128$  \\
\hline
\end{tabular}
\end{small}
\end{center}

In theory Montgomery and Barrett reductions would require roughly the same amount of time to complete.  However, in practice since Montgomery
reduction can be written as a single function with the Comba technique it is much faster.  Barrett reduction suffers from the overhead of
calling the half precision multipliers, addition and division by $\beta$ algorithms.

For almost every cryptographic algorithm Montgomery reduction is the algorithm of choice.  The one set of algorithms where Diminished Radix reduction truly
shines are based on the discrete logarithm problem such as Diffie-Hellman \cite{DH} and ElGamal \cite{ELGAMAL}.  In these algorithms
primes of the form $\beta^m - k$ can be found and shared amongst users.  These primes will allow the Diminished Radix algorithm to be used in
modular exponentiation to greatly speed up the operation.



\section*{Exercises}
\begin{tabular}{cl}
$\left [ 3 \right ]$ & Prove that the ``trick'' in algorithm mp\_montgomery\_setup actually \\
                     & calculates the correct value of $\rho$. \\
                     & \\
$\left [ 2 \right ]$ & Devise an algorithm to reduce modulo $n + k$ for small $k$ quickly.  \\
                     & \\
$\left [ 4 \right ]$ & Prove that the pseudo-code algorithm ``Diminished Radix Reduction'' \\
                     & (\textit{figure~\ref{fig:DR}}) terminates.  Also prove the probability that it will \\
                     & terminate within $1 \le k \le 10$ iterations. \\
                     & \\
\end{tabular}                     


\chapter{Exponentiation}
Exponentiation is the operation of raising one variable to the power of another, for example, $a^b$.  A variant of exponentiation, computed
in a finite field or ring, is called modular exponentiation.  This latter style of operation is typically used in public key 
cryptosystems such as RSA and Diffie-Hellman.  The ability to quickly compute modular exponentiations is of great benefit to any
such cryptosystem and many methods have been sought to speed it up.

\section{Exponentiation Basics}
A trivial algorithm would simply multiply $a$ against itself $b - 1$ times to compute the exponentiation desired.  However, as $b$ grows in size
the number of multiplications becomes prohibitive.  Imagine what would happen if $b$ $\approx$ $2^{1024}$ as is the case when computing an RSA signature
with a $1024$-bit key.  Such a calculation could never be completed as it would take simply far too long.

Fortunately there is a very simple algorithm based on the laws of exponents.  Recall that $lg_a(a^b) = b$ and that $lg_a(a^ba^c) = b + c$ which
are two trivial relationships between the base and the exponent.  Let $b_i$ represent the $i$'th bit of $b$ starting from the least 
significant bit.  If $b$ is a $k$-bit integer than the following equation is true.

\begin{equation}
a^b = \prod_{i=0}^{k-1} a^{2^i \cdot b_i}
\end{equation}

By taking the base $a$ logarithm of both sides of the equation the following equation is the result.

\begin{equation}
b = \sum_{i=0}^{k-1}2^i \cdot b_i
\end{equation}

The term $a^{2^i}$ can be found from the $i - 1$'th term by squaring the term since $\left ( a^{2^i} \right )^2$ is equal to
$a^{2^{i+1}}$.  This observation forms the basis of essentially all fast exponentiation algorithms.  It requires $k$ squarings and on average
$k \over 2$ multiplications to compute the result.  This is indeed quite an improvement over simply multiplying by $a$ a total of $b-1$ times.

While this current method is a considerable speed up there are further improvements to be made.  For example, the $a^{2^i}$ term does not need to 
be computed in an auxilary variable.  Consider the following equivalent algorithm.

\begin{figure}[!here]
\begin{small}
\begin{center}
\begin{tabular}{l}
\hline Algorithm \textbf{Left to Right Exponentiation}. \\
\textbf{Input}.   Integer $a$, $b$ and $k$ \\
\textbf{Output}.  $c = a^b$ \\
\hline \\
1.  $c \leftarrow 1$ \\
2.  for $i$ from $k - 1$ to $0$ do \\
\hspace{3mm}2.1  $c \leftarrow c^2$ \\
\hspace{3mm}2.2  $c \leftarrow c \cdot a^{b_i}$ \\
3.  Return $c$. \\
\hline
\end{tabular}
\end{center}
\end{small}
\caption{Left to Right Exponentiation}
\label{fig:LTOR}
\end{figure}

This algorithm starts from the most significant bit and works towards the least significant bit.  When the $i$'th bit of $b$ is set $a$ is
multiplied against the current product.  In each iteration the product is squared which doubles the exponent of the individual terms of the
product.  

For example, let $b = 101100_2 \equiv 44_{10}$.  The following chart demonstrates the actions of the algorithm.

\newpage\begin{figure}
\begin{center}
\begin{tabular}{|c|c|}
\hline \textbf{Value of $i$} & \textbf{Value of $c$} \\
\hline - & $1$ \\
\hline $5$ & $a$ \\
\hline $4$ & $a^2$ \\
\hline $3$ & $a^4 \cdot a$ \\
\hline $2$ & $a^8 \cdot a^2 \cdot a$ \\
\hline $1$ & $a^{16} \cdot a^4 \cdot a^2$ \\
\hline $0$ & $a^{32} \cdot a^8 \cdot a^4$ \\
\hline
\end{tabular}
\end{center}
\caption{Example of Left to Right Exponentiation}
\end{figure}

When the product $a^{32} \cdot a^8 \cdot a^4$ is simplified it is equal $a^{44}$ which is the desired exponentiation.  This particular algorithm is 
called ``Left to Right'' because it reads the exponent in that order.  All of the exponentiation algorithms that will be presented are of this nature.  

\subsection{Single Digit Exponentiation}
The first algorithm in the series of exponentiation algorithms will be an unbounded algorithm where the exponent is a single digit.  It is intended 
to be used when a small power of an input is required (\textit{e.g. $a^5$}).  It is faster than simply multiplying $b - 1$ times for all values of 
$b$ that are greater than three.  

\newpage\begin{figure}[!here]
\begin{small}
\begin{center}
\begin{tabular}{l}
\hline Algorithm \textbf{mp\_expt\_d}. \\
\textbf{Input}.   mp\_int $a$ and mp\_digit $b$ \\
\textbf{Output}.  $c = a^b$ \\
\hline \\
1.  $g \leftarrow a$ (\textit{mp\_init\_copy}) \\
2.  $c \leftarrow 1$ (\textit{mp\_set}) \\
3.  for $x$ from 1 to $lg(\beta)$ do \\
\hspace{3mm}3.1  $c \leftarrow c^2$ (\textit{mp\_sqr}) \\
\hspace{3mm}3.2  If $b$ AND $2^{lg(\beta) - 1} \ne 0$ then \\
\hspace{6mm}3.2.1  $c \leftarrow c \cdot g$ (\textit{mp\_mul}) \\
\hspace{3mm}3.3  $b \leftarrow b << 1$ \\
4.  Clear $g$. \\
5.  Return(\textit{MP\_OKAY}). \\
\hline
\end{tabular}
\end{center}
\end{small}
\caption{Algorithm mp\_expt\_d}
\end{figure}

\textbf{Algorithm mp\_expt\_d.}
This algorithm computes the value of $a$ raised to the power of a single digit $b$.  It uses the left to right exponentiation algorithm to
quickly compute the exponentiation.  It is loosely based on algorithm 14.79 of HAC \cite[pp. 615]{HAC} with the difference that the 
exponent is a fixed width.  

A copy of $a$ is made first to allow destination variable $c$ be the same as the source variable $a$.  The result is set to the initial value of 
$1$ in the subsequent step.

Inside the loop the exponent is read from the most significant bit first down to the least significant bit.  First $c$ is invariably squared
on step 3.1.  In the following step if the most significant bit of $b$ is one the copy of $a$ is multiplied against $c$.  The value
of $b$ is shifted left one bit to make the next bit down from the most signficant bit the new most significant bit.  In effect each
iteration of the loop moves the bits of the exponent $b$ upwards to the most significant location.

\vspace{+3mm}\begin{small}
\hspace{-5.1mm}{\bf File}: bn\_mp\_expt\_d.c
\vspace{-3mm}
\begin{alltt}
\end{alltt}
\end{small}

Line 29 sets the initial value of the result to $1$.  Next the loop on line 31 steps through each bit of the exponent starting from
the most significant down towards the least significant. The invariant squaring operation placed on line 33 is performed first.  After 
the squaring the result $c$ is multiplied by the base $g$ if and only if the most significant bit of the exponent is set.  The shift on line
47 moves all of the bits of the exponent upwards towards the most significant location.  

\section{$k$-ary Exponentiation}
When calculating an exponentiation the most time consuming bottleneck is the multiplications which are in general a small factor
slower than squaring.  Recall from the previous algorithm that $b_{i}$ refers to the $i$'th bit of the exponent $b$.  Suppose instead it referred to
the $i$'th $k$-bit digit of the exponent of $b$.  For $k = 1$ the definitions are synonymous and for $k > 1$ algorithm~\ref{fig:KARY}
computes the same exponentiation.  A group of $k$ bits from the exponent is called a \textit{window}.  That is it is a small window on only a
portion of the entire exponent.  Consider the following modification to the basic left to right exponentiation algorithm.

\begin{figure}[!here]
\begin{small}
\begin{center}
\begin{tabular}{l}
\hline Algorithm \textbf{$k$-ary Exponentiation}. \\
\textbf{Input}.   Integer $a$, $b$, $k$ and $t$ \\
\textbf{Output}.  $c = a^b$ \\
\hline \\
1.  $c \leftarrow 1$ \\
2.  for $i$ from $t - 1$ to $0$ do \\
\hspace{3mm}2.1  $c \leftarrow c^{2^k} $ \\
\hspace{3mm}2.2  Extract the $i$'th $k$-bit word from $b$ and store it in $g$. \\
\hspace{3mm}2.3  $c \leftarrow c \cdot a^g$ \\
3.  Return $c$. \\
\hline
\end{tabular}
\end{center}
\end{small}
\caption{$k$-ary Exponentiation}
\label{fig:KARY}
\end{figure}

The squaring on step 2.1 can be calculated by squaring the value $c$ successively $k$ times.  If the values of $a^g$ for $0 < g < 2^k$ have been
precomputed this algorithm requires only $t$ multiplications and $tk$ squarings.  The table can be generated with $2^{k - 1} - 1$ squarings and
$2^{k - 1} + 1$ multiplications.  This algorithm assumes that the number of bits in the exponent is evenly divisible by $k$.  
However, when it is not the remaining $0 < x \le k - 1$ bits can be handled with algorithm~\ref{fig:LTOR}.

Suppose $k = 4$ and $t = 100$.  This modified algorithm will require $109$ multiplications and $408$ squarings to compute the exponentiation.  The
original algorithm would on average have required $200$ multiplications and $400$ squrings to compute the same value.  The total number of squarings
has increased slightly but the number of multiplications has nearly halved.

\subsection{Optimal Values of $k$}
An optimal value of $k$ will minimize $2^{k} + \lceil n / k \rceil + n - 1$ for a fixed number of bits in the exponent $n$.  The simplest
approach is to brute force search amongst the values $k = 2, 3, \ldots, 8$ for the lowest result.  Table~\ref{fig:OPTK} lists optimal values of $k$
for various exponent sizes and compares the number of multiplication and squarings required against algorithm~\ref{fig:LTOR}.  

\begin{figure}[here]
\begin{center}
\begin{small}
\begin{tabular}{|c|c|c|c|c|c|}
\hline \textbf{Exponent (bits)} & \textbf{Optimal $k$} & \textbf{Work at $k$} & \textbf{Work with ~\ref{fig:LTOR}} \\
\hline $16$ & $2$ & $27$ & $24$ \\
\hline $32$ & $3$ & $49$ & $48$ \\
\hline $64$ & $3$ & $92$ & $96$ \\
\hline $128$ & $4$ & $175$ & $192$ \\
\hline $256$ & $4$ & $335$ & $384$ \\
\hline $512$ & $5$ & $645$ & $768$ \\
\hline $1024$ & $6$ & $1257$ & $1536$ \\
\hline $2048$ & $6$ & $2452$ & $3072$ \\
\hline $4096$ & $7$ & $4808$ & $6144$ \\
\hline
\end{tabular}
\end{small}
\end{center}
\caption{Optimal Values of $k$ for $k$-ary Exponentiation}
\label{fig:OPTK}
\end{figure}

\subsection{Sliding-Window Exponentiation}
A simple modification to the previous algorithm is only generate the upper half of the table in the range $2^{k-1} \le g < 2^k$.  Essentially
this is a table for all values of $g$ where the most significant bit of $g$ is a one.  However, in order for this to be allowed in the 
algorithm values of $g$ in the range $0 \le g < 2^{k-1}$ must be avoided.  

Table~\ref{fig:OPTK2} lists optimal values of $k$ for various exponent sizes and compares the work required against algorithm~\ref{fig:KARY}.  

\begin{figure}[here]
\begin{center}
\begin{small}
\begin{tabular}{|c|c|c|c|c|c|}
\hline \textbf{Exponent (bits)} & \textbf{Optimal $k$} & \textbf{Work at $k$} & \textbf{Work with ~\ref{fig:KARY}} \\
\hline $16$ & $3$ & $24$ & $27$ \\
\hline $32$ & $3$ & $45$ & $49$ \\
\hline $64$ & $4$ & $87$ & $92$ \\
\hline $128$ & $4$ & $167$ & $175$ \\
\hline $256$ & $5$ & $322$ & $335$ \\
\hline $512$ & $6$ & $628$ & $645$ \\
\hline $1024$ & $6$ & $1225$ & $1257$ \\
\hline $2048$ & $7$ & $2403$ & $2452$ \\
\hline $4096$ & $8$ & $4735$ & $4808$ \\
\hline
\end{tabular}
\end{small}
\end{center}
\caption{Optimal Values of $k$ for Sliding Window Exponentiation}
\label{fig:OPTK2}
\end{figure}

\newpage\begin{figure}[!here]
\begin{small}
\begin{center}
\begin{tabular}{l}
\hline Algorithm \textbf{Sliding Window $k$-ary Exponentiation}. \\
\textbf{Input}.   Integer $a$, $b$, $k$ and $t$ \\
\textbf{Output}.  $c = a^b$ \\
\hline \\
1.  $c \leftarrow 1$ \\
2.  for $i$ from $t - 1$ to $0$ do \\
\hspace{3mm}2.1  If the $i$'th bit of $b$ is a zero then \\
\hspace{6mm}2.1.1   $c \leftarrow c^2$ \\
\hspace{3mm}2.2  else do \\
\hspace{6mm}2.2.1  $c \leftarrow c^{2^k}$ \\
\hspace{6mm}2.2.2  Extract the $k$ bits from $(b_{i}b_{i-1}\ldots b_{i-(k-1)})$ and store it in $g$. \\
\hspace{6mm}2.2.3  $c \leftarrow c \cdot a^g$ \\
\hspace{6mm}2.2.4  $i \leftarrow i - k$ \\
3.  Return $c$. \\
\hline
\end{tabular}
\end{center}
\end{small}
\caption{Sliding Window $k$-ary Exponentiation}
\end{figure}

Similar to the previous algorithm this algorithm must have a special handler when fewer than $k$ bits are left in the exponent.  While this
algorithm requires the same number of squarings it can potentially have fewer multiplications.  The pre-computed table $a^g$ is also half
the size as the previous table.  

Consider the exponent $b = 111101011001000_2 \equiv 31432_{10}$ with $k = 3$ using both algorithms.  The first algorithm will divide the exponent up as 
the following five $3$-bit words $b \equiv \left ( 111, 101, 011, 001, 000 \right )_{2}$.  The second algorithm will break the 
exponent as $b \equiv \left ( 111, 101, 0, 110, 0, 100, 0 \right )_{2}$.  The single digit $0$ in the second representation are where
a single squaring took place instead of a squaring and multiplication.  In total the first method requires $10$ multiplications and $18$ 
squarings.  The second method requires $8$ multiplications and $18$ squarings.  

In general the sliding window method is never slower than the generic $k$-ary method and often it is slightly faster.  

\section{Modular Exponentiation}

Modular exponentiation is essentially computing the power of a base within a finite field or ring.  For example, computing 
$d \equiv a^b \mbox{ (mod }c\mbox{)}$ is a modular exponentiation.  Instead of first computing $a^b$ and then reducing it 
modulo $c$ the intermediate result is reduced modulo $c$ after every squaring or multiplication operation.  

This guarantees that any intermediate result is bounded by $0 \le d \le c^2 - 2c + 1$ and can be reduced modulo $c$ quickly using
one of the algorithms presented in chapter six.  

Before the actual modular exponentiation algorithm can be written a wrapper algorithm must be written first.  This algorithm
will allow the exponent $b$ to be negative which is computed as $c \equiv \left (1 / a \right )^{\vert b \vert} \mbox{(mod }d\mbox{)}$. The
value of $(1/a) \mbox{ mod }c$ is computed using the modular inverse (\textit{see \ref{sec;modinv}}).  If no inverse exists the algorithm
terminates with an error.  

\begin{figure}[!here]
\begin{small}
\begin{center}
\begin{tabular}{l}
\hline Algorithm \textbf{mp\_exptmod}. \\
\textbf{Input}.   mp\_int $a$, $b$ and $c$ \\
\textbf{Output}.  $y \equiv g^x \mbox{ (mod }p\mbox{)}$ \\
\hline \\
1.  If $c.sign = MP\_NEG$ return(\textit{MP\_VAL}). \\
2.  If $b.sign = MP\_NEG$ then \\
\hspace{3mm}2.1  $g' \leftarrow g^{-1} \mbox{ (mod }c\mbox{)}$ \\
\hspace{3mm}2.2  $x' \leftarrow \vert x \vert$ \\
\hspace{3mm}2.3  Compute $d \equiv g'^{x'} \mbox{ (mod }c\mbox{)}$ via recursion. \\
3.  if $p$ is odd \textbf{OR} $p$ is a D.R. modulus then \\
\hspace{3mm}3.1  Compute $y \equiv g^{x} \mbox{ (mod }p\mbox{)}$ via algorithm mp\_exptmod\_fast. \\
4.  else \\
\hspace{3mm}4.1  Compute $y \equiv g^{x} \mbox{ (mod }p\mbox{)}$ via algorithm s\_mp\_exptmod. \\
\hline
\end{tabular}
\end{center}
\end{small}
\caption{Algorithm mp\_exptmod}
\end{figure}

\textbf{Algorithm mp\_exptmod.}
The first algorithm which actually performs modular exponentiation is algorithm s\_mp\_exptmod.  It is a sliding window $k$-ary algorithm 
which uses Barrett reduction to reduce the product modulo $p$.  The second algorithm mp\_exptmod\_fast performs the same operation 
except it uses either Montgomery or Diminished Radix reduction.  The two latter reduction algorithms are clumped in the same exponentiation
algorithm since their arguments are essentially the same (\textit{two mp\_ints and one mp\_digit}).  

\vspace{+3mm}\begin{small}
\hspace{-5.1mm}{\bf File}: bn\_mp\_exptmod.c
\vspace{-3mm}
\begin{alltt}
\end{alltt}
\end{small}

In order to keep the algorithms in a known state the first step on line 29 is to reject any negative modulus as input.  If the exponent is
negative the algorithm tries to perform a modular exponentiation with the modular inverse of the base $G$.  The temporary variable $tmpG$ is assigned
the modular inverse of $G$ and $tmpX$ is assigned the absolute value of $X$.  The algorithm will recuse with these new values with a positive
exponent.

If the exponent is positive the algorithm resumes the exponentiation.  Line 77 determines if the modulus is of the restricted Diminished Radix 
form.  If it is not line 70 attempts to determine if it is of a unrestricted Diminished Radix form.  The integer $dr$ will take on one
of three values.

\begin{enumerate}
\item $dr = 0$ means that the modulus is not of either restricted or unrestricted Diminished Radix form.
\item $dr = 1$ means that the modulus is of restricted Diminished Radix form.
\item $dr = 2$ means that the modulus is of unrestricted Diminished Radix form.
\end{enumerate}

Line 69 determines if the fast modular exponentiation algorithm can be used.  It is allowed if $dr \ne 0$ or if the modulus is odd.  Otherwise,
the slower s\_mp\_exptmod algorithm is used which uses Barrett reduction.  

\subsection{Barrett Modular Exponentiation}

\newpage\begin{figure}[!here]
\begin{small}
\begin{center}
\begin{tabular}{l}
\hline Algorithm \textbf{s\_mp\_exptmod}. \\
\textbf{Input}.   mp\_int $a$, $b$ and $c$ \\
\textbf{Output}.  $y \equiv g^x \mbox{ (mod }p\mbox{)}$ \\
\hline \\
1.  $k \leftarrow lg(x)$ \\
2.  $winsize \leftarrow  \left \lbrace \begin{array}{ll}
                              2 &  \mbox{if }k \le 7 \\
                              3 &  \mbox{if }7 < k \le 36 \\
                              4 &  \mbox{if }36 < k \le 140 \\
                              5 &  \mbox{if }140 < k \le 450 \\
                              6 &  \mbox{if }450 < k \le 1303 \\
                              7 &  \mbox{if }1303 < k \le 3529 \\
                              8 &  \mbox{if }3529 < k \\
                              \end{array} \right .$ \\
3.  Initialize $2^{winsize}$ mp\_ints in an array named $M$ and one mp\_int named $\mu$ \\
4.  Calculate the $\mu$ required for Barrett Reduction (\textit{mp\_reduce\_setup}). \\
5.  $M_1 \leftarrow g \mbox{ (mod }p\mbox{)}$ \\
\\
Setup the table of small powers of $g$.  First find $g^{2^{winsize}}$ and then all multiples of it. \\
6.  $k \leftarrow 2^{winsize - 1}$ \\
7.  $M_{k} \leftarrow M_1$ \\
8.  for $ix$ from 0 to $winsize - 2$ do \\
\hspace{3mm}8.1  $M_k \leftarrow \left ( M_k \right )^2$ (\textit{mp\_sqr})  \\
\hspace{3mm}8.2  $M_k \leftarrow M_k \mbox{ (mod }p\mbox{)}$ (\textit{mp\_reduce}) \\
9.  for $ix$ from $2^{winsize - 1} + 1$ to $2^{winsize} - 1$ do \\
\hspace{3mm}9.1  $M_{ix} \leftarrow M_{ix - 1} \cdot M_{1}$ (\textit{mp\_mul}) \\
\hspace{3mm}9.2  $M_{ix} \leftarrow M_{ix} \mbox{ (mod }p\mbox{)}$ (\textit{mp\_reduce}) \\
10.  $res \leftarrow 1$ \\
\\
Start Sliding Window. \\
11.  $mode \leftarrow 0, bitcnt \leftarrow 1, buf \leftarrow 0, digidx \leftarrow x.used - 1, bitcpy \leftarrow 0, bitbuf \leftarrow 0$ \\
12.  Loop \\
\hspace{3mm}12.1  $bitcnt \leftarrow bitcnt - 1$ \\
\hspace{3mm}12.2  If $bitcnt = 0$ then do \\
\hspace{6mm}12.2.1  If $digidx = -1$ goto step 13. \\
\hspace{6mm}12.2.2  $buf \leftarrow x_{digidx}$ \\
\hspace{6mm}12.2.3  $digidx \leftarrow digidx - 1$ \\
\hspace{6mm}12.2.4  $bitcnt \leftarrow lg(\beta)$ \\
Continued on next page. \\
\hline
\end{tabular}
\end{center}
\end{small}
\caption{Algorithm s\_mp\_exptmod}
\end{figure}

\newpage\begin{figure}[!here]
\begin{small}
\begin{center}
\begin{tabular}{l}
\hline Algorithm \textbf{s\_mp\_exptmod} (\textit{continued}). \\
\textbf{Input}.   mp\_int $a$, $b$ and $c$ \\
\textbf{Output}.  $y \equiv g^x \mbox{ (mod }p\mbox{)}$ \\
\hline \\
\hspace{3mm}12.3  $y \leftarrow (buf >> (lg(\beta) - 1))$ AND $1$ \\
\hspace{3mm}12.4  $buf \leftarrow buf << 1$ \\
\hspace{3mm}12.5  if $mode = 0$ and $y = 0$ then goto step 12. \\
\hspace{3mm}12.6  if $mode = 1$ and $y = 0$ then do \\
\hspace{6mm}12.6.1  $res \leftarrow res^2$ \\
\hspace{6mm}12.6.2  $res \leftarrow res \mbox{ (mod }p\mbox{)}$ \\
\hspace{6mm}12.6.3  Goto step 12. \\
\hspace{3mm}12.7  $bitcpy \leftarrow bitcpy + 1$ \\
\hspace{3mm}12.8  $bitbuf \leftarrow bitbuf + (y << (winsize - bitcpy))$ \\
\hspace{3mm}12.9  $mode \leftarrow 2$ \\
\hspace{3mm}12.10  If $bitcpy = winsize$ then do \\
\hspace{6mm}Window is full so perform the squarings and single multiplication. \\
\hspace{6mm}12.10.1  for $ix$ from $0$ to $winsize -1$ do \\
\hspace{9mm}12.10.1.1  $res \leftarrow res^2$ \\
\hspace{9mm}12.10.1.2  $res \leftarrow res \mbox{ (mod }p\mbox{)}$ \\
\hspace{6mm}12.10.2  $res \leftarrow res \cdot M_{bitbuf}$ \\
\hspace{6mm}12.10.3  $res \leftarrow res \mbox{ (mod }p\mbox{)}$ \\
\hspace{6mm}Reset the window. \\
\hspace{6mm}12.10.4  $bitcpy \leftarrow 0, bitbuf \leftarrow 0, mode \leftarrow 1$ \\
\\
No more windows left.  Check for residual bits of exponent. \\
13.  If $mode = 2$ and $bitcpy > 0$ then do \\
\hspace{3mm}13.1  for $ix$ form $0$ to $bitcpy - 1$ do \\
\hspace{6mm}13.1.1  $res \leftarrow res^2$ \\
\hspace{6mm}13.1.2  $res \leftarrow res \mbox{ (mod }p\mbox{)}$ \\
\hspace{6mm}13.1.3  $bitbuf \leftarrow bitbuf << 1$ \\
\hspace{6mm}13.1.4  If $bitbuf$ AND $2^{winsize} \ne 0$ then do \\
\hspace{9mm}13.1.4.1  $res \leftarrow res \cdot M_{1}$ \\
\hspace{9mm}13.1.4.2  $res \leftarrow res \mbox{ (mod }p\mbox{)}$ \\
14.  $y \leftarrow res$ \\
15.  Clear $res$, $mu$ and the $M$ array. \\
16.  Return(\textit{MP\_OKAY}). \\
\hline
\end{tabular}
\end{center}
\end{small}
\caption{Algorithm s\_mp\_exptmod (continued)}
\end{figure}

\textbf{Algorithm s\_mp\_exptmod.}
This algorithm computes the $x$'th power of $g$ modulo $p$ and stores the result in $y$.  It takes advantage of the Barrett reduction
algorithm to keep the product small throughout the algorithm.

The first two steps determine the optimal window size based on the number of bits in the exponent.  The larger the exponent the 
larger the window size becomes.  After a window size $winsize$ has been chosen an array of $2^{winsize}$ mp\_int variables is allocated.  This
table will hold the values of $g^x \mbox{ (mod }p\mbox{)}$ for $2^{winsize - 1} \le x < 2^{winsize}$.  

After the table is allocated the first power of $g$ is found.  Since $g \ge p$ is allowed it must be first reduced modulo $p$ to make
the rest of the algorithm more efficient.  The first element of the table at $2^{winsize - 1}$ is found by squaring $M_1$ successively $winsize - 2$
times.  The rest of the table elements are found by multiplying the previous element by $M_1$ modulo $p$.

Now that the table is available the sliding window may begin.  The following list describes the functions of all the variables in the window.
\begin{enumerate}
\item The variable $mode$ dictates how the bits of the exponent are interpreted.  
\begin{enumerate}
   \item When $mode = 0$ the bits are ignored since no non-zero bit of the exponent has been seen yet.  For example, if the exponent were simply 
         $1$ then there would be $lg(\beta) - 1$ zero bits before the first non-zero bit.  In this case bits are ignored until a non-zero bit is found.  
   \item When $mode = 1$ a non-zero bit has been seen before and a new $winsize$-bit window has not been formed yet.  In this mode leading $0$ bits 
         are read and a single squaring is performed.  If a non-zero bit is read a new window is created.  
   \item When $mode = 2$ the algorithm is in the middle of forming a window and new bits are appended to the window from the most significant bit
         downwards.
\end{enumerate}
\item The variable $bitcnt$ indicates how many bits are left in the current digit of the exponent left to be read.  When it reaches zero a new digit
      is fetched from the exponent.
\item The variable $buf$ holds the currently read digit of the exponent. 
\item The variable $digidx$ is an index into the exponents digits.  It starts at the leading digit $x.used - 1$ and moves towards the trailing digit.
\item The variable $bitcpy$ indicates how many bits are in the currently formed window.  When it reaches $winsize$ the window is flushed and
      the appropriate operations performed.
\item The variable $bitbuf$ holds the current bits of the window being formed.  
\end{enumerate}

All of step 12 is the window processing loop.  It will iterate while there are digits available form the exponent to read.  The first step
inside this loop is to extract a new digit if no more bits are available in the current digit.  If there are no bits left a new digit is
read and if there are no digits left than the loop terminates.  

After a digit is made available step 12.3 will extract the most significant bit of the current digit and move all other bits in the digit
upwards.  In effect the digit is read from most significant bit to least significant bit and since the digits are read from leading to 
trailing edges the entire exponent is read from most significant bit to least significant bit.

At step 12.5 if the $mode$ and currently extracted bit $y$ are both zero the bit is ignored and the next bit is read.  This prevents the 
algorithm from having to perform trivial squaring and reduction operations before the first non-zero bit is read.  Step 12.6 and 12.7-10 handle
the two cases of $mode = 1$ and $mode = 2$ respectively.  

\begin{center}
\begin{figure}[here]
\includegraphics{pics/expt_state.ps}
\caption{Sliding Window State Diagram}
\label{pic:expt_state}
\end{figure}
\end{center}

By step 13 there are no more digits left in the exponent.  However, there may be partial bits in the window left.  If $mode = 2$ then 
a Left-to-Right algorithm is used to process the remaining few bits.  

\vspace{+3mm}\begin{small}
\hspace{-5.1mm}{\bf File}: bn\_s\_mp\_exptmod.c
\vspace{-3mm}
\begin{alltt}
\end{alltt}
\end{small}

Lines 32 through 46 determine the optimal window size based on the length of the exponent in bits.  The window divisions are sorted
from smallest to greatest so that in each \textbf{if} statement only one condition must be tested.  For example, by the \textbf{if} statement 
on line 38 the value of $x$ is already known to be greater than $140$.  

The conditional piece of code beginning on line 48 allows the window size to be restricted to five bits.  This logic is used to ensure
the table of precomputed powers of $G$ remains relatively small.  

The for loop on line 61 initializes the $M$ array while lines 72 and 77 through 86 initialize the reduction
function that will be used for this modulus.

-- More later.

\section{Quick Power of Two}
Calculating $b = 2^a$ can be performed much quicker than with any of the previous algorithms.  Recall that a logical shift left $m << k$ is
equivalent to $m \cdot 2^k$.  By this logic when $m = 1$ a quick power of two can be achieved.

\begin{figure}[!here]
\begin{small}
\begin{center}
\begin{tabular}{l}
\hline Algorithm \textbf{mp\_2expt}. \\
\textbf{Input}.   integer $b$ \\
\textbf{Output}.  $a \leftarrow 2^b$ \\
\hline \\
1.  $a \leftarrow 0$ \\
2.  If $a.alloc < \lfloor b / lg(\beta) \rfloor + 1$ then grow $a$ appropriately. \\
3.  $a.used \leftarrow \lfloor b / lg(\beta) \rfloor + 1$ \\
4.  $a_{\lfloor b / lg(\beta) \rfloor} \leftarrow 1 << (b \mbox{ mod } lg(\beta))$ \\
5.  Return(\textit{MP\_OKAY}). \\
\hline
\end{tabular}
\end{center}
\end{small}
\caption{Algorithm mp\_2expt}
\end{figure}

\textbf{Algorithm mp\_2expt.}

\vspace{+3mm}\begin{small}
\hspace{-5.1mm}{\bf File}: bn\_mp\_2expt.c
\vspace{-3mm}
\begin{alltt}
\end{alltt}
\end{small}

\chapter{Higher Level Algorithms}

This chapter discusses the various higher level algorithms that are required to complete a well rounded multiple precision integer package.  These
routines are less performance oriented than the algorithms of chapters five, six and seven but are no less important.  

The first section describes a method of integer division with remainder that is universally well known.  It provides the signed division logic
for the package.  The subsequent section discusses a set of algorithms which allow a single digit to be the 2nd operand for a variety of operations.  
These algorithms serve mostly to simplify other algorithms where small constants are required.  The last two sections discuss how to manipulate 
various representations of integers.  For example, converting from an mp\_int to a string of character.

\section{Integer Division with Remainder}
\label{sec:division}

Integer division aside from modular exponentiation is the most intensive algorithm to compute.  Like addition, subtraction and multiplication
the basis of this algorithm is the long-hand division algorithm taught to school children.  Throughout this discussion several common variables
will be used.  Let $x$ represent the divisor and $y$ represent the dividend.  Let $q$ represent the integer quotient $\lfloor y / x \rfloor$ and 
let $r$ represent the remainder $r = y - x \lfloor y / x \rfloor$.  The following simple algorithm will be used to start the discussion.

\newpage\begin{figure}[!here]
\begin{small}
\begin{center}
\begin{tabular}{l}
\hline Algorithm \textbf{Radix-$\beta$ Integer Division}. \\
\textbf{Input}.   integer $x$ and $y$ \\
\textbf{Output}.  $q = \lfloor y/x\rfloor, r = y - xq$ \\
\hline \\
1.  $q \leftarrow 0$ \\
2.  $n \leftarrow \vert \vert y \vert \vert - \vert \vert x \vert \vert$ \\
3.  for $t$ from $n$ down to $0$ do \\
\hspace{3mm}3.1  Maximize $k$ such that $kx\beta^t$ is less than or equal to $y$ and $(k + 1)x\beta^t$ is greater. \\
\hspace{3mm}3.2  $q \leftarrow q + k\beta^t$ \\
\hspace{3mm}3.3  $y \leftarrow y - kx\beta^t$ \\
4.  $r \leftarrow y$ \\
5.  Return($q, r$) \\
\hline
\end{tabular}
\end{center}
\end{small}
\caption{Algorithm Radix-$\beta$ Integer Division}
\label{fig:raddiv}
\end{figure}

As children we are taught this very simple algorithm for the case of $\beta = 10$.  Almost instinctively several optimizations are taught for which
their reason of existing are never explained.  For this example let $y = 5471$ represent the dividend and $x = 23$ represent the divisor.

To find the first digit of the quotient the value of $k$ must be maximized such that $kx\beta^t$ is less than or equal to $y$ and 
simultaneously $(k + 1)x\beta^t$ is greater than $y$.  Implicitly $k$ is the maximum value the $t$'th digit of the quotient may have.  The habitual method
used to find the maximum is to ``eyeball'' the two numbers, typically only the leading digits and quickly estimate a quotient.  By only using leading
digits a much simpler division may be used to form an educated guess at what the value must be.  In this case $k = \lfloor 54/23\rfloor = 2$ quickly 
arises as a possible  solution.  Indeed $2x\beta^2 = 4600$ is less than $y = 5471$ and simultaneously $(k + 1)x\beta^2 = 6900$ is larger than $y$.  
As a  result $k\beta^2$ is added to the quotient which now equals $q = 200$ and $4600$ is subtracted from $y$ to give a remainder of $y = 841$.

Again this process is repeated to produce the quotient digit $k = 3$ which makes the quotient $q = 200 + 3\beta = 230$ and the remainder 
$y = 841 - 3x\beta = 181$.  Finally the last iteration of the loop produces $k = 7$ which leads to the quotient $q = 230 + 7 = 237$ and the
remainder $y = 181 - 7x = 20$.  The final quotient and remainder found are $q = 237$ and $r = y = 20$ which are indeed correct since 
$237 \cdot 23 + 20 = 5471$ is true.  

\subsection{Quotient Estimation}
\label{sec:divest}
As alluded to earlier the quotient digit $k$ can be estimated from only the leading digits of both the divisor and dividend.  When $p$ leading
digits are used from both the divisor and dividend to form an estimation the accuracy of the estimation rises as $p$ grows.  Technically
speaking the estimation is based on assuming the lower $\vert \vert y \vert \vert - p$ and $\vert \vert x \vert \vert - p$ lower digits of the
dividend and divisor are zero.  

The value of the estimation may off by a few values in either direction and in general is fairly correct.  A simplification \cite[pp. 271]{TAOCPV2}
of the estimation technique is to use $t + 1$ digits of the dividend and $t$ digits of the divisor, in particularly when $t = 1$.  The estimate 
using this technique is never too small.  For the following proof let $t = \vert \vert y \vert \vert - 1$ and $s = \vert \vert x \vert \vert - 1$ 
represent the most significant digits of the dividend and divisor respectively.

\textbf{Proof.}\textit{  The quotient $\hat k = \lfloor (y_t\beta + y_{t-1}) / x_s \rfloor$ is greater than or equal to 
$k = \lfloor y / (x \cdot \beta^{\vert \vert y \vert \vert - \vert \vert x \vert \vert - 1}) \rfloor$. }
The first obvious case is when $\hat k = \beta - 1$ in which case the proof is concluded since the real quotient cannot be larger.  For all other 
cases $\hat k = \lfloor (y_t\beta + y_{t-1}) / x_s \rfloor$ and $\hat k x_s \ge y_t\beta + y_{t-1} - x_s + 1$.  The latter portion of the inequalility
$-x_s + 1$ arises from the fact that a truncated integer division will give the same quotient for at most $x_s - 1$ values.  Next a series of 
inequalities will prove the hypothesis.

\begin{equation}
y - \hat k x \le y - \hat k x_s\beta^s
\end{equation}

This is trivially true since $x \ge x_s\beta^s$.  Next we replace $\hat kx_s\beta^s$ by the previous inequality for $\hat kx_s$.  

\begin{equation}
y - \hat k x \le y_t\beta^t + \ldots + y_0 - (y_t\beta^t + y_{t-1}\beta^{t-1} - x_s\beta^t + \beta^s)
\end{equation}

By simplifying the previous inequality the following inequality is formed.

\begin{equation}
y - \hat k x \le y_{t-2}\beta^{t-2} + \ldots + y_0 + x_s\beta^s - \beta^s
\end{equation}

Subsequently,

\begin{equation}
y_{t-2}\beta^{t-2} + \ldots +  y_0  + x_s\beta^s - \beta^s < x_s\beta^s \le x
\end{equation}

Which proves that $y - \hat kx \le x$ and by consequence $\hat k \ge k$ which concludes the proof.  \textbf{QED}


\subsection{Normalized Integers}
For the purposes of division a normalized input is when the divisors leading digit $x_n$ is greater than or equal to $\beta / 2$.  By multiplying both
$x$ and $y$ by $j = \lfloor (\beta / 2) / x_n \rfloor$ the quotient remains unchanged and the remainder is simply $j$ times the original
remainder.  The purpose of normalization is to ensure the leading digit of the divisor is sufficiently large such that the estimated quotient will
lie in the domain of a single digit.  Consider the maximum dividend $(\beta - 1) \cdot \beta + (\beta - 1)$ and the minimum divisor $\beta / 2$.  

\begin{equation} 
{{\beta^2 - 1} \over { \beta / 2}} \le 2\beta - {2 \over \beta} 
\end{equation}

At most the quotient approaches $2\beta$, however, in practice this will not occur since that would imply the previous quotient digit was too small.  

\subsection{Radix-$\beta$ Division with Remainder}
\newpage\begin{figure}[!here]
\begin{small}
\begin{center}
\begin{tabular}{l}
\hline Algorithm \textbf{mp\_div}. \\
\textbf{Input}.   mp\_int $a, b$ \\
\textbf{Output}.  $c = \lfloor a/b \rfloor$, $d = a - bc$ \\
\hline \\
1.  If $b = 0$ return(\textit{MP\_VAL}). \\
2.  If $\vert a \vert < \vert b \vert$ then do \\
\hspace{3mm}2.1  $d \leftarrow a$ \\
\hspace{3mm}2.2  $c \leftarrow 0$ \\
\hspace{3mm}2.3  Return(\textit{MP\_OKAY}). \\
\\
Setup the quotient to receive the digits. \\
3.  Grow $q$ to $a.used + 2$ digits. \\
4.  $q \leftarrow 0$ \\
5.  $x \leftarrow \vert a \vert , y \leftarrow \vert b \vert$ \\
6.  $sign \leftarrow  \left \lbrace \begin{array}{ll}
                              MP\_ZPOS &  \mbox{if }a.sign = b.sign \\
                              MP\_NEG  &  \mbox{otherwise} \\
                              \end{array} \right .$ \\
\\
Normalize the inputs such that the leading digit of $y$ is greater than or equal to $\beta / 2$. \\
7.  $norm \leftarrow (lg(\beta) - 1) - (\lceil lg(y) \rceil \mbox{ (mod }lg(\beta)\mbox{)})$ \\
8.  $x \leftarrow x \cdot 2^{norm}, y \leftarrow y \cdot 2^{norm}$ \\
\\
Find the leading digit of the quotient. \\
9.  $n \leftarrow x.used - 1, t \leftarrow y.used - 1$ \\
10.  $y \leftarrow y \cdot \beta^{n - t}$ \\
11.  While ($x \ge y$) do \\
\hspace{3mm}11.1  $q_{n - t} \leftarrow q_{n - t} + 1$ \\
\hspace{3mm}11.2  $x \leftarrow x - y$ \\
12.  $y \leftarrow \lfloor y / \beta^{n-t} \rfloor$ \\
\\
Continued on the next page. \\
\hline
\end{tabular}
\end{center}
\end{small}
\caption{Algorithm mp\_div}
\end{figure}

\newpage\begin{figure}[!here]
\begin{small}
\begin{center}
\begin{tabular}{l}
\hline Algorithm \textbf{mp\_div} (continued). \\
\textbf{Input}.   mp\_int $a, b$ \\
\textbf{Output}.  $c = \lfloor a/b \rfloor$, $d = a - bc$ \\
\hline \\
Now find the remainder fo the digits. \\
13.  for $i$ from $n$ down to $(t + 1)$ do \\
\hspace{3mm}13.1  If $i > x.used$ then jump to the next iteration of this loop. \\
\hspace{3mm}13.2  If $x_{i} = y_{t}$ then \\
\hspace{6mm}13.2.1  $q_{i - t - 1} \leftarrow \beta - 1$ \\
\hspace{3mm}13.3  else \\
\hspace{6mm}13.3.1  $\hat r \leftarrow x_{i} \cdot \beta + x_{i - 1}$ \\
\hspace{6mm}13.3.2  $\hat r \leftarrow \lfloor \hat r / y_{t} \rfloor$ \\
\hspace{6mm}13.3.3  $q_{i - t - 1} \leftarrow \hat r$ \\
\hspace{3mm}13.4  $q_{i - t - 1} \leftarrow q_{i - t - 1} + 1$ \\
\\
Fixup quotient estimation. \\
\hspace{3mm}13.5  Loop \\
\hspace{6mm}13.5.1  $q_{i - t - 1} \leftarrow q_{i - t - 1} - 1$ \\
\hspace{6mm}13.5.2  t$1 \leftarrow 0$ \\
\hspace{6mm}13.5.3  t$1_0 \leftarrow y_{t - 1}, $ t$1_1 \leftarrow y_t,$ t$1.used \leftarrow 2$ \\
\hspace{6mm}13.5.4  $t1 \leftarrow t1 \cdot q_{i - t - 1}$ \\
\hspace{6mm}13.5.5  t$2_0 \leftarrow x_{i - 2}, $ t$2_1 \leftarrow x_{i - 1}, $ t$2_2 \leftarrow x_i, $ t$2.used \leftarrow 3$ \\
\hspace{6mm}13.5.6  If $\vert t1 \vert > \vert t2 \vert$ then goto step 13.5. \\
\hspace{3mm}13.6  t$1 \leftarrow y \cdot q_{i - t - 1}$ \\
\hspace{3mm}13.7  t$1 \leftarrow $ t$1 \cdot \beta^{i - t - 1}$ \\
\hspace{3mm}13.8  $x \leftarrow x - $ t$1$ \\
\hspace{3mm}13.9  If $x.sign = MP\_NEG$ then \\
\hspace{6mm}13.10  t$1 \leftarrow y$ \\
\hspace{6mm}13.11  t$1 \leftarrow $ t$1 \cdot \beta^{i - t - 1}$ \\
\hspace{6mm}13.12  $x \leftarrow x + $ t$1$ \\
\hspace{6mm}13.13  $q_{i - t - 1} \leftarrow q_{i - t - 1} - 1$ \\
\\
Finalize the result. \\
14.  Clamp excess digits of $q$ \\
15.  $c \leftarrow q, c.sign \leftarrow sign$ \\
16.  $x.sign \leftarrow a.sign$ \\
17.  $d \leftarrow \lfloor x / 2^{norm} \rfloor$ \\
18.  Return(\textit{MP\_OKAY}). \\
\hline
\end{tabular}
\end{center}
\end{small}
\caption{Algorithm mp\_div (continued)}
\end{figure}
\textbf{Algorithm mp\_div.}
This algorithm will calculate quotient and remainder from an integer division given a dividend and divisor.  The algorithm is a signed
division and will produce a fully qualified quotient and remainder.

First the divisor $b$ must be non-zero which is enforced in step one.  If the divisor is larger than the dividend than the quotient is implicitly 
zero and the remainder is the dividend.  

After the first two trivial cases of inputs are handled the variable $q$ is setup to receive the digits of the quotient.  Two unsigned copies of the
divisor $y$ and dividend $x$ are made as well.  The core of the division algorithm is an unsigned division and will only work if the values are
positive.  Now the two values $x$ and $y$ must be normalized such that the leading digit of $y$ is greater than or equal to $\beta / 2$.  
This is performed by shifting both to the left by enough bits to get the desired normalization.  

At this point the division algorithm can begin producing digits of the quotient.  Recall that maximum value of the estimation used is 
$2\beta - {2 \over \beta}$ which means that a digit of the quotient must be first produced by another means.  In this case $y$ is shifted
to the left (\textit{step ten}) so that it has the same number of digits as $x$.  The loop on step eleven will subtract multiples of the 
shifted copy of $y$ until $x$ is smaller.  Since the leading digit of $y$ is greater than or equal to $\beta/2$ this loop will iterate at most two
times to produce the desired leading digit of the quotient.  

Now the remainder of the digits can be produced.  The equation $\hat q = \lfloor {{x_i \beta + x_{i-1}}\over y_t} \rfloor$ is used to fairly
accurately approximate the true quotient digit.  The estimation can in theory produce an estimation as high as $2\beta - {2 \over \beta}$ but by
induction the upper quotient digit is correct (\textit{as established on step eleven}) and the estimate must be less than $\beta$.  

Recall from section~\ref{sec:divest} that the estimation is never too low but may be too high.  The next step of the estimation process is
to refine the estimation.  The loop on step 13.5 uses $x_i\beta^2 + x_{i-1}\beta + x_{i-2}$ and $q_{i - t - 1}(y_t\beta + y_{t-1})$ as a higher
order approximation to adjust the quotient digit.

After both phases of estimation the quotient digit may still be off by a value of one\footnote{This is similar to the error introduced
by optimizing Barrett reduction.}.  Steps 13.6 and 13.7 subtract the multiple of the divisor from the dividend (\textit{Similar to step 3.3 of
algorithm~\ref{fig:raddiv}} and then subsequently add a multiple of the divisor if the quotient was too large.  

Now that the quotient has been determine finializing the result is a matter of clamping the quotient, fixing the sizes and de-normalizing the 
remainder.  An important aspect of this algorithm seemingly overlooked in other descriptions such as that of Algorithm 14.20 HAC \cite[pp. 598]{HAC}
is that when the estimations are being made (\textit{inside the loop on step 13.5}) that the digits $y_{t-1}$, $x_{i-2}$ and $x_{i-1}$ may lie 
outside their respective boundaries.  For example, if $t = 0$ or $i \le 1$ then the digits would be undefined.  In those cases the digits should
respectively be replaced with a zero.  

\vspace{+3mm}\begin{small}
\hspace{-5.1mm}{\bf File}: bn\_mp\_div.c
\vspace{-3mm}
\begin{alltt}
\end{alltt}
\end{small}

The implementation of this algorithm differs slightly from the pseudo code presented previously.  In this algorithm either of the quotient $c$ or
remainder $d$ may be passed as a \textbf{NULL} pointer which indicates their value is not desired.  For example, the C code to call the division
algorithm with only the quotient is 

\begin{verbatim}
mp_div(&a, &b, &c, NULL);  /* c = [a/b] */
\end{verbatim}

Lines 109 and 113 handle the two trivial cases of inputs which are division by zero and dividend smaller than the divisor 
respectively.  After the two trivial cases all of the temporary variables are initialized.  Line 148 determines the sign of 
the quotient and line 148 ensures that both $x$ and $y$ are positive.  

The number of bits in the leading digit is calculated on line 151.  Implictly an mp\_int with $r$ digits will require $lg(\beta)(r-1) + k$ bits
of precision which when reduced modulo $lg(\beta)$ produces the value of $k$.  In this case $k$ is the number of bits in the leading digit which is
exactly what is required.  For the algorithm to operate $k$ must equal $lg(\beta) - 1$ and when it does not the inputs must be normalized by shifting
them to the left by $lg(\beta) - 1 - k$ bits.

Throughout the variables $n$ and $t$ will represent the highest digit of $x$ and $y$ respectively.  These are first used to produce the 
leading digit of the quotient.  The loop beginning on line 184 will produce the remainder of the quotient digits.

The conditional ``continue'' on line 187 is used to prevent the algorithm from reading past the leading edge of $x$ which can occur when the
algorithm eliminates multiple non-zero digits in a single iteration.  This ensures that $x_i$ is always non-zero since by definition the digits
above the $i$'th position $x$ must be zero in order for the quotient to be precise\footnote{Precise as far as integer division is concerned.}.  

Lines 214, 216 and 223 through 225 manually construct the high accuracy estimations by setting the digits of the two mp\_int 
variables directly.  

\section{Single Digit Helpers}

This section briefly describes a series of single digit helper algorithms which come in handy when working with small constants.  All of 
the helper functions assume the single digit input is positive and will treat them as such.

\subsection{Single Digit Addition and Subtraction}

Both addition and subtraction are performed by ``cheating'' and using mp\_set followed by the higher level addition or subtraction 
algorithms.   As a result these algorithms are subtantially simpler with a slight cost in performance.

\newpage\begin{figure}[!here]
\begin{small}
\begin{center}
\begin{tabular}{l}
\hline Algorithm \textbf{mp\_add\_d}. \\
\textbf{Input}.   mp\_int $a$ and a mp\_digit $b$ \\
\textbf{Output}.  $c = a + b$ \\
\hline \\
1.  $t \leftarrow b$ (\textit{mp\_set}) \\
2.  $c \leftarrow a + t$ \\
3.  Return(\textit{MP\_OKAY}) \\
\hline
\end{tabular}
\end{center}
\end{small}
\caption{Algorithm mp\_add\_d}
\end{figure}

\textbf{Algorithm mp\_add\_d.}
This algorithm initiates a temporary mp\_int with the value of the single digit and uses algorithm mp\_add to add the two values together.

\vspace{+3mm}\begin{small}
\hspace{-5.1mm}{\bf File}: bn\_mp\_add\_d.c
\vspace{-3mm}
\begin{alltt}
\end{alltt}
\end{small}

Clever use of the letter 't'.

\subsubsection{Subtraction}
The single digit subtraction algorithm mp\_sub\_d is essentially the same except it uses mp\_sub to subtract the digit from the mp\_int.

\subsection{Single Digit Multiplication}
Single digit multiplication arises enough in division and radix conversion that it ought to be implement as a special case of the baseline
multiplication algorithm.  Essentially this algorithm is a modified version of algorithm s\_mp\_mul\_digs where one of the multiplicands
only has one digit.

\begin{figure}[!here]
\begin{small}
\begin{center}
\begin{tabular}{l}
\hline Algorithm \textbf{mp\_mul\_d}. \\
\textbf{Input}.   mp\_int $a$ and a mp\_digit $b$ \\
\textbf{Output}.  $c = ab$ \\
\hline \\
1.  $pa \leftarrow a.used$ \\
2.  Grow $c$ to at least $pa + 1$ digits. \\
3.  $oldused \leftarrow c.used$ \\
4.  $c.used \leftarrow pa + 1$ \\
5.  $c.sign \leftarrow a.sign$ \\
6.  $\mu \leftarrow 0$ \\
7.  for $ix$ from $0$ to $pa - 1$ do \\
\hspace{3mm}7.1  $\hat r \leftarrow \mu + a_{ix}b$ \\
\hspace{3mm}7.2  $c_{ix} \leftarrow \hat r \mbox{ (mod }\beta\mbox{)}$ \\
\hspace{3mm}7.3  $\mu \leftarrow \lfloor \hat r / \beta \rfloor$ \\
8.  $c_{pa} \leftarrow \mu$ \\
9.  for $ix$ from $pa + 1$ to $oldused$ do \\
\hspace{3mm}9.1  $c_{ix} \leftarrow 0$ \\
10.  Clamp excess digits of $c$. \\
11.  Return(\textit{MP\_OKAY}). \\
\hline
\end{tabular}
\end{center}
\end{small}
\caption{Algorithm mp\_mul\_d}
\end{figure}
\textbf{Algorithm mp\_mul\_d.}
This algorithm quickly multiplies an mp\_int by a small single digit value.  It is specially tailored to the job and has a minimal of overhead.  
Unlike the full multiplication algorithms this algorithm does not require any significnat temporary storage or memory allocations.  

\vspace{+3mm}\begin{small}
\hspace{-5.1mm}{\bf File}: bn\_mp\_mul\_d.c
\vspace{-3mm}
\begin{alltt}
\end{alltt}
\end{small}

In this implementation the destination $c$ may point to the same mp\_int as the source $a$ since the result is written after the digit is 
read from the source.  This function uses pointer aliases $tmpa$ and $tmpc$ for the digits of $a$ and $c$ respectively.  

\subsection{Single Digit Division}
Like the single digit multiplication algorithm, single digit division is also a fairly common algorithm used in radix conversion.  Since the
divisor is only a single digit a specialized variant of the division algorithm can be used to compute the quotient.  

\newpage\begin{figure}[!here]
\begin{small}
\begin{center}
\begin{tabular}{l}
\hline Algorithm \textbf{mp\_div\_d}. \\
\textbf{Input}.   mp\_int $a$ and a mp\_digit $b$ \\
\textbf{Output}.  $c = \lfloor a / b \rfloor, d = a - cb$ \\
\hline \\
1.  If $b = 0$ then return(\textit{MP\_VAL}).\\
2.  If $b = 3$ then use algorithm mp\_div\_3 instead. \\
3.  Init $q$ to $a.used$ digits.  \\
4.  $q.used \leftarrow a.used$ \\
5.  $q.sign \leftarrow a.sign$ \\
6.  $\hat w \leftarrow 0$ \\
7.  for $ix$ from $a.used - 1$ down to $0$ do \\
\hspace{3mm}7.1  $\hat w \leftarrow \hat w \beta + a_{ix}$ \\
\hspace{3mm}7.2  If $\hat w \ge b$ then \\
\hspace{6mm}7.2.1  $t \leftarrow \lfloor \hat w / b \rfloor$ \\
\hspace{6mm}7.2.2  $\hat w \leftarrow \hat w \mbox{ (mod }b\mbox{)}$ \\
\hspace{3mm}7.3  else\\
\hspace{6mm}7.3.1  $t \leftarrow 0$ \\
\hspace{3mm}7.4  $q_{ix} \leftarrow t$ \\
8.  $d \leftarrow \hat w$ \\
9.  Clamp excess digits of $q$. \\
10.  $c \leftarrow q$ \\
11.  Return(\textit{MP\_OKAY}). \\
\hline
\end{tabular}
\end{center}
\end{small}
\caption{Algorithm mp\_div\_d}
\end{figure}
\textbf{Algorithm mp\_div\_d.}
This algorithm divides the mp\_int $a$ by the single mp\_digit $b$ using an optimized approach.  Essentially in every iteration of the
algorithm another digit of the dividend is reduced and another digit of quotient produced.  Provided $b < \beta$ the value of $\hat w$
after step 7.1 will be limited such that $0 \le \lfloor \hat w / b \rfloor < \beta$.  

If the divisor $b$ is equal to three a variant of this algorithm is used which is called mp\_div\_3.  It replaces the division by three with
a multiplication by $\lfloor \beta / 3 \rfloor$ and the appropriate shift and residual fixup.  In essence it is much like the Barrett reduction
from chapter seven.  

\vspace{+3mm}\begin{small}
\hspace{-5.1mm}{\bf File}: bn\_mp\_div\_d.c
\vspace{-3mm}
\begin{alltt}
\end{alltt}
\end{small}

Like the implementation of algorithm mp\_div this algorithm allows either of the quotient or remainder to be passed as a \textbf{NULL} pointer to
indicate the respective value is not required.  This allows a trivial single digit modular reduction algorithm, mp\_mod\_d to be created.

The division and remainder on lines 44 and @45,%@ can be replaced often by a single division on most processors.  For example, the 32-bit x86 based 
processors can divide a 64-bit quantity by a 32-bit quantity and produce the quotient and remainder simultaneously.  Unfortunately the GCC 
compiler does not recognize that optimization and will actually produce two function calls to find the quotient and remainder respectively.  

\subsection{Single Digit Root Extraction}

Finding the $n$'th root of an integer is fairly easy as far as numerical analysis is concerned.  Algorithms such as the Newton-Raphson approximation 
(\ref{eqn:newton}) series will converge very quickly to a root for any continuous function $f(x)$.  

\begin{equation}
x_{i+1} = x_i - {f(x_i) \over f'(x_i)}
\label{eqn:newton}
\end{equation}

In this case the $n$'th root is desired and $f(x) = x^n - a$ where $a$ is the integer of which the root is desired.  The derivative of $f(x)$ is 
simply $f'(x) = nx^{n - 1}$.  Of particular importance is that this algorithm will be used over the integers not over the a more continuous domain
such as the real numbers.  As a result the root found can be above the true root by few and must be manually adjusted.  Ideally at the end of the 
algorithm the $n$'th root $b$ of an integer $a$ is desired such that $b^n \le a$.  

\newpage\begin{figure}[!here]
\begin{small}
\begin{center}
\begin{tabular}{l}
\hline Algorithm \textbf{mp\_n\_root}. \\
\textbf{Input}.   mp\_int $a$ and a mp\_digit $b$ \\
\textbf{Output}.  $c^b \le a$ \\
\hline \\
1.  If $b$ is even and $a.sign = MP\_NEG$ return(\textit{MP\_VAL}). \\
2.  $sign \leftarrow a.sign$ \\
3.  $a.sign \leftarrow MP\_ZPOS$ \\
4.  t$2 \leftarrow 2$ \\
5.  Loop \\
\hspace{3mm}5.1  t$1 \leftarrow $ t$2$ \\
\hspace{3mm}5.2  t$3 \leftarrow $ t$1^{b - 1}$ \\
\hspace{3mm}5.3  t$2 \leftarrow $ t$3 $ $\cdot$ t$1$ \\
\hspace{3mm}5.4  t$2 \leftarrow $ t$2 - a$ \\
\hspace{3mm}5.5  t$3 \leftarrow $ t$3 \cdot b$ \\
\hspace{3mm}5.6  t$3 \leftarrow \lfloor $t$2 / $t$3 \rfloor$ \\
\hspace{3mm}5.7  t$2 \leftarrow $ t$1 - $ t$3$ \\
\hspace{3mm}5.8  If t$1 \ne $ t$2$ then goto step 5.  \\
6.  Loop \\
\hspace{3mm}6.1  t$2 \leftarrow $ t$1^b$ \\
\hspace{3mm}6.2  If t$2 > a$ then \\
\hspace{6mm}6.2.1  t$1 \leftarrow $ t$1 - 1$ \\
\hspace{6mm}6.2.2  Goto step 6. \\
7.  $a.sign \leftarrow sign$ \\
8.  $c \leftarrow $ t$1$ \\
9.  $c.sign \leftarrow sign$  \\
10.  Return(\textit{MP\_OKAY}).  \\
\hline
\end{tabular}
\end{center}
\end{small}
\caption{Algorithm mp\_n\_root}
\end{figure}
\textbf{Algorithm mp\_n\_root.}
This algorithm finds the integer $n$'th root of an input using the Newton-Raphson approach.  It is partially optimized based on the observation
that the numerator of ${f(x) \over f'(x)}$ can be derived from a partial denominator.  That is at first the denominator is calculated by finding
$x^{b - 1}$.  This value can then be multiplied by $x$ and have $a$ subtracted from it to find the numerator.  This saves a total of $b - 1$ 
multiplications by t$1$ inside the loop.  

The initial value of the approximation is t$2 = 2$ which allows the algorithm to start with very small values and quickly converge on the
root.  Ideally this algorithm is meant to find the $n$'th root of an input where $n$ is bounded by $2 \le n \le 5$.  

\vspace{+3mm}\begin{small}
\hspace{-5.1mm}{\bf File}: bn\_mp\_n\_root.c
\vspace{-3mm}
\begin{alltt}
\end{alltt}
\end{small}

\section{Random Number Generation}

Random numbers come up in a variety of activities from public key cryptography to simple simulations and various randomized algorithms.  Pollard-Rho 
factoring for example, can make use of random values as starting points to find factors of a composite integer.  In this case the algorithm presented
is solely for simulations and not intended for cryptographic use.  

\newpage\begin{figure}[!here]
\begin{small}
\begin{center}
\begin{tabular}{l}
\hline Algorithm \textbf{mp\_rand}. \\
\textbf{Input}.   An integer $b$ \\
\textbf{Output}.  A pseudo-random number of $b$ digits \\
\hline \\
1.  $a \leftarrow 0$ \\
2.  If $b \le 0$ return(\textit{MP\_OKAY}) \\
3.  Pick a non-zero random digit $d$. \\
4.  $a \leftarrow a + d$ \\
5.  for $ix$ from 1 to $d - 1$ do \\
\hspace{3mm}5.1  $a \leftarrow a \cdot \beta$ \\
\hspace{3mm}5.2  Pick a random digit $d$. \\
\hspace{3mm}5.3  $a \leftarrow a + d$ \\
6.  Return(\textit{MP\_OKAY}). \\
\hline
\end{tabular}
\end{center}
\end{small}
\caption{Algorithm mp\_rand}
\end{figure}
\textbf{Algorithm mp\_rand.}
This algorithm produces a pseudo-random integer of $b$ digits.  By ensuring that the first digit is non-zero the algorithm also guarantees that the
final result has at least $b$ digits.  It relies heavily on a third-part random number generator which should ideally generate uniformly all of
the integers from $0$ to $\beta - 1$.  

\vspace{+3mm}\begin{small}
\hspace{-5.1mm}{\bf File}: bn\_mp\_rand.c
\vspace{-3mm}
\begin{alltt}
\end{alltt}
\end{small}

\section{Formatted Representations}
The ability to emit a radix-$n$ textual representation of an integer is useful for interacting with human parties.  For example, the ability to
be given a string of characters such as ``114585'' and turn it into the radix-$\beta$ equivalent would make it easier to enter numbers
into a program.

\subsection{Reading Radix-n Input}
For the purposes of this text we will assume that a simple lower ASCII map (\ref{fig:ASC}) is used for the values of from $0$ to $63$ to 
printable characters.  For example, when the character ``N'' is read it represents the integer $23$.  The first $16$ characters of the
map are for the common representations up to hexadecimal.  After that they match the ``base64'' encoding scheme which are suitable chosen
such that they are printable.  While outputting as base64 may not be too helpful for human operators it does allow communication via non binary
mediums.

\newpage\begin{figure}[here]
\begin{center}
\begin{tabular}{cc|cc|cc|cc}
\hline \textbf{Value} & \textbf{Char} & \textbf{Value} & \textbf{Char} & \textbf{Value} & \textbf{Char} &  \textbf{Value} & \textbf{Char} \\
\hline 
0 & 0 & 1 & 1 & 2 & 2 & 3 & 3 \\
4 & 4 & 5 & 5 & 6 & 6 & 7 & 7 \\
8 & 8 & 9 & 9 & 10 & A & 11 & B \\
12 & C & 13 & D & 14 & E & 15 & F \\
16 & G & 17 & H & 18 & I & 19 & J \\
20 & K & 21 & L & 22 & M & 23 & N \\
24 & O & 25 & P & 26 & Q & 27 & R \\
28 & S & 29 & T & 30 & U & 31 & V \\
32 & W & 33 & X & 34 & Y & 35 & Z \\
36 & a & 37 & b & 38 & c & 39 & d \\
40 & e & 41 & f & 42 & g & 43 & h \\
44 & i & 45 & j & 46 & k & 47 & l \\
48 & m & 49 & n & 50 & o & 51 & p \\
52 & q & 53 & r & 54 & s & 55 & t \\
56 & u & 57 & v & 58 & w & 59 & x \\
60 & y & 61 & z & 62 & $+$ & 63 & $/$ \\
\hline
\end{tabular}
\end{center}
\caption{Lower ASCII Map}
\label{fig:ASC}
\end{figure}

\newpage\begin{figure}[!here]
\begin{small}
\begin{center}
\begin{tabular}{l}
\hline Algorithm \textbf{mp\_read\_radix}. \\
\textbf{Input}.   A string $str$ of length $sn$ and radix $r$. \\
\textbf{Output}.  The radix-$\beta$ equivalent mp\_int. \\
\hline \\
1.  If $r < 2$ or $r > 64$ return(\textit{MP\_VAL}). \\
2.  $ix \leftarrow 0$ \\
3.  If $str_0 =$ ``-'' then do \\
\hspace{3mm}3.1  $ix \leftarrow ix + 1$ \\
\hspace{3mm}3.2  $sign \leftarrow MP\_NEG$ \\
4.  else \\
\hspace{3mm}4.1  $sign \leftarrow MP\_ZPOS$ \\
5.  $a \leftarrow 0$ \\
6.  for $iy$ from $ix$ to $sn - 1$ do \\
\hspace{3mm}6.1  Let $y$ denote the position in the map of $str_{iy}$. \\
\hspace{3mm}6.2  If $str_{iy}$ is not in the map or $y \ge r$ then goto step 7. \\
\hspace{3mm}6.3  $a \leftarrow a \cdot r$ \\
\hspace{3mm}6.4  $a \leftarrow a + y$ \\
7.  If $a \ne 0$ then $a.sign \leftarrow sign$ \\
8.  Return(\textit{MP\_OKAY}). \\
\hline
\end{tabular}
\end{center}
\end{small}
\caption{Algorithm mp\_read\_radix}
\end{figure}
\textbf{Algorithm mp\_read\_radix.}
This algorithm will read an ASCII string and produce the radix-$\beta$ mp\_int representation of the same integer.  A minus symbol ``-'' may precede the 
string  to indicate the value is negative, otherwise it is assumed to be positive.  The algorithm will read up to $sn$ characters from the input
and will stop when it reads a character it cannot map the algorithm stops reading characters from the string.  This allows numbers to be embedded
as part of larger input without any significant problem.

\vspace{+3mm}\begin{small}
\hspace{-5.1mm}{\bf File}: bn\_mp\_read\_radix.c
\vspace{-3mm}
\begin{alltt}
\end{alltt}
\end{small}

\subsection{Generating Radix-$n$ Output}
Generating radix-$n$ output is fairly trivial with a division and remainder algorithm.  

\newpage\begin{figure}[!here]
\begin{small}
\begin{center}
\begin{tabular}{l}
\hline Algorithm \textbf{mp\_toradix}. \\
\textbf{Input}.   A mp\_int $a$ and an integer $r$\\
\textbf{Output}.  The radix-$r$ representation of $a$ \\
\hline \\
1.  If $r < 2$ or $r > 64$ return(\textit{MP\_VAL}). \\
2.  If $a = 0$ then $str = $ ``$0$'' and return(\textit{MP\_OKAY}).  \\
3.  $t \leftarrow a$ \\
4.  $str \leftarrow$ ``'' \\
5.  if $t.sign = MP\_NEG$ then \\
\hspace{3mm}5.1  $str \leftarrow str + $ ``-'' \\
\hspace{3mm}5.2  $t.sign = MP\_ZPOS$ \\
6.  While ($t \ne 0$) do \\
\hspace{3mm}6.1  $d \leftarrow t \mbox{ (mod }r\mbox{)}$ \\
\hspace{3mm}6.2  $t \leftarrow \lfloor t / r \rfloor$ \\
\hspace{3mm}6.3  Look up $d$ in the map and store the equivalent character in $y$. \\
\hspace{3mm}6.4  $str \leftarrow str + y$ \\
7.  If $str_0 = $``$-$'' then \\
\hspace{3mm}7.1  Reverse the digits $str_1, str_2, \ldots str_n$. \\
8.  Otherwise \\
\hspace{3mm}8.1  Reverse the digits $str_0, str_1, \ldots str_n$. \\
9.  Return(\textit{MP\_OKAY}).\\
\hline
\end{tabular}
\end{center}
\end{small}
\caption{Algorithm mp\_toradix}
\end{figure}
\textbf{Algorithm mp\_toradix.}
This algorithm computes the radix-$r$ representation of an mp\_int $a$.  The ``digits'' of the representation are extracted by reducing 
successive powers of $\lfloor a / r^k \rfloor$ the input modulo $r$ until $r^k > a$.  Note that instead of actually dividing by $r^k$ in
each iteration the quotient $\lfloor a / r \rfloor$ is saved for the next iteration.  As a result a series of trivial $n \times 1$ divisions
are required instead of a series of $n \times k$ divisions.  One design flaw of this approach is that the digits are produced in the reverse order 
(see~\ref{fig:mpradix}).  To remedy this flaw the digits must be swapped or simply ``reversed''.

\begin{figure}
\begin{center}
\begin{tabular}{|c|c|c|}
\hline \textbf{Value of $a$} & \textbf{Value of $d$} & \textbf{Value of $str$} \\
\hline $1234$ & -- & -- \\
\hline $123$  & $4$ & ``4'' \\
\hline $12$   & $3$ & ``43'' \\
\hline $1$    & $2$ & ``432'' \\
\hline $0$    & $1$ & ``4321'' \\
\hline
\end{tabular}
\end{center}
\caption{Example of Algorithm mp\_toradix.}
\label{fig:mpradix}
\end{figure}

\vspace{+3mm}\begin{small}
\hspace{-5.1mm}{\bf File}: bn\_mp\_toradix.c
\vspace{-3mm}
\begin{alltt}
\end{alltt}
\end{small}

\chapter{Number Theoretic Algorithms}
This chapter discusses several fundamental number theoretic algorithms such as the greatest common divisor, least common multiple and Jacobi 
symbol computation.  These algorithms arise as essential components in several key cryptographic algorithms such as the RSA public key algorithm and
various Sieve based factoring algorithms.

\section{Greatest Common Divisor}
The greatest common divisor of two integers $a$ and $b$, often denoted as $(a, b)$ is the largest integer $k$ that is a proper divisor of
both $a$ and $b$.  That is, $k$ is the largest integer such that $0 \equiv a \mbox{ (mod }k\mbox{)}$ and $0 \equiv b \mbox{ (mod }k\mbox{)}$ occur
simultaneously.

The most common approach (cite) is to reduce one input modulo another.  That is if $a$ and $b$ are divisible by some integer $k$ and if $qa + r = b$ then
$r$ is also divisible by $k$.  The reduction pattern follows $\left < a , b \right > \rightarrow \left < b, a \mbox{ mod } b \right >$.  

\newpage\begin{figure}[!here]
\begin{small}
\begin{center}
\begin{tabular}{l}
\hline Algorithm \textbf{Greatest Common Divisor (I)}. \\
\textbf{Input}.   Two positive integers $a$ and $b$ greater than zero. \\
\textbf{Output}.  The greatest common divisor $(a, b)$.  \\
\hline \\
1.  While ($b > 0$) do \\
\hspace{3mm}1.1  $r \leftarrow a \mbox{ (mod }b\mbox{)}$ \\
\hspace{3mm}1.2  $a \leftarrow b$ \\
\hspace{3mm}1.3  $b \leftarrow r$ \\
2.  Return($a$). \\
\hline
\end{tabular}
\end{center}
\end{small}
\caption{Algorithm Greatest Common Divisor (I)}
\label{fig:gcd1}
\end{figure}

This algorithm will quickly converge on the greatest common divisor since the residue $r$ tends diminish rapidly.  However, divisions are
relatively expensive operations to perform and should ideally be avoided.  There is another approach based on a similar relationship of 
greatest common divisors.  The faster approach is based on the observation that if $k$ divides both $a$ and $b$ it will also divide $a - b$.  
In particular, we would like $a - b$ to decrease in magnitude which implies that $b \ge a$.  

\begin{figure}[!here]
\begin{small}
\begin{center}
\begin{tabular}{l}
\hline Algorithm \textbf{Greatest Common Divisor (II)}. \\
\textbf{Input}.   Two positive integers $a$ and $b$ greater than zero. \\
\textbf{Output}.  The greatest common divisor $(a, b)$.  \\
\hline \\
1.  While ($b > 0$) do \\
\hspace{3mm}1.1  Swap $a$ and $b$ such that $a$ is the smallest of the two. \\
\hspace{3mm}1.2  $b \leftarrow b - a$ \\
2.  Return($a$). \\
\hline
\end{tabular}
\end{center}
\end{small}
\caption{Algorithm Greatest Common Divisor (II)}
\label{fig:gcd2}
\end{figure}

\textbf{Proof} \textit{Algorithm~\ref{fig:gcd2} will return the greatest common divisor of $a$ and $b$.}
The algorithm in figure~\ref{fig:gcd2} will eventually terminate since $b \ge a$ the subtraction in step 1.2 will be a value less than $b$.  In other
words in every iteration that tuple $\left < a, b \right >$ decrease in magnitude until eventually $a = b$.  Since both $a$ and $b$ are always 
divisible by the greatest common divisor (\textit{until the last iteration}) and in the last iteration of the algorithm $b = 0$, therefore, in the 
second to last iteration of the algorithm $b = a$ and clearly $(a, a) = a$ which concludes the proof.  \textbf{QED}.

As a matter of practicality algorithm \ref{fig:gcd1} decreases far too slowly to be useful.  Specially if $b$ is much larger than $a$ such that 
$b - a$ is still very much larger than $a$.  A simple addition to the algorithm is to divide $b - a$ by a power of some integer $p$ which does
not divide the greatest common divisor but will divide $b - a$.  In this case ${b - a} \over p$ is also an integer and still divisible by
the greatest common divisor.

However, instead of factoring $b - a$ to find a suitable value of $p$ the powers of $p$ can be removed from $a$ and $b$ that are in common first.  
Then inside the loop whenever $b - a$ is divisible by some power of $p$ it can be safely removed.  

\begin{figure}[!here]
\begin{small}
\begin{center}
\begin{tabular}{l}
\hline Algorithm \textbf{Greatest Common Divisor (III)}. \\
\textbf{Input}.   Two positive integers $a$ and $b$ greater than zero. \\
\textbf{Output}.  The greatest common divisor $(a, b)$.  \\
\hline \\
1.  $k \leftarrow 0$ \\
2.  While $a$ and $b$ are both divisible by $p$ do \\
\hspace{3mm}2.1  $a \leftarrow \lfloor a / p \rfloor$ \\
\hspace{3mm}2.2  $b \leftarrow \lfloor b / p \rfloor$ \\
\hspace{3mm}2.3  $k \leftarrow k + 1$ \\
3.  While $a$ is divisible by $p$ do \\
\hspace{3mm}3.1  $a \leftarrow \lfloor a / p \rfloor$ \\
4.  While $b$ is divisible by $p$ do \\
\hspace{3mm}4.1  $b \leftarrow \lfloor b / p \rfloor$ \\
5.  While ($b > 0$) do \\
\hspace{3mm}5.1  Swap $a$ and $b$ such that $a$ is the smallest of the two. \\
\hspace{3mm}5.2  $b \leftarrow b - a$ \\
\hspace{3mm}5.3  While $b$ is divisible by $p$ do \\
\hspace{6mm}5.3.1  $b \leftarrow \lfloor b / p \rfloor$ \\
6.  Return($a \cdot p^k$). \\
\hline
\end{tabular}
\end{center}
\end{small}
\caption{Algorithm Greatest Common Divisor (III)}
\label{fig:gcd3}
\end{figure}

This algorithm is based on the first except it removes powers of $p$ first and inside the main loop to ensure the tuple $\left < a, b \right >$ 
decreases more rapidly.  The first loop on step two removes powers of $p$ that are in common.  A count, $k$, is kept which will present a common
divisor of $p^k$.  After step two the remaining common divisor of $a$ and $b$ cannot be divisible by $p$.  This means that $p$ can be safely 
divided out of the difference $b - a$ so long as the division leaves no remainder.  

In particular the value of $p$ should be chosen such that the division on step 5.3.1 occur often.  It also helps that division by $p$ be easy
to compute.  The ideal choice of $p$ is two since division by two amounts to a right logical shift.  Another important observation is that by
step five both $a$ and $b$ are odd.  Therefore, the diffrence $b - a$ must be even which means that each iteration removes one bit from the 
largest of the pair.

\subsection{Complete Greatest Common Divisor}
The algorithms presented so far cannot handle inputs which are zero or negative.  The following algorithm can handle all input cases properly
and will produce the greatest common divisor.

\newpage\begin{figure}[!here]
\begin{small}
\begin{center}
\begin{tabular}{l}
\hline Algorithm \textbf{mp\_gcd}. \\
\textbf{Input}.   mp\_int $a$ and $b$ \\
\textbf{Output}.  The greatest common divisor $c = (a, b)$.  \\
\hline \\
1.  If $a = 0$ then \\
\hspace{3mm}1.1  $c \leftarrow \vert b \vert $ \\
\hspace{3mm}1.2  Return(\textit{MP\_OKAY}). \\
2.  If $b = 0$ then \\
\hspace{3mm}2.1  $c \leftarrow \vert a \vert $ \\
\hspace{3mm}2.2  Return(\textit{MP\_OKAY}). \\
3.  $u \leftarrow \vert a \vert, v \leftarrow \vert b \vert$ \\
4.  $k \leftarrow 0$ \\
5.  While $u.used > 0$ and $v.used > 0$ and $u_0 \equiv v_0 \equiv 0 \mbox{ (mod }2\mbox{)}$ \\
\hspace{3mm}5.1  $k \leftarrow k + 1$ \\
\hspace{3mm}5.2  $u \leftarrow \lfloor u / 2 \rfloor$ \\
\hspace{3mm}5.3  $v \leftarrow \lfloor v / 2 \rfloor$ \\
6.  While $u.used > 0$ and $u_0 \equiv 0 \mbox{ (mod }2\mbox{)}$ \\
\hspace{3mm}6.1  $u \leftarrow \lfloor u / 2 \rfloor$ \\
7.  While $v.used > 0$ and $v_0 \equiv 0 \mbox{ (mod }2\mbox{)}$ \\
\hspace{3mm}7.1  $v \leftarrow \lfloor v / 2 \rfloor$ \\
8.  While $v.used > 0$ \\
\hspace{3mm}8.1  If $\vert u \vert > \vert v \vert$ then \\
\hspace{6mm}8.1.1  Swap $u$ and $v$. \\
\hspace{3mm}8.2  $v \leftarrow \vert v \vert - \vert u \vert$ \\
\hspace{3mm}8.3  While $v.used > 0$ and $v_0 \equiv 0 \mbox{ (mod }2\mbox{)}$ \\
\hspace{6mm}8.3.1  $v \leftarrow \lfloor v / 2 \rfloor$ \\
9.  $c \leftarrow u \cdot 2^k$ \\
10.  Return(\textit{MP\_OKAY}). \\
\hline
\end{tabular}
\end{center}
\end{small}
\caption{Algorithm mp\_gcd}
\end{figure}
\textbf{Algorithm mp\_gcd.}
This algorithm will produce the greatest common divisor of two mp\_ints $a$ and $b$.  The algorithm was originally based on Algorithm B of
Knuth \cite[pp. 338]{TAOCPV2} but has been modified to be simpler to explain.  In theory it achieves the same asymptotic working time as
Algorithm B and in practice this appears to be true.  

The first two steps handle the cases where either one of or both inputs are zero.  If either input is zero the greatest common divisor is the 
largest input or zero if they are both zero.  If the inputs are not trivial than $u$ and $v$ are assigned the absolute values of 
$a$ and $b$ respectively and the algorithm will proceed to reduce the pair.

Step five will divide out any common factors of two and keep track of the count in the variable $k$.  After this step, two is no longer a
factor of the remaining greatest common divisor between $u$ and $v$ and can be safely evenly divided out of either whenever they are even.  Step 
six and seven ensure that the $u$ and $v$ respectively have no more factors of two.  At most only one of the while--loops will iterate since 
they cannot both be even.

By step eight both of $u$ and $v$ are odd which is required for the inner logic.  First the pair are swapped such that $v$ is equal to
or greater than $u$.  This ensures that the subtraction on step 8.2 will always produce a positive and even result.  Step 8.3 removes any
factors of two from the difference $u$ to ensure that in the next iteration of the loop both are once again odd.

After $v = 0$ occurs the variable $u$ has the greatest common divisor of the pair $\left < u, v \right >$ just after step six.  The result
must be adjusted by multiplying by the common factors of two ($2^k$) removed earlier.  

\vspace{+3mm}\begin{small}
\hspace{-5.1mm}{\bf File}: bn\_mp\_gcd.c
\vspace{-3mm}
\begin{alltt}
\end{alltt}
\end{small}

This function makes use of the macros mp\_iszero and mp\_iseven.  The former evaluates to $1$ if the input mp\_int is equivalent to the 
integer zero otherwise it evaluates to $0$.  The latter evaluates to $1$ if the input mp\_int represents a non-zero even integer otherwise
it evaluates to $0$.  Note that just because mp\_iseven may evaluate to $0$ does not mean the input is odd, it could also be zero.  The three 
trivial cases of inputs are handled on lines 24 through 30.  After those lines the inputs are assumed to be non-zero.

Lines 32 and 37 make local copies $u$ and $v$ of the inputs $a$ and $b$ respectively.  At this point the common factors of two 
must be divided out of the two inputs.  The block starting at line 44 removes common factors of two by first counting the number of trailing
zero bits in both.  The local integer $k$ is used to keep track of how many factors of $2$ are pulled out of both values.  It is assumed that 
the number of factors will not exceed the maximum value of a C ``int'' data type\footnote{Strictly speaking no array in C may have more than 
entries than are accessible by an ``int'' so this is not a limitation.}.  

At this point there are no more common factors of two in the two values.  The divisions by a power of two on lines 62 and 68 remove 
any independent factors of two such that both $u$ and $v$ are guaranteed to be an odd integer before hitting the main body of the algorithm.  The while loop
on line 73 performs the reduction of the pair until $v$ is equal to zero.  The unsigned comparison and subtraction algorithms are used in
place of the full signed routines since both values are guaranteed to be positive and the result of the subtraction is guaranteed to be non-negative.

\section{Least Common Multiple}
The least common multiple of a pair of integers is their product divided by their greatest common divisor.  For two integers $a$ and $b$ the
least common multiple is normally denoted as $[ a, b ]$ and numerically equivalent to ${ab} \over {(a, b)}$.  For example, if $a = 2 \cdot 2 \cdot 3 = 12$
and $b = 2 \cdot 3 \cdot 3 \cdot 7 = 126$ the least common multiple is ${126 \over {(12, 126)}} = {126 \over 6} = 21$.

The least common multiple arises often in coding theory as well as number theory.  If two functions have periods of $a$ and $b$ respectively they will
collide, that is be in synchronous states, after only $[ a, b ]$ iterations.  This is why, for example, random number generators based on 
Linear Feedback Shift Registers (LFSR) tend to use registers with periods which are co-prime (\textit{e.g. the greatest common divisor is one.}).  
Similarly in number theory if a composite $n$ has two prime factors $p$ and $q$ then maximal order of any unit of $\Z/n\Z$ will be $[ p - 1, q - 1] $.

\begin{figure}[!here]
\begin{small}
\begin{center}
\begin{tabular}{l}
\hline Algorithm \textbf{mp\_lcm}. \\
\textbf{Input}.   mp\_int $a$ and $b$ \\
\textbf{Output}.  The least common multiple $c = [a, b]$.  \\
\hline \\
1.  $c \leftarrow (a, b)$ \\
2.  $t \leftarrow a \cdot b$ \\
3.  $c \leftarrow \lfloor t / c \rfloor$ \\
4.  Return(\textit{MP\_OKAY}). \\
\hline
\end{tabular}
\end{center}
\end{small}
\caption{Algorithm mp\_lcm}
\end{figure}
\textbf{Algorithm mp\_lcm.}
This algorithm computes the least common multiple of two mp\_int inputs $a$ and $b$.  It computes the least common multiple directly by
dividing the product of the two inputs by their greatest common divisor.

\vspace{+3mm}\begin{small}
\hspace{-5.1mm}{\bf File}: bn\_mp\_lcm.c
\vspace{-3mm}
\begin{alltt}
\end{alltt}
\end{small}

\section{Jacobi Symbol Computation}
To explain the Jacobi Symbol we shall first discuss the Legendre function\footnote{Arrg.  What is the name of this?} off which the Jacobi symbol is 
defined.  The Legendre function computes whether or not an integer $a$ is a quadratic residue modulo an odd prime $p$.  Numerically it is
equivalent to equation \ref{eqn:legendre}.

\textit{-- Tom, don't be an ass, cite your source here...!}

\begin{equation}
a^{(p-1)/2} \equiv \begin{array}{rl}
                              -1 &  \mbox{if }a\mbox{ is a quadratic non-residue.} \\
                              0  &  \mbox{if }a\mbox{ divides }p\mbox{.} \\
                              1  &  \mbox{if }a\mbox{ is a quadratic residue}. 
                              \end{array} \mbox{ (mod }p\mbox{)}
\label{eqn:legendre}                              
\end{equation}

\textbf{Proof.} \textit{Equation \ref{eqn:legendre} correctly identifies the residue status of an integer $a$ modulo a prime $p$.}
An integer $a$ is a quadratic residue if the following equation has a solution.

\begin{equation}
x^2 \equiv a \mbox{ (mod }p\mbox{)}
\label{eqn:root}
\end{equation}

Consider the following equation.

\begin{equation}
0 \equiv x^{p-1} - 1 \equiv \left \lbrace \left (x^2 \right )^{(p-1)/2} - a^{(p-1)/2} \right \rbrace + \left ( a^{(p-1)/2} - 1 \right ) \mbox{ (mod }p\mbox{)}
\label{eqn:rooti}
\end{equation}

Whether equation \ref{eqn:root} has a solution or not equation \ref{eqn:rooti} is always true.  If $a^{(p-1)/2} - 1 \equiv 0 \mbox{ (mod }p\mbox{)}$
then the quantity in the braces must be zero.  By reduction,

\begin{eqnarray}
\left (x^2 \right )^{(p-1)/2} - a^{(p-1)/2} \equiv 0  \nonumber \\
\left (x^2 \right )^{(p-1)/2} \equiv a^{(p-1)/2} \nonumber \\
x^2 \equiv a \mbox{ (mod }p\mbox{)} 
\end{eqnarray}

As a result there must be a solution to the quadratic equation and in turn $a$ must be a quadratic residue.  If $a$ does not divide $p$ and $a$
is not a quadratic residue then the only other value $a^{(p-1)/2}$ may be congruent to is $-1$ since
\begin{equation}
0 \equiv a^{p - 1} - 1 \equiv (a^{(p-1)/2} + 1)(a^{(p-1)/2} - 1) \mbox{ (mod }p\mbox{)}
\end{equation}
One of the terms on the right hand side must be zero.  \textbf{QED}

\subsection{Jacobi Symbol}
The Jacobi symbol is a generalization of the Legendre function for any odd non prime moduli $p$ greater than 2.  If $p = \prod_{i=0}^n p_i$ then
the Jacobi symbol $\left ( { a \over p } \right )$ is equal to the following equation.

\begin{equation}
\left ( { a \over p } \right ) = \left ( { a \over p_0} \right ) \left ( { a \over p_1} \right ) \ldots \left ( { a \over p_n} \right )
\end{equation}

By inspection if $p$ is prime the Jacobi symbol is equivalent to the Legendre function.  The following facts\footnote{See HAC \cite[pp. 72-74]{HAC} for
further details.} will be used to derive an efficient Jacobi symbol algorithm.  Where $p$ is an odd integer greater than two and $a, b \in \Z$ the
following are true.  

\begin{enumerate}
\item $\left ( { a \over p} \right )$ equals $-1$, $0$ or $1$. 
\item $\left ( { ab \over p} \right ) = \left ( { a \over p} \right )\left ( { b \over p} \right )$.
\item If $a \equiv b$ then $\left ( { a \over p} \right ) = \left ( { b \over p} \right )$.
\item $\left ( { 2 \over p} \right )$ equals $1$ if $p \equiv 1$ or $7 \mbox{ (mod }8\mbox{)}$.  Otherwise, it equals $-1$.
\item $\left ( { a \over p} \right ) \equiv \left ( { p \over a} \right ) \cdot (-1)^{(p-1)(a-1)/4}$.  More specifically 
$\left ( { a \over p} \right ) = \left ( { p \over a} \right )$ if $p \equiv a \equiv 1 \mbox{ (mod }4\mbox{)}$.  
\end{enumerate}

Using these facts if $a = 2^k \cdot a'$ then

\begin{eqnarray}
\left ( { a \over p } \right ) = \left ( {{2^k} \over p } \right ) \left ( {a' \over p} \right ) \nonumber \\
                               = \left ( {2 \over p } \right )^k \left ( {a' \over p} \right ) 
\label{eqn:jacobi}
\end{eqnarray}

By fact five, 

\begin{equation}
\left ( { a \over p } \right ) = \left ( { p \over a } \right ) \cdot (-1)^{(p-1)(a-1)/4} 
\end{equation}

Subsequently by fact three since $p \equiv (p \mbox{ mod }a) \mbox{ (mod }a\mbox{)}$ then 

\begin{equation}
\left ( { a \over p } \right ) = \left ( { {p \mbox{ mod } a} \over a } \right ) \cdot (-1)^{(p-1)(a-1)/4} 
\end{equation}

By putting both observations into equation \ref{eqn:jacobi} the following simplified equation is formed.

\begin{equation}
\left ( { a \over p } \right ) = \left ( {2 \over p } \right )^k \left ( {{p\mbox{ mod }a'} \over a'} \right )  \cdot (-1)^{(p-1)(a'-1)/4} 
\end{equation}

The value of $\left ( {{p \mbox{ mod }a'} \over a'} \right )$ can be found by using the same equation recursively.  The value of 
$\left ( {2 \over p } \right )^k$ equals $1$ if $k$ is even otherwise it equals $\left ( {2 \over p } \right )$.  Using this approach the 
factors of $p$ do not have to be known.  Furthermore, if $(a, p) = 1$ then the algorithm will terminate when the recursion requests the 
Jacobi symbol computation of $\left ( {1 \over a'} \right )$ which is simply $1$.  

\newpage\begin{figure}[!here]
\begin{small}
\begin{center}
\begin{tabular}{l}
\hline Algorithm \textbf{mp\_jacobi}. \\
\textbf{Input}.   mp\_int $a$ and $p$, $a \ge 0$, $p \ge 3$, $p \equiv 1 \mbox{ (mod }2\mbox{)}$ \\
\textbf{Output}.  The Jacobi symbol $c = \left ( {a \over p } \right )$. \\
\hline \\
1.  If $a = 0$ then \\
\hspace{3mm}1.1  $c \leftarrow 0$ \\
\hspace{3mm}1.2  Return(\textit{MP\_OKAY}). \\
2.  If $a = 1$ then \\
\hspace{3mm}2.1  $c \leftarrow 1$ \\
\hspace{3mm}2.2  Return(\textit{MP\_OKAY}). \\
3.  $a' \leftarrow a$ \\
4.  $k \leftarrow 0$ \\
5.  While $a'.used > 0$ and $a'_0 \equiv 0 \mbox{ (mod }2\mbox{)}$ \\
\hspace{3mm}5.1  $k \leftarrow k + 1$ \\
\hspace{3mm}5.2  $a' \leftarrow \lfloor a' / 2 \rfloor$ \\
6.  If $k \equiv 0 \mbox{ (mod }2\mbox{)}$ then \\
\hspace{3mm}6.1  $s \leftarrow 1$ \\
7.  else \\
\hspace{3mm}7.1  $r \leftarrow p_0 \mbox{ (mod }8\mbox{)}$ \\
\hspace{3mm}7.2  If $r = 1$ or $r = 7$ then \\
\hspace{6mm}7.2.1  $s \leftarrow 1$ \\
\hspace{3mm}7.3  else \\
\hspace{6mm}7.3.1  $s \leftarrow -1$ \\
8.  If $p_0 \equiv a'_0 \equiv 3 \mbox{ (mod }4\mbox{)}$ then \\
\hspace{3mm}8.1  $s \leftarrow -s$ \\
9.  If $a' \ne 1$ then \\
\hspace{3mm}9.1  $p' \leftarrow p \mbox{ (mod }a'\mbox{)}$ \\
\hspace{3mm}9.2  $s \leftarrow s \cdot \mbox{mp\_jacobi}(p', a')$ \\
10.  $c \leftarrow s$ \\
11.  Return(\textit{MP\_OKAY}). \\
\hline
\end{tabular}
\end{center}
\end{small}
\caption{Algorithm mp\_jacobi}
\end{figure}
\textbf{Algorithm mp\_jacobi.}
This algorithm computes the Jacobi symbol for an arbitrary positive integer $a$ with respect to an odd integer $p$ greater than three.  The algorithm
is based on algorithm 2.149 of HAC \cite[pp. 73]{HAC}.  

Step numbers one and two handle the trivial cases of $a = 0$ and $a = 1$ respectively.  Step five determines the number of two factors in the
input $a$.  If $k$ is even than the term $\left ( { 2 \over p } \right )^k$ must always evaluate to one.  If $k$ is odd than the term evaluates to one 
if $p_0$ is congruent to one or seven modulo eight, otherwise it evaluates to $-1$. After the the $\left ( { 2 \over p } \right )^k$ term is handled 
the $(-1)^{(p-1)(a'-1)/4}$ is computed and multiplied against the current product $s$.  The latter term evaluates to one if both $p$ and $a'$ 
are congruent to one modulo four, otherwise it evaluates to negative one.

By step nine if $a'$ does not equal one a recursion is required.  Step 9.1 computes $p' \equiv p \mbox{ (mod }a'\mbox{)}$ and will recurse to compute
$\left ( {p' \over a'} \right )$ which is multiplied against the current Jacobi product.

\vspace{+3mm}\begin{small}
\hspace{-5.1mm}{\bf File}: bn\_mp\_jacobi.c
\vspace{-3mm}
\begin{alltt}
\end{alltt}
\end{small}

As a matter of practicality the variable $a'$ as per the pseudo-code is reprensented by the variable $a1$ since the $'$ symbol is not valid for a C 
variable name character. 

The two simple cases of $a = 0$ and $a = 1$ are handled at the very beginning to simplify the algorithm.  If the input is non-trivial the algorithm
has to proceed compute the Jacobi.  The variable $s$ is used to hold the current Jacobi product.  Note that $s$ is merely a C ``int'' data type since
the values it may obtain are merely $-1$, $0$ and $1$.  

After a local copy of $a$ is made all of the factors of two are divided out and the total stored in $k$.  Technically only the least significant
bit of $k$ is required, however, it makes the algorithm simpler to follow to perform an addition. In practice an exclusive-or and addition have the same 
processor requirements and neither is faster than the other.

Line 58 through 71 determines the value of $\left ( { 2 \over p } \right )^k$.  If the least significant bit of $k$ is zero than
$k$ is even and the value is one.  Otherwise, the value of $s$ depends on which residue class $p$ belongs to modulo eight.  The value of
$(-1)^{(p-1)(a'-1)/4}$ is compute and multiplied against $s$ on lines 71 through 74.  

Finally, if $a1$ does not equal one the algorithm must recurse and compute $\left ( {p' \over a'} \right )$.  

\textit{-- Comment about default $s$ and such...}

\section{Modular Inverse}
\label{sec:modinv}
The modular inverse of a number actually refers to the modular multiplicative inverse.  Essentially for any integer $a$ such that $(a, p) = 1$ there
exist another integer $b$ such that $ab \equiv 1 \mbox{ (mod }p\mbox{)}$.  The integer $b$ is called the multiplicative inverse of $a$ which is
denoted as $b = a^{-1}$.  Technically speaking modular inversion is a well defined operation for any finite ring or field not just for rings and 
fields of integers.  However, the former will be the matter of discussion.

The simplest approach is to compute the algebraic inverse of the input.  That is to compute $b \equiv a^{\Phi(p) - 1}$.  If $\Phi(p)$ is the 
order of the multiplicative subgroup modulo $p$ then $b$ must be the multiplicative inverse of $a$.  The proof of which is trivial.

\begin{equation}
ab \equiv a \left (a^{\Phi(p) - 1} \right ) \equiv a^{\Phi(p)} \equiv a^0 \equiv 1 \mbox{ (mod }p\mbox{)}
\end{equation}

However, as simple as this approach may be it has two serious flaws.  It requires that the value of $\Phi(p)$ be known which if $p$ is composite 
requires all of the prime factors.  This approach also is very slow as the size of $p$ grows.  

A simpler approach is based on the observation that solving for the multiplicative inverse is equivalent to solving the linear 
Diophantine\footnote{See LeVeque \cite[pp. 40-43]{LeVeque} for more information.} equation.

\begin{equation}
ab + pq = 1
\end{equation}

Where $a$, $b$, $p$ and $q$ are all integers.  If such a pair of integers $ \left < b, q \right >$ exist than $b$ is the multiplicative inverse of 
$a$ modulo $p$.  The extended Euclidean algorithm (Knuth \cite[pp. 342]{TAOCPV2}) can be used to solve such equations provided $(a, p) = 1$.  
However, instead of using that algorithm directly a variant known as the binary Extended Euclidean algorithm will be used in its place.  The
binary approach is very similar to the binary greatest common divisor algorithm except it will produce a full solution to the Diophantine 
equation.  

\subsection{General Case}
\newpage\begin{figure}[!here]
\begin{small}
\begin{center}
\begin{tabular}{l}
\hline Algorithm \textbf{mp\_invmod}. \\
\textbf{Input}.   mp\_int $a$ and $b$, $(a, b) = 1$, $p \ge 2$, $0 < a < p$.  \\
\textbf{Output}.  The modular inverse $c \equiv a^{-1} \mbox{ (mod }b\mbox{)}$. \\
\hline \\
1.  If $b \le 0$ then return(\textit{MP\_VAL}). \\
2.  If $b_0 \equiv 1 \mbox{ (mod }2\mbox{)}$ then use algorithm fast\_mp\_invmod. \\
3.  $x \leftarrow \vert a \vert, y \leftarrow b$ \\
4.  If $x_0 \equiv y_0  \equiv 0 \mbox{ (mod }2\mbox{)}$ then return(\textit{MP\_VAL}). \\
5.  $B \leftarrow 0, C \leftarrow 0, A \leftarrow 1, D \leftarrow 1$ \\
6.  While $u.used > 0$ and $u_0 \equiv 0 \mbox{ (mod }2\mbox{)}$ \\
\hspace{3mm}6.1  $u \leftarrow \lfloor u / 2 \rfloor$ \\
\hspace{3mm}6.2  If ($A.used > 0$ and $A_0 \equiv 1 \mbox{ (mod }2\mbox{)}$) or ($B.used > 0$ and $B_0 \equiv 1 \mbox{ (mod }2\mbox{)}$) then \\
\hspace{6mm}6.2.1  $A \leftarrow A + y$ \\
\hspace{6mm}6.2.2  $B \leftarrow B - x$ \\
\hspace{3mm}6.3  $A \leftarrow \lfloor A / 2 \rfloor$ \\
\hspace{3mm}6.4  $B \leftarrow \lfloor B / 2 \rfloor$ \\
7.  While $v.used > 0$ and $v_0 \equiv 0 \mbox{ (mod }2\mbox{)}$ \\
\hspace{3mm}7.1  $v \leftarrow \lfloor v / 2 \rfloor$ \\
\hspace{3mm}7.2  If ($C.used > 0$ and $C_0 \equiv 1 \mbox{ (mod }2\mbox{)}$) or ($D.used > 0$ and $D_0 \equiv 1 \mbox{ (mod }2\mbox{)}$) then \\
\hspace{6mm}7.2.1  $C \leftarrow C + y$ \\
\hspace{6mm}7.2.2  $D \leftarrow D - x$ \\
\hspace{3mm}7.3  $C \leftarrow \lfloor C / 2 \rfloor$ \\
\hspace{3mm}7.4  $D \leftarrow \lfloor D / 2 \rfloor$ \\
8.  If $u \ge v$ then \\
\hspace{3mm}8.1  $u \leftarrow u - v$ \\
\hspace{3mm}8.2  $A \leftarrow A - C$ \\
\hspace{3mm}8.3  $B \leftarrow B - D$ \\
9.  else \\
\hspace{3mm}9.1  $v \leftarrow v - u$ \\
\hspace{3mm}9.2  $C \leftarrow C - A$ \\
\hspace{3mm}9.3  $D \leftarrow D - B$ \\
10.  If $u \ne 0$ goto step 6. \\
11.  If $v \ne 1$ return(\textit{MP\_VAL}). \\
12.  While $C \le 0$ do \\
\hspace{3mm}12.1  $C \leftarrow C + b$ \\
13.  While $C \ge b$ do \\
\hspace{3mm}13.1  $C \leftarrow C - b$ \\
14.  $c \leftarrow C$ \\
15.  Return(\textit{MP\_OKAY}). \\
\hline
\end{tabular}
\end{center}
\end{small}
\end{figure}
\textbf{Algorithm mp\_invmod.}
This algorithm computes the modular multiplicative inverse of an integer $a$ modulo an integer $b$.  This algorithm is a variation of the 
extended binary Euclidean algorithm from HAC \cite[pp. 608]{HAC}.  It has been modified to only compute the modular inverse and not a complete
Diophantine solution.  

If $b \le 0$ than the modulus is invalid and MP\_VAL is returned.  Similarly if both $a$ and $b$ are even then there cannot be a multiplicative
inverse for $a$ and the error is reported.  

The astute reader will observe that steps seven through nine are very similar to the binary greatest common divisor algorithm mp\_gcd.  In this case
the other variables to the Diophantine equation are solved.  The algorithm terminates when $u = 0$ in which case the solution is

\begin{equation}
Ca + Db = v
\end{equation}

If $v$, the greatest common divisor of $a$ and $b$ is not equal to one then the algorithm will report an error as no inverse exists.  Otherwise, $C$
is the modular inverse of $a$.  The actual value of $C$ is congruent to, but not necessarily equal to, the ideal modular inverse which should lie 
within $1 \le a^{-1} < b$.  Step numbers twelve and thirteen adjust the inverse until it is in range.  If the original input $a$ is within $0 < a < p$ 
then only a couple of additions or subtractions will be required to adjust the inverse.

\vspace{+3mm}\begin{small}
\hspace{-5.1mm}{\bf File}: bn\_mp\_invmod.c
\vspace{-3mm}
\begin{alltt}
\end{alltt}
\end{small}

\subsubsection{Odd Moduli}

When the modulus $b$ is odd the variables $A$ and $C$ are fixed and are not required to compute the inverse.  In particular by attempting to solve
the Diophantine $Cb + Da = 1$ only $B$ and $D$ are required to find the inverse of $a$.  

The algorithm fast\_mp\_invmod is a direct adaptation of algorithm mp\_invmod with all all steps involving either $A$ or $C$ removed.  This 
optimization will halve the time required to compute the modular inverse.

\section{Primality Tests}

A non-zero integer $a$ is said to be prime if it is not divisible by any other integer excluding one and itself.  For example, $a = 7$ is prime 
since the integers $2 \ldots 6$ do not evenly divide $a$.  By contrast, $a = 6$ is not prime since $a = 6 = 2 \cdot 3$. 

Prime numbers arise in cryptography considerably as they allow finite fields to be formed.  The ability to determine whether an integer is prime or
not quickly has been a viable subject in cryptography and number theory for considerable time.  The algorithms that will be presented are all
probablistic algorithms in that when they report an integer is composite it must be composite.  However, when the algorithms report an integer is
prime the algorithm may be incorrect.  

As will be discussed it is possible to limit the probability of error so well that for practical purposes the probablity of error might as 
well be zero.  For the purposes of these discussions let $n$ represent the candidate integer of which the primality is in question.

\subsection{Trial Division}

Trial division means to attempt to evenly divide a candidate integer by small prime integers.  If the candidate can be evenly divided it obviously
cannot be prime.  By dividing by all primes $1 < p \le \sqrt{n}$ this test can actually prove whether an integer is prime.  However, such a test
would require a prohibitive amount of time as $n$ grows.

Instead of dividing by every prime, a smaller, more mangeable set of primes may be used instead.  By performing trial division with only a subset
of the primes less than $\sqrt{n} + 1$ the algorithm cannot prove if a candidate is prime.  However, often it can prove a candidate is not prime.

The benefit of this test is that trial division by small values is fairly efficient.  Specially compared to the other algorithms that will be
discussed shortly.  The probability that this approach correctly identifies a composite candidate when tested with all primes upto $q$ is given by
$1 - {1.12 \over ln(q)}$.  The graph (\ref{pic:primality}, will be added later) demonstrates the probability of success for the range 
$3 \le q \le 100$.  

At approximately $q = 30$ the gain of performing further tests diminishes fairly quickly.  At $q = 90$ further testing is generally not going to 
be of any practical use.  In the case of LibTomMath the default limit $q = 256$ was chosen since it is not too high and will eliminate 
approximately $80\%$ of all candidate integers.  The constant \textbf{PRIME\_SIZE} is equal to the number of primes in the test base.  The 
array \_\_prime\_tab is an array of the first \textbf{PRIME\_SIZE} prime numbers.  

\begin{figure}[!here]
\begin{small}
\begin{center}
\begin{tabular}{l}
\hline Algorithm \textbf{mp\_prime\_is\_divisible}. \\
\textbf{Input}.   mp\_int $a$ \\
\textbf{Output}.  $c = 1$ if $n$ is divisible by a small prime, otherwise $c = 0$.  \\
\hline \\
1.  for $ix$ from $0$ to $PRIME\_SIZE$ do \\
\hspace{3mm}1.1  $d \leftarrow n \mbox{ (mod }\_\_prime\_tab_{ix}\mbox{)}$ \\
\hspace{3mm}1.2  If $d = 0$ then \\
\hspace{6mm}1.2.1  $c \leftarrow 1$ \\
\hspace{6mm}1.2.2  Return(\textit{MP\_OKAY}). \\
2.  $c \leftarrow 0$ \\
3.  Return(\textit{MP\_OKAY}). \\
\hline
\end{tabular}
\end{center}
\end{small}
\caption{Algorithm mp\_prime\_is\_divisible}
\end{figure}
\textbf{Algorithm mp\_prime\_is\_divisible.}
This algorithm attempts to determine if a candidate integer $n$ is composite by performing trial divisions.  

\vspace{+3mm}\begin{small}
\hspace{-5.1mm}{\bf File}: bn\_mp\_prime\_is\_divisible.c
\vspace{-3mm}
\begin{alltt}
\end{alltt}
\end{small}

The algorithm defaults to a return of $0$ in case an error occurs.  The values in the prime table are all specified to be in the range of a 
mp\_digit.  The table \_\_prime\_tab is defined in the following file.

\vspace{+3mm}\begin{small}
\hspace{-5.1mm}{\bf File}: bn\_prime\_tab.c
\vspace{-3mm}
\begin{alltt}
\end{alltt}
\end{small}

Note that there are two possible tables.  When an mp\_digit is 7-bits long only the primes upto $127$ may be included, otherwise the primes
upto $1619$ are used.  Note that the value of \textbf{PRIME\_SIZE} is a constant dependent on the size of a mp\_digit. 

\subsection{The Fermat Test}
The Fermat test is probably one the oldest tests to have a non-trivial probability of success.  It is based on the fact that if $n$ is in 
fact prime then $a^{n} \equiv a \mbox{ (mod }n\mbox{)}$ for all $0 < a < n$.  The reason being that if $n$ is prime than the order of
the multiplicative sub group is $n - 1$.  Any base $a$ must have an order which divides $n - 1$ and as such $a^n$ is equivalent to 
$a^1 = a$.  

If $n$ is composite then any given base $a$ does not have to have a period which divides $n - 1$.  In which case 
it is possible that $a^n \nequiv a \mbox{ (mod }n\mbox{)}$.  However, this test is not absolute as it is possible that the order
of a base will divide $n - 1$ which would then be reported as prime.  Such a base yields what is known as a Fermat pseudo-prime.  Several 
integers known as Carmichael numbers will be a pseudo-prime to all valid bases.  Fortunately such numbers are extremely rare as $n$ grows
in size.

\begin{figure}[!here]
\begin{small}
\begin{center}
\begin{tabular}{l}
\hline Algorithm \textbf{mp\_prime\_fermat}. \\
\textbf{Input}.   mp\_int $a$ and $b$, $a \ge 2$, $0 < b < a$.  \\
\textbf{Output}.  $c = 1$ if $b^a \equiv b \mbox{ (mod }a\mbox{)}$, otherwise $c = 0$.  \\
\hline \\
1.  $t \leftarrow b^a \mbox{ (mod }a\mbox{)}$ \\
2.  If $t = b$ then \\
\hspace{3mm}2.1  $c = 1$ \\
3.  else \\
\hspace{3mm}3.1  $c = 0$ \\
4.  Return(\textit{MP\_OKAY}). \\
\hline
\end{tabular}
\end{center}
\end{small}
\caption{Algorithm mp\_prime\_fermat}
\end{figure}
\textbf{Algorithm mp\_prime\_fermat.}
This algorithm determines whether an mp\_int $a$ is a Fermat prime to the base $b$ or not.  It uses a single modular exponentiation to
determine the result.  

\vspace{+3mm}\begin{small}
\hspace{-5.1mm}{\bf File}: bn\_mp\_prime\_fermat.c
\vspace{-3mm}
\begin{alltt}
\end{alltt}
\end{small}

\subsection{The Miller-Rabin Test}
The Miller-Rabin (citation) test is another primality test which has tighter error bounds than the Fermat test specifically with sequentially chosen 
candidate  integers.  The algorithm is based on the observation that if $n - 1 = 2^kr$ and if $b^r \nequiv \pm 1$ then after upto $k - 1$ squarings the 
value must be equal to $-1$.  The squarings are stopped as soon as $-1$ is observed.  If the value of $1$ is observed first it means that
some value not congruent to $\pm 1$ when squared equals one which cannot occur if $n$ is prime.

\begin{figure}[!here]
\begin{small}
\begin{center}
\begin{tabular}{l}
\hline Algorithm \textbf{mp\_prime\_miller\_rabin}. \\
\textbf{Input}.   mp\_int $a$ and $b$, $a \ge 2$, $0 < b < a$.  \\
\textbf{Output}.  $c = 1$ if $a$ is a Miller-Rabin prime to the base $a$, otherwise $c = 0$.  \\
\hline
1.  $a' \leftarrow a - 1$ \\
2.  $r  \leftarrow n1$    \\
3.  $c \leftarrow 0, s  \leftarrow 0$ \\
4.  While $r.used > 0$ and $r_0 \equiv 0 \mbox{ (mod }2\mbox{)}$ \\
\hspace{3mm}4.1  $s \leftarrow s + 1$ \\
\hspace{3mm}4.2  $r \leftarrow \lfloor r / 2 \rfloor$ \\
5.  $y \leftarrow b^r \mbox{ (mod }a\mbox{)}$ \\
6.  If $y \nequiv \pm 1$ then \\
\hspace{3mm}6.1  $j \leftarrow 1$ \\
\hspace{3mm}6.2  While $j \le (s - 1)$ and $y \nequiv a'$ \\
\hspace{6mm}6.2.1  $y \leftarrow y^2 \mbox{ (mod }a\mbox{)}$ \\
\hspace{6mm}6.2.2  If $y = 1$ then goto step 8. \\
\hspace{6mm}6.2.3  $j \leftarrow j + 1$ \\
\hspace{3mm}6.3  If $y \nequiv a'$ goto step 8. \\
7.  $c \leftarrow 1$\\
8.  Return(\textit{MP\_OKAY}). \\
\hline
\end{tabular}
\end{center}
\end{small}
\caption{Algorithm mp\_prime\_miller\_rabin}
\end{figure}
\textbf{Algorithm mp\_prime\_miller\_rabin.}
This algorithm performs one trial round of the Miller-Rabin algorithm to the base $b$.  It will set $c = 1$ if the algorithm cannot determine
if $b$ is composite or $c = 0$ if $b$ is provably composite.  The values of $s$ and $r$ are computed such that $a' = a - 1 = 2^sr$.  

If the value $y \equiv b^r$ is congruent to $\pm 1$ then the algorithm cannot prove if $a$ is composite or not.  Otherwise, the algorithm will
square $y$ upto $s - 1$ times stopping only when $y \equiv -1$.  If $y^2 \equiv 1$ and $y \nequiv \pm 1$ then the algorithm can report that $a$
is provably composite.  If the algorithm performs $s - 1$ squarings and $y \nequiv -1$ then $a$ is provably composite.  If $a$ is not provably 
composite then it is \textit{probably} prime.

\vspace{+3mm}\begin{small}
\hspace{-5.1mm}{\bf File}: bn\_mp\_prime\_miller\_rabin.c
\vspace{-3mm}
\begin{alltt}
\end{alltt}
\end{small}




\backmatter
\appendix
\begin{thebibliography}{ABCDEF}
\bibitem[1]{TAOCPV2}
Donald Knuth, \textit{The Art of Computer Programming}, Third Edition, Volume Two, Seminumerical Algorithms, Addison-Wesley, 1998

\bibitem[2]{HAC}
A. Menezes, P. van Oorschot, S. Vanstone, \textit{Handbook of Applied Cryptography}, CRC Press, 1996

\bibitem[3]{ROSE}
Michael Rosing, \textit{Implementing Elliptic Curve Cryptography}, Manning Publications, 1999

\bibitem[4]{COMBA}
Paul G. Comba, \textit{Exponentiation Cryptosystems on the IBM PC}. IBM Systems Journal 29(4): 526-538 (1990)

\bibitem[5]{KARA}
A. Karatsuba, Doklay Akad. Nauk SSSR 145 (1962), pp.293-294

\bibitem[6]{KARAP}
Andre Weimerskirch and Christof Paar, \textit{Generalizations of the Karatsuba Algorithm for Polynomial Multiplication}, Submitted to Design, Codes and Cryptography, March 2002

\bibitem[7]{BARRETT}
Paul Barrett, \textit{Implementing the Rivest Shamir and Adleman Public Key Encryption Algorithm on a Standard Digital Signal Processor}, Advances in Cryptology, Crypto '86, Springer-Verlag.

\bibitem[8]{MONT}
P.L.Montgomery. \textit{Modular multiplication without trial division}. Mathematics of Computation, 44(170):519-521, April 1985.

\bibitem[9]{DRMET}
Chae Hoon Lim and Pil Joong Lee, \textit{Generating Efficient Primes for Discrete Log Cryptosystems}, POSTECH Information Research Laboratories

\bibitem[10]{MMB}
J. Daemen and R. Govaerts and J. Vandewalle, \textit{Block ciphers based on Modular Arithmetic}, State and {P}rogress in the {R}esearch of {C}ryptography, 1993, pp. 80-89

\bibitem[11]{RSAREF}
R.L. Rivest, A. Shamir, L. Adleman, \textit{A Method for Obtaining Digital Signatures and Public-Key Cryptosystems}

\bibitem[12]{DHREF}
Whitfield Diffie, Martin E. Hellman, \textit{New Directions in Cryptography}, IEEE Transactions on Information Theory, 1976

\bibitem[13]{IEEE}
IEEE Standard for Binary Floating-Point Arithmetic (ANSI/IEEE Std 754-1985)

\bibitem[14]{GMP}
GNU Multiple Precision (GMP), \url{http://www.swox.com/gmp/}

\bibitem[15]{MPI}
Multiple Precision Integer Library (MPI), Michael Fromberger, \url{http://thayer.dartmouth.edu/~sting/mpi/}

\bibitem[16]{OPENSSL}
OpenSSL Cryptographic Toolkit, \url{http://openssl.org}

\bibitem[17]{LIP}
Large Integer Package, \url{http://home.hetnet.nl/~ecstr/LIP.zip}

\bibitem[18]{ISOC}
JTC1/SC22/WG14, ISO/IEC 9899:1999, ``A draft rationale for the C99 standard.''

\bibitem[19]{JAVA}
The Sun Java Website, \url{http://java.sun.com/}

\end{thebibliography}

\documentclass[b5paper]{book}
\usepackage{hyperref}
\usepackage{makeidx}
\usepackage{amssymb}
\usepackage{color}
\usepackage{alltt}
\usepackage{graphicx}
\usepackage{layout}
\def\union{\cup}
\def\intersect{\cap}
\def\getsrandom{\stackrel{\rm R}{\gets}}
\def\cross{\times}
\def\cat{\hspace{0.5em} \| \hspace{0.5em}}
\def\catn{$\|$}
\def\divides{\hspace{0.3em} | \hspace{0.3em}}
\def\nequiv{\not\equiv}
\def\approx{\raisebox{0.2ex}{\mbox{\small $\sim$}}}
\def\lcm{{\rm lcm}}
\def\gcd{{\rm gcd}}
\def\log{{\rm log}}
\def\ord{{\rm ord}}
\def\abs{{\mathit abs}}
\def\rep{{\mathit rep}}
\def\mod{{\mathit\ mod\ }}
\renewcommand{\pmod}[1]{\ ({\rm mod\ }{#1})}
\newcommand{\floor}[1]{\left\lfloor{#1}\right\rfloor}
\newcommand{\ceil}[1]{\left\lceil{#1}\right\rceil}
\def\Or{{\rm\ or\ }}
\def\And{{\rm\ and\ }}
\def\iff{\hspace{1em}\Longleftrightarrow\hspace{1em}}
\def\implies{\Rightarrow}
\def\undefined{{\rm ``undefined"}}
\def\Proof{\vspace{1ex}\noindent {\bf Proof:}\hspace{1em}}
\let\oldphi\phi
\def\phi{\varphi}
\def\Pr{{\rm Pr}}
\newcommand{\str}[1]{{\mathbf{#1}}}
\def\F{{\mathbb F}}
\def\N{{\mathbb N}}
\def\Z{{\mathbb Z}}
\def\R{{\mathbb R}}
\def\C{{\mathbb C}}
\def\Q{{\mathbb Q}}
\definecolor{DGray}{gray}{0.5}
\newcommand{\emailaddr}[1]{\mbox{$<${#1}$>$}}
\def\twiddle{\raisebox{0.3ex}{\mbox{\tiny $\sim$}}}
\def\gap{\vspace{0.5ex}}
\makeindex
\begin{document}
\frontmatter
\pagestyle{empty}
\title{Multi--Precision Math}
\author{\mbox{
%\begin{small}
\begin{tabular}{c}
Tom St Denis \\
Algonquin College \\
\\
Mads Rasmussen \\
Open Communications Security \\
\\
Greg Rose \\
QUALCOMM Australia \\
\end{tabular}
%\end{small}
}
}
\maketitle
This text has been placed in the public domain.  This text corresponds to the v0.39 release of the 
LibTomMath project.

\begin{alltt}
Tom St Denis
111 Banning Rd
Ottawa, Ontario
K2L 1C3
Canada

Phone: 1-613-836-3160
Email: tomstdenis@gmail.com
\end{alltt}

This text is formatted to the international B5 paper size of 176mm wide by 250mm tall using the \LaTeX{} 
{\em book} macro package and the Perl {\em booker} package.

\tableofcontents
\listoffigures
\chapter*{Prefaces}
When I tell people about my LibTom projects and that I release them as public domain they are often puzzled.  
They ask why I did it and especially why I continue to work on them for free.  The best I can explain it is ``Because I can.''  
Which seems odd and perhaps too terse for adult conversation. I often qualify it with ``I am able, I am willing.'' which 
perhaps explains it better.  I am the first to admit there is not anything that special with what I have done.  Perhaps
others can see that too and then we would have a society to be proud of.  My LibTom projects are what I am doing to give 
back to society in the form of tools and knowledge that can help others in their endeavours.

I started writing this book because it was the most logical task to further my goal of open academia.  The LibTomMath source
code itself was written to be easy to follow and learn from.  There are times, however, where pure C source code does not
explain the algorithms properly.  Hence this book.  The book literally starts with the foundation of the library and works
itself outwards to the more complicated algorithms.  The use of both pseudo--code and verbatim source code provides a duality
of ``theory'' and ``practice'' that the computer science students of the world shall appreciate.  I never deviate too far
from relatively straightforward algebra and I hope that this book can be a valuable learning asset.

This book and indeed much of the LibTom projects would not exist in their current form if it was not for a plethora
of kind people donating their time, resources and kind words to help support my work.  Writing a text of significant
length (along with the source code) is a tiresome and lengthy process.  Currently the LibTom project is four years old,
comprises of literally thousands of users and over 100,000 lines of source code, TeX and other material.  People like Mads and Greg 
were there at the beginning to encourage me to work well.  It is amazing how timely validation from others can boost morale to 
continue the project. Definitely my parents were there for me by providing room and board during the many months of work in 2003.  

To my many friends whom I have met through the years I thank you for the good times and the words of encouragement.  I hope I
honour your kind gestures with this project.

Open Source.  Open Academia.  Open Minds.

\begin{flushright} Tom St Denis \end{flushright}

\newpage
I found the opportunity to work with Tom appealing for several reasons, not only could I broaden my own horizons, but also 
contribute to educate others facing the problem of having to handle big number mathematical calculations.

This book is Tom's child and he has been caring and fostering the project ever since the beginning with a clear mind of 
how he wanted the project to turn out. I have helped by proofreading the text and we have had several discussions about 
the layout and language used.

I hold a masters degree in cryptography from the University of Southern Denmark and have always been interested in the 
practical aspects of cryptography. 

Having worked in the security consultancy business for several years in S\~{a}o Paulo, Brazil, I have been in touch with a 
great deal of work in which multiple precision mathematics was needed. Understanding the possibilities for speeding up 
multiple precision calculations is often very important since we deal with outdated machine architecture where modular 
reductions, for example, become painfully slow.

This text is for people who stop and wonder when first examining algorithms such as RSA for the first time and asks 
themselves, ``You tell me this is only secure for large numbers, fine; but how do you implement these numbers?''

\begin{flushright}
Mads Rasmussen

S\~{a}o Paulo - SP

Brazil
\end{flushright}

\newpage
It's all because I broke my leg. That just happened to be at about the same time that Tom asked for someone to review the section of the book about 
Karatsuba multiplication. I was laid up, alone and immobile, and thought ``Why not?'' I vaguely knew what Karatsuba multiplication was, but not 
really, so I thought I could help, learn, and stop myself from watching daytime cable TV, all at once.

At the time of writing this, I've still not met Tom or Mads in meatspace. I've been following Tom's progress since his first splash on the 
sci.crypt Usenet news group. I watched him go from a clueless newbie, to the cryptographic equivalent of a reformed smoker, to a real
contributor to the field, over a period of about two years. I've been impressed with his obvious intelligence, and astounded by his productivity. 
Of course, he's young enough to be my own child, so he doesn't have my problems with staying awake.

When I reviewed that single section of the book, in its very earliest form, I was very pleasantly surprised. So I decided to collaborate more fully, 
and at least review all of it, and perhaps write some bits too. There's still a long way to go with it, and I have watched a number of close 
friends go through the mill of publication, so I think that the way to go is longer than Tom thinks it is. Nevertheless, it's a good effort, 
and I'm pleased to be involved with it.

\begin{flushright}
Greg Rose, Sydney, Australia, June 2003. 
\end{flushright}

\mainmatter
\pagestyle{headings}
\chapter{Introduction}
\section{Multiple Precision Arithmetic}

\subsection{What is Multiple Precision Arithmetic?}
When we think of long-hand arithmetic such as addition or multiplication we rarely consider the fact that we instinctively
raise or lower the precision of the numbers we are dealing with.  For example, in decimal we almost immediate can 
reason that $7$ times $6$ is $42$.  However, $42$ has two digits of precision as opposed to one digit we started with.  
Further multiplications of say $3$ result in a larger precision result $126$.  In these few examples we have multiple 
precisions for the numbers we are working with.  Despite the various levels of precision a single subset\footnote{With the occasional optimization.}
 of algorithms can be designed to accomodate them.  

By way of comparison a fixed or single precision operation would lose precision on various operations.  For example, in
the decimal system with fixed precision $6 \cdot 7 = 2$.

Essentially at the heart of computer based multiple precision arithmetic are the same long-hand algorithms taught in
schools to manually add, subtract, multiply and divide.  

\subsection{The Need for Multiple Precision Arithmetic}
The most prevalent need for multiple precision arithmetic, often referred to as ``bignum'' math, is within the implementation
of public-key cryptography algorithms.   Algorithms such as RSA \cite{RSAREF} and Diffie-Hellman \cite{DHREF} require 
integers of significant magnitude to resist known cryptanalytic attacks.  For example, at the time of this writing a 
typical RSA modulus would be at least greater than $10^{309}$.  However, modern programming languages such as ISO C \cite{ISOC} and 
Java \cite{JAVA} only provide instrinsic support for integers which are relatively small and single precision.

\begin{figure}[!here]
\begin{center}
\begin{tabular}{|r|c|}
\hline \textbf{Data Type} & \textbf{Range} \\
\hline char  & $-128 \ldots 127$ \\
\hline short & $-32768 \ldots 32767$ \\
\hline long  & $-2147483648 \ldots 2147483647$ \\
\hline long long & $-9223372036854775808 \ldots 9223372036854775807$ \\
\hline
\end{tabular}
\end{center}
\caption{Typical Data Types for the C Programming Language}
\label{fig:ISOC}
\end{figure}

The largest data type guaranteed to be provided by the ISO C programming 
language\footnote{As per the ISO C standard.  However, each compiler vendor is allowed to augment the precision as they 
see fit.}  can only represent values up to $10^{19}$ as shown in figure \ref{fig:ISOC}. On its own the C language is 
insufficient to accomodate the magnitude required for the problem at hand.  An RSA modulus of magnitude $10^{19}$ could be 
trivially factored\footnote{A Pollard-Rho factoring would take only $2^{16}$ time.} on the average desktop computer, 
rendering any protocol based on the algorithm insecure.  Multiple precision algorithms solve this very problem by 
extending the range of representable integers while using single precision data types.

Most advancements in fast multiple precision arithmetic stem from the need for faster and more efficient cryptographic 
primitives.  Faster modular reduction and exponentiation algorithms such as Barrett's algorithm, which have appeared in 
various cryptographic journals, can render algorithms such as RSA and Diffie-Hellman more efficient.  In fact, several 
major companies such as RSA Security, Certicom and Entrust have built entire product lines on the implementation and 
deployment of efficient algorithms.

However, cryptography is not the only field of study that can benefit from fast multiple precision integer routines.  
Another auxiliary use of multiple precision integers is high precision floating point data types.  
The basic IEEE \cite{IEEE} standard floating point type is made up of an integer mantissa $q$, an exponent $e$ and a sign bit $s$.  
Numbers are given in the form $n = q \cdot b^e \cdot -1^s$ where $b = 2$ is the most common base for IEEE.  Since IEEE 
floating point is meant to be implemented in hardware the precision of the mantissa is often fairly small 
(\textit{23, 48 and 64 bits}).  The mantissa is merely an integer and a multiple precision integer could be used to create
a mantissa of much larger precision than hardware alone can efficiently support.  This approach could be useful where 
scientific applications must minimize the total output error over long calculations.

Yet another use for large integers is within arithmetic on polynomials of large characteristic (i.e. $GF(p)[x]$ for large $p$).
In fact the library discussed within this text has already been used to form a polynomial basis library\footnote{See \url{http://poly.libtomcrypt.org} for more details.}.

\subsection{Benefits of Multiple Precision Arithmetic}
\index{precision}
The benefit of multiple precision representations over single or fixed precision representations is that 
no precision is lost while representing the result of an operation which requires excess precision.  For example, 
the product of two $n$-bit integers requires at least $2n$ bits of precision to be represented faithfully.  A multiple 
precision algorithm would augment the precision of the destination to accomodate the result while a single precision system 
would truncate excess bits to maintain a fixed level of precision.

It is possible to implement algorithms which require large integers with fixed precision algorithms.  For example, elliptic
curve cryptography (\textit{ECC}) is often implemented on smartcards by fixing the precision of the integers to the maximum 
size the system will ever need.  Such an approach can lead to vastly simpler algorithms which can accomodate the 
integers required even if the host platform cannot natively accomodate them\footnote{For example, the average smartcard 
processor has an 8 bit accumulator.}.  However, as efficient as such an approach may be, the resulting source code is not
normally very flexible.  It cannot, at runtime, accomodate inputs of higher magnitude than the designer anticipated.

Multiple precision algorithms have the most overhead of any style of arithmetic.  For the the most part the 
overhead can be kept to a minimum with careful planning, but overall, it is not well suited for most memory starved
platforms.  However, multiple precision algorithms do offer the most flexibility in terms of the magnitude of the 
inputs.  That is, the same algorithms based on multiple precision integers can accomodate any reasonable size input 
without the designer's explicit forethought.  This leads to lower cost of ownership for the code as it only has to 
be written and tested once.

\section{Purpose of This Text}
The purpose of this text is to instruct the reader regarding how to implement efficient multiple precision algorithms.  
That is to not only explain a limited subset of the core theory behind the algorithms but also the various ``house keeping'' 
elements that are neglected by authors of other texts on the subject.  Several well reknowned texts \cite{TAOCPV2,HAC} 
give considerably detailed explanations of the theoretical aspects of algorithms and often very little information 
regarding the practical implementation aspects.  

In most cases how an algorithm is explained and how it is actually implemented are two very different concepts.  For 
example, the Handbook of Applied Cryptography (\textit{HAC}), algorithm 14.7 on page 594, gives a relatively simple 
algorithm for performing multiple precision integer addition.  However, the description lacks any discussion concerning 
the fact that the two integer inputs may be of differing magnitudes.  As a result the implementation is not as simple
as the text would lead people to believe.  Similarly the division routine (\textit{algorithm 14.20, pp. 598}) does not 
discuss how to handle sign or handle the dividend's decreasing magnitude in the main loop (\textit{step \#3}).

Both texts also do not discuss several key optimal algorithms required such as ``Comba'' and Karatsuba multipliers 
and fast modular inversion, which we consider practical oversights.  These optimal algorithms are vital to achieve 
any form of useful performance in non-trivial applications.  

To solve this problem the focus of this text is on the practical aspects of implementing a multiple precision integer
package.  As a case study the ``LibTomMath''\footnote{Available at \url{http://math.libtomcrypt.com}} package is used 
to demonstrate algorithms with real implementations\footnote{In the ISO C programming language.} that have been field 
tested and work very well.  The LibTomMath library is freely available on the Internet for all uses and this text 
discusses a very large portion of the inner workings of the library.

The algorithms that are presented will always include at least one ``pseudo-code'' description followed 
by the actual C source code that implements the algorithm.  The pseudo-code can be used to implement the same 
algorithm in other programming languages as the reader sees fit.  

This text shall also serve as a walkthrough of the creation of multiple precision algorithms from scratch.  Showing
the reader how the algorithms fit together as well as where to start on various taskings.  

\section{Discussion and Notation}
\subsection{Notation}
A multiple precision integer of $n$-digits shall be denoted as $x = (x_{n-1}, \ldots, x_1, x_0)_{ \beta }$ and represent
the integer $x \equiv \sum_{i=0}^{n-1} x_i\beta^i$.  The elements of the array $x$ are said to be the radix $\beta$ digits 
of the integer.  For example, $x = (1,2,3)_{10}$ would represent the integer 
$1\cdot 10^2 + 2\cdot10^1 + 3\cdot10^0 = 123$.  

\index{mp\_int}
The term ``mp\_int'' shall refer to a composite structure which contains the digits of the integer it represents, as well 
as auxilary data required to manipulate the data.  These additional members are discussed further in section 
\ref{sec:MPINT}.  For the purposes of this text a ``multiple precision integer'' and an ``mp\_int'' are assumed to be 
synonymous.  When an algorithm is specified to accept an mp\_int variable it is assumed the various auxliary data members 
are present as well.  An expression of the type \textit{variablename.item} implies that it should evaluate to the 
member named ``item'' of the variable.  For example, a string of characters may have a member ``length'' which would 
evaluate to the number of characters in the string.  If the string $a$ equals ``hello'' then it follows that 
$a.length = 5$.  

For certain discussions more generic algorithms are presented to help the reader understand the final algorithm used
to solve a given problem.  When an algorithm is described as accepting an integer input it is assumed the input is 
a plain integer with no additional multiple-precision members.  That is, algorithms that use integers as opposed to 
mp\_ints as inputs do not concern themselves with the housekeeping operations required such as memory management.  These 
algorithms will be used to establish the relevant theory which will subsequently be used to describe a multiple
precision algorithm to solve the same problem.  

\subsection{Precision Notation}
The variable $\beta$ represents the radix of a single digit of a multiple precision integer and 
must be of the form $q^p$ for $q, p \in \Z^+$.  A single precision variable must be able to represent integers in 
the range $0 \le x < q \beta$ while a double precision variable must be able to represent integers in the range 
$0 \le x < q \beta^2$.  The extra radix-$q$ factor allows additions and subtractions to proceed without truncation of the 
carry.  Since all modern computers are binary, it is assumed that $q$ is two.

\index{mp\_digit} \index{mp\_word}
Within the source code that will be presented for each algorithm, the data type \textbf{mp\_digit} will represent 
a single precision integer type, while, the data type \textbf{mp\_word} will represent a double precision integer type.  In 
several algorithms (notably the Comba routines) temporary results will be stored in arrays of double precision mp\_words.  
For the purposes of this text $x_j$ will refer to the $j$'th digit of a single precision array and $\hat x_j$ will refer to 
the $j$'th digit of a double precision array.  Whenever an expression is to be assigned to a double precision
variable it is assumed that all single precision variables are promoted to double precision during the evaluation.  
Expressions that are assigned to a single precision variable are truncated to fit within the precision of a single
precision data type.

For example, if $\beta = 10^2$ a single precision data type may represent a value in the 
range $0 \le x < 10^3$, while a double precision data type may represent a value in the range $0 \le x < 10^5$.  Let
$a = 23$ and $b = 49$ represent two single precision variables.  The single precision product shall be written
as $c \leftarrow a \cdot b$ while the double precision product shall be written as $\hat c \leftarrow a \cdot b$.
In this particular case, $\hat c = 1127$ and $c = 127$.  The most significant digit of the product would not fit 
in a single precision data type and as a result $c \ne \hat c$.  

\subsection{Algorithm Inputs and Outputs}
Within the algorithm descriptions all variables are assumed to be scalars of either single or double precision
as indicated.  The only exception to this rule is when variables have been indicated to be of type mp\_int.  This 
distinction is important as scalars are often used as array indicies and various other counters.  

\subsection{Mathematical Expressions}
The $\lfloor \mbox{ } \rfloor$ brackets imply an expression truncated to an integer not greater than the expression 
itself.  For example, $\lfloor 5.7 \rfloor = 5$.  Similarly the $\lceil \mbox{ } \rceil$ brackets imply an expression
rounded to an integer not less than the expression itself.  For example, $\lceil 5.1 \rceil = 6$.  Typically when 
the $/$ division symbol is used the intention is to perform an integer division with truncation.  For example, 
$5/2 = 2$ which will often be written as $\lfloor 5/2 \rfloor = 2$ for clarity.  When an expression is written as a 
fraction a real value division is implied, for example ${5 \over 2} = 2.5$.  

The norm of a multiple precision integer, for example $\vert \vert x \vert \vert$, will be used to represent the number of digits in the representation
of the integer.  For example, $\vert \vert 123 \vert \vert = 3$ and $\vert \vert 79452 \vert \vert = 5$.  

\subsection{Work Effort}
\index{big-Oh}
To measure the efficiency of the specified algorithms, a modified big-Oh notation is used.  In this system all 
single precision operations are considered to have the same cost\footnote{Except where explicitly noted.}.  
That is a single precision addition, multiplication and division are assumed to take the same time to 
complete.  While this is generally not true in practice, it will simplify the discussions considerably.

Some algorithms have slight advantages over others which is why some constants will not be removed in 
the notation.  For example, a normal baseline multiplication (section \ref{sec:basemult}) requires $O(n^2)$ work while a 
baseline squaring (section \ref{sec:basesquare}) requires $O({{n^2 + n}\over 2})$ work.  In standard big-Oh notation these 
would both be said to be equivalent to $O(n^2)$.  However, 
in the context of the this text this is not the case as the magnitude of the inputs will typically be rather small.  As a 
result small constant factors in the work effort will make an observable difference in algorithm efficiency.

All of the algorithms presented in this text have a polynomial time work level.  That is, of the form 
$O(n^k)$ for $n, k \in \Z^{+}$.  This will help make useful comparisons in terms of the speed of the algorithms and how 
various optimizations will help pay off in the long run.

\section{Exercises}
Within the more advanced chapters a section will be set aside to give the reader some challenging exercises related to
the discussion at hand.  These exercises are not designed to be prize winning problems, but instead to be thought 
provoking.  Wherever possible the problems are forward minded, stating problems that will be answered in subsequent 
chapters.  The reader is encouraged to finish the exercises as they appear to get a better understanding of the 
subject material.  

That being said, the problems are designed to affirm knowledge of a particular subject matter.  Students in particular
are encouraged to verify they can answer the problems correctly before moving on.

Similar to the exercises of \cite[pp. ix]{TAOCPV2} these exercises are given a scoring system based on the difficulty of
the problem.  However, unlike \cite{TAOCPV2} the problems do not get nearly as hard.  The scoring of these 
exercises ranges from one (the easiest) to five (the hardest).  The following table sumarizes the 
scoring system used.

\begin{figure}[here]
\begin{center}
\begin{small}
\begin{tabular}{|c|l|}
\hline $\left [ 1 \right ]$ & An easy problem that should only take the reader a manner of \\
                            & minutes to solve.  Usually does not involve much computer time \\
                            & to solve. \\
\hline $\left [ 2 \right ]$ & An easy problem that involves a marginal amount of computer \\
                     & time usage.  Usually requires a program to be written to \\
                     & solve the problem. \\
\hline $\left [ 3 \right ]$ & A moderately hard problem that requires a non-trivial amount \\
                     & of work.  Usually involves trivial research and development of \\
                     & new theory from the perspective of a student. \\
\hline $\left [ 4 \right ]$ & A moderately hard problem that involves a non-trivial amount \\
                     & of work and research, the solution to which will demonstrate \\
                     & a higher mastery of the subject matter. \\
\hline $\left [ 5 \right ]$ & A hard problem that involves concepts that are difficult for a \\
                     & novice to solve.  Solutions to these problems will demonstrate a \\
                     & complete mastery of the given subject. \\
\hline
\end{tabular}
\end{small}
\end{center}
\caption{Exercise Scoring System}
\end{figure}

Problems at the first level are meant to be simple questions that the reader can answer quickly without programming a solution or
devising new theory.  These problems are quick tests to see if the material is understood.  Problems at the second level 
are also designed to be easy but will require a program or algorithm to be implemented to arrive at the answer.  These
two levels are essentially entry level questions.  

Problems at the third level are meant to be a bit more difficult than the first two levels.  The answer is often 
fairly obvious but arriving at an exacting solution requires some thought and skill.  These problems will almost always 
involve devising a new algorithm or implementing a variation of another algorithm previously presented.  Readers who can
answer these questions will feel comfortable with the concepts behind the topic at hand.

Problems at the fourth level are meant to be similar to those of the level three questions except they will require 
additional research to be completed.  The reader will most likely not know the answer right away, nor will the text provide 
the exact details of the answer until a subsequent chapter.  

Problems at the fifth level are meant to be the hardest 
problems relative to all the other problems in the chapter.  People who can correctly answer fifth level problems have a 
mastery of the subject matter at hand.

Often problems will be tied together.  The purpose of this is to start a chain of thought that will be discussed in future chapters.  The reader
is encouraged to answer the follow-up problems and try to draw the relevance of problems.

\section{Introduction to LibTomMath}

\subsection{What is LibTomMath?}
LibTomMath is a free and open source multiple precision integer library written entirely in portable ISO C.  By portable it 
is meant that the library does not contain any code that is computer platform dependent or otherwise problematic to use on 
any given platform.  

The library has been successfully tested under numerous operating systems including Unix\footnote{All of these
trademarks belong to their respective rightful owners.}, MacOS, Windows, Linux, PalmOS and on standalone hardware such 
as the Gameboy Advance.  The library is designed to contain enough functionality to be able to develop applications such 
as public key cryptosystems and still maintain a relatively small footprint.

\subsection{Goals of LibTomMath}

Libraries which obtain the most efficiency are rarely written in a high level programming language such as C.  However, 
even though this library is written entirely in ISO C, considerable care has been taken to optimize the algorithm implementations within the 
library.  Specifically the code has been written to work well with the GNU C Compiler (\textit{GCC}) on both x86 and ARM 
processors.  Wherever possible, highly efficient algorithms, such as Karatsuba multiplication, sliding window 
exponentiation and Montgomery reduction have been provided to make the library more efficient.  

Even with the nearly optimal and specialized algorithms that have been included the Application Programing Interface 
(\textit{API}) has been kept as simple as possible.  Often generic place holder routines will make use of specialized 
algorithms automatically without the developer's specific attention.  One such example is the generic multiplication 
algorithm \textbf{mp\_mul()} which will automatically use Toom--Cook, Karatsuba, Comba or baseline multiplication 
based on the magnitude of the inputs and the configuration of the library.  

Making LibTomMath as efficient as possible is not the only goal of the LibTomMath project.  Ideally the library should 
be source compatible with another popular library which makes it more attractive for developers to use.  In this case the
MPI library was used as a API template for all the basic functions.  MPI was chosen because it is another library that fits 
in the same niche as LibTomMath.  Even though LibTomMath uses MPI as the template for the function names and argument 
passing conventions, it has been written from scratch by Tom St Denis.

The project is also meant to act as a learning tool for students, the logic being that no easy-to-follow ``bignum'' 
library exists which can be used to teach computer science students how to perform fast and reliable multiple precision 
integer arithmetic.  To this end the source code has been given quite a few comments and algorithm discussion points.  

\section{Choice of LibTomMath}
LibTomMath was chosen as the case study of this text not only because the author of both projects is one and the same but
for more worthy reasons.  Other libraries such as GMP \cite{GMP}, MPI \cite{MPI}, LIP \cite{LIP} and OpenSSL 
\cite{OPENSSL} have multiple precision integer arithmetic routines but would not be ideal for this text for 
reasons that will be explained in the following sub-sections.

\subsection{Code Base}
The LibTomMath code base is all portable ISO C source code.  This means that there are no platform dependent conditional
segments of code littered throughout the source.  This clean and uncluttered approach to the library means that a
developer can more readily discern the true intent of a given section of source code without trying to keep track of
what conditional code will be used.

The code base of LibTomMath is well organized.  Each function is in its own separate source code file 
which allows the reader to find a given function very quickly.  On average there are $76$ lines of code per source
file which makes the source very easily to follow.  By comparison MPI and LIP are single file projects making code tracing
very hard.  GMP has many conditional code segments which also hinder tracing.  

When compiled with GCC for the x86 processor and optimized for speed the entire library is approximately $100$KiB\footnote{The notation ``KiB'' means $2^{10}$ octets, similarly ``MiB'' means $2^{20}$ octets.}
 which is fairly small compared to GMP (over $250$KiB).  LibTomMath is slightly larger than MPI (which compiles to about 
$50$KiB) but LibTomMath is also much faster and more complete than MPI.

\subsection{API Simplicity}
LibTomMath is designed after the MPI library and shares the API design.  Quite often programs that use MPI will build 
with LibTomMath without change. The function names correlate directly to the action they perform.  Almost all of the 
functions share the same parameter passing convention.  The learning curve is fairly shallow with the API provided 
which is an extremely valuable benefit for the student and developer alike.  

The LIP library is an example of a library with an API that is awkward to work with.  LIP uses function names that are often ``compressed'' to 
illegible short hand.  LibTomMath does not share this characteristic.  

The GMP library also does not return error codes.  Instead it uses a POSIX.1 \cite{POSIX1} signal system where errors
are signaled to the host application.  This happens to be the fastest approach but definitely not the most versatile.  In
effect a math error (i.e. invalid input, heap error, etc) can cause a program to stop functioning which is definitely 
undersireable in many situations.

\subsection{Optimizations}
While LibTomMath is certainly not the fastest library (GMP often beats LibTomMath by a factor of two) it does
feature a set of optimal algorithms for tasks such as modular reduction, exponentiation, multiplication and squaring.  GMP 
and LIP also feature such optimizations while MPI only uses baseline algorithms with no optimizations.  GMP lacks a few
of the additional modular reduction optimizations that LibTomMath features\footnote{At the time of this writing GMP
only had Barrett and Montgomery modular reduction algorithms.}.  

LibTomMath is almost always an order of magnitude faster than the MPI library at computationally expensive tasks such as modular
exponentiation.  In the grand scheme of ``bignum'' libraries LibTomMath is faster than the average library and usually  
slower than the best libraries such as GMP and OpenSSL by only a small factor.

\subsection{Portability and Stability}
LibTomMath will build ``out of the box'' on any platform equipped with a modern version of the GNU C Compiler 
(\textit{GCC}).  This means that without changes the library will build without configuration or setting up any 
variables.  LIP and MPI will build ``out of the box'' as well but have numerous known bugs.  Most notably the author of 
MPI has recently stopped working on his library and LIP has long since been discontinued.  

GMP requires a configuration script to run and will not build out of the box.   GMP and LibTomMath are still in active
development and are very stable across a variety of platforms.

\subsection{Choice}
LibTomMath is a relatively compact, well documented, highly optimized and portable library which seems only natural for
the case study of this text.  Various source files from the LibTomMath project will be included within the text.  However, 
the reader is encouraged to download their own copy of the library to actually be able to work with the library.  

\chapter{Getting Started}
\section{Library Basics}
The trick to writing any useful library of source code is to build a solid foundation and work outwards from it.  First, 
a problem along with allowable solution parameters should be identified and analyzed.  In this particular case the 
inability to accomodate multiple precision integers is the problem.  Futhermore, the solution must be written
as portable source code that is reasonably efficient across several different computer platforms.

After a foundation is formed the remainder of the library can be designed and implemented in a hierarchical fashion.  
That is, to implement the lowest level dependencies first and work towards the most abstract functions last.  For example, 
before implementing a modular exponentiation algorithm one would implement a modular reduction algorithm.
By building outwards from a base foundation instead of using a parallel design methodology the resulting project is 
highly modular.  Being highly modular is a desirable property of any project as it often means the resulting product
has a small footprint and updates are easy to perform.  

Usually when I start a project I will begin with the header files.  I define the data types I think I will need and 
prototype the initial functions that are not dependent on other functions (within the library).  After I 
implement these base functions I prototype more dependent functions and implement them.   The process repeats until
I implement all of the functions I require.  For example, in the case of LibTomMath I implemented functions such as 
mp\_init() well before I implemented mp\_mul() and even further before I implemented mp\_exptmod().  As an example as to 
why this design works note that the Karatsuba and Toom-Cook multipliers were written \textit{after} the 
dependent function mp\_exptmod() was written.  Adding the new multiplication algorithms did not require changes to the 
mp\_exptmod() function itself and lowered the total cost of ownership (\textit{so to speak}) and of development 
for new algorithms.  This methodology allows new algorithms to be tested in a complete framework with relative ease.

\begin{center}
\begin{figure}[here]
\includegraphics{pics/design_process.ps}
\caption{Design Flow of the First Few Original LibTomMath Functions.}
\label{pic:design_process}
\end{figure}
\end{center}

Only after the majority of the functions were in place did I pursue a less hierarchical approach to auditing and optimizing
the source code.  For example, one day I may audit the multipliers and the next day the polynomial basis functions.  

It only makes sense to begin the text with the preliminary data types and support algorithms required as well.  
This chapter discusses the core algorithms of the library which are the dependents for every other algorithm.

\section{What is a Multiple Precision Integer?}
Recall that most programming languages, in particular ISO C \cite{ISOC}, only have fixed precision data types that on their own cannot 
be used to represent values larger than their precision will allow. The purpose of multiple precision algorithms is 
to use fixed precision data types to create and manipulate multiple precision integers which may represent values 
that are very large.  

As a well known analogy, school children are taught how to form numbers larger than nine by prepending more radix ten digits.  In the decimal system
the largest single digit value is $9$.  However, by concatenating digits together larger numbers may be represented.  Newly prepended digits 
(\textit{to the left}) are said to be in a different power of ten column.  That is, the number $123$ can be described as having a $1$ in the hundreds 
column, $2$ in the tens column and $3$ in the ones column.  Or more formally $123 = 1 \cdot 10^2 + 2 \cdot 10^1 + 3 \cdot 10^0$.  Computer based 
multiple precision arithmetic is essentially the same concept.  Larger integers are represented by adjoining fixed 
precision computer words with the exception that a different radix is used.

What most people probably do not think about explicitly are the various other attributes that describe a multiple precision 
integer.  For example, the integer $154_{10}$ has two immediately obvious properties.  First, the integer is positive, 
that is the sign of this particular integer is positive as opposed to negative.  Second, the integer has three digits in 
its representation.  There is an additional property that the integer posesses that does not concern pencil-and-paper 
arithmetic.  The third property is how many digits placeholders are available to hold the integer.  

The human analogy of this third property is ensuring there is enough space on the paper to write the integer.  For example,
if one starts writing a large number too far to the right on a piece of paper they will have to erase it and move left.  
Similarly, computer algorithms must maintain strict control over memory usage to ensure that the digits of an integer
will not exceed the allowed boundaries.  These three properties make up what is known as a multiple precision 
integer or mp\_int for short.  

\subsection{The mp\_int Structure}
\label{sec:MPINT}
The mp\_int structure is the ISO C based manifestation of what represents a multiple precision integer.  The ISO C standard does not provide for 
any such data type but it does provide for making composite data types known as structures.  The following is the structure definition 
used within LibTomMath.

\index{mp\_int}
\begin{figure}[here]
\begin{center}
\begin{small}
%\begin{verbatim}
\begin{tabular}{|l|}
\hline
typedef struct \{ \\
\hspace{3mm}int used, alloc, sign;\\
\hspace{3mm}mp\_digit *dp;\\
\} \textbf{mp\_int}; \\
\hline
\end{tabular}
%\end{verbatim}
\end{small}
\caption{The mp\_int Structure}
\label{fig:mpint}
\end{center}
\end{figure}

The mp\_int structure (fig. \ref{fig:mpint}) can be broken down as follows.

\begin{enumerate}
\item The \textbf{used} parameter denotes how many digits of the array \textbf{dp} contain the digits used to represent
a given integer.  The \textbf{used} count must be positive (or zero) and may not exceed the \textbf{alloc} count.  

\item The \textbf{alloc} parameter denotes how 
many digits are available in the array to use by functions before it has to increase in size.  When the \textbf{used} count 
of a result would exceed the \textbf{alloc} count all of the algorithms will automatically increase the size of the 
array to accommodate the precision of the result.  

\item The pointer \textbf{dp} points to a dynamically allocated array of digits that represent the given multiple 
precision integer.  It is padded with $(\textbf{alloc} - \textbf{used})$ zero digits.  The array is maintained in a least 
significant digit order.  As a pencil and paper analogy the array is organized such that the right most digits are stored
first starting at the location indexed by zero\footnote{In C all arrays begin at zero.} in the array.  For example, 
if \textbf{dp} contains $\lbrace a, b, c, \ldots \rbrace$ where \textbf{dp}$_0 = a$, \textbf{dp}$_1 = b$, \textbf{dp}$_2 = c$, $\ldots$ then 
it would represent the integer $a + b\beta + c\beta^2 + \ldots$  

\index{MP\_ZPOS} \index{MP\_NEG}
\item The \textbf{sign} parameter denotes the sign as either zero/positive (\textbf{MP\_ZPOS}) or negative (\textbf{MP\_NEG}).  
\end{enumerate}

\subsubsection{Valid mp\_int Structures}
Several rules are placed on the state of an mp\_int structure and are assumed to be followed for reasons of efficiency.  
The only exceptions are when the structure is passed to initialization functions such as mp\_init() and mp\_init\_copy().

\begin{enumerate}
\item The value of \textbf{alloc} may not be less than one.  That is \textbf{dp} always points to a previously allocated
array of digits.
\item The value of \textbf{used} may not exceed \textbf{alloc} and must be greater than or equal to zero.
\item The value of \textbf{used} implies the digit at index $(used - 1)$ of the \textbf{dp} array is non-zero.  That is, 
leading zero digits in the most significant positions must be trimmed.
   \begin{enumerate}
   \item Digits in the \textbf{dp} array at and above the \textbf{used} location must be zero.
   \end{enumerate}
\item The value of \textbf{sign} must be \textbf{MP\_ZPOS} if \textbf{used} is zero; 
this represents the mp\_int value of zero.
\end{enumerate}

\section{Argument Passing}
A convention of argument passing must be adopted early on in the development of any library.  Making the function 
prototypes consistent will help eliminate many headaches in the future as the library grows to significant complexity.  
In LibTomMath the multiple precision integer functions accept parameters from left to right as pointers to mp\_int 
structures.  That means that the source (input) operands are placed on the left and the destination (output) on the right.   
Consider the following examples.

\begin{verbatim}
   mp_mul(&a, &b, &c);   /* c = a * b */
   mp_add(&a, &b, &a);   /* a = a + b */
   mp_sqr(&a, &b);       /* b = a * a */
\end{verbatim}

The left to right order is a fairly natural way to implement the functions since it lets the developer read aloud the
functions and make sense of them.  For example, the first function would read ``multiply a and b and store in c''.

Certain libraries (\textit{LIP by Lenstra for instance}) accept parameters the other way around, to mimic the order
of assignment expressions.  That is, the destination (output) is on the left and arguments (inputs) are on the right.  In 
truth, it is entirely a matter of preference.  In the case of LibTomMath the convention from the MPI library has been 
adopted.  

Another very useful design consideration, provided for in LibTomMath, is whether to allow argument sources to also be a 
destination.  For example, the second example (\textit{mp\_add}) adds $a$ to $b$ and stores in $a$.  This is an important 
feature to implement since it allows the calling functions to cut down on the number of variables it must maintain.  
However, to implement this feature specific care has to be given to ensure the destination is not modified before the 
source is fully read.

\section{Return Values}
A well implemented application, no matter what its purpose, should trap as many runtime errors as possible and return them 
to the caller.  By catching runtime errors a library can be guaranteed to prevent undefined behaviour.  However, the end 
developer can still manage to cause a library to crash.  For example, by passing an invalid pointer an application may
fault by dereferencing memory not owned by the application.

In the case of LibTomMath the only errors that are checked for are related to inappropriate inputs (division by zero for 
instance) and memory allocation errors.  It will not check that the mp\_int passed to any function is valid nor 
will it check pointers for validity.  Any function that can cause a runtime error will return an error code as an 
\textbf{int} data type with one of the following values (fig \ref{fig:errcodes}).

\index{MP\_OKAY} \index{MP\_VAL} \index{MP\_MEM}
\begin{figure}[here]
\begin{center}
\begin{tabular}{|l|l|}
\hline \textbf{Value} & \textbf{Meaning} \\
\hline \textbf{MP\_OKAY} & The function was successful \\
\hline \textbf{MP\_VAL}  & One of the input value(s) was invalid \\
\hline \textbf{MP\_MEM}  & The function ran out of heap memory \\
\hline
\end{tabular}
\end{center}
\caption{LibTomMath Error Codes}
\label{fig:errcodes}
\end{figure}

When an error is detected within a function it should free any memory it allocated, often during the initialization of
temporary mp\_ints, and return as soon as possible.  The goal is to leave the system in the same state it was when the 
function was called.  Error checking with this style of API is fairly simple.

\begin{verbatim}
   int err;
   if ((err = mp_add(&a, &b, &c)) != MP_OKAY) {
      printf("Error: %s\n", mp_error_to_string(err));
      exit(EXIT_FAILURE);
   }
\end{verbatim}

The GMP \cite{GMP} library uses C style \textit{signals} to flag errors which is of questionable use.  Not all errors are fatal 
and it was not deemed ideal by the author of LibTomMath to force developers to have signal handlers for such cases.

\section{Initialization and Clearing}
The logical starting point when actually writing multiple precision integer functions is the initialization and 
clearing of the mp\_int structures.  These two algorithms will be used by the majority of the higher level algorithms.

Given the basic mp\_int structure an initialization routine must first allocate memory to hold the digits of
the integer.  Often it is optimal to allocate a sufficiently large pre-set number of digits even though
the initial integer will represent zero.  If only a single digit were allocated quite a few subsequent re-allocations
would occur when operations are performed on the integers.  There is a tradeoff between how many default digits to allocate
and how many re-allocations are tolerable.  Obviously allocating an excessive amount of digits initially will waste 
memory and become unmanageable.  

If the memory for the digits has been successfully allocated then the rest of the members of the structure must
be initialized.  Since the initial state of an mp\_int is to represent the zero integer, the allocated digits must be set
to zero.  The \textbf{used} count set to zero and \textbf{sign} set to \textbf{MP\_ZPOS}.

\subsection{Initializing an mp\_int}
An mp\_int is said to be initialized if it is set to a valid, preferably default, state such that all of the members of the
structure are set to valid values.  The mp\_init algorithm will perform such an action.

\index{mp\_init}
\begin{figure}[here]
\begin{center}
\begin{tabular}{l}
\hline Algorithm \textbf{mp\_init}. \\
\textbf{Input}.   An mp\_int $a$ \\
\textbf{Output}.  Allocate memory and initialize $a$ to a known valid mp\_int state.  \\
\hline \\
1.  Allocate memory for \textbf{MP\_PREC} digits. \\
2.  If the allocation failed return(\textit{MP\_MEM}) \\
3.  for $n$ from $0$ to $MP\_PREC - 1$ do  \\
\hspace{3mm}3.1  $a_n \leftarrow 0$\\
4.  $a.sign \leftarrow MP\_ZPOS$\\
5.  $a.used \leftarrow 0$\\
6.  $a.alloc \leftarrow MP\_PREC$\\
7.  Return(\textit{MP\_OKAY})\\
\hline
\end{tabular}
\end{center}
\caption{Algorithm mp\_init}
\end{figure}

\textbf{Algorithm mp\_init.}
The purpose of this function is to initialize an mp\_int structure so that the rest of the library can properly
manipulte it.  It is assumed that the input may not have had any of its members previously initialized which is certainly
a valid assumption if the input resides on the stack.  

Before any of the members such as \textbf{sign}, \textbf{used} or \textbf{alloc} are initialized the memory for
the digits is allocated.  If this fails the function returns before setting any of the other members.  The \textbf{MP\_PREC} 
name represents a constant\footnote{Defined in the ``tommath.h'' header file within LibTomMath.} 
used to dictate the minimum precision of newly initialized mp\_int integers.  Ideally, it is at least equal to the smallest
precision number you'll be working with.

Allocating a block of digits at first instead of a single digit has the benefit of lowering the number of usually slow
heap operations later functions will have to perform in the future.  If \textbf{MP\_PREC} is set correctly the slack 
memory and the number of heap operations will be trivial.

Once the allocation has been made the digits have to be set to zero as well as the \textbf{used}, \textbf{sign} and
\textbf{alloc} members initialized.  This ensures that the mp\_int will always represent the default state of zero regardless
of the original condition of the input.

\textbf{Remark.}
This function introduces the idiosyncrasy that all iterative loops, commonly initiated with the ``for'' keyword, iterate incrementally
when the ``to'' keyword is placed between two expressions.  For example, ``for $a$ from $b$ to $c$ do'' means that
a subsequent expression (or body of expressions) are to be evaluated upto $c - b$ times so long as $b \le c$.  In each
iteration the variable $a$ is substituted for a new integer that lies inclusively between $b$ and $c$.  If $b > c$ occured
the loop would not iterate.  By contrast if the ``downto'' keyword were used in place of ``to'' the loop would iterate 
decrementally.

\vspace{+3mm}\begin{small}
\hspace{-5.1mm}{\bf File}: bn\_mp\_init.c
\vspace{-3mm}
\begin{alltt}
\end{alltt}
\end{small}

One immediate observation of this initializtion function is that it does not return a pointer to a mp\_int structure.  It 
is assumed that the caller has already allocated memory for the mp\_int structure, typically on the application stack.  The 
call to mp\_init() is used only to initialize the members of the structure to a known default state.  

Here we see (line 24) the memory allocation is performed first.  This allows us to exit cleanly and quickly
if there is an error.  If the allocation fails the routine will return \textbf{MP\_MEM} to the caller to indicate there
was a memory error.  The function XMALLOC is what actually allocates the memory.  Technically XMALLOC is not a function
but a macro defined in ``tommath.h``.  By default, XMALLOC will evaluate to malloc() which is the C library's built--in
memory allocation routine.

In order to assure the mp\_int is in a known state the digits must be set to zero.  On most platforms this could have been
accomplished by using calloc() instead of malloc().  However,  to correctly initialize a integer type to a given value in a 
portable fashion you have to actually assign the value.  The for loop (line 30) performs this required
operation.

After the memory has been successfully initialized the remainder of the members are initialized 
(lines 34 through 35) to their respective default states.  At this point the algorithm has succeeded and
a success code is returned to the calling function.  If this function returns \textbf{MP\_OKAY} it is safe to assume the 
mp\_int structure has been properly initialized and is safe to use with other functions within the library.  

\subsection{Clearing an mp\_int}
When an mp\_int is no longer required by the application, the memory that has been allocated for its digits must be 
returned to the application's memory pool with the mp\_clear algorithm.

\begin{figure}[here]
\begin{center}
\begin{tabular}{l}
\hline Algorithm \textbf{mp\_clear}. \\
\textbf{Input}.   An mp\_int $a$ \\
\textbf{Output}.  The memory for $a$ shall be deallocated.  \\
\hline \\
1.  If $a$ has been previously freed then return(\textit{MP\_OKAY}). \\
2.  for $n$ from 0 to $a.used - 1$ do \\
\hspace{3mm}2.1  $a_n \leftarrow 0$ \\
3.  Free the memory allocated for the digits of $a$. \\
4.  $a.used \leftarrow 0$ \\
5.  $a.alloc \leftarrow 0$ \\
6.  $a.sign \leftarrow MP\_ZPOS$ \\
7.  Return(\textit{MP\_OKAY}). \\
\hline
\end{tabular}
\end{center}
\caption{Algorithm mp\_clear}
\end{figure}

\textbf{Algorithm mp\_clear.}
This algorithm accomplishes two goals.  First, it clears the digits and the other mp\_int members.  This ensures that 
if a developer accidentally re-uses a cleared structure it is less likely to cause problems.  The second goal
is to free the allocated memory.

The logic behind the algorithm is extended by marking cleared mp\_int structures so that subsequent calls to this
algorithm will not try to free the memory multiple times.  Cleared mp\_ints are detectable by having a pre-defined invalid 
digit pointer \textbf{dp} setting.

Once an mp\_int has been cleared the mp\_int structure is no longer in a valid state for any other algorithm
with the exception of algorithms mp\_init, mp\_init\_copy, mp\_init\_size and mp\_clear.

\vspace{+3mm}\begin{small}
\hspace{-5.1mm}{\bf File}: bn\_mp\_clear.c
\vspace{-3mm}
\begin{alltt}
\end{alltt}
\end{small}

The algorithm only operates on the mp\_int if it hasn't been previously cleared.  The if statement (line 25)
checks to see if the \textbf{dp} member is not \textbf{NULL}.  If the mp\_int is a valid mp\_int then \textbf{dp} cannot be
\textbf{NULL} in which case the if statement will evaluate to true.

The digits of the mp\_int are cleared by the for loop (line 27) which assigns a zero to every digit.  Similar to mp\_init()
the digits are assigned zero instead of using block memory operations (such as memset()) since this is more portable.  

The digits are deallocated off the heap via the XFREE macro.  Similar to XMALLOC the XFREE macro actually evaluates to
a standard C library function.  In this case the free() function.  Since free() only deallocates the memory the pointer
still has to be reset to \textbf{NULL} manually (line 35).  

Now that the digits have been cleared and deallocated the other members are set to their final values (lines 36 and 37).

\section{Maintenance Algorithms}

The previous sections describes how to initialize and clear an mp\_int structure.  To further support operations
that are to be performed on mp\_int structures (such as addition and multiplication) the dependent algorithms must be
able to augment the precision of an mp\_int and 
initialize mp\_ints with differing initial conditions.  

These algorithms complete the set of low level algorithms required to work with mp\_int structures in the higher level
algorithms such as addition, multiplication and modular exponentiation.

\subsection{Augmenting an mp\_int's Precision}
When storing a value in an mp\_int structure, a sufficient number of digits must be available to accomodate the entire 
result of an operation without loss of precision.  Quite often the size of the array given by the \textbf{alloc} member 
is large enough to simply increase the \textbf{used} digit count.  However, when the size of the array is too small it 
must be re-sized appropriately to accomodate the result.  The mp\_grow algorithm will provide this functionality.

\newpage\begin{figure}[here]
\begin{center}
\begin{tabular}{l}
\hline Algorithm \textbf{mp\_grow}. \\
\textbf{Input}.   An mp\_int $a$ and an integer $b$. \\
\textbf{Output}.  $a$ is expanded to accomodate $b$ digits. \\
\hline \\
1.  if $a.alloc \ge b$ then return(\textit{MP\_OKAY}) \\
2.  $u \leftarrow b\mbox{ (mod }MP\_PREC\mbox{)}$ \\
3.  $v \leftarrow b + 2 \cdot MP\_PREC - u$ \\
4.  Re-allocate the array of digits $a$ to size $v$ \\
5.  If the allocation failed then return(\textit{MP\_MEM}). \\
6.  for n from a.alloc to $v - 1$ do  \\
\hspace{+3mm}6.1  $a_n \leftarrow 0$ \\
7.  $a.alloc \leftarrow v$ \\
8.  Return(\textit{MP\_OKAY}) \\
\hline
\end{tabular}
\end{center}
\caption{Algorithm mp\_grow}
\end{figure}

\textbf{Algorithm mp\_grow.}
It is ideal to prevent re-allocations from being performed if they are not required (step one).  This is useful to 
prevent mp\_ints from growing excessively in code that erroneously calls mp\_grow.  

The requested digit count is padded up to next multiple of \textbf{MP\_PREC} plus an additional \textbf{MP\_PREC} (steps two and three).  
This helps prevent many trivial reallocations that would grow an mp\_int by trivially small values.  

It is assumed that the reallocation (step four) leaves the lower $a.alloc$ digits of the mp\_int intact.  This is much 
akin to how the \textit{realloc} function from the standard C library works.  Since the newly allocated digits are 
assumed to contain undefined values they are initially set to zero.

\vspace{+3mm}\begin{small}
\hspace{-5.1mm}{\bf File}: bn\_mp\_grow.c
\vspace{-3mm}
\begin{alltt}
\end{alltt}
\end{small}

A quick optimization is to first determine if a memory re-allocation is required at all.  The if statement (line 24) checks
if the \textbf{alloc} member of the mp\_int is smaller than the requested digit count.  If the count is not larger than \textbf{alloc}
the function skips the re-allocation part thus saving time.

When a re-allocation is performed it is turned into an optimal request to save time in the future.  The requested digit count is
padded upwards to 2nd multiple of \textbf{MP\_PREC} larger than \textbf{alloc} (line 25).  The XREALLOC function is used
to re-allocate the memory.  As per the other functions XREALLOC is actually a macro which evaluates to realloc by default.  The realloc
function leaves the base of the allocation intact which means the first \textbf{alloc} digits of the mp\_int are the same as before
the re-allocation.  All	that is left is to clear the newly allocated digits and return.

Note that the re-allocation result is actually stored in a temporary pointer $tmp$.  This is to allow this function to return
an error with a valid pointer.  Earlier releases of the library stored the result of XREALLOC into the mp\_int $a$.  That would
result in a memory leak if XREALLOC ever failed.  

\subsection{Initializing Variable Precision mp\_ints}
Occasionally the number of digits required will be known in advance of an initialization, based on, for example, the size 
of input mp\_ints to a given algorithm.  The purpose of algorithm mp\_init\_size is similar to mp\_init except that it 
will allocate \textit{at least} a specified number of digits.  

\begin{figure}[here]
\begin{small}
\begin{center}
\begin{tabular}{l}
\hline Algorithm \textbf{mp\_init\_size}. \\
\textbf{Input}.   An mp\_int $a$ and the requested number of digits $b$. \\
\textbf{Output}.  $a$ is initialized to hold at least $b$ digits. \\
\hline \\
1.  $u \leftarrow b \mbox{ (mod }MP\_PREC\mbox{)}$ \\
2.  $v \leftarrow b + 2 \cdot MP\_PREC - u$ \\
3.  Allocate $v$ digits. \\
4.  for $n$ from $0$ to $v - 1$ do \\
\hspace{3mm}4.1  $a_n \leftarrow 0$ \\
5.  $a.sign \leftarrow MP\_ZPOS$\\
6.  $a.used \leftarrow 0$\\
7.  $a.alloc \leftarrow v$\\
8.  Return(\textit{MP\_OKAY})\\
\hline
\end{tabular}
\end{center}
\end{small}
\caption{Algorithm mp\_init\_size}
\end{figure}

\textbf{Algorithm mp\_init\_size.}
This algorithm will initialize an mp\_int structure $a$ like algorithm mp\_init with the exception that the number of 
digits allocated can be controlled by the second input argument $b$.  The input size is padded upwards so it is a 
multiple of \textbf{MP\_PREC} plus an additional \textbf{MP\_PREC} digits.  This padding is used to prevent trivial 
allocations from becoming a bottleneck in the rest of the algorithms.

Like algorithm mp\_init, the mp\_int structure is initialized to a default state representing the integer zero.  This 
particular algorithm is useful if it is known ahead of time the approximate size of the input.  If the approximation is
correct no further memory re-allocations are required to work with the mp\_int.

\vspace{+3mm}\begin{small}
\hspace{-5.1mm}{\bf File}: bn\_mp\_init\_size.c
\vspace{-3mm}
\begin{alltt}
\end{alltt}
\end{small}

The number of digits $b$ requested is padded (line 24) by first augmenting it to the next multiple of 
\textbf{MP\_PREC} and then adding \textbf{MP\_PREC} to the result.  If the memory can be successfully allocated the 
mp\_int is placed in a default state representing the integer zero.  Otherwise, the error code \textbf{MP\_MEM} will be 
returned (line 29).  

The digits are allocated and set to zero at the same time with the calloc() function (line @25,XCALLOC@).  The 
\textbf{used} count is set to zero, the \textbf{alloc} count set to the padded digit count and the \textbf{sign} flag set 
to \textbf{MP\_ZPOS} to achieve a default valid mp\_int state (lines 33, 34 and 35).  If the function 
returns succesfully then it is correct to assume that the mp\_int structure is in a valid state for the remainder of the 
functions to work with.

\subsection{Multiple Integer Initializations and Clearings}
Occasionally a function will require a series of mp\_int data types to be made available simultaneously.  
The purpose of algorithm mp\_init\_multi is to initialize a variable length array of mp\_int structures in a single
statement.  It is essentially a shortcut to multiple initializations.

\newpage\begin{figure}[here]
\begin{center}
\begin{tabular}{l}
\hline Algorithm \textbf{mp\_init\_multi}. \\
\textbf{Input}.   Variable length array $V_k$ of mp\_int variables of length $k$. \\
\textbf{Output}.  The array is initialized such that each mp\_int of $V_k$ is ready to use. \\
\hline \\
1.  for $n$ from 0 to $k - 1$ do \\
\hspace{+3mm}1.1.  Initialize the mp\_int $V_n$ (\textit{mp\_init}) \\
\hspace{+3mm}1.2.  If initialization failed then do \\
\hspace{+6mm}1.2.1.  for $j$ from $0$ to $n$ do \\
\hspace{+9mm}1.2.1.1.  Free the mp\_int $V_j$ (\textit{mp\_clear}) \\
\hspace{+6mm}1.2.2.   Return(\textit{MP\_MEM}) \\
2.  Return(\textit{MP\_OKAY}) \\
\hline
\end{tabular}
\end{center}
\caption{Algorithm mp\_init\_multi}
\end{figure}

\textbf{Algorithm mp\_init\_multi.}
The algorithm will initialize the array of mp\_int variables one at a time.  If a runtime error has been detected 
(\textit{step 1.2}) all of the previously initialized variables are cleared.  The goal is an ``all or nothing'' 
initialization which allows for quick recovery from runtime errors.

\vspace{+3mm}\begin{small}
\hspace{-5.1mm}{\bf File}: bn\_mp\_init\_multi.c
\vspace{-3mm}
\begin{alltt}
\end{alltt}
\end{small}

This function intializes a variable length list of mp\_int structure pointers.  However, instead of having the mp\_int
structures in an actual C array they are simply passed as arguments to the function.  This function makes use of the 
``...'' argument syntax of the C programming language.  The list is terminated with a final \textbf{NULL} argument 
appended on the right.  

The function uses the ``stdarg.h'' \textit{va} functions to step portably through the arguments to the function.  A count
$n$ of succesfully initialized mp\_int structures is maintained (line 48) such that if a failure does occur,
the algorithm can backtrack and free the previously initialized structures (lines 28 to 47).  


\subsection{Clamping Excess Digits}
When a function anticipates a result will be $n$ digits it is simpler to assume this is true within the body of 
the function instead of checking during the computation.  For example, a multiplication of a $i$ digit number by a 
$j$ digit produces a result of at most $i + j$ digits.  It is entirely possible that the result is $i + j - 1$ 
though, with no final carry into the last position.  However, suppose the destination had to be first expanded 
(\textit{via mp\_grow}) to accomodate $i + j - 1$ digits than further expanded to accomodate the final carry.  
That would be a considerable waste of time since heap operations are relatively slow.

The ideal solution is to always assume the result is $i + j$ and fix up the \textbf{used} count after the function
terminates.  This way a single heap operation (\textit{at most}) is required.  However, if the result was not checked
there would be an excess high order zero digit.  

For example, suppose the product of two integers was $x_n = (0x_{n-1}x_{n-2}...x_0)_{\beta}$.  The leading zero digit 
will not contribute to the precision of the result.  In fact, through subsequent operations more leading zero digits would
accumulate to the point the size of the integer would be prohibitive.  As a result even though the precision is very 
low the representation is excessively large.  

The mp\_clamp algorithm is designed to solve this very problem.  It will trim high-order zeros by decrementing the 
\textbf{used} count until a non-zero most significant digit is found.  Also in this system, zero is considered to be a 
positive number which means that if the \textbf{used} count is decremented to zero, the sign must be set to 
\textbf{MP\_ZPOS}.

\begin{figure}[here]
\begin{center}
\begin{tabular}{l}
\hline Algorithm \textbf{mp\_clamp}. \\
\textbf{Input}.   An mp\_int $a$ \\
\textbf{Output}.  Any excess leading zero digits of $a$ are removed \\
\hline \\
1.  while $a.used > 0$ and $a_{a.used - 1} = 0$ do \\
\hspace{+3mm}1.1  $a.used \leftarrow a.used - 1$ \\
2.  if $a.used = 0$ then do \\
\hspace{+3mm}2.1  $a.sign \leftarrow MP\_ZPOS$ \\
\hline \\
\end{tabular}
\end{center}
\caption{Algorithm mp\_clamp}
\end{figure}

\textbf{Algorithm mp\_clamp.}
As can be expected this algorithm is very simple.  The loop on step one is expected to iterate only once or twice at
the most.  For example, this will happen in cases where there is not a carry to fill the last position.  Step two fixes the sign for 
when all of the digits are zero to ensure that the mp\_int is valid at all times.

\vspace{+3mm}\begin{small}
\hspace{-5.1mm}{\bf File}: bn\_mp\_clamp.c
\vspace{-3mm}
\begin{alltt}
\end{alltt}
\end{small}

Note on line 28 how to test for the \textbf{used} count is made on the left of the \&\& operator.  In the C programming
language the terms to \&\& are evaluated left to right with a boolean short-circuit if any condition fails.  This is 
important since if the \textbf{used} is zero the test on the right would fetch below the array.  That is obviously 
undesirable.  The parenthesis on line 31 is used to make sure the \textbf{used} count is decremented and not
the pointer ``a''.  

\section*{Exercises}
\begin{tabular}{cl}
$\left [ 1 \right ]$ & Discuss the relevance of the \textbf{used} member of the mp\_int structure. \\
                     & \\
$\left [ 1 \right ]$ & Discuss the consequences of not using padding when performing allocations.  \\
                     & \\
$\left [ 2 \right ]$ & Estimate an ideal value for \textbf{MP\_PREC} when performing 1024-bit RSA \\
                     & encryption when $\beta = 2^{28}$.  \\
                     & \\
$\left [ 1 \right ]$ & Discuss the relevance of the algorithm mp\_clamp.  What does it prevent? \\
                     & \\
$\left [ 1 \right ]$ & Give an example of when the algorithm  mp\_init\_copy might be useful. \\
                     & \\
\end{tabular}


%%%
% CHAPTER FOUR
%%%

\chapter{Basic Operations}

\section{Introduction}
In the previous chapter a series of low level algorithms were established that dealt with initializing and maintaining
mp\_int structures.  This chapter will discuss another set of seemingly non-algebraic algorithms which will form the low 
level basis of the entire library.  While these algorithm are relatively trivial it is important to understand how they
work before proceeding since these algorithms will be used almost intrinsically in the following chapters.

The algorithms in this chapter deal primarily with more ``programmer'' related tasks such as creating copies of
mp\_int structures, assigning small values to mp\_int structures and comparisons of the values mp\_int structures
represent.   

\section{Assigning Values to mp\_int Structures}
\subsection{Copying an mp\_int}
Assigning the value that a given mp\_int structure represents to another mp\_int structure shall be known as making
a copy for the purposes of this text.  The copy of the mp\_int will be a separate entity that represents the same
value as the mp\_int it was copied from.  The mp\_copy algorithm provides this functionality. 

\newpage\begin{figure}[here]
\begin{center}
\begin{tabular}{l}
\hline Algorithm \textbf{mp\_copy}. \\
\textbf{Input}.  An mp\_int $a$ and $b$. \\
\textbf{Output}.  Store a copy of $a$ in $b$. \\
\hline \\
1.  If $b.alloc < a.used$ then grow $b$ to $a.used$ digits.  (\textit{mp\_grow}) \\
2.  for $n$ from 0 to $a.used - 1$ do \\
\hspace{3mm}2.1  $b_{n} \leftarrow a_{n}$ \\
3.  for $n$ from $a.used$ to $b.used - 1$ do \\
\hspace{3mm}3.1  $b_{n} \leftarrow 0$ \\
4.  $b.used \leftarrow a.used$ \\
5.  $b.sign \leftarrow a.sign$ \\
6.  return(\textit{MP\_OKAY}) \\
\hline
\end{tabular}
\end{center}
\caption{Algorithm mp\_copy}
\end{figure}

\textbf{Algorithm mp\_copy.}
This algorithm copies the mp\_int $a$ such that upon succesful termination of the algorithm the mp\_int $b$ will
represent the same integer as the mp\_int $a$.  The mp\_int $b$ shall be a complete and distinct copy of the 
mp\_int $a$ meaing that the mp\_int $a$ can be modified and it shall not affect the value of the mp\_int $b$.

If $b$ does not have enough room for the digits of $a$ it must first have its precision augmented via the mp\_grow 
algorithm.  The digits of $a$ are copied over the digits of $b$ and any excess digits of $b$ are set to zero (step two
and three).  The \textbf{used} and \textbf{sign} members of $a$ are finally copied over the respective members of
$b$.

\textbf{Remark.}  This algorithm also introduces a new idiosyncrasy that will be used throughout the rest of the
text.  The error return codes of other algorithms are not explicitly checked in the pseudo-code presented.  For example, in 
step one of the mp\_copy algorithm the return of mp\_grow is not explicitly checked to ensure it succeeded.  Text space is 
limited so it is assumed that if a algorithm fails it will clear all temporarily allocated mp\_ints and return
the error code itself.  However, the C code presented will demonstrate all of the error handling logic required to 
implement the pseudo-code.

\vspace{+3mm}\begin{small}
\hspace{-5.1mm}{\bf File}: bn\_mp\_copy.c
\vspace{-3mm}
\begin{alltt}
\end{alltt}
\end{small}

Occasionally a dependent algorithm may copy an mp\_int effectively into itself such as when the input and output
mp\_int structures passed to a function are one and the same.  For this case it is optimal to return immediately without 
copying digits (line 25).  

The mp\_int $b$ must have enough digits to accomodate the used digits of the mp\_int $a$.  If $b.alloc$ is less than
$a.used$ the algorithm mp\_grow is used to augment the precision of $b$ (lines 30 to 33).  In order to
simplify the inner loop that copies the digits from $a$ to $b$, two aliases $tmpa$ and $tmpb$ point directly at the digits
of the mp\_ints $a$ and $b$ respectively.  These aliases (lines 43 and 46) allow the compiler to access the digits without first dereferencing the
mp\_int pointers and then subsequently the pointer to the digits.  

After the aliases are established the digits from $a$ are copied into $b$ (lines 49 to 51) and then the excess 
digits of $b$ are set to zero (lines 54 to 56).  Both ``for'' loops make use of the pointer aliases and in 
fact the alias for $b$ is carried through into the second ``for'' loop to clear the excess digits.  This optimization 
allows the alias to stay in a machine register fairly easy between the two loops.

\textbf{Remarks.}  The use of pointer aliases is an implementation methodology first introduced in this function that will
be used considerably in other functions.  Technically, a pointer alias is simply a short hand alias used to lower the 
number of pointer dereferencing operations required to access data.  For example, a for loop may resemble

\begin{alltt}
for (x = 0; x < 100; x++) \{
    a->num[4]->dp[x] = 0;
\}
\end{alltt}

This could be re-written using aliases as 

\begin{alltt}
mp_digit *tmpa;
a = a->num[4]->dp;
for (x = 0; x < 100; x++) \{
    *a++ = 0;
\}
\end{alltt}

In this case an alias is used to access the 
array of digits within an mp\_int structure directly.  It may seem that a pointer alias is strictly not required 
as a compiler may optimize out the redundant pointer operations.  However, there are two dominant reasons to use aliases.

The first reason is that most compilers will not effectively optimize pointer arithmetic.  For example, some optimizations 
may work for the Microsoft Visual C++ compiler (MSVC) and not for the GNU C Compiler (GCC).  Also some optimizations may 
work for GCC and not MSVC.  As such it is ideal to find a common ground for as many compilers as possible.  Pointer 
aliases optimize the code considerably before the compiler even reads the source code which means the end compiled code 
stands a better chance of being faster.

The second reason is that pointer aliases often can make an algorithm simpler to read.  Consider the first ``for'' 
loop of the function mp\_copy() re-written to not use pointer aliases.

\begin{alltt}
    /* copy all the digits */
    for (n = 0; n < a->used; n++) \{
      b->dp[n] = a->dp[n];
    \}
\end{alltt}

Whether this code is harder to read depends strongly on the individual.  However, it is quantifiably slightly more 
complicated as there are four variables within the statement instead of just two.

\subsubsection{Nested Statements}
Another commonly used technique in the source routines is that certain sections of code are nested.  This is used in
particular with the pointer aliases to highlight code phases.  For example, a Comba multiplier (discussed in chapter six)
will typically have three different phases.  First the temporaries are initialized, then the columns calculated and 
finally the carries are propagated.  In this example the middle column production phase will typically be nested as it
uses temporary variables and aliases the most.

The nesting also simplies the source code as variables that are nested are only valid for their scope.  As a result
the various temporary variables required do not propagate into other sections of code.


\subsection{Creating a Clone}
Another common operation is to make a local temporary copy of an mp\_int argument.  To initialize an mp\_int 
and then copy another existing mp\_int into the newly intialized mp\_int will be known as creating a clone.  This is 
useful within functions that need to modify an argument but do not wish to actually modify the original copy.  The 
mp\_init\_copy algorithm has been designed to help perform this task.

\begin{figure}[here]
\begin{center}
\begin{tabular}{l}
\hline Algorithm \textbf{mp\_init\_copy}. \\
\textbf{Input}.   An mp\_int $a$ and $b$\\
\textbf{Output}.  $a$ is initialized to be a copy of $b$. \\
\hline \\
1.  Init $a$.  (\textit{mp\_init}) \\
2.  Copy $b$ to $a$.  (\textit{mp\_copy}) \\
3.  Return the status of the copy operation. \\
\hline
\end{tabular}
\end{center}
\caption{Algorithm mp\_init\_copy}
\end{figure}

\textbf{Algorithm mp\_init\_copy.}
This algorithm will initialize an mp\_int variable and copy another previously initialized mp\_int variable into it.  As 
such this algorithm will perform two operations in one step.  

\vspace{+3mm}\begin{small}
\hspace{-5.1mm}{\bf File}: bn\_mp\_init\_copy.c
\vspace{-3mm}
\begin{alltt}
\end{alltt}
\end{small}

This will initialize \textbf{a} and make it a verbatim copy of the contents of \textbf{b}.  Note that 
\textbf{a} will have its own memory allocated which means that \textbf{b} may be cleared after the call
and \textbf{a} will be left intact.  

\section{Zeroing an Integer}
Reseting an mp\_int to the default state is a common step in many algorithms.  The mp\_zero algorithm will be the algorithm used to
perform this task.

\begin{figure}[here]
\begin{center}
\begin{tabular}{l}
\hline Algorithm \textbf{mp\_zero}. \\
\textbf{Input}.   An mp\_int $a$ \\
\textbf{Output}.  Zero the contents of $a$ \\
\hline \\
1.  $a.used \leftarrow 0$ \\
2.  $a.sign \leftarrow$ MP\_ZPOS \\
3.  for $n$ from 0 to $a.alloc - 1$ do \\
\hspace{3mm}3.1  $a_n \leftarrow 0$ \\
\hline
\end{tabular}
\end{center}
\caption{Algorithm mp\_zero}
\end{figure}

\textbf{Algorithm mp\_zero.}
This algorithm simply resets a mp\_int to the default state.  

\vspace{+3mm}\begin{small}
\hspace{-5.1mm}{\bf File}: bn\_mp\_zero.c
\vspace{-3mm}
\begin{alltt}
\end{alltt}
\end{small}

After the function is completed, all of the digits are zeroed, the \textbf{used} count is zeroed and the 
\textbf{sign} variable is set to \textbf{MP\_ZPOS}.

\section{Sign Manipulation}
\subsection{Absolute Value}
With the mp\_int representation of an integer, calculating the absolute value is trivial.  The mp\_abs algorithm will compute
the absolute value of an mp\_int.

\begin{figure}[here]
\begin{center}
\begin{tabular}{l}
\hline Algorithm \textbf{mp\_abs}. \\
\textbf{Input}.   An mp\_int $a$ \\
\textbf{Output}.  Computes $b = \vert a \vert$ \\
\hline \\
1.  Copy $a$ to $b$.  (\textit{mp\_copy}) \\
2.  If the copy failed return(\textit{MP\_MEM}). \\
3.  $b.sign \leftarrow MP\_ZPOS$ \\
4.  Return(\textit{MP\_OKAY}) \\
\hline
\end{tabular}
\end{center}
\caption{Algorithm mp\_abs}
\end{figure}

\textbf{Algorithm mp\_abs.}
This algorithm computes the absolute of an mp\_int input.  First it copies $a$ over $b$.  This is an example of an
algorithm where the check in mp\_copy that determines if the source and destination are equal proves useful.  This allows,
for instance, the developer to pass the same mp\_int as the source and destination to this function without addition 
logic to handle it.

\vspace{+3mm}\begin{small}
\hspace{-5.1mm}{\bf File}: bn\_mp\_abs.c
\vspace{-3mm}
\begin{alltt}
\end{alltt}
\end{small}

This fairly trivial algorithm first eliminates non--required duplications (line 28) and then sets the
\textbf{sign} flag to \textbf{MP\_ZPOS}.

\subsection{Integer Negation}
With the mp\_int representation of an integer, calculating the negation is also trivial.  The mp\_neg algorithm will compute
the negative of an mp\_int input.

\begin{figure}[here]
\begin{center}
\begin{tabular}{l}
\hline Algorithm \textbf{mp\_neg}. \\
\textbf{Input}.   An mp\_int $a$ \\
\textbf{Output}.  Computes $b = -a$ \\
\hline \\
1.  Copy $a$ to $b$.  (\textit{mp\_copy}) \\
2.  If the copy failed return(\textit{MP\_MEM}). \\
3.  If $a.used = 0$ then return(\textit{MP\_OKAY}). \\
4.  If $a.sign = MP\_ZPOS$ then do \\
\hspace{3mm}4.1  $b.sign = MP\_NEG$. \\
5.  else do \\
\hspace{3mm}5.1  $b.sign = MP\_ZPOS$. \\
6.  Return(\textit{MP\_OKAY}) \\
\hline
\end{tabular}
\end{center}
\caption{Algorithm mp\_neg}
\end{figure}

\textbf{Algorithm mp\_neg.}
This algorithm computes the negation of an input.  First it copies $a$ over $b$.  If $a$ has no used digits then
the algorithm returns immediately.  Otherwise it flips the sign flag and stores the result in $b$.  Note that if 
$a$ had no digits then it must be positive by definition.  Had step three been omitted then the algorithm would return
zero as negative.

\vspace{+3mm}\begin{small}
\hspace{-5.1mm}{\bf File}: bn\_mp\_neg.c
\vspace{-3mm}
\begin{alltt}
\end{alltt}
\end{small}

Like mp\_abs() this function avoids non--required duplications (line 22) and then sets the sign.  We
have to make sure that only non--zero values get a \textbf{sign} of \textbf{MP\_NEG}.  If the mp\_int is zero
than the \textbf{sign} is hard--coded to \textbf{MP\_ZPOS}.

\section{Small Constants}
\subsection{Setting Small Constants}
Often a mp\_int must be set to a relatively small value such as $1$ or $2$.  For these cases the mp\_set algorithm is useful.

\newpage\begin{figure}[here]
\begin{center}
\begin{tabular}{l}
\hline Algorithm \textbf{mp\_set}. \\
\textbf{Input}.   An mp\_int $a$ and a digit $b$ \\
\textbf{Output}.  Make $a$ equivalent to $b$ \\
\hline \\
1.  Zero $a$ (\textit{mp\_zero}). \\
2.  $a_0 \leftarrow b \mbox{ (mod }\beta\mbox{)}$ \\
3.  $a.used \leftarrow  \left \lbrace \begin{array}{ll}
                              1 &  \mbox{if }a_0 > 0 \\
                              0 &  \mbox{if }a_0 = 0 
                              \end{array} \right .$ \\
\hline                              
\end{tabular}
\end{center}
\caption{Algorithm mp\_set}
\end{figure}

\textbf{Algorithm mp\_set.}
This algorithm sets a mp\_int to a small single digit value.  Step number 1 ensures that the integer is reset to the default state.  The
single digit is set (\textit{modulo $\beta$}) and the \textbf{used} count is adjusted accordingly.

\vspace{+3mm}\begin{small}
\hspace{-5.1mm}{\bf File}: bn\_mp\_set.c
\vspace{-3mm}
\begin{alltt}
\end{alltt}
\end{small}

First we zero (line 21) the mp\_int to make sure that the other members are initialized for a 
small positive constant.  mp\_zero() ensures that the \textbf{sign} is positive and the \textbf{used} count
is zero.  Next we set the digit and reduce it modulo $\beta$ (line 22).  After this step we have to 
check if the resulting digit is zero or not.  If it is not then we set the \textbf{used} count to one, otherwise
to zero.

We can quickly reduce modulo $\beta$ since it is of the form $2^k$ and a quick binary AND operation with 
$2^k - 1$ will perform the same operation.

One important limitation of this function is that it will only set one digit.  The size of a digit is not fixed, meaning source that uses 
this function should take that into account.  Only trivially small constants can be set using this function.

\subsection{Setting Large Constants}
To overcome the limitations of the mp\_set algorithm the mp\_set\_int algorithm is ideal.  It accepts a ``long''
data type as input and will always treat it as a 32-bit integer.

\begin{figure}[here]
\begin{center}
\begin{tabular}{l}
\hline Algorithm \textbf{mp\_set\_int}. \\
\textbf{Input}.   An mp\_int $a$ and a ``long'' integer $b$ \\
\textbf{Output}.  Make $a$ equivalent to $b$ \\
\hline \\
1.  Zero $a$ (\textit{mp\_zero}) \\
2.  for $n$ from 0 to 7 do \\
\hspace{3mm}2.1  $a \leftarrow a \cdot 16$ (\textit{mp\_mul2d}) \\
\hspace{3mm}2.2  $u \leftarrow \lfloor b / 2^{4(7 - n)} \rfloor \mbox{ (mod }16\mbox{)}$\\
\hspace{3mm}2.3  $a_0 \leftarrow a_0 + u$ \\
\hspace{3mm}2.4  $a.used \leftarrow a.used + 1$ \\
3.  Clamp excess used digits (\textit{mp\_clamp}) \\
\hline
\end{tabular}
\end{center}
\caption{Algorithm mp\_set\_int}
\end{figure}

\textbf{Algorithm mp\_set\_int.}
The algorithm performs eight iterations of a simple loop where in each iteration four bits from the source are added to the 
mp\_int.  Step 2.1 will multiply the current result by sixteen making room for four more bits in the less significant positions.  In step 2.2 the
next four bits from the source are extracted and are added to the mp\_int. The \textbf{used} digit count is 
incremented to reflect the addition.  The \textbf{used} digit counter is incremented since if any of the leading digits were zero the mp\_int would have
zero digits used and the newly added four bits would be ignored.

Excess zero digits are trimmed in steps 2.1 and 3 by using higher level algorithms mp\_mul2d and mp\_clamp.

\vspace{+3mm}\begin{small}
\hspace{-5.1mm}{\bf File}: bn\_mp\_set\_int.c
\vspace{-3mm}
\begin{alltt}
\end{alltt}
\end{small}

This function sets four bits of the number at a time to handle all practical \textbf{DIGIT\_BIT} sizes.  The weird
addition on line 39 ensures that the newly added in bits are added to the number of digits.  While it may not 
seem obvious as to why the digit counter does not grow exceedingly large it is because of the shift on line 28 
as well as the  call to mp\_clamp() on line 41.  Both functions will clamp excess leading digits which keeps 
the number of used digits low.

\section{Comparisons}
\subsection{Unsigned Comparisions}
Comparing a multiple precision integer is performed with the exact same algorithm used to compare two decimal numbers.  For example,
to compare $1,234$ to $1,264$ the digits are extracted by their positions.  That is we compare $1 \cdot 10^3 + 2 \cdot 10^2 + 3 \cdot 10^1 + 4 \cdot 10^0$
to $1 \cdot 10^3 + 2 \cdot 10^2 + 6 \cdot 10^1 + 4 \cdot 10^0$ by comparing single digits at a time starting with the highest magnitude 
positions.  If any leading digit of one integer is greater than a digit in the same position of another integer then obviously it must be greater.  

The first comparision routine that will be developed is the unsigned magnitude compare which will perform a comparison based on the digits of two
mp\_int variables alone.  It will ignore the sign of the two inputs.  Such a function is useful when an absolute comparison is required or if the 
signs are known to agree in advance.

To facilitate working with the results of the comparison functions three constants are required.  

\begin{figure}[here]
\begin{center}
\begin{tabular}{|r|l|}
\hline \textbf{Constant} & \textbf{Meaning} \\
\hline \textbf{MP\_GT} & Greater Than \\
\hline \textbf{MP\_EQ} & Equal To \\
\hline \textbf{MP\_LT} & Less Than \\
\hline
\end{tabular}
\end{center}
\caption{Comparison Return Codes}
\end{figure}

\begin{figure}[here]
\begin{center}
\begin{tabular}{l}
\hline Algorithm \textbf{mp\_cmp\_mag}. \\
\textbf{Input}.   Two mp\_ints $a$ and $b$.  \\
\textbf{Output}.  Unsigned comparison results ($a$ to the left of $b$). \\
\hline \\
1.  If $a.used > b.used$ then return(\textit{MP\_GT}) \\
2.  If $a.used < b.used$ then return(\textit{MP\_LT}) \\
3.  for n from $a.used - 1$ to 0 do \\
\hspace{+3mm}3.1  if $a_n > b_n$ then return(\textit{MP\_GT}) \\
\hspace{+3mm}3.2  if $a_n < b_n$ then return(\textit{MP\_LT}) \\
4.  Return(\textit{MP\_EQ}) \\
\hline
\end{tabular}
\end{center}
\caption{Algorithm mp\_cmp\_mag}
\end{figure}

\textbf{Algorithm mp\_cmp\_mag.}
By saying ``$a$ to the left of $b$'' it is meant that the comparison is with respect to $a$, that is if $a$ is greater than $b$ it will return
\textbf{MP\_GT} and similar with respect to when $a = b$ and $a < b$.  The first two steps compare the number of digits used in both $a$ and $b$.  
Obviously if the digit counts differ there would be an imaginary zero digit in the smaller number where the leading digit of the larger number is.  
If both have the same number of digits than the actual digits themselves must be compared starting at the leading digit.  

By step three both inputs must have the same number of digits so its safe to start from either $a.used - 1$ or $b.used - 1$ and count down to
the zero'th digit.  If after all of the digits have been compared, no difference is found, the algorithm returns \textbf{MP\_EQ}.

\vspace{+3mm}\begin{small}
\hspace{-5.1mm}{\bf File}: bn\_mp\_cmp\_mag.c
\vspace{-3mm}
\begin{alltt}
\end{alltt}
\end{small}

The two if statements (lines 25 and 29) compare the number of digits in the two inputs.  These two are 
performed before all of the digits are compared since it is a very cheap test to perform and can potentially save 
considerable time.  The implementation given is also not valid without those two statements.  $b.alloc$ may be 
smaller than $a.used$, meaning that undefined values will be read from $b$ past the end of the array of digits.



\subsection{Signed Comparisons}
Comparing with sign considerations is also fairly critical in several routines (\textit{division for example}).  Based on an unsigned magnitude 
comparison a trivial signed comparison algorithm can be written.

\begin{figure}[here]
\begin{center}
\begin{tabular}{l}
\hline Algorithm \textbf{mp\_cmp}. \\
\textbf{Input}.   Two mp\_ints $a$ and $b$ \\
\textbf{Output}.  Signed Comparison Results ($a$ to the left of $b$) \\
\hline \\
1.  if $a.sign = MP\_NEG$ and $b.sign = MP\_ZPOS$ then return(\textit{MP\_LT}) \\
2.  if $a.sign = MP\_ZPOS$ and $b.sign = MP\_NEG$ then return(\textit{MP\_GT}) \\
3.  if $a.sign = MP\_NEG$ then \\
\hspace{+3mm}3.1  Return the unsigned comparison of $b$ and $a$ (\textit{mp\_cmp\_mag}) \\
4   Otherwise \\
\hspace{+3mm}4.1  Return the unsigned comparison of $a$ and $b$ \\
\hline
\end{tabular}
\end{center}
\caption{Algorithm mp\_cmp}
\end{figure}

\textbf{Algorithm mp\_cmp.}
The first two steps compare the signs of the two inputs.  If the signs do not agree then it can return right away with the appropriate 
comparison code.  When the signs are equal the digits of the inputs must be compared to determine the correct result.  In step 
three the unsigned comparision flips the order of the arguments since they are both negative.  For instance, if $-a > -b$ then 
$\vert a \vert < \vert b \vert$.  Step number four will compare the two when they are both positive.

\vspace{+3mm}\begin{small}
\hspace{-5.1mm}{\bf File}: bn\_mp\_cmp.c
\vspace{-3mm}
\begin{alltt}
\end{alltt}
\end{small}

The two if statements (lines 23 and 24) perform the initial sign comparison.  If the signs are not the equal then which ever
has the positive sign is larger.   The inputs are compared (line 32) based on magnitudes.  If the signs were both 
negative then the unsigned comparison is performed in the opposite direction (line 34).  Otherwise, the signs are assumed to 
be both positive and a forward direction unsigned comparison is performed.

\section*{Exercises}
\begin{tabular}{cl}
$\left [ 2 \right ]$ & Modify algorithm mp\_set\_int to accept as input a variable length array of bits. \\
                     & \\
$\left [ 3 \right ]$ & Give the probability that algorithm mp\_cmp\_mag will have to compare $k$ digits  \\
                     & of two random digits (of equal magnitude) before a difference is found. \\
                     & \\
$\left [ 1 \right ]$ & Suggest a simple method to speed up the implementation of mp\_cmp\_mag based  \\
                     & on the observations made in the previous problem. \\
                     &
\end{tabular}

\chapter{Basic Arithmetic}
\section{Introduction}
At this point algorithms for initialization, clearing, zeroing, copying, comparing and setting small constants have been 
established.  The next logical set of algorithms to develop are addition, subtraction and digit shifting algorithms.  These 
algorithms make use of the lower level algorithms and are the cruicial building block for the multiplication algorithms.  It is very important 
that these algorithms are highly optimized.  On their own they are simple $O(n)$ algorithms but they can be called from higher level algorithms 
which easily places them at $O(n^2)$ or even $O(n^3)$ work levels.  

All of the algorithms within this chapter make use of the logical bit shift operations denoted by $<<$ and $>>$ for left and right 
logical shifts respectively.  A logical shift is analogous to sliding the decimal point of radix-10 representations.  For example, the real 
number $0.9345$ is equivalent to $93.45\%$ which is found by sliding the the decimal two places to the right (\textit{multiplying by $\beta^2 = 10^2$}).  
Algebraically a binary logical shift is equivalent to a division or multiplication by a power of two.  
For example, $a << k = a \cdot 2^k$ while $a >> k = \lfloor a/2^k \rfloor$.

One significant difference between a logical shift and the way decimals are shifted is that digits below the zero'th position are removed
from the number.  For example, consider $1101_2 >> 1$ using decimal notation this would produce $110.1_2$.  However, with a logical shift the 
result is $110_2$.  

\section{Addition and Subtraction}
In common twos complement fixed precision arithmetic negative numbers are easily represented by subtraction from the modulus.  For example, with 32-bit integers
$a - b\mbox{ (mod }2^{32}\mbox{)}$ is the same as $a + (2^{32} - b) \mbox{ (mod }2^{32}\mbox{)}$  since $2^{32} \equiv 0 \mbox{ (mod }2^{32}\mbox{)}$.  
As a result subtraction can be performed with a trivial series of logical operations and an addition.

However, in multiple precision arithmetic negative numbers are not represented in the same way.  Instead a sign flag is used to keep track of the
sign of the integer.  As a result signed addition and subtraction are actually implemented as conditional usage of lower level addition or 
subtraction algorithms with the sign fixed up appropriately.

The lower level algorithms will add or subtract integers without regard to the sign flag.  That is they will add or subtract the magnitude of
the integers respectively.

\subsection{Low Level Addition}
An unsigned addition of multiple precision integers is performed with the same long-hand algorithm used to add decimal numbers.  That is to add the 
trailing digits first and propagate the resulting carry upwards.  Since this is a lower level algorithm the name will have a ``s\_'' prefix.  
Historically that convention stems from the MPI library where ``s\_'' stood for static functions that were hidden from the developer entirely.

\newpage
\begin{figure}[!here]
\begin{center}
\begin{small}
\begin{tabular}{l}
\hline Algorithm \textbf{s\_mp\_add}. \\
\textbf{Input}.   Two mp\_ints $a$ and $b$ \\
\textbf{Output}.  The unsigned addition $c = \vert a \vert + \vert b \vert$. \\
\hline \\
1.  if $a.used > b.used$ then \\
\hspace{+3mm}1.1  $min \leftarrow b.used$ \\
\hspace{+3mm}1.2  $max \leftarrow a.used$ \\
\hspace{+3mm}1.3  $x   \leftarrow a$ \\
2.  else  \\
\hspace{+3mm}2.1  $min \leftarrow a.used$ \\
\hspace{+3mm}2.2  $max \leftarrow b.used$ \\
\hspace{+3mm}2.3  $x   \leftarrow b$ \\
3.  If $c.alloc < max + 1$ then grow $c$ to hold at least $max + 1$ digits (\textit{mp\_grow}) \\
4.  $oldused \leftarrow c.used$ \\
5.  $c.used \leftarrow max + 1$ \\
6.  $u \leftarrow 0$ \\
7.  for $n$ from $0$ to $min - 1$ do \\
\hspace{+3mm}7.1  $c_n \leftarrow a_n + b_n + u$ \\
\hspace{+3mm}7.2  $u \leftarrow c_n >> lg(\beta)$ \\
\hspace{+3mm}7.3  $c_n \leftarrow c_n \mbox{ (mod }\beta\mbox{)}$ \\
8.  if $min \ne max$ then do \\
\hspace{+3mm}8.1  for $n$ from $min$ to $max - 1$ do \\
\hspace{+6mm}8.1.1  $c_n \leftarrow x_n + u$ \\
\hspace{+6mm}8.1.2  $u \leftarrow c_n >> lg(\beta)$ \\
\hspace{+6mm}8.1.3  $c_n \leftarrow c_n \mbox{ (mod }\beta\mbox{)}$ \\
9.  $c_{max} \leftarrow u$ \\
10.  if $olduse > max$ then \\
\hspace{+3mm}10.1  for $n$ from $max + 1$ to $oldused - 1$ do \\
\hspace{+6mm}10.1.1  $c_n \leftarrow 0$ \\
11.  Clamp excess digits in $c$.  (\textit{mp\_clamp}) \\
12.  Return(\textit{MP\_OKAY}) \\
\hline
\end{tabular}
\end{small}
\end{center}
\caption{Algorithm s\_mp\_add}
\end{figure}

\textbf{Algorithm s\_mp\_add.}
This algorithm is loosely based on algorithm 14.7 of HAC \cite[pp. 594]{HAC} but has been extended to allow the inputs to have different magnitudes.  
Coincidentally the description of algorithm A in Knuth \cite[pp. 266]{TAOCPV2} shares the same deficiency as the algorithm from \cite{HAC}.  Even the 
MIX pseudo  machine code presented by Knuth \cite[pp. 266-267]{TAOCPV2} is incapable of handling inputs which are of different magnitudes.

The first thing that has to be accomplished is to sort out which of the two inputs is the largest.  The addition logic
will simply add all of the smallest input to the largest input and store that first part of the result in the
destination.  Then it will apply a simpler addition loop to excess digits of the larger input.

The first two steps will handle sorting the inputs such that $min$ and $max$ hold the digit counts of the two 
inputs.  The variable $x$ will be an mp\_int alias for the largest input or the second input $b$ if they have the
same number of digits.  After the inputs are sorted the destination $c$ is grown as required to accomodate the sum 
of the two inputs.  The original \textbf{used} count of $c$ is copied and set to the new used count.  

At this point the first addition loop will go through as many digit positions that both inputs have.  The carry
variable $\mu$ is set to zero outside the loop.  Inside the loop an ``addition'' step requires three statements to produce
one digit of the summand.  First
two digits from $a$ and $b$ are added together along with the carry $\mu$.  The carry of this step is extracted and stored
in $\mu$ and finally the digit of the result $c_n$ is truncated within the range $0 \le c_n < \beta$.

Now all of the digit positions that both inputs have in common have been exhausted.  If $min \ne max$ then $x$ is an alias
for one of the inputs that has more digits.  A simplified addition loop is then used to essentially copy the remaining digits
and the carry to the destination.

The final carry is stored in $c_{max}$ and digits above $max$ upto $oldused$ are zeroed which completes the addition.


\vspace{+3mm}\begin{small}
\hspace{-5.1mm}{\bf File}: bn\_s\_mp\_add.c
\vspace{-3mm}
\begin{alltt}
\end{alltt}
\end{small}

We first sort (lines 28 to 36) the inputs based on magnitude and determine the $min$ and $max$ variables.
Note that $x$ is a pointer to an mp\_int assigned to the largest input, in effect it is a local alias.  Next we
grow the destination (38 to 42) ensure that it can accomodate the result of the addition. 

Similar to the implementation of mp\_copy this function uses the braced code and local aliases coding style.  The three aliases that are on 
lines 56, 59 and 62 represent the two inputs and destination variables respectively.  These aliases are used to ensure the
compiler does not have to dereference $a$, $b$ or $c$ (respectively) to access the digits of the respective mp\_int.

The initial carry $u$ will be cleared (line 65), note that $u$ is of type mp\_digit which ensures type 
compatibility within the implementation.  The initial addition (line 66 to 75) adds digits from
both inputs until the smallest input runs out of digits.  Similarly the conditional addition loop
(line 81 to 90) adds the remaining digits from the larger of the two inputs.  The addition is finished 
with the final carry being stored in $tmpc$ (line 94).  Note the ``++'' operator within the same expression.
After line 94, $tmpc$ will point to the $c.used$'th digit of the mp\_int $c$.  This is useful
for the next loop (line 97 to 99) which set any old upper digits to zero.

\subsection{Low Level Subtraction}
The low level unsigned subtraction algorithm is very similar to the low level unsigned addition algorithm.  The principle difference is that the
unsigned subtraction algorithm requires the result to be positive.  That is when computing $a - b$ the condition $\vert a \vert \ge \vert b\vert$ must 
be met for this algorithm to function properly.  Keep in mind this low level algorithm is not meant to be used in higher level algorithms directly.  
This algorithm as will be shown can be used to create functional signed addition and subtraction algorithms.


For this algorithm a new variable is required to make the description simpler.  Recall from section 1.3.1 that a mp\_digit must be able to represent
the range $0 \le x < 2\beta$ for the algorithms to work correctly.  However, it is allowable that a mp\_digit represent a larger range of values.  For 
this algorithm we will assume that the variable $\gamma$ represents the number of bits available in a 
mp\_digit (\textit{this implies $2^{\gamma} > \beta$}).  

For example, the default for LibTomMath is to use a ``unsigned long'' for the mp\_digit ``type'' while $\beta = 2^{28}$.  In ISO C an ``unsigned long''
data type must be able to represent $0 \le x < 2^{32}$ meaning that in this case $\gamma \ge 32$.

\newpage\begin{figure}[!here]
\begin{center}
\begin{small}
\begin{tabular}{l}
\hline Algorithm \textbf{s\_mp\_sub}. \\
\textbf{Input}.   Two mp\_ints $a$ and $b$ ($\vert a \vert \ge \vert b \vert$) \\
\textbf{Output}.  The unsigned subtraction $c = \vert a \vert - \vert b \vert$. \\
\hline \\
1.  $min \leftarrow b.used$ \\
2.  $max \leftarrow a.used$ \\
3.  If $c.alloc < max$ then grow $c$ to hold at least $max$ digits.  (\textit{mp\_grow}) \\
4.  $oldused \leftarrow c.used$ \\ 
5.  $c.used \leftarrow max$ \\
6.  $u \leftarrow 0$ \\
7.  for $n$ from $0$ to $min - 1$ do \\
\hspace{3mm}7.1  $c_n \leftarrow a_n - b_n - u$ \\
\hspace{3mm}7.2  $u   \leftarrow c_n >> (\gamma - 1)$ \\
\hspace{3mm}7.3  $c_n \leftarrow c_n \mbox{ (mod }\beta\mbox{)}$ \\
8.  if $min < max$ then do \\
\hspace{3mm}8.1  for $n$ from $min$ to $max - 1$ do \\
\hspace{6mm}8.1.1  $c_n \leftarrow a_n - u$ \\
\hspace{6mm}8.1.2  $u   \leftarrow c_n >> (\gamma - 1)$ \\
\hspace{6mm}8.1.3  $c_n \leftarrow c_n \mbox{ (mod }\beta\mbox{)}$ \\
9. if $oldused > max$ then do \\
\hspace{3mm}9.1  for $n$ from $max$ to $oldused - 1$ do \\
\hspace{6mm}9.1.1  $c_n \leftarrow 0$ \\
10. Clamp excess digits of $c$.  (\textit{mp\_clamp}). \\
11. Return(\textit{MP\_OKAY}). \\
\hline
\end{tabular}
\end{small}
\end{center}
\caption{Algorithm s\_mp\_sub}
\end{figure}

\textbf{Algorithm s\_mp\_sub.}
This algorithm performs the unsigned subtraction of two mp\_int variables under the restriction that the result must be positive.  That is when
passing variables $a$ and $b$ the condition that $\vert a \vert \ge \vert b \vert$ must be met for the algorithm to function correctly.  This
algorithm is loosely based on algorithm 14.9 \cite[pp. 595]{HAC} and is similar to algorithm S in \cite[pp. 267]{TAOCPV2} as well.  As was the case
of the algorithm s\_mp\_add both other references lack discussion concerning various practical details such as when the inputs differ in magnitude.

The initial sorting of the inputs is trivial in this algorithm since $a$ is guaranteed to have at least the same magnitude of $b$.  Steps 1 and 2 
set the $min$ and $max$ variables.  Unlike the addition routine there is guaranteed to be no carry which means that the final result can be at 
most $max$ digits in length as opposed to $max + 1$.  Similar to the addition algorithm the \textbf{used} count of $c$ is copied locally and 
set to the maximal count for the operation.

The subtraction loop that begins on step seven is essentially the same as the addition loop of algorithm s\_mp\_add except single precision 
subtraction is used instead.  Note the use of the $\gamma$ variable to extract the carry (\textit{also known as the borrow}) within the subtraction 
loops.  Under the assumption that two's complement single precision arithmetic is used this will successfully extract the desired carry.  

For example, consider subtracting $0101_2$ from $0100_2$ where $\gamma = 4$ and $\beta = 2$.  The least significant bit will force a carry upwards to 
the third bit which will be set to zero after the borrow.  After the very first bit has been subtracted $4 - 1 \equiv 0011_2$ will remain,  When the 
third bit of $0101_2$ is subtracted from the result it will cause another carry.  In this case though the carry will be forced to propagate all the 
way to the most significant bit.  

Recall that $\beta < 2^{\gamma}$.  This means that if a carry does occur just before the $lg(\beta)$'th bit it will propagate all the way to the most 
significant bit.  Thus, the high order bits of the mp\_digit that are not part of the actual digit will either be all zero, or all one. All that
is needed is a single zero or one bit for the carry.  Therefore a single logical shift right by $\gamma - 1$ positions is sufficient to extract the 
carry.  This method of carry extraction may seem awkward but the reason for it becomes apparent when the implementation is discussed.  

If $b$ has a smaller magnitude than $a$ then step 9 will force the carry and copy operation to propagate through the larger input $a$ into $c$.  Step
10 will ensure that any leading digits of $c$ above the $max$'th position are zeroed.

\vspace{+3mm}\begin{small}
\hspace{-5.1mm}{\bf File}: bn\_s\_mp\_sub.c
\vspace{-3mm}
\begin{alltt}
\end{alltt}
\end{small}

Like low level addition we ``sort'' the inputs.  Except in this case the sorting is hardcoded 
(lines 25 and 26).  In reality the $min$ and $max$ variables are only aliases and are only 
used to make the source code easier to read.  Again the pointer alias optimization is used 
within this algorithm.  The aliases $tmpa$, $tmpb$ and $tmpc$ are initialized
(lines 42, 43 and 44) for $a$, $b$ and $c$ respectively.

The first subtraction loop (lines 47 through 61) subtract digits from both inputs until the smaller of
the two inputs has been exhausted.  As remarked earlier there is an implementation reason for using the ``awkward'' 
method of extracting the carry (line 57).  The traditional method for extracting the carry would be to shift 
by $lg(\beta)$ positions and logically AND the least significant bit.  The AND operation is required because all of 
the bits above the $\lg(\beta)$'th bit will be set to one after a carry occurs from subtraction.  This carry 
extraction requires two relatively cheap operations to extract the carry.  The other method is to simply shift the 
most significant bit to the least significant bit thus extracting the carry with a single cheap operation.  This 
optimization only works on twos compliment machines which is a safe assumption to make.

If $a$ has a larger magnitude than $b$ an additional loop (lines 64 through 73) is required to propagate 
the carry through $a$ and copy the result to $c$.  

\subsection{High Level Addition}
Now that both lower level addition and subtraction algorithms have been established an effective high level signed addition algorithm can be
established.  This high level addition algorithm will be what other algorithms and developers will use to perform addition of mp\_int data 
types.  

Recall from section 5.2 that an mp\_int represents an integer with an unsigned mantissa (\textit{the array of digits}) and a \textbf{sign} 
flag.  A high level addition is actually performed as a series of eight separate cases which can be optimized down to three unique cases.

\begin{figure}[!here]
\begin{center}
\begin{tabular}{l}
\hline Algorithm \textbf{mp\_add}. \\
\textbf{Input}.   Two mp\_ints $a$ and $b$  \\
\textbf{Output}.  The signed addition $c = a + b$. \\
\hline \\
1.  if $a.sign = b.sign$ then do \\
\hspace{3mm}1.1  $c.sign \leftarrow a.sign$  \\
\hspace{3mm}1.2  $c \leftarrow \vert a \vert + \vert b \vert$ (\textit{s\_mp\_add})\\
2.  else do \\
\hspace{3mm}2.1  if $\vert a \vert < \vert b \vert$ then do (\textit{mp\_cmp\_mag})  \\
\hspace{6mm}2.1.1  $c.sign \leftarrow b.sign$ \\
\hspace{6mm}2.1.2  $c \leftarrow \vert b \vert - \vert a \vert$ (\textit{s\_mp\_sub}) \\
\hspace{3mm}2.2  else do \\
\hspace{6mm}2.2.1  $c.sign \leftarrow a.sign$ \\
\hspace{6mm}2.2.2  $c \leftarrow \vert a \vert - \vert b \vert$ \\
3.  Return(\textit{MP\_OKAY}). \\
\hline
\end{tabular}
\end{center}
\caption{Algorithm mp\_add}
\end{figure}

\textbf{Algorithm mp\_add.}
This algorithm performs the signed addition of two mp\_int variables.  There is no reference algorithm to draw upon from 
either \cite{TAOCPV2} or \cite{HAC} since they both only provide unsigned operations.  The algorithm is fairly 
straightforward but restricted since subtraction can only produce positive results.

\begin{figure}[here]
\begin{small}
\begin{center}
\begin{tabular}{|c|c|c|c|c|}
\hline \textbf{Sign of $a$} & \textbf{Sign of $b$} & \textbf{$\vert a \vert > \vert b \vert $} & \textbf{Unsigned Operation} & \textbf{Result Sign Flag} \\
\hline $+$ & $+$ & Yes & $c = a + b$ & $a.sign$ \\
\hline $+$ & $+$ & No  & $c = a + b$ & $a.sign$ \\
\hline $-$ & $-$ & Yes & $c = a + b$ & $a.sign$ \\
\hline $-$ & $-$ & No  & $c = a + b$ & $a.sign$ \\
\hline &&&&\\

\hline $+$ & $-$ & No  & $c = b - a$ & $b.sign$ \\
\hline $-$ & $+$ & No  & $c = b - a$ & $b.sign$ \\

\hline &&&&\\

\hline $+$ & $-$ & Yes & $c = a - b$ & $a.sign$ \\
\hline $-$ & $+$ & Yes & $c = a - b$ & $a.sign$ \\

\hline
\end{tabular}
\end{center}
\end{small}
\caption{Addition Guide Chart}
\label{fig:AddChart}
\end{figure}

Figure~\ref{fig:AddChart} lists all of the eight possible input combinations and is sorted to show that only three 
specific cases need to be handled.  The return code of the unsigned operations at step 1.2, 2.1.2 and 2.2.2 are 
forwarded to step three to check for errors.  This simplifies the description of the algorithm considerably and best 
follows how the implementation actually was achieved.

Also note how the \textbf{sign} is set before the unsigned addition or subtraction is performed.  Recall from the descriptions of algorithms
s\_mp\_add and s\_mp\_sub that the mp\_clamp function is used at the end to trim excess digits.  The mp\_clamp algorithm will set the \textbf{sign}
to \textbf{MP\_ZPOS} when the \textbf{used} digit count reaches zero.

For example, consider performing $-a + a$ with algorithm mp\_add.  By the description of the algorithm the sign is set to \textbf{MP\_NEG} which would
produce a result of $-0$.  However, since the sign is set first then the unsigned addition is performed the subsequent usage of algorithm mp\_clamp 
within algorithm s\_mp\_add will force $-0$ to become $0$.  

\vspace{+3mm}\begin{small}
\hspace{-5.1mm}{\bf File}: bn\_mp\_add.c
\vspace{-3mm}
\begin{alltt}
\end{alltt}
\end{small}

The source code follows the algorithm fairly closely.  The most notable new source code addition is the usage of the $res$ integer variable which
is used to pass result of the unsigned operations forward.  Unlike in the algorithm, the variable $res$ is merely returned as is without
explicitly checking it and returning the constant \textbf{MP\_OKAY}.  The observation is this algorithm will succeed or fail only if the lower
level functions do so.  Returning their return code is sufficient.

\subsection{High Level Subtraction}
The high level signed subtraction algorithm is essentially the same as the high level signed addition algorithm.  

\newpage\begin{figure}[!here]
\begin{center}
\begin{tabular}{l}
\hline Algorithm \textbf{mp\_sub}. \\
\textbf{Input}.   Two mp\_ints $a$ and $b$  \\
\textbf{Output}.  The signed subtraction $c = a - b$. \\
\hline \\
1.  if $a.sign \ne b.sign$ then do \\
\hspace{3mm}1.1  $c.sign \leftarrow a.sign$ \\
\hspace{3mm}1.2  $c \leftarrow \vert a \vert + \vert b \vert$ (\textit{s\_mp\_add}) \\
2.  else do \\
\hspace{3mm}2.1  if $\vert a \vert \ge \vert b \vert$ then do (\textit{mp\_cmp\_mag}) \\
\hspace{6mm}2.1.1  $c.sign \leftarrow a.sign$ \\
\hspace{6mm}2.1.2  $c \leftarrow \vert a \vert  - \vert b \vert$ (\textit{s\_mp\_sub}) \\
\hspace{3mm}2.2  else do \\
\hspace{6mm}2.2.1  $c.sign \leftarrow  \left \lbrace \begin{array}{ll}
                              MP\_ZPOS &  \mbox{if }a.sign = MP\_NEG \\
                              MP\_NEG  &  \mbox{otherwise} \\
                              \end{array} \right .$ \\
\hspace{6mm}2.2.2  $c \leftarrow \vert b \vert  - \vert a \vert$ \\
3.  Return(\textit{MP\_OKAY}). \\
\hline
\end{tabular}
\end{center}
\caption{Algorithm mp\_sub}
\end{figure}

\textbf{Algorithm mp\_sub.}
This algorithm performs the signed subtraction of two inputs.  Similar to algorithm mp\_add there is no reference in either \cite{TAOCPV2} or 
\cite{HAC}.  Also this algorithm is restricted by algorithm s\_mp\_sub.  Chart \ref{fig:SubChart} lists the eight possible inputs and
the operations required.

\begin{figure}[!here]
\begin{small}
\begin{center}
\begin{tabular}{|c|c|c|c|c|}
\hline \textbf{Sign of $a$} & \textbf{Sign of $b$} & \textbf{$\vert a \vert \ge \vert b \vert $} & \textbf{Unsigned Operation} & \textbf{Result Sign Flag} \\
\hline $+$ & $-$ & Yes & $c = a + b$ & $a.sign$ \\
\hline $+$ & $-$ & No  & $c = a + b$ & $a.sign$ \\
\hline $-$ & $+$ & Yes & $c = a + b$ & $a.sign$ \\
\hline $-$ & $+$ & No  & $c = a + b$ & $a.sign$ \\
\hline &&&& \\
\hline $+$ & $+$ & Yes & $c = a - b$ & $a.sign$ \\
\hline $-$ & $-$ & Yes & $c = a - b$ & $a.sign$ \\
\hline &&&& \\
\hline $+$ & $+$ & No  & $c = b - a$ & $\mbox{opposite of }a.sign$ \\
\hline $-$ & $-$ & No  & $c = b - a$ & $\mbox{opposite of }a.sign$ \\
\hline
\end{tabular}
\end{center}
\end{small}
\caption{Subtraction Guide Chart}
\label{fig:SubChart}
\end{figure}

Similar to the case of algorithm mp\_add the \textbf{sign} is set first before the unsigned addition or subtraction.  That is to prevent the 
algorithm from producing $-a - -a = -0$ as a result.  

\vspace{+3mm}\begin{small}
\hspace{-5.1mm}{\bf File}: bn\_mp\_sub.c
\vspace{-3mm}
\begin{alltt}
\end{alltt}
\end{small}

Much like the implementation of algorithm mp\_add the variable $res$ is used to catch the return code of the unsigned addition or subtraction operations
and forward it to the end of the function.  On line 39 the ``not equal to'' \textbf{MP\_LT} expression is used to emulate a 
``greater than or equal to'' comparison.  

\section{Bit and Digit Shifting}
It is quite common to think of a multiple precision integer as a polynomial in $x$, that is $y = f(\beta)$ where $f(x) = \sum_{i=0}^{n-1} a_i x^i$.  
This notation arises within discussion of Montgomery and Diminished Radix Reduction as well as Karatsuba multiplication and squaring.  

In order to facilitate operations on polynomials in $x$ as above a series of simple ``digit'' algorithms have to be established.  That is to shift
the digits left or right as well to shift individual bits of the digits left and right.  It is important to note that not all ``shift'' operations
are on radix-$\beta$ digits.  

\subsection{Multiplication by Two}

In a binary system where the radix is a power of two multiplication by two not only arises often in other algorithms it is a fairly efficient 
operation to perform.  A single precision logical shift left is sufficient to multiply a single digit by two.  

\newpage\begin{figure}[!here]
\begin{small}
\begin{center}
\begin{tabular}{l}
\hline Algorithm \textbf{mp\_mul\_2}. \\
\textbf{Input}.   One mp\_int $a$ \\
\textbf{Output}.  $b = 2a$. \\
\hline \\
1.  If $b.alloc < a.used + 1$ then grow $b$ to hold $a.used + 1$ digits.  (\textit{mp\_grow}) \\
2.  $oldused \leftarrow b.used$ \\
3.  $b.used \leftarrow a.used$ \\
4.  $r \leftarrow 0$ \\
5.  for $n$ from 0 to $a.used - 1$ do \\
\hspace{3mm}5.1  $rr \leftarrow a_n >> (lg(\beta) - 1)$ \\
\hspace{3mm}5.2  $b_n \leftarrow (a_n << 1) + r \mbox{ (mod }\beta\mbox{)}$ \\
\hspace{3mm}5.3  $r \leftarrow rr$ \\
6.  If $r \ne 0$ then do \\
\hspace{3mm}6.1  $b_{n + 1} \leftarrow r$ \\
\hspace{3mm}6.2  $b.used \leftarrow b.used + 1$ \\
7.  If $b.used < oldused - 1$ then do \\
\hspace{3mm}7.1  for $n$ from $b.used$ to $oldused - 1$ do \\
\hspace{6mm}7.1.1  $b_n \leftarrow 0$ \\
8.  $b.sign \leftarrow a.sign$ \\
9.  Return(\textit{MP\_OKAY}).\\
\hline
\end{tabular}
\end{center}
\end{small}
\caption{Algorithm mp\_mul\_2}
\end{figure}

\textbf{Algorithm mp\_mul\_2.}
This algorithm will quickly multiply a mp\_int by two provided $\beta$ is a power of two.  Neither \cite{TAOCPV2} nor \cite{HAC} describe such 
an algorithm despite the fact it arises often in other algorithms.  The algorithm is setup much like the lower level algorithm s\_mp\_add since 
it is for all intents and purposes equivalent to the operation $b = \vert a \vert + \vert a \vert$.  

Step 1 and 2 grow the input as required to accomodate the maximum number of \textbf{used} digits in the result.  The initial \textbf{used} count
is set to $a.used$ at step 4.  Only if there is a final carry will the \textbf{used} count require adjustment.

Step 6 is an optimization implementation of the addition loop for this specific case.  That is since the two values being added together 
are the same there is no need to perform two reads from the digits of $a$.  Step 6.1 performs a single precision shift on the current digit $a_n$ to
obtain what will be the carry for the next iteration.  Step 6.2 calculates the $n$'th digit of the result as single precision shift of $a_n$ plus
the previous carry.  Recall from section 4.1 that $a_n << 1$ is equivalent to $a_n \cdot 2$.  An iteration of the addition loop is finished with 
forwarding the carry to the next iteration.

Step 7 takes care of any final carry by setting the $a.used$'th digit of the result to the carry and augmenting the \textbf{used} count of $b$.  
Step 8 clears any leading digits of $b$ in case it originally had a larger magnitude than $a$.

\vspace{+3mm}\begin{small}
\hspace{-5.1mm}{\bf File}: bn\_mp\_mul\_2.c
\vspace{-3mm}
\begin{alltt}
\end{alltt}
\end{small}

This implementation is essentially an optimized implementation of s\_mp\_add for the case of doubling an input.  The only noteworthy difference
is the use of the logical shift operator on line 52 to perform a single precision doubling.  

\subsection{Division by Two}
A division by two can just as easily be accomplished with a logical shift right as multiplication by two can be with a logical shift left.

\newpage\begin{figure}[!here]
\begin{small}
\begin{center}
\begin{tabular}{l}
\hline Algorithm \textbf{mp\_div\_2}. \\
\textbf{Input}.   One mp\_int $a$ \\
\textbf{Output}.  $b = a/2$. \\
\hline \\
1.  If $b.alloc < a.used$ then grow $b$ to hold $a.used$ digits.  (\textit{mp\_grow}) \\
2.  If the reallocation failed return(\textit{MP\_MEM}). \\
3.  $oldused \leftarrow b.used$ \\
4.  $b.used \leftarrow a.used$ \\
5.  $r \leftarrow 0$ \\
6.  for $n$ from $b.used - 1$ to $0$ do \\
\hspace{3mm}6.1  $rr \leftarrow a_n \mbox{ (mod }2\mbox{)}$\\
\hspace{3mm}6.2  $b_n \leftarrow (a_n >> 1) + (r << (lg(\beta) - 1)) \mbox{ (mod }\beta\mbox{)}$ \\
\hspace{3mm}6.3  $r \leftarrow rr$ \\
7.  If $b.used < oldused - 1$ then do \\
\hspace{3mm}7.1  for $n$ from $b.used$ to $oldused - 1$ do \\
\hspace{6mm}7.1.1  $b_n \leftarrow 0$ \\
8.  $b.sign \leftarrow a.sign$ \\
9.  Clamp excess digits of $b$.  (\textit{mp\_clamp}) \\
10.  Return(\textit{MP\_OKAY}).\\
\hline
\end{tabular}
\end{center}
\end{small}
\caption{Algorithm mp\_div\_2}
\end{figure}

\textbf{Algorithm mp\_div\_2.}
This algorithm will divide an mp\_int by two using logical shifts to the right.  Like mp\_mul\_2 it uses a modified low level addition
core as the basis of the algorithm.  Unlike mp\_mul\_2 the shift operations work from the leading digit to the trailing digit.  The algorithm
could be written to work from the trailing digit to the leading digit however, it would have to stop one short of $a.used - 1$ digits to prevent
reading past the end of the array of digits.

Essentially the loop at step 6 is similar to that of mp\_mul\_2 except the logical shifts go in the opposite direction and the carry is at the 
least significant bit not the most significant bit.  

\vspace{+3mm}\begin{small}
\hspace{-5.1mm}{\bf File}: bn\_mp\_div\_2.c
\vspace{-3mm}
\begin{alltt}
\end{alltt}
\end{small}

\section{Polynomial Basis Operations}
Recall from section 4.3 that any integer can be represented as a polynomial in $x$ as $y = f(\beta)$.  Such a representation is also known as
the polynomial basis \cite[pp. 48]{ROSE}. Given such a notation a multiplication or division by $x$ amounts to shifting whole digits a single 
place.  The need for such operations arises in several other higher level algorithms such as Barrett and Montgomery reduction, integer
division and Karatsuba multiplication.  

Converting from an array of digits to polynomial basis is very simple.  Consider the integer $y \equiv (a_2, a_1, a_0)_{\beta}$ and recall that
$y = \sum_{i=0}^{2} a_i \beta^i$.  Simply replace $\beta$ with $x$ and the expression is in polynomial basis.  For example, $f(x) = 8x + 9$ is the
polynomial basis representation for $89$ using radix ten.  That is, $f(10) = 8(10) + 9 = 89$.  

\subsection{Multiplication by $x$}

Given a polynomial in $x$ such as $f(x) = a_n x^n + a_{n-1} x^{n-1} + ... + a_0$ multiplying by $x$ amounts to shifting the coefficients up one 
degree.  In this case $f(x) \cdot x = a_n x^{n+1} + a_{n-1} x^n + ... + a_0 x$.  From a scalar basis point of view multiplying by $x$ is equivalent to
multiplying by the integer $\beta$.  

\newpage\begin{figure}[!here]
\begin{small}
\begin{center}
\begin{tabular}{l}
\hline Algorithm \textbf{mp\_lshd}. \\
\textbf{Input}.   One mp\_int $a$ and an integer $b$ \\
\textbf{Output}.  $a \leftarrow a \cdot \beta^b$ (equivalent to multiplication by $x^b$). \\
\hline \\
1.  If $b \le 0$ then return(\textit{MP\_OKAY}). \\
2.  If $a.alloc < a.used + b$ then grow $a$ to at least $a.used + b$ digits.  (\textit{mp\_grow}). \\
3.  If the reallocation failed return(\textit{MP\_MEM}). \\
4.  $a.used \leftarrow a.used + b$ \\
5.  $i \leftarrow a.used - 1$ \\
6.  $j \leftarrow a.used - 1 - b$ \\
7.  for $n$ from $a.used - 1$ to $b$ do \\
\hspace{3mm}7.1  $a_{i} \leftarrow a_{j}$ \\
\hspace{3mm}7.2  $i \leftarrow i - 1$ \\
\hspace{3mm}7.3  $j \leftarrow j - 1$ \\
8.  for $n$ from 0 to $b - 1$ do \\
\hspace{3mm}8.1  $a_n \leftarrow 0$ \\
9.  Return(\textit{MP\_OKAY}). \\
\hline
\end{tabular}
\end{center}
\end{small}
\caption{Algorithm mp\_lshd}
\end{figure}

\textbf{Algorithm mp\_lshd.}
This algorithm multiplies an mp\_int by the $b$'th power of $x$.  This is equivalent to multiplying by $\beta^b$.  The algorithm differs 
from the other algorithms presented so far as it performs the operation in place instead storing the result in a separate location.  The
motivation behind this change is due to the way this function is typically used.  Algorithms such as mp\_add store the result in an optionally
different third mp\_int because the original inputs are often still required.  Algorithm mp\_lshd (\textit{and similarly algorithm mp\_rshd}) is
typically used on values where the original value is no longer required.  The algorithm will return success immediately if 
$b \le 0$ since the rest of algorithm is only valid when $b > 0$.  

First the destination $a$ is grown as required to accomodate the result.  The counters $i$ and $j$ are used to form a \textit{sliding window} over
the digits of $a$ of length $b$.  The head of the sliding window is at $i$ (\textit{the leading digit}) and the tail at $j$ (\textit{the trailing digit}).  
The loop on step 7 copies the digit from the tail to the head.  In each iteration the window is moved down one digit.   The last loop on 
step 8 sets the lower $b$ digits to zero.

\newpage
\begin{center}
\begin{figure}[here]
\includegraphics{pics/sliding_window.ps}
\caption{Sliding Window Movement}
\label{pic:sliding_window}
\end{figure}
\end{center}

\vspace{+3mm}\begin{small}
\hspace{-5.1mm}{\bf File}: bn\_mp\_lshd.c
\vspace{-3mm}
\begin{alltt}
\end{alltt}
\end{small}

The if statement (line 24) ensures that the $b$ variable is greater than zero since we do not interpret negative
shift counts properly.  The \textbf{used} count is incremented by $b$ before the copy loop begins.  This elminates 
the need for an additional variable in the for loop.  The variable $top$ (line 42) is an alias
for the leading digit while $bottom$ (line 45) is an alias for the trailing edge.  The aliases form a 
window of exactly $b$ digits over the input.  

\subsection{Division by $x$}

Division by powers of $x$ is easily achieved by shifting the digits right and removing any that will end up to the right of the zero'th digit.  

\newpage\begin{figure}[!here]
\begin{small}
\begin{center}
\begin{tabular}{l}
\hline Algorithm \textbf{mp\_rshd}. \\
\textbf{Input}.   One mp\_int $a$ and an integer $b$ \\
\textbf{Output}.  $a \leftarrow a / \beta^b$ (Divide by $x^b$). \\
\hline \\
1.  If $b \le 0$ then return. \\
2.  If $a.used \le b$ then do \\
\hspace{3mm}2.1  Zero $a$.  (\textit{mp\_zero}). \\
\hspace{3mm}2.2  Return. \\
3.  $i \leftarrow 0$ \\
4.  $j \leftarrow b$ \\
5.  for $n$ from 0 to $a.used - b - 1$ do \\
\hspace{3mm}5.1  $a_i \leftarrow a_j$ \\
\hspace{3mm}5.2  $i \leftarrow i + 1$ \\
\hspace{3mm}5.3  $j \leftarrow j + 1$ \\
6.  for $n$ from $a.used - b$ to $a.used - 1$ do \\
\hspace{3mm}6.1  $a_n \leftarrow 0$ \\
7.  $a.used \leftarrow a.used - b$ \\
8.  Return. \\
\hline
\end{tabular}
\end{center}
\end{small}
\caption{Algorithm mp\_rshd}
\end{figure}

\textbf{Algorithm mp\_rshd.}
This algorithm divides the input in place by the $b$'th power of $x$.  It is analogous to dividing by a $\beta^b$ but much quicker since
it does not require single precision division.  This algorithm does not actually return an error code as it cannot fail.  

If the input $b$ is less than one the algorithm quickly returns without performing any work.  If the \textbf{used} count is less than or equal
to the shift count $b$ then it will simply zero the input and return.

After the trivial cases of inputs have been handled the sliding window is setup.  Much like the case of algorithm mp\_lshd a sliding window that
is $b$ digits wide is used to copy the digits.  Unlike mp\_lshd the window slides in the opposite direction from the trailing to the leading digit.  
Also the digits are copied from the leading to the trailing edge.

Once the window copy is complete the upper digits must be zeroed and the \textbf{used} count decremented.

\vspace{+3mm}\begin{small}
\hspace{-5.1mm}{\bf File}: bn\_mp\_rshd.c
\vspace{-3mm}
\begin{alltt}
\end{alltt}
\end{small}

The only noteworthy element of this routine is the lack of a return type since it cannot fail.  Like mp\_lshd() we
form a sliding window except we copy in the other direction.  After the window (line 60) we then zero
the upper digits of the input to make sure the result is correct.

\section{Powers of Two}

Now that algorithms for moving single bits as well as whole digits exist algorithms for moving the ``in between'' distances are required.  For 
example, to quickly multiply by $2^k$ for any $k$ without using a full multiplier algorithm would prove useful.  Instead of performing single
shifts $k$ times to achieve a multiplication by $2^{\pm k}$ a mixture of whole digit shifting and partial digit shifting is employed.  

\subsection{Multiplication by Power of Two}

\newpage\begin{figure}[!here]
\begin{small}
\begin{center}
\begin{tabular}{l}
\hline Algorithm \textbf{mp\_mul\_2d}. \\
\textbf{Input}.   One mp\_int $a$ and an integer $b$ \\
\textbf{Output}.  $c \leftarrow a \cdot 2^b$. \\
\hline \\
1.  $c \leftarrow a$.  (\textit{mp\_copy}) \\
2.  If $c.alloc < c.used + \lfloor b / lg(\beta) \rfloor + 2$ then grow $c$ accordingly. \\
3.  If the reallocation failed return(\textit{MP\_MEM}). \\
4.  If $b \ge lg(\beta)$ then \\
\hspace{3mm}4.1  $c \leftarrow c \cdot \beta^{\lfloor b / lg(\beta) \rfloor}$ (\textit{mp\_lshd}). \\
\hspace{3mm}4.2  If step 4.1 failed return(\textit{MP\_MEM}). \\
5.  $d \leftarrow b \mbox{ (mod }lg(\beta)\mbox{)}$ \\
6.  If $d \ne 0$ then do \\
\hspace{3mm}6.1  $mask \leftarrow 2^d$ \\
\hspace{3mm}6.2  $r \leftarrow 0$ \\
\hspace{3mm}6.3  for $n$ from $0$ to $c.used - 1$ do \\
\hspace{6mm}6.3.1  $rr \leftarrow c_n >> (lg(\beta) - d) \mbox{ (mod }mask\mbox{)}$ \\
\hspace{6mm}6.3.2  $c_n \leftarrow (c_n << d) + r \mbox{ (mod }\beta\mbox{)}$ \\
\hspace{6mm}6.3.3  $r \leftarrow rr$ \\
\hspace{3mm}6.4  If $r > 0$ then do \\
\hspace{6mm}6.4.1  $c_{c.used} \leftarrow r$ \\
\hspace{6mm}6.4.2  $c.used \leftarrow c.used + 1$ \\
7.  Return(\textit{MP\_OKAY}). \\
\hline
\end{tabular}
\end{center}
\end{small}
\caption{Algorithm mp\_mul\_2d}
\end{figure}

\textbf{Algorithm mp\_mul\_2d.}
This algorithm multiplies $a$ by $2^b$ and stores the result in $c$.  The algorithm uses algorithm mp\_lshd and a derivative of algorithm mp\_mul\_2 to
quickly compute the product.

First the algorithm will multiply $a$ by $x^{\lfloor b / lg(\beta) \rfloor}$ which will ensure that the remainder multiplicand is less than 
$\beta$.  For example, if $b = 37$ and $\beta = 2^{28}$ then this step will multiply by $x$ leaving a multiplication by $2^{37 - 28} = 2^{9}$ 
left.

After the digits have been shifted appropriately at most $lg(\beta) - 1$ shifts are left to perform.  Step 5 calculates the number of remaining shifts 
required.  If it is non-zero a modified shift loop is used to calculate the remaining product.  
Essentially the loop is a generic version of algorithm mp\_mul\_2 designed to handle any shift count in the range $1 \le x < lg(\beta)$.  The $mask$
variable is used to extract the upper $d$ bits to form the carry for the next iteration.  

This algorithm is loosely measured as a $O(2n)$ algorithm which means that if the input is $n$-digits that it takes $2n$ ``time'' to 
complete.  It is possible to optimize this algorithm down to a $O(n)$ algorithm at a cost of making the algorithm slightly harder to follow.

\vspace{+3mm}\begin{small}
\hspace{-5.1mm}{\bf File}: bn\_mp\_mul\_2d.c
\vspace{-3mm}
\begin{alltt}
\end{alltt}
\end{small}

The shifting is performed in--place which means the first step (line 25) is to copy the input to the 
destination.  We avoid calling mp\_copy() by making sure the mp\_ints are different.  The destination then
has to be grown (line 32) to accomodate the result.

If the shift count $b$ is larger than $lg(\beta)$ then a call to mp\_lshd() is used to handle all of the multiples 
of $lg(\beta)$.  Leaving only a remaining shift of $lg(\beta) - 1$ or fewer bits left.  Inside the actual shift 
loop (lines 46 to 76) we make use of pre--computed values $shift$ and $mask$.   These are used to
extract the carry bit(s) to pass into the next iteration of the loop.  The $r$ and $rr$ variables form a 
chain between consecutive iterations to propagate the carry.  

\subsection{Division by Power of Two}

\newpage\begin{figure}[!here]
\begin{small}
\begin{center}
\begin{tabular}{l}
\hline Algorithm \textbf{mp\_div\_2d}. \\
\textbf{Input}.   One mp\_int $a$ and an integer $b$ \\
\textbf{Output}.  $c \leftarrow \lfloor a / 2^b \rfloor, d \leftarrow a \mbox{ (mod }2^b\mbox{)}$. \\
\hline \\
1.  If $b \le 0$ then do \\
\hspace{3mm}1.1  $c \leftarrow a$ (\textit{mp\_copy}) \\
\hspace{3mm}1.2  $d \leftarrow 0$ (\textit{mp\_zero}) \\
\hspace{3mm}1.3  Return(\textit{MP\_OKAY}). \\
2.  $c \leftarrow a$ \\
3.  $d \leftarrow a \mbox{ (mod }2^b\mbox{)}$ (\textit{mp\_mod\_2d}) \\
4.  If $b \ge lg(\beta)$ then do \\
\hspace{3mm}4.1  $c \leftarrow \lfloor c/\beta^{\lfloor b/lg(\beta) \rfloor} \rfloor$ (\textit{mp\_rshd}). \\
5.  $k \leftarrow b \mbox{ (mod }lg(\beta)\mbox{)}$ \\
6.  If $k \ne 0$ then do \\
\hspace{3mm}6.1  $mask \leftarrow 2^k$ \\
\hspace{3mm}6.2  $r \leftarrow 0$ \\
\hspace{3mm}6.3  for $n$ from $c.used - 1$ to $0$ do \\
\hspace{6mm}6.3.1  $rr \leftarrow c_n \mbox{ (mod }mask\mbox{)}$ \\
\hspace{6mm}6.3.2  $c_n \leftarrow (c_n >> k) + (r << (lg(\beta) - k))$ \\
\hspace{6mm}6.3.3  $r \leftarrow rr$ \\
7.  Clamp excess digits of $c$.  (\textit{mp\_clamp}) \\
8.  Return(\textit{MP\_OKAY}). \\
\hline
\end{tabular}
\end{center}
\end{small}
\caption{Algorithm mp\_div\_2d}
\end{figure}

\textbf{Algorithm mp\_div\_2d.}
This algorithm will divide an input $a$ by $2^b$ and produce the quotient and remainder.  The algorithm is designed much like algorithm 
mp\_mul\_2d by first using whole digit shifts then single precision shifts.  This algorithm will also produce the remainder of the division
by using algorithm mp\_mod\_2d.

\vspace{+3mm}\begin{small}
\hspace{-5.1mm}{\bf File}: bn\_mp\_div\_2d.c
\vspace{-3mm}
\begin{alltt}
\end{alltt}
\end{small}

The implementation of algorithm mp\_div\_2d is slightly different than the algorithm specifies.  The remainder $d$ may be optionally 
ignored by passing \textbf{NULL} as the pointer to the mp\_int variable.    The temporary mp\_int variable $t$ is used to hold the 
result of the remainder operation until the end.  This allows $d$ and $a$ to represent the same mp\_int without modifying $a$ before
the quotient is obtained.

The remainder of the source code is essentially the same as the source code for mp\_mul\_2d.  The only significant difference is
the direction of the shifts.

\subsection{Remainder of Division by Power of Two}

The last algorithm in the series of polynomial basis power of two algorithms is calculating the remainder of division by $2^b$.  This
algorithm benefits from the fact that in twos complement arithmetic $a \mbox{ (mod }2^b\mbox{)}$ is the same as $a$ AND $2^b - 1$.  

\begin{figure}[!here]
\begin{small}
\begin{center}
\begin{tabular}{l}
\hline Algorithm \textbf{mp\_mod\_2d}. \\
\textbf{Input}.   One mp\_int $a$ and an integer $b$ \\
\textbf{Output}.  $c \leftarrow a \mbox{ (mod }2^b\mbox{)}$. \\
\hline \\
1.  If $b \le 0$ then do \\
\hspace{3mm}1.1  $c \leftarrow 0$ (\textit{mp\_zero}) \\
\hspace{3mm}1.2  Return(\textit{MP\_OKAY}). \\
2.  If $b > a.used \cdot lg(\beta)$ then do \\
\hspace{3mm}2.1  $c \leftarrow a$ (\textit{mp\_copy}) \\
\hspace{3mm}2.2  Return the result of step 2.1. \\
3.  $c \leftarrow a$ \\
4.  If step 3 failed return(\textit{MP\_MEM}). \\
5.  for $n$ from $\lceil b / lg(\beta) \rceil$ to $c.used$ do \\
\hspace{3mm}5.1  $c_n \leftarrow 0$ \\
6.  $k \leftarrow b \mbox{ (mod }lg(\beta)\mbox{)}$ \\
7.  $c_{\lfloor b / lg(\beta) \rfloor} \leftarrow c_{\lfloor b / lg(\beta) \rfloor} \mbox{ (mod }2^{k}\mbox{)}$. \\
8.  Clamp excess digits of $c$.  (\textit{mp\_clamp}) \\
9.  Return(\textit{MP\_OKAY}). \\
\hline
\end{tabular}
\end{center}
\end{small}
\caption{Algorithm mp\_mod\_2d}
\end{figure}

\textbf{Algorithm mp\_mod\_2d.}
This algorithm will quickly calculate the value of $a \mbox{ (mod }2^b\mbox{)}$.  First if $b$ is less than or equal to zero the 
result is set to zero.  If $b$ is greater than the number of bits in $a$ then it simply copies $a$ to $c$ and returns.  Otherwise, $a$ 
is copied to $b$, leading digits are removed and the remaining leading digit is trimed to the exact bit count.

\vspace{+3mm}\begin{small}
\hspace{-5.1mm}{\bf File}: bn\_mp\_mod\_2d.c
\vspace{-3mm}
\begin{alltt}
\end{alltt}
\end{small}

We first avoid cases of $b \le 0$ by simply mp\_zero()'ing the destination in such cases.  Next if $2^b$ is larger
than the input we just mp\_copy() the input and return right away.  After this point we know we must actually
perform some work to produce the remainder.

Recalling that reducing modulo $2^k$ and a binary ``and'' with $2^k - 1$ are numerically equivalent we can quickly reduce 
the number.  First we zero any digits above the last digit in $2^b$ (line 42).  Next we reduce the 
leading digit of both (line 46) and then mp\_clamp().

\section*{Exercises}
\begin{tabular}{cl}
$\left [ 3 \right ] $ & Devise an algorithm that performs $a \cdot 2^b$ for generic values of $b$ \\
                      & in $O(n)$ time. \\
                      &\\
$\left [ 3 \right ] $ & Devise an efficient algorithm to multiply by small low hamming  \\
                      & weight values such as $3$, $5$ and $9$.  Extend it to handle all values \\
                      & upto $64$ with a hamming weight less than three. \\
                      &\\
$\left [ 2 \right ] $ & Modify the preceding algorithm to handle values of the form \\
                      & $2^k - 1$ as well. \\
                      &\\
$\left [ 3 \right ] $ & Using only algorithms mp\_mul\_2, mp\_div\_2 and mp\_add create an \\
                      & algorithm to multiply two integers in roughly $O(2n^2)$ time for \\
                      & any $n$-bit input.  Note that the time of addition is ignored in the \\
                      & calculation.  \\
                      & \\
$\left [ 5 \right ] $ & Improve the previous algorithm to have a working time of at most \\
                      & $O \left (2^{(k-1)}n + \left ({2n^2 \over k} \right ) \right )$ for an appropriate choice of $k$.  Again ignore \\
                      & the cost of addition. \\
                      & \\
$\left [ 2 \right ] $ & Devise a chart to find optimal values of $k$ for the previous problem \\
                      & for $n = 64 \ldots 1024$ in steps of $64$. \\
                      & \\
$\left [ 2 \right ] $ & Using only algorithms mp\_abs and mp\_sub devise another method for \\
                      & calculating the result of a signed comparison. \\
                      &
\end{tabular}

\chapter{Multiplication and Squaring}
\section{The Multipliers}
For most number theoretic problems including certain public key cryptographic algorithms, the ``multipliers'' form the most important subset of 
algorithms of any multiple precision integer package.  The set of multiplier algorithms include integer multiplication, squaring and modular reduction 
where in each of the algorithms single precision multiplication is the dominant operation performed.  This chapter will discuss integer multiplication 
and squaring, leaving modular reductions for the subsequent chapter.  

The importance of the multiplier algorithms is for the most part driven by the fact that certain popular public key algorithms are based on modular 
exponentiation, that is computing $d \equiv a^b \mbox{ (mod }c\mbox{)}$ for some arbitrary choice of $a$, $b$, $c$ and $d$.  During a modular
exponentiation the majority\footnote{Roughly speaking a modular exponentiation will spend about 40\% of the time performing modular reductions, 
35\% of the time performing squaring and 25\% of the time performing multiplications.} of the processor time is spent performing single precision 
multiplications.

For centuries general purpose multiplication has required a lengthly $O(n^2)$ process, whereby each digit of one multiplicand has to be multiplied 
against every digit of the other multiplicand.  Traditional long-hand multiplication is based on this process;  while the techniques can differ the 
overall algorithm used is essentially the same.  Only ``recently'' have faster algorithms been studied.  First Karatsuba multiplication was discovered in 
1962.  This algorithm can multiply two numbers with considerably fewer single precision multiplications when compared to the long-hand approach.  
This technique led to the discovery of polynomial basis algorithms (\textit{good reference?}) and subquently Fourier Transform based solutions.  

\section{Multiplication}
\subsection{The Baseline Multiplication}
\label{sec:basemult}
\index{baseline multiplication}
Computing the product of two integers in software can be achieved using a trivial adaptation of the standard $O(n^2)$ long-hand multiplication
algorithm that school children are taught.  The algorithm is considered an $O(n^2)$ algorithm since for two $n$-digit inputs $n^2$ single precision 
multiplications are required.  More specifically for a $m$ and $n$ digit input $m \cdot n$ single precision multiplications are required.  To 
simplify most discussions, it will be assumed that the inputs have comparable number of digits.  

The ``baseline multiplication'' algorithm is designed to act as the ``catch-all'' algorithm, only to be used when the faster algorithms cannot be 
used.  This algorithm does not use any particularly interesting optimizations and should ideally be avoided if possible.    One important 
facet of this algorithm, is that it has been modified to only produce a certain amount of output digits as resolution.  The importance of this 
modification will become evident during the discussion of Barrett modular reduction.  Recall that for a $n$ and $m$ digit input the product 
will be at most $n + m$ digits.  Therefore, this algorithm can be reduced to a full multiplier by having it produce $n + m$ digits of the product.  

Recall from sub-section 4.2.2 the definition of $\gamma$ as the number of bits in the type \textbf{mp\_digit}.  We shall now extend the variable set to 
include $\alpha$ which shall represent the number of bits in the type \textbf{mp\_word}.  This implies that $2^{\alpha} > 2 \cdot \beta^2$.  The 
constant $\delta = 2^{\alpha - 2lg(\beta)}$ will represent the maximal weight of any column in a product (\textit{see sub-section 5.2.2 for more information}).

\newpage\begin{figure}[!here]
\begin{small}
\begin{center}
\begin{tabular}{l}
\hline Algorithm \textbf{s\_mp\_mul\_digs}. \\
\textbf{Input}.   mp\_int $a$, mp\_int $b$ and an integer $digs$ \\
\textbf{Output}.  $c \leftarrow \vert a \vert \cdot \vert b \vert \mbox{ (mod }\beta^{digs}\mbox{)}$. \\
\hline \\
1.  If min$(a.used, b.used) < \delta$ then do \\
\hspace{3mm}1.1  Calculate $c = \vert a \vert \cdot \vert b \vert$ by the Comba method (\textit{see algorithm~\ref{fig:COMBAMULT}}).  \\
\hspace{3mm}1.2  Return the result of step 1.1 \\
\\
Allocate and initialize a temporary mp\_int. \\
2.  Init $t$ to be of size $digs$ \\
3.  If step 2 failed return(\textit{MP\_MEM}). \\
4.  $t.used \leftarrow digs$ \\
\\
Compute the product. \\
5.  for $ix$ from $0$ to $a.used - 1$ do \\
\hspace{3mm}5.1  $u \leftarrow 0$ \\
\hspace{3mm}5.2  $pb \leftarrow \mbox{min}(b.used, digs - ix)$ \\
\hspace{3mm}5.3  If $pb < 1$ then goto step 6. \\
\hspace{3mm}5.4  for $iy$ from $0$ to $pb - 1$ do \\
\hspace{6mm}5.4.1  $\hat r \leftarrow t_{iy + ix} + a_{ix} \cdot b_{iy} + u$ \\
\hspace{6mm}5.4.2  $t_{iy + ix} \leftarrow \hat r \mbox{ (mod }\beta\mbox{)}$ \\
\hspace{6mm}5.4.3  $u \leftarrow \lfloor \hat r / \beta \rfloor$ \\
\hspace{3mm}5.5  if $ix + pb < digs$ then do \\
\hspace{6mm}5.5.1  $t_{ix + pb} \leftarrow u$ \\
6.  Clamp excess digits of $t$. \\
7.  Swap $c$ with $t$ \\
8.  Clear $t$ \\
9.  Return(\textit{MP\_OKAY}). \\
\hline
\end{tabular}
\end{center}
\end{small}
\caption{Algorithm s\_mp\_mul\_digs}
\end{figure}

\textbf{Algorithm s\_mp\_mul\_digs.}
This algorithm computes the unsigned product of two inputs $a$ and $b$, limited to an output precision of $digs$ digits.  While it may seem
a bit awkward to modify the function from its simple $O(n^2)$ description, the usefulness of partial multipliers will arise in a subsequent 
algorithm.  The algorithm is loosely based on algorithm 14.12 from \cite[pp. 595]{HAC} and is similar to Algorithm M of Knuth \cite[pp. 268]{TAOCPV2}.  
Algorithm s\_mp\_mul\_digs differs from these cited references since it can produce a variable output precision regardless of the precision of the 
inputs.

The first thing this algorithm checks for is whether a Comba multiplier can be used instead.   If the minimum digit count of either
input is less than $\delta$, then the Comba method may be used instead.    After the Comba method is ruled out, the baseline algorithm begins.  A 
temporary mp\_int variable $t$ is used to hold the intermediate result of the product.  This allows the algorithm to be used to 
compute products when either $a = c$ or $b = c$ without overwriting the inputs.  

All of step 5 is the infamous $O(n^2)$ multiplication loop slightly modified to only produce upto $digs$ digits of output.  The $pb$ variable
is given the count of digits to read from $b$ inside the nested loop.  If $pb \le 1$ then no more output digits can be produced and the algorithm
will exit the loop.  The best way to think of the loops are as a series of $pb \times 1$ multiplications.    That is, in each pass of the 
innermost loop $a_{ix}$ is multiplied against $b$ and the result is added (\textit{with an appropriate shift}) to $t$.  

For example, consider multiplying $576$ by $241$.  That is equivalent to computing $10^0(1)(576) + 10^1(4)(576) + 10^2(2)(576)$ which is best
visualized in the following table.

\begin{figure}[here]
\begin{center}
\begin{tabular}{|c|c|c|c|c|c|l|}
\hline   &&          & 5 & 7 & 6 & \\
\hline   $\times$&&  & 2 & 4 & 1 & \\
\hline &&&&&&\\
  &&          & 5 & 7 & 6 & $10^0(1)(576)$ \\
  &2 &   3    & 6 & 1 & 6 & $10^1(4)(576) + 10^0(1)(576)$ \\
  1 & 3 & 8 & 8 & 1 & 6 &   $10^2(2)(576) + 10^1(4)(576) + 10^0(1)(576)$ \\
\hline  
\end{tabular}
\end{center}
\caption{Long-Hand Multiplication Diagram}
\end{figure}

Each row of the product is added to the result after being shifted to the left (\textit{multiplied by a power of the radix}) by the appropriate 
count.  That is in pass $ix$ of the inner loop the product is added starting at the $ix$'th digit of the reult.

Step 5.4.1 introduces the hat symbol (\textit{e.g. $\hat r$}) which represents a double precision variable.  The multiplication on that step
is assumed to be a double wide output single precision multiplication.  That is, two single precision variables are multiplied to produce a
double precision result.  The step is somewhat optimized from a long-hand multiplication algorithm because the carry from the addition in step
5.4.1 is propagated through the nested loop.  If the carry was not propagated immediately it would overflow the single precision digit 
$t_{ix+iy}$ and the result would be lost.  

At step 5.5 the nested loop is finished and any carry that was left over should be forwarded.  The carry does not have to be added to the $ix+pb$'th
digit since that digit is assumed to be zero at this point.  However, if $ix + pb \ge digs$ the carry is not set as it would make the result
exceed the precision requested.

\vspace{+3mm}\begin{small}
\hspace{-5.1mm}{\bf File}: bn\_s\_mp\_mul\_digs.c
\vspace{-3mm}
\begin{alltt}
\end{alltt}
\end{small}

First we determine (line 31) if the Comba method can be used first since it's faster.  The conditions for 
sing the Comba routine are that min$(a.used, b.used) < \delta$ and the number of digits of output is less than 
\textbf{MP\_WARRAY}.  This new constant is used to control the stack usage in the Comba routines.  By default it is 
set to $\delta$ but can be reduced when memory is at a premium.

If we cannot use the Comba method we proceed to setup the baseline routine.  We allocate the the destination mp\_int
$t$ (line 37) to the exact size of the output to avoid further re--allocations.  At this point we now 
begin the $O(n^2)$ loop.

This implementation of multiplication has the caveat that it can be trimmed to only produce a variable number of
digits as output.  In each iteration of the outer loop the $pb$ variable is set (line 49) to the maximum 
number of inner loop iterations.  

Inside the inner loop we calculate $\hat r$ as the mp\_word product of the two mp\_digits and the addition of the
carry from the previous iteration.  A particularly important observation is that most modern optimizing 
C compilers (GCC for instance) can recognize that a $N \times N \rightarrow 2N$ multiplication is all that 
is required for the product.  In x86 terms for example, this means using the MUL instruction.

Each digit of the product is stored in turn (line 69) and the carry propagated (line 72) to the 
next iteration.

\subsection{Faster Multiplication by the ``Comba'' Method}

One of the huge drawbacks of the ``baseline'' algorithms is that at the $O(n^2)$ level the carry must be 
computed and propagated upwards.  This makes the nested loop very sequential and hard to unroll and implement 
in parallel.  The ``Comba'' \cite{COMBA} method is named after little known (\textit{in cryptographic venues}) Paul G. 
Comba who described a method of implementing fast multipliers that do not require nested carry fixup operations.  As an 
interesting aside it seems that Paul Barrett describes a similar technique in his 1986 paper \cite{BARRETT} written 
five years before.

At the heart of the Comba technique is once again the long-hand algorithm.  Except in this case a slight 
twist is placed on how the columns of the result are produced.  In the standard long-hand algorithm rows of products 
are produced then added together to form the final result.  In the baseline algorithm the columns are added together 
after each iteration to get the result instantaneously.  

In the Comba algorithm the columns of the result are produced entirely independently of each other.  That is at 
the $O(n^2)$ level a simple multiplication and addition step is performed.  The carries of the columns are propagated 
after the nested loop to reduce the amount of work requiored. Succintly the first step of the algorithm is to compute 
the product vector $\vec x$ as follows. 

\begin{equation}
\vec x_n = \sum_{i+j = n} a_ib_j, \forall n \in \lbrace 0, 1, 2, \ldots, i + j \rbrace
\end{equation}

Where $\vec x_n$ is the $n'th$ column of the output vector.  Consider the following example which computes the vector $\vec x$ for the multiplication
of $576$ and $241$.  

\newpage\begin{figure}[here]
\begin{small}
\begin{center}
\begin{tabular}{|c|c|c|c|c|c|}
  \hline &          & 5 & 7 & 6 & First Input\\
  \hline $\times$ & & 2 & 4 & 1 & Second Input\\
\hline            &                        & $1 \cdot 5 = 5$   & $1 \cdot 7 = 7$   & $1 \cdot 6 = 6$ & First pass \\
                  &  $4 \cdot 5 = 20$      & $4 \cdot 7+5=33$  & $4 \cdot 6+7=31$  & 6               & Second pass \\
   $2 \cdot 5 = 10$ &  $2 \cdot 7 + 20 = 34$ & $2 \cdot 6+33=45$ & 31                & 6             & Third pass \\
\hline 10 & 34 & 45 & 31 & 6 & Final Result \\   
\hline   
\end{tabular}
\end{center}
\end{small}
\caption{Comba Multiplication Diagram}
\end{figure}

At this point the vector $x = \left < 10, 34, 45, 31, 6 \right >$ is the result of the first step of the Comba multipler.  
Now the columns must be fixed by propagating the carry upwards.  The resultant vector will have one extra dimension over the input vector which is
congruent to adding a leading zero digit.

\begin{figure}[!here]
\begin{small}
\begin{center}
\begin{tabular}{l}
\hline Algorithm \textbf{Comba Fixup}. \\
\textbf{Input}.   Vector $\vec x$ of dimension $k$ \\
\textbf{Output}.  Vector $\vec x$ such that the carries have been propagated. \\
\hline \\
1.  for $n$ from $0$ to $k - 1$ do \\
\hspace{3mm}1.1 $\vec x_{n+1} \leftarrow \vec x_{n+1} + \lfloor \vec x_{n}/\beta \rfloor$ \\
\hspace{3mm}1.2 $\vec x_{n} \leftarrow \vec x_{n} \mbox{ (mod }\beta\mbox{)}$ \\
2.  Return($\vec x$). \\
\hline
\end{tabular}
\end{center}
\end{small}
\caption{Algorithm Comba Fixup}
\end{figure}

With that algorithm and $k = 5$ and $\beta = 10$ the following vector is produced $\vec x= \left < 1, 3, 8, 8, 1, 6 \right >$.  In this case 
$241 \cdot 576$ is in fact $138816$ and the procedure succeeded.  If the algorithm is correct and as will be demonstrated shortly more
efficient than the baseline algorithm why not simply always use this algorithm?

\subsubsection{Column Weight.}
At the nested $O(n^2)$ level the Comba method adds the product of two single precision variables to each column of the output 
independently.  A serious obstacle is if the carry is lost, due to lack of precision before the algorithm has a chance to fix
the carries.  For example, in the multiplication of two three-digit numbers the third column of output will be the sum of
three single precision multiplications.  If the precision of the accumulator for the output digits is less then $3 \cdot (\beta - 1)^2$ then
an overflow can occur and the carry information will be lost.  For any $m$ and $n$ digit inputs the maximum weight of any column is 
min$(m, n)$ which is fairly obvious.

The maximum number of terms in any column of a product is known as the ``column weight'' and strictly governs when the algorithm can be used.  Recall
from earlier that a double precision type has $\alpha$ bits of resolution and a single precision digit has $lg(\beta)$ bits of precision.  Given these
two quantities we must not violate the following

\begin{equation}
k \cdot \left (\beta - 1 \right )^2 < 2^{\alpha}
\end{equation}

Which reduces to 

\begin{equation}
k \cdot \left ( \beta^2 - 2\beta + 1 \right ) < 2^{\alpha}
\end{equation}

Let $\rho = lg(\beta)$ represent the number of bits in a single precision digit.  By further re-arrangement of the equation the final solution is
found.

\begin{equation}
k  < {{2^{\alpha}} \over {\left (2^{2\rho} - 2^{\rho + 1} + 1 \right )}}
\end{equation}

The defaults for LibTomMath are $\beta = 2^{28}$ and $\alpha = 2^{64}$ which means that $k$ is bounded by $k < 257$.  In this configuration 
the smaller input may not have more than $256$ digits if the Comba method is to be used.  This is quite satisfactory for most applications since 
$256$ digits would allow for numbers in the range of $0 \le x < 2^{7168}$ which, is much larger than most public key cryptographic algorithms require.  

\newpage\begin{figure}[!here]
\begin{small}
\begin{center}
\begin{tabular}{l}
\hline Algorithm \textbf{fast\_s\_mp\_mul\_digs}. \\
\textbf{Input}.   mp\_int $a$, mp\_int $b$ and an integer $digs$ \\
\textbf{Output}.  $c \leftarrow \vert a \vert \cdot \vert b \vert \mbox{ (mod }\beta^{digs}\mbox{)}$. \\
\hline \\
Place an array of \textbf{MP\_WARRAY} single precision digits named $W$ on the stack. \\
1.  If $c.alloc < digs$ then grow $c$ to $digs$ digits. (\textit{mp\_grow}) \\
2.  If step 1 failed return(\textit{MP\_MEM}).\\
\\
3.  $pa \leftarrow \mbox{MIN}(digs, a.used + b.used)$ \\
\\
4.  $\_ \hat W \leftarrow 0$ \\
5.  for $ix$ from 0 to $pa - 1$ do \\
\hspace{3mm}5.1  $ty \leftarrow \mbox{MIN}(b.used - 1, ix)$ \\
\hspace{3mm}5.2  $tx \leftarrow ix - ty$ \\
\hspace{3mm}5.3  $iy \leftarrow \mbox{MIN}(a.used - tx, ty + 1)$ \\
\hspace{3mm}5.4  for $iz$ from 0 to $iy - 1$ do \\
\hspace{6mm}5.4.1  $\_ \hat W \leftarrow \_ \hat W + a_{tx+iy}b_{ty-iy}$ \\
\hspace{3mm}5.5  $W_{ix} \leftarrow \_ \hat W (\mbox{mod }\beta)$\\
\hspace{3mm}5.6  $\_ \hat W \leftarrow \lfloor \_ \hat W / \beta \rfloor$ \\
\\
6.  $oldused \leftarrow c.used$ \\
7.  $c.used \leftarrow digs$ \\
8.  for $ix$ from $0$ to $pa$ do \\
\hspace{3mm}8.1  $c_{ix} \leftarrow W_{ix}$ \\
9.  for $ix$ from $pa + 1$ to $oldused - 1$ do \\
\hspace{3mm}9.1 $c_{ix} \leftarrow 0$ \\
\\
10.  Clamp $c$. \\
11.  Return MP\_OKAY. \\
\hline
\end{tabular}
\end{center}
\end{small}
\caption{Algorithm fast\_s\_mp\_mul\_digs}
\label{fig:COMBAMULT}
\end{figure}

\textbf{Algorithm fast\_s\_mp\_mul\_digs.}
This algorithm performs the unsigned multiplication of $a$ and $b$ using the Comba method limited to $digs$ digits of precision.

The outer loop of this algorithm is more complicated than that of the baseline multiplier.  This is because on the inside of the 
loop we want to produce one column per pass.  This allows the accumulator $\_ \hat W$ to be placed in CPU registers and
reduce the memory bandwidth to two \textbf{mp\_digit} reads per iteration.

The $ty$ variable is set to the minimum count of $ix$ or the number of digits in $b$.  That way if $a$ has more digits than
$b$ this will be limited to $b.used - 1$.  The $tx$ variable is set to the to the distance past $b.used$ the variable
$ix$ is.  This is used for the immediately subsequent statement where we find $iy$.  

The variable $iy$ is the minimum digits we can read from either $a$ or $b$ before running out.  Computing one column at a time
means we have to scan one integer upwards and the other downwards.  $a$ starts at $tx$ and $b$ starts at $ty$.  In each
pass we are producing the $ix$'th output column and we note that $tx + ty = ix$.  As we move $tx$ upwards we have to 
move $ty$ downards so the equality remains valid.  The $iy$ variable is the number of iterations until 
$tx \ge a.used$ or $ty < 0$ occurs.

After every inner pass we store the lower half of the accumulator into $W_{ix}$ and then propagate the carry of the accumulator
into the next round by dividing $\_ \hat W$ by $\beta$.

To measure the benefits of the Comba method over the baseline method consider the number of operations that are required.  If the 
cost in terms of time of a multiply and addition is $p$ and the cost of a carry propagation is $q$ then a baseline multiplication would require 
$O \left ((p + q)n^2 \right )$ time to multiply two $n$-digit numbers.  The Comba method requires only $O(pn^2 + qn)$ time, however in practice, 
the speed increase is actually much more.  With $O(n)$ space the algorithm can be reduced to $O(pn + qn)$ time by implementing the $n$ multiply
and addition operations in the nested loop in parallel.  

\vspace{+3mm}\begin{small}
\hspace{-5.1mm}{\bf File}: bn\_fast\_s\_mp\_mul\_digs.c
\vspace{-3mm}
\begin{alltt}
\end{alltt}
\end{small}

As per the pseudo--code we first calculate $pa$ (line 48) as the number of digits to output.  Next we begin the outer loop
to produce the individual columns of the product.  We use the two aliases $tmpx$ and $tmpy$ (lines 62, 63) to point
inside the two multiplicands quickly.  

The inner loop (lines 71 to 74) of this implementation is where the tradeoff come into play.  Originally this comba 
implementation was ``row--major'' which means it adds to each of the columns in each pass.  After the outer loop it would then fix 
the carries.  This was very fast except it had an annoying drawback.  You had to read a mp\_word and two mp\_digits and write 
one mp\_word per iteration.  On processors such as the Athlon XP and P4 this did not matter much since the cache bandwidth 
is very high and it can keep the ALU fed with data.  It did, however, matter on older and embedded cpus where cache is often 
slower and also often doesn't exist.  This new algorithm only performs two reads per iteration under the assumption that the 
compiler has aliased $\_ \hat W$ to a CPU register.

After the inner loop we store the current accumulator in $W$ and shift $\_ \hat W$ (lines 77, 80) to forward it as 
a carry for the next pass.  After the outer loop we use the final carry (line 77) as the last digit of the product.  

\subsection{Polynomial Basis Multiplication}
To break the $O(n^2)$ barrier in multiplication requires a completely different look at integer multiplication.  In the following algorithms
the use of polynomial basis representation for two integers $a$ and $b$ as $f(x) = \sum_{i=0}^{n} a_i x^i$ and  
$g(x) = \sum_{i=0}^{n} b_i x^i$ respectively, is required.  In this system both $f(x)$ and $g(x)$ have $n + 1$ terms and are of the $n$'th degree.
 
The product $a \cdot b \equiv f(x)g(x)$ is the polynomial $W(x) = \sum_{i=0}^{2n} w_i x^i$.  The coefficients $w_i$ will
directly yield the desired product when $\beta$ is substituted for $x$.  The direct solution to solve for the $2n + 1$ coefficients
requires $O(n^2)$ time and would in practice be slower than the Comba technique.

However, numerical analysis theory indicates that only $2n + 1$ distinct points in $W(x)$ are required to determine the values of the $2n + 1$ unknown 
coefficients.   This means by finding $\zeta_y = W(y)$ for $2n + 1$ small values of $y$ the coefficients of $W(x)$ can be found with 
Gaussian elimination.  This technique is also occasionally refered to as the \textit{interpolation technique} (\textit{references please...}) since in 
effect an interpolation based on $2n + 1$ points will yield a polynomial equivalent to $W(x)$.  

The coefficients of the polynomial $W(x)$ are unknown which makes finding $W(y)$ for any value of $y$ impossible.  However, since 
$W(x) = f(x)g(x)$ the equivalent $\zeta_y = f(y) g(y)$ can be used in its place.  The benefit of this technique stems from the 
fact that $f(y)$ and $g(y)$ are much smaller than either $a$ or $b$ respectively.  As a result finding the $2n + 1$ relations required 
by multiplying $f(y)g(y)$ involves multiplying integers that are much smaller than either of the inputs.

When picking points to gather relations there are always three obvious points to choose, $y = 0, 1$ and $ \infty$.  The $\zeta_0$ term
is simply the product $W(0) = w_0 = a_0 \cdot b_0$.  The $\zeta_1$ term is the product 
$W(1) = \left (\sum_{i = 0}^{n} a_i \right ) \left (\sum_{i = 0}^{n} b_i \right )$.  The third point $\zeta_{\infty}$ is less obvious but rather
simple to explain.  The $2n + 1$'th coefficient of $W(x)$ is numerically equivalent to the most significant column in an integer multiplication.  
The point at $\infty$ is used symbolically to represent the most significant column, that is $W(\infty) = w_{2n} = a_nb_n$.  Note that the 
points at $y = 0$ and $\infty$ yield the coefficients $w_0$ and $w_{2n}$ directly.

If more points are required they should be of small values and powers of two such as $2^q$ and the related \textit{mirror points} 
$\left (2^q \right )^{2n}  \cdot \zeta_{2^{-q}}$ for small values of $q$.  The term ``mirror point'' stems from the fact that 
$\left (2^q \right )^{2n}  \cdot \zeta_{2^{-q}}$ can be calculated in the exact opposite fashion as $\zeta_{2^q}$.  For
example, when $n = 2$ and $q = 1$ then following two equations are equivalent to the point $\zeta_{2}$ and its mirror.

\begin{eqnarray}
\zeta_{2}                  = f(2)g(2) = (4a_2 + 2a_1 + a_0)(4b_2 + 2b_1 + b_0) \nonumber \\
16 \cdot \zeta_{1 \over 2} = 4f({1\over 2}) \cdot 4g({1 \over 2}) = (a_2 + 2a_1 + 4a_0)(b_2 + 2b_1 + 4b_0)
\end{eqnarray}

Using such points will allow the values of $f(y)$ and $g(y)$ to be independently calculated using only left shifts.  For example, when $n = 2$ the
polynomial $f(2^q)$ is equal to $2^q((2^qa_2) + a_1) + a_0$.  This technique of polynomial representation is known as Horner's method.  

As a general rule of the algorithm when the inputs are split into $n$ parts each there are $2n - 1$ multiplications.  Each multiplication is of 
multiplicands that have $n$ times fewer digits than the inputs.  The asymptotic running time of this algorithm is 
$O \left ( k^{lg_n(2n - 1)} \right )$ for $k$ digit inputs (\textit{assuming they have the same number of digits}).  Figure~\ref{fig:exponent}
summarizes the exponents for various values of $n$.

\begin{figure}
\begin{center}
\begin{tabular}{|c|c|c|}
\hline \textbf{Split into $n$ Parts} & \textbf{Exponent}  & \textbf{Notes}\\
\hline $2$ & $1.584962501$ & This is Karatsuba Multiplication. \\
\hline $3$ & $1.464973520$ & This is Toom-Cook Multiplication. \\
\hline $4$ & $1.403677461$ &\\
\hline $5$ & $1.365212389$ &\\
\hline $10$ & $1.278753601$ &\\
\hline $100$ & $1.149426538$ &\\
\hline $1000$ & $1.100270931$ &\\
\hline $10000$ & $1.075252070$ &\\
\hline
\end{tabular}
\end{center}
\caption{Asymptotic Running Time of Polynomial Basis Multiplication}
\label{fig:exponent}
\end{figure}

At first it may seem like a good idea to choose $n = 1000$ since the exponent is approximately $1.1$.  However, the overhead
of solving for the 2001 terms of $W(x)$ will certainly consume any savings the algorithm could offer for all but exceedingly large
numbers.  

\subsubsection{Cutoff Point}
The polynomial basis multiplication algorithms all require fewer single precision multiplications than a straight Comba approach.  However, 
the algorithms incur an overhead (\textit{at the $O(n)$ work level}) since they require a system of equations to be solved.  This makes the
polynomial basis approach more costly to use with small inputs.

Let $m$ represent the number of digits in the multiplicands (\textit{assume both multiplicands have the same number of digits}).  There exists a 
point $y$ such that when $m < y$ the polynomial basis algorithms are more costly than Comba, when $m = y$ they are roughly the same cost and 
when $m > y$ the Comba methods are slower than the polynomial basis algorithms.  

The exact location of $y$ depends on several key architectural elements of the computer platform in question.

\begin{enumerate}
\item  The ratio of clock cycles for single precision multiplication versus other simpler operations such as addition, shifting, etc.  For example
on the AMD Athlon the ratio is roughly $17 : 1$ while on the Intel P4 it is $29 : 1$.  The higher the ratio in favour of multiplication the lower
the cutoff point $y$ will be.  

\item  The complexity of the linear system of equations (\textit{for the coefficients of $W(x)$}) is.  Generally speaking as the number of splits
grows the complexity grows substantially.  Ideally solving the system will only involve addition, subtraction and shifting of integers.  This
directly reflects on the ratio previous mentioned.

\item  To a lesser extent memory bandwidth and function call overheads.  Provided the values are in the processor cache this is less of an
influence over the cutoff point.

\end{enumerate}

A clean cutoff point separation occurs when a point $y$ is found such that all of the cutoff point conditions are met.  For example, if the point
is too low then there will be values of $m$ such that $m > y$ and the Comba method is still faster.  Finding the cutoff points is fairly simple when
a high resolution timer is available.  

\subsection{Karatsuba Multiplication}
Karatsuba \cite{KARA} multiplication when originally proposed in 1962 was among the first set of algorithms to break the $O(n^2)$ barrier for
general purpose multiplication.  Given two polynomial basis representations $f(x) = ax + b$ and $g(x) = cx + d$, Karatsuba proved with 
light algebra \cite{KARAP} that the following polynomial is equivalent to multiplication of the two integers the polynomials represent.

\begin{equation}
f(x) \cdot g(x) = acx^2 + ((a + b)(c + d) - (ac + bd))x + bd
\end{equation}

Using the observation that $ac$ and $bd$ could be re-used only three half sized multiplications would be required to produce the product.  Applying
this algorithm recursively, the work factor becomes $O(n^{lg(3)})$ which is substantially better than the work factor $O(n^2)$ of the Comba technique.  It turns 
out what Karatsuba did not know or at least did not publish was that this is simply polynomial basis multiplication with the points 
$\zeta_0$, $\zeta_{\infty}$ and $\zeta_{1}$.  Consider the resultant system of equations.

\begin{center}
\begin{tabular}{rcrcrcrc}
$\zeta_{0}$ &      $=$ &  &  &  & & $w_0$ \\
$\zeta_{1}$ &      $=$ & $w_2$ & $+$ & $w_1$ & $+$ & $w_0$ \\
$\zeta_{\infty}$ & $=$ & $w_2$ &  & &  & \\
\end{tabular}
\end{center}

By adding the first and last equation to the equation in the middle the term $w_1$ can be isolated and all three coefficients solved for.  The simplicity
of this system of equations has made Karatsuba fairly popular.  In fact the cutoff point is often fairly low\footnote{With LibTomMath 0.18 it is 70 and 109 digits for the Intel P4 and AMD Athlon respectively.}
making it an ideal algorithm to speed up certain public key cryptosystems such as RSA and Diffie-Hellman.  

\newpage\begin{figure}[!here]
\begin{small}
\begin{center}
\begin{tabular}{l}
\hline Algorithm \textbf{mp\_karatsuba\_mul}. \\
\textbf{Input}.   mp\_int $a$ and mp\_int $b$ \\
\textbf{Output}.  $c \leftarrow \vert a \vert \cdot \vert b \vert$ \\
\hline \\
1.  Init the following mp\_int variables: $x0$, $x1$, $y0$, $y1$, $t1$, $x0y0$, $x1y1$.\\
2.  If step 2 failed then return(\textit{MP\_MEM}). \\
\\
Split the input.  e.g. $a = x1 \cdot \beta^B + x0$ \\
3.  $B \leftarrow \mbox{min}(a.used, b.used)/2$ \\
4.  $x0 \leftarrow a \mbox{ (mod }\beta^B\mbox{)}$ (\textit{mp\_mod\_2d}) \\
5.  $y0 \leftarrow b \mbox{ (mod }\beta^B\mbox{)}$ \\
6.  $x1 \leftarrow \lfloor a / \beta^B \rfloor$ (\textit{mp\_rshd}) \\
7.  $y1 \leftarrow \lfloor b / \beta^B \rfloor$ \\
\\
Calculate the three products. \\
8.  $x0y0 \leftarrow x0 \cdot y0$ (\textit{mp\_mul}) \\
9.  $x1y1 \leftarrow x1 \cdot y1$ \\
10.  $t1 \leftarrow x1 + x0$ (\textit{mp\_add}) \\
11.  $x0 \leftarrow y1 + y0$ \\
12.  $t1 \leftarrow t1 \cdot x0$ \\
\\
Calculate the middle term. \\
13.  $x0 \leftarrow x0y0 + x1y1$ \\
14.  $t1 \leftarrow t1 - x0$ (\textit{s\_mp\_sub}) \\
\\
Calculate the final product. \\
15.  $t1 \leftarrow t1 \cdot \beta^B$ (\textit{mp\_lshd}) \\
16.  $x1y1 \leftarrow x1y1 \cdot \beta^{2B}$ \\
17.  $t1 \leftarrow x0y0 + t1$ \\
18.  $c \leftarrow t1 + x1y1$ \\
19.  Clear all of the temporary variables. \\
20.  Return(\textit{MP\_OKAY}).\\
\hline 
\end{tabular}
\end{center}
\end{small}
\caption{Algorithm mp\_karatsuba\_mul}
\end{figure}

\textbf{Algorithm mp\_karatsuba\_mul.}
This algorithm computes the unsigned product of two inputs using the Karatsuba multiplication algorithm.  It is loosely based on the description
from Knuth \cite[pp. 294-295]{TAOCPV2}.  

\index{radix point}
In order to split the two inputs into their respective halves, a suitable \textit{radix point} must be chosen.  The radix point chosen must
be used for both of the inputs meaning that it must be smaller than the smallest input.  Step 3 chooses the radix point $B$ as half of the 
smallest input \textbf{used} count.  After the radix point is chosen the inputs are split into lower and upper halves.  Step 4 and 5 
compute the lower halves.  Step 6 and 7 computer the upper halves.  

After the halves have been computed the three intermediate half-size products must be computed.  Step 8 and 9 compute the trivial products
$x0 \cdot y0$ and $x1 \cdot y1$.  The mp\_int $x0$ is used as a temporary variable after $x1 + x0$ has been computed.  By using $x0$ instead
of an additional temporary variable, the algorithm can avoid an addition memory allocation operation.

The remaining steps 13 through 18 compute the Karatsuba polynomial through a variety of digit shifting and addition operations.

\vspace{+3mm}\begin{small}
\hspace{-5.1mm}{\bf File}: bn\_mp\_karatsuba\_mul.c
\vspace{-3mm}
\begin{alltt}
\end{alltt}
\end{small}

The new coding element in this routine, not  seen in previous routines, is the usage of goto statements.  The conventional
wisdom is that goto statements should be avoided.  This is generally true, however when every single function call can fail, it makes sense
to handle error recovery with a single piece of code.  Lines 62 to 76 handle initializing all of the temporary variables 
required.  Note how each of the if statements goes to a different label in case of failure.  This allows the routine to correctly free only
the temporaries that have been successfully allocated so far.

The temporary variables are all initialized using the mp\_init\_size routine since they are expected to be large.  This saves the 
additional reallocation that would have been necessary.  Also $x0$, $x1$, $y0$ and $y1$ have to be able to hold at least their respective
number of digits for the next section of code.

The first algebraic portion of the algorithm is to split the two inputs into their halves.  However, instead of using mp\_mod\_2d and mp\_rshd
to extract the halves, the respective code has been placed inline within the body of the function.  To initialize the halves, the \textbf{used} and 
\textbf{sign} members are copied first.  The first for loop on line 96 copies the lower halves.  Since they are both the same magnitude it 
is simpler to calculate both lower halves in a single loop.  The for loop on lines 102 and 107 calculate the upper halves $x1$ and 
$y1$ respectively.

By inlining the calculation of the halves, the Karatsuba multiplier has a slightly lower overhead and can be used for smaller magnitude inputs.

When line 151 is reached, the algorithm has completed succesfully.  The ``error status'' variable $err$ is set to \textbf{MP\_OKAY} so that
the same code that handles errors can be used to clear the temporary variables and return.  

\subsection{Toom-Cook $3$-Way Multiplication}
Toom-Cook $3$-Way \cite{TOOM} multiplication is essentially the polynomial basis algorithm for $n = 2$ except that the points  are 
chosen such that $\zeta$ is easy to compute and the resulting system of equations easy to reduce.  Here, the points $\zeta_{0}$, 
$16 \cdot \zeta_{1 \over 2}$, $\zeta_1$, $\zeta_2$ and $\zeta_{\infty}$ make up the five required points to solve for the coefficients 
of the $W(x)$.

With the five relations that Toom-Cook specifies, the following system of equations is formed.

\begin{center}
\begin{tabular}{rcrcrcrcrcr}
$\zeta_0$                    & $=$ & $0w_4$ & $+$ & $0w_3$ & $+$ & $0w_2$ & $+$ & $0w_1$ & $+$ & $1w_0$  \\
$16 \cdot \zeta_{1 \over 2}$ & $=$ & $1w_4$ & $+$ & $2w_3$ & $+$ & $4w_2$ & $+$ & $8w_1$ & $+$ & $16w_0$  \\
$\zeta_1$                    & $=$ & $1w_4$ & $+$ & $1w_3$ & $+$ & $1w_2$ & $+$ & $1w_1$ & $+$ & $1w_0$  \\
$\zeta_2$                    & $=$ & $16w_4$ & $+$ & $8w_3$ & $+$ & $4w_2$ & $+$ & $2w_1$ & $+$ & $1w_0$  \\
$\zeta_{\infty}$             & $=$ & $1w_4$ & $+$ & $0w_3$ & $+$ & $0w_2$ & $+$ & $0w_1$ & $+$ & $0w_0$  \\
\end{tabular}
\end{center}

A trivial solution to this matrix requires $12$ subtractions, two multiplications by a small power of two, two divisions by a small power
of two, two divisions by three and one multiplication by three.  All of these $19$ sub-operations require less than quadratic time, meaning that
the algorithm can be faster than a baseline multiplication.  However, the greater complexity of this algorithm places the cutoff point
(\textbf{TOOM\_MUL\_CUTOFF}) where Toom-Cook becomes more efficient much higher than the Karatsuba cutoff point.  

\begin{figure}[!here]
\begin{small}
\begin{center}
\begin{tabular}{l}
\hline Algorithm \textbf{mp\_toom\_mul}. \\
\textbf{Input}.   mp\_int $a$ and mp\_int $b$ \\
\textbf{Output}.  $c \leftarrow  a  \cdot  b $ \\
\hline \\
Split $a$ and $b$ into three pieces.  E.g. $a = a_2 \beta^{2k} + a_1 \beta^{k} + a_0$ \\
1.  $k \leftarrow \lfloor \mbox{min}(a.used, b.used) / 3 \rfloor$ \\
2.  $a_0 \leftarrow a \mbox{ (mod }\beta^{k}\mbox{)}$ \\
3.  $a_1 \leftarrow \lfloor a / \beta^k \rfloor$, $a_1 \leftarrow a_1 \mbox{ (mod }\beta^{k}\mbox{)}$ \\
4.  $a_2 \leftarrow \lfloor a / \beta^{2k} \rfloor$, $a_2 \leftarrow a_2 \mbox{ (mod }\beta^{k}\mbox{)}$ \\
5.  $b_0 \leftarrow a \mbox{ (mod }\beta^{k}\mbox{)}$ \\
6.  $b_1 \leftarrow \lfloor a / \beta^k \rfloor$, $b_1 \leftarrow b_1 \mbox{ (mod }\beta^{k}\mbox{)}$ \\
7.  $b_2 \leftarrow \lfloor a / \beta^{2k} \rfloor$, $b_2 \leftarrow b_2 \mbox{ (mod }\beta^{k}\mbox{)}$ \\
\\
Find the five equations for $w_0, w_1, ..., w_4$. \\
8.  $w_0 \leftarrow a_0 \cdot b_0$ \\
9.  $w_4 \leftarrow a_2 \cdot b_2$ \\
10. $tmp_1 \leftarrow 2 \cdot a_0$, $tmp_1 \leftarrow a_1 + tmp_1$, $tmp_1 \leftarrow 2 \cdot tmp_1$, $tmp_1 \leftarrow tmp_1 + a_2$ \\
11. $tmp_2 \leftarrow 2 \cdot b_0$, $tmp_2 \leftarrow b_1 + tmp_2$, $tmp_2 \leftarrow 2 \cdot tmp_2$, $tmp_2 \leftarrow tmp_2 + b_2$ \\
12. $w_1 \leftarrow tmp_1 \cdot tmp_2$ \\
13. $tmp_1 \leftarrow 2 \cdot a_2$, $tmp_1 \leftarrow a_1 + tmp_1$, $tmp_1 \leftarrow 2 \cdot tmp_1$, $tmp_1 \leftarrow tmp_1 + a_0$ \\
14. $tmp_2 \leftarrow 2 \cdot b_2$, $tmp_2 \leftarrow b_1 + tmp_2$, $tmp_2 \leftarrow 2 \cdot tmp_2$, $tmp_2 \leftarrow tmp_2 + b_0$ \\
15. $w_3 \leftarrow tmp_1 \cdot tmp_2$ \\
16. $tmp_1 \leftarrow a_0 + a_1$, $tmp_1 \leftarrow tmp_1 + a_2$, $tmp_2 \leftarrow b_0 + b_1$, $tmp_2 \leftarrow tmp_2 + b_2$ \\
17. $w_2 \leftarrow tmp_1 \cdot tmp_2$ \\
\\
Continued on the next page.\\
\hline
\end{tabular}
\end{center}
\end{small}
\caption{Algorithm mp\_toom\_mul}
\end{figure}

\newpage\begin{figure}[!here]
\begin{small}
\begin{center}
\begin{tabular}{l}
\hline Algorithm \textbf{mp\_toom\_mul} (continued). \\
\textbf{Input}.   mp\_int $a$ and mp\_int $b$ \\
\textbf{Output}.  $c \leftarrow a \cdot  b $ \\
\hline \\
Now solve the system of equations. \\
18. $w_1 \leftarrow w_4 - w_1$, $w_3 \leftarrow w_3 - w_0$ \\
19. $w_1 \leftarrow \lfloor w_1 / 2 \rfloor$, $w_3 \leftarrow \lfloor w_3 / 2 \rfloor$ \\
20. $w_2 \leftarrow w_2 - w_0$, $w_2 \leftarrow w_2 - w_4$ \\
21. $w_1 \leftarrow w_1 - w_2$, $w_3 \leftarrow w_3 - w_2$ \\
22. $tmp_1 \leftarrow 8 \cdot w_0$, $w_1 \leftarrow w_1 - tmp_1$, $tmp_1 \leftarrow 8 \cdot w_4$, $w_3 \leftarrow w_3 - tmp_1$ \\
23. $w_2 \leftarrow 3 \cdot w_2$, $w_2 \leftarrow w_2 - w_1$, $w_2 \leftarrow w_2 - w_3$ \\
24. $w_1 \leftarrow w_1 - w_2$, $w_3 \leftarrow w_3 - w_2$ \\
25. $w_1 \leftarrow \lfloor w_1 / 3 \rfloor, w_3 \leftarrow \lfloor w_3 / 3 \rfloor$ \\
\\
Now substitute $\beta^k$ for $x$ by shifting $w_0, w_1, ..., w_4$. \\
26. for $n$ from $1$ to $4$ do \\
\hspace{3mm}26.1  $w_n \leftarrow w_n \cdot \beta^{nk}$ \\
27. $c \leftarrow w_0 + w_1$, $c \leftarrow c + w_2$, $c \leftarrow c + w_3$, $c \leftarrow c + w_4$ \\
28. Return(\textit{MP\_OKAY}) \\
\hline
\end{tabular}
\end{center}
\end{small}
\caption{Algorithm mp\_toom\_mul (continued)}
\end{figure}

\textbf{Algorithm mp\_toom\_mul.}
This algorithm computes the product of two mp\_int variables $a$ and $b$ using the Toom-Cook approach.  Compared to the Karatsuba multiplication, this 
algorithm has a lower asymptotic running time of approximately $O(n^{1.464})$ but at an obvious cost in overhead.  In this
description, several statements have been compounded to save space.  The intention is that the statements are executed from left to right across
any given step.

The two inputs $a$ and $b$ are first split into three $k$-digit integers $a_0, a_1, a_2$ and $b_0, b_1, b_2$ respectively.  From these smaller
integers the coefficients of the polynomial basis representations $f(x)$ and $g(x)$ are known and can be used to find the relations required.

The first two relations $w_0$ and $w_4$ are the points $\zeta_{0}$ and $\zeta_{\infty}$ respectively.  The relation $w_1, w_2$ and $w_3$ correspond
to the points $16 \cdot \zeta_{1 \over 2}, \zeta_{2}$ and $\zeta_{1}$ respectively.  These are found using logical shifts to independently find
$f(y)$ and $g(y)$ which significantly speeds up the algorithm.

After the five relations $w_0, w_1, \ldots, w_4$ have been computed, the system they represent must be solved in order for the unknown coefficients 
$w_1, w_2$ and $w_3$ to be isolated.  The steps 18 through 25 perform the system reduction required as previously described.  Each step of
the reduction represents the comparable matrix operation that would be performed had this been performed by pencil.  For example, step 18 indicates
that row $1$ must be subtracted from row $4$ and simultaneously row $0$ subtracted from row $3$.  

Once the coeffients have been isolated, the polynomial $W(x) = \sum_{i=0}^{2n} w_i x^i$ is known.  By substituting $\beta^{k}$ for $x$, the integer 
result $a \cdot b$ is produced.

\vspace{+3mm}\begin{small}
\hspace{-5.1mm}{\bf File}: bn\_mp\_toom\_mul.c
\vspace{-3mm}
\begin{alltt}
\end{alltt}
\end{small}

The first obvious thing to note is that this algorithm is complicated.  The complexity is worth it if you are multiplying very 
large numbers.  For example, a 10,000 digit multiplication takes approximaly 99,282,205 fewer single precision multiplications with
Toom--Cook than a Comba or baseline approach (this is a savings of more than 99$\%$).  For most ``crypto'' sized numbers this
algorithm is not practical as Karatsuba has a much lower cutoff point.

First we split $a$ and $b$ into three roughly equal portions.  This has been accomplished (lines 41 to 70) with 
combinations of mp\_rshd() and mp\_mod\_2d() function calls.  At this point $a = a2 \cdot \beta^2 + a1 \cdot \beta + a0$ and similiarly
for $b$.  

Next we compute the five points $w0, w1, w2, w3$ and $w4$.  Recall that $w0$ and $w4$ can be computed directly from the portions so
we get those out of the way first (lines 73 and 78).  Next we compute $w1, w2$ and $w3$ using Horners method.

After this point we solve for the actual values of $w1, w2$ and $w3$ by reducing the $5 \times 5$ system which is relatively
straight forward.  

\subsection{Signed Multiplication}
Now that algorithms to handle multiplications of every useful dimensions have been developed, a rather simple finishing touch is required.  So far all
of the multiplication algorithms have been unsigned multiplications which leaves only a signed multiplication algorithm to be established.  

\begin{figure}[!here]
\begin{small}
\begin{center}
\begin{tabular}{l}
\hline Algorithm \textbf{mp\_mul}. \\
\textbf{Input}.   mp\_int $a$ and mp\_int $b$ \\
\textbf{Output}.  $c \leftarrow a \cdot b$ \\
\hline \\
1.  If $a.sign = b.sign$ then \\
\hspace{3mm}1.1  $sign = MP\_ZPOS$ \\
2.  else \\
\hspace{3mm}2.1  $sign = MP\_ZNEG$ \\
3.  If min$(a.used, b.used) \ge TOOM\_MUL\_CUTOFF$ then  \\
\hspace{3mm}3.1  $c \leftarrow a \cdot b$ using algorithm mp\_toom\_mul \\
4.  else if min$(a.used, b.used) \ge KARATSUBA\_MUL\_CUTOFF$ then \\
\hspace{3mm}4.1  $c \leftarrow a \cdot b$ using algorithm mp\_karatsuba\_mul \\
5.  else \\
\hspace{3mm}5.1  $digs \leftarrow a.used + b.used + 1$ \\
\hspace{3mm}5.2  If $digs < MP\_ARRAY$ and min$(a.used, b.used) \le \delta$ then \\
\hspace{6mm}5.2.1  $c \leftarrow a \cdot b \mbox{ (mod }\beta^{digs}\mbox{)}$ using algorithm fast\_s\_mp\_mul\_digs.  \\
\hspace{3mm}5.3  else \\
\hspace{6mm}5.3.1  $c \leftarrow a \cdot b \mbox{ (mod }\beta^{digs}\mbox{)}$ using algorithm s\_mp\_mul\_digs.  \\
6.  $c.sign \leftarrow sign$ \\
7.  Return the result of the unsigned multiplication performed. \\
\hline
\end{tabular}
\end{center}
\end{small}
\caption{Algorithm mp\_mul}
\end{figure}

\textbf{Algorithm mp\_mul.}
This algorithm performs the signed multiplication of two inputs.  It will make use of any of the three unsigned multiplication algorithms 
available when the input is of appropriate size.  The \textbf{sign} of the result is not set until the end of the algorithm since algorithm
s\_mp\_mul\_digs will clear it.  

\vspace{+3mm}\begin{small}
\hspace{-5.1mm}{\bf File}: bn\_mp\_mul.c
\vspace{-3mm}
\begin{alltt}
\end{alltt}
\end{small}

The implementation is rather simplistic and is not particularly noteworthy.  Line 22 computes the sign of the result using the ``?'' 
operator from the C programming language.  Line 48 computes $\delta$ using the fact that $1 << k$ is equal to $2^k$.  

\section{Squaring}
\label{sec:basesquare}

Squaring is a special case of multiplication where both multiplicands are equal.  At first it may seem like there is no significant optimization
available but in fact there is.  Consider the multiplication of $576$ against $241$.  In total there will be nine single precision multiplications
performed which are $1\cdot 6$, $1 \cdot 7$, $1 \cdot 5$, $4 \cdot 6$, $4 \cdot 7$, $4 \cdot 5$, $2 \cdot  6$, $2 \cdot 7$ and $2 \cdot 5$.  Now consider 
the multiplication of $123$ against $123$.  The nine products are $3 \cdot 3$, $3 \cdot 2$, $3 \cdot 1$, $2 \cdot 3$, $2 \cdot 2$, $2 \cdot 1$, 
$1 \cdot 3$, $1 \cdot 2$ and $1 \cdot 1$.  On closer inspection some of the products are equivalent.  For example, $3 \cdot 2 = 2 \cdot 3$ 
and $3 \cdot 1 = 1 \cdot 3$. 

For any $n$-digit input, there are ${{\left (n^2 + n \right)}\over 2}$ possible unique single precision multiplications required compared to the $n^2$
required for multiplication.  The following diagram gives an example of the operations required.

\begin{figure}[here]
\begin{center}
\begin{tabular}{ccccc|c}
&&1&2&3&\\
$\times$ &&1&2&3&\\
\hline && $3 \cdot 1$ & $3 \cdot 2$ & $3 \cdot 3$ & Row 0\\
       & $2 \cdot 1$  & $2 \cdot 2$ & $2 \cdot 3$ && Row 1 \\
         $1 \cdot 1$  & $1 \cdot 2$ & $1 \cdot 3$ &&& Row 2 \\
\end{tabular}
\end{center}
\caption{Squaring Optimization Diagram}
\end{figure}

Starting from zero and numbering the columns from right to left a very simple pattern becomes obvious.  For the purposes of this discussion let $x$
represent the number being squared.  The first observation is that in row $k$ the $2k$'th column of the product has a $\left (x_k \right)^2$ term in it.  

The second observation is that every column $j$ in row $k$ where $j \ne 2k$ is part of a double product.  Every non-square term of a column will
appear twice hence the name ``double product''.  Every odd column is made up entirely of double products.  In fact every column is made up of double 
products and at most one square (\textit{see the exercise section}).  

The third and final observation is that for row $k$ the first unique non-square term, that is, one that hasn't already appeared in an earlier row, 
occurs at column $2k + 1$.  For example, on row $1$ of the previous squaring, column one is part of the double product with column one from row zero. 
Column two of row one is a square and column three is the first unique column.

\subsection{The Baseline Squaring Algorithm}
The baseline squaring algorithm is meant to be a catch-all squaring algorithm.  It will handle any of the input sizes that the faster routines
will not handle.  

\begin{figure}[!here]
\begin{small}
\begin{center}
\begin{tabular}{l}
\hline Algorithm \textbf{s\_mp\_sqr}. \\
\textbf{Input}.   mp\_int $a$ \\
\textbf{Output}.  $b \leftarrow a^2$ \\
\hline \\
1.  Init a temporary mp\_int of at least $2 \cdot a.used +1$ digits.  (\textit{mp\_init\_size}) \\
2.  If step 1 failed return(\textit{MP\_MEM}) \\
3.  $t.used \leftarrow 2 \cdot a.used + 1$ \\
4.  For $ix$ from 0 to $a.used - 1$ do \\
\hspace{3mm}Calculate the square. \\
\hspace{3mm}4.1  $\hat r \leftarrow t_{2ix} + \left (a_{ix} \right )^2$ \\
\hspace{3mm}4.2  $t_{2ix} \leftarrow \hat r \mbox{ (mod }\beta\mbox{)}$ \\
\hspace{3mm}Calculate the double products after the square. \\
\hspace{3mm}4.3  $u \leftarrow \lfloor \hat r / \beta \rfloor$ \\
\hspace{3mm}4.4  For $iy$ from $ix + 1$ to $a.used - 1$ do \\
\hspace{6mm}4.4.1  $\hat r \leftarrow 2 \cdot a_{ix}a_{iy} + t_{ix + iy} + u$ \\
\hspace{6mm}4.4.2  $t_{ix + iy} \leftarrow \hat r \mbox{ (mod }\beta\mbox{)}$ \\
\hspace{6mm}4.4.3  $u \leftarrow \lfloor \hat r / \beta \rfloor$ \\
\hspace{3mm}Set the last carry. \\
\hspace{3mm}4.5  While $u > 0$ do \\
\hspace{6mm}4.5.1  $iy \leftarrow iy + 1$ \\
\hspace{6mm}4.5.2  $\hat r \leftarrow t_{ix + iy} + u$ \\
\hspace{6mm}4.5.3  $t_{ix + iy} \leftarrow \hat r \mbox{ (mod }\beta\mbox{)}$ \\
\hspace{6mm}4.5.4  $u \leftarrow \lfloor \hat r / \beta \rfloor$ \\
5.  Clamp excess digits of $t$.  (\textit{mp\_clamp}) \\
6.  Exchange $b$ and $t$. \\
7.  Clear $t$ (\textit{mp\_clear}) \\
8.  Return(\textit{MP\_OKAY}) \\
\hline
\end{tabular}
\end{center}
\end{small}
\caption{Algorithm s\_mp\_sqr}
\end{figure}

\textbf{Algorithm s\_mp\_sqr.}
This algorithm computes the square of an input using the three observations on squaring.  It is based fairly faithfully on  algorithm 14.16 of HAC
\cite[pp.596-597]{HAC}.  Similar to algorithm s\_mp\_mul\_digs, a temporary mp\_int is allocated to hold the result of the squaring.  This allows the 
destination mp\_int to be the same as the source mp\_int.

The outer loop of this algorithm begins on step 4. It is best to think of the outer loop as walking down the rows of the partial results, while
the inner loop computes the columns of the partial result.  Step 4.1 and 4.2 compute the square term for each row, and step 4.3 and 4.4 propagate
the carry and compute the double products.  

The requirement that a mp\_word be able to represent the range $0 \le x < 2 \beta^2$ arises from this
very algorithm.  The product $a_{ix}a_{iy}$ will lie in the range $0 \le x \le \beta^2 - 2\beta + 1$ which is obviously less than $\beta^2$ meaning that
when it is multiplied by two, it can be properly represented by a mp\_word.

Similar to algorithm s\_mp\_mul\_digs, after every pass of the inner loop, the destination is correctly set to the sum of all of the partial 
results calculated so far.  This involves expensive carry propagation which will be eliminated in the next algorithm.  

\vspace{+3mm}\begin{small}
\hspace{-5.1mm}{\bf File}: bn\_s\_mp\_sqr.c
\vspace{-3mm}
\begin{alltt}
\end{alltt}
\end{small}

Inside the outer loop (line 34) the square term is calculated on line 37.  The carry (line 44) has been
extracted from the mp\_word accumulator using a right shift.  Aliases for $a_{ix}$ and $t_{ix+iy}$ are initialized 
(lines 47 and 50) to simplify the inner loop.  The doubling is performed using two
additions (line 59) since it is usually faster than shifting, if not at least as fast.  

The important observation is that the inner loop does not begin at $iy = 0$ like for multiplication.  As such the inner loops
get progressively shorter as the algorithm proceeds.  This is what leads to the savings compared to using a multiplication to
square a number. 

\subsection{Faster Squaring by the ``Comba'' Method}
A major drawback to the baseline method is the requirement for single precision shifting inside the $O(n^2)$ nested loop.  Squaring has an additional
drawback that it must double the product inside the inner loop as well.  As for multiplication, the Comba technique can be used to eliminate these
performance hazards.

The first obvious solution is to make an array of mp\_words which will hold all of the columns.  This will indeed eliminate all of the carry
propagation operations from the inner loop.  However, the inner product must still be doubled $O(n^2)$ times.  The solution stems from the simple fact
that $2a + 2b + 2c = 2(a + b + c)$.  That is the sum of all of the double products is equal to double the sum of all the products.  For example,
$ab + ba + ac + ca = 2ab + 2ac = 2(ab + ac)$.  

However, we cannot simply double all of the columns, since the squares appear only once per row.  The most practical solution is to have two 
mp\_word arrays.  One array will hold the squares and the other array will hold the double products.  With both arrays the doubling and 
carry propagation can be moved to a $O(n)$ work level outside the $O(n^2)$ level.  In this case, we have an even simpler solution in mind.

\newpage\begin{figure}[!here]
\begin{small}
\begin{center}
\begin{tabular}{l}
\hline Algorithm \textbf{fast\_s\_mp\_sqr}. \\
\textbf{Input}.   mp\_int $a$ \\
\textbf{Output}.  $b \leftarrow a^2$ \\
\hline \\
Place an array of \textbf{MP\_WARRAY} mp\_digits named $W$ on the stack. \\
1.  If $b.alloc < 2a.used + 1$ then grow $b$ to $2a.used + 1$ digits.  (\textit{mp\_grow}). \\
2.  If step 1 failed return(\textit{MP\_MEM}). \\
\\
3.  $pa \leftarrow 2 \cdot a.used$ \\
4.  $\hat W1 \leftarrow 0$ \\
5.  for $ix$ from $0$ to $pa - 1$ do \\
\hspace{3mm}5.1  $\_ \hat W \leftarrow 0$ \\
\hspace{3mm}5.2  $ty \leftarrow \mbox{MIN}(a.used - 1, ix)$ \\
\hspace{3mm}5.3  $tx \leftarrow ix - ty$ \\
\hspace{3mm}5.4  $iy \leftarrow \mbox{MIN}(a.used - tx, ty + 1)$ \\
\hspace{3mm}5.5  $iy \leftarrow \mbox{MIN}(iy, \lfloor \left (ty - tx + 1 \right )/2 \rfloor)$ \\
\hspace{3mm}5.6  for $iz$ from $0$ to $iz - 1$ do \\
\hspace{6mm}5.6.1  $\_ \hat W \leftarrow \_ \hat W + a_{tx + iz}a_{ty - iz}$ \\
\hspace{3mm}5.7  $\_ \hat W \leftarrow 2 \cdot \_ \hat W  + \hat W1$ \\
\hspace{3mm}5.8  if $ix$ is even then \\
\hspace{6mm}5.8.1  $\_ \hat W \leftarrow \_ \hat W + \left ( a_{\lfloor ix/2 \rfloor}\right )^2$ \\
\hspace{3mm}5.9  $W_{ix} \leftarrow \_ \hat W (\mbox{mod }\beta)$ \\
\hspace{3mm}5.10  $\hat W1 \leftarrow \lfloor \_ \hat W / \beta \rfloor$ \\
\\
6.  $oldused \leftarrow b.used$ \\
7.  $b.used \leftarrow 2 \cdot a.used$ \\
8.  for $ix$ from $0$ to $pa - 1$ do \\
\hspace{3mm}8.1  $b_{ix} \leftarrow W_{ix}$ \\
9.  for $ix$ from $pa$ to $oldused - 1$ do \\
\hspace{3mm}9.1  $b_{ix} \leftarrow 0$ \\
10.  Clamp excess digits from $b$.  (\textit{mp\_clamp}) \\
11.  Return(\textit{MP\_OKAY}). \\ 
\hline
\end{tabular}
\end{center}
\end{small}
\caption{Algorithm fast\_s\_mp\_sqr}
\end{figure}

\textbf{Algorithm fast\_s\_mp\_sqr.}
This algorithm computes the square of an input using the Comba technique.  It is designed to be a replacement for algorithm 
s\_mp\_sqr when the number of input digits is less than \textbf{MP\_WARRAY} and less than $\delta \over 2$.  
This algorithm is very similar to the Comba multiplier except with a few key differences we shall make note of.

First, we have an accumulator and carry variables $\_ \hat W$ and $\hat W1$ respectively.  This is because the inner loop
products are to be doubled.  If we had added the previous carry in we would be doubling too much.  Next we perform an
addition MIN condition on $iy$ (step 5.5) to prevent overlapping digits.  For example, $a_3 \cdot a_5$ is equal
$a_5 \cdot a_3$.  Whereas in the multiplication case we would have $5 < a.used$ and $3 \ge 0$ is maintained since we double the sum
of the products just outside the inner loop we have to avoid doing this.  This is also a good thing since we perform
fewer multiplications and the routine ends up being faster.

Finally the last difference is the addition of the ``square'' term outside the inner loop (step 5.8).  We add in the square
only to even outputs and it is the square of the term at the $\lfloor ix / 2 \rfloor$ position.

\vspace{+3mm}\begin{small}
\hspace{-5.1mm}{\bf File}: bn\_fast\_s\_mp\_sqr.c
\vspace{-3mm}
\begin{alltt}
\end{alltt}
\end{small}

This implementation is essentially a copy of Comba multiplication with the appropriate changes added to make it faster for 
the special case of squaring.  

\subsection{Polynomial Basis Squaring}
The same algorithm that performs optimal polynomial basis multiplication can be used to perform polynomial basis squaring.  The minor exception
is that $\zeta_y = f(y)g(y)$ is actually equivalent to $\zeta_y = f(y)^2$ since $f(y) = g(y)$.  Instead of performing $2n + 1$
multiplications to find the $\zeta$ relations, squaring operations are performed instead.  

\subsection{Karatsuba Squaring}
Let $f(x) = ax + b$ represent the polynomial basis representation of a number to square.  
Let $h(x) = \left ( f(x) \right )^2$ represent the square of the polynomial.  The Karatsuba equation can be modified to square a 
number with the following equation.

\begin{equation}
h(x) = a^2x^2 + \left ((a + b)^2 - (a^2 + b^2) \right )x + b^2
\end{equation}

Upon closer inspection this equation only requires the calculation of three half-sized squares: $a^2$, $b^2$ and $(a + b)^2$.  As in 
Karatsuba multiplication, this algorithm can be applied recursively on the input and will achieve an asymptotic running time of 
$O \left ( n^{lg(3)} \right )$.

If the asymptotic times of Karatsuba squaring and multiplication are the same, why not simply use the multiplication algorithm 
instead?  The answer to this arises from the cutoff point for squaring.  As in multiplication there exists a cutoff point, at which the 
time required for a Comba based squaring and a Karatsuba based squaring meet.  Due to the overhead inherent in the Karatsuba method, the cutoff 
point is fairly high.  For example, on an AMD Athlon XP processor with $\beta = 2^{28}$, the cutoff point is around 127 digits.  

Consider squaring a 200 digit number with this technique.  It will be split into two 100 digit halves which are subsequently squared.  
The 100 digit halves will not be squared using Karatsuba, but instead using the faster Comba based squaring algorithm.  If Karatsuba multiplication
were used instead, the 100 digit numbers would be squared with a slower Comba based multiplication.  

\newpage\begin{figure}[!here]
\begin{small}
\begin{center}
\begin{tabular}{l}
\hline Algorithm \textbf{mp\_karatsuba\_sqr}. \\
\textbf{Input}.   mp\_int $a$ \\
\textbf{Output}.  $b \leftarrow a^2$ \\
\hline \\
1.  Initialize the following temporary mp\_ints:  $x0$, $x1$, $t1$, $t2$, $x0x0$ and $x1x1$. \\
2.  If any of the initializations on step 1 failed return(\textit{MP\_MEM}). \\
\\
Split the input.  e.g. $a = x1\beta^B + x0$ \\
3.  $B \leftarrow \lfloor a.used / 2 \rfloor$ \\
4.  $x0 \leftarrow a \mbox{ (mod }\beta^B\mbox{)}$ (\textit{mp\_mod\_2d}) \\
5.  $x1 \leftarrow \lfloor a / \beta^B \rfloor$ (\textit{mp\_lshd}) \\
\\
Calculate the three squares. \\
6.  $x0x0 \leftarrow x0^2$ (\textit{mp\_sqr}) \\
7.  $x1x1 \leftarrow x1^2$ \\
8.  $t1 \leftarrow x1 + x0$ (\textit{s\_mp\_add}) \\
9.  $t1 \leftarrow t1^2$ \\
\\
Compute the middle term. \\
10.  $t2 \leftarrow x0x0 + x1x1$ (\textit{s\_mp\_add}) \\
11.  $t1 \leftarrow t1 - t2$ \\
\\
Compute final product. \\
12.  $t1 \leftarrow t1\beta^B$ (\textit{mp\_lshd}) \\
13.  $x1x1 \leftarrow x1x1\beta^{2B}$ \\
14.  $t1 \leftarrow t1 + x0x0$ \\
15.  $b \leftarrow t1 + x1x1$ \\
16.  Return(\textit{MP\_OKAY}). \\
\hline
\end{tabular}
\end{center}
\end{small}
\caption{Algorithm mp\_karatsuba\_sqr}
\end{figure}

\textbf{Algorithm mp\_karatsuba\_sqr.}
This algorithm computes the square of an input $a$ using the Karatsuba technique.  This algorithm is very similar to the Karatsuba based
multiplication algorithm with the exception that the three half-size multiplications have been replaced with three half-size squarings.

The radix point for squaring is simply placed exactly in the middle of the digits when the input has an odd number of digits, otherwise it is
placed just below the middle.  Step 3, 4 and 5 compute the two halves required using $B$
as the radix point.  The first two squares in steps 6 and 7 are rather straightforward while the last square is of a more compact form.

By expanding $\left (x1 + x0 \right )^2$, the $x1^2$ and $x0^2$ terms in the middle disappear, that is $(x0 - x1)^2 - (x1^2 + x0^2)  = 2 \cdot x0 \cdot x1$.
Now if $5n$ single precision additions and a squaring of $n$-digits is faster than multiplying two $n$-digit numbers and doubling then
this method is faster.  Assuming no further recursions occur, the difference can be estimated with the following inequality.

Let $p$ represent the cost of a single precision addition and $q$ the cost of a single precision multiplication both in terms of time\footnote{Or
machine clock cycles.}. 

\begin{equation}
5pn +{{q(n^2 + n)} \over 2} \le pn + qn^2
\end{equation}

For example, on an AMD Athlon XP processor $p = {1 \over 3}$ and $q = 6$.  This implies that the following inequality should hold.
\begin{center}
\begin{tabular}{rcl}
${5n \over 3} + 3n^2 + 3n$     & $<$ & ${n \over 3} + 6n^2$ \\
${5 \over 3} + 3n + 3$     & $<$ & ${1 \over 3} + 6n$ \\
${13 \over 9}$     & $<$ & $n$ \\
\end{tabular}
\end{center}

This results in a cutoff point around $n = 2$.  As a consequence it is actually faster to compute the middle term the ``long way'' on processors
where multiplication is substantially slower\footnote{On the Athlon there is a 1:17 ratio between clock cycles for addition and multiplication.  On
the Intel P4 processor this ratio is 1:29 making this method even more beneficial.  The only common exception is the ARMv4 processor which has a
ratio of 1:7.  } than simpler operations such as addition.  

\vspace{+3mm}\begin{small}
\hspace{-5.1mm}{\bf File}: bn\_mp\_karatsuba\_sqr.c
\vspace{-3mm}
\begin{alltt}
\end{alltt}
\end{small}

This implementation is largely based on the implementation of algorithm mp\_karatsuba\_mul.  It uses the same inline style to copy and 
shift the input into the two halves.  The loop from line 54 to line 70 has been modified since only one input exists.  The \textbf{used}
count of both $x0$ and $x1$ is fixed up and $x0$ is clamped before the calculations begin.  At this point $x1$ and $x0$ are valid equivalents
to the respective halves as if mp\_rshd and mp\_mod\_2d had been used.  

By inlining the copy and shift operations the cutoff point for Karatsuba multiplication can be lowered.  On the Athlon the cutoff point
is exactly at the point where Comba squaring can no longer be used (\textit{128 digits}).  On slower processors such as the Intel P4
it is actually below the Comba limit (\textit{at 110 digits}).

This routine uses the same error trap coding style as mp\_karatsuba\_sqr.  As the temporary variables are initialized errors are 
redirected to the error trap higher up.  If the algorithm completes without error the error code is set to \textbf{MP\_OKAY} and 
mp\_clears are executed normally.

\subsection{Toom-Cook Squaring}
The Toom-Cook squaring algorithm mp\_toom\_sqr is heavily based on the algorithm mp\_toom\_mul with the exception that squarings are used
instead of multiplication to find the five relations.  The reader is encouraged to read the description of the latter algorithm and try to 
derive their own Toom-Cook squaring algorithm.  

\subsection{High Level Squaring}
\newpage\begin{figure}[!here]
\begin{small}
\begin{center}
\begin{tabular}{l}
\hline Algorithm \textbf{mp\_sqr}. \\
\textbf{Input}.   mp\_int $a$ \\
\textbf{Output}.  $b \leftarrow a^2$ \\
\hline \\
1.  If $a.used \ge TOOM\_SQR\_CUTOFF$ then  \\
\hspace{3mm}1.1  $b \leftarrow a^2$ using algorithm mp\_toom\_sqr \\
2.  else if $a.used \ge KARATSUBA\_SQR\_CUTOFF$ then \\
\hspace{3mm}2.1  $b \leftarrow a^2$ using algorithm mp\_karatsuba\_sqr \\
3.  else \\
\hspace{3mm}3.1  $digs \leftarrow a.used + b.used + 1$ \\
\hspace{3mm}3.2  If $digs < MP\_ARRAY$ and $a.used \le \delta$ then \\
\hspace{6mm}3.2.1  $b \leftarrow a^2$ using algorithm fast\_s\_mp\_sqr.  \\
\hspace{3mm}3.3  else \\
\hspace{6mm}3.3.1  $b \leftarrow a^2$ using algorithm s\_mp\_sqr.  \\
4.  $b.sign \leftarrow MP\_ZPOS$ \\
5.  Return the result of the unsigned squaring performed. \\
\hline
\end{tabular}
\end{center}
\end{small}
\caption{Algorithm mp\_sqr}
\end{figure}

\textbf{Algorithm mp\_sqr.}
This algorithm computes the square of the input using one of four different algorithms.  If the input is very large and has at least
\textbf{TOOM\_SQR\_CUTOFF} or \textbf{KARATSUBA\_SQR\_CUTOFF} digits then either the Toom-Cook or the Karatsuba Squaring algorithm is used.  If
neither of the polynomial basis algorithms should be used then either the Comba or baseline algorithm is used.  

\vspace{+3mm}\begin{small}
\hspace{-5.1mm}{\bf File}: bn\_mp\_sqr.c
\vspace{-3mm}
\begin{alltt}
\end{alltt}
\end{small}

\section*{Exercises}
\begin{tabular}{cl}
$\left [ 3 \right ] $ & Devise an efficient algorithm for selection of the radix point to handle inputs \\
                      & that have different number of digits in Karatsuba multiplication. \\
                      & \\
$\left [ 2 \right ] $ & In section 5.3 the fact that every column of a squaring is made up \\
                      & of double products and at most one square is stated.  Prove this statement. \\
                      & \\                      
$\left [ 3 \right ] $ & Prove the equation for Karatsuba squaring. \\
                      & \\
$\left [ 1 \right ] $ & Prove that Karatsuba squaring requires $O \left (n^{lg(3)} \right )$ time. \\
                      & \\ 
$\left [ 2 \right ] $ & Determine the minimal ratio between addition and multiplication clock cycles \\
                      & required for equation $6.7$ to be true.  \\
                      & \\
$\left [ 3 \right ] $ & Implement a threaded version of Comba multiplication (and squaring) where you \\
                      & compute subsets of the columns in each thread.  Determine a cutoff point where \\
                      & it is effective and add the logic to mp\_mul() and mp\_sqr(). \\
                      &\\
$\left [ 4 \right ] $ & Same as the previous but also modify the Karatsuba and Toom-Cook.  You must \\
                      & increase the throughput of mp\_exptmod() for random odd moduli in the range \\
                      & $512 \ldots 4096$ bits significantly ($> 2x$) to complete this challenge. \\
                      & \\
\end{tabular}

\chapter{Modular Reduction}
\section{Basics of Modular Reduction}
\index{modular residue}
Modular reduction is an operation that arises quite often within public key cryptography algorithms and various number theoretic algorithms, 
such as factoring.  Modular reduction algorithms are the third class of algorithms of the ``multipliers'' set.  A number $a$ is said to be \textit{reduced}
modulo another number $b$ by finding the remainder of the division $a/b$.  Full integer division with remainder is a topic to be covered 
in~\ref{sec:division}.

Modular reduction is equivalent to solving for $r$ in the following equation.  $a = bq + r$ where $q = \lfloor a/b \rfloor$.  The result 
$r$ is said to be ``congruent to $a$ modulo $b$'' which is also written as $r \equiv a \mbox{ (mod }b\mbox{)}$.  In other vernacular $r$ is known as the 
``modular residue'' which leads to ``quadratic residue''\footnote{That's fancy talk for $b \equiv a^2 \mbox{ (mod }p\mbox{)}$.} and
other forms of residues.  

Modular reductions are normally used to create either finite groups, rings or fields.  The most common usage for performance driven modular reductions 
is in modular exponentiation algorithms.  That is to compute $d = a^b \mbox{ (mod }c\mbox{)}$ as fast as possible.  This operation is used in the 
RSA and Diffie-Hellman public key algorithms, for example.  Modular multiplication and squaring also appears as a fundamental operation in 
elliptic curve cryptographic algorithms.  As will be discussed in the subsequent chapter there exist fast algorithms for computing modular 
exponentiations without having to perform (\textit{in this example}) $b - 1$ multiplications.  These algorithms will produce partial results in the 
range $0 \le x < c^2$ which can be taken advantage of to create several efficient algorithms.   They have also been used to create redundancy check 
algorithms known as CRCs, error correction codes such as Reed-Solomon and solve a variety of number theoeretic problems.  

\section{The Barrett Reduction}
The Barrett reduction algorithm \cite{BARRETT} was inspired by fast division algorithms which multiply by the reciprocal to emulate
division.  Barretts observation was that the residue $c$ of $a$ modulo $b$ is equal to 

\begin{equation}
c = a - b \cdot \lfloor a/b \rfloor
\end{equation}

Since algorithms such as modular exponentiation would be using the same modulus extensively, typical DSP\footnote{It is worth noting that Barrett's paper 
targeted the DSP56K processor.}  intuition would indicate the next step would be to replace $a/b$ by a multiplication by the reciprocal.  However, 
DSP intuition on its own will not work as these numbers are considerably larger than the precision of common DSP floating point data types.  
It would take another common optimization to optimize the algorithm.

\subsection{Fixed Point Arithmetic}
The trick used to optimize the above equation is based on a technique of emulating floating point data types with fixed precision integers.  Fixed
point arithmetic would become very popular as it greatly optimize the ``3d-shooter'' genre of games in the mid 1990s when floating point units were 
fairly slow if not unavailable.   The idea behind fixed point arithmetic is to take a normal $k$-bit integer data type and break it into $p$-bit 
integer and a $q$-bit fraction part (\textit{where $p+q = k$}).  

In this system a $k$-bit integer $n$ would actually represent $n/2^q$.  For example, with $q = 4$ the integer $n = 37$ would actually represent the
value $2.3125$.  To multiply two fixed point numbers the integers are multiplied using traditional arithmetic and subsequently normalized by 
moving the implied decimal point back to where it should be.  For example, with $q = 4$ to multiply the integers $9$ and $5$ they must be converted 
to fixed point first by multiplying by $2^q$.  Let $a = 9(2^q)$ represent the fixed point representation of $9$ and $b = 5(2^q)$ represent the 
fixed point representation of $5$.  The product $ab$ is equal to $45(2^{2q})$ which when normalized by dividing by $2^q$ produces $45(2^q)$.  

This technique became popular since a normal integer multiplication and logical shift right are the only required operations to perform a multiplication
of two fixed point numbers.  Using fixed point arithmetic, division can be easily approximated by multiplying by the reciprocal.  If $2^q$ is 
equivalent to one than $2^q/b$ is equivalent to the fixed point approximation of $1/b$ using real arithmetic.  Using this fact dividing an integer 
$a$ by another integer $b$ can be achieved with the following expression.

\begin{equation}
\lfloor a / b \rfloor \mbox{ }\approx\mbox{ } \lfloor (a \cdot \lfloor 2^q / b \rfloor)/2^q \rfloor
\end{equation}

The precision of the division is proportional to the value of $q$.  If the divisor $b$ is used frequently as is the case with 
modular exponentiation pre-computing $2^q/b$ will allow a division to be performed with a multiplication and a right shift.  Both operations
are considerably faster than division on most processors.  

Consider dividing $19$ by $5$.  The correct result is $\lfloor 19/5 \rfloor = 3$.  With $q = 3$ the reciprocal is $\lfloor 2^q/5 \rfloor = 1$ which
leads to a product of $19$ which when divided by $2^q$ produces $2$.  However, with $q = 4$ the reciprocal is $\lfloor 2^q/5 \rfloor = 3$ and
the result of the emulated division is $\lfloor 3 \cdot 19 / 2^q \rfloor = 3$ which is correct.  The value of $2^q$ must be close to or ideally
larger than the dividend.  In effect if $a$ is the dividend then $q$ should allow $0 \le \lfloor a/2^q \rfloor \le 1$ in order for this approach
to work correctly.  Plugging this form of divison into the original equation the following modular residue equation arises.

\begin{equation}
c = a - b \cdot \lfloor (a \cdot \lfloor 2^q / b \rfloor)/2^q \rfloor
\end{equation}

Using the notation from \cite{BARRETT} the value of $\lfloor 2^q / b \rfloor$ will be represented by the $\mu$ symbol.  Using the $\mu$
variable also helps re-inforce the idea that it is meant to be computed once and re-used.

\begin{equation}
c = a - b \cdot \lfloor (a \cdot \mu)/2^q \rfloor
\end{equation}

Provided that $2^q \ge a$ this algorithm will produce a quotient that is either exactly correct or off by a value of one.  In the context of Barrett
reduction the value of $a$ is bound by $0 \le a \le (b - 1)^2$ meaning that $2^q \ge b^2$ is sufficient to ensure the reciprocal will have enough
precision.  

Let $n$ represent the number of digits in $b$.  This algorithm requires approximately $2n^2$ single precision multiplications to produce the quotient and 
another $n^2$ single precision multiplications to find the residue.  In total $3n^2$ single precision multiplications are required to 
reduce the number.  

For example, if $b = 1179677$ and $q = 41$ ($2^q > b^2$), then the reciprocal $\mu$ is equal to $\lfloor 2^q / b \rfloor = 1864089$.  Consider reducing
$a = 180388626447$ modulo $b$ using the above reduction equation.  The quotient using the new formula is $\lfloor (a \cdot \mu) / 2^q \rfloor = 152913$.
By subtracting $152913b$ from $a$ the correct residue $a \equiv 677346 \mbox{ (mod }b\mbox{)}$ is found.

\subsection{Choosing a Radix Point}
Using the fixed point representation a modular reduction can be performed with $3n^2$ single precision multiplications.  If that were the best
that could be achieved a full division\footnote{A division requires approximately $O(2cn^2)$ single precision multiplications for a small value of $c$.  
See~\ref{sec:division} for further details.} might as well be used in its place.  The key to optimizing the reduction is to reduce the precision of
the initial multiplication that finds the quotient.  

Let $a$ represent the number of which the residue is sought.  Let $b$ represent the modulus used to find the residue.  Let $m$ represent
the number of digits in $b$.  For the purposes of this discussion we will assume that the number of digits in $a$ is $2m$, which is generally true if 
two $m$-digit numbers have been multiplied.  Dividing $a$ by $b$ is the same as dividing a $2m$ digit integer by a $m$ digit integer.  Digits below the 
$m - 1$'th digit of $a$ will contribute at most a value of $1$ to the quotient because $\beta^k < b$ for any $0 \le k \le m - 1$.  Another way to
express this is by re-writing $a$ as two parts.  If $a' \equiv a \mbox{ (mod }b^m\mbox{)}$ and $a'' = a - a'$ then 
${a \over b} \equiv {{a' + a''} \over b}$ which is equivalent to ${a' \over b} + {a'' \over b}$.  Since $a'$ is bound to be less than $b$ the quotient
is bound by $0 \le {a' \over b} < 1$.

Since the digits of $a'$ do not contribute much to the quotient the observation is that they might as well be zero.  However, if the digits 
``might as well be zero'' they might as well not be there in the first place.  Let $q_0 = \lfloor a/\beta^{m-1} \rfloor$ represent the input
with the irrelevant digits trimmed.  Now the modular reduction is trimmed to the almost equivalent equation

\begin{equation}
c = a - b \cdot \lfloor (q_0 \cdot \mu) / \beta^{m+1} \rfloor
\end{equation}

Note that the original divisor $2^q$ has been replaced with $\beta^{m+1}$ where in this case $q$ is a multiple of $lg(\beta)$. Also note that the 
exponent on the divisor when added to the amount $q_0$ was shifted by equals $2m$.  If the optimization had not been performed the divisor 
would have the exponent $2m$ so in the end the exponents do ``add up''. Using the above equation the quotient 
$\lfloor (q_0 \cdot \mu) / \beta^{m+1} \rfloor$ can be off from the true quotient by at most two.  The original fixed point quotient can be off
by as much as one (\textit{provided the radix point is chosen suitably}) and now that the lower irrelevent digits have been trimmed the quotient
can be off by an additional value of one for a total of at most two.  This implies that 
$0 \le a - b \cdot \lfloor (q_0 \cdot \mu) / \beta^{m+1} \rfloor < 3b$.  By first subtracting $b$ times the quotient and then conditionally subtracting 
$b$ once or twice the residue is found.

The quotient is now found using $(m + 1)(m) = m^2 + m$ single precision multiplications and the residue with an additional $m^2$ single
precision multiplications, ignoring the subtractions required.  In total $2m^2 + m$ single precision multiplications are required to find the residue.  
This is considerably faster than the original attempt.

For example, let $\beta = 10$ represent the radix of the digits.  Let $b = 9999$ represent the modulus which implies $m = 4$. Let $a = 99929878$ 
represent the value of which the residue is desired.  In this case $q = 8$ since $10^7 < 9999^2$ meaning that $\mu = \lfloor \beta^{q}/b \rfloor = 10001$.  
With the new observation the multiplicand for the quotient is equal to $q_0 = \lfloor a / \beta^{m - 1} \rfloor = 99929$.  The quotient is then 
$\lfloor (q_0 \cdot \mu) / \beta^{m+1} \rfloor = 9993$.  Subtracting $9993b$ from $a$ and the correct residue $a \equiv 9871 \mbox{ (mod }b\mbox{)}$ 
is found.  

\subsection{Trimming the Quotient}
So far the reduction algorithm has been optimized from $3m^2$ single precision multiplications down to $2m^2 + m$ single precision multiplications.  As 
it stands now the algorithm is already fairly fast compared to a full integer division algorithm.  However, there is still room for
optimization.  

After the first multiplication inside the quotient ($q_0 \cdot \mu$) the value is shifted right by $m + 1$ places effectively nullifying the lower
half of the product.  It would be nice to be able to remove those digits from the product to effectively cut down the number of single precision 
multiplications.  If the number of digits in the modulus $m$ is far less than $\beta$ a full product is not required for the algorithm to work properly.  
In fact the lower $m - 2$ digits will not affect the upper half of the product at all and do not need to be computed.  

The value of $\mu$ is a $m$-digit number and $q_0$ is a $m + 1$ digit number.  Using a full multiplier $(m + 1)(m) = m^2 + m$ single precision
multiplications would be required.  Using a multiplier that will only produce digits at and above the $m - 1$'th digit reduces the number
of single precision multiplications to ${m^2 + m} \over 2$ single precision multiplications.  

\subsection{Trimming the Residue}
After the quotient has been calculated it is used to reduce the input.  As previously noted the algorithm is not exact and it can be off by a small
multiple of the modulus, that is $0 \le a - b \cdot \lfloor (q_0 \cdot \mu) / \beta^{m+1} \rfloor < 3b$.  If $b$ is $m$ digits than the 
result of reduction equation is a value of at most $m + 1$ digits (\textit{provided $3 < \beta$}) implying that the upper $m - 1$ digits are
implicitly zero.  

The next optimization arises from this very fact.  Instead of computing $b \cdot \lfloor (q_0 \cdot \mu) / \beta^{m+1} \rfloor$ using a full
$O(m^2)$ multiplication algorithm only the lower $m+1$ digits of the product have to be computed.  Similarly the value of $a$ can
be reduced modulo $\beta^{m+1}$ before the multiple of $b$ is subtracted which simplifes the subtraction as well.  A multiplication that produces 
only the lower $m+1$ digits requires ${m^2 + 3m - 2} \over 2$ single precision multiplications.  

With both optimizations in place the algorithm is the algorithm Barrett proposed.  It requires $m^2 + 2m - 1$ single precision multiplications which
is considerably faster than the straightforward $3m^2$ method.  

\subsection{The Barrett Algorithm}
\newpage\begin{figure}[!here]
\begin{small}
\begin{center}
\begin{tabular}{l}
\hline Algorithm \textbf{mp\_reduce}. \\
\textbf{Input}.   mp\_int $a$, mp\_int $b$ and $\mu = \lfloor \beta^{2m}/b \rfloor, m = \lceil lg_{\beta}(b) \rceil, (0 \le a < b^2, b > 1)$ \\
\textbf{Output}.  $a \mbox{ (mod }b\mbox{)}$ \\
\hline \\
Let $m$ represent the number of digits in $b$.  \\
1.  Make a copy of $a$ and store it in $q$.  (\textit{mp\_init\_copy}) \\
2.  $q \leftarrow \lfloor q / \beta^{m - 1} \rfloor$ (\textit{mp\_rshd}) \\
\\
Produce the quotient. \\
3.  $q \leftarrow q \cdot \mu$  (\textit{note: only produce digits at or above $m-1$}) \\
4.  $q \leftarrow \lfloor q / \beta^{m + 1} \rfloor$ \\
\\
Subtract the multiple of modulus from the input. \\
5.  $a \leftarrow a \mbox{ (mod }\beta^{m+1}\mbox{)}$ (\textit{mp\_mod\_2d}) \\
6.  $q \leftarrow q \cdot b \mbox{ (mod }\beta^{m+1}\mbox{)}$ (\textit{s\_mp\_mul\_digs}) \\
7.  $a \leftarrow a - q$ (\textit{mp\_sub}) \\
\\
Add $\beta^{m+1}$ if a carry occured. \\
8.  If $a < 0$ then (\textit{mp\_cmp\_d}) \\
\hspace{3mm}8.1  $q \leftarrow 1$ (\textit{mp\_set}) \\
\hspace{3mm}8.2  $q \leftarrow q \cdot \beta^{m+1}$ (\textit{mp\_lshd}) \\
\hspace{3mm}8.3  $a \leftarrow a + q$ \\
\\
Now subtract the modulus if the residue is too large (e.g. quotient too small). \\
9.  While $a \ge b$ do (\textit{mp\_cmp}) \\
\hspace{3mm}9.1  $c \leftarrow a - b$ \\
10.  Clear $q$. \\
11.  Return(\textit{MP\_OKAY}) \\
\hline
\end{tabular}
\end{center}
\end{small}
\caption{Algorithm mp\_reduce}
\end{figure}

\textbf{Algorithm mp\_reduce.}
This algorithm will reduce the input $a$ modulo $b$ in place using the Barrett algorithm.  It is loosely based on algorithm 14.42 of HAC
\cite[pp.  602]{HAC} which is based on the paper from Paul Barrett \cite{BARRETT}.  The algorithm has several restrictions and assumptions which must 
be adhered to for the algorithm to work.

First the modulus $b$ is assumed to be positive and greater than one.  If the modulus were less than or equal to one than subtracting
a multiple of it would either accomplish nothing or actually enlarge the input.  The input $a$ must be in the range $0 \le a < b^2$ in order
for the quotient to have enough precision.  If $a$ is the product of two numbers that were already reduced modulo $b$, this will not be a problem.
Technically the algorithm will still work if $a \ge b^2$ but it will take much longer to finish.  The value of $\mu$ is passed as an argument to this 
algorithm and is assumed to be calculated and stored before the algorithm is used.  

Recall that the multiplication for the quotient on step 3 must only produce digits at or above the $m-1$'th position.  An algorithm called 
$s\_mp\_mul\_high\_digs$ which has not been presented is used to accomplish this task.  The algorithm is based on $s\_mp\_mul\_digs$ except that
instead of stopping at a given level of precision it starts at a given level of precision.  This optimal algorithm can only be used if the number
of digits in $b$ is very much smaller than $\beta$.  

While it is known that 
$a \ge b \cdot \lfloor (q_0 \cdot \mu) / \beta^{m+1} \rfloor$ only the lower $m+1$ digits are being used to compute the residue, so an implied 
``borrow'' from the higher digits might leave a negative result.  After the multiple of the modulus has been subtracted from $a$ the residue must be 
fixed up in case it is negative.  The invariant $\beta^{m+1}$ must be added to the residue to make it positive again.  

The while loop at step 9 will subtract $b$ until the residue is less than $b$.  If the algorithm is performed correctly this step is 
performed at most twice, and on average once. However, if $a \ge b^2$ than it will iterate substantially more times than it should.

\vspace{+3mm}\begin{small}
\hspace{-5.1mm}{\bf File}: bn\_mp\_reduce.c
\vspace{-3mm}
\begin{alltt}
\end{alltt}
\end{small}

The first multiplication that determines the quotient can be performed by only producing the digits from $m - 1$ and up.  This essentially halves
the number of single precision multiplications required.  However, the optimization is only safe if $\beta$ is much larger than the number of digits
in the modulus.  In the source code this is evaluated on lines 36 to 44 where algorithm s\_mp\_mul\_high\_digs is used when it is
safe to do so.  

\subsection{The Barrett Setup Algorithm}
In order to use algorithm mp\_reduce the value of $\mu$ must be calculated in advance.  Ideally this value should be computed once and stored for
future use so that the Barrett algorithm can be used without delay.  

\newpage\begin{figure}[!here]
\begin{small}
\begin{center}
\begin{tabular}{l}
\hline Algorithm \textbf{mp\_reduce\_setup}. \\
\textbf{Input}.   mp\_int $a$ ($a > 1$)  \\
\textbf{Output}.  $\mu \leftarrow \lfloor \beta^{2m}/a \rfloor$ \\
\hline \\
1.  $\mu \leftarrow 2^{2 \cdot lg(\beta) \cdot  m}$ (\textit{mp\_2expt}) \\
2.  $\mu \leftarrow \lfloor \mu / b \rfloor$ (\textit{mp\_div}) \\
3.  Return(\textit{MP\_OKAY}) \\
\hline
\end{tabular}
\end{center}
\end{small}
\caption{Algorithm mp\_reduce\_setup}
\end{figure}

\textbf{Algorithm mp\_reduce\_setup.}
This algorithm computes the reciprocal $\mu$ required for Barrett reduction.  First $\beta^{2m}$ is calculated as $2^{2 \cdot lg(\beta) \cdot  m}$ which
is equivalent and much faster.  The final value is computed by taking the integer quotient of $\lfloor \mu / b \rfloor$.

\vspace{+3mm}\begin{small}
\hspace{-5.1mm}{\bf File}: bn\_mp\_reduce\_setup.c
\vspace{-3mm}
\begin{alltt}
\end{alltt}
\end{small}

This simple routine calculates the reciprocal $\mu$ required by Barrett reduction.  Note the extended usage of algorithm mp\_div where the variable
which would received the remainder is passed as NULL.  As will be discussed in~\ref{sec:division} the division routine allows both the quotient and the 
remainder to be passed as NULL meaning to ignore the value.  

\section{The Montgomery Reduction}
Montgomery reduction\footnote{Thanks to Niels Ferguson for his insightful explanation of the algorithm.} \cite{MONT} is by far the most interesting 
form of reduction in common use.  It computes a modular residue which is not actually equal to the residue of the input yet instead equal to a 
residue times a constant.  However, as perplexing as this may sound the algorithm is relatively simple and very efficient.  

Throughout this entire section the variable $n$ will represent the modulus used to form the residue.  As will be discussed shortly the value of
$n$ must be odd.  The variable $x$ will represent the quantity of which the residue is sought.  Similar to the Barrett algorithm the input
is restricted to $0 \le x < n^2$.  To begin the description some simple number theory facts must be established.

\textbf{Fact 1.}  Adding $n$ to $x$ does not change the residue since in effect it adds one to the quotient $\lfloor x / n \rfloor$.  Another way
to explain this is that $n$ is (\textit{or multiples of $n$ are}) congruent to zero modulo $n$.  Adding zero will not change the value of the residue.  

\textbf{Fact 2.}  If $x$ is even then performing a division by two in $\Z$ is congruent to $x \cdot 2^{-1} \mbox{ (mod }n\mbox{)}$.  Actually
this is an application of the fact that if $x$ is evenly divisible by any $k \in \Z$ then division in $\Z$ will be congruent to 
multiplication by $k^{-1}$ modulo $n$.  

From these two simple facts the following simple algorithm can be derived.

\newpage\begin{figure}[!here]
\begin{small}
\begin{center}
\begin{tabular}{l}
\hline Algorithm \textbf{Montgomery Reduction}. \\
\textbf{Input}.   Integer $x$, $n$ and $k$ \\
\textbf{Output}.  $2^{-k}x \mbox{ (mod }n\mbox{)}$ \\
\hline \\
1.  for $t$ from $1$ to $k$ do \\
\hspace{3mm}1.1  If $x$ is odd then \\
\hspace{6mm}1.1.1  $x \leftarrow x + n$ \\
\hspace{3mm}1.2  $x \leftarrow x/2$ \\
2.  Return $x$. \\
\hline
\end{tabular}
\end{center}
\end{small}
\caption{Algorithm Montgomery Reduction}
\end{figure}

The algorithm reduces the input one bit at a time using the two congruencies stated previously.  Inside the loop $n$, which is odd, is
added to $x$ if $x$ is odd.  This forces $x$ to be even which allows the division by two in $\Z$ to be congruent to a modular division by two.  Since
$x$ is assumed to be initially much larger than $n$ the addition of $n$ will contribute an insignificant magnitude to $x$.  Let $r$ represent the 
final result of the Montgomery algorithm.  If $k > lg(n)$ and $0 \le x < n^2$ then the final result is limited to 
$0 \le r < \lfloor x/2^k \rfloor + n$.  As a result at most a single subtraction is required to get the residue desired.

\begin{figure}[here]
\begin{small}
\begin{center}
\begin{tabular}{|c|l|}
\hline \textbf{Step number ($t$)} & \textbf{Result ($x$)} \\
\hline $1$ & $x + n = 5812$, $x/2 = 2906$ \\
\hline $2$ & $x/2 = 1453$ \\
\hline $3$ & $x + n = 1710$, $x/2 = 855$ \\
\hline $4$ & $x + n = 1112$, $x/2 = 556$ \\
\hline $5$ & $x/2 = 278$ \\
\hline $6$ & $x/2 = 139$ \\
\hline $7$ & $x + n = 396$, $x/2 = 198$ \\
\hline $8$ & $x/2 = 99$ \\
\hline $9$ & $x + n = 356$, $x/2 = 178$ \\
\hline
\end{tabular}
\end{center}
\end{small}
\caption{Example of Montgomery Reduction (I)}
\label{fig:MONT1}
\end{figure}

Consider the example in figure~\ref{fig:MONT1} which reduces $x = 5555$ modulo $n = 257$ when $k = 9$ (note $\beta^k = 512$ which is larger than $n$).  The result of 
the algorithm $r = 178$ is congruent to the value of $2^{-9} \cdot 5555 \mbox{ (mod }257\mbox{)}$.  When $r$ is multiplied by $2^9$ modulo $257$ the correct residue 
$r \equiv 158$ is produced.  

Let $k = \lfloor lg(n) \rfloor + 1$ represent the number of bits in $n$.  The current algorithm requires $2k^2$ single precision shifts
and $k^2$ single precision additions.  At this rate the algorithm is most certainly slower than Barrett reduction and not terribly useful.  
Fortunately there exists an alternative representation of the algorithm.

\begin{figure}[!here]
\begin{small}
\begin{center}
\begin{tabular}{l}
\hline Algorithm \textbf{Montgomery Reduction} (modified I). \\
\textbf{Input}.   Integer $x$, $n$ and $k$ ($2^k > n$) \\
\textbf{Output}.  $2^{-k}x \mbox{ (mod }n\mbox{)}$ \\
\hline \\
1.  for $t$ from $1$ to $k$ do \\
\hspace{3mm}1.1  If the $t$'th bit of $x$ is one then \\
\hspace{6mm}1.1.1  $x \leftarrow x + 2^tn$ \\
2.  Return $x/2^k$. \\
\hline
\end{tabular}
\end{center}
\end{small}
\caption{Algorithm Montgomery Reduction (modified I)}
\end{figure}

This algorithm is equivalent since $2^tn$ is a multiple of $n$ and the lower $k$ bits of $x$ are zero by step 2.  The number of single
precision shifts has now been reduced from $2k^2$ to $k^2 + k$ which is only a small improvement.

\begin{figure}[here]
\begin{small}
\begin{center}
\begin{tabular}{|c|l|r|}
\hline \textbf{Step number ($t$)} & \textbf{Result ($x$)} & \textbf{Result ($x$) in Binary} \\
\hline -- & $5555$ & $1010110110011$ \\
\hline $1$ & $x + 2^{0}n = 5812$ &  $1011010110100$ \\
\hline $2$ & $5812$ & $1011010110100$ \\
\hline $3$ & $x + 2^{2}n = 6840$ & $1101010111000$ \\
\hline $4$ & $x + 2^{3}n = 8896$ & $10001011000000$ \\
\hline $5$ & $8896$ & $10001011000000$ \\
\hline $6$ & $8896$ & $10001011000000$ \\
\hline $7$ & $x + 2^{6}n = 25344$ & $110001100000000$ \\
\hline $8$ & $25344$ & $110001100000000$ \\
\hline $9$ & $x + 2^{7}n = 91136$ & $10110010000000000$ \\
\hline -- & $x/2^k = 178$ & \\
\hline
\end{tabular}
\end{center}
\end{small}
\caption{Example of Montgomery Reduction (II)}
\label{fig:MONT2}
\end{figure}

Figure~\ref{fig:MONT2} demonstrates the modified algorithm reducing $x = 5555$ modulo $n = 257$ with $k = 9$. 
With this algorithm a single shift right at the end is the only right shift required to reduce the input instead of $k$ right shifts inside the 
loop.  Note that for the iterations $t = 2, 5, 6$ and $8$ where the result $x$ is not changed.  In those iterations the $t$'th bit of $x$ is 
zero and the appropriate multiple of $n$ does not need to be added to force the $t$'th bit of the result to zero.  

\subsection{Digit Based Montgomery Reduction}
Instead of computing the reduction on a bit-by-bit basis it is actually much faster to compute it on digit-by-digit basis.  Consider the
previous algorithm re-written to compute the Montgomery reduction in this new fashion.

\begin{figure}[!here]
\begin{small}
\begin{center}
\begin{tabular}{l}
\hline Algorithm \textbf{Montgomery Reduction} (modified II). \\
\textbf{Input}.   Integer $x$, $n$ and $k$ ($\beta^k > n$) \\
\textbf{Output}.  $\beta^{-k}x \mbox{ (mod }n\mbox{)}$ \\
\hline \\
1.  for $t$ from $0$ to $k - 1$ do \\
\hspace{3mm}1.1  $x \leftarrow x + \mu n \beta^t$ \\
2.  Return $x/\beta^k$. \\
\hline
\end{tabular}
\end{center}
\end{small}
\caption{Algorithm Montgomery Reduction (modified II)}
\end{figure}

The value $\mu n \beta^t$ is a multiple of the modulus $n$ meaning that it will not change the residue.  If the first digit of 
the value $\mu n \beta^t$ equals the negative (modulo $\beta$) of the $t$'th digit of $x$ then the addition will result in a zero digit.  This
problem breaks down to solving the following congruency.  

\begin{center}
\begin{tabular}{rcl}
$x_t + \mu n_0$ & $\equiv$ & $0 \mbox{ (mod }\beta\mbox{)}$ \\
$\mu n_0$ & $\equiv$ & $-x_t \mbox{ (mod }\beta\mbox{)}$ \\
$\mu$ & $\equiv$ & $-x_t/n_0 \mbox{ (mod }\beta\mbox{)}$ \\
\end{tabular}
\end{center}

In each iteration of the loop on step 1 a new value of $\mu$ must be calculated.  The value of $-1/n_0 \mbox{ (mod }\beta\mbox{)}$ is used 
extensively in this algorithm and should be precomputed.  Let $\rho$ represent the negative of the modular inverse of $n_0$ modulo $\beta$.  

For example, let $\beta = 10$ represent the radix.  Let $n = 17$ represent the modulus which implies $k = 2$ and $\rho \equiv 7$.  Let $x = 33$ 
represent the value to reduce.

\newpage\begin{figure}
\begin{center}
\begin{tabular}{|c|c|c|}
\hline \textbf{Step ($t$)} & \textbf{Value of $x$} & \textbf{Value of $\mu$} \\
\hline --                 & $33$ & --\\
\hline $0$                 & $33 + \mu n = 50$ & $1$ \\
\hline $1$                 & $50 + \mu n \beta = 900$ & $5$ \\
\hline
\end{tabular}
\end{center}
\caption{Example of Montgomery Reduction}
\end{figure}

The final result $900$ is then divided by $\beta^k$ to produce the final result $9$.  The first observation is that $9 \nequiv x \mbox{ (mod }n\mbox{)}$ 
which implies the result is not the modular residue of $x$ modulo $n$.  However, recall that the residue is actually multiplied by $\beta^{-k}$ in
the algorithm.  To get the true residue the value must be multiplied by $\beta^k$.  In this case $\beta^k \equiv 15 \mbox{ (mod }n\mbox{)}$ and
the correct residue is $9 \cdot 15 \equiv 16 \mbox{ (mod }n\mbox{)}$.  

\subsection{Baseline Montgomery Reduction}
The baseline Montgomery reduction algorithm will produce the residue for any size input.  It is designed to be a catch-all algororithm for 
Montgomery reductions.  

\newpage\begin{figure}[!here]
\begin{small}
\begin{center}
\begin{tabular}{l}
\hline Algorithm \textbf{mp\_montgomery\_reduce}. \\
\textbf{Input}.   mp\_int $x$, mp\_int $n$ and a digit $\rho \equiv -1/n_0 \mbox{ (mod }n\mbox{)}$. \\
\hspace{11.5mm}($0 \le x < n^2, n > 1, (n, \beta) = 1, \beta^k > n$) \\
\textbf{Output}.  $\beta^{-k}x \mbox{ (mod }n\mbox{)}$ \\
\hline \\
1.  $digs \leftarrow 2n.used + 1$ \\
2.  If $digs < MP\_ARRAY$ and $m.used < \delta$ then \\
\hspace{3mm}2.1  Use algorithm fast\_mp\_montgomery\_reduce instead. \\
\\
Setup $x$ for the reduction. \\
3.  If $x.alloc < digs$ then grow $x$ to $digs$ digits. \\
4.  $x.used \leftarrow digs$ \\
\\
Eliminate the lower $k$ digits. \\
5.  For $ix$ from $0$ to $k - 1$ do \\
\hspace{3mm}5.1  $\mu \leftarrow x_{ix} \cdot \rho \mbox{ (mod }\beta\mbox{)}$ \\
\hspace{3mm}5.2  $u \leftarrow 0$ \\
\hspace{3mm}5.3  For $iy$ from $0$ to $k - 1$ do \\
\hspace{6mm}5.3.1  $\hat r \leftarrow \mu n_{iy} + x_{ix + iy} + u$ \\
\hspace{6mm}5.3.2  $x_{ix + iy} \leftarrow \hat r \mbox{ (mod }\beta\mbox{)}$ \\
\hspace{6mm}5.3.3  $u \leftarrow \lfloor \hat r / \beta \rfloor$ \\
\hspace{3mm}5.4  While $u > 0$ do \\
\hspace{6mm}5.4.1  $iy \leftarrow iy + 1$ \\
\hspace{6mm}5.4.2  $x_{ix + iy} \leftarrow x_{ix + iy} + u$ \\
\hspace{6mm}5.4.3  $u \leftarrow \lfloor x_{ix+iy} / \beta \rfloor$ \\
\hspace{6mm}5.4.4  $x_{ix + iy} \leftarrow x_{ix+iy} \mbox{ (mod }\beta\mbox{)}$ \\
\\
Divide by $\beta^k$ and fix up as required. \\
6.  $x \leftarrow \lfloor x / \beta^k \rfloor$ \\
7.  If $x \ge n$ then \\
\hspace{3mm}7.1  $x \leftarrow x - n$ \\
8.  Return(\textit{MP\_OKAY}). \\
\hline
\end{tabular}
\end{center}
\end{small}
\caption{Algorithm mp\_montgomery\_reduce}
\end{figure}

\textbf{Algorithm mp\_montgomery\_reduce.}
This algorithm reduces the input $x$ modulo $n$ in place using the Montgomery reduction algorithm.  The algorithm is loosely based
on algorithm 14.32 of \cite[pp.601]{HAC} except it merges the multiplication of $\mu n \beta^t$ with the addition in the inner loop.  The
restrictions on this algorithm are fairly easy to adapt to.  First $0 \le x < n^2$ bounds the input to numbers in the same range as 
for the Barrett algorithm.  Additionally if $n > 1$ and $n$ is odd there will exist a modular inverse $\rho$.  $\rho$ must be calculated in
advance of this algorithm.  Finally the variable $k$ is fixed and a pseudonym for $n.used$.  

Step 2 decides whether a faster Montgomery algorithm can be used.  It is based on the Comba technique meaning that there are limits on
the size of the input.  This algorithm is discussed in sub-section 6.3.3.

Step 5 is the main reduction loop of the algorithm.  The value of $\mu$ is calculated once per iteration in the outer loop.  The inner loop
calculates $x + \mu n \beta^{ix}$ by multiplying $\mu n$ and adding the result to $x$ shifted by $ix$ digits.  Both the addition and
multiplication are performed in the same loop to save time and memory.  Step 5.4 will handle any additional carries that escape the inner loop.

Using a quick inspection this algorithm requires $n$ single precision multiplications for the outer loop and $n^2$ single precision multiplications 
in the inner loop.  In total $n^2 + n$ single precision multiplications which compares favourably to Barrett at $n^2 + 2n - 1$ single precision
multiplications.  

\vspace{+3mm}\begin{small}
\hspace{-5.1mm}{\bf File}: bn\_mp\_montgomery\_reduce.c
\vspace{-3mm}
\begin{alltt}
\end{alltt}
\end{small}

This is the baseline implementation of the Montgomery reduction algorithm.  Lines 31 to 36 determine if the Comba based
routine can be used instead.  Line 47 computes the value of $\mu$ for that particular iteration of the outer loop.  

The multiplication $\mu n \beta^{ix}$ is performed in one step in the inner loop.  The alias $tmpx$ refers to the $ix$'th digit of $x$ and
the alias $tmpn$ refers to the modulus $n$.  

\subsection{Faster ``Comba'' Montgomery Reduction}

The Montgomery reduction requires fewer single precision multiplications than a Barrett reduction, however it is much slower due to the serial
nature of the inner loop.  The Barrett reduction algorithm requires two slightly modified multipliers which can be implemented with the Comba
technique.  The Montgomery reduction algorithm cannot directly use the Comba technique to any significant advantage since the inner loop calculates
a $k \times 1$ product $k$ times. 

The biggest obstacle is that at the $ix$'th iteration of the outer loop the value of $x_{ix}$ is required to calculate $\mu$.  This means the 
carries from $0$ to $ix - 1$ must have been propagated upwards to form a valid $ix$'th digit.  The solution as it turns out is very simple.  
Perform a Comba like multiplier and inside the outer loop just after the inner loop fix up the $ix + 1$'th digit by forwarding the carry.  

With this change in place the Montgomery reduction algorithm can be performed with a Comba style multiplication loop which substantially increases
the speed of the algorithm.  

\newpage\begin{figure}[!here]
\begin{small}
\begin{center}
\begin{tabular}{l}
\hline Algorithm \textbf{fast\_mp\_montgomery\_reduce}. \\
\textbf{Input}.   mp\_int $x$, mp\_int $n$ and a digit $\rho \equiv -1/n_0 \mbox{ (mod }n\mbox{)}$. \\
\hspace{11.5mm}($0 \le x < n^2, n > 1, (n, \beta) = 1, \beta^k > n$) \\
\textbf{Output}.  $\beta^{-k}x \mbox{ (mod }n\mbox{)}$ \\
\hline \\
Place an array of \textbf{MP\_WARRAY} mp\_word variables called $\hat W$ on the stack. \\
1.  if $x.alloc < n.used + 1$ then grow $x$ to $n.used + 1$ digits. \\
Copy the digits of $x$ into the array $\hat W$ \\
2.  For $ix$ from $0$ to $x.used - 1$ do \\
\hspace{3mm}2.1  $\hat W_{ix} \leftarrow x_{ix}$ \\
3.  For $ix$ from $x.used$ to $2n.used - 1$ do \\
\hspace{3mm}3.1  $\hat W_{ix} \leftarrow 0$ \\
Elimiate the lower $k$ digits. \\
4.  for $ix$ from $0$ to $n.used - 1$ do \\
\hspace{3mm}4.1  $\mu \leftarrow \hat W_{ix} \cdot \rho \mbox{ (mod }\beta\mbox{)}$ \\
\hspace{3mm}4.2  For $iy$ from $0$ to $n.used - 1$ do \\
\hspace{6mm}4.2.1  $\hat W_{iy + ix} \leftarrow \hat W_{iy + ix} + \mu \cdot n_{iy}$ \\
\hspace{3mm}4.3  $\hat W_{ix + 1} \leftarrow \hat W_{ix + 1} + \lfloor \hat W_{ix} / \beta \rfloor$ \\
Propagate carries upwards. \\
5.  for $ix$ from $n.used$ to $2n.used + 1$ do \\
\hspace{3mm}5.1  $\hat W_{ix + 1} \leftarrow \hat W_{ix + 1} + \lfloor \hat W_{ix} / \beta \rfloor$ \\
Shift right and reduce modulo $\beta$ simultaneously. \\
6.  for $ix$ from $0$ to $n.used + 1$ do \\
\hspace{3mm}6.1  $x_{ix} \leftarrow \hat W_{ix + n.used} \mbox{ (mod }\beta\mbox{)}$ \\
Zero excess digits and fixup $x$. \\
7.  if $x.used > n.used + 1$ then do \\
\hspace{3mm}7.1  for $ix$ from $n.used + 1$ to $x.used - 1$ do \\
\hspace{6mm}7.1.1  $x_{ix} \leftarrow 0$ \\
8.  $x.used \leftarrow n.used + 1$ \\
9.  Clamp excessive digits of $x$. \\
10.  If $x \ge n$ then \\
\hspace{3mm}10.1  $x \leftarrow x - n$ \\
11.  Return(\textit{MP\_OKAY}). \\
\hline
\end{tabular}
\end{center}
\end{small}
\caption{Algorithm fast\_mp\_montgomery\_reduce}
\end{figure}

\textbf{Algorithm fast\_mp\_montgomery\_reduce.}
This algorithm will compute the Montgomery reduction of $x$ modulo $n$ using the Comba technique.  It is on most computer platforms significantly
faster than algorithm mp\_montgomery\_reduce and algorithm mp\_reduce (\textit{Barrett reduction}).  The algorithm has the same restrictions
on the input as the baseline reduction algorithm.  An additional two restrictions are imposed on this algorithm.  The number of digits $k$ in the 
the modulus $n$ must not violate $MP\_WARRAY > 2k +1$ and $n < \delta$.   When $\beta = 2^{28}$ this algorithm can be used to reduce modulo
a modulus of at most $3,556$ bits in length.  

As in the other Comba reduction algorithms there is a $\hat W$ array which stores the columns of the product.  It is initially filled with the
contents of $x$ with the excess digits zeroed.  The reduction loop is very similar the to the baseline loop at heart.  The multiplication on step
4.1 can be single precision only since $ab \mbox{ (mod }\beta\mbox{)} \equiv (a \mbox{ mod }\beta)(b \mbox{ mod }\beta)$.  Some multipliers such
as those on the ARM processors take a variable length time to complete depending on the number of bytes of result it must produce.  By performing
a single precision multiplication instead half the amount of time is spent.

Also note that digit $\hat W_{ix}$ must have the carry from the $ix - 1$'th digit propagated upwards in order for this to work.  That is what step
4.3 will do.  In effect over the $n.used$ iterations of the outer loop the $n.used$'th lower columns all have the their carries propagated forwards.  Note
how the upper bits of those same words are not reduced modulo $\beta$.  This is because those values will be discarded shortly and there is no
point.

Step 5 will propagate the remainder of the carries upwards.  On step 6 the columns are reduced modulo $\beta$ and shifted simultaneously as they are
stored in the destination $x$.  

\vspace{+3mm}\begin{small}
\hspace{-5.1mm}{\bf File}: bn\_fast\_mp\_montgomery\_reduce.c
\vspace{-3mm}
\begin{alltt}
\end{alltt}
\end{small}

The $\hat W$ array is first filled with digits of $x$ on line 48 then the rest of the digits are zeroed on line 55.  Both loops share
the same alias variables to make the code easier to read.  

The value of $\mu$ is calculated in an interesting fashion.  First the value $\hat W_{ix}$ is reduced modulo $\beta$ and cast to a mp\_digit.  This
forces the compiler to use a single precision multiplication and prevents any concerns about loss of precision.   Line 110 fixes the carry 
for the next iteration of the loop by propagating the carry from $\hat W_{ix}$ to $\hat W_{ix+1}$.

The for loop on line 109 propagates the rest of the carries upwards through the columns.  The for loop on line 126 reduces the columns
modulo $\beta$ and shifts them $k$ places at the same time.  The alias $\_ \hat W$ actually refers to the array $\hat W$ starting at the $n.used$'th
digit, that is $\_ \hat W_{t} = \hat W_{n.used + t}$.  

\subsection{Montgomery Setup}
To calculate the variable $\rho$ a relatively simple algorithm will be required.  

\begin{figure}[!here]
\begin{small}
\begin{center}
\begin{tabular}{l}
\hline Algorithm \textbf{mp\_montgomery\_setup}. \\
\textbf{Input}.   mp\_int $n$ ($n > 1$ and $(n, 2) = 1$) \\
\textbf{Output}.  $\rho \equiv -1/n_0 \mbox{ (mod }\beta\mbox{)}$ \\
\hline \\
1.  $b \leftarrow n_0$ \\
2.  If $b$ is even return(\textit{MP\_VAL}) \\
3.  $x \leftarrow (((b + 2) \mbox{ AND } 4) << 1) + b$ \\
4.  for $k$ from 0 to $\lceil lg(lg(\beta)) \rceil - 2$ do \\
\hspace{3mm}4.1  $x \leftarrow x \cdot (2 - bx)$ \\
5.  $\rho \leftarrow \beta - x \mbox{ (mod }\beta\mbox{)}$ \\
6.  Return(\textit{MP\_OKAY}). \\
\hline
\end{tabular}
\end{center}
\end{small}
\caption{Algorithm mp\_montgomery\_setup} 
\end{figure}

\textbf{Algorithm mp\_montgomery\_setup.}
This algorithm will calculate the value of $\rho$ required within the Montgomery reduction algorithms.  It uses a very interesting trick 
to calculate $1/n_0$ when $\beta$ is a power of two.  

\vspace{+3mm}\begin{small}
\hspace{-5.1mm}{\bf File}: bn\_mp\_montgomery\_setup.c
\vspace{-3mm}
\begin{alltt}
\end{alltt}
\end{small}

This source code computes the value of $\rho$ required to perform Montgomery reduction.  It has been modified to avoid performing excess
multiplications when $\beta$ is not the default 28-bits.  

\section{The Diminished Radix Algorithm}
The Diminished Radix method of modular reduction \cite{DRMET} is a fairly clever technique which can be more efficient than either the Barrett
or Montgomery methods for certain forms of moduli.  The technique is based on the following simple congruence.

\begin{equation}
(x \mbox{ mod } n) + k \lfloor x / n \rfloor \equiv x \mbox{ (mod }(n - k)\mbox{)}
\end{equation}

This observation was used in the MMB \cite{MMB} block cipher to create a diffusion primitive.  It used the fact that if $n = 2^{31}$ and $k=1$ that 
then a x86 multiplier could produce the 62-bit product and use  the ``shrd'' instruction to perform a double-precision right shift.  The proof
of the above equation is very simple.  First write $x$ in the product form.

\begin{equation}
x = qn + r
\end{equation}

Now reduce both sides modulo $(n - k)$.

\begin{equation}
x \equiv qk + r  \mbox{ (mod }(n-k)\mbox{)}
\end{equation}

The variable $n$ reduces modulo $n - k$ to $k$.  By putting $q = \lfloor x/n \rfloor$ and $r = x \mbox{ mod } n$ 
into the equation the original congruence is reproduced, thus concluding the proof.  The following algorithm is based on this observation.

\begin{figure}[!here]
\begin{small}
\begin{center}
\begin{tabular}{l}
\hline Algorithm \textbf{Diminished Radix Reduction}. \\
\textbf{Input}.   Integer $x$, $n$, $k$ \\
\textbf{Output}.  $x \mbox{ mod } (n - k)$ \\
\hline \\
1.  $q \leftarrow \lfloor x / n \rfloor$ \\
2.  $q \leftarrow k \cdot q$ \\
3.  $x \leftarrow x \mbox{ (mod }n\mbox{)}$ \\
4.  $x \leftarrow x + q$ \\
5.  If $x \ge (n - k)$ then \\
\hspace{3mm}5.1  $x \leftarrow x - (n - k)$ \\
\hspace{3mm}5.2  Goto step 1. \\
6.  Return $x$ \\
\hline
\end{tabular}
\end{center}
\end{small}
\caption{Algorithm Diminished Radix Reduction}
\label{fig:DR}
\end{figure}

This algorithm will reduce $x$ modulo $n - k$ and return the residue.  If $0 \le x < (n - k)^2$ then the algorithm will loop almost always
once or twice and occasionally three times.  For simplicity sake the value of $x$ is bounded by the following simple polynomial.

\begin{equation} 
0 \le x < n^2 + k^2 - 2nk
\end{equation}

The true bound is  $0 \le x < (n - k - 1)^2$ but this has quite a few more terms.  The value of $q$ after step 1 is bounded by the following.

\begin{equation}
q < n - 2k - k^2/n
\end{equation}

Since $k^2$ is going to be considerably smaller than $n$ that term will always be zero.  The value of $x$ after step 3 is bounded trivially as
$0 \le x < n$.  By step four the sum $x + q$ is bounded by 

\begin{equation}
0 \le q + x < (k + 1)n - 2k^2 - 1
\end{equation}

With a second pass $q$ will be loosely bounded by $0 \le q < k^2$ after step 2 while $x$ will still be loosely bounded by $0 \le x < n$ after step 3.  After the second pass it is highly unlike that the
sum in step 4 will exceed $n - k$.  In practice fewer than three passes of the algorithm are required to reduce virtually every input in the 
range $0 \le x < (n - k - 1)^2$.  

\begin{figure}
\begin{small}
\begin{center}
\begin{tabular}{|l|}
\hline
$x = 123456789, n = 256, k = 3$ \\
\hline $q \leftarrow \lfloor x/n \rfloor = 482253$ \\
$q \leftarrow q*k = 1446759$ \\
$x \leftarrow x \mbox{ mod } n = 21$ \\
$x \leftarrow x + q = 1446780$ \\
$x \leftarrow x - (n - k) = 1446527$ \\
\hline 
$q \leftarrow \lfloor x/n \rfloor = 5650$ \\
$q \leftarrow q*k = 16950$ \\
$x \leftarrow x \mbox{ mod } n = 127$ \\
$x \leftarrow x + q = 17077$ \\
$x \leftarrow x - (n - k) = 16824$ \\
\hline 
$q \leftarrow \lfloor x/n \rfloor = 65$ \\
$q \leftarrow q*k = 195$ \\
$x \leftarrow x \mbox{ mod } n = 184$ \\
$x \leftarrow x + q = 379$ \\
$x \leftarrow x - (n - k) = 126$ \\
\hline
\end{tabular}
\end{center}
\end{small}
\caption{Example Diminished Radix Reduction}
\label{fig:EXDR}
\end{figure}

Figure~\ref{fig:EXDR} demonstrates the reduction of $x = 123456789$ modulo $n - k = 253$ when $n = 256$ and $k = 3$.  Note that even while $x$
is considerably larger than $(n - k - 1)^2 = 63504$ the algorithm still converges on the modular residue exceedingly fast.  In this case only
three passes were required to find the residue $x \equiv 126$.


\subsection{Choice of Moduli}
On the surface this algorithm looks like a very expensive algorithm.  It requires a couple of subtractions followed by multiplication and other
modular reductions.  The usefulness of this algorithm becomes exceedingly clear when an appropriate modulus is chosen.

Division in general is a very expensive operation to perform.  The one exception is when the division is by a power of the radix of representation used.  
Division by ten for example is simple for pencil and paper mathematics since it amounts to shifting the decimal place to the right.  Similarly division 
by two (\textit{or powers of two}) is very simple for binary computers to perform.  It would therefore seem logical to choose $n$ of the form $2^p$ 
which would imply that $\lfloor x / n \rfloor$ is a simple shift of $x$ right $p$ bits.  

However, there is one operation related to division of power of twos that is even faster than this.  If $n = \beta^p$ then the division may be 
performed by moving whole digits to the right $p$ places.  In practice division by $\beta^p$ is much faster than division by $2^p$ for any $p$.  
Also with the choice of $n = \beta^p$ reducing $x$ modulo $n$ merely requires zeroing the digits above the $p-1$'th digit of $x$.  

Throughout the next section the term ``restricted modulus'' will refer to a modulus of the form $\beta^p - k$ whereas the term ``unrestricted
modulus'' will refer to a modulus of the form $2^p - k$.  The word ``restricted'' in this case refers to the fact that it is based on the 
$2^p$ logic except $p$ must be a multiple of $lg(\beta)$.  

\subsection{Choice of $k$}
Now that division and reduction (\textit{step 1 and 3 of figure~\ref{fig:DR}}) have been optimized to simple digit operations the multiplication by $k$
in step 2 is the most expensive operation.  Fortunately the choice of $k$ is not terribly limited.  For all intents and purposes it might
as well be a single digit.  The smaller the value of $k$ is the faster the algorithm will be.  

\subsection{Restricted Diminished Radix Reduction}
The restricted Diminished Radix algorithm can quickly reduce an input modulo a modulus of the form $n = \beta^p - k$.  This algorithm can reduce 
an input $x$ within the range $0 \le x < n^2$ using only a couple passes of the algorithm demonstrated in figure~\ref{fig:DR}.  The implementation
of this algorithm has been optimized to avoid additional overhead associated with a division by $\beta^p$, the multiplication by $k$ or the addition 
of $x$ and $q$.  The resulting algorithm is very efficient and can lead to substantial improvements over Barrett and Montgomery reduction when modular 
exponentiations are performed.

\newpage\begin{figure}[!here]
\begin{small}
\begin{center}
\begin{tabular}{l}
\hline Algorithm \textbf{mp\_dr\_reduce}. \\
\textbf{Input}.   mp\_int $x$, $n$ and a mp\_digit $k = \beta - n_0$ \\
\hspace{11.5mm}($0 \le x < n^2$, $n > 1$, $0 < k < \beta$) \\
\textbf{Output}.  $x \mbox{ mod } n$ \\
\hline \\
1.  $m \leftarrow n.used$ \\
2.  If $x.alloc < 2m$ then grow $x$ to $2m$ digits. \\
3.  $\mu \leftarrow 0$ \\
4.  for $i$ from $0$ to $m - 1$ do \\
\hspace{3mm}4.1  $\hat r \leftarrow k \cdot x_{m+i} + x_{i} + \mu$ \\
\hspace{3mm}4.2  $x_{i} \leftarrow \hat r \mbox{ (mod }\beta\mbox{)}$ \\
\hspace{3mm}4.3  $\mu \leftarrow \lfloor \hat r / \beta \rfloor$ \\
5.  $x_{m} \leftarrow \mu$ \\
6.  for $i$ from $m + 1$ to $x.used - 1$ do \\
\hspace{3mm}6.1  $x_{i} \leftarrow 0$ \\
7.  Clamp excess digits of $x$. \\
8.  If $x \ge n$ then \\
\hspace{3mm}8.1  $x \leftarrow x - n$ \\
\hspace{3mm}8.2  Goto step 3. \\
9.  Return(\textit{MP\_OKAY}). \\
\hline
\end{tabular}
\end{center}
\end{small}
\caption{Algorithm mp\_dr\_reduce}
\end{figure}

\textbf{Algorithm mp\_dr\_reduce.}
This algorithm will perform the Dimished Radix reduction of $x$ modulo $n$.  It has similar restrictions to that of the Barrett reduction
with the addition that $n$ must be of the form $n = \beta^m - k$ where $0 < k <\beta$.  

This algorithm essentially implements the pseudo-code in figure~\ref{fig:DR} except with a slight optimization.  The division by $\beta^m$, multiplication by $k$
and addition of $x \mbox{ mod }\beta^m$ are all performed simultaneously inside the loop on step 4.  The division by $\beta^m$ is emulated by accessing
the term at the $m+i$'th position which is subsequently multiplied by $k$ and added to the term at the $i$'th position.  After the loop the $m$'th
digit is set to the carry and the upper digits are zeroed.  Steps 5 and 6 emulate the reduction modulo $\beta^m$ that should have happend to 
$x$ before the addition of the multiple of the upper half.  

At step 8 if $x$ is still larger than $n$ another pass of the algorithm is required.  First $n$ is subtracted from $x$ and then the algorithm resumes
at step 3.  

\vspace{+3mm}\begin{small}
\hspace{-5.1mm}{\bf File}: bn\_mp\_dr\_reduce.c
\vspace{-3mm}
\begin{alltt}
\end{alltt}
\end{small}

The first step is to grow $x$ as required to $2m$ digits since the reduction is performed in place on $x$.  The label on line 52 is where
the algorithm will resume if further reduction passes are required.  In theory it could be placed at the top of the function however, the size of
the modulus and question of whether $x$ is large enough are invariant after the first pass meaning that it would be a waste of time.  

The aliases $tmpx1$ and $tmpx2$ refer to the digits of $x$ where the latter is offset by $m$ digits.  By reading digits from $x$ offset by $m$ digits
a division by $\beta^m$ can be simulated virtually for free.  The loop on line 64 performs the bulk of the work (\textit{corresponds to step 4 of algorithm 7.11})
in this algorithm.

By line 67 the pointer $tmpx1$ points to the $m$'th digit of $x$ which is where the final carry will be placed.  Similarly by line 74 the 
same pointer will point to the $m+1$'th digit where the zeroes will be placed.  

Since the algorithm is only valid if both $x$ and $n$ are greater than zero an unsigned comparison suffices to determine if another pass is required.  
With the same logic at line 81 the value of $x$ is known to be greater than or equal to $n$ meaning that an unsigned subtraction can be used
as well.  Since the destination of the subtraction is the larger of the inputs the call to algorithm s\_mp\_sub cannot fail and the return code
does not need to be checked.

\subsubsection{Setup}
To setup the restricted Diminished Radix algorithm the value $k = \beta - n_0$ is required.  This algorithm is not really complicated but provided for
completeness.

\begin{figure}[!here]
\begin{small}
\begin{center}
\begin{tabular}{l}
\hline Algorithm \textbf{mp\_dr\_setup}. \\
\textbf{Input}.   mp\_int $n$ \\
\textbf{Output}.  $k = \beta - n_0$ \\
\hline \\
1.  $k \leftarrow \beta - n_0$ \\
\hline
\end{tabular}
\end{center}
\end{small}
\caption{Algorithm mp\_dr\_setup}
\end{figure}

\vspace{+3mm}\begin{small}
\hspace{-5.1mm}{\bf File}: bn\_mp\_dr\_setup.c
\vspace{-3mm}
\begin{alltt}
\end{alltt}
\end{small}

\subsubsection{Modulus Detection}
Another algorithm which will be useful is the ability to detect a restricted Diminished Radix modulus.  An integer is said to be
of restricted Diminished Radix form if all of the digits are equal to $\beta - 1$ except the trailing digit which may be any value.

\begin{figure}[!here]
\begin{small}
\begin{center}
\begin{tabular}{l}
\hline Algorithm \textbf{mp\_dr\_is\_modulus}. \\
\textbf{Input}.   mp\_int $n$ \\
\textbf{Output}.  $1$ if $n$ is in D.R form, $0$ otherwise \\
\hline
1.  If $n.used < 2$ then return($0$). \\
2.  for $ix$ from $1$ to $n.used - 1$ do \\
\hspace{3mm}2.1  If $n_{ix} \ne \beta - 1$ return($0$). \\
3.  Return($1$). \\
\hline
\end{tabular}
\end{center}
\end{small}
\caption{Algorithm mp\_dr\_is\_modulus}
\end{figure}

\textbf{Algorithm mp\_dr\_is\_modulus.}
This algorithm determines if a value is in Diminished Radix form.  Step 1 rejects obvious cases where fewer than two digits are
in the mp\_int.  Step 2 tests all but the first digit to see if they are equal to $\beta - 1$.  If the algorithm manages to get to
step 3 then $n$ must be of Diminished Radix form.

\vspace{+3mm}\begin{small}
\hspace{-5.1mm}{\bf File}: bn\_mp\_dr\_is\_modulus.c
\vspace{-3mm}
\begin{alltt}
\end{alltt}
\end{small}

\subsection{Unrestricted Diminished Radix Reduction}
The unrestricted Diminished Radix algorithm allows modular reductions to be performed when the modulus is of the form $2^p - k$.  This algorithm
is a straightforward adaptation of algorithm~\ref{fig:DR}.

In general the restricted Diminished Radix reduction algorithm is much faster since it has considerably lower overhead.  However, this new
algorithm is much faster than either Montgomery or Barrett reduction when the moduli are of the appropriate form.

\begin{figure}[!here]
\begin{small}
\begin{center}
\begin{tabular}{l}
\hline Algorithm \textbf{mp\_reduce\_2k}. \\
\textbf{Input}.   mp\_int $a$ and $n$.  mp\_digit $k$  \\
\hspace{11.5mm}($a \ge 0$, $n > 1$, $0 < k < \beta$, $n + k$ is a power of two) \\
\textbf{Output}.  $a \mbox{ (mod }n\mbox{)}$ \\
\hline
1.  $p \leftarrow \lceil lg(n) \rceil$  (\textit{mp\_count\_bits}) \\
2.  While $a \ge n$ do \\
\hspace{3mm}2.1  $q \leftarrow \lfloor a / 2^p \rfloor$ (\textit{mp\_div\_2d}) \\
\hspace{3mm}2.2  $a \leftarrow a \mbox{ (mod }2^p\mbox{)}$ (\textit{mp\_mod\_2d}) \\
\hspace{3mm}2.3  $q \leftarrow q \cdot k$ (\textit{mp\_mul\_d}) \\
\hspace{3mm}2.4  $a \leftarrow a - q$ (\textit{s\_mp\_sub}) \\
\hspace{3mm}2.5  If $a \ge n$ then do \\
\hspace{6mm}2.5.1  $a \leftarrow a - n$ \\
3.  Return(\textit{MP\_OKAY}). \\
\hline
\end{tabular}
\end{center}
\end{small}
\caption{Algorithm mp\_reduce\_2k}
\end{figure}

\textbf{Algorithm mp\_reduce\_2k.}
This algorithm quickly reduces an input $a$ modulo an unrestricted Diminished Radix modulus $n$.  Division by $2^p$ is emulated with a right
shift which makes the algorithm fairly inexpensive to use.  

\vspace{+3mm}\begin{small}
\hspace{-5.1mm}{\bf File}: bn\_mp\_reduce\_2k.c
\vspace{-3mm}
\begin{alltt}
\end{alltt}
\end{small}

The algorithm mp\_count\_bits calculates the number of bits in an mp\_int which is used to find the initial value of $p$.  The call to mp\_div\_2d
on line 31 calculates both the quotient $q$ and the remainder $a$ required.  By doing both in a single function call the code size
is kept fairly small.  The multiplication by $k$ is only performed if $k > 1$. This allows reductions modulo $2^p - 1$ to be performed without
any multiplications.  

The unsigned s\_mp\_add, mp\_cmp\_mag and s\_mp\_sub are used in place of their full sign counterparts since the inputs are only valid if they are 
positive.  By using the unsigned versions the overhead is kept to a minimum.  

\subsubsection{Unrestricted Setup}
To setup this reduction algorithm the value of $k = 2^p - n$ is required.  

\begin{figure}[!here]
\begin{small}
\begin{center}
\begin{tabular}{l}
\hline Algorithm \textbf{mp\_reduce\_2k\_setup}. \\
\textbf{Input}.   mp\_int $n$   \\
\textbf{Output}.  $k = 2^p - n$ \\
\hline
1.  $p \leftarrow \lceil lg(n) \rceil$  (\textit{mp\_count\_bits}) \\
2.  $x \leftarrow 2^p$ (\textit{mp\_2expt}) \\
3.  $x \leftarrow x - n$ (\textit{mp\_sub}) \\
4.  $k \leftarrow x_0$ \\
5.  Return(\textit{MP\_OKAY}). \\
\hline
\end{tabular}
\end{center}
\end{small}
\caption{Algorithm mp\_reduce\_2k\_setup}
\end{figure}

\textbf{Algorithm mp\_reduce\_2k\_setup.}
This algorithm computes the value of $k$ required for the algorithm mp\_reduce\_2k.  By making a temporary variable $x$ equal to $2^p$ a subtraction
is sufficient to solve for $k$.  Alternatively if $n$ has more than one digit the value of $k$ is simply $\beta - n_0$.  

\vspace{+3mm}\begin{small}
\hspace{-5.1mm}{\bf File}: bn\_mp\_reduce\_2k\_setup.c
\vspace{-3mm}
\begin{alltt}
\end{alltt}
\end{small}

\subsubsection{Unrestricted Detection}
An integer $n$ is a valid unrestricted Diminished Radix modulus if either of the following are true.

\begin{enumerate}
\item  The number has only one digit.
\item  The number has more than one digit and every bit from the $\beta$'th to the most significant is one.
\end{enumerate}

If either condition is true than there is a power of two $2^p$ such that $0 < 2^p - n < \beta$.   If the input is only
one digit than it will always be of the correct form.  Otherwise all of the bits above the first digit must be one.  This arises from the fact
that there will be value of $k$ that when added to the modulus causes a carry in the first digit which propagates all the way to the most
significant bit.  The resulting sum will be a power of two.

\begin{figure}[!here]
\begin{small}
\begin{center}
\begin{tabular}{l}
\hline Algorithm \textbf{mp\_reduce\_is\_2k}. \\
\textbf{Input}.   mp\_int $n$   \\
\textbf{Output}.  $1$ if of proper form, $0$ otherwise \\
\hline
1.  If $n.used = 0$ then return($0$). \\
2.  If $n.used = 1$ then return($1$). \\
3.  $p \leftarrow \lceil lg(n) \rceil$  (\textit{mp\_count\_bits}) \\
4.  for $x$ from $lg(\beta)$ to $p$ do \\
\hspace{3mm}4.1  If the ($x \mbox{ mod }lg(\beta)$)'th bit of the $\lfloor x / lg(\beta) \rfloor$ of $n$ is zero then return($0$). \\
5.  Return($1$). \\
\hline
\end{tabular}
\end{center}
\end{small}
\caption{Algorithm mp\_reduce\_is\_2k}
\end{figure}

\textbf{Algorithm mp\_reduce\_is\_2k.}
This algorithm quickly determines if a modulus is of the form required for algorithm mp\_reduce\_2k to function properly.  

\vspace{+3mm}\begin{small}
\hspace{-5.1mm}{\bf File}: bn\_mp\_reduce\_is\_2k.c
\vspace{-3mm}
\begin{alltt}
\end{alltt}
\end{small}



\section{Algorithm Comparison}
So far three very different algorithms for modular reduction have been discussed.  Each of the algorithms have their own strengths and weaknesses
that makes having such a selection very useful.  The following table sumarizes the three algorithms along with comparisons of work factors.  Since
all three algorithms have the restriction that $0 \le x < n^2$ and $n > 1$ those limitations are not included in the table.  

\begin{center}
\begin{small}
\begin{tabular}{|c|c|c|c|c|c|}
\hline \textbf{Method} & \textbf{Work Required} & \textbf{Limitations} & \textbf{$m = 8$} & \textbf{$m = 32$} & \textbf{$m = 64$} \\
\hline Barrett    & $m^2 + 2m - 1$ & None              & $79$ & $1087$ & $4223$ \\
\hline Montgomery & $m^2 + m$      & $n$ must be odd   & $72$ & $1056$ & $4160$ \\
\hline D.R.       & $2m$           & $n = \beta^m - k$ & $16$ & $64$   & $128$  \\
\hline
\end{tabular}
\end{small}
\end{center}

In theory Montgomery and Barrett reductions would require roughly the same amount of time to complete.  However, in practice since Montgomery
reduction can be written as a single function with the Comba technique it is much faster.  Barrett reduction suffers from the overhead of
calling the half precision multipliers, addition and division by $\beta$ algorithms.

For almost every cryptographic algorithm Montgomery reduction is the algorithm of choice.  The one set of algorithms where Diminished Radix reduction truly
shines are based on the discrete logarithm problem such as Diffie-Hellman \cite{DH} and ElGamal \cite{ELGAMAL}.  In these algorithms
primes of the form $\beta^m - k$ can be found and shared amongst users.  These primes will allow the Diminished Radix algorithm to be used in
modular exponentiation to greatly speed up the operation.



\section*{Exercises}
\begin{tabular}{cl}
$\left [ 3 \right ]$ & Prove that the ``trick'' in algorithm mp\_montgomery\_setup actually \\
                     & calculates the correct value of $\rho$. \\
                     & \\
$\left [ 2 \right ]$ & Devise an algorithm to reduce modulo $n + k$ for small $k$ quickly.  \\
                     & \\
$\left [ 4 \right ]$ & Prove that the pseudo-code algorithm ``Diminished Radix Reduction'' \\
                     & (\textit{figure~\ref{fig:DR}}) terminates.  Also prove the probability that it will \\
                     & terminate within $1 \le k \le 10$ iterations. \\
                     & \\
\end{tabular}                     


\chapter{Exponentiation}
Exponentiation is the operation of raising one variable to the power of another, for example, $a^b$.  A variant of exponentiation, computed
in a finite field or ring, is called modular exponentiation.  This latter style of operation is typically used in public key 
cryptosystems such as RSA and Diffie-Hellman.  The ability to quickly compute modular exponentiations is of great benefit to any
such cryptosystem and many methods have been sought to speed it up.

\section{Exponentiation Basics}
A trivial algorithm would simply multiply $a$ against itself $b - 1$ times to compute the exponentiation desired.  However, as $b$ grows in size
the number of multiplications becomes prohibitive.  Imagine what would happen if $b$ $\approx$ $2^{1024}$ as is the case when computing an RSA signature
with a $1024$-bit key.  Such a calculation could never be completed as it would take simply far too long.

Fortunately there is a very simple algorithm based on the laws of exponents.  Recall that $lg_a(a^b) = b$ and that $lg_a(a^ba^c) = b + c$ which
are two trivial relationships between the base and the exponent.  Let $b_i$ represent the $i$'th bit of $b$ starting from the least 
significant bit.  If $b$ is a $k$-bit integer than the following equation is true.

\begin{equation}
a^b = \prod_{i=0}^{k-1} a^{2^i \cdot b_i}
\end{equation}

By taking the base $a$ logarithm of both sides of the equation the following equation is the result.

\begin{equation}
b = \sum_{i=0}^{k-1}2^i \cdot b_i
\end{equation}

The term $a^{2^i}$ can be found from the $i - 1$'th term by squaring the term since $\left ( a^{2^i} \right )^2$ is equal to
$a^{2^{i+1}}$.  This observation forms the basis of essentially all fast exponentiation algorithms.  It requires $k$ squarings and on average
$k \over 2$ multiplications to compute the result.  This is indeed quite an improvement over simply multiplying by $a$ a total of $b-1$ times.

While this current method is a considerable speed up there are further improvements to be made.  For example, the $a^{2^i}$ term does not need to 
be computed in an auxilary variable.  Consider the following equivalent algorithm.

\begin{figure}[!here]
\begin{small}
\begin{center}
\begin{tabular}{l}
\hline Algorithm \textbf{Left to Right Exponentiation}. \\
\textbf{Input}.   Integer $a$, $b$ and $k$ \\
\textbf{Output}.  $c = a^b$ \\
\hline \\
1.  $c \leftarrow 1$ \\
2.  for $i$ from $k - 1$ to $0$ do \\
\hspace{3mm}2.1  $c \leftarrow c^2$ \\
\hspace{3mm}2.2  $c \leftarrow c \cdot a^{b_i}$ \\
3.  Return $c$. \\
\hline
\end{tabular}
\end{center}
\end{small}
\caption{Left to Right Exponentiation}
\label{fig:LTOR}
\end{figure}

This algorithm starts from the most significant bit and works towards the least significant bit.  When the $i$'th bit of $b$ is set $a$ is
multiplied against the current product.  In each iteration the product is squared which doubles the exponent of the individual terms of the
product.  

For example, let $b = 101100_2 \equiv 44_{10}$.  The following chart demonstrates the actions of the algorithm.

\newpage\begin{figure}
\begin{center}
\begin{tabular}{|c|c|}
\hline \textbf{Value of $i$} & \textbf{Value of $c$} \\
\hline - & $1$ \\
\hline $5$ & $a$ \\
\hline $4$ & $a^2$ \\
\hline $3$ & $a^4 \cdot a$ \\
\hline $2$ & $a^8 \cdot a^2 \cdot a$ \\
\hline $1$ & $a^{16} \cdot a^4 \cdot a^2$ \\
\hline $0$ & $a^{32} \cdot a^8 \cdot a^4$ \\
\hline
\end{tabular}
\end{center}
\caption{Example of Left to Right Exponentiation}
\end{figure}

When the product $a^{32} \cdot a^8 \cdot a^4$ is simplified it is equal $a^{44}$ which is the desired exponentiation.  This particular algorithm is 
called ``Left to Right'' because it reads the exponent in that order.  All of the exponentiation algorithms that will be presented are of this nature.  

\subsection{Single Digit Exponentiation}
The first algorithm in the series of exponentiation algorithms will be an unbounded algorithm where the exponent is a single digit.  It is intended 
to be used when a small power of an input is required (\textit{e.g. $a^5$}).  It is faster than simply multiplying $b - 1$ times for all values of 
$b$ that are greater than three.  

\newpage\begin{figure}[!here]
\begin{small}
\begin{center}
\begin{tabular}{l}
\hline Algorithm \textbf{mp\_expt\_d}. \\
\textbf{Input}.   mp\_int $a$ and mp\_digit $b$ \\
\textbf{Output}.  $c = a^b$ \\
\hline \\
1.  $g \leftarrow a$ (\textit{mp\_init\_copy}) \\
2.  $c \leftarrow 1$ (\textit{mp\_set}) \\
3.  for $x$ from 1 to $lg(\beta)$ do \\
\hspace{3mm}3.1  $c \leftarrow c^2$ (\textit{mp\_sqr}) \\
\hspace{3mm}3.2  If $b$ AND $2^{lg(\beta) - 1} \ne 0$ then \\
\hspace{6mm}3.2.1  $c \leftarrow c \cdot g$ (\textit{mp\_mul}) \\
\hspace{3mm}3.3  $b \leftarrow b << 1$ \\
4.  Clear $g$. \\
5.  Return(\textit{MP\_OKAY}). \\
\hline
\end{tabular}
\end{center}
\end{small}
\caption{Algorithm mp\_expt\_d}
\end{figure}

\textbf{Algorithm mp\_expt\_d.}
This algorithm computes the value of $a$ raised to the power of a single digit $b$.  It uses the left to right exponentiation algorithm to
quickly compute the exponentiation.  It is loosely based on algorithm 14.79 of HAC \cite[pp. 615]{HAC} with the difference that the 
exponent is a fixed width.  

A copy of $a$ is made first to allow destination variable $c$ be the same as the source variable $a$.  The result is set to the initial value of 
$1$ in the subsequent step.

Inside the loop the exponent is read from the most significant bit first down to the least significant bit.  First $c$ is invariably squared
on step 3.1.  In the following step if the most significant bit of $b$ is one the copy of $a$ is multiplied against $c$.  The value
of $b$ is shifted left one bit to make the next bit down from the most signficant bit the new most significant bit.  In effect each
iteration of the loop moves the bits of the exponent $b$ upwards to the most significant location.

\vspace{+3mm}\begin{small}
\hspace{-5.1mm}{\bf File}: bn\_mp\_expt\_d.c
\vspace{-3mm}
\begin{alltt}
\end{alltt}
\end{small}

Line 29 sets the initial value of the result to $1$.  Next the loop on line 31 steps through each bit of the exponent starting from
the most significant down towards the least significant. The invariant squaring operation placed on line 33 is performed first.  After 
the squaring the result $c$ is multiplied by the base $g$ if and only if the most significant bit of the exponent is set.  The shift on line
47 moves all of the bits of the exponent upwards towards the most significant location.  

\section{$k$-ary Exponentiation}
When calculating an exponentiation the most time consuming bottleneck is the multiplications which are in general a small factor
slower than squaring.  Recall from the previous algorithm that $b_{i}$ refers to the $i$'th bit of the exponent $b$.  Suppose instead it referred to
the $i$'th $k$-bit digit of the exponent of $b$.  For $k = 1$ the definitions are synonymous and for $k > 1$ algorithm~\ref{fig:KARY}
computes the same exponentiation.  A group of $k$ bits from the exponent is called a \textit{window}.  That is it is a small window on only a
portion of the entire exponent.  Consider the following modification to the basic left to right exponentiation algorithm.

\begin{figure}[!here]
\begin{small}
\begin{center}
\begin{tabular}{l}
\hline Algorithm \textbf{$k$-ary Exponentiation}. \\
\textbf{Input}.   Integer $a$, $b$, $k$ and $t$ \\
\textbf{Output}.  $c = a^b$ \\
\hline \\
1.  $c \leftarrow 1$ \\
2.  for $i$ from $t - 1$ to $0$ do \\
\hspace{3mm}2.1  $c \leftarrow c^{2^k} $ \\
\hspace{3mm}2.2  Extract the $i$'th $k$-bit word from $b$ and store it in $g$. \\
\hspace{3mm}2.3  $c \leftarrow c \cdot a^g$ \\
3.  Return $c$. \\
\hline
\end{tabular}
\end{center}
\end{small}
\caption{$k$-ary Exponentiation}
\label{fig:KARY}
\end{figure}

The squaring on step 2.1 can be calculated by squaring the value $c$ successively $k$ times.  If the values of $a^g$ for $0 < g < 2^k$ have been
precomputed this algorithm requires only $t$ multiplications and $tk$ squarings.  The table can be generated with $2^{k - 1} - 1$ squarings and
$2^{k - 1} + 1$ multiplications.  This algorithm assumes that the number of bits in the exponent is evenly divisible by $k$.  
However, when it is not the remaining $0 < x \le k - 1$ bits can be handled with algorithm~\ref{fig:LTOR}.

Suppose $k = 4$ and $t = 100$.  This modified algorithm will require $109$ multiplications and $408$ squarings to compute the exponentiation.  The
original algorithm would on average have required $200$ multiplications and $400$ squrings to compute the same value.  The total number of squarings
has increased slightly but the number of multiplications has nearly halved.

\subsection{Optimal Values of $k$}
An optimal value of $k$ will minimize $2^{k} + \lceil n / k \rceil + n - 1$ for a fixed number of bits in the exponent $n$.  The simplest
approach is to brute force search amongst the values $k = 2, 3, \ldots, 8$ for the lowest result.  Table~\ref{fig:OPTK} lists optimal values of $k$
for various exponent sizes and compares the number of multiplication and squarings required against algorithm~\ref{fig:LTOR}.  

\begin{figure}[here]
\begin{center}
\begin{small}
\begin{tabular}{|c|c|c|c|c|c|}
\hline \textbf{Exponent (bits)} & \textbf{Optimal $k$} & \textbf{Work at $k$} & \textbf{Work with ~\ref{fig:LTOR}} \\
\hline $16$ & $2$ & $27$ & $24$ \\
\hline $32$ & $3$ & $49$ & $48$ \\
\hline $64$ & $3$ & $92$ & $96$ \\
\hline $128$ & $4$ & $175$ & $192$ \\
\hline $256$ & $4$ & $335$ & $384$ \\
\hline $512$ & $5$ & $645$ & $768$ \\
\hline $1024$ & $6$ & $1257$ & $1536$ \\
\hline $2048$ & $6$ & $2452$ & $3072$ \\
\hline $4096$ & $7$ & $4808$ & $6144$ \\
\hline
\end{tabular}
\end{small}
\end{center}
\caption{Optimal Values of $k$ for $k$-ary Exponentiation}
\label{fig:OPTK}
\end{figure}

\subsection{Sliding-Window Exponentiation}
A simple modification to the previous algorithm is only generate the upper half of the table in the range $2^{k-1} \le g < 2^k$.  Essentially
this is a table for all values of $g$ where the most significant bit of $g$ is a one.  However, in order for this to be allowed in the 
algorithm values of $g$ in the range $0 \le g < 2^{k-1}$ must be avoided.  

Table~\ref{fig:OPTK2} lists optimal values of $k$ for various exponent sizes and compares the work required against algorithm~\ref{fig:KARY}.  

\begin{figure}[here]
\begin{center}
\begin{small}
\begin{tabular}{|c|c|c|c|c|c|}
\hline \textbf{Exponent (bits)} & \textbf{Optimal $k$} & \textbf{Work at $k$} & \textbf{Work with ~\ref{fig:KARY}} \\
\hline $16$ & $3$ & $24$ & $27$ \\
\hline $32$ & $3$ & $45$ & $49$ \\
\hline $64$ & $4$ & $87$ & $92$ \\
\hline $128$ & $4$ & $167$ & $175$ \\
\hline $256$ & $5$ & $322$ & $335$ \\
\hline $512$ & $6$ & $628$ & $645$ \\
\hline $1024$ & $6$ & $1225$ & $1257$ \\
\hline $2048$ & $7$ & $2403$ & $2452$ \\
\hline $4096$ & $8$ & $4735$ & $4808$ \\
\hline
\end{tabular}
\end{small}
\end{center}
\caption{Optimal Values of $k$ for Sliding Window Exponentiation}
\label{fig:OPTK2}
\end{figure}

\newpage\begin{figure}[!here]
\begin{small}
\begin{center}
\begin{tabular}{l}
\hline Algorithm \textbf{Sliding Window $k$-ary Exponentiation}. \\
\textbf{Input}.   Integer $a$, $b$, $k$ and $t$ \\
\textbf{Output}.  $c = a^b$ \\
\hline \\
1.  $c \leftarrow 1$ \\
2.  for $i$ from $t - 1$ to $0$ do \\
\hspace{3mm}2.1  If the $i$'th bit of $b$ is a zero then \\
\hspace{6mm}2.1.1   $c \leftarrow c^2$ \\
\hspace{3mm}2.2  else do \\
\hspace{6mm}2.2.1  $c \leftarrow c^{2^k}$ \\
\hspace{6mm}2.2.2  Extract the $k$ bits from $(b_{i}b_{i-1}\ldots b_{i-(k-1)})$ and store it in $g$. \\
\hspace{6mm}2.2.3  $c \leftarrow c \cdot a^g$ \\
\hspace{6mm}2.2.4  $i \leftarrow i - k$ \\
3.  Return $c$. \\
\hline
\end{tabular}
\end{center}
\end{small}
\caption{Sliding Window $k$-ary Exponentiation}
\end{figure}

Similar to the previous algorithm this algorithm must have a special handler when fewer than $k$ bits are left in the exponent.  While this
algorithm requires the same number of squarings it can potentially have fewer multiplications.  The pre-computed table $a^g$ is also half
the size as the previous table.  

Consider the exponent $b = 111101011001000_2 \equiv 31432_{10}$ with $k = 3$ using both algorithms.  The first algorithm will divide the exponent up as 
the following five $3$-bit words $b \equiv \left ( 111, 101, 011, 001, 000 \right )_{2}$.  The second algorithm will break the 
exponent as $b \equiv \left ( 111, 101, 0, 110, 0, 100, 0 \right )_{2}$.  The single digit $0$ in the second representation are where
a single squaring took place instead of a squaring and multiplication.  In total the first method requires $10$ multiplications and $18$ 
squarings.  The second method requires $8$ multiplications and $18$ squarings.  

In general the sliding window method is never slower than the generic $k$-ary method and often it is slightly faster.  

\section{Modular Exponentiation}

Modular exponentiation is essentially computing the power of a base within a finite field or ring.  For example, computing 
$d \equiv a^b \mbox{ (mod }c\mbox{)}$ is a modular exponentiation.  Instead of first computing $a^b$ and then reducing it 
modulo $c$ the intermediate result is reduced modulo $c$ after every squaring or multiplication operation.  

This guarantees that any intermediate result is bounded by $0 \le d \le c^2 - 2c + 1$ and can be reduced modulo $c$ quickly using
one of the algorithms presented in chapter six.  

Before the actual modular exponentiation algorithm can be written a wrapper algorithm must be written first.  This algorithm
will allow the exponent $b$ to be negative which is computed as $c \equiv \left (1 / a \right )^{\vert b \vert} \mbox{(mod }d\mbox{)}$. The
value of $(1/a) \mbox{ mod }c$ is computed using the modular inverse (\textit{see \ref{sec;modinv}}).  If no inverse exists the algorithm
terminates with an error.  

\begin{figure}[!here]
\begin{small}
\begin{center}
\begin{tabular}{l}
\hline Algorithm \textbf{mp\_exptmod}. \\
\textbf{Input}.   mp\_int $a$, $b$ and $c$ \\
\textbf{Output}.  $y \equiv g^x \mbox{ (mod }p\mbox{)}$ \\
\hline \\
1.  If $c.sign = MP\_NEG$ return(\textit{MP\_VAL}). \\
2.  If $b.sign = MP\_NEG$ then \\
\hspace{3mm}2.1  $g' \leftarrow g^{-1} \mbox{ (mod }c\mbox{)}$ \\
\hspace{3mm}2.2  $x' \leftarrow \vert x \vert$ \\
\hspace{3mm}2.3  Compute $d \equiv g'^{x'} \mbox{ (mod }c\mbox{)}$ via recursion. \\
3.  if $p$ is odd \textbf{OR} $p$ is a D.R. modulus then \\
\hspace{3mm}3.1  Compute $y \equiv g^{x} \mbox{ (mod }p\mbox{)}$ via algorithm mp\_exptmod\_fast. \\
4.  else \\
\hspace{3mm}4.1  Compute $y \equiv g^{x} \mbox{ (mod }p\mbox{)}$ via algorithm s\_mp\_exptmod. \\
\hline
\end{tabular}
\end{center}
\end{small}
\caption{Algorithm mp\_exptmod}
\end{figure}

\textbf{Algorithm mp\_exptmod.}
The first algorithm which actually performs modular exponentiation is algorithm s\_mp\_exptmod.  It is a sliding window $k$-ary algorithm 
which uses Barrett reduction to reduce the product modulo $p$.  The second algorithm mp\_exptmod\_fast performs the same operation 
except it uses either Montgomery or Diminished Radix reduction.  The two latter reduction algorithms are clumped in the same exponentiation
algorithm since their arguments are essentially the same (\textit{two mp\_ints and one mp\_digit}).  

\vspace{+3mm}\begin{small}
\hspace{-5.1mm}{\bf File}: bn\_mp\_exptmod.c
\vspace{-3mm}
\begin{alltt}
\end{alltt}
\end{small}

In order to keep the algorithms in a known state the first step on line 29 is to reject any negative modulus as input.  If the exponent is
negative the algorithm tries to perform a modular exponentiation with the modular inverse of the base $G$.  The temporary variable $tmpG$ is assigned
the modular inverse of $G$ and $tmpX$ is assigned the absolute value of $X$.  The algorithm will recuse with these new values with a positive
exponent.

If the exponent is positive the algorithm resumes the exponentiation.  Line 77 determines if the modulus is of the restricted Diminished Radix 
form.  If it is not line 70 attempts to determine if it is of a unrestricted Diminished Radix form.  The integer $dr$ will take on one
of three values.

\begin{enumerate}
\item $dr = 0$ means that the modulus is not of either restricted or unrestricted Diminished Radix form.
\item $dr = 1$ means that the modulus is of restricted Diminished Radix form.
\item $dr = 2$ means that the modulus is of unrestricted Diminished Radix form.
\end{enumerate}

Line 69 determines if the fast modular exponentiation algorithm can be used.  It is allowed if $dr \ne 0$ or if the modulus is odd.  Otherwise,
the slower s\_mp\_exptmod algorithm is used which uses Barrett reduction.  

\subsection{Barrett Modular Exponentiation}

\newpage\begin{figure}[!here]
\begin{small}
\begin{center}
\begin{tabular}{l}
\hline Algorithm \textbf{s\_mp\_exptmod}. \\
\textbf{Input}.   mp\_int $a$, $b$ and $c$ \\
\textbf{Output}.  $y \equiv g^x \mbox{ (mod }p\mbox{)}$ \\
\hline \\
1.  $k \leftarrow lg(x)$ \\
2.  $winsize \leftarrow  \left \lbrace \begin{array}{ll}
                              2 &  \mbox{if }k \le 7 \\
                              3 &  \mbox{if }7 < k \le 36 \\
                              4 &  \mbox{if }36 < k \le 140 \\
                              5 &  \mbox{if }140 < k \le 450 \\
                              6 &  \mbox{if }450 < k \le 1303 \\
                              7 &  \mbox{if }1303 < k \le 3529 \\
                              8 &  \mbox{if }3529 < k \\
                              \end{array} \right .$ \\
3.  Initialize $2^{winsize}$ mp\_ints in an array named $M$ and one mp\_int named $\mu$ \\
4.  Calculate the $\mu$ required for Barrett Reduction (\textit{mp\_reduce\_setup}). \\
5.  $M_1 \leftarrow g \mbox{ (mod }p\mbox{)}$ \\
\\
Setup the table of small powers of $g$.  First find $g^{2^{winsize}}$ and then all multiples of it. \\
6.  $k \leftarrow 2^{winsize - 1}$ \\
7.  $M_{k} \leftarrow M_1$ \\
8.  for $ix$ from 0 to $winsize - 2$ do \\
\hspace{3mm}8.1  $M_k \leftarrow \left ( M_k \right )^2$ (\textit{mp\_sqr})  \\
\hspace{3mm}8.2  $M_k \leftarrow M_k \mbox{ (mod }p\mbox{)}$ (\textit{mp\_reduce}) \\
9.  for $ix$ from $2^{winsize - 1} + 1$ to $2^{winsize} - 1$ do \\
\hspace{3mm}9.1  $M_{ix} \leftarrow M_{ix - 1} \cdot M_{1}$ (\textit{mp\_mul}) \\
\hspace{3mm}9.2  $M_{ix} \leftarrow M_{ix} \mbox{ (mod }p\mbox{)}$ (\textit{mp\_reduce}) \\
10.  $res \leftarrow 1$ \\
\\
Start Sliding Window. \\
11.  $mode \leftarrow 0, bitcnt \leftarrow 1, buf \leftarrow 0, digidx \leftarrow x.used - 1, bitcpy \leftarrow 0, bitbuf \leftarrow 0$ \\
12.  Loop \\
\hspace{3mm}12.1  $bitcnt \leftarrow bitcnt - 1$ \\
\hspace{3mm}12.2  If $bitcnt = 0$ then do \\
\hspace{6mm}12.2.1  If $digidx = -1$ goto step 13. \\
\hspace{6mm}12.2.2  $buf \leftarrow x_{digidx}$ \\
\hspace{6mm}12.2.3  $digidx \leftarrow digidx - 1$ \\
\hspace{6mm}12.2.4  $bitcnt \leftarrow lg(\beta)$ \\
Continued on next page. \\
\hline
\end{tabular}
\end{center}
\end{small}
\caption{Algorithm s\_mp\_exptmod}
\end{figure}

\newpage\begin{figure}[!here]
\begin{small}
\begin{center}
\begin{tabular}{l}
\hline Algorithm \textbf{s\_mp\_exptmod} (\textit{continued}). \\
\textbf{Input}.   mp\_int $a$, $b$ and $c$ \\
\textbf{Output}.  $y \equiv g^x \mbox{ (mod }p\mbox{)}$ \\
\hline \\
\hspace{3mm}12.3  $y \leftarrow (buf >> (lg(\beta) - 1))$ AND $1$ \\
\hspace{3mm}12.4  $buf \leftarrow buf << 1$ \\
\hspace{3mm}12.5  if $mode = 0$ and $y = 0$ then goto step 12. \\
\hspace{3mm}12.6  if $mode = 1$ and $y = 0$ then do \\
\hspace{6mm}12.6.1  $res \leftarrow res^2$ \\
\hspace{6mm}12.6.2  $res \leftarrow res \mbox{ (mod }p\mbox{)}$ \\
\hspace{6mm}12.6.3  Goto step 12. \\
\hspace{3mm}12.7  $bitcpy \leftarrow bitcpy + 1$ \\
\hspace{3mm}12.8  $bitbuf \leftarrow bitbuf + (y << (winsize - bitcpy))$ \\
\hspace{3mm}12.9  $mode \leftarrow 2$ \\
\hspace{3mm}12.10  If $bitcpy = winsize$ then do \\
\hspace{6mm}Window is full so perform the squarings and single multiplication. \\
\hspace{6mm}12.10.1  for $ix$ from $0$ to $winsize -1$ do \\
\hspace{9mm}12.10.1.1  $res \leftarrow res^2$ \\
\hspace{9mm}12.10.1.2  $res \leftarrow res \mbox{ (mod }p\mbox{)}$ \\
\hspace{6mm}12.10.2  $res \leftarrow res \cdot M_{bitbuf}$ \\
\hspace{6mm}12.10.3  $res \leftarrow res \mbox{ (mod }p\mbox{)}$ \\
\hspace{6mm}Reset the window. \\
\hspace{6mm}12.10.4  $bitcpy \leftarrow 0, bitbuf \leftarrow 0, mode \leftarrow 1$ \\
\\
No more windows left.  Check for residual bits of exponent. \\
13.  If $mode = 2$ and $bitcpy > 0$ then do \\
\hspace{3mm}13.1  for $ix$ form $0$ to $bitcpy - 1$ do \\
\hspace{6mm}13.1.1  $res \leftarrow res^2$ \\
\hspace{6mm}13.1.2  $res \leftarrow res \mbox{ (mod }p\mbox{)}$ \\
\hspace{6mm}13.1.3  $bitbuf \leftarrow bitbuf << 1$ \\
\hspace{6mm}13.1.4  If $bitbuf$ AND $2^{winsize} \ne 0$ then do \\
\hspace{9mm}13.1.4.1  $res \leftarrow res \cdot M_{1}$ \\
\hspace{9mm}13.1.4.2  $res \leftarrow res \mbox{ (mod }p\mbox{)}$ \\
14.  $y \leftarrow res$ \\
15.  Clear $res$, $mu$ and the $M$ array. \\
16.  Return(\textit{MP\_OKAY}). \\
\hline
\end{tabular}
\end{center}
\end{small}
\caption{Algorithm s\_mp\_exptmod (continued)}
\end{figure}

\textbf{Algorithm s\_mp\_exptmod.}
This algorithm computes the $x$'th power of $g$ modulo $p$ and stores the result in $y$.  It takes advantage of the Barrett reduction
algorithm to keep the product small throughout the algorithm.

The first two steps determine the optimal window size based on the number of bits in the exponent.  The larger the exponent the 
larger the window size becomes.  After a window size $winsize$ has been chosen an array of $2^{winsize}$ mp\_int variables is allocated.  This
table will hold the values of $g^x \mbox{ (mod }p\mbox{)}$ for $2^{winsize - 1} \le x < 2^{winsize}$.  

After the table is allocated the first power of $g$ is found.  Since $g \ge p$ is allowed it must be first reduced modulo $p$ to make
the rest of the algorithm more efficient.  The first element of the table at $2^{winsize - 1}$ is found by squaring $M_1$ successively $winsize - 2$
times.  The rest of the table elements are found by multiplying the previous element by $M_1$ modulo $p$.

Now that the table is available the sliding window may begin.  The following list describes the functions of all the variables in the window.
\begin{enumerate}
\item The variable $mode$ dictates how the bits of the exponent are interpreted.  
\begin{enumerate}
   \item When $mode = 0$ the bits are ignored since no non-zero bit of the exponent has been seen yet.  For example, if the exponent were simply 
         $1$ then there would be $lg(\beta) - 1$ zero bits before the first non-zero bit.  In this case bits are ignored until a non-zero bit is found.  
   \item When $mode = 1$ a non-zero bit has been seen before and a new $winsize$-bit window has not been formed yet.  In this mode leading $0$ bits 
         are read and a single squaring is performed.  If a non-zero bit is read a new window is created.  
   \item When $mode = 2$ the algorithm is in the middle of forming a window and new bits are appended to the window from the most significant bit
         downwards.
\end{enumerate}
\item The variable $bitcnt$ indicates how many bits are left in the current digit of the exponent left to be read.  When it reaches zero a new digit
      is fetched from the exponent.
\item The variable $buf$ holds the currently read digit of the exponent. 
\item The variable $digidx$ is an index into the exponents digits.  It starts at the leading digit $x.used - 1$ and moves towards the trailing digit.
\item The variable $bitcpy$ indicates how many bits are in the currently formed window.  When it reaches $winsize$ the window is flushed and
      the appropriate operations performed.
\item The variable $bitbuf$ holds the current bits of the window being formed.  
\end{enumerate}

All of step 12 is the window processing loop.  It will iterate while there are digits available form the exponent to read.  The first step
inside this loop is to extract a new digit if no more bits are available in the current digit.  If there are no bits left a new digit is
read and if there are no digits left than the loop terminates.  

After a digit is made available step 12.3 will extract the most significant bit of the current digit and move all other bits in the digit
upwards.  In effect the digit is read from most significant bit to least significant bit and since the digits are read from leading to 
trailing edges the entire exponent is read from most significant bit to least significant bit.

At step 12.5 if the $mode$ and currently extracted bit $y$ are both zero the bit is ignored and the next bit is read.  This prevents the 
algorithm from having to perform trivial squaring and reduction operations before the first non-zero bit is read.  Step 12.6 and 12.7-10 handle
the two cases of $mode = 1$ and $mode = 2$ respectively.  

\begin{center}
\begin{figure}[here]
\includegraphics{pics/expt_state.ps}
\caption{Sliding Window State Diagram}
\label{pic:expt_state}
\end{figure}
\end{center}

By step 13 there are no more digits left in the exponent.  However, there may be partial bits in the window left.  If $mode = 2$ then 
a Left-to-Right algorithm is used to process the remaining few bits.  

\vspace{+3mm}\begin{small}
\hspace{-5.1mm}{\bf File}: bn\_s\_mp\_exptmod.c
\vspace{-3mm}
\begin{alltt}
\end{alltt}
\end{small}

Lines 32 through 46 determine the optimal window size based on the length of the exponent in bits.  The window divisions are sorted
from smallest to greatest so that in each \textbf{if} statement only one condition must be tested.  For example, by the \textbf{if} statement 
on line 38 the value of $x$ is already known to be greater than $140$.  

The conditional piece of code beginning on line 48 allows the window size to be restricted to five bits.  This logic is used to ensure
the table of precomputed powers of $G$ remains relatively small.  

The for loop on line 61 initializes the $M$ array while lines 72 and 77 through 86 initialize the reduction
function that will be used for this modulus.

-- More later.

\section{Quick Power of Two}
Calculating $b = 2^a$ can be performed much quicker than with any of the previous algorithms.  Recall that a logical shift left $m << k$ is
equivalent to $m \cdot 2^k$.  By this logic when $m = 1$ a quick power of two can be achieved.

\begin{figure}[!here]
\begin{small}
\begin{center}
\begin{tabular}{l}
\hline Algorithm \textbf{mp\_2expt}. \\
\textbf{Input}.   integer $b$ \\
\textbf{Output}.  $a \leftarrow 2^b$ \\
\hline \\
1.  $a \leftarrow 0$ \\
2.  If $a.alloc < \lfloor b / lg(\beta) \rfloor + 1$ then grow $a$ appropriately. \\
3.  $a.used \leftarrow \lfloor b / lg(\beta) \rfloor + 1$ \\
4.  $a_{\lfloor b / lg(\beta) \rfloor} \leftarrow 1 << (b \mbox{ mod } lg(\beta))$ \\
5.  Return(\textit{MP\_OKAY}). \\
\hline
\end{tabular}
\end{center}
\end{small}
\caption{Algorithm mp\_2expt}
\end{figure}

\textbf{Algorithm mp\_2expt.}

\vspace{+3mm}\begin{small}
\hspace{-5.1mm}{\bf File}: bn\_mp\_2expt.c
\vspace{-3mm}
\begin{alltt}
\end{alltt}
\end{small}

\chapter{Higher Level Algorithms}

This chapter discusses the various higher level algorithms that are required to complete a well rounded multiple precision integer package.  These
routines are less performance oriented than the algorithms of chapters five, six and seven but are no less important.  

The first section describes a method of integer division with remainder that is universally well known.  It provides the signed division logic
for the package.  The subsequent section discusses a set of algorithms which allow a single digit to be the 2nd operand for a variety of operations.  
These algorithms serve mostly to simplify other algorithms where small constants are required.  The last two sections discuss how to manipulate 
various representations of integers.  For example, converting from an mp\_int to a string of character.

\section{Integer Division with Remainder}
\label{sec:division}

Integer division aside from modular exponentiation is the most intensive algorithm to compute.  Like addition, subtraction and multiplication
the basis of this algorithm is the long-hand division algorithm taught to school children.  Throughout this discussion several common variables
will be used.  Let $x$ represent the divisor and $y$ represent the dividend.  Let $q$ represent the integer quotient $\lfloor y / x \rfloor$ and 
let $r$ represent the remainder $r = y - x \lfloor y / x \rfloor$.  The following simple algorithm will be used to start the discussion.

\newpage\begin{figure}[!here]
\begin{small}
\begin{center}
\begin{tabular}{l}
\hline Algorithm \textbf{Radix-$\beta$ Integer Division}. \\
\textbf{Input}.   integer $x$ and $y$ \\
\textbf{Output}.  $q = \lfloor y/x\rfloor, r = y - xq$ \\
\hline \\
1.  $q \leftarrow 0$ \\
2.  $n \leftarrow \vert \vert y \vert \vert - \vert \vert x \vert \vert$ \\
3.  for $t$ from $n$ down to $0$ do \\
\hspace{3mm}3.1  Maximize $k$ such that $kx\beta^t$ is less than or equal to $y$ and $(k + 1)x\beta^t$ is greater. \\
\hspace{3mm}3.2  $q \leftarrow q + k\beta^t$ \\
\hspace{3mm}3.3  $y \leftarrow y - kx\beta^t$ \\
4.  $r \leftarrow y$ \\
5.  Return($q, r$) \\
\hline
\end{tabular}
\end{center}
\end{small}
\caption{Algorithm Radix-$\beta$ Integer Division}
\label{fig:raddiv}
\end{figure}

As children we are taught this very simple algorithm for the case of $\beta = 10$.  Almost instinctively several optimizations are taught for which
their reason of existing are never explained.  For this example let $y = 5471$ represent the dividend and $x = 23$ represent the divisor.

To find the first digit of the quotient the value of $k$ must be maximized such that $kx\beta^t$ is less than or equal to $y$ and 
simultaneously $(k + 1)x\beta^t$ is greater than $y$.  Implicitly $k$ is the maximum value the $t$'th digit of the quotient may have.  The habitual method
used to find the maximum is to ``eyeball'' the two numbers, typically only the leading digits and quickly estimate a quotient.  By only using leading
digits a much simpler division may be used to form an educated guess at what the value must be.  In this case $k = \lfloor 54/23\rfloor = 2$ quickly 
arises as a possible  solution.  Indeed $2x\beta^2 = 4600$ is less than $y = 5471$ and simultaneously $(k + 1)x\beta^2 = 6900$ is larger than $y$.  
As a  result $k\beta^2$ is added to the quotient which now equals $q = 200$ and $4600$ is subtracted from $y$ to give a remainder of $y = 841$.

Again this process is repeated to produce the quotient digit $k = 3$ which makes the quotient $q = 200 + 3\beta = 230$ and the remainder 
$y = 841 - 3x\beta = 181$.  Finally the last iteration of the loop produces $k = 7$ which leads to the quotient $q = 230 + 7 = 237$ and the
remainder $y = 181 - 7x = 20$.  The final quotient and remainder found are $q = 237$ and $r = y = 20$ which are indeed correct since 
$237 \cdot 23 + 20 = 5471$ is true.  

\subsection{Quotient Estimation}
\label{sec:divest}
As alluded to earlier the quotient digit $k$ can be estimated from only the leading digits of both the divisor and dividend.  When $p$ leading
digits are used from both the divisor and dividend to form an estimation the accuracy of the estimation rises as $p$ grows.  Technically
speaking the estimation is based on assuming the lower $\vert \vert y \vert \vert - p$ and $\vert \vert x \vert \vert - p$ lower digits of the
dividend and divisor are zero.  

The value of the estimation may off by a few values in either direction and in general is fairly correct.  A simplification \cite[pp. 271]{TAOCPV2}
of the estimation technique is to use $t + 1$ digits of the dividend and $t$ digits of the divisor, in particularly when $t = 1$.  The estimate 
using this technique is never too small.  For the following proof let $t = \vert \vert y \vert \vert - 1$ and $s = \vert \vert x \vert \vert - 1$ 
represent the most significant digits of the dividend and divisor respectively.

\textbf{Proof.}\textit{  The quotient $\hat k = \lfloor (y_t\beta + y_{t-1}) / x_s \rfloor$ is greater than or equal to 
$k = \lfloor y / (x \cdot \beta^{\vert \vert y \vert \vert - \vert \vert x \vert \vert - 1}) \rfloor$. }
The first obvious case is when $\hat k = \beta - 1$ in which case the proof is concluded since the real quotient cannot be larger.  For all other 
cases $\hat k = \lfloor (y_t\beta + y_{t-1}) / x_s \rfloor$ and $\hat k x_s \ge y_t\beta + y_{t-1} - x_s + 1$.  The latter portion of the inequalility
$-x_s + 1$ arises from the fact that a truncated integer division will give the same quotient for at most $x_s - 1$ values.  Next a series of 
inequalities will prove the hypothesis.

\begin{equation}
y - \hat k x \le y - \hat k x_s\beta^s
\end{equation}

This is trivially true since $x \ge x_s\beta^s$.  Next we replace $\hat kx_s\beta^s$ by the previous inequality for $\hat kx_s$.  

\begin{equation}
y - \hat k x \le y_t\beta^t + \ldots + y_0 - (y_t\beta^t + y_{t-1}\beta^{t-1} - x_s\beta^t + \beta^s)
\end{equation}

By simplifying the previous inequality the following inequality is formed.

\begin{equation}
y - \hat k x \le y_{t-2}\beta^{t-2} + \ldots + y_0 + x_s\beta^s - \beta^s
\end{equation}

Subsequently,

\begin{equation}
y_{t-2}\beta^{t-2} + \ldots +  y_0  + x_s\beta^s - \beta^s < x_s\beta^s \le x
\end{equation}

Which proves that $y - \hat kx \le x$ and by consequence $\hat k \ge k$ which concludes the proof.  \textbf{QED}


\subsection{Normalized Integers}
For the purposes of division a normalized input is when the divisors leading digit $x_n$ is greater than or equal to $\beta / 2$.  By multiplying both
$x$ and $y$ by $j = \lfloor (\beta / 2) / x_n \rfloor$ the quotient remains unchanged and the remainder is simply $j$ times the original
remainder.  The purpose of normalization is to ensure the leading digit of the divisor is sufficiently large such that the estimated quotient will
lie in the domain of a single digit.  Consider the maximum dividend $(\beta - 1) \cdot \beta + (\beta - 1)$ and the minimum divisor $\beta / 2$.  

\begin{equation} 
{{\beta^2 - 1} \over { \beta / 2}} \le 2\beta - {2 \over \beta} 
\end{equation}

At most the quotient approaches $2\beta$, however, in practice this will not occur since that would imply the previous quotient digit was too small.  

\subsection{Radix-$\beta$ Division with Remainder}
\newpage\begin{figure}[!here]
\begin{small}
\begin{center}
\begin{tabular}{l}
\hline Algorithm \textbf{mp\_div}. \\
\textbf{Input}.   mp\_int $a, b$ \\
\textbf{Output}.  $c = \lfloor a/b \rfloor$, $d = a - bc$ \\
\hline \\
1.  If $b = 0$ return(\textit{MP\_VAL}). \\
2.  If $\vert a \vert < \vert b \vert$ then do \\
\hspace{3mm}2.1  $d \leftarrow a$ \\
\hspace{3mm}2.2  $c \leftarrow 0$ \\
\hspace{3mm}2.3  Return(\textit{MP\_OKAY}). \\
\\
Setup the quotient to receive the digits. \\
3.  Grow $q$ to $a.used + 2$ digits. \\
4.  $q \leftarrow 0$ \\
5.  $x \leftarrow \vert a \vert , y \leftarrow \vert b \vert$ \\
6.  $sign \leftarrow  \left \lbrace \begin{array}{ll}
                              MP\_ZPOS &  \mbox{if }a.sign = b.sign \\
                              MP\_NEG  &  \mbox{otherwise} \\
                              \end{array} \right .$ \\
\\
Normalize the inputs such that the leading digit of $y$ is greater than or equal to $\beta / 2$. \\
7.  $norm \leftarrow (lg(\beta) - 1) - (\lceil lg(y) \rceil \mbox{ (mod }lg(\beta)\mbox{)})$ \\
8.  $x \leftarrow x \cdot 2^{norm}, y \leftarrow y \cdot 2^{norm}$ \\
\\
Find the leading digit of the quotient. \\
9.  $n \leftarrow x.used - 1, t \leftarrow y.used - 1$ \\
10.  $y \leftarrow y \cdot \beta^{n - t}$ \\
11.  While ($x \ge y$) do \\
\hspace{3mm}11.1  $q_{n - t} \leftarrow q_{n - t} + 1$ \\
\hspace{3mm}11.2  $x \leftarrow x - y$ \\
12.  $y \leftarrow \lfloor y / \beta^{n-t} \rfloor$ \\
\\
Continued on the next page. \\
\hline
\end{tabular}
\end{center}
\end{small}
\caption{Algorithm mp\_div}
\end{figure}

\newpage\begin{figure}[!here]
\begin{small}
\begin{center}
\begin{tabular}{l}
\hline Algorithm \textbf{mp\_div} (continued). \\
\textbf{Input}.   mp\_int $a, b$ \\
\textbf{Output}.  $c = \lfloor a/b \rfloor$, $d = a - bc$ \\
\hline \\
Now find the remainder fo the digits. \\
13.  for $i$ from $n$ down to $(t + 1)$ do \\
\hspace{3mm}13.1  If $i > x.used$ then jump to the next iteration of this loop. \\
\hspace{3mm}13.2  If $x_{i} = y_{t}$ then \\
\hspace{6mm}13.2.1  $q_{i - t - 1} \leftarrow \beta - 1$ \\
\hspace{3mm}13.3  else \\
\hspace{6mm}13.3.1  $\hat r \leftarrow x_{i} \cdot \beta + x_{i - 1}$ \\
\hspace{6mm}13.3.2  $\hat r \leftarrow \lfloor \hat r / y_{t} \rfloor$ \\
\hspace{6mm}13.3.3  $q_{i - t - 1} \leftarrow \hat r$ \\
\hspace{3mm}13.4  $q_{i - t - 1} \leftarrow q_{i - t - 1} + 1$ \\
\\
Fixup quotient estimation. \\
\hspace{3mm}13.5  Loop \\
\hspace{6mm}13.5.1  $q_{i - t - 1} \leftarrow q_{i - t - 1} - 1$ \\
\hspace{6mm}13.5.2  t$1 \leftarrow 0$ \\
\hspace{6mm}13.5.3  t$1_0 \leftarrow y_{t - 1}, $ t$1_1 \leftarrow y_t,$ t$1.used \leftarrow 2$ \\
\hspace{6mm}13.5.4  $t1 \leftarrow t1 \cdot q_{i - t - 1}$ \\
\hspace{6mm}13.5.5  t$2_0 \leftarrow x_{i - 2}, $ t$2_1 \leftarrow x_{i - 1}, $ t$2_2 \leftarrow x_i, $ t$2.used \leftarrow 3$ \\
\hspace{6mm}13.5.6  If $\vert t1 \vert > \vert t2 \vert$ then goto step 13.5. \\
\hspace{3mm}13.6  t$1 \leftarrow y \cdot q_{i - t - 1}$ \\
\hspace{3mm}13.7  t$1 \leftarrow $ t$1 \cdot \beta^{i - t - 1}$ \\
\hspace{3mm}13.8  $x \leftarrow x - $ t$1$ \\
\hspace{3mm}13.9  If $x.sign = MP\_NEG$ then \\
\hspace{6mm}13.10  t$1 \leftarrow y$ \\
\hspace{6mm}13.11  t$1 \leftarrow $ t$1 \cdot \beta^{i - t - 1}$ \\
\hspace{6mm}13.12  $x \leftarrow x + $ t$1$ \\
\hspace{6mm}13.13  $q_{i - t - 1} \leftarrow q_{i - t - 1} - 1$ \\
\\
Finalize the result. \\
14.  Clamp excess digits of $q$ \\
15.  $c \leftarrow q, c.sign \leftarrow sign$ \\
16.  $x.sign \leftarrow a.sign$ \\
17.  $d \leftarrow \lfloor x / 2^{norm} \rfloor$ \\
18.  Return(\textit{MP\_OKAY}). \\
\hline
\end{tabular}
\end{center}
\end{small}
\caption{Algorithm mp\_div (continued)}
\end{figure}
\textbf{Algorithm mp\_div.}
This algorithm will calculate quotient and remainder from an integer division given a dividend and divisor.  The algorithm is a signed
division and will produce a fully qualified quotient and remainder.

First the divisor $b$ must be non-zero which is enforced in step one.  If the divisor is larger than the dividend than the quotient is implicitly 
zero and the remainder is the dividend.  

After the first two trivial cases of inputs are handled the variable $q$ is setup to receive the digits of the quotient.  Two unsigned copies of the
divisor $y$ and dividend $x$ are made as well.  The core of the division algorithm is an unsigned division and will only work if the values are
positive.  Now the two values $x$ and $y$ must be normalized such that the leading digit of $y$ is greater than or equal to $\beta / 2$.  
This is performed by shifting both to the left by enough bits to get the desired normalization.  

At this point the division algorithm can begin producing digits of the quotient.  Recall that maximum value of the estimation used is 
$2\beta - {2 \over \beta}$ which means that a digit of the quotient must be first produced by another means.  In this case $y$ is shifted
to the left (\textit{step ten}) so that it has the same number of digits as $x$.  The loop on step eleven will subtract multiples of the 
shifted copy of $y$ until $x$ is smaller.  Since the leading digit of $y$ is greater than or equal to $\beta/2$ this loop will iterate at most two
times to produce the desired leading digit of the quotient.  

Now the remainder of the digits can be produced.  The equation $\hat q = \lfloor {{x_i \beta + x_{i-1}}\over y_t} \rfloor$ is used to fairly
accurately approximate the true quotient digit.  The estimation can in theory produce an estimation as high as $2\beta - {2 \over \beta}$ but by
induction the upper quotient digit is correct (\textit{as established on step eleven}) and the estimate must be less than $\beta$.  

Recall from section~\ref{sec:divest} that the estimation is never too low but may be too high.  The next step of the estimation process is
to refine the estimation.  The loop on step 13.5 uses $x_i\beta^2 + x_{i-1}\beta + x_{i-2}$ and $q_{i - t - 1}(y_t\beta + y_{t-1})$ as a higher
order approximation to adjust the quotient digit.

After both phases of estimation the quotient digit may still be off by a value of one\footnote{This is similar to the error introduced
by optimizing Barrett reduction.}.  Steps 13.6 and 13.7 subtract the multiple of the divisor from the dividend (\textit{Similar to step 3.3 of
algorithm~\ref{fig:raddiv}} and then subsequently add a multiple of the divisor if the quotient was too large.  

Now that the quotient has been determine finializing the result is a matter of clamping the quotient, fixing the sizes and de-normalizing the 
remainder.  An important aspect of this algorithm seemingly overlooked in other descriptions such as that of Algorithm 14.20 HAC \cite[pp. 598]{HAC}
is that when the estimations are being made (\textit{inside the loop on step 13.5}) that the digits $y_{t-1}$, $x_{i-2}$ and $x_{i-1}$ may lie 
outside their respective boundaries.  For example, if $t = 0$ or $i \le 1$ then the digits would be undefined.  In those cases the digits should
respectively be replaced with a zero.  

\vspace{+3mm}\begin{small}
\hspace{-5.1mm}{\bf File}: bn\_mp\_div.c
\vspace{-3mm}
\begin{alltt}
\end{alltt}
\end{small}

The implementation of this algorithm differs slightly from the pseudo code presented previously.  In this algorithm either of the quotient $c$ or
remainder $d$ may be passed as a \textbf{NULL} pointer which indicates their value is not desired.  For example, the C code to call the division
algorithm with only the quotient is 

\begin{verbatim}
mp_div(&a, &b, &c, NULL);  /* c = [a/b] */
\end{verbatim}

Lines 109 and 113 handle the two trivial cases of inputs which are division by zero and dividend smaller than the divisor 
respectively.  After the two trivial cases all of the temporary variables are initialized.  Line 148 determines the sign of 
the quotient and line 148 ensures that both $x$ and $y$ are positive.  

The number of bits in the leading digit is calculated on line 151.  Implictly an mp\_int with $r$ digits will require $lg(\beta)(r-1) + k$ bits
of precision which when reduced modulo $lg(\beta)$ produces the value of $k$.  In this case $k$ is the number of bits in the leading digit which is
exactly what is required.  For the algorithm to operate $k$ must equal $lg(\beta) - 1$ and when it does not the inputs must be normalized by shifting
them to the left by $lg(\beta) - 1 - k$ bits.

Throughout the variables $n$ and $t$ will represent the highest digit of $x$ and $y$ respectively.  These are first used to produce the 
leading digit of the quotient.  The loop beginning on line 184 will produce the remainder of the quotient digits.

The conditional ``continue'' on line 187 is used to prevent the algorithm from reading past the leading edge of $x$ which can occur when the
algorithm eliminates multiple non-zero digits in a single iteration.  This ensures that $x_i$ is always non-zero since by definition the digits
above the $i$'th position $x$ must be zero in order for the quotient to be precise\footnote{Precise as far as integer division is concerned.}.  

Lines 214, 216 and 223 through 225 manually construct the high accuracy estimations by setting the digits of the two mp\_int 
variables directly.  

\section{Single Digit Helpers}

This section briefly describes a series of single digit helper algorithms which come in handy when working with small constants.  All of 
the helper functions assume the single digit input is positive and will treat them as such.

\subsection{Single Digit Addition and Subtraction}

Both addition and subtraction are performed by ``cheating'' and using mp\_set followed by the higher level addition or subtraction 
algorithms.   As a result these algorithms are subtantially simpler with a slight cost in performance.

\newpage\begin{figure}[!here]
\begin{small}
\begin{center}
\begin{tabular}{l}
\hline Algorithm \textbf{mp\_add\_d}. \\
\textbf{Input}.   mp\_int $a$ and a mp\_digit $b$ \\
\textbf{Output}.  $c = a + b$ \\
\hline \\
1.  $t \leftarrow b$ (\textit{mp\_set}) \\
2.  $c \leftarrow a + t$ \\
3.  Return(\textit{MP\_OKAY}) \\
\hline
\end{tabular}
\end{center}
\end{small}
\caption{Algorithm mp\_add\_d}
\end{figure}

\textbf{Algorithm mp\_add\_d.}
This algorithm initiates a temporary mp\_int with the value of the single digit and uses algorithm mp\_add to add the two values together.

\vspace{+3mm}\begin{small}
\hspace{-5.1mm}{\bf File}: bn\_mp\_add\_d.c
\vspace{-3mm}
\begin{alltt}
\end{alltt}
\end{small}

Clever use of the letter 't'.

\subsubsection{Subtraction}
The single digit subtraction algorithm mp\_sub\_d is essentially the same except it uses mp\_sub to subtract the digit from the mp\_int.

\subsection{Single Digit Multiplication}
Single digit multiplication arises enough in division and radix conversion that it ought to be implement as a special case of the baseline
multiplication algorithm.  Essentially this algorithm is a modified version of algorithm s\_mp\_mul\_digs where one of the multiplicands
only has one digit.

\begin{figure}[!here]
\begin{small}
\begin{center}
\begin{tabular}{l}
\hline Algorithm \textbf{mp\_mul\_d}. \\
\textbf{Input}.   mp\_int $a$ and a mp\_digit $b$ \\
\textbf{Output}.  $c = ab$ \\
\hline \\
1.  $pa \leftarrow a.used$ \\
2.  Grow $c$ to at least $pa + 1$ digits. \\
3.  $oldused \leftarrow c.used$ \\
4.  $c.used \leftarrow pa + 1$ \\
5.  $c.sign \leftarrow a.sign$ \\
6.  $\mu \leftarrow 0$ \\
7.  for $ix$ from $0$ to $pa - 1$ do \\
\hspace{3mm}7.1  $\hat r \leftarrow \mu + a_{ix}b$ \\
\hspace{3mm}7.2  $c_{ix} \leftarrow \hat r \mbox{ (mod }\beta\mbox{)}$ \\
\hspace{3mm}7.3  $\mu \leftarrow \lfloor \hat r / \beta \rfloor$ \\
8.  $c_{pa} \leftarrow \mu$ \\
9.  for $ix$ from $pa + 1$ to $oldused$ do \\
\hspace{3mm}9.1  $c_{ix} \leftarrow 0$ \\
10.  Clamp excess digits of $c$. \\
11.  Return(\textit{MP\_OKAY}). \\
\hline
\end{tabular}
\end{center}
\end{small}
\caption{Algorithm mp\_mul\_d}
\end{figure}
\textbf{Algorithm mp\_mul\_d.}
This algorithm quickly multiplies an mp\_int by a small single digit value.  It is specially tailored to the job and has a minimal of overhead.  
Unlike the full multiplication algorithms this algorithm does not require any significnat temporary storage or memory allocations.  

\vspace{+3mm}\begin{small}
\hspace{-5.1mm}{\bf File}: bn\_mp\_mul\_d.c
\vspace{-3mm}
\begin{alltt}
\end{alltt}
\end{small}

In this implementation the destination $c$ may point to the same mp\_int as the source $a$ since the result is written after the digit is 
read from the source.  This function uses pointer aliases $tmpa$ and $tmpc$ for the digits of $a$ and $c$ respectively.  

\subsection{Single Digit Division}
Like the single digit multiplication algorithm, single digit division is also a fairly common algorithm used in radix conversion.  Since the
divisor is only a single digit a specialized variant of the division algorithm can be used to compute the quotient.  

\newpage\begin{figure}[!here]
\begin{small}
\begin{center}
\begin{tabular}{l}
\hline Algorithm \textbf{mp\_div\_d}. \\
\textbf{Input}.   mp\_int $a$ and a mp\_digit $b$ \\
\textbf{Output}.  $c = \lfloor a / b \rfloor, d = a - cb$ \\
\hline \\
1.  If $b = 0$ then return(\textit{MP\_VAL}).\\
2.  If $b = 3$ then use algorithm mp\_div\_3 instead. \\
3.  Init $q$ to $a.used$ digits.  \\
4.  $q.used \leftarrow a.used$ \\
5.  $q.sign \leftarrow a.sign$ \\
6.  $\hat w \leftarrow 0$ \\
7.  for $ix$ from $a.used - 1$ down to $0$ do \\
\hspace{3mm}7.1  $\hat w \leftarrow \hat w \beta + a_{ix}$ \\
\hspace{3mm}7.2  If $\hat w \ge b$ then \\
\hspace{6mm}7.2.1  $t \leftarrow \lfloor \hat w / b \rfloor$ \\
\hspace{6mm}7.2.2  $\hat w \leftarrow \hat w \mbox{ (mod }b\mbox{)}$ \\
\hspace{3mm}7.3  else\\
\hspace{6mm}7.3.1  $t \leftarrow 0$ \\
\hspace{3mm}7.4  $q_{ix} \leftarrow t$ \\
8.  $d \leftarrow \hat w$ \\
9.  Clamp excess digits of $q$. \\
10.  $c \leftarrow q$ \\
11.  Return(\textit{MP\_OKAY}). \\
\hline
\end{tabular}
\end{center}
\end{small}
\caption{Algorithm mp\_div\_d}
\end{figure}
\textbf{Algorithm mp\_div\_d.}
This algorithm divides the mp\_int $a$ by the single mp\_digit $b$ using an optimized approach.  Essentially in every iteration of the
algorithm another digit of the dividend is reduced and another digit of quotient produced.  Provided $b < \beta$ the value of $\hat w$
after step 7.1 will be limited such that $0 \le \lfloor \hat w / b \rfloor < \beta$.  

If the divisor $b$ is equal to three a variant of this algorithm is used which is called mp\_div\_3.  It replaces the division by three with
a multiplication by $\lfloor \beta / 3 \rfloor$ and the appropriate shift and residual fixup.  In essence it is much like the Barrett reduction
from chapter seven.  

\vspace{+3mm}\begin{small}
\hspace{-5.1mm}{\bf File}: bn\_mp\_div\_d.c
\vspace{-3mm}
\begin{alltt}
\end{alltt}
\end{small}

Like the implementation of algorithm mp\_div this algorithm allows either of the quotient or remainder to be passed as a \textbf{NULL} pointer to
indicate the respective value is not required.  This allows a trivial single digit modular reduction algorithm, mp\_mod\_d to be created.

The division and remainder on lines 44 and @45,%@ can be replaced often by a single division on most processors.  For example, the 32-bit x86 based 
processors can divide a 64-bit quantity by a 32-bit quantity and produce the quotient and remainder simultaneously.  Unfortunately the GCC 
compiler does not recognize that optimization and will actually produce two function calls to find the quotient and remainder respectively.  

\subsection{Single Digit Root Extraction}

Finding the $n$'th root of an integer is fairly easy as far as numerical analysis is concerned.  Algorithms such as the Newton-Raphson approximation 
(\ref{eqn:newton}) series will converge very quickly to a root for any continuous function $f(x)$.  

\begin{equation}
x_{i+1} = x_i - {f(x_i) \over f'(x_i)}
\label{eqn:newton}
\end{equation}

In this case the $n$'th root is desired and $f(x) = x^n - a$ where $a$ is the integer of which the root is desired.  The derivative of $f(x)$ is 
simply $f'(x) = nx^{n - 1}$.  Of particular importance is that this algorithm will be used over the integers not over the a more continuous domain
such as the real numbers.  As a result the root found can be above the true root by few and must be manually adjusted.  Ideally at the end of the 
algorithm the $n$'th root $b$ of an integer $a$ is desired such that $b^n \le a$.  

\newpage\begin{figure}[!here]
\begin{small}
\begin{center}
\begin{tabular}{l}
\hline Algorithm \textbf{mp\_n\_root}. \\
\textbf{Input}.   mp\_int $a$ and a mp\_digit $b$ \\
\textbf{Output}.  $c^b \le a$ \\
\hline \\
1.  If $b$ is even and $a.sign = MP\_NEG$ return(\textit{MP\_VAL}). \\
2.  $sign \leftarrow a.sign$ \\
3.  $a.sign \leftarrow MP\_ZPOS$ \\
4.  t$2 \leftarrow 2$ \\
5.  Loop \\
\hspace{3mm}5.1  t$1 \leftarrow $ t$2$ \\
\hspace{3mm}5.2  t$3 \leftarrow $ t$1^{b - 1}$ \\
\hspace{3mm}5.3  t$2 \leftarrow $ t$3 $ $\cdot$ t$1$ \\
\hspace{3mm}5.4  t$2 \leftarrow $ t$2 - a$ \\
\hspace{3mm}5.5  t$3 \leftarrow $ t$3 \cdot b$ \\
\hspace{3mm}5.6  t$3 \leftarrow \lfloor $t$2 / $t$3 \rfloor$ \\
\hspace{3mm}5.7  t$2 \leftarrow $ t$1 - $ t$3$ \\
\hspace{3mm}5.8  If t$1 \ne $ t$2$ then goto step 5.  \\
6.  Loop \\
\hspace{3mm}6.1  t$2 \leftarrow $ t$1^b$ \\
\hspace{3mm}6.2  If t$2 > a$ then \\
\hspace{6mm}6.2.1  t$1 \leftarrow $ t$1 - 1$ \\
\hspace{6mm}6.2.2  Goto step 6. \\
7.  $a.sign \leftarrow sign$ \\
8.  $c \leftarrow $ t$1$ \\
9.  $c.sign \leftarrow sign$  \\
10.  Return(\textit{MP\_OKAY}).  \\
\hline
\end{tabular}
\end{center}
\end{small}
\caption{Algorithm mp\_n\_root}
\end{figure}
\textbf{Algorithm mp\_n\_root.}
This algorithm finds the integer $n$'th root of an input using the Newton-Raphson approach.  It is partially optimized based on the observation
that the numerator of ${f(x) \over f'(x)}$ can be derived from a partial denominator.  That is at first the denominator is calculated by finding
$x^{b - 1}$.  This value can then be multiplied by $x$ and have $a$ subtracted from it to find the numerator.  This saves a total of $b - 1$ 
multiplications by t$1$ inside the loop.  

The initial value of the approximation is t$2 = 2$ which allows the algorithm to start with very small values and quickly converge on the
root.  Ideally this algorithm is meant to find the $n$'th root of an input where $n$ is bounded by $2 \le n \le 5$.  

\vspace{+3mm}\begin{small}
\hspace{-5.1mm}{\bf File}: bn\_mp\_n\_root.c
\vspace{-3mm}
\begin{alltt}
\end{alltt}
\end{small}

\section{Random Number Generation}

Random numbers come up in a variety of activities from public key cryptography to simple simulations and various randomized algorithms.  Pollard-Rho 
factoring for example, can make use of random values as starting points to find factors of a composite integer.  In this case the algorithm presented
is solely for simulations and not intended for cryptographic use.  

\newpage\begin{figure}[!here]
\begin{small}
\begin{center}
\begin{tabular}{l}
\hline Algorithm \textbf{mp\_rand}. \\
\textbf{Input}.   An integer $b$ \\
\textbf{Output}.  A pseudo-random number of $b$ digits \\
\hline \\
1.  $a \leftarrow 0$ \\
2.  If $b \le 0$ return(\textit{MP\_OKAY}) \\
3.  Pick a non-zero random digit $d$. \\
4.  $a \leftarrow a + d$ \\
5.  for $ix$ from 1 to $d - 1$ do \\
\hspace{3mm}5.1  $a \leftarrow a \cdot \beta$ \\
\hspace{3mm}5.2  Pick a random digit $d$. \\
\hspace{3mm}5.3  $a \leftarrow a + d$ \\
6.  Return(\textit{MP\_OKAY}). \\
\hline
\end{tabular}
\end{center}
\end{small}
\caption{Algorithm mp\_rand}
\end{figure}
\textbf{Algorithm mp\_rand.}
This algorithm produces a pseudo-random integer of $b$ digits.  By ensuring that the first digit is non-zero the algorithm also guarantees that the
final result has at least $b$ digits.  It relies heavily on a third-part random number generator which should ideally generate uniformly all of
the integers from $0$ to $\beta - 1$.  

\vspace{+3mm}\begin{small}
\hspace{-5.1mm}{\bf File}: bn\_mp\_rand.c
\vspace{-3mm}
\begin{alltt}
\end{alltt}
\end{small}

\section{Formatted Representations}
The ability to emit a radix-$n$ textual representation of an integer is useful for interacting with human parties.  For example, the ability to
be given a string of characters such as ``114585'' and turn it into the radix-$\beta$ equivalent would make it easier to enter numbers
into a program.

\subsection{Reading Radix-n Input}
For the purposes of this text we will assume that a simple lower ASCII map (\ref{fig:ASC}) is used for the values of from $0$ to $63$ to 
printable characters.  For example, when the character ``N'' is read it represents the integer $23$.  The first $16$ characters of the
map are for the common representations up to hexadecimal.  After that they match the ``base64'' encoding scheme which are suitable chosen
such that they are printable.  While outputting as base64 may not be too helpful for human operators it does allow communication via non binary
mediums.

\newpage\begin{figure}[here]
\begin{center}
\begin{tabular}{cc|cc|cc|cc}
\hline \textbf{Value} & \textbf{Char} & \textbf{Value} & \textbf{Char} & \textbf{Value} & \textbf{Char} &  \textbf{Value} & \textbf{Char} \\
\hline 
0 & 0 & 1 & 1 & 2 & 2 & 3 & 3 \\
4 & 4 & 5 & 5 & 6 & 6 & 7 & 7 \\
8 & 8 & 9 & 9 & 10 & A & 11 & B \\
12 & C & 13 & D & 14 & E & 15 & F \\
16 & G & 17 & H & 18 & I & 19 & J \\
20 & K & 21 & L & 22 & M & 23 & N \\
24 & O & 25 & P & 26 & Q & 27 & R \\
28 & S & 29 & T & 30 & U & 31 & V \\
32 & W & 33 & X & 34 & Y & 35 & Z \\
36 & a & 37 & b & 38 & c & 39 & d \\
40 & e & 41 & f & 42 & g & 43 & h \\
44 & i & 45 & j & 46 & k & 47 & l \\
48 & m & 49 & n & 50 & o & 51 & p \\
52 & q & 53 & r & 54 & s & 55 & t \\
56 & u & 57 & v & 58 & w & 59 & x \\
60 & y & 61 & z & 62 & $+$ & 63 & $/$ \\
\hline
\end{tabular}
\end{center}
\caption{Lower ASCII Map}
\label{fig:ASC}
\end{figure}

\newpage\begin{figure}[!here]
\begin{small}
\begin{center}
\begin{tabular}{l}
\hline Algorithm \textbf{mp\_read\_radix}. \\
\textbf{Input}.   A string $str$ of length $sn$ and radix $r$. \\
\textbf{Output}.  The radix-$\beta$ equivalent mp\_int. \\
\hline \\
1.  If $r < 2$ or $r > 64$ return(\textit{MP\_VAL}). \\
2.  $ix \leftarrow 0$ \\
3.  If $str_0 =$ ``-'' then do \\
\hspace{3mm}3.1  $ix \leftarrow ix + 1$ \\
\hspace{3mm}3.2  $sign \leftarrow MP\_NEG$ \\
4.  else \\
\hspace{3mm}4.1  $sign \leftarrow MP\_ZPOS$ \\
5.  $a \leftarrow 0$ \\
6.  for $iy$ from $ix$ to $sn - 1$ do \\
\hspace{3mm}6.1  Let $y$ denote the position in the map of $str_{iy}$. \\
\hspace{3mm}6.2  If $str_{iy}$ is not in the map or $y \ge r$ then goto step 7. \\
\hspace{3mm}6.3  $a \leftarrow a \cdot r$ \\
\hspace{3mm}6.4  $a \leftarrow a + y$ \\
7.  If $a \ne 0$ then $a.sign \leftarrow sign$ \\
8.  Return(\textit{MP\_OKAY}). \\
\hline
\end{tabular}
\end{center}
\end{small}
\caption{Algorithm mp\_read\_radix}
\end{figure}
\textbf{Algorithm mp\_read\_radix.}
This algorithm will read an ASCII string and produce the radix-$\beta$ mp\_int representation of the same integer.  A minus symbol ``-'' may precede the 
string  to indicate the value is negative, otherwise it is assumed to be positive.  The algorithm will read up to $sn$ characters from the input
and will stop when it reads a character it cannot map the algorithm stops reading characters from the string.  This allows numbers to be embedded
as part of larger input without any significant problem.

\vspace{+3mm}\begin{small}
\hspace{-5.1mm}{\bf File}: bn\_mp\_read\_radix.c
\vspace{-3mm}
\begin{alltt}
\end{alltt}
\end{small}

\subsection{Generating Radix-$n$ Output}
Generating radix-$n$ output is fairly trivial with a division and remainder algorithm.  

\newpage\begin{figure}[!here]
\begin{small}
\begin{center}
\begin{tabular}{l}
\hline Algorithm \textbf{mp\_toradix}. \\
\textbf{Input}.   A mp\_int $a$ and an integer $r$\\
\textbf{Output}.  The radix-$r$ representation of $a$ \\
\hline \\
1.  If $r < 2$ or $r > 64$ return(\textit{MP\_VAL}). \\
2.  If $a = 0$ then $str = $ ``$0$'' and return(\textit{MP\_OKAY}).  \\
3.  $t \leftarrow a$ \\
4.  $str \leftarrow$ ``'' \\
5.  if $t.sign = MP\_NEG$ then \\
\hspace{3mm}5.1  $str \leftarrow str + $ ``-'' \\
\hspace{3mm}5.2  $t.sign = MP\_ZPOS$ \\
6.  While ($t \ne 0$) do \\
\hspace{3mm}6.1  $d \leftarrow t \mbox{ (mod }r\mbox{)}$ \\
\hspace{3mm}6.2  $t \leftarrow \lfloor t / r \rfloor$ \\
\hspace{3mm}6.3  Look up $d$ in the map and store the equivalent character in $y$. \\
\hspace{3mm}6.4  $str \leftarrow str + y$ \\
7.  If $str_0 = $``$-$'' then \\
\hspace{3mm}7.1  Reverse the digits $str_1, str_2, \ldots str_n$. \\
8.  Otherwise \\
\hspace{3mm}8.1  Reverse the digits $str_0, str_1, \ldots str_n$. \\
9.  Return(\textit{MP\_OKAY}).\\
\hline
\end{tabular}
\end{center}
\end{small}
\caption{Algorithm mp\_toradix}
\end{figure}
\textbf{Algorithm mp\_toradix.}
This algorithm computes the radix-$r$ representation of an mp\_int $a$.  The ``digits'' of the representation are extracted by reducing 
successive powers of $\lfloor a / r^k \rfloor$ the input modulo $r$ until $r^k > a$.  Note that instead of actually dividing by $r^k$ in
each iteration the quotient $\lfloor a / r \rfloor$ is saved for the next iteration.  As a result a series of trivial $n \times 1$ divisions
are required instead of a series of $n \times k$ divisions.  One design flaw of this approach is that the digits are produced in the reverse order 
(see~\ref{fig:mpradix}).  To remedy this flaw the digits must be swapped or simply ``reversed''.

\begin{figure}
\begin{center}
\begin{tabular}{|c|c|c|}
\hline \textbf{Value of $a$} & \textbf{Value of $d$} & \textbf{Value of $str$} \\
\hline $1234$ & -- & -- \\
\hline $123$  & $4$ & ``4'' \\
\hline $12$   & $3$ & ``43'' \\
\hline $1$    & $2$ & ``432'' \\
\hline $0$    & $1$ & ``4321'' \\
\hline
\end{tabular}
\end{center}
\caption{Example of Algorithm mp\_toradix.}
\label{fig:mpradix}
\end{figure}

\vspace{+3mm}\begin{small}
\hspace{-5.1mm}{\bf File}: bn\_mp\_toradix.c
\vspace{-3mm}
\begin{alltt}
\end{alltt}
\end{small}

\chapter{Number Theoretic Algorithms}
This chapter discusses several fundamental number theoretic algorithms such as the greatest common divisor, least common multiple and Jacobi 
symbol computation.  These algorithms arise as essential components in several key cryptographic algorithms such as the RSA public key algorithm and
various Sieve based factoring algorithms.

\section{Greatest Common Divisor}
The greatest common divisor of two integers $a$ and $b$, often denoted as $(a, b)$ is the largest integer $k$ that is a proper divisor of
both $a$ and $b$.  That is, $k$ is the largest integer such that $0 \equiv a \mbox{ (mod }k\mbox{)}$ and $0 \equiv b \mbox{ (mod }k\mbox{)}$ occur
simultaneously.

The most common approach (cite) is to reduce one input modulo another.  That is if $a$ and $b$ are divisible by some integer $k$ and if $qa + r = b$ then
$r$ is also divisible by $k$.  The reduction pattern follows $\left < a , b \right > \rightarrow \left < b, a \mbox{ mod } b \right >$.  

\newpage\begin{figure}[!here]
\begin{small}
\begin{center}
\begin{tabular}{l}
\hline Algorithm \textbf{Greatest Common Divisor (I)}. \\
\textbf{Input}.   Two positive integers $a$ and $b$ greater than zero. \\
\textbf{Output}.  The greatest common divisor $(a, b)$.  \\
\hline \\
1.  While ($b > 0$) do \\
\hspace{3mm}1.1  $r \leftarrow a \mbox{ (mod }b\mbox{)}$ \\
\hspace{3mm}1.2  $a \leftarrow b$ \\
\hspace{3mm}1.3  $b \leftarrow r$ \\
2.  Return($a$). \\
\hline
\end{tabular}
\end{center}
\end{small}
\caption{Algorithm Greatest Common Divisor (I)}
\label{fig:gcd1}
\end{figure}

This algorithm will quickly converge on the greatest common divisor since the residue $r$ tends diminish rapidly.  However, divisions are
relatively expensive operations to perform and should ideally be avoided.  There is another approach based on a similar relationship of 
greatest common divisors.  The faster approach is based on the observation that if $k$ divides both $a$ and $b$ it will also divide $a - b$.  
In particular, we would like $a - b$ to decrease in magnitude which implies that $b \ge a$.  

\begin{figure}[!here]
\begin{small}
\begin{center}
\begin{tabular}{l}
\hline Algorithm \textbf{Greatest Common Divisor (II)}. \\
\textbf{Input}.   Two positive integers $a$ and $b$ greater than zero. \\
\textbf{Output}.  The greatest common divisor $(a, b)$.  \\
\hline \\
1.  While ($b > 0$) do \\
\hspace{3mm}1.1  Swap $a$ and $b$ such that $a$ is the smallest of the two. \\
\hspace{3mm}1.2  $b \leftarrow b - a$ \\
2.  Return($a$). \\
\hline
\end{tabular}
\end{center}
\end{small}
\caption{Algorithm Greatest Common Divisor (II)}
\label{fig:gcd2}
\end{figure}

\textbf{Proof} \textit{Algorithm~\ref{fig:gcd2} will return the greatest common divisor of $a$ and $b$.}
The algorithm in figure~\ref{fig:gcd2} will eventually terminate since $b \ge a$ the subtraction in step 1.2 will be a value less than $b$.  In other
words in every iteration that tuple $\left < a, b \right >$ decrease in magnitude until eventually $a = b$.  Since both $a$ and $b$ are always 
divisible by the greatest common divisor (\textit{until the last iteration}) and in the last iteration of the algorithm $b = 0$, therefore, in the 
second to last iteration of the algorithm $b = a$ and clearly $(a, a) = a$ which concludes the proof.  \textbf{QED}.

As a matter of practicality algorithm \ref{fig:gcd1} decreases far too slowly to be useful.  Specially if $b$ is much larger than $a$ such that 
$b - a$ is still very much larger than $a$.  A simple addition to the algorithm is to divide $b - a$ by a power of some integer $p$ which does
not divide the greatest common divisor but will divide $b - a$.  In this case ${b - a} \over p$ is also an integer and still divisible by
the greatest common divisor.

However, instead of factoring $b - a$ to find a suitable value of $p$ the powers of $p$ can be removed from $a$ and $b$ that are in common first.  
Then inside the loop whenever $b - a$ is divisible by some power of $p$ it can be safely removed.  

\begin{figure}[!here]
\begin{small}
\begin{center}
\begin{tabular}{l}
\hline Algorithm \textbf{Greatest Common Divisor (III)}. \\
\textbf{Input}.   Two positive integers $a$ and $b$ greater than zero. \\
\textbf{Output}.  The greatest common divisor $(a, b)$.  \\
\hline \\
1.  $k \leftarrow 0$ \\
2.  While $a$ and $b$ are both divisible by $p$ do \\
\hspace{3mm}2.1  $a \leftarrow \lfloor a / p \rfloor$ \\
\hspace{3mm}2.2  $b \leftarrow \lfloor b / p \rfloor$ \\
\hspace{3mm}2.3  $k \leftarrow k + 1$ \\
3.  While $a$ is divisible by $p$ do \\
\hspace{3mm}3.1  $a \leftarrow \lfloor a / p \rfloor$ \\
4.  While $b$ is divisible by $p$ do \\
\hspace{3mm}4.1  $b \leftarrow \lfloor b / p \rfloor$ \\
5.  While ($b > 0$) do \\
\hspace{3mm}5.1  Swap $a$ and $b$ such that $a$ is the smallest of the two. \\
\hspace{3mm}5.2  $b \leftarrow b - a$ \\
\hspace{3mm}5.3  While $b$ is divisible by $p$ do \\
\hspace{6mm}5.3.1  $b \leftarrow \lfloor b / p \rfloor$ \\
6.  Return($a \cdot p^k$). \\
\hline
\end{tabular}
\end{center}
\end{small}
\caption{Algorithm Greatest Common Divisor (III)}
\label{fig:gcd3}
\end{figure}

This algorithm is based on the first except it removes powers of $p$ first and inside the main loop to ensure the tuple $\left < a, b \right >$ 
decreases more rapidly.  The first loop on step two removes powers of $p$ that are in common.  A count, $k$, is kept which will present a common
divisor of $p^k$.  After step two the remaining common divisor of $a$ and $b$ cannot be divisible by $p$.  This means that $p$ can be safely 
divided out of the difference $b - a$ so long as the division leaves no remainder.  

In particular the value of $p$ should be chosen such that the division on step 5.3.1 occur often.  It also helps that division by $p$ be easy
to compute.  The ideal choice of $p$ is two since division by two amounts to a right logical shift.  Another important observation is that by
step five both $a$ and $b$ are odd.  Therefore, the diffrence $b - a$ must be even which means that each iteration removes one bit from the 
largest of the pair.

\subsection{Complete Greatest Common Divisor}
The algorithms presented so far cannot handle inputs which are zero or negative.  The following algorithm can handle all input cases properly
and will produce the greatest common divisor.

\newpage\begin{figure}[!here]
\begin{small}
\begin{center}
\begin{tabular}{l}
\hline Algorithm \textbf{mp\_gcd}. \\
\textbf{Input}.   mp\_int $a$ and $b$ \\
\textbf{Output}.  The greatest common divisor $c = (a, b)$.  \\
\hline \\
1.  If $a = 0$ then \\
\hspace{3mm}1.1  $c \leftarrow \vert b \vert $ \\
\hspace{3mm}1.2  Return(\textit{MP\_OKAY}). \\
2.  If $b = 0$ then \\
\hspace{3mm}2.1  $c \leftarrow \vert a \vert $ \\
\hspace{3mm}2.2  Return(\textit{MP\_OKAY}). \\
3.  $u \leftarrow \vert a \vert, v \leftarrow \vert b \vert$ \\
4.  $k \leftarrow 0$ \\
5.  While $u.used > 0$ and $v.used > 0$ and $u_0 \equiv v_0 \equiv 0 \mbox{ (mod }2\mbox{)}$ \\
\hspace{3mm}5.1  $k \leftarrow k + 1$ \\
\hspace{3mm}5.2  $u \leftarrow \lfloor u / 2 \rfloor$ \\
\hspace{3mm}5.3  $v \leftarrow \lfloor v / 2 \rfloor$ \\
6.  While $u.used > 0$ and $u_0 \equiv 0 \mbox{ (mod }2\mbox{)}$ \\
\hspace{3mm}6.1  $u \leftarrow \lfloor u / 2 \rfloor$ \\
7.  While $v.used > 0$ and $v_0 \equiv 0 \mbox{ (mod }2\mbox{)}$ \\
\hspace{3mm}7.1  $v \leftarrow \lfloor v / 2 \rfloor$ \\
8.  While $v.used > 0$ \\
\hspace{3mm}8.1  If $\vert u \vert > \vert v \vert$ then \\
\hspace{6mm}8.1.1  Swap $u$ and $v$. \\
\hspace{3mm}8.2  $v \leftarrow \vert v \vert - \vert u \vert$ \\
\hspace{3mm}8.3  While $v.used > 0$ and $v_0 \equiv 0 \mbox{ (mod }2\mbox{)}$ \\
\hspace{6mm}8.3.1  $v \leftarrow \lfloor v / 2 \rfloor$ \\
9.  $c \leftarrow u \cdot 2^k$ \\
10.  Return(\textit{MP\_OKAY}). \\
\hline
\end{tabular}
\end{center}
\end{small}
\caption{Algorithm mp\_gcd}
\end{figure}
\textbf{Algorithm mp\_gcd.}
This algorithm will produce the greatest common divisor of two mp\_ints $a$ and $b$.  The algorithm was originally based on Algorithm B of
Knuth \cite[pp. 338]{TAOCPV2} but has been modified to be simpler to explain.  In theory it achieves the same asymptotic working time as
Algorithm B and in practice this appears to be true.  

The first two steps handle the cases where either one of or both inputs are zero.  If either input is zero the greatest common divisor is the 
largest input or zero if they are both zero.  If the inputs are not trivial than $u$ and $v$ are assigned the absolute values of 
$a$ and $b$ respectively and the algorithm will proceed to reduce the pair.

Step five will divide out any common factors of two and keep track of the count in the variable $k$.  After this step, two is no longer a
factor of the remaining greatest common divisor between $u$ and $v$ and can be safely evenly divided out of either whenever they are even.  Step 
six and seven ensure that the $u$ and $v$ respectively have no more factors of two.  At most only one of the while--loops will iterate since 
they cannot both be even.

By step eight both of $u$ and $v$ are odd which is required for the inner logic.  First the pair are swapped such that $v$ is equal to
or greater than $u$.  This ensures that the subtraction on step 8.2 will always produce a positive and even result.  Step 8.3 removes any
factors of two from the difference $u$ to ensure that in the next iteration of the loop both are once again odd.

After $v = 0$ occurs the variable $u$ has the greatest common divisor of the pair $\left < u, v \right >$ just after step six.  The result
must be adjusted by multiplying by the common factors of two ($2^k$) removed earlier.  

\vspace{+3mm}\begin{small}
\hspace{-5.1mm}{\bf File}: bn\_mp\_gcd.c
\vspace{-3mm}
\begin{alltt}
\end{alltt}
\end{small}

This function makes use of the macros mp\_iszero and mp\_iseven.  The former evaluates to $1$ if the input mp\_int is equivalent to the 
integer zero otherwise it evaluates to $0$.  The latter evaluates to $1$ if the input mp\_int represents a non-zero even integer otherwise
it evaluates to $0$.  Note that just because mp\_iseven may evaluate to $0$ does not mean the input is odd, it could also be zero.  The three 
trivial cases of inputs are handled on lines 24 through 30.  After those lines the inputs are assumed to be non-zero.

Lines 32 and 37 make local copies $u$ and $v$ of the inputs $a$ and $b$ respectively.  At this point the common factors of two 
must be divided out of the two inputs.  The block starting at line 44 removes common factors of two by first counting the number of trailing
zero bits in both.  The local integer $k$ is used to keep track of how many factors of $2$ are pulled out of both values.  It is assumed that 
the number of factors will not exceed the maximum value of a C ``int'' data type\footnote{Strictly speaking no array in C may have more than 
entries than are accessible by an ``int'' so this is not a limitation.}.  

At this point there are no more common factors of two in the two values.  The divisions by a power of two on lines 62 and 68 remove 
any independent factors of two such that both $u$ and $v$ are guaranteed to be an odd integer before hitting the main body of the algorithm.  The while loop
on line 73 performs the reduction of the pair until $v$ is equal to zero.  The unsigned comparison and subtraction algorithms are used in
place of the full signed routines since both values are guaranteed to be positive and the result of the subtraction is guaranteed to be non-negative.

\section{Least Common Multiple}
The least common multiple of a pair of integers is their product divided by their greatest common divisor.  For two integers $a$ and $b$ the
least common multiple is normally denoted as $[ a, b ]$ and numerically equivalent to ${ab} \over {(a, b)}$.  For example, if $a = 2 \cdot 2 \cdot 3 = 12$
and $b = 2 \cdot 3 \cdot 3 \cdot 7 = 126$ the least common multiple is ${126 \over {(12, 126)}} = {126 \over 6} = 21$.

The least common multiple arises often in coding theory as well as number theory.  If two functions have periods of $a$ and $b$ respectively they will
collide, that is be in synchronous states, after only $[ a, b ]$ iterations.  This is why, for example, random number generators based on 
Linear Feedback Shift Registers (LFSR) tend to use registers with periods which are co-prime (\textit{e.g. the greatest common divisor is one.}).  
Similarly in number theory if a composite $n$ has two prime factors $p$ and $q$ then maximal order of any unit of $\Z/n\Z$ will be $[ p - 1, q - 1] $.

\begin{figure}[!here]
\begin{small}
\begin{center}
\begin{tabular}{l}
\hline Algorithm \textbf{mp\_lcm}. \\
\textbf{Input}.   mp\_int $a$ and $b$ \\
\textbf{Output}.  The least common multiple $c = [a, b]$.  \\
\hline \\
1.  $c \leftarrow (a, b)$ \\
2.  $t \leftarrow a \cdot b$ \\
3.  $c \leftarrow \lfloor t / c \rfloor$ \\
4.  Return(\textit{MP\_OKAY}). \\
\hline
\end{tabular}
\end{center}
\end{small}
\caption{Algorithm mp\_lcm}
\end{figure}
\textbf{Algorithm mp\_lcm.}
This algorithm computes the least common multiple of two mp\_int inputs $a$ and $b$.  It computes the least common multiple directly by
dividing the product of the two inputs by their greatest common divisor.

\vspace{+3mm}\begin{small}
\hspace{-5.1mm}{\bf File}: bn\_mp\_lcm.c
\vspace{-3mm}
\begin{alltt}
\end{alltt}
\end{small}

\section{Jacobi Symbol Computation}
To explain the Jacobi Symbol we shall first discuss the Legendre function\footnote{Arrg.  What is the name of this?} off which the Jacobi symbol is 
defined.  The Legendre function computes whether or not an integer $a$ is a quadratic residue modulo an odd prime $p$.  Numerically it is
equivalent to equation \ref{eqn:legendre}.

\textit{-- Tom, don't be an ass, cite your source here...!}

\begin{equation}
a^{(p-1)/2} \equiv \begin{array}{rl}
                              -1 &  \mbox{if }a\mbox{ is a quadratic non-residue.} \\
                              0  &  \mbox{if }a\mbox{ divides }p\mbox{.} \\
                              1  &  \mbox{if }a\mbox{ is a quadratic residue}. 
                              \end{array} \mbox{ (mod }p\mbox{)}
\label{eqn:legendre}                              
\end{equation}

\textbf{Proof.} \textit{Equation \ref{eqn:legendre} correctly identifies the residue status of an integer $a$ modulo a prime $p$.}
An integer $a$ is a quadratic residue if the following equation has a solution.

\begin{equation}
x^2 \equiv a \mbox{ (mod }p\mbox{)}
\label{eqn:root}
\end{equation}

Consider the following equation.

\begin{equation}
0 \equiv x^{p-1} - 1 \equiv \left \lbrace \left (x^2 \right )^{(p-1)/2} - a^{(p-1)/2} \right \rbrace + \left ( a^{(p-1)/2} - 1 \right ) \mbox{ (mod }p\mbox{)}
\label{eqn:rooti}
\end{equation}

Whether equation \ref{eqn:root} has a solution or not equation \ref{eqn:rooti} is always true.  If $a^{(p-1)/2} - 1 \equiv 0 \mbox{ (mod }p\mbox{)}$
then the quantity in the braces must be zero.  By reduction,

\begin{eqnarray}
\left (x^2 \right )^{(p-1)/2} - a^{(p-1)/2} \equiv 0  \nonumber \\
\left (x^2 \right )^{(p-1)/2} \equiv a^{(p-1)/2} \nonumber \\
x^2 \equiv a \mbox{ (mod }p\mbox{)} 
\end{eqnarray}

As a result there must be a solution to the quadratic equation and in turn $a$ must be a quadratic residue.  If $a$ does not divide $p$ and $a$
is not a quadratic residue then the only other value $a^{(p-1)/2}$ may be congruent to is $-1$ since
\begin{equation}
0 \equiv a^{p - 1} - 1 \equiv (a^{(p-1)/2} + 1)(a^{(p-1)/2} - 1) \mbox{ (mod }p\mbox{)}
\end{equation}
One of the terms on the right hand side must be zero.  \textbf{QED}

\subsection{Jacobi Symbol}
The Jacobi symbol is a generalization of the Legendre function for any odd non prime moduli $p$ greater than 2.  If $p = \prod_{i=0}^n p_i$ then
the Jacobi symbol $\left ( { a \over p } \right )$ is equal to the following equation.

\begin{equation}
\left ( { a \over p } \right ) = \left ( { a \over p_0} \right ) \left ( { a \over p_1} \right ) \ldots \left ( { a \over p_n} \right )
\end{equation}

By inspection if $p$ is prime the Jacobi symbol is equivalent to the Legendre function.  The following facts\footnote{See HAC \cite[pp. 72-74]{HAC} for
further details.} will be used to derive an efficient Jacobi symbol algorithm.  Where $p$ is an odd integer greater than two and $a, b \in \Z$ the
following are true.  

\begin{enumerate}
\item $\left ( { a \over p} \right )$ equals $-1$, $0$ or $1$. 
\item $\left ( { ab \over p} \right ) = \left ( { a \over p} \right )\left ( { b \over p} \right )$.
\item If $a \equiv b$ then $\left ( { a \over p} \right ) = \left ( { b \over p} \right )$.
\item $\left ( { 2 \over p} \right )$ equals $1$ if $p \equiv 1$ or $7 \mbox{ (mod }8\mbox{)}$.  Otherwise, it equals $-1$.
\item $\left ( { a \over p} \right ) \equiv \left ( { p \over a} \right ) \cdot (-1)^{(p-1)(a-1)/4}$.  More specifically 
$\left ( { a \over p} \right ) = \left ( { p \over a} \right )$ if $p \equiv a \equiv 1 \mbox{ (mod }4\mbox{)}$.  
\end{enumerate}

Using these facts if $a = 2^k \cdot a'$ then

\begin{eqnarray}
\left ( { a \over p } \right ) = \left ( {{2^k} \over p } \right ) \left ( {a' \over p} \right ) \nonumber \\
                               = \left ( {2 \over p } \right )^k \left ( {a' \over p} \right ) 
\label{eqn:jacobi}
\end{eqnarray}

By fact five, 

\begin{equation}
\left ( { a \over p } \right ) = \left ( { p \over a } \right ) \cdot (-1)^{(p-1)(a-1)/4} 
\end{equation}

Subsequently by fact three since $p \equiv (p \mbox{ mod }a) \mbox{ (mod }a\mbox{)}$ then 

\begin{equation}
\left ( { a \over p } \right ) = \left ( { {p \mbox{ mod } a} \over a } \right ) \cdot (-1)^{(p-1)(a-1)/4} 
\end{equation}

By putting both observations into equation \ref{eqn:jacobi} the following simplified equation is formed.

\begin{equation}
\left ( { a \over p } \right ) = \left ( {2 \over p } \right )^k \left ( {{p\mbox{ mod }a'} \over a'} \right )  \cdot (-1)^{(p-1)(a'-1)/4} 
\end{equation}

The value of $\left ( {{p \mbox{ mod }a'} \over a'} \right )$ can be found by using the same equation recursively.  The value of 
$\left ( {2 \over p } \right )^k$ equals $1$ if $k$ is even otherwise it equals $\left ( {2 \over p } \right )$.  Using this approach the 
factors of $p$ do not have to be known.  Furthermore, if $(a, p) = 1$ then the algorithm will terminate when the recursion requests the 
Jacobi symbol computation of $\left ( {1 \over a'} \right )$ which is simply $1$.  

\newpage\begin{figure}[!here]
\begin{small}
\begin{center}
\begin{tabular}{l}
\hline Algorithm \textbf{mp\_jacobi}. \\
\textbf{Input}.   mp\_int $a$ and $p$, $a \ge 0$, $p \ge 3$, $p \equiv 1 \mbox{ (mod }2\mbox{)}$ \\
\textbf{Output}.  The Jacobi symbol $c = \left ( {a \over p } \right )$. \\
\hline \\
1.  If $a = 0$ then \\
\hspace{3mm}1.1  $c \leftarrow 0$ \\
\hspace{3mm}1.2  Return(\textit{MP\_OKAY}). \\
2.  If $a = 1$ then \\
\hspace{3mm}2.1  $c \leftarrow 1$ \\
\hspace{3mm}2.2  Return(\textit{MP\_OKAY}). \\
3.  $a' \leftarrow a$ \\
4.  $k \leftarrow 0$ \\
5.  While $a'.used > 0$ and $a'_0 \equiv 0 \mbox{ (mod }2\mbox{)}$ \\
\hspace{3mm}5.1  $k \leftarrow k + 1$ \\
\hspace{3mm}5.2  $a' \leftarrow \lfloor a' / 2 \rfloor$ \\
6.  If $k \equiv 0 \mbox{ (mod }2\mbox{)}$ then \\
\hspace{3mm}6.1  $s \leftarrow 1$ \\
7.  else \\
\hspace{3mm}7.1  $r \leftarrow p_0 \mbox{ (mod }8\mbox{)}$ \\
\hspace{3mm}7.2  If $r = 1$ or $r = 7$ then \\
\hspace{6mm}7.2.1  $s \leftarrow 1$ \\
\hspace{3mm}7.3  else \\
\hspace{6mm}7.3.1  $s \leftarrow -1$ \\
8.  If $p_0 \equiv a'_0 \equiv 3 \mbox{ (mod }4\mbox{)}$ then \\
\hspace{3mm}8.1  $s \leftarrow -s$ \\
9.  If $a' \ne 1$ then \\
\hspace{3mm}9.1  $p' \leftarrow p \mbox{ (mod }a'\mbox{)}$ \\
\hspace{3mm}9.2  $s \leftarrow s \cdot \mbox{mp\_jacobi}(p', a')$ \\
10.  $c \leftarrow s$ \\
11.  Return(\textit{MP\_OKAY}). \\
\hline
\end{tabular}
\end{center}
\end{small}
\caption{Algorithm mp\_jacobi}
\end{figure}
\textbf{Algorithm mp\_jacobi.}
This algorithm computes the Jacobi symbol for an arbitrary positive integer $a$ with respect to an odd integer $p$ greater than three.  The algorithm
is based on algorithm 2.149 of HAC \cite[pp. 73]{HAC}.  

Step numbers one and two handle the trivial cases of $a = 0$ and $a = 1$ respectively.  Step five determines the number of two factors in the
input $a$.  If $k$ is even than the term $\left ( { 2 \over p } \right )^k$ must always evaluate to one.  If $k$ is odd than the term evaluates to one 
if $p_0$ is congruent to one or seven modulo eight, otherwise it evaluates to $-1$. After the the $\left ( { 2 \over p } \right )^k$ term is handled 
the $(-1)^{(p-1)(a'-1)/4}$ is computed and multiplied against the current product $s$.  The latter term evaluates to one if both $p$ and $a'$ 
are congruent to one modulo four, otherwise it evaluates to negative one.

By step nine if $a'$ does not equal one a recursion is required.  Step 9.1 computes $p' \equiv p \mbox{ (mod }a'\mbox{)}$ and will recurse to compute
$\left ( {p' \over a'} \right )$ which is multiplied against the current Jacobi product.

\vspace{+3mm}\begin{small}
\hspace{-5.1mm}{\bf File}: bn\_mp\_jacobi.c
\vspace{-3mm}
\begin{alltt}
\end{alltt}
\end{small}

As a matter of practicality the variable $a'$ as per the pseudo-code is reprensented by the variable $a1$ since the $'$ symbol is not valid for a C 
variable name character. 

The two simple cases of $a = 0$ and $a = 1$ are handled at the very beginning to simplify the algorithm.  If the input is non-trivial the algorithm
has to proceed compute the Jacobi.  The variable $s$ is used to hold the current Jacobi product.  Note that $s$ is merely a C ``int'' data type since
the values it may obtain are merely $-1$, $0$ and $1$.  

After a local copy of $a$ is made all of the factors of two are divided out and the total stored in $k$.  Technically only the least significant
bit of $k$ is required, however, it makes the algorithm simpler to follow to perform an addition. In practice an exclusive-or and addition have the same 
processor requirements and neither is faster than the other.

Line 58 through 71 determines the value of $\left ( { 2 \over p } \right )^k$.  If the least significant bit of $k$ is zero than
$k$ is even and the value is one.  Otherwise, the value of $s$ depends on which residue class $p$ belongs to modulo eight.  The value of
$(-1)^{(p-1)(a'-1)/4}$ is compute and multiplied against $s$ on lines 71 through 74.  

Finally, if $a1$ does not equal one the algorithm must recurse and compute $\left ( {p' \over a'} \right )$.  

\textit{-- Comment about default $s$ and such...}

\section{Modular Inverse}
\label{sec:modinv}
The modular inverse of a number actually refers to the modular multiplicative inverse.  Essentially for any integer $a$ such that $(a, p) = 1$ there
exist another integer $b$ such that $ab \equiv 1 \mbox{ (mod }p\mbox{)}$.  The integer $b$ is called the multiplicative inverse of $a$ which is
denoted as $b = a^{-1}$.  Technically speaking modular inversion is a well defined operation for any finite ring or field not just for rings and 
fields of integers.  However, the former will be the matter of discussion.

The simplest approach is to compute the algebraic inverse of the input.  That is to compute $b \equiv a^{\Phi(p) - 1}$.  If $\Phi(p)$ is the 
order of the multiplicative subgroup modulo $p$ then $b$ must be the multiplicative inverse of $a$.  The proof of which is trivial.

\begin{equation}
ab \equiv a \left (a^{\Phi(p) - 1} \right ) \equiv a^{\Phi(p)} \equiv a^0 \equiv 1 \mbox{ (mod }p\mbox{)}
\end{equation}

However, as simple as this approach may be it has two serious flaws.  It requires that the value of $\Phi(p)$ be known which if $p$ is composite 
requires all of the prime factors.  This approach also is very slow as the size of $p$ grows.  

A simpler approach is based on the observation that solving for the multiplicative inverse is equivalent to solving the linear 
Diophantine\footnote{See LeVeque \cite[pp. 40-43]{LeVeque} for more information.} equation.

\begin{equation}
ab + pq = 1
\end{equation}

Where $a$, $b$, $p$ and $q$ are all integers.  If such a pair of integers $ \left < b, q \right >$ exist than $b$ is the multiplicative inverse of 
$a$ modulo $p$.  The extended Euclidean algorithm (Knuth \cite[pp. 342]{TAOCPV2}) can be used to solve such equations provided $(a, p) = 1$.  
However, instead of using that algorithm directly a variant known as the binary Extended Euclidean algorithm will be used in its place.  The
binary approach is very similar to the binary greatest common divisor algorithm except it will produce a full solution to the Diophantine 
equation.  

\subsection{General Case}
\newpage\begin{figure}[!here]
\begin{small}
\begin{center}
\begin{tabular}{l}
\hline Algorithm \textbf{mp\_invmod}. \\
\textbf{Input}.   mp\_int $a$ and $b$, $(a, b) = 1$, $p \ge 2$, $0 < a < p$.  \\
\textbf{Output}.  The modular inverse $c \equiv a^{-1} \mbox{ (mod }b\mbox{)}$. \\
\hline \\
1.  If $b \le 0$ then return(\textit{MP\_VAL}). \\
2.  If $b_0 \equiv 1 \mbox{ (mod }2\mbox{)}$ then use algorithm fast\_mp\_invmod. \\
3.  $x \leftarrow \vert a \vert, y \leftarrow b$ \\
4.  If $x_0 \equiv y_0  \equiv 0 \mbox{ (mod }2\mbox{)}$ then return(\textit{MP\_VAL}). \\
5.  $B \leftarrow 0, C \leftarrow 0, A \leftarrow 1, D \leftarrow 1$ \\
6.  While $u.used > 0$ and $u_0 \equiv 0 \mbox{ (mod }2\mbox{)}$ \\
\hspace{3mm}6.1  $u \leftarrow \lfloor u / 2 \rfloor$ \\
\hspace{3mm}6.2  If ($A.used > 0$ and $A_0 \equiv 1 \mbox{ (mod }2\mbox{)}$) or ($B.used > 0$ and $B_0 \equiv 1 \mbox{ (mod }2\mbox{)}$) then \\
\hspace{6mm}6.2.1  $A \leftarrow A + y$ \\
\hspace{6mm}6.2.2  $B \leftarrow B - x$ \\
\hspace{3mm}6.3  $A \leftarrow \lfloor A / 2 \rfloor$ \\
\hspace{3mm}6.4  $B \leftarrow \lfloor B / 2 \rfloor$ \\
7.  While $v.used > 0$ and $v_0 \equiv 0 \mbox{ (mod }2\mbox{)}$ \\
\hspace{3mm}7.1  $v \leftarrow \lfloor v / 2 \rfloor$ \\
\hspace{3mm}7.2  If ($C.used > 0$ and $C_0 \equiv 1 \mbox{ (mod }2\mbox{)}$) or ($D.used > 0$ and $D_0 \equiv 1 \mbox{ (mod }2\mbox{)}$) then \\
\hspace{6mm}7.2.1  $C \leftarrow C + y$ \\
\hspace{6mm}7.2.2  $D \leftarrow D - x$ \\
\hspace{3mm}7.3  $C \leftarrow \lfloor C / 2 \rfloor$ \\
\hspace{3mm}7.4  $D \leftarrow \lfloor D / 2 \rfloor$ \\
8.  If $u \ge v$ then \\
\hspace{3mm}8.1  $u \leftarrow u - v$ \\
\hspace{3mm}8.2  $A \leftarrow A - C$ \\
\hspace{3mm}8.3  $B \leftarrow B - D$ \\
9.  else \\
\hspace{3mm}9.1  $v \leftarrow v - u$ \\
\hspace{3mm}9.2  $C \leftarrow C - A$ \\
\hspace{3mm}9.3  $D \leftarrow D - B$ \\
10.  If $u \ne 0$ goto step 6. \\
11.  If $v \ne 1$ return(\textit{MP\_VAL}). \\
12.  While $C \le 0$ do \\
\hspace{3mm}12.1  $C \leftarrow C + b$ \\
13.  While $C \ge b$ do \\
\hspace{3mm}13.1  $C \leftarrow C - b$ \\
14.  $c \leftarrow C$ \\
15.  Return(\textit{MP\_OKAY}). \\
\hline
\end{tabular}
\end{center}
\end{small}
\end{figure}
\textbf{Algorithm mp\_invmod.}
This algorithm computes the modular multiplicative inverse of an integer $a$ modulo an integer $b$.  This algorithm is a variation of the 
extended binary Euclidean algorithm from HAC \cite[pp. 608]{HAC}.  It has been modified to only compute the modular inverse and not a complete
Diophantine solution.  

If $b \le 0$ than the modulus is invalid and MP\_VAL is returned.  Similarly if both $a$ and $b$ are even then there cannot be a multiplicative
inverse for $a$ and the error is reported.  

The astute reader will observe that steps seven through nine are very similar to the binary greatest common divisor algorithm mp\_gcd.  In this case
the other variables to the Diophantine equation are solved.  The algorithm terminates when $u = 0$ in which case the solution is

\begin{equation}
Ca + Db = v
\end{equation}

If $v$, the greatest common divisor of $a$ and $b$ is not equal to one then the algorithm will report an error as no inverse exists.  Otherwise, $C$
is the modular inverse of $a$.  The actual value of $C$ is congruent to, but not necessarily equal to, the ideal modular inverse which should lie 
within $1 \le a^{-1} < b$.  Step numbers twelve and thirteen adjust the inverse until it is in range.  If the original input $a$ is within $0 < a < p$ 
then only a couple of additions or subtractions will be required to adjust the inverse.

\vspace{+3mm}\begin{small}
\hspace{-5.1mm}{\bf File}: bn\_mp\_invmod.c
\vspace{-3mm}
\begin{alltt}
\end{alltt}
\end{small}

\subsubsection{Odd Moduli}

When the modulus $b$ is odd the variables $A$ and $C$ are fixed and are not required to compute the inverse.  In particular by attempting to solve
the Diophantine $Cb + Da = 1$ only $B$ and $D$ are required to find the inverse of $a$.  

The algorithm fast\_mp\_invmod is a direct adaptation of algorithm mp\_invmod with all all steps involving either $A$ or $C$ removed.  This 
optimization will halve the time required to compute the modular inverse.

\section{Primality Tests}

A non-zero integer $a$ is said to be prime if it is not divisible by any other integer excluding one and itself.  For example, $a = 7$ is prime 
since the integers $2 \ldots 6$ do not evenly divide $a$.  By contrast, $a = 6$ is not prime since $a = 6 = 2 \cdot 3$. 

Prime numbers arise in cryptography considerably as they allow finite fields to be formed.  The ability to determine whether an integer is prime or
not quickly has been a viable subject in cryptography and number theory for considerable time.  The algorithms that will be presented are all
probablistic algorithms in that when they report an integer is composite it must be composite.  However, when the algorithms report an integer is
prime the algorithm may be incorrect.  

As will be discussed it is possible to limit the probability of error so well that for practical purposes the probablity of error might as 
well be zero.  For the purposes of these discussions let $n$ represent the candidate integer of which the primality is in question.

\subsection{Trial Division}

Trial division means to attempt to evenly divide a candidate integer by small prime integers.  If the candidate can be evenly divided it obviously
cannot be prime.  By dividing by all primes $1 < p \le \sqrt{n}$ this test can actually prove whether an integer is prime.  However, such a test
would require a prohibitive amount of time as $n$ grows.

Instead of dividing by every prime, a smaller, more mangeable set of primes may be used instead.  By performing trial division with only a subset
of the primes less than $\sqrt{n} + 1$ the algorithm cannot prove if a candidate is prime.  However, often it can prove a candidate is not prime.

The benefit of this test is that trial division by small values is fairly efficient.  Specially compared to the other algorithms that will be
discussed shortly.  The probability that this approach correctly identifies a composite candidate when tested with all primes upto $q$ is given by
$1 - {1.12 \over ln(q)}$.  The graph (\ref{pic:primality}, will be added later) demonstrates the probability of success for the range 
$3 \le q \le 100$.  

At approximately $q = 30$ the gain of performing further tests diminishes fairly quickly.  At $q = 90$ further testing is generally not going to 
be of any practical use.  In the case of LibTomMath the default limit $q = 256$ was chosen since it is not too high and will eliminate 
approximately $80\%$ of all candidate integers.  The constant \textbf{PRIME\_SIZE} is equal to the number of primes in the test base.  The 
array \_\_prime\_tab is an array of the first \textbf{PRIME\_SIZE} prime numbers.  

\begin{figure}[!here]
\begin{small}
\begin{center}
\begin{tabular}{l}
\hline Algorithm \textbf{mp\_prime\_is\_divisible}. \\
\textbf{Input}.   mp\_int $a$ \\
\textbf{Output}.  $c = 1$ if $n$ is divisible by a small prime, otherwise $c = 0$.  \\
\hline \\
1.  for $ix$ from $0$ to $PRIME\_SIZE$ do \\
\hspace{3mm}1.1  $d \leftarrow n \mbox{ (mod }\_\_prime\_tab_{ix}\mbox{)}$ \\
\hspace{3mm}1.2  If $d = 0$ then \\
\hspace{6mm}1.2.1  $c \leftarrow 1$ \\
\hspace{6mm}1.2.2  Return(\textit{MP\_OKAY}). \\
2.  $c \leftarrow 0$ \\
3.  Return(\textit{MP\_OKAY}). \\
\hline
\end{tabular}
\end{center}
\end{small}
\caption{Algorithm mp\_prime\_is\_divisible}
\end{figure}
\textbf{Algorithm mp\_prime\_is\_divisible.}
This algorithm attempts to determine if a candidate integer $n$ is composite by performing trial divisions.  

\vspace{+3mm}\begin{small}
\hspace{-5.1mm}{\bf File}: bn\_mp\_prime\_is\_divisible.c
\vspace{-3mm}
\begin{alltt}
\end{alltt}
\end{small}

The algorithm defaults to a return of $0$ in case an error occurs.  The values in the prime table are all specified to be in the range of a 
mp\_digit.  The table \_\_prime\_tab is defined in the following file.

\vspace{+3mm}\begin{small}
\hspace{-5.1mm}{\bf File}: bn\_prime\_tab.c
\vspace{-3mm}
\begin{alltt}
\end{alltt}
\end{small}

Note that there are two possible tables.  When an mp\_digit is 7-bits long only the primes upto $127$ may be included, otherwise the primes
upto $1619$ are used.  Note that the value of \textbf{PRIME\_SIZE} is a constant dependent on the size of a mp\_digit. 

\subsection{The Fermat Test}
The Fermat test is probably one the oldest tests to have a non-trivial probability of success.  It is based on the fact that if $n$ is in 
fact prime then $a^{n} \equiv a \mbox{ (mod }n\mbox{)}$ for all $0 < a < n$.  The reason being that if $n$ is prime than the order of
the multiplicative sub group is $n - 1$.  Any base $a$ must have an order which divides $n - 1$ and as such $a^n$ is equivalent to 
$a^1 = a$.  

If $n$ is composite then any given base $a$ does not have to have a period which divides $n - 1$.  In which case 
it is possible that $a^n \nequiv a \mbox{ (mod }n\mbox{)}$.  However, this test is not absolute as it is possible that the order
of a base will divide $n - 1$ which would then be reported as prime.  Such a base yields what is known as a Fermat pseudo-prime.  Several 
integers known as Carmichael numbers will be a pseudo-prime to all valid bases.  Fortunately such numbers are extremely rare as $n$ grows
in size.

\begin{figure}[!here]
\begin{small}
\begin{center}
\begin{tabular}{l}
\hline Algorithm \textbf{mp\_prime\_fermat}. \\
\textbf{Input}.   mp\_int $a$ and $b$, $a \ge 2$, $0 < b < a$.  \\
\textbf{Output}.  $c = 1$ if $b^a \equiv b \mbox{ (mod }a\mbox{)}$, otherwise $c = 0$.  \\
\hline \\
1.  $t \leftarrow b^a \mbox{ (mod }a\mbox{)}$ \\
2.  If $t = b$ then \\
\hspace{3mm}2.1  $c = 1$ \\
3.  else \\
\hspace{3mm}3.1  $c = 0$ \\
4.  Return(\textit{MP\_OKAY}). \\
\hline
\end{tabular}
\end{center}
\end{small}
\caption{Algorithm mp\_prime\_fermat}
\end{figure}
\textbf{Algorithm mp\_prime\_fermat.}
This algorithm determines whether an mp\_int $a$ is a Fermat prime to the base $b$ or not.  It uses a single modular exponentiation to
determine the result.  

\vspace{+3mm}\begin{small}
\hspace{-5.1mm}{\bf File}: bn\_mp\_prime\_fermat.c
\vspace{-3mm}
\begin{alltt}
\end{alltt}
\end{small}

\subsection{The Miller-Rabin Test}
The Miller-Rabin (citation) test is another primality test which has tighter error bounds than the Fermat test specifically with sequentially chosen 
candidate  integers.  The algorithm is based on the observation that if $n - 1 = 2^kr$ and if $b^r \nequiv \pm 1$ then after upto $k - 1$ squarings the 
value must be equal to $-1$.  The squarings are stopped as soon as $-1$ is observed.  If the value of $1$ is observed first it means that
some value not congruent to $\pm 1$ when squared equals one which cannot occur if $n$ is prime.

\begin{figure}[!here]
\begin{small}
\begin{center}
\begin{tabular}{l}
\hline Algorithm \textbf{mp\_prime\_miller\_rabin}. \\
\textbf{Input}.   mp\_int $a$ and $b$, $a \ge 2$, $0 < b < a$.  \\
\textbf{Output}.  $c = 1$ if $a$ is a Miller-Rabin prime to the base $a$, otherwise $c = 0$.  \\
\hline
1.  $a' \leftarrow a - 1$ \\
2.  $r  \leftarrow n1$    \\
3.  $c \leftarrow 0, s  \leftarrow 0$ \\
4.  While $r.used > 0$ and $r_0 \equiv 0 \mbox{ (mod }2\mbox{)}$ \\
\hspace{3mm}4.1  $s \leftarrow s + 1$ \\
\hspace{3mm}4.2  $r \leftarrow \lfloor r / 2 \rfloor$ \\
5.  $y \leftarrow b^r \mbox{ (mod }a\mbox{)}$ \\
6.  If $y \nequiv \pm 1$ then \\
\hspace{3mm}6.1  $j \leftarrow 1$ \\
\hspace{3mm}6.2  While $j \le (s - 1)$ and $y \nequiv a'$ \\
\hspace{6mm}6.2.1  $y \leftarrow y^2 \mbox{ (mod }a\mbox{)}$ \\
\hspace{6mm}6.2.2  If $y = 1$ then goto step 8. \\
\hspace{6mm}6.2.3  $j \leftarrow j + 1$ \\
\hspace{3mm}6.3  If $y \nequiv a'$ goto step 8. \\
7.  $c \leftarrow 1$\\
8.  Return(\textit{MP\_OKAY}). \\
\hline
\end{tabular}
\end{center}
\end{small}
\caption{Algorithm mp\_prime\_miller\_rabin}
\end{figure}
\textbf{Algorithm mp\_prime\_miller\_rabin.}
This algorithm performs one trial round of the Miller-Rabin algorithm to the base $b$.  It will set $c = 1$ if the algorithm cannot determine
if $b$ is composite or $c = 0$ if $b$ is provably composite.  The values of $s$ and $r$ are computed such that $a' = a - 1 = 2^sr$.  

If the value $y \equiv b^r$ is congruent to $\pm 1$ then the algorithm cannot prove if $a$ is composite or not.  Otherwise, the algorithm will
square $y$ upto $s - 1$ times stopping only when $y \equiv -1$.  If $y^2 \equiv 1$ and $y \nequiv \pm 1$ then the algorithm can report that $a$
is provably composite.  If the algorithm performs $s - 1$ squarings and $y \nequiv -1$ then $a$ is provably composite.  If $a$ is not provably 
composite then it is \textit{probably} prime.

\vspace{+3mm}\begin{small}
\hspace{-5.1mm}{\bf File}: bn\_mp\_prime\_miller\_rabin.c
\vspace{-3mm}
\begin{alltt}
\end{alltt}
\end{small}




\backmatter
\appendix
\begin{thebibliography}{ABCDEF}
\bibitem[1]{TAOCPV2}
Donald Knuth, \textit{The Art of Computer Programming}, Third Edition, Volume Two, Seminumerical Algorithms, Addison-Wesley, 1998

\bibitem[2]{HAC}
A. Menezes, P. van Oorschot, S. Vanstone, \textit{Handbook of Applied Cryptography}, CRC Press, 1996

\bibitem[3]{ROSE}
Michael Rosing, \textit{Implementing Elliptic Curve Cryptography}, Manning Publications, 1999

\bibitem[4]{COMBA}
Paul G. Comba, \textit{Exponentiation Cryptosystems on the IBM PC}. IBM Systems Journal 29(4): 526-538 (1990)

\bibitem[5]{KARA}
A. Karatsuba, Doklay Akad. Nauk SSSR 145 (1962), pp.293-294

\bibitem[6]{KARAP}
Andre Weimerskirch and Christof Paar, \textit{Generalizations of the Karatsuba Algorithm for Polynomial Multiplication}, Submitted to Design, Codes and Cryptography, March 2002

\bibitem[7]{BARRETT}
Paul Barrett, \textit{Implementing the Rivest Shamir and Adleman Public Key Encryption Algorithm on a Standard Digital Signal Processor}, Advances in Cryptology, Crypto '86, Springer-Verlag.

\bibitem[8]{MONT}
P.L.Montgomery. \textit{Modular multiplication without trial division}. Mathematics of Computation, 44(170):519-521, April 1985.

\bibitem[9]{DRMET}
Chae Hoon Lim and Pil Joong Lee, \textit{Generating Efficient Primes for Discrete Log Cryptosystems}, POSTECH Information Research Laboratories

\bibitem[10]{MMB}
J. Daemen and R. Govaerts and J. Vandewalle, \textit{Block ciphers based on Modular Arithmetic}, State and {P}rogress in the {R}esearch of {C}ryptography, 1993, pp. 80-89

\bibitem[11]{RSAREF}
R.L. Rivest, A. Shamir, L. Adleman, \textit{A Method for Obtaining Digital Signatures and Public-Key Cryptosystems}

\bibitem[12]{DHREF}
Whitfield Diffie, Martin E. Hellman, \textit{New Directions in Cryptography}, IEEE Transactions on Information Theory, 1976

\bibitem[13]{IEEE}
IEEE Standard for Binary Floating-Point Arithmetic (ANSI/IEEE Std 754-1985)

\bibitem[14]{GMP}
GNU Multiple Precision (GMP), \url{http://www.swox.com/gmp/}

\bibitem[15]{MPI}
Multiple Precision Integer Library (MPI), Michael Fromberger, \url{http://thayer.dartmouth.edu/~sting/mpi/}

\bibitem[16]{OPENSSL}
OpenSSL Cryptographic Toolkit, \url{http://openssl.org}

\bibitem[17]{LIP}
Large Integer Package, \url{http://home.hetnet.nl/~ecstr/LIP.zip}

\bibitem[18]{ISOC}
JTC1/SC22/WG14, ISO/IEC 9899:1999, ``A draft rationale for the C99 standard.''

\bibitem[19]{JAVA}
The Sun Java Website, \url{http://java.sun.com/}

\end{thebibliography}

\documentclass[b5paper]{book}
\usepackage{hyperref}
\usepackage{makeidx}
\usepackage{amssymb}
\usepackage{color}
\usepackage{alltt}
\usepackage{graphicx}
\usepackage{layout}
\def\union{\cup}
\def\intersect{\cap}
\def\getsrandom{\stackrel{\rm R}{\gets}}
\def\cross{\times}
\def\cat{\hspace{0.5em} \| \hspace{0.5em}}
\def\catn{$\|$}
\def\divides{\hspace{0.3em} | \hspace{0.3em}}
\def\nequiv{\not\equiv}
\def\approx{\raisebox{0.2ex}{\mbox{\small $\sim$}}}
\def\lcm{{\rm lcm}}
\def\gcd{{\rm gcd}}
\def\log{{\rm log}}
\def\ord{{\rm ord}}
\def\abs{{\mathit abs}}
\def\rep{{\mathit rep}}
\def\mod{{\mathit\ mod\ }}
\renewcommand{\pmod}[1]{\ ({\rm mod\ }{#1})}
\newcommand{\floor}[1]{\left\lfloor{#1}\right\rfloor}
\newcommand{\ceil}[1]{\left\lceil{#1}\right\rceil}
\def\Or{{\rm\ or\ }}
\def\And{{\rm\ and\ }}
\def\iff{\hspace{1em}\Longleftrightarrow\hspace{1em}}
\def\implies{\Rightarrow}
\def\undefined{{\rm ``undefined"}}
\def\Proof{\vspace{1ex}\noindent {\bf Proof:}\hspace{1em}}
\let\oldphi\phi
\def\phi{\varphi}
\def\Pr{{\rm Pr}}
\newcommand{\str}[1]{{\mathbf{#1}}}
\def\F{{\mathbb F}}
\def\N{{\mathbb N}}
\def\Z{{\mathbb Z}}
\def\R{{\mathbb R}}
\def\C{{\mathbb C}}
\def\Q{{\mathbb Q}}
\definecolor{DGray}{gray}{0.5}
\newcommand{\emailaddr}[1]{\mbox{$<${#1}$>$}}
\def\twiddle{\raisebox{0.3ex}{\mbox{\tiny $\sim$}}}
\def\gap{\vspace{0.5ex}}
\makeindex
\begin{document}
\frontmatter
\pagestyle{empty}
\title{Multi--Precision Math}
\author{\mbox{
%\begin{small}
\begin{tabular}{c}
Tom St Denis \\
Algonquin College \\
\\
Mads Rasmussen \\
Open Communications Security \\
\\
Greg Rose \\
QUALCOMM Australia \\
\end{tabular}
%\end{small}
}
}
\maketitle
This text has been placed in the public domain.  This text corresponds to the v0.39 release of the 
LibTomMath project.

\begin{alltt}
Tom St Denis
111 Banning Rd
Ottawa, Ontario
K2L 1C3
Canada

Phone: 1-613-836-3160
Email: tomstdenis@gmail.com
\end{alltt}

This text is formatted to the international B5 paper size of 176mm wide by 250mm tall using the \LaTeX{} 
{\em book} macro package and the Perl {\em booker} package.

\tableofcontents
\listoffigures
\chapter*{Prefaces}
When I tell people about my LibTom projects and that I release them as public domain they are often puzzled.  
They ask why I did it and especially why I continue to work on them for free.  The best I can explain it is ``Because I can.''  
Which seems odd and perhaps too terse for adult conversation. I often qualify it with ``I am able, I am willing.'' which 
perhaps explains it better.  I am the first to admit there is not anything that special with what I have done.  Perhaps
others can see that too and then we would have a society to be proud of.  My LibTom projects are what I am doing to give 
back to society in the form of tools and knowledge that can help others in their endeavours.

I started writing this book because it was the most logical task to further my goal of open academia.  The LibTomMath source
code itself was written to be easy to follow and learn from.  There are times, however, where pure C source code does not
explain the algorithms properly.  Hence this book.  The book literally starts with the foundation of the library and works
itself outwards to the more complicated algorithms.  The use of both pseudo--code and verbatim source code provides a duality
of ``theory'' and ``practice'' that the computer science students of the world shall appreciate.  I never deviate too far
from relatively straightforward algebra and I hope that this book can be a valuable learning asset.

This book and indeed much of the LibTom projects would not exist in their current form if it was not for a plethora
of kind people donating their time, resources and kind words to help support my work.  Writing a text of significant
length (along with the source code) is a tiresome and lengthy process.  Currently the LibTom project is four years old,
comprises of literally thousands of users and over 100,000 lines of source code, TeX and other material.  People like Mads and Greg 
were there at the beginning to encourage me to work well.  It is amazing how timely validation from others can boost morale to 
continue the project. Definitely my parents were there for me by providing room and board during the many months of work in 2003.  

To my many friends whom I have met through the years I thank you for the good times and the words of encouragement.  I hope I
honour your kind gestures with this project.

Open Source.  Open Academia.  Open Minds.

\begin{flushright} Tom St Denis \end{flushright}

\newpage
I found the opportunity to work with Tom appealing for several reasons, not only could I broaden my own horizons, but also 
contribute to educate others facing the problem of having to handle big number mathematical calculations.

This book is Tom's child and he has been caring and fostering the project ever since the beginning with a clear mind of 
how he wanted the project to turn out. I have helped by proofreading the text and we have had several discussions about 
the layout and language used.

I hold a masters degree in cryptography from the University of Southern Denmark and have always been interested in the 
practical aspects of cryptography. 

Having worked in the security consultancy business for several years in S\~{a}o Paulo, Brazil, I have been in touch with a 
great deal of work in which multiple precision mathematics was needed. Understanding the possibilities for speeding up 
multiple precision calculations is often very important since we deal with outdated machine architecture where modular 
reductions, for example, become painfully slow.

This text is for people who stop and wonder when first examining algorithms such as RSA for the first time and asks 
themselves, ``You tell me this is only secure for large numbers, fine; but how do you implement these numbers?''

\begin{flushright}
Mads Rasmussen

S\~{a}o Paulo - SP

Brazil
\end{flushright}

\newpage
It's all because I broke my leg. That just happened to be at about the same time that Tom asked for someone to review the section of the book about 
Karatsuba multiplication. I was laid up, alone and immobile, and thought ``Why not?'' I vaguely knew what Karatsuba multiplication was, but not 
really, so I thought I could help, learn, and stop myself from watching daytime cable TV, all at once.

At the time of writing this, I've still not met Tom or Mads in meatspace. I've been following Tom's progress since his first splash on the 
sci.crypt Usenet news group. I watched him go from a clueless newbie, to the cryptographic equivalent of a reformed smoker, to a real
contributor to the field, over a period of about two years. I've been impressed with his obvious intelligence, and astounded by his productivity. 
Of course, he's young enough to be my own child, so he doesn't have my problems with staying awake.

When I reviewed that single section of the book, in its very earliest form, I was very pleasantly surprised. So I decided to collaborate more fully, 
and at least review all of it, and perhaps write some bits too. There's still a long way to go with it, and I have watched a number of close 
friends go through the mill of publication, so I think that the way to go is longer than Tom thinks it is. Nevertheless, it's a good effort, 
and I'm pleased to be involved with it.

\begin{flushright}
Greg Rose, Sydney, Australia, June 2003. 
\end{flushright}

\mainmatter
\pagestyle{headings}
\chapter{Introduction}
\section{Multiple Precision Arithmetic}

\subsection{What is Multiple Precision Arithmetic?}
When we think of long-hand arithmetic such as addition or multiplication we rarely consider the fact that we instinctively
raise or lower the precision of the numbers we are dealing with.  For example, in decimal we almost immediate can 
reason that $7$ times $6$ is $42$.  However, $42$ has two digits of precision as opposed to one digit we started with.  
Further multiplications of say $3$ result in a larger precision result $126$.  In these few examples we have multiple 
precisions for the numbers we are working with.  Despite the various levels of precision a single subset\footnote{With the occasional optimization.}
 of algorithms can be designed to accomodate them.  

By way of comparison a fixed or single precision operation would lose precision on various operations.  For example, in
the decimal system with fixed precision $6 \cdot 7 = 2$.

Essentially at the heart of computer based multiple precision arithmetic are the same long-hand algorithms taught in
schools to manually add, subtract, multiply and divide.  

\subsection{The Need for Multiple Precision Arithmetic}
The most prevalent need for multiple precision arithmetic, often referred to as ``bignum'' math, is within the implementation
of public-key cryptography algorithms.   Algorithms such as RSA \cite{RSAREF} and Diffie-Hellman \cite{DHREF} require 
integers of significant magnitude to resist known cryptanalytic attacks.  For example, at the time of this writing a 
typical RSA modulus would be at least greater than $10^{309}$.  However, modern programming languages such as ISO C \cite{ISOC} and 
Java \cite{JAVA} only provide instrinsic support for integers which are relatively small and single precision.

\begin{figure}[!here]
\begin{center}
\begin{tabular}{|r|c|}
\hline \textbf{Data Type} & \textbf{Range} \\
\hline char  & $-128 \ldots 127$ \\
\hline short & $-32768 \ldots 32767$ \\
\hline long  & $-2147483648 \ldots 2147483647$ \\
\hline long long & $-9223372036854775808 \ldots 9223372036854775807$ \\
\hline
\end{tabular}
\end{center}
\caption{Typical Data Types for the C Programming Language}
\label{fig:ISOC}
\end{figure}

The largest data type guaranteed to be provided by the ISO C programming 
language\footnote{As per the ISO C standard.  However, each compiler vendor is allowed to augment the precision as they 
see fit.}  can only represent values up to $10^{19}$ as shown in figure \ref{fig:ISOC}. On its own the C language is 
insufficient to accomodate the magnitude required for the problem at hand.  An RSA modulus of magnitude $10^{19}$ could be 
trivially factored\footnote{A Pollard-Rho factoring would take only $2^{16}$ time.} on the average desktop computer, 
rendering any protocol based on the algorithm insecure.  Multiple precision algorithms solve this very problem by 
extending the range of representable integers while using single precision data types.

Most advancements in fast multiple precision arithmetic stem from the need for faster and more efficient cryptographic 
primitives.  Faster modular reduction and exponentiation algorithms such as Barrett's algorithm, which have appeared in 
various cryptographic journals, can render algorithms such as RSA and Diffie-Hellman more efficient.  In fact, several 
major companies such as RSA Security, Certicom and Entrust have built entire product lines on the implementation and 
deployment of efficient algorithms.

However, cryptography is not the only field of study that can benefit from fast multiple precision integer routines.  
Another auxiliary use of multiple precision integers is high precision floating point data types.  
The basic IEEE \cite{IEEE} standard floating point type is made up of an integer mantissa $q$, an exponent $e$ and a sign bit $s$.  
Numbers are given in the form $n = q \cdot b^e \cdot -1^s$ where $b = 2$ is the most common base for IEEE.  Since IEEE 
floating point is meant to be implemented in hardware the precision of the mantissa is often fairly small 
(\textit{23, 48 and 64 bits}).  The mantissa is merely an integer and a multiple precision integer could be used to create
a mantissa of much larger precision than hardware alone can efficiently support.  This approach could be useful where 
scientific applications must minimize the total output error over long calculations.

Yet another use for large integers is within arithmetic on polynomials of large characteristic (i.e. $GF(p)[x]$ for large $p$).
In fact the library discussed within this text has already been used to form a polynomial basis library\footnote{See \url{http://poly.libtomcrypt.org} for more details.}.

\subsection{Benefits of Multiple Precision Arithmetic}
\index{precision}
The benefit of multiple precision representations over single or fixed precision representations is that 
no precision is lost while representing the result of an operation which requires excess precision.  For example, 
the product of two $n$-bit integers requires at least $2n$ bits of precision to be represented faithfully.  A multiple 
precision algorithm would augment the precision of the destination to accomodate the result while a single precision system 
would truncate excess bits to maintain a fixed level of precision.

It is possible to implement algorithms which require large integers with fixed precision algorithms.  For example, elliptic
curve cryptography (\textit{ECC}) is often implemented on smartcards by fixing the precision of the integers to the maximum 
size the system will ever need.  Such an approach can lead to vastly simpler algorithms which can accomodate the 
integers required even if the host platform cannot natively accomodate them\footnote{For example, the average smartcard 
processor has an 8 bit accumulator.}.  However, as efficient as such an approach may be, the resulting source code is not
normally very flexible.  It cannot, at runtime, accomodate inputs of higher magnitude than the designer anticipated.

Multiple precision algorithms have the most overhead of any style of arithmetic.  For the the most part the 
overhead can be kept to a minimum with careful planning, but overall, it is not well suited for most memory starved
platforms.  However, multiple precision algorithms do offer the most flexibility in terms of the magnitude of the 
inputs.  That is, the same algorithms based on multiple precision integers can accomodate any reasonable size input 
without the designer's explicit forethought.  This leads to lower cost of ownership for the code as it only has to 
be written and tested once.

\section{Purpose of This Text}
The purpose of this text is to instruct the reader regarding how to implement efficient multiple precision algorithms.  
That is to not only explain a limited subset of the core theory behind the algorithms but also the various ``house keeping'' 
elements that are neglected by authors of other texts on the subject.  Several well reknowned texts \cite{TAOCPV2,HAC} 
give considerably detailed explanations of the theoretical aspects of algorithms and often very little information 
regarding the practical implementation aspects.  

In most cases how an algorithm is explained and how it is actually implemented are two very different concepts.  For 
example, the Handbook of Applied Cryptography (\textit{HAC}), algorithm 14.7 on page 594, gives a relatively simple 
algorithm for performing multiple precision integer addition.  However, the description lacks any discussion concerning 
the fact that the two integer inputs may be of differing magnitudes.  As a result the implementation is not as simple
as the text would lead people to believe.  Similarly the division routine (\textit{algorithm 14.20, pp. 598}) does not 
discuss how to handle sign or handle the dividend's decreasing magnitude in the main loop (\textit{step \#3}).

Both texts also do not discuss several key optimal algorithms required such as ``Comba'' and Karatsuba multipliers 
and fast modular inversion, which we consider practical oversights.  These optimal algorithms are vital to achieve 
any form of useful performance in non-trivial applications.  

To solve this problem the focus of this text is on the practical aspects of implementing a multiple precision integer
package.  As a case study the ``LibTomMath''\footnote{Available at \url{http://math.libtomcrypt.com}} package is used 
to demonstrate algorithms with real implementations\footnote{In the ISO C programming language.} that have been field 
tested and work very well.  The LibTomMath library is freely available on the Internet for all uses and this text 
discusses a very large portion of the inner workings of the library.

The algorithms that are presented will always include at least one ``pseudo-code'' description followed 
by the actual C source code that implements the algorithm.  The pseudo-code can be used to implement the same 
algorithm in other programming languages as the reader sees fit.  

This text shall also serve as a walkthrough of the creation of multiple precision algorithms from scratch.  Showing
the reader how the algorithms fit together as well as where to start on various taskings.  

\section{Discussion and Notation}
\subsection{Notation}
A multiple precision integer of $n$-digits shall be denoted as $x = (x_{n-1}, \ldots, x_1, x_0)_{ \beta }$ and represent
the integer $x \equiv \sum_{i=0}^{n-1} x_i\beta^i$.  The elements of the array $x$ are said to be the radix $\beta$ digits 
of the integer.  For example, $x = (1,2,3)_{10}$ would represent the integer 
$1\cdot 10^2 + 2\cdot10^1 + 3\cdot10^0 = 123$.  

\index{mp\_int}
The term ``mp\_int'' shall refer to a composite structure which contains the digits of the integer it represents, as well 
as auxilary data required to manipulate the data.  These additional members are discussed further in section 
\ref{sec:MPINT}.  For the purposes of this text a ``multiple precision integer'' and an ``mp\_int'' are assumed to be 
synonymous.  When an algorithm is specified to accept an mp\_int variable it is assumed the various auxliary data members 
are present as well.  An expression of the type \textit{variablename.item} implies that it should evaluate to the 
member named ``item'' of the variable.  For example, a string of characters may have a member ``length'' which would 
evaluate to the number of characters in the string.  If the string $a$ equals ``hello'' then it follows that 
$a.length = 5$.  

For certain discussions more generic algorithms are presented to help the reader understand the final algorithm used
to solve a given problem.  When an algorithm is described as accepting an integer input it is assumed the input is 
a plain integer with no additional multiple-precision members.  That is, algorithms that use integers as opposed to 
mp\_ints as inputs do not concern themselves with the housekeeping operations required such as memory management.  These 
algorithms will be used to establish the relevant theory which will subsequently be used to describe a multiple
precision algorithm to solve the same problem.  

\subsection{Precision Notation}
The variable $\beta$ represents the radix of a single digit of a multiple precision integer and 
must be of the form $q^p$ for $q, p \in \Z^+$.  A single precision variable must be able to represent integers in 
the range $0 \le x < q \beta$ while a double precision variable must be able to represent integers in the range 
$0 \le x < q \beta^2$.  The extra radix-$q$ factor allows additions and subtractions to proceed without truncation of the 
carry.  Since all modern computers are binary, it is assumed that $q$ is two.

\index{mp\_digit} \index{mp\_word}
Within the source code that will be presented for each algorithm, the data type \textbf{mp\_digit} will represent 
a single precision integer type, while, the data type \textbf{mp\_word} will represent a double precision integer type.  In 
several algorithms (notably the Comba routines) temporary results will be stored in arrays of double precision mp\_words.  
For the purposes of this text $x_j$ will refer to the $j$'th digit of a single precision array and $\hat x_j$ will refer to 
the $j$'th digit of a double precision array.  Whenever an expression is to be assigned to a double precision
variable it is assumed that all single precision variables are promoted to double precision during the evaluation.  
Expressions that are assigned to a single precision variable are truncated to fit within the precision of a single
precision data type.

For example, if $\beta = 10^2$ a single precision data type may represent a value in the 
range $0 \le x < 10^3$, while a double precision data type may represent a value in the range $0 \le x < 10^5$.  Let
$a = 23$ and $b = 49$ represent two single precision variables.  The single precision product shall be written
as $c \leftarrow a \cdot b$ while the double precision product shall be written as $\hat c \leftarrow a \cdot b$.
In this particular case, $\hat c = 1127$ and $c = 127$.  The most significant digit of the product would not fit 
in a single precision data type and as a result $c \ne \hat c$.  

\subsection{Algorithm Inputs and Outputs}
Within the algorithm descriptions all variables are assumed to be scalars of either single or double precision
as indicated.  The only exception to this rule is when variables have been indicated to be of type mp\_int.  This 
distinction is important as scalars are often used as array indicies and various other counters.  

\subsection{Mathematical Expressions}
The $\lfloor \mbox{ } \rfloor$ brackets imply an expression truncated to an integer not greater than the expression 
itself.  For example, $\lfloor 5.7 \rfloor = 5$.  Similarly the $\lceil \mbox{ } \rceil$ brackets imply an expression
rounded to an integer not less than the expression itself.  For example, $\lceil 5.1 \rceil = 6$.  Typically when 
the $/$ division symbol is used the intention is to perform an integer division with truncation.  For example, 
$5/2 = 2$ which will often be written as $\lfloor 5/2 \rfloor = 2$ for clarity.  When an expression is written as a 
fraction a real value division is implied, for example ${5 \over 2} = 2.5$.  

The norm of a multiple precision integer, for example $\vert \vert x \vert \vert$, will be used to represent the number of digits in the representation
of the integer.  For example, $\vert \vert 123 \vert \vert = 3$ and $\vert \vert 79452 \vert \vert = 5$.  

\subsection{Work Effort}
\index{big-Oh}
To measure the efficiency of the specified algorithms, a modified big-Oh notation is used.  In this system all 
single precision operations are considered to have the same cost\footnote{Except where explicitly noted.}.  
That is a single precision addition, multiplication and division are assumed to take the same time to 
complete.  While this is generally not true in practice, it will simplify the discussions considerably.

Some algorithms have slight advantages over others which is why some constants will not be removed in 
the notation.  For example, a normal baseline multiplication (section \ref{sec:basemult}) requires $O(n^2)$ work while a 
baseline squaring (section \ref{sec:basesquare}) requires $O({{n^2 + n}\over 2})$ work.  In standard big-Oh notation these 
would both be said to be equivalent to $O(n^2)$.  However, 
in the context of the this text this is not the case as the magnitude of the inputs will typically be rather small.  As a 
result small constant factors in the work effort will make an observable difference in algorithm efficiency.

All of the algorithms presented in this text have a polynomial time work level.  That is, of the form 
$O(n^k)$ for $n, k \in \Z^{+}$.  This will help make useful comparisons in terms of the speed of the algorithms and how 
various optimizations will help pay off in the long run.

\section{Exercises}
Within the more advanced chapters a section will be set aside to give the reader some challenging exercises related to
the discussion at hand.  These exercises are not designed to be prize winning problems, but instead to be thought 
provoking.  Wherever possible the problems are forward minded, stating problems that will be answered in subsequent 
chapters.  The reader is encouraged to finish the exercises as they appear to get a better understanding of the 
subject material.  

That being said, the problems are designed to affirm knowledge of a particular subject matter.  Students in particular
are encouraged to verify they can answer the problems correctly before moving on.

Similar to the exercises of \cite[pp. ix]{TAOCPV2} these exercises are given a scoring system based on the difficulty of
the problem.  However, unlike \cite{TAOCPV2} the problems do not get nearly as hard.  The scoring of these 
exercises ranges from one (the easiest) to five (the hardest).  The following table sumarizes the 
scoring system used.

\begin{figure}[here]
\begin{center}
\begin{small}
\begin{tabular}{|c|l|}
\hline $\left [ 1 \right ]$ & An easy problem that should only take the reader a manner of \\
                            & minutes to solve.  Usually does not involve much computer time \\
                            & to solve. \\
\hline $\left [ 2 \right ]$ & An easy problem that involves a marginal amount of computer \\
                     & time usage.  Usually requires a program to be written to \\
                     & solve the problem. \\
\hline $\left [ 3 \right ]$ & A moderately hard problem that requires a non-trivial amount \\
                     & of work.  Usually involves trivial research and development of \\
                     & new theory from the perspective of a student. \\
\hline $\left [ 4 \right ]$ & A moderately hard problem that involves a non-trivial amount \\
                     & of work and research, the solution to which will demonstrate \\
                     & a higher mastery of the subject matter. \\
\hline $\left [ 5 \right ]$ & A hard problem that involves concepts that are difficult for a \\
                     & novice to solve.  Solutions to these problems will demonstrate a \\
                     & complete mastery of the given subject. \\
\hline
\end{tabular}
\end{small}
\end{center}
\caption{Exercise Scoring System}
\end{figure}

Problems at the first level are meant to be simple questions that the reader can answer quickly without programming a solution or
devising new theory.  These problems are quick tests to see if the material is understood.  Problems at the second level 
are also designed to be easy but will require a program or algorithm to be implemented to arrive at the answer.  These
two levels are essentially entry level questions.  

Problems at the third level are meant to be a bit more difficult than the first two levels.  The answer is often 
fairly obvious but arriving at an exacting solution requires some thought and skill.  These problems will almost always 
involve devising a new algorithm or implementing a variation of another algorithm previously presented.  Readers who can
answer these questions will feel comfortable with the concepts behind the topic at hand.

Problems at the fourth level are meant to be similar to those of the level three questions except they will require 
additional research to be completed.  The reader will most likely not know the answer right away, nor will the text provide 
the exact details of the answer until a subsequent chapter.  

Problems at the fifth level are meant to be the hardest 
problems relative to all the other problems in the chapter.  People who can correctly answer fifth level problems have a 
mastery of the subject matter at hand.

Often problems will be tied together.  The purpose of this is to start a chain of thought that will be discussed in future chapters.  The reader
is encouraged to answer the follow-up problems and try to draw the relevance of problems.

\section{Introduction to LibTomMath}

\subsection{What is LibTomMath?}
LibTomMath is a free and open source multiple precision integer library written entirely in portable ISO C.  By portable it 
is meant that the library does not contain any code that is computer platform dependent or otherwise problematic to use on 
any given platform.  

The library has been successfully tested under numerous operating systems including Unix\footnote{All of these
trademarks belong to their respective rightful owners.}, MacOS, Windows, Linux, PalmOS and on standalone hardware such 
as the Gameboy Advance.  The library is designed to contain enough functionality to be able to develop applications such 
as public key cryptosystems and still maintain a relatively small footprint.

\subsection{Goals of LibTomMath}

Libraries which obtain the most efficiency are rarely written in a high level programming language such as C.  However, 
even though this library is written entirely in ISO C, considerable care has been taken to optimize the algorithm implementations within the 
library.  Specifically the code has been written to work well with the GNU C Compiler (\textit{GCC}) on both x86 and ARM 
processors.  Wherever possible, highly efficient algorithms, such as Karatsuba multiplication, sliding window 
exponentiation and Montgomery reduction have been provided to make the library more efficient.  

Even with the nearly optimal and specialized algorithms that have been included the Application Programing Interface 
(\textit{API}) has been kept as simple as possible.  Often generic place holder routines will make use of specialized 
algorithms automatically without the developer's specific attention.  One such example is the generic multiplication 
algorithm \textbf{mp\_mul()} which will automatically use Toom--Cook, Karatsuba, Comba or baseline multiplication 
based on the magnitude of the inputs and the configuration of the library.  

Making LibTomMath as efficient as possible is not the only goal of the LibTomMath project.  Ideally the library should 
be source compatible with another popular library which makes it more attractive for developers to use.  In this case the
MPI library was used as a API template for all the basic functions.  MPI was chosen because it is another library that fits 
in the same niche as LibTomMath.  Even though LibTomMath uses MPI as the template for the function names and argument 
passing conventions, it has been written from scratch by Tom St Denis.

The project is also meant to act as a learning tool for students, the logic being that no easy-to-follow ``bignum'' 
library exists which can be used to teach computer science students how to perform fast and reliable multiple precision 
integer arithmetic.  To this end the source code has been given quite a few comments and algorithm discussion points.  

\section{Choice of LibTomMath}
LibTomMath was chosen as the case study of this text not only because the author of both projects is one and the same but
for more worthy reasons.  Other libraries such as GMP \cite{GMP}, MPI \cite{MPI}, LIP \cite{LIP} and OpenSSL 
\cite{OPENSSL} have multiple precision integer arithmetic routines but would not be ideal for this text for 
reasons that will be explained in the following sub-sections.

\subsection{Code Base}
The LibTomMath code base is all portable ISO C source code.  This means that there are no platform dependent conditional
segments of code littered throughout the source.  This clean and uncluttered approach to the library means that a
developer can more readily discern the true intent of a given section of source code without trying to keep track of
what conditional code will be used.

The code base of LibTomMath is well organized.  Each function is in its own separate source code file 
which allows the reader to find a given function very quickly.  On average there are $76$ lines of code per source
file which makes the source very easily to follow.  By comparison MPI and LIP are single file projects making code tracing
very hard.  GMP has many conditional code segments which also hinder tracing.  

When compiled with GCC for the x86 processor and optimized for speed the entire library is approximately $100$KiB\footnote{The notation ``KiB'' means $2^{10}$ octets, similarly ``MiB'' means $2^{20}$ octets.}
 which is fairly small compared to GMP (over $250$KiB).  LibTomMath is slightly larger than MPI (which compiles to about 
$50$KiB) but LibTomMath is also much faster and more complete than MPI.

\subsection{API Simplicity}
LibTomMath is designed after the MPI library and shares the API design.  Quite often programs that use MPI will build 
with LibTomMath without change. The function names correlate directly to the action they perform.  Almost all of the 
functions share the same parameter passing convention.  The learning curve is fairly shallow with the API provided 
which is an extremely valuable benefit for the student and developer alike.  

The LIP library is an example of a library with an API that is awkward to work with.  LIP uses function names that are often ``compressed'' to 
illegible short hand.  LibTomMath does not share this characteristic.  

The GMP library also does not return error codes.  Instead it uses a POSIX.1 \cite{POSIX1} signal system where errors
are signaled to the host application.  This happens to be the fastest approach but definitely not the most versatile.  In
effect a math error (i.e. invalid input, heap error, etc) can cause a program to stop functioning which is definitely 
undersireable in many situations.

\subsection{Optimizations}
While LibTomMath is certainly not the fastest library (GMP often beats LibTomMath by a factor of two) it does
feature a set of optimal algorithms for tasks such as modular reduction, exponentiation, multiplication and squaring.  GMP 
and LIP also feature such optimizations while MPI only uses baseline algorithms with no optimizations.  GMP lacks a few
of the additional modular reduction optimizations that LibTomMath features\footnote{At the time of this writing GMP
only had Barrett and Montgomery modular reduction algorithms.}.  

LibTomMath is almost always an order of magnitude faster than the MPI library at computationally expensive tasks such as modular
exponentiation.  In the grand scheme of ``bignum'' libraries LibTomMath is faster than the average library and usually  
slower than the best libraries such as GMP and OpenSSL by only a small factor.

\subsection{Portability and Stability}
LibTomMath will build ``out of the box'' on any platform equipped with a modern version of the GNU C Compiler 
(\textit{GCC}).  This means that without changes the library will build without configuration or setting up any 
variables.  LIP and MPI will build ``out of the box'' as well but have numerous known bugs.  Most notably the author of 
MPI has recently stopped working on his library and LIP has long since been discontinued.  

GMP requires a configuration script to run and will not build out of the box.   GMP and LibTomMath are still in active
development and are very stable across a variety of platforms.

\subsection{Choice}
LibTomMath is a relatively compact, well documented, highly optimized and portable library which seems only natural for
the case study of this text.  Various source files from the LibTomMath project will be included within the text.  However, 
the reader is encouraged to download their own copy of the library to actually be able to work with the library.  

\chapter{Getting Started}
\section{Library Basics}
The trick to writing any useful library of source code is to build a solid foundation and work outwards from it.  First, 
a problem along with allowable solution parameters should be identified and analyzed.  In this particular case the 
inability to accomodate multiple precision integers is the problem.  Futhermore, the solution must be written
as portable source code that is reasonably efficient across several different computer platforms.

After a foundation is formed the remainder of the library can be designed and implemented in a hierarchical fashion.  
That is, to implement the lowest level dependencies first and work towards the most abstract functions last.  For example, 
before implementing a modular exponentiation algorithm one would implement a modular reduction algorithm.
By building outwards from a base foundation instead of using a parallel design methodology the resulting project is 
highly modular.  Being highly modular is a desirable property of any project as it often means the resulting product
has a small footprint and updates are easy to perform.  

Usually when I start a project I will begin with the header files.  I define the data types I think I will need and 
prototype the initial functions that are not dependent on other functions (within the library).  After I 
implement these base functions I prototype more dependent functions and implement them.   The process repeats until
I implement all of the functions I require.  For example, in the case of LibTomMath I implemented functions such as 
mp\_init() well before I implemented mp\_mul() and even further before I implemented mp\_exptmod().  As an example as to 
why this design works note that the Karatsuba and Toom-Cook multipliers were written \textit{after} the 
dependent function mp\_exptmod() was written.  Adding the new multiplication algorithms did not require changes to the 
mp\_exptmod() function itself and lowered the total cost of ownership (\textit{so to speak}) and of development 
for new algorithms.  This methodology allows new algorithms to be tested in a complete framework with relative ease.

\begin{center}
\begin{figure}[here]
\includegraphics{pics/design_process.ps}
\caption{Design Flow of the First Few Original LibTomMath Functions.}
\label{pic:design_process}
\end{figure}
\end{center}

Only after the majority of the functions were in place did I pursue a less hierarchical approach to auditing and optimizing
the source code.  For example, one day I may audit the multipliers and the next day the polynomial basis functions.  

It only makes sense to begin the text with the preliminary data types and support algorithms required as well.  
This chapter discusses the core algorithms of the library which are the dependents for every other algorithm.

\section{What is a Multiple Precision Integer?}
Recall that most programming languages, in particular ISO C \cite{ISOC}, only have fixed precision data types that on their own cannot 
be used to represent values larger than their precision will allow. The purpose of multiple precision algorithms is 
to use fixed precision data types to create and manipulate multiple precision integers which may represent values 
that are very large.  

As a well known analogy, school children are taught how to form numbers larger than nine by prepending more radix ten digits.  In the decimal system
the largest single digit value is $9$.  However, by concatenating digits together larger numbers may be represented.  Newly prepended digits 
(\textit{to the left}) are said to be in a different power of ten column.  That is, the number $123$ can be described as having a $1$ in the hundreds 
column, $2$ in the tens column and $3$ in the ones column.  Or more formally $123 = 1 \cdot 10^2 + 2 \cdot 10^1 + 3 \cdot 10^0$.  Computer based 
multiple precision arithmetic is essentially the same concept.  Larger integers are represented by adjoining fixed 
precision computer words with the exception that a different radix is used.

What most people probably do not think about explicitly are the various other attributes that describe a multiple precision 
integer.  For example, the integer $154_{10}$ has two immediately obvious properties.  First, the integer is positive, 
that is the sign of this particular integer is positive as opposed to negative.  Second, the integer has three digits in 
its representation.  There is an additional property that the integer posesses that does not concern pencil-and-paper 
arithmetic.  The third property is how many digits placeholders are available to hold the integer.  

The human analogy of this third property is ensuring there is enough space on the paper to write the integer.  For example,
if one starts writing a large number too far to the right on a piece of paper they will have to erase it and move left.  
Similarly, computer algorithms must maintain strict control over memory usage to ensure that the digits of an integer
will not exceed the allowed boundaries.  These three properties make up what is known as a multiple precision 
integer or mp\_int for short.  

\subsection{The mp\_int Structure}
\label{sec:MPINT}
The mp\_int structure is the ISO C based manifestation of what represents a multiple precision integer.  The ISO C standard does not provide for 
any such data type but it does provide for making composite data types known as structures.  The following is the structure definition 
used within LibTomMath.

\index{mp\_int}
\begin{figure}[here]
\begin{center}
\begin{small}
%\begin{verbatim}
\begin{tabular}{|l|}
\hline
typedef struct \{ \\
\hspace{3mm}int used, alloc, sign;\\
\hspace{3mm}mp\_digit *dp;\\
\} \textbf{mp\_int}; \\
\hline
\end{tabular}
%\end{verbatim}
\end{small}
\caption{The mp\_int Structure}
\label{fig:mpint}
\end{center}
\end{figure}

The mp\_int structure (fig. \ref{fig:mpint}) can be broken down as follows.

\begin{enumerate}
\item The \textbf{used} parameter denotes how many digits of the array \textbf{dp} contain the digits used to represent
a given integer.  The \textbf{used} count must be positive (or zero) and may not exceed the \textbf{alloc} count.  

\item The \textbf{alloc} parameter denotes how 
many digits are available in the array to use by functions before it has to increase in size.  When the \textbf{used} count 
of a result would exceed the \textbf{alloc} count all of the algorithms will automatically increase the size of the 
array to accommodate the precision of the result.  

\item The pointer \textbf{dp} points to a dynamically allocated array of digits that represent the given multiple 
precision integer.  It is padded with $(\textbf{alloc} - \textbf{used})$ zero digits.  The array is maintained in a least 
significant digit order.  As a pencil and paper analogy the array is organized such that the right most digits are stored
first starting at the location indexed by zero\footnote{In C all arrays begin at zero.} in the array.  For example, 
if \textbf{dp} contains $\lbrace a, b, c, \ldots \rbrace$ where \textbf{dp}$_0 = a$, \textbf{dp}$_1 = b$, \textbf{dp}$_2 = c$, $\ldots$ then 
it would represent the integer $a + b\beta + c\beta^2 + \ldots$  

\index{MP\_ZPOS} \index{MP\_NEG}
\item The \textbf{sign} parameter denotes the sign as either zero/positive (\textbf{MP\_ZPOS}) or negative (\textbf{MP\_NEG}).  
\end{enumerate}

\subsubsection{Valid mp\_int Structures}
Several rules are placed on the state of an mp\_int structure and are assumed to be followed for reasons of efficiency.  
The only exceptions are when the structure is passed to initialization functions such as mp\_init() and mp\_init\_copy().

\begin{enumerate}
\item The value of \textbf{alloc} may not be less than one.  That is \textbf{dp} always points to a previously allocated
array of digits.
\item The value of \textbf{used} may not exceed \textbf{alloc} and must be greater than or equal to zero.
\item The value of \textbf{used} implies the digit at index $(used - 1)$ of the \textbf{dp} array is non-zero.  That is, 
leading zero digits in the most significant positions must be trimmed.
   \begin{enumerate}
   \item Digits in the \textbf{dp} array at and above the \textbf{used} location must be zero.
   \end{enumerate}
\item The value of \textbf{sign} must be \textbf{MP\_ZPOS} if \textbf{used} is zero; 
this represents the mp\_int value of zero.
\end{enumerate}

\section{Argument Passing}
A convention of argument passing must be adopted early on in the development of any library.  Making the function 
prototypes consistent will help eliminate many headaches in the future as the library grows to significant complexity.  
In LibTomMath the multiple precision integer functions accept parameters from left to right as pointers to mp\_int 
structures.  That means that the source (input) operands are placed on the left and the destination (output) on the right.   
Consider the following examples.

\begin{verbatim}
   mp_mul(&a, &b, &c);   /* c = a * b */
   mp_add(&a, &b, &a);   /* a = a + b */
   mp_sqr(&a, &b);       /* b = a * a */
\end{verbatim}

The left to right order is a fairly natural way to implement the functions since it lets the developer read aloud the
functions and make sense of them.  For example, the first function would read ``multiply a and b and store in c''.

Certain libraries (\textit{LIP by Lenstra for instance}) accept parameters the other way around, to mimic the order
of assignment expressions.  That is, the destination (output) is on the left and arguments (inputs) are on the right.  In 
truth, it is entirely a matter of preference.  In the case of LibTomMath the convention from the MPI library has been 
adopted.  

Another very useful design consideration, provided for in LibTomMath, is whether to allow argument sources to also be a 
destination.  For example, the second example (\textit{mp\_add}) adds $a$ to $b$ and stores in $a$.  This is an important 
feature to implement since it allows the calling functions to cut down on the number of variables it must maintain.  
However, to implement this feature specific care has to be given to ensure the destination is not modified before the 
source is fully read.

\section{Return Values}
A well implemented application, no matter what its purpose, should trap as many runtime errors as possible and return them 
to the caller.  By catching runtime errors a library can be guaranteed to prevent undefined behaviour.  However, the end 
developer can still manage to cause a library to crash.  For example, by passing an invalid pointer an application may
fault by dereferencing memory not owned by the application.

In the case of LibTomMath the only errors that are checked for are related to inappropriate inputs (division by zero for 
instance) and memory allocation errors.  It will not check that the mp\_int passed to any function is valid nor 
will it check pointers for validity.  Any function that can cause a runtime error will return an error code as an 
\textbf{int} data type with one of the following values (fig \ref{fig:errcodes}).

\index{MP\_OKAY} \index{MP\_VAL} \index{MP\_MEM}
\begin{figure}[here]
\begin{center}
\begin{tabular}{|l|l|}
\hline \textbf{Value} & \textbf{Meaning} \\
\hline \textbf{MP\_OKAY} & The function was successful \\
\hline \textbf{MP\_VAL}  & One of the input value(s) was invalid \\
\hline \textbf{MP\_MEM}  & The function ran out of heap memory \\
\hline
\end{tabular}
\end{center}
\caption{LibTomMath Error Codes}
\label{fig:errcodes}
\end{figure}

When an error is detected within a function it should free any memory it allocated, often during the initialization of
temporary mp\_ints, and return as soon as possible.  The goal is to leave the system in the same state it was when the 
function was called.  Error checking with this style of API is fairly simple.

\begin{verbatim}
   int err;
   if ((err = mp_add(&a, &b, &c)) != MP_OKAY) {
      printf("Error: %s\n", mp_error_to_string(err));
      exit(EXIT_FAILURE);
   }
\end{verbatim}

The GMP \cite{GMP} library uses C style \textit{signals} to flag errors which is of questionable use.  Not all errors are fatal 
and it was not deemed ideal by the author of LibTomMath to force developers to have signal handlers for such cases.

\section{Initialization and Clearing}
The logical starting point when actually writing multiple precision integer functions is the initialization and 
clearing of the mp\_int structures.  These two algorithms will be used by the majority of the higher level algorithms.

Given the basic mp\_int structure an initialization routine must first allocate memory to hold the digits of
the integer.  Often it is optimal to allocate a sufficiently large pre-set number of digits even though
the initial integer will represent zero.  If only a single digit were allocated quite a few subsequent re-allocations
would occur when operations are performed on the integers.  There is a tradeoff between how many default digits to allocate
and how many re-allocations are tolerable.  Obviously allocating an excessive amount of digits initially will waste 
memory and become unmanageable.  

If the memory for the digits has been successfully allocated then the rest of the members of the structure must
be initialized.  Since the initial state of an mp\_int is to represent the zero integer, the allocated digits must be set
to zero.  The \textbf{used} count set to zero and \textbf{sign} set to \textbf{MP\_ZPOS}.

\subsection{Initializing an mp\_int}
An mp\_int is said to be initialized if it is set to a valid, preferably default, state such that all of the members of the
structure are set to valid values.  The mp\_init algorithm will perform such an action.

\index{mp\_init}
\begin{figure}[here]
\begin{center}
\begin{tabular}{l}
\hline Algorithm \textbf{mp\_init}. \\
\textbf{Input}.   An mp\_int $a$ \\
\textbf{Output}.  Allocate memory and initialize $a$ to a known valid mp\_int state.  \\
\hline \\
1.  Allocate memory for \textbf{MP\_PREC} digits. \\
2.  If the allocation failed return(\textit{MP\_MEM}) \\
3.  for $n$ from $0$ to $MP\_PREC - 1$ do  \\
\hspace{3mm}3.1  $a_n \leftarrow 0$\\
4.  $a.sign \leftarrow MP\_ZPOS$\\
5.  $a.used \leftarrow 0$\\
6.  $a.alloc \leftarrow MP\_PREC$\\
7.  Return(\textit{MP\_OKAY})\\
\hline
\end{tabular}
\end{center}
\caption{Algorithm mp\_init}
\end{figure}

\textbf{Algorithm mp\_init.}
The purpose of this function is to initialize an mp\_int structure so that the rest of the library can properly
manipulte it.  It is assumed that the input may not have had any of its members previously initialized which is certainly
a valid assumption if the input resides on the stack.  

Before any of the members such as \textbf{sign}, \textbf{used} or \textbf{alloc} are initialized the memory for
the digits is allocated.  If this fails the function returns before setting any of the other members.  The \textbf{MP\_PREC} 
name represents a constant\footnote{Defined in the ``tommath.h'' header file within LibTomMath.} 
used to dictate the minimum precision of newly initialized mp\_int integers.  Ideally, it is at least equal to the smallest
precision number you'll be working with.

Allocating a block of digits at first instead of a single digit has the benefit of lowering the number of usually slow
heap operations later functions will have to perform in the future.  If \textbf{MP\_PREC} is set correctly the slack 
memory and the number of heap operations will be trivial.

Once the allocation has been made the digits have to be set to zero as well as the \textbf{used}, \textbf{sign} and
\textbf{alloc} members initialized.  This ensures that the mp\_int will always represent the default state of zero regardless
of the original condition of the input.

\textbf{Remark.}
This function introduces the idiosyncrasy that all iterative loops, commonly initiated with the ``for'' keyword, iterate incrementally
when the ``to'' keyword is placed between two expressions.  For example, ``for $a$ from $b$ to $c$ do'' means that
a subsequent expression (or body of expressions) are to be evaluated upto $c - b$ times so long as $b \le c$.  In each
iteration the variable $a$ is substituted for a new integer that lies inclusively between $b$ and $c$.  If $b > c$ occured
the loop would not iterate.  By contrast if the ``downto'' keyword were used in place of ``to'' the loop would iterate 
decrementally.

\vspace{+3mm}\begin{small}
\hspace{-5.1mm}{\bf File}: bn\_mp\_init.c
\vspace{-3mm}
\begin{alltt}
\end{alltt}
\end{small}

One immediate observation of this initializtion function is that it does not return a pointer to a mp\_int structure.  It 
is assumed that the caller has already allocated memory for the mp\_int structure, typically on the application stack.  The 
call to mp\_init() is used only to initialize the members of the structure to a known default state.  

Here we see (line 24) the memory allocation is performed first.  This allows us to exit cleanly and quickly
if there is an error.  If the allocation fails the routine will return \textbf{MP\_MEM} to the caller to indicate there
was a memory error.  The function XMALLOC is what actually allocates the memory.  Technically XMALLOC is not a function
but a macro defined in ``tommath.h``.  By default, XMALLOC will evaluate to malloc() which is the C library's built--in
memory allocation routine.

In order to assure the mp\_int is in a known state the digits must be set to zero.  On most platforms this could have been
accomplished by using calloc() instead of malloc().  However,  to correctly initialize a integer type to a given value in a 
portable fashion you have to actually assign the value.  The for loop (line 30) performs this required
operation.

After the memory has been successfully initialized the remainder of the members are initialized 
(lines 34 through 35) to their respective default states.  At this point the algorithm has succeeded and
a success code is returned to the calling function.  If this function returns \textbf{MP\_OKAY} it is safe to assume the 
mp\_int structure has been properly initialized and is safe to use with other functions within the library.  

\subsection{Clearing an mp\_int}
When an mp\_int is no longer required by the application, the memory that has been allocated for its digits must be 
returned to the application's memory pool with the mp\_clear algorithm.

\begin{figure}[here]
\begin{center}
\begin{tabular}{l}
\hline Algorithm \textbf{mp\_clear}. \\
\textbf{Input}.   An mp\_int $a$ \\
\textbf{Output}.  The memory for $a$ shall be deallocated.  \\
\hline \\
1.  If $a$ has been previously freed then return(\textit{MP\_OKAY}). \\
2.  for $n$ from 0 to $a.used - 1$ do \\
\hspace{3mm}2.1  $a_n \leftarrow 0$ \\
3.  Free the memory allocated for the digits of $a$. \\
4.  $a.used \leftarrow 0$ \\
5.  $a.alloc \leftarrow 0$ \\
6.  $a.sign \leftarrow MP\_ZPOS$ \\
7.  Return(\textit{MP\_OKAY}). \\
\hline
\end{tabular}
\end{center}
\caption{Algorithm mp\_clear}
\end{figure}

\textbf{Algorithm mp\_clear.}
This algorithm accomplishes two goals.  First, it clears the digits and the other mp\_int members.  This ensures that 
if a developer accidentally re-uses a cleared structure it is less likely to cause problems.  The second goal
is to free the allocated memory.

The logic behind the algorithm is extended by marking cleared mp\_int structures so that subsequent calls to this
algorithm will not try to free the memory multiple times.  Cleared mp\_ints are detectable by having a pre-defined invalid 
digit pointer \textbf{dp} setting.

Once an mp\_int has been cleared the mp\_int structure is no longer in a valid state for any other algorithm
with the exception of algorithms mp\_init, mp\_init\_copy, mp\_init\_size and mp\_clear.

\vspace{+3mm}\begin{small}
\hspace{-5.1mm}{\bf File}: bn\_mp\_clear.c
\vspace{-3mm}
\begin{alltt}
\end{alltt}
\end{small}

The algorithm only operates on the mp\_int if it hasn't been previously cleared.  The if statement (line 25)
checks to see if the \textbf{dp} member is not \textbf{NULL}.  If the mp\_int is a valid mp\_int then \textbf{dp} cannot be
\textbf{NULL} in which case the if statement will evaluate to true.

The digits of the mp\_int are cleared by the for loop (line 27) which assigns a zero to every digit.  Similar to mp\_init()
the digits are assigned zero instead of using block memory operations (such as memset()) since this is more portable.  

The digits are deallocated off the heap via the XFREE macro.  Similar to XMALLOC the XFREE macro actually evaluates to
a standard C library function.  In this case the free() function.  Since free() only deallocates the memory the pointer
still has to be reset to \textbf{NULL} manually (line 35).  

Now that the digits have been cleared and deallocated the other members are set to their final values (lines 36 and 37).

\section{Maintenance Algorithms}

The previous sections describes how to initialize and clear an mp\_int structure.  To further support operations
that are to be performed on mp\_int structures (such as addition and multiplication) the dependent algorithms must be
able to augment the precision of an mp\_int and 
initialize mp\_ints with differing initial conditions.  

These algorithms complete the set of low level algorithms required to work with mp\_int structures in the higher level
algorithms such as addition, multiplication and modular exponentiation.

\subsection{Augmenting an mp\_int's Precision}
When storing a value in an mp\_int structure, a sufficient number of digits must be available to accomodate the entire 
result of an operation without loss of precision.  Quite often the size of the array given by the \textbf{alloc} member 
is large enough to simply increase the \textbf{used} digit count.  However, when the size of the array is too small it 
must be re-sized appropriately to accomodate the result.  The mp\_grow algorithm will provide this functionality.

\newpage\begin{figure}[here]
\begin{center}
\begin{tabular}{l}
\hline Algorithm \textbf{mp\_grow}. \\
\textbf{Input}.   An mp\_int $a$ and an integer $b$. \\
\textbf{Output}.  $a$ is expanded to accomodate $b$ digits. \\
\hline \\
1.  if $a.alloc \ge b$ then return(\textit{MP\_OKAY}) \\
2.  $u \leftarrow b\mbox{ (mod }MP\_PREC\mbox{)}$ \\
3.  $v \leftarrow b + 2 \cdot MP\_PREC - u$ \\
4.  Re-allocate the array of digits $a$ to size $v$ \\
5.  If the allocation failed then return(\textit{MP\_MEM}). \\
6.  for n from a.alloc to $v - 1$ do  \\
\hspace{+3mm}6.1  $a_n \leftarrow 0$ \\
7.  $a.alloc \leftarrow v$ \\
8.  Return(\textit{MP\_OKAY}) \\
\hline
\end{tabular}
\end{center}
\caption{Algorithm mp\_grow}
\end{figure}

\textbf{Algorithm mp\_grow.}
It is ideal to prevent re-allocations from being performed if they are not required (step one).  This is useful to 
prevent mp\_ints from growing excessively in code that erroneously calls mp\_grow.  

The requested digit count is padded up to next multiple of \textbf{MP\_PREC} plus an additional \textbf{MP\_PREC} (steps two and three).  
This helps prevent many trivial reallocations that would grow an mp\_int by trivially small values.  

It is assumed that the reallocation (step four) leaves the lower $a.alloc$ digits of the mp\_int intact.  This is much 
akin to how the \textit{realloc} function from the standard C library works.  Since the newly allocated digits are 
assumed to contain undefined values they are initially set to zero.

\vspace{+3mm}\begin{small}
\hspace{-5.1mm}{\bf File}: bn\_mp\_grow.c
\vspace{-3mm}
\begin{alltt}
\end{alltt}
\end{small}

A quick optimization is to first determine if a memory re-allocation is required at all.  The if statement (line 24) checks
if the \textbf{alloc} member of the mp\_int is smaller than the requested digit count.  If the count is not larger than \textbf{alloc}
the function skips the re-allocation part thus saving time.

When a re-allocation is performed it is turned into an optimal request to save time in the future.  The requested digit count is
padded upwards to 2nd multiple of \textbf{MP\_PREC} larger than \textbf{alloc} (line 25).  The XREALLOC function is used
to re-allocate the memory.  As per the other functions XREALLOC is actually a macro which evaluates to realloc by default.  The realloc
function leaves the base of the allocation intact which means the first \textbf{alloc} digits of the mp\_int are the same as before
the re-allocation.  All	that is left is to clear the newly allocated digits and return.

Note that the re-allocation result is actually stored in a temporary pointer $tmp$.  This is to allow this function to return
an error with a valid pointer.  Earlier releases of the library stored the result of XREALLOC into the mp\_int $a$.  That would
result in a memory leak if XREALLOC ever failed.  

\subsection{Initializing Variable Precision mp\_ints}
Occasionally the number of digits required will be known in advance of an initialization, based on, for example, the size 
of input mp\_ints to a given algorithm.  The purpose of algorithm mp\_init\_size is similar to mp\_init except that it 
will allocate \textit{at least} a specified number of digits.  

\begin{figure}[here]
\begin{small}
\begin{center}
\begin{tabular}{l}
\hline Algorithm \textbf{mp\_init\_size}. \\
\textbf{Input}.   An mp\_int $a$ and the requested number of digits $b$. \\
\textbf{Output}.  $a$ is initialized to hold at least $b$ digits. \\
\hline \\
1.  $u \leftarrow b \mbox{ (mod }MP\_PREC\mbox{)}$ \\
2.  $v \leftarrow b + 2 \cdot MP\_PREC - u$ \\
3.  Allocate $v$ digits. \\
4.  for $n$ from $0$ to $v - 1$ do \\
\hspace{3mm}4.1  $a_n \leftarrow 0$ \\
5.  $a.sign \leftarrow MP\_ZPOS$\\
6.  $a.used \leftarrow 0$\\
7.  $a.alloc \leftarrow v$\\
8.  Return(\textit{MP\_OKAY})\\
\hline
\end{tabular}
\end{center}
\end{small}
\caption{Algorithm mp\_init\_size}
\end{figure}

\textbf{Algorithm mp\_init\_size.}
This algorithm will initialize an mp\_int structure $a$ like algorithm mp\_init with the exception that the number of 
digits allocated can be controlled by the second input argument $b$.  The input size is padded upwards so it is a 
multiple of \textbf{MP\_PREC} plus an additional \textbf{MP\_PREC} digits.  This padding is used to prevent trivial 
allocations from becoming a bottleneck in the rest of the algorithms.

Like algorithm mp\_init, the mp\_int structure is initialized to a default state representing the integer zero.  This 
particular algorithm is useful if it is known ahead of time the approximate size of the input.  If the approximation is
correct no further memory re-allocations are required to work with the mp\_int.

\vspace{+3mm}\begin{small}
\hspace{-5.1mm}{\bf File}: bn\_mp\_init\_size.c
\vspace{-3mm}
\begin{alltt}
\end{alltt}
\end{small}

The number of digits $b$ requested is padded (line 24) by first augmenting it to the next multiple of 
\textbf{MP\_PREC} and then adding \textbf{MP\_PREC} to the result.  If the memory can be successfully allocated the 
mp\_int is placed in a default state representing the integer zero.  Otherwise, the error code \textbf{MP\_MEM} will be 
returned (line 29).  

The digits are allocated and set to zero at the same time with the calloc() function (line @25,XCALLOC@).  The 
\textbf{used} count is set to zero, the \textbf{alloc} count set to the padded digit count and the \textbf{sign} flag set 
to \textbf{MP\_ZPOS} to achieve a default valid mp\_int state (lines 33, 34 and 35).  If the function 
returns succesfully then it is correct to assume that the mp\_int structure is in a valid state for the remainder of the 
functions to work with.

\subsection{Multiple Integer Initializations and Clearings}
Occasionally a function will require a series of mp\_int data types to be made available simultaneously.  
The purpose of algorithm mp\_init\_multi is to initialize a variable length array of mp\_int structures in a single
statement.  It is essentially a shortcut to multiple initializations.

\newpage\begin{figure}[here]
\begin{center}
\begin{tabular}{l}
\hline Algorithm \textbf{mp\_init\_multi}. \\
\textbf{Input}.   Variable length array $V_k$ of mp\_int variables of length $k$. \\
\textbf{Output}.  The array is initialized such that each mp\_int of $V_k$ is ready to use. \\
\hline \\
1.  for $n$ from 0 to $k - 1$ do \\
\hspace{+3mm}1.1.  Initialize the mp\_int $V_n$ (\textit{mp\_init}) \\
\hspace{+3mm}1.2.  If initialization failed then do \\
\hspace{+6mm}1.2.1.  for $j$ from $0$ to $n$ do \\
\hspace{+9mm}1.2.1.1.  Free the mp\_int $V_j$ (\textit{mp\_clear}) \\
\hspace{+6mm}1.2.2.   Return(\textit{MP\_MEM}) \\
2.  Return(\textit{MP\_OKAY}) \\
\hline
\end{tabular}
\end{center}
\caption{Algorithm mp\_init\_multi}
\end{figure}

\textbf{Algorithm mp\_init\_multi.}
The algorithm will initialize the array of mp\_int variables one at a time.  If a runtime error has been detected 
(\textit{step 1.2}) all of the previously initialized variables are cleared.  The goal is an ``all or nothing'' 
initialization which allows for quick recovery from runtime errors.

\vspace{+3mm}\begin{small}
\hspace{-5.1mm}{\bf File}: bn\_mp\_init\_multi.c
\vspace{-3mm}
\begin{alltt}
\end{alltt}
\end{small}

This function intializes a variable length list of mp\_int structure pointers.  However, instead of having the mp\_int
structures in an actual C array they are simply passed as arguments to the function.  This function makes use of the 
``...'' argument syntax of the C programming language.  The list is terminated with a final \textbf{NULL} argument 
appended on the right.  

The function uses the ``stdarg.h'' \textit{va} functions to step portably through the arguments to the function.  A count
$n$ of succesfully initialized mp\_int structures is maintained (line 48) such that if a failure does occur,
the algorithm can backtrack and free the previously initialized structures (lines 28 to 47).  


\subsection{Clamping Excess Digits}
When a function anticipates a result will be $n$ digits it is simpler to assume this is true within the body of 
the function instead of checking during the computation.  For example, a multiplication of a $i$ digit number by a 
$j$ digit produces a result of at most $i + j$ digits.  It is entirely possible that the result is $i + j - 1$ 
though, with no final carry into the last position.  However, suppose the destination had to be first expanded 
(\textit{via mp\_grow}) to accomodate $i + j - 1$ digits than further expanded to accomodate the final carry.  
That would be a considerable waste of time since heap operations are relatively slow.

The ideal solution is to always assume the result is $i + j$ and fix up the \textbf{used} count after the function
terminates.  This way a single heap operation (\textit{at most}) is required.  However, if the result was not checked
there would be an excess high order zero digit.  

For example, suppose the product of two integers was $x_n = (0x_{n-1}x_{n-2}...x_0)_{\beta}$.  The leading zero digit 
will not contribute to the precision of the result.  In fact, through subsequent operations more leading zero digits would
accumulate to the point the size of the integer would be prohibitive.  As a result even though the precision is very 
low the representation is excessively large.  

The mp\_clamp algorithm is designed to solve this very problem.  It will trim high-order zeros by decrementing the 
\textbf{used} count until a non-zero most significant digit is found.  Also in this system, zero is considered to be a 
positive number which means that if the \textbf{used} count is decremented to zero, the sign must be set to 
\textbf{MP\_ZPOS}.

\begin{figure}[here]
\begin{center}
\begin{tabular}{l}
\hline Algorithm \textbf{mp\_clamp}. \\
\textbf{Input}.   An mp\_int $a$ \\
\textbf{Output}.  Any excess leading zero digits of $a$ are removed \\
\hline \\
1.  while $a.used > 0$ and $a_{a.used - 1} = 0$ do \\
\hspace{+3mm}1.1  $a.used \leftarrow a.used - 1$ \\
2.  if $a.used = 0$ then do \\
\hspace{+3mm}2.1  $a.sign \leftarrow MP\_ZPOS$ \\
\hline \\
\end{tabular}
\end{center}
\caption{Algorithm mp\_clamp}
\end{figure}

\textbf{Algorithm mp\_clamp.}
As can be expected this algorithm is very simple.  The loop on step one is expected to iterate only once or twice at
the most.  For example, this will happen in cases where there is not a carry to fill the last position.  Step two fixes the sign for 
when all of the digits are zero to ensure that the mp\_int is valid at all times.

\vspace{+3mm}\begin{small}
\hspace{-5.1mm}{\bf File}: bn\_mp\_clamp.c
\vspace{-3mm}
\begin{alltt}
\end{alltt}
\end{small}

Note on line 28 how to test for the \textbf{used} count is made on the left of the \&\& operator.  In the C programming
language the terms to \&\& are evaluated left to right with a boolean short-circuit if any condition fails.  This is 
important since if the \textbf{used} is zero the test on the right would fetch below the array.  That is obviously 
undesirable.  The parenthesis on line 31 is used to make sure the \textbf{used} count is decremented and not
the pointer ``a''.  

\section*{Exercises}
\begin{tabular}{cl}
$\left [ 1 \right ]$ & Discuss the relevance of the \textbf{used} member of the mp\_int structure. \\
                     & \\
$\left [ 1 \right ]$ & Discuss the consequences of not using padding when performing allocations.  \\
                     & \\
$\left [ 2 \right ]$ & Estimate an ideal value for \textbf{MP\_PREC} when performing 1024-bit RSA \\
                     & encryption when $\beta = 2^{28}$.  \\
                     & \\
$\left [ 1 \right ]$ & Discuss the relevance of the algorithm mp\_clamp.  What does it prevent? \\
                     & \\
$\left [ 1 \right ]$ & Give an example of when the algorithm  mp\_init\_copy might be useful. \\
                     & \\
\end{tabular}


%%%
% CHAPTER FOUR
%%%

\chapter{Basic Operations}

\section{Introduction}
In the previous chapter a series of low level algorithms were established that dealt with initializing and maintaining
mp\_int structures.  This chapter will discuss another set of seemingly non-algebraic algorithms which will form the low 
level basis of the entire library.  While these algorithm are relatively trivial it is important to understand how they
work before proceeding since these algorithms will be used almost intrinsically in the following chapters.

The algorithms in this chapter deal primarily with more ``programmer'' related tasks such as creating copies of
mp\_int structures, assigning small values to mp\_int structures and comparisons of the values mp\_int structures
represent.   

\section{Assigning Values to mp\_int Structures}
\subsection{Copying an mp\_int}
Assigning the value that a given mp\_int structure represents to another mp\_int structure shall be known as making
a copy for the purposes of this text.  The copy of the mp\_int will be a separate entity that represents the same
value as the mp\_int it was copied from.  The mp\_copy algorithm provides this functionality. 

\newpage\begin{figure}[here]
\begin{center}
\begin{tabular}{l}
\hline Algorithm \textbf{mp\_copy}. \\
\textbf{Input}.  An mp\_int $a$ and $b$. \\
\textbf{Output}.  Store a copy of $a$ in $b$. \\
\hline \\
1.  If $b.alloc < a.used$ then grow $b$ to $a.used$ digits.  (\textit{mp\_grow}) \\
2.  for $n$ from 0 to $a.used - 1$ do \\
\hspace{3mm}2.1  $b_{n} \leftarrow a_{n}$ \\
3.  for $n$ from $a.used$ to $b.used - 1$ do \\
\hspace{3mm}3.1  $b_{n} \leftarrow 0$ \\
4.  $b.used \leftarrow a.used$ \\
5.  $b.sign \leftarrow a.sign$ \\
6.  return(\textit{MP\_OKAY}) \\
\hline
\end{tabular}
\end{center}
\caption{Algorithm mp\_copy}
\end{figure}

\textbf{Algorithm mp\_copy.}
This algorithm copies the mp\_int $a$ such that upon succesful termination of the algorithm the mp\_int $b$ will
represent the same integer as the mp\_int $a$.  The mp\_int $b$ shall be a complete and distinct copy of the 
mp\_int $a$ meaing that the mp\_int $a$ can be modified and it shall not affect the value of the mp\_int $b$.

If $b$ does not have enough room for the digits of $a$ it must first have its precision augmented via the mp\_grow 
algorithm.  The digits of $a$ are copied over the digits of $b$ and any excess digits of $b$ are set to zero (step two
and three).  The \textbf{used} and \textbf{sign} members of $a$ are finally copied over the respective members of
$b$.

\textbf{Remark.}  This algorithm also introduces a new idiosyncrasy that will be used throughout the rest of the
text.  The error return codes of other algorithms are not explicitly checked in the pseudo-code presented.  For example, in 
step one of the mp\_copy algorithm the return of mp\_grow is not explicitly checked to ensure it succeeded.  Text space is 
limited so it is assumed that if a algorithm fails it will clear all temporarily allocated mp\_ints and return
the error code itself.  However, the C code presented will demonstrate all of the error handling logic required to 
implement the pseudo-code.

\vspace{+3mm}\begin{small}
\hspace{-5.1mm}{\bf File}: bn\_mp\_copy.c
\vspace{-3mm}
\begin{alltt}
\end{alltt}
\end{small}

Occasionally a dependent algorithm may copy an mp\_int effectively into itself such as when the input and output
mp\_int structures passed to a function are one and the same.  For this case it is optimal to return immediately without 
copying digits (line 25).  

The mp\_int $b$ must have enough digits to accomodate the used digits of the mp\_int $a$.  If $b.alloc$ is less than
$a.used$ the algorithm mp\_grow is used to augment the precision of $b$ (lines 30 to 33).  In order to
simplify the inner loop that copies the digits from $a$ to $b$, two aliases $tmpa$ and $tmpb$ point directly at the digits
of the mp\_ints $a$ and $b$ respectively.  These aliases (lines 43 and 46) allow the compiler to access the digits without first dereferencing the
mp\_int pointers and then subsequently the pointer to the digits.  

After the aliases are established the digits from $a$ are copied into $b$ (lines 49 to 51) and then the excess 
digits of $b$ are set to zero (lines 54 to 56).  Both ``for'' loops make use of the pointer aliases and in 
fact the alias for $b$ is carried through into the second ``for'' loop to clear the excess digits.  This optimization 
allows the alias to stay in a machine register fairly easy between the two loops.

\textbf{Remarks.}  The use of pointer aliases is an implementation methodology first introduced in this function that will
be used considerably in other functions.  Technically, a pointer alias is simply a short hand alias used to lower the 
number of pointer dereferencing operations required to access data.  For example, a for loop may resemble

\begin{alltt}
for (x = 0; x < 100; x++) \{
    a->num[4]->dp[x] = 0;
\}
\end{alltt}

This could be re-written using aliases as 

\begin{alltt}
mp_digit *tmpa;
a = a->num[4]->dp;
for (x = 0; x < 100; x++) \{
    *a++ = 0;
\}
\end{alltt}

In this case an alias is used to access the 
array of digits within an mp\_int structure directly.  It may seem that a pointer alias is strictly not required 
as a compiler may optimize out the redundant pointer operations.  However, there are two dominant reasons to use aliases.

The first reason is that most compilers will not effectively optimize pointer arithmetic.  For example, some optimizations 
may work for the Microsoft Visual C++ compiler (MSVC) and not for the GNU C Compiler (GCC).  Also some optimizations may 
work for GCC and not MSVC.  As such it is ideal to find a common ground for as many compilers as possible.  Pointer 
aliases optimize the code considerably before the compiler even reads the source code which means the end compiled code 
stands a better chance of being faster.

The second reason is that pointer aliases often can make an algorithm simpler to read.  Consider the first ``for'' 
loop of the function mp\_copy() re-written to not use pointer aliases.

\begin{alltt}
    /* copy all the digits */
    for (n = 0; n < a->used; n++) \{
      b->dp[n] = a->dp[n];
    \}
\end{alltt}

Whether this code is harder to read depends strongly on the individual.  However, it is quantifiably slightly more 
complicated as there are four variables within the statement instead of just two.

\subsubsection{Nested Statements}
Another commonly used technique in the source routines is that certain sections of code are nested.  This is used in
particular with the pointer aliases to highlight code phases.  For example, a Comba multiplier (discussed in chapter six)
will typically have three different phases.  First the temporaries are initialized, then the columns calculated and 
finally the carries are propagated.  In this example the middle column production phase will typically be nested as it
uses temporary variables and aliases the most.

The nesting also simplies the source code as variables that are nested are only valid for their scope.  As a result
the various temporary variables required do not propagate into other sections of code.


\subsection{Creating a Clone}
Another common operation is to make a local temporary copy of an mp\_int argument.  To initialize an mp\_int 
and then copy another existing mp\_int into the newly intialized mp\_int will be known as creating a clone.  This is 
useful within functions that need to modify an argument but do not wish to actually modify the original copy.  The 
mp\_init\_copy algorithm has been designed to help perform this task.

\begin{figure}[here]
\begin{center}
\begin{tabular}{l}
\hline Algorithm \textbf{mp\_init\_copy}. \\
\textbf{Input}.   An mp\_int $a$ and $b$\\
\textbf{Output}.  $a$ is initialized to be a copy of $b$. \\
\hline \\
1.  Init $a$.  (\textit{mp\_init}) \\
2.  Copy $b$ to $a$.  (\textit{mp\_copy}) \\
3.  Return the status of the copy operation. \\
\hline
\end{tabular}
\end{center}
\caption{Algorithm mp\_init\_copy}
\end{figure}

\textbf{Algorithm mp\_init\_copy.}
This algorithm will initialize an mp\_int variable and copy another previously initialized mp\_int variable into it.  As 
such this algorithm will perform two operations in one step.  

\vspace{+3mm}\begin{small}
\hspace{-5.1mm}{\bf File}: bn\_mp\_init\_copy.c
\vspace{-3mm}
\begin{alltt}
\end{alltt}
\end{small}

This will initialize \textbf{a} and make it a verbatim copy of the contents of \textbf{b}.  Note that 
\textbf{a} will have its own memory allocated which means that \textbf{b} may be cleared after the call
and \textbf{a} will be left intact.  

\section{Zeroing an Integer}
Reseting an mp\_int to the default state is a common step in many algorithms.  The mp\_zero algorithm will be the algorithm used to
perform this task.

\begin{figure}[here]
\begin{center}
\begin{tabular}{l}
\hline Algorithm \textbf{mp\_zero}. \\
\textbf{Input}.   An mp\_int $a$ \\
\textbf{Output}.  Zero the contents of $a$ \\
\hline \\
1.  $a.used \leftarrow 0$ \\
2.  $a.sign \leftarrow$ MP\_ZPOS \\
3.  for $n$ from 0 to $a.alloc - 1$ do \\
\hspace{3mm}3.1  $a_n \leftarrow 0$ \\
\hline
\end{tabular}
\end{center}
\caption{Algorithm mp\_zero}
\end{figure}

\textbf{Algorithm mp\_zero.}
This algorithm simply resets a mp\_int to the default state.  

\vspace{+3mm}\begin{small}
\hspace{-5.1mm}{\bf File}: bn\_mp\_zero.c
\vspace{-3mm}
\begin{alltt}
\end{alltt}
\end{small}

After the function is completed, all of the digits are zeroed, the \textbf{used} count is zeroed and the 
\textbf{sign} variable is set to \textbf{MP\_ZPOS}.

\section{Sign Manipulation}
\subsection{Absolute Value}
With the mp\_int representation of an integer, calculating the absolute value is trivial.  The mp\_abs algorithm will compute
the absolute value of an mp\_int.

\begin{figure}[here]
\begin{center}
\begin{tabular}{l}
\hline Algorithm \textbf{mp\_abs}. \\
\textbf{Input}.   An mp\_int $a$ \\
\textbf{Output}.  Computes $b = \vert a \vert$ \\
\hline \\
1.  Copy $a$ to $b$.  (\textit{mp\_copy}) \\
2.  If the copy failed return(\textit{MP\_MEM}). \\
3.  $b.sign \leftarrow MP\_ZPOS$ \\
4.  Return(\textit{MP\_OKAY}) \\
\hline
\end{tabular}
\end{center}
\caption{Algorithm mp\_abs}
\end{figure}

\textbf{Algorithm mp\_abs.}
This algorithm computes the absolute of an mp\_int input.  First it copies $a$ over $b$.  This is an example of an
algorithm where the check in mp\_copy that determines if the source and destination are equal proves useful.  This allows,
for instance, the developer to pass the same mp\_int as the source and destination to this function without addition 
logic to handle it.

\vspace{+3mm}\begin{small}
\hspace{-5.1mm}{\bf File}: bn\_mp\_abs.c
\vspace{-3mm}
\begin{alltt}
\end{alltt}
\end{small}

This fairly trivial algorithm first eliminates non--required duplications (line 28) and then sets the
\textbf{sign} flag to \textbf{MP\_ZPOS}.

\subsection{Integer Negation}
With the mp\_int representation of an integer, calculating the negation is also trivial.  The mp\_neg algorithm will compute
the negative of an mp\_int input.

\begin{figure}[here]
\begin{center}
\begin{tabular}{l}
\hline Algorithm \textbf{mp\_neg}. \\
\textbf{Input}.   An mp\_int $a$ \\
\textbf{Output}.  Computes $b = -a$ \\
\hline \\
1.  Copy $a$ to $b$.  (\textit{mp\_copy}) \\
2.  If the copy failed return(\textit{MP\_MEM}). \\
3.  If $a.used = 0$ then return(\textit{MP\_OKAY}). \\
4.  If $a.sign = MP\_ZPOS$ then do \\
\hspace{3mm}4.1  $b.sign = MP\_NEG$. \\
5.  else do \\
\hspace{3mm}5.1  $b.sign = MP\_ZPOS$. \\
6.  Return(\textit{MP\_OKAY}) \\
\hline
\end{tabular}
\end{center}
\caption{Algorithm mp\_neg}
\end{figure}

\textbf{Algorithm mp\_neg.}
This algorithm computes the negation of an input.  First it copies $a$ over $b$.  If $a$ has no used digits then
the algorithm returns immediately.  Otherwise it flips the sign flag and stores the result in $b$.  Note that if 
$a$ had no digits then it must be positive by definition.  Had step three been omitted then the algorithm would return
zero as negative.

\vspace{+3mm}\begin{small}
\hspace{-5.1mm}{\bf File}: bn\_mp\_neg.c
\vspace{-3mm}
\begin{alltt}
\end{alltt}
\end{small}

Like mp\_abs() this function avoids non--required duplications (line 22) and then sets the sign.  We
have to make sure that only non--zero values get a \textbf{sign} of \textbf{MP\_NEG}.  If the mp\_int is zero
than the \textbf{sign} is hard--coded to \textbf{MP\_ZPOS}.

\section{Small Constants}
\subsection{Setting Small Constants}
Often a mp\_int must be set to a relatively small value such as $1$ or $2$.  For these cases the mp\_set algorithm is useful.

\newpage\begin{figure}[here]
\begin{center}
\begin{tabular}{l}
\hline Algorithm \textbf{mp\_set}. \\
\textbf{Input}.   An mp\_int $a$ and a digit $b$ \\
\textbf{Output}.  Make $a$ equivalent to $b$ \\
\hline \\
1.  Zero $a$ (\textit{mp\_zero}). \\
2.  $a_0 \leftarrow b \mbox{ (mod }\beta\mbox{)}$ \\
3.  $a.used \leftarrow  \left \lbrace \begin{array}{ll}
                              1 &  \mbox{if }a_0 > 0 \\
                              0 &  \mbox{if }a_0 = 0 
                              \end{array} \right .$ \\
\hline                              
\end{tabular}
\end{center}
\caption{Algorithm mp\_set}
\end{figure}

\textbf{Algorithm mp\_set.}
This algorithm sets a mp\_int to a small single digit value.  Step number 1 ensures that the integer is reset to the default state.  The
single digit is set (\textit{modulo $\beta$}) and the \textbf{used} count is adjusted accordingly.

\vspace{+3mm}\begin{small}
\hspace{-5.1mm}{\bf File}: bn\_mp\_set.c
\vspace{-3mm}
\begin{alltt}
\end{alltt}
\end{small}

First we zero (line 21) the mp\_int to make sure that the other members are initialized for a 
small positive constant.  mp\_zero() ensures that the \textbf{sign} is positive and the \textbf{used} count
is zero.  Next we set the digit and reduce it modulo $\beta$ (line 22).  After this step we have to 
check if the resulting digit is zero or not.  If it is not then we set the \textbf{used} count to one, otherwise
to zero.

We can quickly reduce modulo $\beta$ since it is of the form $2^k$ and a quick binary AND operation with 
$2^k - 1$ will perform the same operation.

One important limitation of this function is that it will only set one digit.  The size of a digit is not fixed, meaning source that uses 
this function should take that into account.  Only trivially small constants can be set using this function.

\subsection{Setting Large Constants}
To overcome the limitations of the mp\_set algorithm the mp\_set\_int algorithm is ideal.  It accepts a ``long''
data type as input and will always treat it as a 32-bit integer.

\begin{figure}[here]
\begin{center}
\begin{tabular}{l}
\hline Algorithm \textbf{mp\_set\_int}. \\
\textbf{Input}.   An mp\_int $a$ and a ``long'' integer $b$ \\
\textbf{Output}.  Make $a$ equivalent to $b$ \\
\hline \\
1.  Zero $a$ (\textit{mp\_zero}) \\
2.  for $n$ from 0 to 7 do \\
\hspace{3mm}2.1  $a \leftarrow a \cdot 16$ (\textit{mp\_mul2d}) \\
\hspace{3mm}2.2  $u \leftarrow \lfloor b / 2^{4(7 - n)} \rfloor \mbox{ (mod }16\mbox{)}$\\
\hspace{3mm}2.3  $a_0 \leftarrow a_0 + u$ \\
\hspace{3mm}2.4  $a.used \leftarrow a.used + 1$ \\
3.  Clamp excess used digits (\textit{mp\_clamp}) \\
\hline
\end{tabular}
\end{center}
\caption{Algorithm mp\_set\_int}
\end{figure}

\textbf{Algorithm mp\_set\_int.}
The algorithm performs eight iterations of a simple loop where in each iteration four bits from the source are added to the 
mp\_int.  Step 2.1 will multiply the current result by sixteen making room for four more bits in the less significant positions.  In step 2.2 the
next four bits from the source are extracted and are added to the mp\_int. The \textbf{used} digit count is 
incremented to reflect the addition.  The \textbf{used} digit counter is incremented since if any of the leading digits were zero the mp\_int would have
zero digits used and the newly added four bits would be ignored.

Excess zero digits are trimmed in steps 2.1 and 3 by using higher level algorithms mp\_mul2d and mp\_clamp.

\vspace{+3mm}\begin{small}
\hspace{-5.1mm}{\bf File}: bn\_mp\_set\_int.c
\vspace{-3mm}
\begin{alltt}
\end{alltt}
\end{small}

This function sets four bits of the number at a time to handle all practical \textbf{DIGIT\_BIT} sizes.  The weird
addition on line 39 ensures that the newly added in bits are added to the number of digits.  While it may not 
seem obvious as to why the digit counter does not grow exceedingly large it is because of the shift on line 28 
as well as the  call to mp\_clamp() on line 41.  Both functions will clamp excess leading digits which keeps 
the number of used digits low.

\section{Comparisons}
\subsection{Unsigned Comparisions}
Comparing a multiple precision integer is performed with the exact same algorithm used to compare two decimal numbers.  For example,
to compare $1,234$ to $1,264$ the digits are extracted by their positions.  That is we compare $1 \cdot 10^3 + 2 \cdot 10^2 + 3 \cdot 10^1 + 4 \cdot 10^0$
to $1 \cdot 10^3 + 2 \cdot 10^2 + 6 \cdot 10^1 + 4 \cdot 10^0$ by comparing single digits at a time starting with the highest magnitude 
positions.  If any leading digit of one integer is greater than a digit in the same position of another integer then obviously it must be greater.  

The first comparision routine that will be developed is the unsigned magnitude compare which will perform a comparison based on the digits of two
mp\_int variables alone.  It will ignore the sign of the two inputs.  Such a function is useful when an absolute comparison is required or if the 
signs are known to agree in advance.

To facilitate working with the results of the comparison functions three constants are required.  

\begin{figure}[here]
\begin{center}
\begin{tabular}{|r|l|}
\hline \textbf{Constant} & \textbf{Meaning} \\
\hline \textbf{MP\_GT} & Greater Than \\
\hline \textbf{MP\_EQ} & Equal To \\
\hline \textbf{MP\_LT} & Less Than \\
\hline
\end{tabular}
\end{center}
\caption{Comparison Return Codes}
\end{figure}

\begin{figure}[here]
\begin{center}
\begin{tabular}{l}
\hline Algorithm \textbf{mp\_cmp\_mag}. \\
\textbf{Input}.   Two mp\_ints $a$ and $b$.  \\
\textbf{Output}.  Unsigned comparison results ($a$ to the left of $b$). \\
\hline \\
1.  If $a.used > b.used$ then return(\textit{MP\_GT}) \\
2.  If $a.used < b.used$ then return(\textit{MP\_LT}) \\
3.  for n from $a.used - 1$ to 0 do \\
\hspace{+3mm}3.1  if $a_n > b_n$ then return(\textit{MP\_GT}) \\
\hspace{+3mm}3.2  if $a_n < b_n$ then return(\textit{MP\_LT}) \\
4.  Return(\textit{MP\_EQ}) \\
\hline
\end{tabular}
\end{center}
\caption{Algorithm mp\_cmp\_mag}
\end{figure}

\textbf{Algorithm mp\_cmp\_mag.}
By saying ``$a$ to the left of $b$'' it is meant that the comparison is with respect to $a$, that is if $a$ is greater than $b$ it will return
\textbf{MP\_GT} and similar with respect to when $a = b$ and $a < b$.  The first two steps compare the number of digits used in both $a$ and $b$.  
Obviously if the digit counts differ there would be an imaginary zero digit in the smaller number where the leading digit of the larger number is.  
If both have the same number of digits than the actual digits themselves must be compared starting at the leading digit.  

By step three both inputs must have the same number of digits so its safe to start from either $a.used - 1$ or $b.used - 1$ and count down to
the zero'th digit.  If after all of the digits have been compared, no difference is found, the algorithm returns \textbf{MP\_EQ}.

\vspace{+3mm}\begin{small}
\hspace{-5.1mm}{\bf File}: bn\_mp\_cmp\_mag.c
\vspace{-3mm}
\begin{alltt}
\end{alltt}
\end{small}

The two if statements (lines 25 and 29) compare the number of digits in the two inputs.  These two are 
performed before all of the digits are compared since it is a very cheap test to perform and can potentially save 
considerable time.  The implementation given is also not valid without those two statements.  $b.alloc$ may be 
smaller than $a.used$, meaning that undefined values will be read from $b$ past the end of the array of digits.



\subsection{Signed Comparisons}
Comparing with sign considerations is also fairly critical in several routines (\textit{division for example}).  Based on an unsigned magnitude 
comparison a trivial signed comparison algorithm can be written.

\begin{figure}[here]
\begin{center}
\begin{tabular}{l}
\hline Algorithm \textbf{mp\_cmp}. \\
\textbf{Input}.   Two mp\_ints $a$ and $b$ \\
\textbf{Output}.  Signed Comparison Results ($a$ to the left of $b$) \\
\hline \\
1.  if $a.sign = MP\_NEG$ and $b.sign = MP\_ZPOS$ then return(\textit{MP\_LT}) \\
2.  if $a.sign = MP\_ZPOS$ and $b.sign = MP\_NEG$ then return(\textit{MP\_GT}) \\
3.  if $a.sign = MP\_NEG$ then \\
\hspace{+3mm}3.1  Return the unsigned comparison of $b$ and $a$ (\textit{mp\_cmp\_mag}) \\
4   Otherwise \\
\hspace{+3mm}4.1  Return the unsigned comparison of $a$ and $b$ \\
\hline
\end{tabular}
\end{center}
\caption{Algorithm mp\_cmp}
\end{figure}

\textbf{Algorithm mp\_cmp.}
The first two steps compare the signs of the two inputs.  If the signs do not agree then it can return right away with the appropriate 
comparison code.  When the signs are equal the digits of the inputs must be compared to determine the correct result.  In step 
three the unsigned comparision flips the order of the arguments since they are both negative.  For instance, if $-a > -b$ then 
$\vert a \vert < \vert b \vert$.  Step number four will compare the two when they are both positive.

\vspace{+3mm}\begin{small}
\hspace{-5.1mm}{\bf File}: bn\_mp\_cmp.c
\vspace{-3mm}
\begin{alltt}
\end{alltt}
\end{small}

The two if statements (lines 23 and 24) perform the initial sign comparison.  If the signs are not the equal then which ever
has the positive sign is larger.   The inputs are compared (line 32) based on magnitudes.  If the signs were both 
negative then the unsigned comparison is performed in the opposite direction (line 34).  Otherwise, the signs are assumed to 
be both positive and a forward direction unsigned comparison is performed.

\section*{Exercises}
\begin{tabular}{cl}
$\left [ 2 \right ]$ & Modify algorithm mp\_set\_int to accept as input a variable length array of bits. \\
                     & \\
$\left [ 3 \right ]$ & Give the probability that algorithm mp\_cmp\_mag will have to compare $k$ digits  \\
                     & of two random digits (of equal magnitude) before a difference is found. \\
                     & \\
$\left [ 1 \right ]$ & Suggest a simple method to speed up the implementation of mp\_cmp\_mag based  \\
                     & on the observations made in the previous problem. \\
                     &
\end{tabular}

\chapter{Basic Arithmetic}
\section{Introduction}
At this point algorithms for initialization, clearing, zeroing, copying, comparing and setting small constants have been 
established.  The next logical set of algorithms to develop are addition, subtraction and digit shifting algorithms.  These 
algorithms make use of the lower level algorithms and are the cruicial building block for the multiplication algorithms.  It is very important 
that these algorithms are highly optimized.  On their own they are simple $O(n)$ algorithms but they can be called from higher level algorithms 
which easily places them at $O(n^2)$ or even $O(n^3)$ work levels.  

All of the algorithms within this chapter make use of the logical bit shift operations denoted by $<<$ and $>>$ for left and right 
logical shifts respectively.  A logical shift is analogous to sliding the decimal point of radix-10 representations.  For example, the real 
number $0.9345$ is equivalent to $93.45\%$ which is found by sliding the the decimal two places to the right (\textit{multiplying by $\beta^2 = 10^2$}).  
Algebraically a binary logical shift is equivalent to a division or multiplication by a power of two.  
For example, $a << k = a \cdot 2^k$ while $a >> k = \lfloor a/2^k \rfloor$.

One significant difference between a logical shift and the way decimals are shifted is that digits below the zero'th position are removed
from the number.  For example, consider $1101_2 >> 1$ using decimal notation this would produce $110.1_2$.  However, with a logical shift the 
result is $110_2$.  

\section{Addition and Subtraction}
In common twos complement fixed precision arithmetic negative numbers are easily represented by subtraction from the modulus.  For example, with 32-bit integers
$a - b\mbox{ (mod }2^{32}\mbox{)}$ is the same as $a + (2^{32} - b) \mbox{ (mod }2^{32}\mbox{)}$  since $2^{32} \equiv 0 \mbox{ (mod }2^{32}\mbox{)}$.  
As a result subtraction can be performed with a trivial series of logical operations and an addition.

However, in multiple precision arithmetic negative numbers are not represented in the same way.  Instead a sign flag is used to keep track of the
sign of the integer.  As a result signed addition and subtraction are actually implemented as conditional usage of lower level addition or 
subtraction algorithms with the sign fixed up appropriately.

The lower level algorithms will add or subtract integers without regard to the sign flag.  That is they will add or subtract the magnitude of
the integers respectively.

\subsection{Low Level Addition}
An unsigned addition of multiple precision integers is performed with the same long-hand algorithm used to add decimal numbers.  That is to add the 
trailing digits first and propagate the resulting carry upwards.  Since this is a lower level algorithm the name will have a ``s\_'' prefix.  
Historically that convention stems from the MPI library where ``s\_'' stood for static functions that were hidden from the developer entirely.

\newpage
\begin{figure}[!here]
\begin{center}
\begin{small}
\begin{tabular}{l}
\hline Algorithm \textbf{s\_mp\_add}. \\
\textbf{Input}.   Two mp\_ints $a$ and $b$ \\
\textbf{Output}.  The unsigned addition $c = \vert a \vert + \vert b \vert$. \\
\hline \\
1.  if $a.used > b.used$ then \\
\hspace{+3mm}1.1  $min \leftarrow b.used$ \\
\hspace{+3mm}1.2  $max \leftarrow a.used$ \\
\hspace{+3mm}1.3  $x   \leftarrow a$ \\
2.  else  \\
\hspace{+3mm}2.1  $min \leftarrow a.used$ \\
\hspace{+3mm}2.2  $max \leftarrow b.used$ \\
\hspace{+3mm}2.3  $x   \leftarrow b$ \\
3.  If $c.alloc < max + 1$ then grow $c$ to hold at least $max + 1$ digits (\textit{mp\_grow}) \\
4.  $oldused \leftarrow c.used$ \\
5.  $c.used \leftarrow max + 1$ \\
6.  $u \leftarrow 0$ \\
7.  for $n$ from $0$ to $min - 1$ do \\
\hspace{+3mm}7.1  $c_n \leftarrow a_n + b_n + u$ \\
\hspace{+3mm}7.2  $u \leftarrow c_n >> lg(\beta)$ \\
\hspace{+3mm}7.3  $c_n \leftarrow c_n \mbox{ (mod }\beta\mbox{)}$ \\
8.  if $min \ne max$ then do \\
\hspace{+3mm}8.1  for $n$ from $min$ to $max - 1$ do \\
\hspace{+6mm}8.1.1  $c_n \leftarrow x_n + u$ \\
\hspace{+6mm}8.1.2  $u \leftarrow c_n >> lg(\beta)$ \\
\hspace{+6mm}8.1.3  $c_n \leftarrow c_n \mbox{ (mod }\beta\mbox{)}$ \\
9.  $c_{max} \leftarrow u$ \\
10.  if $olduse > max$ then \\
\hspace{+3mm}10.1  for $n$ from $max + 1$ to $oldused - 1$ do \\
\hspace{+6mm}10.1.1  $c_n \leftarrow 0$ \\
11.  Clamp excess digits in $c$.  (\textit{mp\_clamp}) \\
12.  Return(\textit{MP\_OKAY}) \\
\hline
\end{tabular}
\end{small}
\end{center}
\caption{Algorithm s\_mp\_add}
\end{figure}

\textbf{Algorithm s\_mp\_add.}
This algorithm is loosely based on algorithm 14.7 of HAC \cite[pp. 594]{HAC} but has been extended to allow the inputs to have different magnitudes.  
Coincidentally the description of algorithm A in Knuth \cite[pp. 266]{TAOCPV2} shares the same deficiency as the algorithm from \cite{HAC}.  Even the 
MIX pseudo  machine code presented by Knuth \cite[pp. 266-267]{TAOCPV2} is incapable of handling inputs which are of different magnitudes.

The first thing that has to be accomplished is to sort out which of the two inputs is the largest.  The addition logic
will simply add all of the smallest input to the largest input and store that first part of the result in the
destination.  Then it will apply a simpler addition loop to excess digits of the larger input.

The first two steps will handle sorting the inputs such that $min$ and $max$ hold the digit counts of the two 
inputs.  The variable $x$ will be an mp\_int alias for the largest input or the second input $b$ if they have the
same number of digits.  After the inputs are sorted the destination $c$ is grown as required to accomodate the sum 
of the two inputs.  The original \textbf{used} count of $c$ is copied and set to the new used count.  

At this point the first addition loop will go through as many digit positions that both inputs have.  The carry
variable $\mu$ is set to zero outside the loop.  Inside the loop an ``addition'' step requires three statements to produce
one digit of the summand.  First
two digits from $a$ and $b$ are added together along with the carry $\mu$.  The carry of this step is extracted and stored
in $\mu$ and finally the digit of the result $c_n$ is truncated within the range $0 \le c_n < \beta$.

Now all of the digit positions that both inputs have in common have been exhausted.  If $min \ne max$ then $x$ is an alias
for one of the inputs that has more digits.  A simplified addition loop is then used to essentially copy the remaining digits
and the carry to the destination.

The final carry is stored in $c_{max}$ and digits above $max$ upto $oldused$ are zeroed which completes the addition.


\vspace{+3mm}\begin{small}
\hspace{-5.1mm}{\bf File}: bn\_s\_mp\_add.c
\vspace{-3mm}
\begin{alltt}
\end{alltt}
\end{small}

We first sort (lines 28 to 36) the inputs based on magnitude and determine the $min$ and $max$ variables.
Note that $x$ is a pointer to an mp\_int assigned to the largest input, in effect it is a local alias.  Next we
grow the destination (38 to 42) ensure that it can accomodate the result of the addition. 

Similar to the implementation of mp\_copy this function uses the braced code and local aliases coding style.  The three aliases that are on 
lines 56, 59 and 62 represent the two inputs and destination variables respectively.  These aliases are used to ensure the
compiler does not have to dereference $a$, $b$ or $c$ (respectively) to access the digits of the respective mp\_int.

The initial carry $u$ will be cleared (line 65), note that $u$ is of type mp\_digit which ensures type 
compatibility within the implementation.  The initial addition (line 66 to 75) adds digits from
both inputs until the smallest input runs out of digits.  Similarly the conditional addition loop
(line 81 to 90) adds the remaining digits from the larger of the two inputs.  The addition is finished 
with the final carry being stored in $tmpc$ (line 94).  Note the ``++'' operator within the same expression.
After line 94, $tmpc$ will point to the $c.used$'th digit of the mp\_int $c$.  This is useful
for the next loop (line 97 to 99) which set any old upper digits to zero.

\subsection{Low Level Subtraction}
The low level unsigned subtraction algorithm is very similar to the low level unsigned addition algorithm.  The principle difference is that the
unsigned subtraction algorithm requires the result to be positive.  That is when computing $a - b$ the condition $\vert a \vert \ge \vert b\vert$ must 
be met for this algorithm to function properly.  Keep in mind this low level algorithm is not meant to be used in higher level algorithms directly.  
This algorithm as will be shown can be used to create functional signed addition and subtraction algorithms.


For this algorithm a new variable is required to make the description simpler.  Recall from section 1.3.1 that a mp\_digit must be able to represent
the range $0 \le x < 2\beta$ for the algorithms to work correctly.  However, it is allowable that a mp\_digit represent a larger range of values.  For 
this algorithm we will assume that the variable $\gamma$ represents the number of bits available in a 
mp\_digit (\textit{this implies $2^{\gamma} > \beta$}).  

For example, the default for LibTomMath is to use a ``unsigned long'' for the mp\_digit ``type'' while $\beta = 2^{28}$.  In ISO C an ``unsigned long''
data type must be able to represent $0 \le x < 2^{32}$ meaning that in this case $\gamma \ge 32$.

\newpage\begin{figure}[!here]
\begin{center}
\begin{small}
\begin{tabular}{l}
\hline Algorithm \textbf{s\_mp\_sub}. \\
\textbf{Input}.   Two mp\_ints $a$ and $b$ ($\vert a \vert \ge \vert b \vert$) \\
\textbf{Output}.  The unsigned subtraction $c = \vert a \vert - \vert b \vert$. \\
\hline \\
1.  $min \leftarrow b.used$ \\
2.  $max \leftarrow a.used$ \\
3.  If $c.alloc < max$ then grow $c$ to hold at least $max$ digits.  (\textit{mp\_grow}) \\
4.  $oldused \leftarrow c.used$ \\ 
5.  $c.used \leftarrow max$ \\
6.  $u \leftarrow 0$ \\
7.  for $n$ from $0$ to $min - 1$ do \\
\hspace{3mm}7.1  $c_n \leftarrow a_n - b_n - u$ \\
\hspace{3mm}7.2  $u   \leftarrow c_n >> (\gamma - 1)$ \\
\hspace{3mm}7.3  $c_n \leftarrow c_n \mbox{ (mod }\beta\mbox{)}$ \\
8.  if $min < max$ then do \\
\hspace{3mm}8.1  for $n$ from $min$ to $max - 1$ do \\
\hspace{6mm}8.1.1  $c_n \leftarrow a_n - u$ \\
\hspace{6mm}8.1.2  $u   \leftarrow c_n >> (\gamma - 1)$ \\
\hspace{6mm}8.1.3  $c_n \leftarrow c_n \mbox{ (mod }\beta\mbox{)}$ \\
9. if $oldused > max$ then do \\
\hspace{3mm}9.1  for $n$ from $max$ to $oldused - 1$ do \\
\hspace{6mm}9.1.1  $c_n \leftarrow 0$ \\
10. Clamp excess digits of $c$.  (\textit{mp\_clamp}). \\
11. Return(\textit{MP\_OKAY}). \\
\hline
\end{tabular}
\end{small}
\end{center}
\caption{Algorithm s\_mp\_sub}
\end{figure}

\textbf{Algorithm s\_mp\_sub.}
This algorithm performs the unsigned subtraction of two mp\_int variables under the restriction that the result must be positive.  That is when
passing variables $a$ and $b$ the condition that $\vert a \vert \ge \vert b \vert$ must be met for the algorithm to function correctly.  This
algorithm is loosely based on algorithm 14.9 \cite[pp. 595]{HAC} and is similar to algorithm S in \cite[pp. 267]{TAOCPV2} as well.  As was the case
of the algorithm s\_mp\_add both other references lack discussion concerning various practical details such as when the inputs differ in magnitude.

The initial sorting of the inputs is trivial in this algorithm since $a$ is guaranteed to have at least the same magnitude of $b$.  Steps 1 and 2 
set the $min$ and $max$ variables.  Unlike the addition routine there is guaranteed to be no carry which means that the final result can be at 
most $max$ digits in length as opposed to $max + 1$.  Similar to the addition algorithm the \textbf{used} count of $c$ is copied locally and 
set to the maximal count for the operation.

The subtraction loop that begins on step seven is essentially the same as the addition loop of algorithm s\_mp\_add except single precision 
subtraction is used instead.  Note the use of the $\gamma$ variable to extract the carry (\textit{also known as the borrow}) within the subtraction 
loops.  Under the assumption that two's complement single precision arithmetic is used this will successfully extract the desired carry.  

For example, consider subtracting $0101_2$ from $0100_2$ where $\gamma = 4$ and $\beta = 2$.  The least significant bit will force a carry upwards to 
the third bit which will be set to zero after the borrow.  After the very first bit has been subtracted $4 - 1 \equiv 0011_2$ will remain,  When the 
third bit of $0101_2$ is subtracted from the result it will cause another carry.  In this case though the carry will be forced to propagate all the 
way to the most significant bit.  

Recall that $\beta < 2^{\gamma}$.  This means that if a carry does occur just before the $lg(\beta)$'th bit it will propagate all the way to the most 
significant bit.  Thus, the high order bits of the mp\_digit that are not part of the actual digit will either be all zero, or all one. All that
is needed is a single zero or one bit for the carry.  Therefore a single logical shift right by $\gamma - 1$ positions is sufficient to extract the 
carry.  This method of carry extraction may seem awkward but the reason for it becomes apparent when the implementation is discussed.  

If $b$ has a smaller magnitude than $a$ then step 9 will force the carry and copy operation to propagate through the larger input $a$ into $c$.  Step
10 will ensure that any leading digits of $c$ above the $max$'th position are zeroed.

\vspace{+3mm}\begin{small}
\hspace{-5.1mm}{\bf File}: bn\_s\_mp\_sub.c
\vspace{-3mm}
\begin{alltt}
\end{alltt}
\end{small}

Like low level addition we ``sort'' the inputs.  Except in this case the sorting is hardcoded 
(lines 25 and 26).  In reality the $min$ and $max$ variables are only aliases and are only 
used to make the source code easier to read.  Again the pointer alias optimization is used 
within this algorithm.  The aliases $tmpa$, $tmpb$ and $tmpc$ are initialized
(lines 42, 43 and 44) for $a$, $b$ and $c$ respectively.

The first subtraction loop (lines 47 through 61) subtract digits from both inputs until the smaller of
the two inputs has been exhausted.  As remarked earlier there is an implementation reason for using the ``awkward'' 
method of extracting the carry (line 57).  The traditional method for extracting the carry would be to shift 
by $lg(\beta)$ positions and logically AND the least significant bit.  The AND operation is required because all of 
the bits above the $\lg(\beta)$'th bit will be set to one after a carry occurs from subtraction.  This carry 
extraction requires two relatively cheap operations to extract the carry.  The other method is to simply shift the 
most significant bit to the least significant bit thus extracting the carry with a single cheap operation.  This 
optimization only works on twos compliment machines which is a safe assumption to make.

If $a$ has a larger magnitude than $b$ an additional loop (lines 64 through 73) is required to propagate 
the carry through $a$ and copy the result to $c$.  

\subsection{High Level Addition}
Now that both lower level addition and subtraction algorithms have been established an effective high level signed addition algorithm can be
established.  This high level addition algorithm will be what other algorithms and developers will use to perform addition of mp\_int data 
types.  

Recall from section 5.2 that an mp\_int represents an integer with an unsigned mantissa (\textit{the array of digits}) and a \textbf{sign} 
flag.  A high level addition is actually performed as a series of eight separate cases which can be optimized down to three unique cases.

\begin{figure}[!here]
\begin{center}
\begin{tabular}{l}
\hline Algorithm \textbf{mp\_add}. \\
\textbf{Input}.   Two mp\_ints $a$ and $b$  \\
\textbf{Output}.  The signed addition $c = a + b$. \\
\hline \\
1.  if $a.sign = b.sign$ then do \\
\hspace{3mm}1.1  $c.sign \leftarrow a.sign$  \\
\hspace{3mm}1.2  $c \leftarrow \vert a \vert + \vert b \vert$ (\textit{s\_mp\_add})\\
2.  else do \\
\hspace{3mm}2.1  if $\vert a \vert < \vert b \vert$ then do (\textit{mp\_cmp\_mag})  \\
\hspace{6mm}2.1.1  $c.sign \leftarrow b.sign$ \\
\hspace{6mm}2.1.2  $c \leftarrow \vert b \vert - \vert a \vert$ (\textit{s\_mp\_sub}) \\
\hspace{3mm}2.2  else do \\
\hspace{6mm}2.2.1  $c.sign \leftarrow a.sign$ \\
\hspace{6mm}2.2.2  $c \leftarrow \vert a \vert - \vert b \vert$ \\
3.  Return(\textit{MP\_OKAY}). \\
\hline
\end{tabular}
\end{center}
\caption{Algorithm mp\_add}
\end{figure}

\textbf{Algorithm mp\_add.}
This algorithm performs the signed addition of two mp\_int variables.  There is no reference algorithm to draw upon from 
either \cite{TAOCPV2} or \cite{HAC} since they both only provide unsigned operations.  The algorithm is fairly 
straightforward but restricted since subtraction can only produce positive results.

\begin{figure}[here]
\begin{small}
\begin{center}
\begin{tabular}{|c|c|c|c|c|}
\hline \textbf{Sign of $a$} & \textbf{Sign of $b$} & \textbf{$\vert a \vert > \vert b \vert $} & \textbf{Unsigned Operation} & \textbf{Result Sign Flag} \\
\hline $+$ & $+$ & Yes & $c = a + b$ & $a.sign$ \\
\hline $+$ & $+$ & No  & $c = a + b$ & $a.sign$ \\
\hline $-$ & $-$ & Yes & $c = a + b$ & $a.sign$ \\
\hline $-$ & $-$ & No  & $c = a + b$ & $a.sign$ \\
\hline &&&&\\

\hline $+$ & $-$ & No  & $c = b - a$ & $b.sign$ \\
\hline $-$ & $+$ & No  & $c = b - a$ & $b.sign$ \\

\hline &&&&\\

\hline $+$ & $-$ & Yes & $c = a - b$ & $a.sign$ \\
\hline $-$ & $+$ & Yes & $c = a - b$ & $a.sign$ \\

\hline
\end{tabular}
\end{center}
\end{small}
\caption{Addition Guide Chart}
\label{fig:AddChart}
\end{figure}

Figure~\ref{fig:AddChart} lists all of the eight possible input combinations and is sorted to show that only three 
specific cases need to be handled.  The return code of the unsigned operations at step 1.2, 2.1.2 and 2.2.2 are 
forwarded to step three to check for errors.  This simplifies the description of the algorithm considerably and best 
follows how the implementation actually was achieved.

Also note how the \textbf{sign} is set before the unsigned addition or subtraction is performed.  Recall from the descriptions of algorithms
s\_mp\_add and s\_mp\_sub that the mp\_clamp function is used at the end to trim excess digits.  The mp\_clamp algorithm will set the \textbf{sign}
to \textbf{MP\_ZPOS} when the \textbf{used} digit count reaches zero.

For example, consider performing $-a + a$ with algorithm mp\_add.  By the description of the algorithm the sign is set to \textbf{MP\_NEG} which would
produce a result of $-0$.  However, since the sign is set first then the unsigned addition is performed the subsequent usage of algorithm mp\_clamp 
within algorithm s\_mp\_add will force $-0$ to become $0$.  

\vspace{+3mm}\begin{small}
\hspace{-5.1mm}{\bf File}: bn\_mp\_add.c
\vspace{-3mm}
\begin{alltt}
\end{alltt}
\end{small}

The source code follows the algorithm fairly closely.  The most notable new source code addition is the usage of the $res$ integer variable which
is used to pass result of the unsigned operations forward.  Unlike in the algorithm, the variable $res$ is merely returned as is without
explicitly checking it and returning the constant \textbf{MP\_OKAY}.  The observation is this algorithm will succeed or fail only if the lower
level functions do so.  Returning their return code is sufficient.

\subsection{High Level Subtraction}
The high level signed subtraction algorithm is essentially the same as the high level signed addition algorithm.  

\newpage\begin{figure}[!here]
\begin{center}
\begin{tabular}{l}
\hline Algorithm \textbf{mp\_sub}. \\
\textbf{Input}.   Two mp\_ints $a$ and $b$  \\
\textbf{Output}.  The signed subtraction $c = a - b$. \\
\hline \\
1.  if $a.sign \ne b.sign$ then do \\
\hspace{3mm}1.1  $c.sign \leftarrow a.sign$ \\
\hspace{3mm}1.2  $c \leftarrow \vert a \vert + \vert b \vert$ (\textit{s\_mp\_add}) \\
2.  else do \\
\hspace{3mm}2.1  if $\vert a \vert \ge \vert b \vert$ then do (\textit{mp\_cmp\_mag}) \\
\hspace{6mm}2.1.1  $c.sign \leftarrow a.sign$ \\
\hspace{6mm}2.1.2  $c \leftarrow \vert a \vert  - \vert b \vert$ (\textit{s\_mp\_sub}) \\
\hspace{3mm}2.2  else do \\
\hspace{6mm}2.2.1  $c.sign \leftarrow  \left \lbrace \begin{array}{ll}
                              MP\_ZPOS &  \mbox{if }a.sign = MP\_NEG \\
                              MP\_NEG  &  \mbox{otherwise} \\
                              \end{array} \right .$ \\
\hspace{6mm}2.2.2  $c \leftarrow \vert b \vert  - \vert a \vert$ \\
3.  Return(\textit{MP\_OKAY}). \\
\hline
\end{tabular}
\end{center}
\caption{Algorithm mp\_sub}
\end{figure}

\textbf{Algorithm mp\_sub.}
This algorithm performs the signed subtraction of two inputs.  Similar to algorithm mp\_add there is no reference in either \cite{TAOCPV2} or 
\cite{HAC}.  Also this algorithm is restricted by algorithm s\_mp\_sub.  Chart \ref{fig:SubChart} lists the eight possible inputs and
the operations required.

\begin{figure}[!here]
\begin{small}
\begin{center}
\begin{tabular}{|c|c|c|c|c|}
\hline \textbf{Sign of $a$} & \textbf{Sign of $b$} & \textbf{$\vert a \vert \ge \vert b \vert $} & \textbf{Unsigned Operation} & \textbf{Result Sign Flag} \\
\hline $+$ & $-$ & Yes & $c = a + b$ & $a.sign$ \\
\hline $+$ & $-$ & No  & $c = a + b$ & $a.sign$ \\
\hline $-$ & $+$ & Yes & $c = a + b$ & $a.sign$ \\
\hline $-$ & $+$ & No  & $c = a + b$ & $a.sign$ \\
\hline &&&& \\
\hline $+$ & $+$ & Yes & $c = a - b$ & $a.sign$ \\
\hline $-$ & $-$ & Yes & $c = a - b$ & $a.sign$ \\
\hline &&&& \\
\hline $+$ & $+$ & No  & $c = b - a$ & $\mbox{opposite of }a.sign$ \\
\hline $-$ & $-$ & No  & $c = b - a$ & $\mbox{opposite of }a.sign$ \\
\hline
\end{tabular}
\end{center}
\end{small}
\caption{Subtraction Guide Chart}
\label{fig:SubChart}
\end{figure}

Similar to the case of algorithm mp\_add the \textbf{sign} is set first before the unsigned addition or subtraction.  That is to prevent the 
algorithm from producing $-a - -a = -0$ as a result.  

\vspace{+3mm}\begin{small}
\hspace{-5.1mm}{\bf File}: bn\_mp\_sub.c
\vspace{-3mm}
\begin{alltt}
\end{alltt}
\end{small}

Much like the implementation of algorithm mp\_add the variable $res$ is used to catch the return code of the unsigned addition or subtraction operations
and forward it to the end of the function.  On line 39 the ``not equal to'' \textbf{MP\_LT} expression is used to emulate a 
``greater than or equal to'' comparison.  

\section{Bit and Digit Shifting}
It is quite common to think of a multiple precision integer as a polynomial in $x$, that is $y = f(\beta)$ where $f(x) = \sum_{i=0}^{n-1} a_i x^i$.  
This notation arises within discussion of Montgomery and Diminished Radix Reduction as well as Karatsuba multiplication and squaring.  

In order to facilitate operations on polynomials in $x$ as above a series of simple ``digit'' algorithms have to be established.  That is to shift
the digits left or right as well to shift individual bits of the digits left and right.  It is important to note that not all ``shift'' operations
are on radix-$\beta$ digits.  

\subsection{Multiplication by Two}

In a binary system where the radix is a power of two multiplication by two not only arises often in other algorithms it is a fairly efficient 
operation to perform.  A single precision logical shift left is sufficient to multiply a single digit by two.  

\newpage\begin{figure}[!here]
\begin{small}
\begin{center}
\begin{tabular}{l}
\hline Algorithm \textbf{mp\_mul\_2}. \\
\textbf{Input}.   One mp\_int $a$ \\
\textbf{Output}.  $b = 2a$. \\
\hline \\
1.  If $b.alloc < a.used + 1$ then grow $b$ to hold $a.used + 1$ digits.  (\textit{mp\_grow}) \\
2.  $oldused \leftarrow b.used$ \\
3.  $b.used \leftarrow a.used$ \\
4.  $r \leftarrow 0$ \\
5.  for $n$ from 0 to $a.used - 1$ do \\
\hspace{3mm}5.1  $rr \leftarrow a_n >> (lg(\beta) - 1)$ \\
\hspace{3mm}5.2  $b_n \leftarrow (a_n << 1) + r \mbox{ (mod }\beta\mbox{)}$ \\
\hspace{3mm}5.3  $r \leftarrow rr$ \\
6.  If $r \ne 0$ then do \\
\hspace{3mm}6.1  $b_{n + 1} \leftarrow r$ \\
\hspace{3mm}6.2  $b.used \leftarrow b.used + 1$ \\
7.  If $b.used < oldused - 1$ then do \\
\hspace{3mm}7.1  for $n$ from $b.used$ to $oldused - 1$ do \\
\hspace{6mm}7.1.1  $b_n \leftarrow 0$ \\
8.  $b.sign \leftarrow a.sign$ \\
9.  Return(\textit{MP\_OKAY}).\\
\hline
\end{tabular}
\end{center}
\end{small}
\caption{Algorithm mp\_mul\_2}
\end{figure}

\textbf{Algorithm mp\_mul\_2.}
This algorithm will quickly multiply a mp\_int by two provided $\beta$ is a power of two.  Neither \cite{TAOCPV2} nor \cite{HAC} describe such 
an algorithm despite the fact it arises often in other algorithms.  The algorithm is setup much like the lower level algorithm s\_mp\_add since 
it is for all intents and purposes equivalent to the operation $b = \vert a \vert + \vert a \vert$.  

Step 1 and 2 grow the input as required to accomodate the maximum number of \textbf{used} digits in the result.  The initial \textbf{used} count
is set to $a.used$ at step 4.  Only if there is a final carry will the \textbf{used} count require adjustment.

Step 6 is an optimization implementation of the addition loop for this specific case.  That is since the two values being added together 
are the same there is no need to perform two reads from the digits of $a$.  Step 6.1 performs a single precision shift on the current digit $a_n$ to
obtain what will be the carry for the next iteration.  Step 6.2 calculates the $n$'th digit of the result as single precision shift of $a_n$ plus
the previous carry.  Recall from section 4.1 that $a_n << 1$ is equivalent to $a_n \cdot 2$.  An iteration of the addition loop is finished with 
forwarding the carry to the next iteration.

Step 7 takes care of any final carry by setting the $a.used$'th digit of the result to the carry and augmenting the \textbf{used} count of $b$.  
Step 8 clears any leading digits of $b$ in case it originally had a larger magnitude than $a$.

\vspace{+3mm}\begin{small}
\hspace{-5.1mm}{\bf File}: bn\_mp\_mul\_2.c
\vspace{-3mm}
\begin{alltt}
\end{alltt}
\end{small}

This implementation is essentially an optimized implementation of s\_mp\_add for the case of doubling an input.  The only noteworthy difference
is the use of the logical shift operator on line 52 to perform a single precision doubling.  

\subsection{Division by Two}
A division by two can just as easily be accomplished with a logical shift right as multiplication by two can be with a logical shift left.

\newpage\begin{figure}[!here]
\begin{small}
\begin{center}
\begin{tabular}{l}
\hline Algorithm \textbf{mp\_div\_2}. \\
\textbf{Input}.   One mp\_int $a$ \\
\textbf{Output}.  $b = a/2$. \\
\hline \\
1.  If $b.alloc < a.used$ then grow $b$ to hold $a.used$ digits.  (\textit{mp\_grow}) \\
2.  If the reallocation failed return(\textit{MP\_MEM}). \\
3.  $oldused \leftarrow b.used$ \\
4.  $b.used \leftarrow a.used$ \\
5.  $r \leftarrow 0$ \\
6.  for $n$ from $b.used - 1$ to $0$ do \\
\hspace{3mm}6.1  $rr \leftarrow a_n \mbox{ (mod }2\mbox{)}$\\
\hspace{3mm}6.2  $b_n \leftarrow (a_n >> 1) + (r << (lg(\beta) - 1)) \mbox{ (mod }\beta\mbox{)}$ \\
\hspace{3mm}6.3  $r \leftarrow rr$ \\
7.  If $b.used < oldused - 1$ then do \\
\hspace{3mm}7.1  for $n$ from $b.used$ to $oldused - 1$ do \\
\hspace{6mm}7.1.1  $b_n \leftarrow 0$ \\
8.  $b.sign \leftarrow a.sign$ \\
9.  Clamp excess digits of $b$.  (\textit{mp\_clamp}) \\
10.  Return(\textit{MP\_OKAY}).\\
\hline
\end{tabular}
\end{center}
\end{small}
\caption{Algorithm mp\_div\_2}
\end{figure}

\textbf{Algorithm mp\_div\_2.}
This algorithm will divide an mp\_int by two using logical shifts to the right.  Like mp\_mul\_2 it uses a modified low level addition
core as the basis of the algorithm.  Unlike mp\_mul\_2 the shift operations work from the leading digit to the trailing digit.  The algorithm
could be written to work from the trailing digit to the leading digit however, it would have to stop one short of $a.used - 1$ digits to prevent
reading past the end of the array of digits.

Essentially the loop at step 6 is similar to that of mp\_mul\_2 except the logical shifts go in the opposite direction and the carry is at the 
least significant bit not the most significant bit.  

\vspace{+3mm}\begin{small}
\hspace{-5.1mm}{\bf File}: bn\_mp\_div\_2.c
\vspace{-3mm}
\begin{alltt}
\end{alltt}
\end{small}

\section{Polynomial Basis Operations}
Recall from section 4.3 that any integer can be represented as a polynomial in $x$ as $y = f(\beta)$.  Such a representation is also known as
the polynomial basis \cite[pp. 48]{ROSE}. Given such a notation a multiplication or division by $x$ amounts to shifting whole digits a single 
place.  The need for such operations arises in several other higher level algorithms such as Barrett and Montgomery reduction, integer
division and Karatsuba multiplication.  

Converting from an array of digits to polynomial basis is very simple.  Consider the integer $y \equiv (a_2, a_1, a_0)_{\beta}$ and recall that
$y = \sum_{i=0}^{2} a_i \beta^i$.  Simply replace $\beta$ with $x$ and the expression is in polynomial basis.  For example, $f(x) = 8x + 9$ is the
polynomial basis representation for $89$ using radix ten.  That is, $f(10) = 8(10) + 9 = 89$.  

\subsection{Multiplication by $x$}

Given a polynomial in $x$ such as $f(x) = a_n x^n + a_{n-1} x^{n-1} + ... + a_0$ multiplying by $x$ amounts to shifting the coefficients up one 
degree.  In this case $f(x) \cdot x = a_n x^{n+1} + a_{n-1} x^n + ... + a_0 x$.  From a scalar basis point of view multiplying by $x$ is equivalent to
multiplying by the integer $\beta$.  

\newpage\begin{figure}[!here]
\begin{small}
\begin{center}
\begin{tabular}{l}
\hline Algorithm \textbf{mp\_lshd}. \\
\textbf{Input}.   One mp\_int $a$ and an integer $b$ \\
\textbf{Output}.  $a \leftarrow a \cdot \beta^b$ (equivalent to multiplication by $x^b$). \\
\hline \\
1.  If $b \le 0$ then return(\textit{MP\_OKAY}). \\
2.  If $a.alloc < a.used + b$ then grow $a$ to at least $a.used + b$ digits.  (\textit{mp\_grow}). \\
3.  If the reallocation failed return(\textit{MP\_MEM}). \\
4.  $a.used \leftarrow a.used + b$ \\
5.  $i \leftarrow a.used - 1$ \\
6.  $j \leftarrow a.used - 1 - b$ \\
7.  for $n$ from $a.used - 1$ to $b$ do \\
\hspace{3mm}7.1  $a_{i} \leftarrow a_{j}$ \\
\hspace{3mm}7.2  $i \leftarrow i - 1$ \\
\hspace{3mm}7.3  $j \leftarrow j - 1$ \\
8.  for $n$ from 0 to $b - 1$ do \\
\hspace{3mm}8.1  $a_n \leftarrow 0$ \\
9.  Return(\textit{MP\_OKAY}). \\
\hline
\end{tabular}
\end{center}
\end{small}
\caption{Algorithm mp\_lshd}
\end{figure}

\textbf{Algorithm mp\_lshd.}
This algorithm multiplies an mp\_int by the $b$'th power of $x$.  This is equivalent to multiplying by $\beta^b$.  The algorithm differs 
from the other algorithms presented so far as it performs the operation in place instead storing the result in a separate location.  The
motivation behind this change is due to the way this function is typically used.  Algorithms such as mp\_add store the result in an optionally
different third mp\_int because the original inputs are often still required.  Algorithm mp\_lshd (\textit{and similarly algorithm mp\_rshd}) is
typically used on values where the original value is no longer required.  The algorithm will return success immediately if 
$b \le 0$ since the rest of algorithm is only valid when $b > 0$.  

First the destination $a$ is grown as required to accomodate the result.  The counters $i$ and $j$ are used to form a \textit{sliding window} over
the digits of $a$ of length $b$.  The head of the sliding window is at $i$ (\textit{the leading digit}) and the tail at $j$ (\textit{the trailing digit}).  
The loop on step 7 copies the digit from the tail to the head.  In each iteration the window is moved down one digit.   The last loop on 
step 8 sets the lower $b$ digits to zero.

\newpage
\begin{center}
\begin{figure}[here]
\includegraphics{pics/sliding_window.ps}
\caption{Sliding Window Movement}
\label{pic:sliding_window}
\end{figure}
\end{center}

\vspace{+3mm}\begin{small}
\hspace{-5.1mm}{\bf File}: bn\_mp\_lshd.c
\vspace{-3mm}
\begin{alltt}
\end{alltt}
\end{small}

The if statement (line 24) ensures that the $b$ variable is greater than zero since we do not interpret negative
shift counts properly.  The \textbf{used} count is incremented by $b$ before the copy loop begins.  This elminates 
the need for an additional variable in the for loop.  The variable $top$ (line 42) is an alias
for the leading digit while $bottom$ (line 45) is an alias for the trailing edge.  The aliases form a 
window of exactly $b$ digits over the input.  

\subsection{Division by $x$}

Division by powers of $x$ is easily achieved by shifting the digits right and removing any that will end up to the right of the zero'th digit.  

\newpage\begin{figure}[!here]
\begin{small}
\begin{center}
\begin{tabular}{l}
\hline Algorithm \textbf{mp\_rshd}. \\
\textbf{Input}.   One mp\_int $a$ and an integer $b$ \\
\textbf{Output}.  $a \leftarrow a / \beta^b$ (Divide by $x^b$). \\
\hline \\
1.  If $b \le 0$ then return. \\
2.  If $a.used \le b$ then do \\
\hspace{3mm}2.1  Zero $a$.  (\textit{mp\_zero}). \\
\hspace{3mm}2.2  Return. \\
3.  $i \leftarrow 0$ \\
4.  $j \leftarrow b$ \\
5.  for $n$ from 0 to $a.used - b - 1$ do \\
\hspace{3mm}5.1  $a_i \leftarrow a_j$ \\
\hspace{3mm}5.2  $i \leftarrow i + 1$ \\
\hspace{3mm}5.3  $j \leftarrow j + 1$ \\
6.  for $n$ from $a.used - b$ to $a.used - 1$ do \\
\hspace{3mm}6.1  $a_n \leftarrow 0$ \\
7.  $a.used \leftarrow a.used - b$ \\
8.  Return. \\
\hline
\end{tabular}
\end{center}
\end{small}
\caption{Algorithm mp\_rshd}
\end{figure}

\textbf{Algorithm mp\_rshd.}
This algorithm divides the input in place by the $b$'th power of $x$.  It is analogous to dividing by a $\beta^b$ but much quicker since
it does not require single precision division.  This algorithm does not actually return an error code as it cannot fail.  

If the input $b$ is less than one the algorithm quickly returns without performing any work.  If the \textbf{used} count is less than or equal
to the shift count $b$ then it will simply zero the input and return.

After the trivial cases of inputs have been handled the sliding window is setup.  Much like the case of algorithm mp\_lshd a sliding window that
is $b$ digits wide is used to copy the digits.  Unlike mp\_lshd the window slides in the opposite direction from the trailing to the leading digit.  
Also the digits are copied from the leading to the trailing edge.

Once the window copy is complete the upper digits must be zeroed and the \textbf{used} count decremented.

\vspace{+3mm}\begin{small}
\hspace{-5.1mm}{\bf File}: bn\_mp\_rshd.c
\vspace{-3mm}
\begin{alltt}
\end{alltt}
\end{small}

The only noteworthy element of this routine is the lack of a return type since it cannot fail.  Like mp\_lshd() we
form a sliding window except we copy in the other direction.  After the window (line 60) we then zero
the upper digits of the input to make sure the result is correct.

\section{Powers of Two}

Now that algorithms for moving single bits as well as whole digits exist algorithms for moving the ``in between'' distances are required.  For 
example, to quickly multiply by $2^k$ for any $k$ without using a full multiplier algorithm would prove useful.  Instead of performing single
shifts $k$ times to achieve a multiplication by $2^{\pm k}$ a mixture of whole digit shifting and partial digit shifting is employed.  

\subsection{Multiplication by Power of Two}

\newpage\begin{figure}[!here]
\begin{small}
\begin{center}
\begin{tabular}{l}
\hline Algorithm \textbf{mp\_mul\_2d}. \\
\textbf{Input}.   One mp\_int $a$ and an integer $b$ \\
\textbf{Output}.  $c \leftarrow a \cdot 2^b$. \\
\hline \\
1.  $c \leftarrow a$.  (\textit{mp\_copy}) \\
2.  If $c.alloc < c.used + \lfloor b / lg(\beta) \rfloor + 2$ then grow $c$ accordingly. \\
3.  If the reallocation failed return(\textit{MP\_MEM}). \\
4.  If $b \ge lg(\beta)$ then \\
\hspace{3mm}4.1  $c \leftarrow c \cdot \beta^{\lfloor b / lg(\beta) \rfloor}$ (\textit{mp\_lshd}). \\
\hspace{3mm}4.2  If step 4.1 failed return(\textit{MP\_MEM}). \\
5.  $d \leftarrow b \mbox{ (mod }lg(\beta)\mbox{)}$ \\
6.  If $d \ne 0$ then do \\
\hspace{3mm}6.1  $mask \leftarrow 2^d$ \\
\hspace{3mm}6.2  $r \leftarrow 0$ \\
\hspace{3mm}6.3  for $n$ from $0$ to $c.used - 1$ do \\
\hspace{6mm}6.3.1  $rr \leftarrow c_n >> (lg(\beta) - d) \mbox{ (mod }mask\mbox{)}$ \\
\hspace{6mm}6.3.2  $c_n \leftarrow (c_n << d) + r \mbox{ (mod }\beta\mbox{)}$ \\
\hspace{6mm}6.3.3  $r \leftarrow rr$ \\
\hspace{3mm}6.4  If $r > 0$ then do \\
\hspace{6mm}6.4.1  $c_{c.used} \leftarrow r$ \\
\hspace{6mm}6.4.2  $c.used \leftarrow c.used + 1$ \\
7.  Return(\textit{MP\_OKAY}). \\
\hline
\end{tabular}
\end{center}
\end{small}
\caption{Algorithm mp\_mul\_2d}
\end{figure}

\textbf{Algorithm mp\_mul\_2d.}
This algorithm multiplies $a$ by $2^b$ and stores the result in $c$.  The algorithm uses algorithm mp\_lshd and a derivative of algorithm mp\_mul\_2 to
quickly compute the product.

First the algorithm will multiply $a$ by $x^{\lfloor b / lg(\beta) \rfloor}$ which will ensure that the remainder multiplicand is less than 
$\beta$.  For example, if $b = 37$ and $\beta = 2^{28}$ then this step will multiply by $x$ leaving a multiplication by $2^{37 - 28} = 2^{9}$ 
left.

After the digits have been shifted appropriately at most $lg(\beta) - 1$ shifts are left to perform.  Step 5 calculates the number of remaining shifts 
required.  If it is non-zero a modified shift loop is used to calculate the remaining product.  
Essentially the loop is a generic version of algorithm mp\_mul\_2 designed to handle any shift count in the range $1 \le x < lg(\beta)$.  The $mask$
variable is used to extract the upper $d$ bits to form the carry for the next iteration.  

This algorithm is loosely measured as a $O(2n)$ algorithm which means that if the input is $n$-digits that it takes $2n$ ``time'' to 
complete.  It is possible to optimize this algorithm down to a $O(n)$ algorithm at a cost of making the algorithm slightly harder to follow.

\vspace{+3mm}\begin{small}
\hspace{-5.1mm}{\bf File}: bn\_mp\_mul\_2d.c
\vspace{-3mm}
\begin{alltt}
\end{alltt}
\end{small}

The shifting is performed in--place which means the first step (line 25) is to copy the input to the 
destination.  We avoid calling mp\_copy() by making sure the mp\_ints are different.  The destination then
has to be grown (line 32) to accomodate the result.

If the shift count $b$ is larger than $lg(\beta)$ then a call to mp\_lshd() is used to handle all of the multiples 
of $lg(\beta)$.  Leaving only a remaining shift of $lg(\beta) - 1$ or fewer bits left.  Inside the actual shift 
loop (lines 46 to 76) we make use of pre--computed values $shift$ and $mask$.   These are used to
extract the carry bit(s) to pass into the next iteration of the loop.  The $r$ and $rr$ variables form a 
chain between consecutive iterations to propagate the carry.  

\subsection{Division by Power of Two}

\newpage\begin{figure}[!here]
\begin{small}
\begin{center}
\begin{tabular}{l}
\hline Algorithm \textbf{mp\_div\_2d}. \\
\textbf{Input}.   One mp\_int $a$ and an integer $b$ \\
\textbf{Output}.  $c \leftarrow \lfloor a / 2^b \rfloor, d \leftarrow a \mbox{ (mod }2^b\mbox{)}$. \\
\hline \\
1.  If $b \le 0$ then do \\
\hspace{3mm}1.1  $c \leftarrow a$ (\textit{mp\_copy}) \\
\hspace{3mm}1.2  $d \leftarrow 0$ (\textit{mp\_zero}) \\
\hspace{3mm}1.3  Return(\textit{MP\_OKAY}). \\
2.  $c \leftarrow a$ \\
3.  $d \leftarrow a \mbox{ (mod }2^b\mbox{)}$ (\textit{mp\_mod\_2d}) \\
4.  If $b \ge lg(\beta)$ then do \\
\hspace{3mm}4.1  $c \leftarrow \lfloor c/\beta^{\lfloor b/lg(\beta) \rfloor} \rfloor$ (\textit{mp\_rshd}). \\
5.  $k \leftarrow b \mbox{ (mod }lg(\beta)\mbox{)}$ \\
6.  If $k \ne 0$ then do \\
\hspace{3mm}6.1  $mask \leftarrow 2^k$ \\
\hspace{3mm}6.2  $r \leftarrow 0$ \\
\hspace{3mm}6.3  for $n$ from $c.used - 1$ to $0$ do \\
\hspace{6mm}6.3.1  $rr \leftarrow c_n \mbox{ (mod }mask\mbox{)}$ \\
\hspace{6mm}6.3.2  $c_n \leftarrow (c_n >> k) + (r << (lg(\beta) - k))$ \\
\hspace{6mm}6.3.3  $r \leftarrow rr$ \\
7.  Clamp excess digits of $c$.  (\textit{mp\_clamp}) \\
8.  Return(\textit{MP\_OKAY}). \\
\hline
\end{tabular}
\end{center}
\end{small}
\caption{Algorithm mp\_div\_2d}
\end{figure}

\textbf{Algorithm mp\_div\_2d.}
This algorithm will divide an input $a$ by $2^b$ and produce the quotient and remainder.  The algorithm is designed much like algorithm 
mp\_mul\_2d by first using whole digit shifts then single precision shifts.  This algorithm will also produce the remainder of the division
by using algorithm mp\_mod\_2d.

\vspace{+3mm}\begin{small}
\hspace{-5.1mm}{\bf File}: bn\_mp\_div\_2d.c
\vspace{-3mm}
\begin{alltt}
\end{alltt}
\end{small}

The implementation of algorithm mp\_div\_2d is slightly different than the algorithm specifies.  The remainder $d$ may be optionally 
ignored by passing \textbf{NULL} as the pointer to the mp\_int variable.    The temporary mp\_int variable $t$ is used to hold the 
result of the remainder operation until the end.  This allows $d$ and $a$ to represent the same mp\_int without modifying $a$ before
the quotient is obtained.

The remainder of the source code is essentially the same as the source code for mp\_mul\_2d.  The only significant difference is
the direction of the shifts.

\subsection{Remainder of Division by Power of Two}

The last algorithm in the series of polynomial basis power of two algorithms is calculating the remainder of division by $2^b$.  This
algorithm benefits from the fact that in twos complement arithmetic $a \mbox{ (mod }2^b\mbox{)}$ is the same as $a$ AND $2^b - 1$.  

\begin{figure}[!here]
\begin{small}
\begin{center}
\begin{tabular}{l}
\hline Algorithm \textbf{mp\_mod\_2d}. \\
\textbf{Input}.   One mp\_int $a$ and an integer $b$ \\
\textbf{Output}.  $c \leftarrow a \mbox{ (mod }2^b\mbox{)}$. \\
\hline \\
1.  If $b \le 0$ then do \\
\hspace{3mm}1.1  $c \leftarrow 0$ (\textit{mp\_zero}) \\
\hspace{3mm}1.2  Return(\textit{MP\_OKAY}). \\
2.  If $b > a.used \cdot lg(\beta)$ then do \\
\hspace{3mm}2.1  $c \leftarrow a$ (\textit{mp\_copy}) \\
\hspace{3mm}2.2  Return the result of step 2.1. \\
3.  $c \leftarrow a$ \\
4.  If step 3 failed return(\textit{MP\_MEM}). \\
5.  for $n$ from $\lceil b / lg(\beta) \rceil$ to $c.used$ do \\
\hspace{3mm}5.1  $c_n \leftarrow 0$ \\
6.  $k \leftarrow b \mbox{ (mod }lg(\beta)\mbox{)}$ \\
7.  $c_{\lfloor b / lg(\beta) \rfloor} \leftarrow c_{\lfloor b / lg(\beta) \rfloor} \mbox{ (mod }2^{k}\mbox{)}$. \\
8.  Clamp excess digits of $c$.  (\textit{mp\_clamp}) \\
9.  Return(\textit{MP\_OKAY}). \\
\hline
\end{tabular}
\end{center}
\end{small}
\caption{Algorithm mp\_mod\_2d}
\end{figure}

\textbf{Algorithm mp\_mod\_2d.}
This algorithm will quickly calculate the value of $a \mbox{ (mod }2^b\mbox{)}$.  First if $b$ is less than or equal to zero the 
result is set to zero.  If $b$ is greater than the number of bits in $a$ then it simply copies $a$ to $c$ and returns.  Otherwise, $a$ 
is copied to $b$, leading digits are removed and the remaining leading digit is trimed to the exact bit count.

\vspace{+3mm}\begin{small}
\hspace{-5.1mm}{\bf File}: bn\_mp\_mod\_2d.c
\vspace{-3mm}
\begin{alltt}
\end{alltt}
\end{small}

We first avoid cases of $b \le 0$ by simply mp\_zero()'ing the destination in such cases.  Next if $2^b$ is larger
than the input we just mp\_copy() the input and return right away.  After this point we know we must actually
perform some work to produce the remainder.

Recalling that reducing modulo $2^k$ and a binary ``and'' with $2^k - 1$ are numerically equivalent we can quickly reduce 
the number.  First we zero any digits above the last digit in $2^b$ (line 42).  Next we reduce the 
leading digit of both (line 46) and then mp\_clamp().

\section*{Exercises}
\begin{tabular}{cl}
$\left [ 3 \right ] $ & Devise an algorithm that performs $a \cdot 2^b$ for generic values of $b$ \\
                      & in $O(n)$ time. \\
                      &\\
$\left [ 3 \right ] $ & Devise an efficient algorithm to multiply by small low hamming  \\
                      & weight values such as $3$, $5$ and $9$.  Extend it to handle all values \\
                      & upto $64$ with a hamming weight less than three. \\
                      &\\
$\left [ 2 \right ] $ & Modify the preceding algorithm to handle values of the form \\
                      & $2^k - 1$ as well. \\
                      &\\
$\left [ 3 \right ] $ & Using only algorithms mp\_mul\_2, mp\_div\_2 and mp\_add create an \\
                      & algorithm to multiply two integers in roughly $O(2n^2)$ time for \\
                      & any $n$-bit input.  Note that the time of addition is ignored in the \\
                      & calculation.  \\
                      & \\
$\left [ 5 \right ] $ & Improve the previous algorithm to have a working time of at most \\
                      & $O \left (2^{(k-1)}n + \left ({2n^2 \over k} \right ) \right )$ for an appropriate choice of $k$.  Again ignore \\
                      & the cost of addition. \\
                      & \\
$\left [ 2 \right ] $ & Devise a chart to find optimal values of $k$ for the previous problem \\
                      & for $n = 64 \ldots 1024$ in steps of $64$. \\
                      & \\
$\left [ 2 \right ] $ & Using only algorithms mp\_abs and mp\_sub devise another method for \\
                      & calculating the result of a signed comparison. \\
                      &
\end{tabular}

\chapter{Multiplication and Squaring}
\section{The Multipliers}
For most number theoretic problems including certain public key cryptographic algorithms, the ``multipliers'' form the most important subset of 
algorithms of any multiple precision integer package.  The set of multiplier algorithms include integer multiplication, squaring and modular reduction 
where in each of the algorithms single precision multiplication is the dominant operation performed.  This chapter will discuss integer multiplication 
and squaring, leaving modular reductions for the subsequent chapter.  

The importance of the multiplier algorithms is for the most part driven by the fact that certain popular public key algorithms are based on modular 
exponentiation, that is computing $d \equiv a^b \mbox{ (mod }c\mbox{)}$ for some arbitrary choice of $a$, $b$, $c$ and $d$.  During a modular
exponentiation the majority\footnote{Roughly speaking a modular exponentiation will spend about 40\% of the time performing modular reductions, 
35\% of the time performing squaring and 25\% of the time performing multiplications.} of the processor time is spent performing single precision 
multiplications.

For centuries general purpose multiplication has required a lengthly $O(n^2)$ process, whereby each digit of one multiplicand has to be multiplied 
against every digit of the other multiplicand.  Traditional long-hand multiplication is based on this process;  while the techniques can differ the 
overall algorithm used is essentially the same.  Only ``recently'' have faster algorithms been studied.  First Karatsuba multiplication was discovered in 
1962.  This algorithm can multiply two numbers with considerably fewer single precision multiplications when compared to the long-hand approach.  
This technique led to the discovery of polynomial basis algorithms (\textit{good reference?}) and subquently Fourier Transform based solutions.  

\section{Multiplication}
\subsection{The Baseline Multiplication}
\label{sec:basemult}
\index{baseline multiplication}
Computing the product of two integers in software can be achieved using a trivial adaptation of the standard $O(n^2)$ long-hand multiplication
algorithm that school children are taught.  The algorithm is considered an $O(n^2)$ algorithm since for two $n$-digit inputs $n^2$ single precision 
multiplications are required.  More specifically for a $m$ and $n$ digit input $m \cdot n$ single precision multiplications are required.  To 
simplify most discussions, it will be assumed that the inputs have comparable number of digits.  

The ``baseline multiplication'' algorithm is designed to act as the ``catch-all'' algorithm, only to be used when the faster algorithms cannot be 
used.  This algorithm does not use any particularly interesting optimizations and should ideally be avoided if possible.    One important 
facet of this algorithm, is that it has been modified to only produce a certain amount of output digits as resolution.  The importance of this 
modification will become evident during the discussion of Barrett modular reduction.  Recall that for a $n$ and $m$ digit input the product 
will be at most $n + m$ digits.  Therefore, this algorithm can be reduced to a full multiplier by having it produce $n + m$ digits of the product.  

Recall from sub-section 4.2.2 the definition of $\gamma$ as the number of bits in the type \textbf{mp\_digit}.  We shall now extend the variable set to 
include $\alpha$ which shall represent the number of bits in the type \textbf{mp\_word}.  This implies that $2^{\alpha} > 2 \cdot \beta^2$.  The 
constant $\delta = 2^{\alpha - 2lg(\beta)}$ will represent the maximal weight of any column in a product (\textit{see sub-section 5.2.2 for more information}).

\newpage\begin{figure}[!here]
\begin{small}
\begin{center}
\begin{tabular}{l}
\hline Algorithm \textbf{s\_mp\_mul\_digs}. \\
\textbf{Input}.   mp\_int $a$, mp\_int $b$ and an integer $digs$ \\
\textbf{Output}.  $c \leftarrow \vert a \vert \cdot \vert b \vert \mbox{ (mod }\beta^{digs}\mbox{)}$. \\
\hline \\
1.  If min$(a.used, b.used) < \delta$ then do \\
\hspace{3mm}1.1  Calculate $c = \vert a \vert \cdot \vert b \vert$ by the Comba method (\textit{see algorithm~\ref{fig:COMBAMULT}}).  \\
\hspace{3mm}1.2  Return the result of step 1.1 \\
\\
Allocate and initialize a temporary mp\_int. \\
2.  Init $t$ to be of size $digs$ \\
3.  If step 2 failed return(\textit{MP\_MEM}). \\
4.  $t.used \leftarrow digs$ \\
\\
Compute the product. \\
5.  for $ix$ from $0$ to $a.used - 1$ do \\
\hspace{3mm}5.1  $u \leftarrow 0$ \\
\hspace{3mm}5.2  $pb \leftarrow \mbox{min}(b.used, digs - ix)$ \\
\hspace{3mm}5.3  If $pb < 1$ then goto step 6. \\
\hspace{3mm}5.4  for $iy$ from $0$ to $pb - 1$ do \\
\hspace{6mm}5.4.1  $\hat r \leftarrow t_{iy + ix} + a_{ix} \cdot b_{iy} + u$ \\
\hspace{6mm}5.4.2  $t_{iy + ix} \leftarrow \hat r \mbox{ (mod }\beta\mbox{)}$ \\
\hspace{6mm}5.4.3  $u \leftarrow \lfloor \hat r / \beta \rfloor$ \\
\hspace{3mm}5.5  if $ix + pb < digs$ then do \\
\hspace{6mm}5.5.1  $t_{ix + pb} \leftarrow u$ \\
6.  Clamp excess digits of $t$. \\
7.  Swap $c$ with $t$ \\
8.  Clear $t$ \\
9.  Return(\textit{MP\_OKAY}). \\
\hline
\end{tabular}
\end{center}
\end{small}
\caption{Algorithm s\_mp\_mul\_digs}
\end{figure}

\textbf{Algorithm s\_mp\_mul\_digs.}
This algorithm computes the unsigned product of two inputs $a$ and $b$, limited to an output precision of $digs$ digits.  While it may seem
a bit awkward to modify the function from its simple $O(n^2)$ description, the usefulness of partial multipliers will arise in a subsequent 
algorithm.  The algorithm is loosely based on algorithm 14.12 from \cite[pp. 595]{HAC} and is similar to Algorithm M of Knuth \cite[pp. 268]{TAOCPV2}.  
Algorithm s\_mp\_mul\_digs differs from these cited references since it can produce a variable output precision regardless of the precision of the 
inputs.

The first thing this algorithm checks for is whether a Comba multiplier can be used instead.   If the minimum digit count of either
input is less than $\delta$, then the Comba method may be used instead.    After the Comba method is ruled out, the baseline algorithm begins.  A 
temporary mp\_int variable $t$ is used to hold the intermediate result of the product.  This allows the algorithm to be used to 
compute products when either $a = c$ or $b = c$ without overwriting the inputs.  

All of step 5 is the infamous $O(n^2)$ multiplication loop slightly modified to only produce upto $digs$ digits of output.  The $pb$ variable
is given the count of digits to read from $b$ inside the nested loop.  If $pb \le 1$ then no more output digits can be produced and the algorithm
will exit the loop.  The best way to think of the loops are as a series of $pb \times 1$ multiplications.    That is, in each pass of the 
innermost loop $a_{ix}$ is multiplied against $b$ and the result is added (\textit{with an appropriate shift}) to $t$.  

For example, consider multiplying $576$ by $241$.  That is equivalent to computing $10^0(1)(576) + 10^1(4)(576) + 10^2(2)(576)$ which is best
visualized in the following table.

\begin{figure}[here]
\begin{center}
\begin{tabular}{|c|c|c|c|c|c|l|}
\hline   &&          & 5 & 7 & 6 & \\
\hline   $\times$&&  & 2 & 4 & 1 & \\
\hline &&&&&&\\
  &&          & 5 & 7 & 6 & $10^0(1)(576)$ \\
  &2 &   3    & 6 & 1 & 6 & $10^1(4)(576) + 10^0(1)(576)$ \\
  1 & 3 & 8 & 8 & 1 & 6 &   $10^2(2)(576) + 10^1(4)(576) + 10^0(1)(576)$ \\
\hline  
\end{tabular}
\end{center}
\caption{Long-Hand Multiplication Diagram}
\end{figure}

Each row of the product is added to the result after being shifted to the left (\textit{multiplied by a power of the radix}) by the appropriate 
count.  That is in pass $ix$ of the inner loop the product is added starting at the $ix$'th digit of the reult.

Step 5.4.1 introduces the hat symbol (\textit{e.g. $\hat r$}) which represents a double precision variable.  The multiplication on that step
is assumed to be a double wide output single precision multiplication.  That is, two single precision variables are multiplied to produce a
double precision result.  The step is somewhat optimized from a long-hand multiplication algorithm because the carry from the addition in step
5.4.1 is propagated through the nested loop.  If the carry was not propagated immediately it would overflow the single precision digit 
$t_{ix+iy}$ and the result would be lost.  

At step 5.5 the nested loop is finished and any carry that was left over should be forwarded.  The carry does not have to be added to the $ix+pb$'th
digit since that digit is assumed to be zero at this point.  However, if $ix + pb \ge digs$ the carry is not set as it would make the result
exceed the precision requested.

\vspace{+3mm}\begin{small}
\hspace{-5.1mm}{\bf File}: bn\_s\_mp\_mul\_digs.c
\vspace{-3mm}
\begin{alltt}
\end{alltt}
\end{small}

First we determine (line 31) if the Comba method can be used first since it's faster.  The conditions for 
sing the Comba routine are that min$(a.used, b.used) < \delta$ and the number of digits of output is less than 
\textbf{MP\_WARRAY}.  This new constant is used to control the stack usage in the Comba routines.  By default it is 
set to $\delta$ but can be reduced when memory is at a premium.

If we cannot use the Comba method we proceed to setup the baseline routine.  We allocate the the destination mp\_int
$t$ (line 37) to the exact size of the output to avoid further re--allocations.  At this point we now 
begin the $O(n^2)$ loop.

This implementation of multiplication has the caveat that it can be trimmed to only produce a variable number of
digits as output.  In each iteration of the outer loop the $pb$ variable is set (line 49) to the maximum 
number of inner loop iterations.  

Inside the inner loop we calculate $\hat r$ as the mp\_word product of the two mp\_digits and the addition of the
carry from the previous iteration.  A particularly important observation is that most modern optimizing 
C compilers (GCC for instance) can recognize that a $N \times N \rightarrow 2N$ multiplication is all that 
is required for the product.  In x86 terms for example, this means using the MUL instruction.

Each digit of the product is stored in turn (line 69) and the carry propagated (line 72) to the 
next iteration.

\subsection{Faster Multiplication by the ``Comba'' Method}

One of the huge drawbacks of the ``baseline'' algorithms is that at the $O(n^2)$ level the carry must be 
computed and propagated upwards.  This makes the nested loop very sequential and hard to unroll and implement 
in parallel.  The ``Comba'' \cite{COMBA} method is named after little known (\textit{in cryptographic venues}) Paul G. 
Comba who described a method of implementing fast multipliers that do not require nested carry fixup operations.  As an 
interesting aside it seems that Paul Barrett describes a similar technique in his 1986 paper \cite{BARRETT} written 
five years before.

At the heart of the Comba technique is once again the long-hand algorithm.  Except in this case a slight 
twist is placed on how the columns of the result are produced.  In the standard long-hand algorithm rows of products 
are produced then added together to form the final result.  In the baseline algorithm the columns are added together 
after each iteration to get the result instantaneously.  

In the Comba algorithm the columns of the result are produced entirely independently of each other.  That is at 
the $O(n^2)$ level a simple multiplication and addition step is performed.  The carries of the columns are propagated 
after the nested loop to reduce the amount of work requiored. Succintly the first step of the algorithm is to compute 
the product vector $\vec x$ as follows. 

\begin{equation}
\vec x_n = \sum_{i+j = n} a_ib_j, \forall n \in \lbrace 0, 1, 2, \ldots, i + j \rbrace
\end{equation}

Where $\vec x_n$ is the $n'th$ column of the output vector.  Consider the following example which computes the vector $\vec x$ for the multiplication
of $576$ and $241$.  

\newpage\begin{figure}[here]
\begin{small}
\begin{center}
\begin{tabular}{|c|c|c|c|c|c|}
  \hline &          & 5 & 7 & 6 & First Input\\
  \hline $\times$ & & 2 & 4 & 1 & Second Input\\
\hline            &                        & $1 \cdot 5 = 5$   & $1 \cdot 7 = 7$   & $1 \cdot 6 = 6$ & First pass \\
                  &  $4 \cdot 5 = 20$      & $4 \cdot 7+5=33$  & $4 \cdot 6+7=31$  & 6               & Second pass \\
   $2 \cdot 5 = 10$ &  $2 \cdot 7 + 20 = 34$ & $2 \cdot 6+33=45$ & 31                & 6             & Third pass \\
\hline 10 & 34 & 45 & 31 & 6 & Final Result \\   
\hline   
\end{tabular}
\end{center}
\end{small}
\caption{Comba Multiplication Diagram}
\end{figure}

At this point the vector $x = \left < 10, 34, 45, 31, 6 \right >$ is the result of the first step of the Comba multipler.  
Now the columns must be fixed by propagating the carry upwards.  The resultant vector will have one extra dimension over the input vector which is
congruent to adding a leading zero digit.

\begin{figure}[!here]
\begin{small}
\begin{center}
\begin{tabular}{l}
\hline Algorithm \textbf{Comba Fixup}. \\
\textbf{Input}.   Vector $\vec x$ of dimension $k$ \\
\textbf{Output}.  Vector $\vec x$ such that the carries have been propagated. \\
\hline \\
1.  for $n$ from $0$ to $k - 1$ do \\
\hspace{3mm}1.1 $\vec x_{n+1} \leftarrow \vec x_{n+1} + \lfloor \vec x_{n}/\beta \rfloor$ \\
\hspace{3mm}1.2 $\vec x_{n} \leftarrow \vec x_{n} \mbox{ (mod }\beta\mbox{)}$ \\
2.  Return($\vec x$). \\
\hline
\end{tabular}
\end{center}
\end{small}
\caption{Algorithm Comba Fixup}
\end{figure}

With that algorithm and $k = 5$ and $\beta = 10$ the following vector is produced $\vec x= \left < 1, 3, 8, 8, 1, 6 \right >$.  In this case 
$241 \cdot 576$ is in fact $138816$ and the procedure succeeded.  If the algorithm is correct and as will be demonstrated shortly more
efficient than the baseline algorithm why not simply always use this algorithm?

\subsubsection{Column Weight.}
At the nested $O(n^2)$ level the Comba method adds the product of two single precision variables to each column of the output 
independently.  A serious obstacle is if the carry is lost, due to lack of precision before the algorithm has a chance to fix
the carries.  For example, in the multiplication of two three-digit numbers the third column of output will be the sum of
three single precision multiplications.  If the precision of the accumulator for the output digits is less then $3 \cdot (\beta - 1)^2$ then
an overflow can occur and the carry information will be lost.  For any $m$ and $n$ digit inputs the maximum weight of any column is 
min$(m, n)$ which is fairly obvious.

The maximum number of terms in any column of a product is known as the ``column weight'' and strictly governs when the algorithm can be used.  Recall
from earlier that a double precision type has $\alpha$ bits of resolution and a single precision digit has $lg(\beta)$ bits of precision.  Given these
two quantities we must not violate the following

\begin{equation}
k \cdot \left (\beta - 1 \right )^2 < 2^{\alpha}
\end{equation}

Which reduces to 

\begin{equation}
k \cdot \left ( \beta^2 - 2\beta + 1 \right ) < 2^{\alpha}
\end{equation}

Let $\rho = lg(\beta)$ represent the number of bits in a single precision digit.  By further re-arrangement of the equation the final solution is
found.

\begin{equation}
k  < {{2^{\alpha}} \over {\left (2^{2\rho} - 2^{\rho + 1} + 1 \right )}}
\end{equation}

The defaults for LibTomMath are $\beta = 2^{28}$ and $\alpha = 2^{64}$ which means that $k$ is bounded by $k < 257$.  In this configuration 
the smaller input may not have more than $256$ digits if the Comba method is to be used.  This is quite satisfactory for most applications since 
$256$ digits would allow for numbers in the range of $0 \le x < 2^{7168}$ which, is much larger than most public key cryptographic algorithms require.  

\newpage\begin{figure}[!here]
\begin{small}
\begin{center}
\begin{tabular}{l}
\hline Algorithm \textbf{fast\_s\_mp\_mul\_digs}. \\
\textbf{Input}.   mp\_int $a$, mp\_int $b$ and an integer $digs$ \\
\textbf{Output}.  $c \leftarrow \vert a \vert \cdot \vert b \vert \mbox{ (mod }\beta^{digs}\mbox{)}$. \\
\hline \\
Place an array of \textbf{MP\_WARRAY} single precision digits named $W$ on the stack. \\
1.  If $c.alloc < digs$ then grow $c$ to $digs$ digits. (\textit{mp\_grow}) \\
2.  If step 1 failed return(\textit{MP\_MEM}).\\
\\
3.  $pa \leftarrow \mbox{MIN}(digs, a.used + b.used)$ \\
\\
4.  $\_ \hat W \leftarrow 0$ \\
5.  for $ix$ from 0 to $pa - 1$ do \\
\hspace{3mm}5.1  $ty \leftarrow \mbox{MIN}(b.used - 1, ix)$ \\
\hspace{3mm}5.2  $tx \leftarrow ix - ty$ \\
\hspace{3mm}5.3  $iy \leftarrow \mbox{MIN}(a.used - tx, ty + 1)$ \\
\hspace{3mm}5.4  for $iz$ from 0 to $iy - 1$ do \\
\hspace{6mm}5.4.1  $\_ \hat W \leftarrow \_ \hat W + a_{tx+iy}b_{ty-iy}$ \\
\hspace{3mm}5.5  $W_{ix} \leftarrow \_ \hat W (\mbox{mod }\beta)$\\
\hspace{3mm}5.6  $\_ \hat W \leftarrow \lfloor \_ \hat W / \beta \rfloor$ \\
\\
6.  $oldused \leftarrow c.used$ \\
7.  $c.used \leftarrow digs$ \\
8.  for $ix$ from $0$ to $pa$ do \\
\hspace{3mm}8.1  $c_{ix} \leftarrow W_{ix}$ \\
9.  for $ix$ from $pa + 1$ to $oldused - 1$ do \\
\hspace{3mm}9.1 $c_{ix} \leftarrow 0$ \\
\\
10.  Clamp $c$. \\
11.  Return MP\_OKAY. \\
\hline
\end{tabular}
\end{center}
\end{small}
\caption{Algorithm fast\_s\_mp\_mul\_digs}
\label{fig:COMBAMULT}
\end{figure}

\textbf{Algorithm fast\_s\_mp\_mul\_digs.}
This algorithm performs the unsigned multiplication of $a$ and $b$ using the Comba method limited to $digs$ digits of precision.

The outer loop of this algorithm is more complicated than that of the baseline multiplier.  This is because on the inside of the 
loop we want to produce one column per pass.  This allows the accumulator $\_ \hat W$ to be placed in CPU registers and
reduce the memory bandwidth to two \textbf{mp\_digit} reads per iteration.

The $ty$ variable is set to the minimum count of $ix$ or the number of digits in $b$.  That way if $a$ has more digits than
$b$ this will be limited to $b.used - 1$.  The $tx$ variable is set to the to the distance past $b.used$ the variable
$ix$ is.  This is used for the immediately subsequent statement where we find $iy$.  

The variable $iy$ is the minimum digits we can read from either $a$ or $b$ before running out.  Computing one column at a time
means we have to scan one integer upwards and the other downwards.  $a$ starts at $tx$ and $b$ starts at $ty$.  In each
pass we are producing the $ix$'th output column and we note that $tx + ty = ix$.  As we move $tx$ upwards we have to 
move $ty$ downards so the equality remains valid.  The $iy$ variable is the number of iterations until 
$tx \ge a.used$ or $ty < 0$ occurs.

After every inner pass we store the lower half of the accumulator into $W_{ix}$ and then propagate the carry of the accumulator
into the next round by dividing $\_ \hat W$ by $\beta$.

To measure the benefits of the Comba method over the baseline method consider the number of operations that are required.  If the 
cost in terms of time of a multiply and addition is $p$ and the cost of a carry propagation is $q$ then a baseline multiplication would require 
$O \left ((p + q)n^2 \right )$ time to multiply two $n$-digit numbers.  The Comba method requires only $O(pn^2 + qn)$ time, however in practice, 
the speed increase is actually much more.  With $O(n)$ space the algorithm can be reduced to $O(pn + qn)$ time by implementing the $n$ multiply
and addition operations in the nested loop in parallel.  

\vspace{+3mm}\begin{small}
\hspace{-5.1mm}{\bf File}: bn\_fast\_s\_mp\_mul\_digs.c
\vspace{-3mm}
\begin{alltt}
\end{alltt}
\end{small}

As per the pseudo--code we first calculate $pa$ (line 48) as the number of digits to output.  Next we begin the outer loop
to produce the individual columns of the product.  We use the two aliases $tmpx$ and $tmpy$ (lines 62, 63) to point
inside the two multiplicands quickly.  

The inner loop (lines 71 to 74) of this implementation is where the tradeoff come into play.  Originally this comba 
implementation was ``row--major'' which means it adds to each of the columns in each pass.  After the outer loop it would then fix 
the carries.  This was very fast except it had an annoying drawback.  You had to read a mp\_word and two mp\_digits and write 
one mp\_word per iteration.  On processors such as the Athlon XP and P4 this did not matter much since the cache bandwidth 
is very high and it can keep the ALU fed with data.  It did, however, matter on older and embedded cpus where cache is often 
slower and also often doesn't exist.  This new algorithm only performs two reads per iteration under the assumption that the 
compiler has aliased $\_ \hat W$ to a CPU register.

After the inner loop we store the current accumulator in $W$ and shift $\_ \hat W$ (lines 77, 80) to forward it as 
a carry for the next pass.  After the outer loop we use the final carry (line 77) as the last digit of the product.  

\subsection{Polynomial Basis Multiplication}
To break the $O(n^2)$ barrier in multiplication requires a completely different look at integer multiplication.  In the following algorithms
the use of polynomial basis representation for two integers $a$ and $b$ as $f(x) = \sum_{i=0}^{n} a_i x^i$ and  
$g(x) = \sum_{i=0}^{n} b_i x^i$ respectively, is required.  In this system both $f(x)$ and $g(x)$ have $n + 1$ terms and are of the $n$'th degree.
 
The product $a \cdot b \equiv f(x)g(x)$ is the polynomial $W(x) = \sum_{i=0}^{2n} w_i x^i$.  The coefficients $w_i$ will
directly yield the desired product when $\beta$ is substituted for $x$.  The direct solution to solve for the $2n + 1$ coefficients
requires $O(n^2)$ time and would in practice be slower than the Comba technique.

However, numerical analysis theory indicates that only $2n + 1$ distinct points in $W(x)$ are required to determine the values of the $2n + 1$ unknown 
coefficients.   This means by finding $\zeta_y = W(y)$ for $2n + 1$ small values of $y$ the coefficients of $W(x)$ can be found with 
Gaussian elimination.  This technique is also occasionally refered to as the \textit{interpolation technique} (\textit{references please...}) since in 
effect an interpolation based on $2n + 1$ points will yield a polynomial equivalent to $W(x)$.  

The coefficients of the polynomial $W(x)$ are unknown which makes finding $W(y)$ for any value of $y$ impossible.  However, since 
$W(x) = f(x)g(x)$ the equivalent $\zeta_y = f(y) g(y)$ can be used in its place.  The benefit of this technique stems from the 
fact that $f(y)$ and $g(y)$ are much smaller than either $a$ or $b$ respectively.  As a result finding the $2n + 1$ relations required 
by multiplying $f(y)g(y)$ involves multiplying integers that are much smaller than either of the inputs.

When picking points to gather relations there are always three obvious points to choose, $y = 0, 1$ and $ \infty$.  The $\zeta_0$ term
is simply the product $W(0) = w_0 = a_0 \cdot b_0$.  The $\zeta_1$ term is the product 
$W(1) = \left (\sum_{i = 0}^{n} a_i \right ) \left (\sum_{i = 0}^{n} b_i \right )$.  The third point $\zeta_{\infty}$ is less obvious but rather
simple to explain.  The $2n + 1$'th coefficient of $W(x)$ is numerically equivalent to the most significant column in an integer multiplication.  
The point at $\infty$ is used symbolically to represent the most significant column, that is $W(\infty) = w_{2n} = a_nb_n$.  Note that the 
points at $y = 0$ and $\infty$ yield the coefficients $w_0$ and $w_{2n}$ directly.

If more points are required they should be of small values and powers of two such as $2^q$ and the related \textit{mirror points} 
$\left (2^q \right )^{2n}  \cdot \zeta_{2^{-q}}$ for small values of $q$.  The term ``mirror point'' stems from the fact that 
$\left (2^q \right )^{2n}  \cdot \zeta_{2^{-q}}$ can be calculated in the exact opposite fashion as $\zeta_{2^q}$.  For
example, when $n = 2$ and $q = 1$ then following two equations are equivalent to the point $\zeta_{2}$ and its mirror.

\begin{eqnarray}
\zeta_{2}                  = f(2)g(2) = (4a_2 + 2a_1 + a_0)(4b_2 + 2b_1 + b_0) \nonumber \\
16 \cdot \zeta_{1 \over 2} = 4f({1\over 2}) \cdot 4g({1 \over 2}) = (a_2 + 2a_1 + 4a_0)(b_2 + 2b_1 + 4b_0)
\end{eqnarray}

Using such points will allow the values of $f(y)$ and $g(y)$ to be independently calculated using only left shifts.  For example, when $n = 2$ the
polynomial $f(2^q)$ is equal to $2^q((2^qa_2) + a_1) + a_0$.  This technique of polynomial representation is known as Horner's method.  

As a general rule of the algorithm when the inputs are split into $n$ parts each there are $2n - 1$ multiplications.  Each multiplication is of 
multiplicands that have $n$ times fewer digits than the inputs.  The asymptotic running time of this algorithm is 
$O \left ( k^{lg_n(2n - 1)} \right )$ for $k$ digit inputs (\textit{assuming they have the same number of digits}).  Figure~\ref{fig:exponent}
summarizes the exponents for various values of $n$.

\begin{figure}
\begin{center}
\begin{tabular}{|c|c|c|}
\hline \textbf{Split into $n$ Parts} & \textbf{Exponent}  & \textbf{Notes}\\
\hline $2$ & $1.584962501$ & This is Karatsuba Multiplication. \\
\hline $3$ & $1.464973520$ & This is Toom-Cook Multiplication. \\
\hline $4$ & $1.403677461$ &\\
\hline $5$ & $1.365212389$ &\\
\hline $10$ & $1.278753601$ &\\
\hline $100$ & $1.149426538$ &\\
\hline $1000$ & $1.100270931$ &\\
\hline $10000$ & $1.075252070$ &\\
\hline
\end{tabular}
\end{center}
\caption{Asymptotic Running Time of Polynomial Basis Multiplication}
\label{fig:exponent}
\end{figure}

At first it may seem like a good idea to choose $n = 1000$ since the exponent is approximately $1.1$.  However, the overhead
of solving for the 2001 terms of $W(x)$ will certainly consume any savings the algorithm could offer for all but exceedingly large
numbers.  

\subsubsection{Cutoff Point}
The polynomial basis multiplication algorithms all require fewer single precision multiplications than a straight Comba approach.  However, 
the algorithms incur an overhead (\textit{at the $O(n)$ work level}) since they require a system of equations to be solved.  This makes the
polynomial basis approach more costly to use with small inputs.

Let $m$ represent the number of digits in the multiplicands (\textit{assume both multiplicands have the same number of digits}).  There exists a 
point $y$ such that when $m < y$ the polynomial basis algorithms are more costly than Comba, when $m = y$ they are roughly the same cost and 
when $m > y$ the Comba methods are slower than the polynomial basis algorithms.  

The exact location of $y$ depends on several key architectural elements of the computer platform in question.

\begin{enumerate}
\item  The ratio of clock cycles for single precision multiplication versus other simpler operations such as addition, shifting, etc.  For example
on the AMD Athlon the ratio is roughly $17 : 1$ while on the Intel P4 it is $29 : 1$.  The higher the ratio in favour of multiplication the lower
the cutoff point $y$ will be.  

\item  The complexity of the linear system of equations (\textit{for the coefficients of $W(x)$}) is.  Generally speaking as the number of splits
grows the complexity grows substantially.  Ideally solving the system will only involve addition, subtraction and shifting of integers.  This
directly reflects on the ratio previous mentioned.

\item  To a lesser extent memory bandwidth and function call overheads.  Provided the values are in the processor cache this is less of an
influence over the cutoff point.

\end{enumerate}

A clean cutoff point separation occurs when a point $y$ is found such that all of the cutoff point conditions are met.  For example, if the point
is too low then there will be values of $m$ such that $m > y$ and the Comba method is still faster.  Finding the cutoff points is fairly simple when
a high resolution timer is available.  

\subsection{Karatsuba Multiplication}
Karatsuba \cite{KARA} multiplication when originally proposed in 1962 was among the first set of algorithms to break the $O(n^2)$ barrier for
general purpose multiplication.  Given two polynomial basis representations $f(x) = ax + b$ and $g(x) = cx + d$, Karatsuba proved with 
light algebra \cite{KARAP} that the following polynomial is equivalent to multiplication of the two integers the polynomials represent.

\begin{equation}
f(x) \cdot g(x) = acx^2 + ((a + b)(c + d) - (ac + bd))x + bd
\end{equation}

Using the observation that $ac$ and $bd$ could be re-used only three half sized multiplications would be required to produce the product.  Applying
this algorithm recursively, the work factor becomes $O(n^{lg(3)})$ which is substantially better than the work factor $O(n^2)$ of the Comba technique.  It turns 
out what Karatsuba did not know or at least did not publish was that this is simply polynomial basis multiplication with the points 
$\zeta_0$, $\zeta_{\infty}$ and $\zeta_{1}$.  Consider the resultant system of equations.

\begin{center}
\begin{tabular}{rcrcrcrc}
$\zeta_{0}$ &      $=$ &  &  &  & & $w_0$ \\
$\zeta_{1}$ &      $=$ & $w_2$ & $+$ & $w_1$ & $+$ & $w_0$ \\
$\zeta_{\infty}$ & $=$ & $w_2$ &  & &  & \\
\end{tabular}
\end{center}

By adding the first and last equation to the equation in the middle the term $w_1$ can be isolated and all three coefficients solved for.  The simplicity
of this system of equations has made Karatsuba fairly popular.  In fact the cutoff point is often fairly low\footnote{With LibTomMath 0.18 it is 70 and 109 digits for the Intel P4 and AMD Athlon respectively.}
making it an ideal algorithm to speed up certain public key cryptosystems such as RSA and Diffie-Hellman.  

\newpage\begin{figure}[!here]
\begin{small}
\begin{center}
\begin{tabular}{l}
\hline Algorithm \textbf{mp\_karatsuba\_mul}. \\
\textbf{Input}.   mp\_int $a$ and mp\_int $b$ \\
\textbf{Output}.  $c \leftarrow \vert a \vert \cdot \vert b \vert$ \\
\hline \\
1.  Init the following mp\_int variables: $x0$, $x1$, $y0$, $y1$, $t1$, $x0y0$, $x1y1$.\\
2.  If step 2 failed then return(\textit{MP\_MEM}). \\
\\
Split the input.  e.g. $a = x1 \cdot \beta^B + x0$ \\
3.  $B \leftarrow \mbox{min}(a.used, b.used)/2$ \\
4.  $x0 \leftarrow a \mbox{ (mod }\beta^B\mbox{)}$ (\textit{mp\_mod\_2d}) \\
5.  $y0 \leftarrow b \mbox{ (mod }\beta^B\mbox{)}$ \\
6.  $x1 \leftarrow \lfloor a / \beta^B \rfloor$ (\textit{mp\_rshd}) \\
7.  $y1 \leftarrow \lfloor b / \beta^B \rfloor$ \\
\\
Calculate the three products. \\
8.  $x0y0 \leftarrow x0 \cdot y0$ (\textit{mp\_mul}) \\
9.  $x1y1 \leftarrow x1 \cdot y1$ \\
10.  $t1 \leftarrow x1 + x0$ (\textit{mp\_add}) \\
11.  $x0 \leftarrow y1 + y0$ \\
12.  $t1 \leftarrow t1 \cdot x0$ \\
\\
Calculate the middle term. \\
13.  $x0 \leftarrow x0y0 + x1y1$ \\
14.  $t1 \leftarrow t1 - x0$ (\textit{s\_mp\_sub}) \\
\\
Calculate the final product. \\
15.  $t1 \leftarrow t1 \cdot \beta^B$ (\textit{mp\_lshd}) \\
16.  $x1y1 \leftarrow x1y1 \cdot \beta^{2B}$ \\
17.  $t1 \leftarrow x0y0 + t1$ \\
18.  $c \leftarrow t1 + x1y1$ \\
19.  Clear all of the temporary variables. \\
20.  Return(\textit{MP\_OKAY}).\\
\hline 
\end{tabular}
\end{center}
\end{small}
\caption{Algorithm mp\_karatsuba\_mul}
\end{figure}

\textbf{Algorithm mp\_karatsuba\_mul.}
This algorithm computes the unsigned product of two inputs using the Karatsuba multiplication algorithm.  It is loosely based on the description
from Knuth \cite[pp. 294-295]{TAOCPV2}.  

\index{radix point}
In order to split the two inputs into their respective halves, a suitable \textit{radix point} must be chosen.  The radix point chosen must
be used for both of the inputs meaning that it must be smaller than the smallest input.  Step 3 chooses the radix point $B$ as half of the 
smallest input \textbf{used} count.  After the radix point is chosen the inputs are split into lower and upper halves.  Step 4 and 5 
compute the lower halves.  Step 6 and 7 computer the upper halves.  

After the halves have been computed the three intermediate half-size products must be computed.  Step 8 and 9 compute the trivial products
$x0 \cdot y0$ and $x1 \cdot y1$.  The mp\_int $x0$ is used as a temporary variable after $x1 + x0$ has been computed.  By using $x0$ instead
of an additional temporary variable, the algorithm can avoid an addition memory allocation operation.

The remaining steps 13 through 18 compute the Karatsuba polynomial through a variety of digit shifting and addition operations.

\vspace{+3mm}\begin{small}
\hspace{-5.1mm}{\bf File}: bn\_mp\_karatsuba\_mul.c
\vspace{-3mm}
\begin{alltt}
\end{alltt}
\end{small}

The new coding element in this routine, not  seen in previous routines, is the usage of goto statements.  The conventional
wisdom is that goto statements should be avoided.  This is generally true, however when every single function call can fail, it makes sense
to handle error recovery with a single piece of code.  Lines 62 to 76 handle initializing all of the temporary variables 
required.  Note how each of the if statements goes to a different label in case of failure.  This allows the routine to correctly free only
the temporaries that have been successfully allocated so far.

The temporary variables are all initialized using the mp\_init\_size routine since they are expected to be large.  This saves the 
additional reallocation that would have been necessary.  Also $x0$, $x1$, $y0$ and $y1$ have to be able to hold at least their respective
number of digits for the next section of code.

The first algebraic portion of the algorithm is to split the two inputs into their halves.  However, instead of using mp\_mod\_2d and mp\_rshd
to extract the halves, the respective code has been placed inline within the body of the function.  To initialize the halves, the \textbf{used} and 
\textbf{sign} members are copied first.  The first for loop on line 96 copies the lower halves.  Since they are both the same magnitude it 
is simpler to calculate both lower halves in a single loop.  The for loop on lines 102 and 107 calculate the upper halves $x1$ and 
$y1$ respectively.

By inlining the calculation of the halves, the Karatsuba multiplier has a slightly lower overhead and can be used for smaller magnitude inputs.

When line 151 is reached, the algorithm has completed succesfully.  The ``error status'' variable $err$ is set to \textbf{MP\_OKAY} so that
the same code that handles errors can be used to clear the temporary variables and return.  

\subsection{Toom-Cook $3$-Way Multiplication}
Toom-Cook $3$-Way \cite{TOOM} multiplication is essentially the polynomial basis algorithm for $n = 2$ except that the points  are 
chosen such that $\zeta$ is easy to compute and the resulting system of equations easy to reduce.  Here, the points $\zeta_{0}$, 
$16 \cdot \zeta_{1 \over 2}$, $\zeta_1$, $\zeta_2$ and $\zeta_{\infty}$ make up the five required points to solve for the coefficients 
of the $W(x)$.

With the five relations that Toom-Cook specifies, the following system of equations is formed.

\begin{center}
\begin{tabular}{rcrcrcrcrcr}
$\zeta_0$                    & $=$ & $0w_4$ & $+$ & $0w_3$ & $+$ & $0w_2$ & $+$ & $0w_1$ & $+$ & $1w_0$  \\
$16 \cdot \zeta_{1 \over 2}$ & $=$ & $1w_4$ & $+$ & $2w_3$ & $+$ & $4w_2$ & $+$ & $8w_1$ & $+$ & $16w_0$  \\
$\zeta_1$                    & $=$ & $1w_4$ & $+$ & $1w_3$ & $+$ & $1w_2$ & $+$ & $1w_1$ & $+$ & $1w_0$  \\
$\zeta_2$                    & $=$ & $16w_4$ & $+$ & $8w_3$ & $+$ & $4w_2$ & $+$ & $2w_1$ & $+$ & $1w_0$  \\
$\zeta_{\infty}$             & $=$ & $1w_4$ & $+$ & $0w_3$ & $+$ & $0w_2$ & $+$ & $0w_1$ & $+$ & $0w_0$  \\
\end{tabular}
\end{center}

A trivial solution to this matrix requires $12$ subtractions, two multiplications by a small power of two, two divisions by a small power
of two, two divisions by three and one multiplication by three.  All of these $19$ sub-operations require less than quadratic time, meaning that
the algorithm can be faster than a baseline multiplication.  However, the greater complexity of this algorithm places the cutoff point
(\textbf{TOOM\_MUL\_CUTOFF}) where Toom-Cook becomes more efficient much higher than the Karatsuba cutoff point.  

\begin{figure}[!here]
\begin{small}
\begin{center}
\begin{tabular}{l}
\hline Algorithm \textbf{mp\_toom\_mul}. \\
\textbf{Input}.   mp\_int $a$ and mp\_int $b$ \\
\textbf{Output}.  $c \leftarrow  a  \cdot  b $ \\
\hline \\
Split $a$ and $b$ into three pieces.  E.g. $a = a_2 \beta^{2k} + a_1 \beta^{k} + a_0$ \\
1.  $k \leftarrow \lfloor \mbox{min}(a.used, b.used) / 3 \rfloor$ \\
2.  $a_0 \leftarrow a \mbox{ (mod }\beta^{k}\mbox{)}$ \\
3.  $a_1 \leftarrow \lfloor a / \beta^k \rfloor$, $a_1 \leftarrow a_1 \mbox{ (mod }\beta^{k}\mbox{)}$ \\
4.  $a_2 \leftarrow \lfloor a / \beta^{2k} \rfloor$, $a_2 \leftarrow a_2 \mbox{ (mod }\beta^{k}\mbox{)}$ \\
5.  $b_0 \leftarrow a \mbox{ (mod }\beta^{k}\mbox{)}$ \\
6.  $b_1 \leftarrow \lfloor a / \beta^k \rfloor$, $b_1 \leftarrow b_1 \mbox{ (mod }\beta^{k}\mbox{)}$ \\
7.  $b_2 \leftarrow \lfloor a / \beta^{2k} \rfloor$, $b_2 \leftarrow b_2 \mbox{ (mod }\beta^{k}\mbox{)}$ \\
\\
Find the five equations for $w_0, w_1, ..., w_4$. \\
8.  $w_0 \leftarrow a_0 \cdot b_0$ \\
9.  $w_4 \leftarrow a_2 \cdot b_2$ \\
10. $tmp_1 \leftarrow 2 \cdot a_0$, $tmp_1 \leftarrow a_1 + tmp_1$, $tmp_1 \leftarrow 2 \cdot tmp_1$, $tmp_1 \leftarrow tmp_1 + a_2$ \\
11. $tmp_2 \leftarrow 2 \cdot b_0$, $tmp_2 \leftarrow b_1 + tmp_2$, $tmp_2 \leftarrow 2 \cdot tmp_2$, $tmp_2 \leftarrow tmp_2 + b_2$ \\
12. $w_1 \leftarrow tmp_1 \cdot tmp_2$ \\
13. $tmp_1 \leftarrow 2 \cdot a_2$, $tmp_1 \leftarrow a_1 + tmp_1$, $tmp_1 \leftarrow 2 \cdot tmp_1$, $tmp_1 \leftarrow tmp_1 + a_0$ \\
14. $tmp_2 \leftarrow 2 \cdot b_2$, $tmp_2 \leftarrow b_1 + tmp_2$, $tmp_2 \leftarrow 2 \cdot tmp_2$, $tmp_2 \leftarrow tmp_2 + b_0$ \\
15. $w_3 \leftarrow tmp_1 \cdot tmp_2$ \\
16. $tmp_1 \leftarrow a_0 + a_1$, $tmp_1 \leftarrow tmp_1 + a_2$, $tmp_2 \leftarrow b_0 + b_1$, $tmp_2 \leftarrow tmp_2 + b_2$ \\
17. $w_2 \leftarrow tmp_1 \cdot tmp_2$ \\
\\
Continued on the next page.\\
\hline
\end{tabular}
\end{center}
\end{small}
\caption{Algorithm mp\_toom\_mul}
\end{figure}

\newpage\begin{figure}[!here]
\begin{small}
\begin{center}
\begin{tabular}{l}
\hline Algorithm \textbf{mp\_toom\_mul} (continued). \\
\textbf{Input}.   mp\_int $a$ and mp\_int $b$ \\
\textbf{Output}.  $c \leftarrow a \cdot  b $ \\
\hline \\
Now solve the system of equations. \\
18. $w_1 \leftarrow w_4 - w_1$, $w_3 \leftarrow w_3 - w_0$ \\
19. $w_1 \leftarrow \lfloor w_1 / 2 \rfloor$, $w_3 \leftarrow \lfloor w_3 / 2 \rfloor$ \\
20. $w_2 \leftarrow w_2 - w_0$, $w_2 \leftarrow w_2 - w_4$ \\
21. $w_1 \leftarrow w_1 - w_2$, $w_3 \leftarrow w_3 - w_2$ \\
22. $tmp_1 \leftarrow 8 \cdot w_0$, $w_1 \leftarrow w_1 - tmp_1$, $tmp_1 \leftarrow 8 \cdot w_4$, $w_3 \leftarrow w_3 - tmp_1$ \\
23. $w_2 \leftarrow 3 \cdot w_2$, $w_2 \leftarrow w_2 - w_1$, $w_2 \leftarrow w_2 - w_3$ \\
24. $w_1 \leftarrow w_1 - w_2$, $w_3 \leftarrow w_3 - w_2$ \\
25. $w_1 \leftarrow \lfloor w_1 / 3 \rfloor, w_3 \leftarrow \lfloor w_3 / 3 \rfloor$ \\
\\
Now substitute $\beta^k$ for $x$ by shifting $w_0, w_1, ..., w_4$. \\
26. for $n$ from $1$ to $4$ do \\
\hspace{3mm}26.1  $w_n \leftarrow w_n \cdot \beta^{nk}$ \\
27. $c \leftarrow w_0 + w_1$, $c \leftarrow c + w_2$, $c \leftarrow c + w_3$, $c \leftarrow c + w_4$ \\
28. Return(\textit{MP\_OKAY}) \\
\hline
\end{tabular}
\end{center}
\end{small}
\caption{Algorithm mp\_toom\_mul (continued)}
\end{figure}

\textbf{Algorithm mp\_toom\_mul.}
This algorithm computes the product of two mp\_int variables $a$ and $b$ using the Toom-Cook approach.  Compared to the Karatsuba multiplication, this 
algorithm has a lower asymptotic running time of approximately $O(n^{1.464})$ but at an obvious cost in overhead.  In this
description, several statements have been compounded to save space.  The intention is that the statements are executed from left to right across
any given step.

The two inputs $a$ and $b$ are first split into three $k$-digit integers $a_0, a_1, a_2$ and $b_0, b_1, b_2$ respectively.  From these smaller
integers the coefficients of the polynomial basis representations $f(x)$ and $g(x)$ are known and can be used to find the relations required.

The first two relations $w_0$ and $w_4$ are the points $\zeta_{0}$ and $\zeta_{\infty}$ respectively.  The relation $w_1, w_2$ and $w_3$ correspond
to the points $16 \cdot \zeta_{1 \over 2}, \zeta_{2}$ and $\zeta_{1}$ respectively.  These are found using logical shifts to independently find
$f(y)$ and $g(y)$ which significantly speeds up the algorithm.

After the five relations $w_0, w_1, \ldots, w_4$ have been computed, the system they represent must be solved in order for the unknown coefficients 
$w_1, w_2$ and $w_3$ to be isolated.  The steps 18 through 25 perform the system reduction required as previously described.  Each step of
the reduction represents the comparable matrix operation that would be performed had this been performed by pencil.  For example, step 18 indicates
that row $1$ must be subtracted from row $4$ and simultaneously row $0$ subtracted from row $3$.  

Once the coeffients have been isolated, the polynomial $W(x) = \sum_{i=0}^{2n} w_i x^i$ is known.  By substituting $\beta^{k}$ for $x$, the integer 
result $a \cdot b$ is produced.

\vspace{+3mm}\begin{small}
\hspace{-5.1mm}{\bf File}: bn\_mp\_toom\_mul.c
\vspace{-3mm}
\begin{alltt}
\end{alltt}
\end{small}

The first obvious thing to note is that this algorithm is complicated.  The complexity is worth it if you are multiplying very 
large numbers.  For example, a 10,000 digit multiplication takes approximaly 99,282,205 fewer single precision multiplications with
Toom--Cook than a Comba or baseline approach (this is a savings of more than 99$\%$).  For most ``crypto'' sized numbers this
algorithm is not practical as Karatsuba has a much lower cutoff point.

First we split $a$ and $b$ into three roughly equal portions.  This has been accomplished (lines 41 to 70) with 
combinations of mp\_rshd() and mp\_mod\_2d() function calls.  At this point $a = a2 \cdot \beta^2 + a1 \cdot \beta + a0$ and similiarly
for $b$.  

Next we compute the five points $w0, w1, w2, w3$ and $w4$.  Recall that $w0$ and $w4$ can be computed directly from the portions so
we get those out of the way first (lines 73 and 78).  Next we compute $w1, w2$ and $w3$ using Horners method.

After this point we solve for the actual values of $w1, w2$ and $w3$ by reducing the $5 \times 5$ system which is relatively
straight forward.  

\subsection{Signed Multiplication}
Now that algorithms to handle multiplications of every useful dimensions have been developed, a rather simple finishing touch is required.  So far all
of the multiplication algorithms have been unsigned multiplications which leaves only a signed multiplication algorithm to be established.  

\begin{figure}[!here]
\begin{small}
\begin{center}
\begin{tabular}{l}
\hline Algorithm \textbf{mp\_mul}. \\
\textbf{Input}.   mp\_int $a$ and mp\_int $b$ \\
\textbf{Output}.  $c \leftarrow a \cdot b$ \\
\hline \\
1.  If $a.sign = b.sign$ then \\
\hspace{3mm}1.1  $sign = MP\_ZPOS$ \\
2.  else \\
\hspace{3mm}2.1  $sign = MP\_ZNEG$ \\
3.  If min$(a.used, b.used) \ge TOOM\_MUL\_CUTOFF$ then  \\
\hspace{3mm}3.1  $c \leftarrow a \cdot b$ using algorithm mp\_toom\_mul \\
4.  else if min$(a.used, b.used) \ge KARATSUBA\_MUL\_CUTOFF$ then \\
\hspace{3mm}4.1  $c \leftarrow a \cdot b$ using algorithm mp\_karatsuba\_mul \\
5.  else \\
\hspace{3mm}5.1  $digs \leftarrow a.used + b.used + 1$ \\
\hspace{3mm}5.2  If $digs < MP\_ARRAY$ and min$(a.used, b.used) \le \delta$ then \\
\hspace{6mm}5.2.1  $c \leftarrow a \cdot b \mbox{ (mod }\beta^{digs}\mbox{)}$ using algorithm fast\_s\_mp\_mul\_digs.  \\
\hspace{3mm}5.3  else \\
\hspace{6mm}5.3.1  $c \leftarrow a \cdot b \mbox{ (mod }\beta^{digs}\mbox{)}$ using algorithm s\_mp\_mul\_digs.  \\
6.  $c.sign \leftarrow sign$ \\
7.  Return the result of the unsigned multiplication performed. \\
\hline
\end{tabular}
\end{center}
\end{small}
\caption{Algorithm mp\_mul}
\end{figure}

\textbf{Algorithm mp\_mul.}
This algorithm performs the signed multiplication of two inputs.  It will make use of any of the three unsigned multiplication algorithms 
available when the input is of appropriate size.  The \textbf{sign} of the result is not set until the end of the algorithm since algorithm
s\_mp\_mul\_digs will clear it.  

\vspace{+3mm}\begin{small}
\hspace{-5.1mm}{\bf File}: bn\_mp\_mul.c
\vspace{-3mm}
\begin{alltt}
\end{alltt}
\end{small}

The implementation is rather simplistic and is not particularly noteworthy.  Line 22 computes the sign of the result using the ``?'' 
operator from the C programming language.  Line 48 computes $\delta$ using the fact that $1 << k$ is equal to $2^k$.  

\section{Squaring}
\label{sec:basesquare}

Squaring is a special case of multiplication where both multiplicands are equal.  At first it may seem like there is no significant optimization
available but in fact there is.  Consider the multiplication of $576$ against $241$.  In total there will be nine single precision multiplications
performed which are $1\cdot 6$, $1 \cdot 7$, $1 \cdot 5$, $4 \cdot 6$, $4 \cdot 7$, $4 \cdot 5$, $2 \cdot  6$, $2 \cdot 7$ and $2 \cdot 5$.  Now consider 
the multiplication of $123$ against $123$.  The nine products are $3 \cdot 3$, $3 \cdot 2$, $3 \cdot 1$, $2 \cdot 3$, $2 \cdot 2$, $2 \cdot 1$, 
$1 \cdot 3$, $1 \cdot 2$ and $1 \cdot 1$.  On closer inspection some of the products are equivalent.  For example, $3 \cdot 2 = 2 \cdot 3$ 
and $3 \cdot 1 = 1 \cdot 3$. 

For any $n$-digit input, there are ${{\left (n^2 + n \right)}\over 2}$ possible unique single precision multiplications required compared to the $n^2$
required for multiplication.  The following diagram gives an example of the operations required.

\begin{figure}[here]
\begin{center}
\begin{tabular}{ccccc|c}
&&1&2&3&\\
$\times$ &&1&2&3&\\
\hline && $3 \cdot 1$ & $3 \cdot 2$ & $3 \cdot 3$ & Row 0\\
       & $2 \cdot 1$  & $2 \cdot 2$ & $2 \cdot 3$ && Row 1 \\
         $1 \cdot 1$  & $1 \cdot 2$ & $1 \cdot 3$ &&& Row 2 \\
\end{tabular}
\end{center}
\caption{Squaring Optimization Diagram}
\end{figure}

Starting from zero and numbering the columns from right to left a very simple pattern becomes obvious.  For the purposes of this discussion let $x$
represent the number being squared.  The first observation is that in row $k$ the $2k$'th column of the product has a $\left (x_k \right)^2$ term in it.  

The second observation is that every column $j$ in row $k$ where $j \ne 2k$ is part of a double product.  Every non-square term of a column will
appear twice hence the name ``double product''.  Every odd column is made up entirely of double products.  In fact every column is made up of double 
products and at most one square (\textit{see the exercise section}).  

The third and final observation is that for row $k$ the first unique non-square term, that is, one that hasn't already appeared in an earlier row, 
occurs at column $2k + 1$.  For example, on row $1$ of the previous squaring, column one is part of the double product with column one from row zero. 
Column two of row one is a square and column three is the first unique column.

\subsection{The Baseline Squaring Algorithm}
The baseline squaring algorithm is meant to be a catch-all squaring algorithm.  It will handle any of the input sizes that the faster routines
will not handle.  

\begin{figure}[!here]
\begin{small}
\begin{center}
\begin{tabular}{l}
\hline Algorithm \textbf{s\_mp\_sqr}. \\
\textbf{Input}.   mp\_int $a$ \\
\textbf{Output}.  $b \leftarrow a^2$ \\
\hline \\
1.  Init a temporary mp\_int of at least $2 \cdot a.used +1$ digits.  (\textit{mp\_init\_size}) \\
2.  If step 1 failed return(\textit{MP\_MEM}) \\
3.  $t.used \leftarrow 2 \cdot a.used + 1$ \\
4.  For $ix$ from 0 to $a.used - 1$ do \\
\hspace{3mm}Calculate the square. \\
\hspace{3mm}4.1  $\hat r \leftarrow t_{2ix} + \left (a_{ix} \right )^2$ \\
\hspace{3mm}4.2  $t_{2ix} \leftarrow \hat r \mbox{ (mod }\beta\mbox{)}$ \\
\hspace{3mm}Calculate the double products after the square. \\
\hspace{3mm}4.3  $u \leftarrow \lfloor \hat r / \beta \rfloor$ \\
\hspace{3mm}4.4  For $iy$ from $ix + 1$ to $a.used - 1$ do \\
\hspace{6mm}4.4.1  $\hat r \leftarrow 2 \cdot a_{ix}a_{iy} + t_{ix + iy} + u$ \\
\hspace{6mm}4.4.2  $t_{ix + iy} \leftarrow \hat r \mbox{ (mod }\beta\mbox{)}$ \\
\hspace{6mm}4.4.3  $u \leftarrow \lfloor \hat r / \beta \rfloor$ \\
\hspace{3mm}Set the last carry. \\
\hspace{3mm}4.5  While $u > 0$ do \\
\hspace{6mm}4.5.1  $iy \leftarrow iy + 1$ \\
\hspace{6mm}4.5.2  $\hat r \leftarrow t_{ix + iy} + u$ \\
\hspace{6mm}4.5.3  $t_{ix + iy} \leftarrow \hat r \mbox{ (mod }\beta\mbox{)}$ \\
\hspace{6mm}4.5.4  $u \leftarrow \lfloor \hat r / \beta \rfloor$ \\
5.  Clamp excess digits of $t$.  (\textit{mp\_clamp}) \\
6.  Exchange $b$ and $t$. \\
7.  Clear $t$ (\textit{mp\_clear}) \\
8.  Return(\textit{MP\_OKAY}) \\
\hline
\end{tabular}
\end{center}
\end{small}
\caption{Algorithm s\_mp\_sqr}
\end{figure}

\textbf{Algorithm s\_mp\_sqr.}
This algorithm computes the square of an input using the three observations on squaring.  It is based fairly faithfully on  algorithm 14.16 of HAC
\cite[pp.596-597]{HAC}.  Similar to algorithm s\_mp\_mul\_digs, a temporary mp\_int is allocated to hold the result of the squaring.  This allows the 
destination mp\_int to be the same as the source mp\_int.

The outer loop of this algorithm begins on step 4. It is best to think of the outer loop as walking down the rows of the partial results, while
the inner loop computes the columns of the partial result.  Step 4.1 and 4.2 compute the square term for each row, and step 4.3 and 4.4 propagate
the carry and compute the double products.  

The requirement that a mp\_word be able to represent the range $0 \le x < 2 \beta^2$ arises from this
very algorithm.  The product $a_{ix}a_{iy}$ will lie in the range $0 \le x \le \beta^2 - 2\beta + 1$ which is obviously less than $\beta^2$ meaning that
when it is multiplied by two, it can be properly represented by a mp\_word.

Similar to algorithm s\_mp\_mul\_digs, after every pass of the inner loop, the destination is correctly set to the sum of all of the partial 
results calculated so far.  This involves expensive carry propagation which will be eliminated in the next algorithm.  

\vspace{+3mm}\begin{small}
\hspace{-5.1mm}{\bf File}: bn\_s\_mp\_sqr.c
\vspace{-3mm}
\begin{alltt}
\end{alltt}
\end{small}

Inside the outer loop (line 34) the square term is calculated on line 37.  The carry (line 44) has been
extracted from the mp\_word accumulator using a right shift.  Aliases for $a_{ix}$ and $t_{ix+iy}$ are initialized 
(lines 47 and 50) to simplify the inner loop.  The doubling is performed using two
additions (line 59) since it is usually faster than shifting, if not at least as fast.  

The important observation is that the inner loop does not begin at $iy = 0$ like for multiplication.  As such the inner loops
get progressively shorter as the algorithm proceeds.  This is what leads to the savings compared to using a multiplication to
square a number. 

\subsection{Faster Squaring by the ``Comba'' Method}
A major drawback to the baseline method is the requirement for single precision shifting inside the $O(n^2)$ nested loop.  Squaring has an additional
drawback that it must double the product inside the inner loop as well.  As for multiplication, the Comba technique can be used to eliminate these
performance hazards.

The first obvious solution is to make an array of mp\_words which will hold all of the columns.  This will indeed eliminate all of the carry
propagation operations from the inner loop.  However, the inner product must still be doubled $O(n^2)$ times.  The solution stems from the simple fact
that $2a + 2b + 2c = 2(a + b + c)$.  That is the sum of all of the double products is equal to double the sum of all the products.  For example,
$ab + ba + ac + ca = 2ab + 2ac = 2(ab + ac)$.  

However, we cannot simply double all of the columns, since the squares appear only once per row.  The most practical solution is to have two 
mp\_word arrays.  One array will hold the squares and the other array will hold the double products.  With both arrays the doubling and 
carry propagation can be moved to a $O(n)$ work level outside the $O(n^2)$ level.  In this case, we have an even simpler solution in mind.

\newpage\begin{figure}[!here]
\begin{small}
\begin{center}
\begin{tabular}{l}
\hline Algorithm \textbf{fast\_s\_mp\_sqr}. \\
\textbf{Input}.   mp\_int $a$ \\
\textbf{Output}.  $b \leftarrow a^2$ \\
\hline \\
Place an array of \textbf{MP\_WARRAY} mp\_digits named $W$ on the stack. \\
1.  If $b.alloc < 2a.used + 1$ then grow $b$ to $2a.used + 1$ digits.  (\textit{mp\_grow}). \\
2.  If step 1 failed return(\textit{MP\_MEM}). \\
\\
3.  $pa \leftarrow 2 \cdot a.used$ \\
4.  $\hat W1 \leftarrow 0$ \\
5.  for $ix$ from $0$ to $pa - 1$ do \\
\hspace{3mm}5.1  $\_ \hat W \leftarrow 0$ \\
\hspace{3mm}5.2  $ty \leftarrow \mbox{MIN}(a.used - 1, ix)$ \\
\hspace{3mm}5.3  $tx \leftarrow ix - ty$ \\
\hspace{3mm}5.4  $iy \leftarrow \mbox{MIN}(a.used - tx, ty + 1)$ \\
\hspace{3mm}5.5  $iy \leftarrow \mbox{MIN}(iy, \lfloor \left (ty - tx + 1 \right )/2 \rfloor)$ \\
\hspace{3mm}5.6  for $iz$ from $0$ to $iz - 1$ do \\
\hspace{6mm}5.6.1  $\_ \hat W \leftarrow \_ \hat W + a_{tx + iz}a_{ty - iz}$ \\
\hspace{3mm}5.7  $\_ \hat W \leftarrow 2 \cdot \_ \hat W  + \hat W1$ \\
\hspace{3mm}5.8  if $ix$ is even then \\
\hspace{6mm}5.8.1  $\_ \hat W \leftarrow \_ \hat W + \left ( a_{\lfloor ix/2 \rfloor}\right )^2$ \\
\hspace{3mm}5.9  $W_{ix} \leftarrow \_ \hat W (\mbox{mod }\beta)$ \\
\hspace{3mm}5.10  $\hat W1 \leftarrow \lfloor \_ \hat W / \beta \rfloor$ \\
\\
6.  $oldused \leftarrow b.used$ \\
7.  $b.used \leftarrow 2 \cdot a.used$ \\
8.  for $ix$ from $0$ to $pa - 1$ do \\
\hspace{3mm}8.1  $b_{ix} \leftarrow W_{ix}$ \\
9.  for $ix$ from $pa$ to $oldused - 1$ do \\
\hspace{3mm}9.1  $b_{ix} \leftarrow 0$ \\
10.  Clamp excess digits from $b$.  (\textit{mp\_clamp}) \\
11.  Return(\textit{MP\_OKAY}). \\ 
\hline
\end{tabular}
\end{center}
\end{small}
\caption{Algorithm fast\_s\_mp\_sqr}
\end{figure}

\textbf{Algorithm fast\_s\_mp\_sqr.}
This algorithm computes the square of an input using the Comba technique.  It is designed to be a replacement for algorithm 
s\_mp\_sqr when the number of input digits is less than \textbf{MP\_WARRAY} and less than $\delta \over 2$.  
This algorithm is very similar to the Comba multiplier except with a few key differences we shall make note of.

First, we have an accumulator and carry variables $\_ \hat W$ and $\hat W1$ respectively.  This is because the inner loop
products are to be doubled.  If we had added the previous carry in we would be doubling too much.  Next we perform an
addition MIN condition on $iy$ (step 5.5) to prevent overlapping digits.  For example, $a_3 \cdot a_5$ is equal
$a_5 \cdot a_3$.  Whereas in the multiplication case we would have $5 < a.used$ and $3 \ge 0$ is maintained since we double the sum
of the products just outside the inner loop we have to avoid doing this.  This is also a good thing since we perform
fewer multiplications and the routine ends up being faster.

Finally the last difference is the addition of the ``square'' term outside the inner loop (step 5.8).  We add in the square
only to even outputs and it is the square of the term at the $\lfloor ix / 2 \rfloor$ position.

\vspace{+3mm}\begin{small}
\hspace{-5.1mm}{\bf File}: bn\_fast\_s\_mp\_sqr.c
\vspace{-3mm}
\begin{alltt}
\end{alltt}
\end{small}

This implementation is essentially a copy of Comba multiplication with the appropriate changes added to make it faster for 
the special case of squaring.  

\subsection{Polynomial Basis Squaring}
The same algorithm that performs optimal polynomial basis multiplication can be used to perform polynomial basis squaring.  The minor exception
is that $\zeta_y = f(y)g(y)$ is actually equivalent to $\zeta_y = f(y)^2$ since $f(y) = g(y)$.  Instead of performing $2n + 1$
multiplications to find the $\zeta$ relations, squaring operations are performed instead.  

\subsection{Karatsuba Squaring}
Let $f(x) = ax + b$ represent the polynomial basis representation of a number to square.  
Let $h(x) = \left ( f(x) \right )^2$ represent the square of the polynomial.  The Karatsuba equation can be modified to square a 
number with the following equation.

\begin{equation}
h(x) = a^2x^2 + \left ((a + b)^2 - (a^2 + b^2) \right )x + b^2
\end{equation}

Upon closer inspection this equation only requires the calculation of three half-sized squares: $a^2$, $b^2$ and $(a + b)^2$.  As in 
Karatsuba multiplication, this algorithm can be applied recursively on the input and will achieve an asymptotic running time of 
$O \left ( n^{lg(3)} \right )$.

If the asymptotic times of Karatsuba squaring and multiplication are the same, why not simply use the multiplication algorithm 
instead?  The answer to this arises from the cutoff point for squaring.  As in multiplication there exists a cutoff point, at which the 
time required for a Comba based squaring and a Karatsuba based squaring meet.  Due to the overhead inherent in the Karatsuba method, the cutoff 
point is fairly high.  For example, on an AMD Athlon XP processor with $\beta = 2^{28}$, the cutoff point is around 127 digits.  

Consider squaring a 200 digit number with this technique.  It will be split into two 100 digit halves which are subsequently squared.  
The 100 digit halves will not be squared using Karatsuba, but instead using the faster Comba based squaring algorithm.  If Karatsuba multiplication
were used instead, the 100 digit numbers would be squared with a slower Comba based multiplication.  

\newpage\begin{figure}[!here]
\begin{small}
\begin{center}
\begin{tabular}{l}
\hline Algorithm \textbf{mp\_karatsuba\_sqr}. \\
\textbf{Input}.   mp\_int $a$ \\
\textbf{Output}.  $b \leftarrow a^2$ \\
\hline \\
1.  Initialize the following temporary mp\_ints:  $x0$, $x1$, $t1$, $t2$, $x0x0$ and $x1x1$. \\
2.  If any of the initializations on step 1 failed return(\textit{MP\_MEM}). \\
\\
Split the input.  e.g. $a = x1\beta^B + x0$ \\
3.  $B \leftarrow \lfloor a.used / 2 \rfloor$ \\
4.  $x0 \leftarrow a \mbox{ (mod }\beta^B\mbox{)}$ (\textit{mp\_mod\_2d}) \\
5.  $x1 \leftarrow \lfloor a / \beta^B \rfloor$ (\textit{mp\_lshd}) \\
\\
Calculate the three squares. \\
6.  $x0x0 \leftarrow x0^2$ (\textit{mp\_sqr}) \\
7.  $x1x1 \leftarrow x1^2$ \\
8.  $t1 \leftarrow x1 + x0$ (\textit{s\_mp\_add}) \\
9.  $t1 \leftarrow t1^2$ \\
\\
Compute the middle term. \\
10.  $t2 \leftarrow x0x0 + x1x1$ (\textit{s\_mp\_add}) \\
11.  $t1 \leftarrow t1 - t2$ \\
\\
Compute final product. \\
12.  $t1 \leftarrow t1\beta^B$ (\textit{mp\_lshd}) \\
13.  $x1x1 \leftarrow x1x1\beta^{2B}$ \\
14.  $t1 \leftarrow t1 + x0x0$ \\
15.  $b \leftarrow t1 + x1x1$ \\
16.  Return(\textit{MP\_OKAY}). \\
\hline
\end{tabular}
\end{center}
\end{small}
\caption{Algorithm mp\_karatsuba\_sqr}
\end{figure}

\textbf{Algorithm mp\_karatsuba\_sqr.}
This algorithm computes the square of an input $a$ using the Karatsuba technique.  This algorithm is very similar to the Karatsuba based
multiplication algorithm with the exception that the three half-size multiplications have been replaced with three half-size squarings.

The radix point for squaring is simply placed exactly in the middle of the digits when the input has an odd number of digits, otherwise it is
placed just below the middle.  Step 3, 4 and 5 compute the two halves required using $B$
as the radix point.  The first two squares in steps 6 and 7 are rather straightforward while the last square is of a more compact form.

By expanding $\left (x1 + x0 \right )^2$, the $x1^2$ and $x0^2$ terms in the middle disappear, that is $(x0 - x1)^2 - (x1^2 + x0^2)  = 2 \cdot x0 \cdot x1$.
Now if $5n$ single precision additions and a squaring of $n$-digits is faster than multiplying two $n$-digit numbers and doubling then
this method is faster.  Assuming no further recursions occur, the difference can be estimated with the following inequality.

Let $p$ represent the cost of a single precision addition and $q$ the cost of a single precision multiplication both in terms of time\footnote{Or
machine clock cycles.}. 

\begin{equation}
5pn +{{q(n^2 + n)} \over 2} \le pn + qn^2
\end{equation}

For example, on an AMD Athlon XP processor $p = {1 \over 3}$ and $q = 6$.  This implies that the following inequality should hold.
\begin{center}
\begin{tabular}{rcl}
${5n \over 3} + 3n^2 + 3n$     & $<$ & ${n \over 3} + 6n^2$ \\
${5 \over 3} + 3n + 3$     & $<$ & ${1 \over 3} + 6n$ \\
${13 \over 9}$     & $<$ & $n$ \\
\end{tabular}
\end{center}

This results in a cutoff point around $n = 2$.  As a consequence it is actually faster to compute the middle term the ``long way'' on processors
where multiplication is substantially slower\footnote{On the Athlon there is a 1:17 ratio between clock cycles for addition and multiplication.  On
the Intel P4 processor this ratio is 1:29 making this method even more beneficial.  The only common exception is the ARMv4 processor which has a
ratio of 1:7.  } than simpler operations such as addition.  

\vspace{+3mm}\begin{small}
\hspace{-5.1mm}{\bf File}: bn\_mp\_karatsuba\_sqr.c
\vspace{-3mm}
\begin{alltt}
\end{alltt}
\end{small}

This implementation is largely based on the implementation of algorithm mp\_karatsuba\_mul.  It uses the same inline style to copy and 
shift the input into the two halves.  The loop from line 54 to line 70 has been modified since only one input exists.  The \textbf{used}
count of both $x0$ and $x1$ is fixed up and $x0$ is clamped before the calculations begin.  At this point $x1$ and $x0$ are valid equivalents
to the respective halves as if mp\_rshd and mp\_mod\_2d had been used.  

By inlining the copy and shift operations the cutoff point for Karatsuba multiplication can be lowered.  On the Athlon the cutoff point
is exactly at the point where Comba squaring can no longer be used (\textit{128 digits}).  On slower processors such as the Intel P4
it is actually below the Comba limit (\textit{at 110 digits}).

This routine uses the same error trap coding style as mp\_karatsuba\_sqr.  As the temporary variables are initialized errors are 
redirected to the error trap higher up.  If the algorithm completes without error the error code is set to \textbf{MP\_OKAY} and 
mp\_clears are executed normally.

\subsection{Toom-Cook Squaring}
The Toom-Cook squaring algorithm mp\_toom\_sqr is heavily based on the algorithm mp\_toom\_mul with the exception that squarings are used
instead of multiplication to find the five relations.  The reader is encouraged to read the description of the latter algorithm and try to 
derive their own Toom-Cook squaring algorithm.  

\subsection{High Level Squaring}
\newpage\begin{figure}[!here]
\begin{small}
\begin{center}
\begin{tabular}{l}
\hline Algorithm \textbf{mp\_sqr}. \\
\textbf{Input}.   mp\_int $a$ \\
\textbf{Output}.  $b \leftarrow a^2$ \\
\hline \\
1.  If $a.used \ge TOOM\_SQR\_CUTOFF$ then  \\
\hspace{3mm}1.1  $b \leftarrow a^2$ using algorithm mp\_toom\_sqr \\
2.  else if $a.used \ge KARATSUBA\_SQR\_CUTOFF$ then \\
\hspace{3mm}2.1  $b \leftarrow a^2$ using algorithm mp\_karatsuba\_sqr \\
3.  else \\
\hspace{3mm}3.1  $digs \leftarrow a.used + b.used + 1$ \\
\hspace{3mm}3.2  If $digs < MP\_ARRAY$ and $a.used \le \delta$ then \\
\hspace{6mm}3.2.1  $b \leftarrow a^2$ using algorithm fast\_s\_mp\_sqr.  \\
\hspace{3mm}3.3  else \\
\hspace{6mm}3.3.1  $b \leftarrow a^2$ using algorithm s\_mp\_sqr.  \\
4.  $b.sign \leftarrow MP\_ZPOS$ \\
5.  Return the result of the unsigned squaring performed. \\
\hline
\end{tabular}
\end{center}
\end{small}
\caption{Algorithm mp\_sqr}
\end{figure}

\textbf{Algorithm mp\_sqr.}
This algorithm computes the square of the input using one of four different algorithms.  If the input is very large and has at least
\textbf{TOOM\_SQR\_CUTOFF} or \textbf{KARATSUBA\_SQR\_CUTOFF} digits then either the Toom-Cook or the Karatsuba Squaring algorithm is used.  If
neither of the polynomial basis algorithms should be used then either the Comba or baseline algorithm is used.  

\vspace{+3mm}\begin{small}
\hspace{-5.1mm}{\bf File}: bn\_mp\_sqr.c
\vspace{-3mm}
\begin{alltt}
\end{alltt}
\end{small}

\section*{Exercises}
\begin{tabular}{cl}
$\left [ 3 \right ] $ & Devise an efficient algorithm for selection of the radix point to handle inputs \\
                      & that have different number of digits in Karatsuba multiplication. \\
                      & \\
$\left [ 2 \right ] $ & In section 5.3 the fact that every column of a squaring is made up \\
                      & of double products and at most one square is stated.  Prove this statement. \\
                      & \\                      
$\left [ 3 \right ] $ & Prove the equation for Karatsuba squaring. \\
                      & \\
$\left [ 1 \right ] $ & Prove that Karatsuba squaring requires $O \left (n^{lg(3)} \right )$ time. \\
                      & \\ 
$\left [ 2 \right ] $ & Determine the minimal ratio between addition and multiplication clock cycles \\
                      & required for equation $6.7$ to be true.  \\
                      & \\
$\left [ 3 \right ] $ & Implement a threaded version of Comba multiplication (and squaring) where you \\
                      & compute subsets of the columns in each thread.  Determine a cutoff point where \\
                      & it is effective and add the logic to mp\_mul() and mp\_sqr(). \\
                      &\\
$\left [ 4 \right ] $ & Same as the previous but also modify the Karatsuba and Toom-Cook.  You must \\
                      & increase the throughput of mp\_exptmod() for random odd moduli in the range \\
                      & $512 \ldots 4096$ bits significantly ($> 2x$) to complete this challenge. \\
                      & \\
\end{tabular}

\chapter{Modular Reduction}
\section{Basics of Modular Reduction}
\index{modular residue}
Modular reduction is an operation that arises quite often within public key cryptography algorithms and various number theoretic algorithms, 
such as factoring.  Modular reduction algorithms are the third class of algorithms of the ``multipliers'' set.  A number $a$ is said to be \textit{reduced}
modulo another number $b$ by finding the remainder of the division $a/b$.  Full integer division with remainder is a topic to be covered 
in~\ref{sec:division}.

Modular reduction is equivalent to solving for $r$ in the following equation.  $a = bq + r$ where $q = \lfloor a/b \rfloor$.  The result 
$r$ is said to be ``congruent to $a$ modulo $b$'' which is also written as $r \equiv a \mbox{ (mod }b\mbox{)}$.  In other vernacular $r$ is known as the 
``modular residue'' which leads to ``quadratic residue''\footnote{That's fancy talk for $b \equiv a^2 \mbox{ (mod }p\mbox{)}$.} and
other forms of residues.  

Modular reductions are normally used to create either finite groups, rings or fields.  The most common usage for performance driven modular reductions 
is in modular exponentiation algorithms.  That is to compute $d = a^b \mbox{ (mod }c\mbox{)}$ as fast as possible.  This operation is used in the 
RSA and Diffie-Hellman public key algorithms, for example.  Modular multiplication and squaring also appears as a fundamental operation in 
elliptic curve cryptographic algorithms.  As will be discussed in the subsequent chapter there exist fast algorithms for computing modular 
exponentiations without having to perform (\textit{in this example}) $b - 1$ multiplications.  These algorithms will produce partial results in the 
range $0 \le x < c^2$ which can be taken advantage of to create several efficient algorithms.   They have also been used to create redundancy check 
algorithms known as CRCs, error correction codes such as Reed-Solomon and solve a variety of number theoeretic problems.  

\section{The Barrett Reduction}
The Barrett reduction algorithm \cite{BARRETT} was inspired by fast division algorithms which multiply by the reciprocal to emulate
division.  Barretts observation was that the residue $c$ of $a$ modulo $b$ is equal to 

\begin{equation}
c = a - b \cdot \lfloor a/b \rfloor
\end{equation}

Since algorithms such as modular exponentiation would be using the same modulus extensively, typical DSP\footnote{It is worth noting that Barrett's paper 
targeted the DSP56K processor.}  intuition would indicate the next step would be to replace $a/b$ by a multiplication by the reciprocal.  However, 
DSP intuition on its own will not work as these numbers are considerably larger than the precision of common DSP floating point data types.  
It would take another common optimization to optimize the algorithm.

\subsection{Fixed Point Arithmetic}
The trick used to optimize the above equation is based on a technique of emulating floating point data types with fixed precision integers.  Fixed
point arithmetic would become very popular as it greatly optimize the ``3d-shooter'' genre of games in the mid 1990s when floating point units were 
fairly slow if not unavailable.   The idea behind fixed point arithmetic is to take a normal $k$-bit integer data type and break it into $p$-bit 
integer and a $q$-bit fraction part (\textit{where $p+q = k$}).  

In this system a $k$-bit integer $n$ would actually represent $n/2^q$.  For example, with $q = 4$ the integer $n = 37$ would actually represent the
value $2.3125$.  To multiply two fixed point numbers the integers are multiplied using traditional arithmetic and subsequently normalized by 
moving the implied decimal point back to where it should be.  For example, with $q = 4$ to multiply the integers $9$ and $5$ they must be converted 
to fixed point first by multiplying by $2^q$.  Let $a = 9(2^q)$ represent the fixed point representation of $9$ and $b = 5(2^q)$ represent the 
fixed point representation of $5$.  The product $ab$ is equal to $45(2^{2q})$ which when normalized by dividing by $2^q$ produces $45(2^q)$.  

This technique became popular since a normal integer multiplication and logical shift right are the only required operations to perform a multiplication
of two fixed point numbers.  Using fixed point arithmetic, division can be easily approximated by multiplying by the reciprocal.  If $2^q$ is 
equivalent to one than $2^q/b$ is equivalent to the fixed point approximation of $1/b$ using real arithmetic.  Using this fact dividing an integer 
$a$ by another integer $b$ can be achieved with the following expression.

\begin{equation}
\lfloor a / b \rfloor \mbox{ }\approx\mbox{ } \lfloor (a \cdot \lfloor 2^q / b \rfloor)/2^q \rfloor
\end{equation}

The precision of the division is proportional to the value of $q$.  If the divisor $b$ is used frequently as is the case with 
modular exponentiation pre-computing $2^q/b$ will allow a division to be performed with a multiplication and a right shift.  Both operations
are considerably faster than division on most processors.  

Consider dividing $19$ by $5$.  The correct result is $\lfloor 19/5 \rfloor = 3$.  With $q = 3$ the reciprocal is $\lfloor 2^q/5 \rfloor = 1$ which
leads to a product of $19$ which when divided by $2^q$ produces $2$.  However, with $q = 4$ the reciprocal is $\lfloor 2^q/5 \rfloor = 3$ and
the result of the emulated division is $\lfloor 3 \cdot 19 / 2^q \rfloor = 3$ which is correct.  The value of $2^q$ must be close to or ideally
larger than the dividend.  In effect if $a$ is the dividend then $q$ should allow $0 \le \lfloor a/2^q \rfloor \le 1$ in order for this approach
to work correctly.  Plugging this form of divison into the original equation the following modular residue equation arises.

\begin{equation}
c = a - b \cdot \lfloor (a \cdot \lfloor 2^q / b \rfloor)/2^q \rfloor
\end{equation}

Using the notation from \cite{BARRETT} the value of $\lfloor 2^q / b \rfloor$ will be represented by the $\mu$ symbol.  Using the $\mu$
variable also helps re-inforce the idea that it is meant to be computed once and re-used.

\begin{equation}
c = a - b \cdot \lfloor (a \cdot \mu)/2^q \rfloor
\end{equation}

Provided that $2^q \ge a$ this algorithm will produce a quotient that is either exactly correct or off by a value of one.  In the context of Barrett
reduction the value of $a$ is bound by $0 \le a \le (b - 1)^2$ meaning that $2^q \ge b^2$ is sufficient to ensure the reciprocal will have enough
precision.  

Let $n$ represent the number of digits in $b$.  This algorithm requires approximately $2n^2$ single precision multiplications to produce the quotient and 
another $n^2$ single precision multiplications to find the residue.  In total $3n^2$ single precision multiplications are required to 
reduce the number.  

For example, if $b = 1179677$ and $q = 41$ ($2^q > b^2$), then the reciprocal $\mu$ is equal to $\lfloor 2^q / b \rfloor = 1864089$.  Consider reducing
$a = 180388626447$ modulo $b$ using the above reduction equation.  The quotient using the new formula is $\lfloor (a \cdot \mu) / 2^q \rfloor = 152913$.
By subtracting $152913b$ from $a$ the correct residue $a \equiv 677346 \mbox{ (mod }b\mbox{)}$ is found.

\subsection{Choosing a Radix Point}
Using the fixed point representation a modular reduction can be performed with $3n^2$ single precision multiplications.  If that were the best
that could be achieved a full division\footnote{A division requires approximately $O(2cn^2)$ single precision multiplications for a small value of $c$.  
See~\ref{sec:division} for further details.} might as well be used in its place.  The key to optimizing the reduction is to reduce the precision of
the initial multiplication that finds the quotient.  

Let $a$ represent the number of which the residue is sought.  Let $b$ represent the modulus used to find the residue.  Let $m$ represent
the number of digits in $b$.  For the purposes of this discussion we will assume that the number of digits in $a$ is $2m$, which is generally true if 
two $m$-digit numbers have been multiplied.  Dividing $a$ by $b$ is the same as dividing a $2m$ digit integer by a $m$ digit integer.  Digits below the 
$m - 1$'th digit of $a$ will contribute at most a value of $1$ to the quotient because $\beta^k < b$ for any $0 \le k \le m - 1$.  Another way to
express this is by re-writing $a$ as two parts.  If $a' \equiv a \mbox{ (mod }b^m\mbox{)}$ and $a'' = a - a'$ then 
${a \over b} \equiv {{a' + a''} \over b}$ which is equivalent to ${a' \over b} + {a'' \over b}$.  Since $a'$ is bound to be less than $b$ the quotient
is bound by $0 \le {a' \over b} < 1$.

Since the digits of $a'$ do not contribute much to the quotient the observation is that they might as well be zero.  However, if the digits 
``might as well be zero'' they might as well not be there in the first place.  Let $q_0 = \lfloor a/\beta^{m-1} \rfloor$ represent the input
with the irrelevant digits trimmed.  Now the modular reduction is trimmed to the almost equivalent equation

\begin{equation}
c = a - b \cdot \lfloor (q_0 \cdot \mu) / \beta^{m+1} \rfloor
\end{equation}

Note that the original divisor $2^q$ has been replaced with $\beta^{m+1}$ where in this case $q$ is a multiple of $lg(\beta)$. Also note that the 
exponent on the divisor when added to the amount $q_0$ was shifted by equals $2m$.  If the optimization had not been performed the divisor 
would have the exponent $2m$ so in the end the exponents do ``add up''. Using the above equation the quotient 
$\lfloor (q_0 \cdot \mu) / \beta^{m+1} \rfloor$ can be off from the true quotient by at most two.  The original fixed point quotient can be off
by as much as one (\textit{provided the radix point is chosen suitably}) and now that the lower irrelevent digits have been trimmed the quotient
can be off by an additional value of one for a total of at most two.  This implies that 
$0 \le a - b \cdot \lfloor (q_0 \cdot \mu) / \beta^{m+1} \rfloor < 3b$.  By first subtracting $b$ times the quotient and then conditionally subtracting 
$b$ once or twice the residue is found.

The quotient is now found using $(m + 1)(m) = m^2 + m$ single precision multiplications and the residue with an additional $m^2$ single
precision multiplications, ignoring the subtractions required.  In total $2m^2 + m$ single precision multiplications are required to find the residue.  
This is considerably faster than the original attempt.

For example, let $\beta = 10$ represent the radix of the digits.  Let $b = 9999$ represent the modulus which implies $m = 4$. Let $a = 99929878$ 
represent the value of which the residue is desired.  In this case $q = 8$ since $10^7 < 9999^2$ meaning that $\mu = \lfloor \beta^{q}/b \rfloor = 10001$.  
With the new observation the multiplicand for the quotient is equal to $q_0 = \lfloor a / \beta^{m - 1} \rfloor = 99929$.  The quotient is then 
$\lfloor (q_0 \cdot \mu) / \beta^{m+1} \rfloor = 9993$.  Subtracting $9993b$ from $a$ and the correct residue $a \equiv 9871 \mbox{ (mod }b\mbox{)}$ 
is found.  

\subsection{Trimming the Quotient}
So far the reduction algorithm has been optimized from $3m^2$ single precision multiplications down to $2m^2 + m$ single precision multiplications.  As 
it stands now the algorithm is already fairly fast compared to a full integer division algorithm.  However, there is still room for
optimization.  

After the first multiplication inside the quotient ($q_0 \cdot \mu$) the value is shifted right by $m + 1$ places effectively nullifying the lower
half of the product.  It would be nice to be able to remove those digits from the product to effectively cut down the number of single precision 
multiplications.  If the number of digits in the modulus $m$ is far less than $\beta$ a full product is not required for the algorithm to work properly.  
In fact the lower $m - 2$ digits will not affect the upper half of the product at all and do not need to be computed.  

The value of $\mu$ is a $m$-digit number and $q_0$ is a $m + 1$ digit number.  Using a full multiplier $(m + 1)(m) = m^2 + m$ single precision
multiplications would be required.  Using a multiplier that will only produce digits at and above the $m - 1$'th digit reduces the number
of single precision multiplications to ${m^2 + m} \over 2$ single precision multiplications.  

\subsection{Trimming the Residue}
After the quotient has been calculated it is used to reduce the input.  As previously noted the algorithm is not exact and it can be off by a small
multiple of the modulus, that is $0 \le a - b \cdot \lfloor (q_0 \cdot \mu) / \beta^{m+1} \rfloor < 3b$.  If $b$ is $m$ digits than the 
result of reduction equation is a value of at most $m + 1$ digits (\textit{provided $3 < \beta$}) implying that the upper $m - 1$ digits are
implicitly zero.  

The next optimization arises from this very fact.  Instead of computing $b \cdot \lfloor (q_0 \cdot \mu) / \beta^{m+1} \rfloor$ using a full
$O(m^2)$ multiplication algorithm only the lower $m+1$ digits of the product have to be computed.  Similarly the value of $a$ can
be reduced modulo $\beta^{m+1}$ before the multiple of $b$ is subtracted which simplifes the subtraction as well.  A multiplication that produces 
only the lower $m+1$ digits requires ${m^2 + 3m - 2} \over 2$ single precision multiplications.  

With both optimizations in place the algorithm is the algorithm Barrett proposed.  It requires $m^2 + 2m - 1$ single precision multiplications which
is considerably faster than the straightforward $3m^2$ method.  

\subsection{The Barrett Algorithm}
\newpage\begin{figure}[!here]
\begin{small}
\begin{center}
\begin{tabular}{l}
\hline Algorithm \textbf{mp\_reduce}. \\
\textbf{Input}.   mp\_int $a$, mp\_int $b$ and $\mu = \lfloor \beta^{2m}/b \rfloor, m = \lceil lg_{\beta}(b) \rceil, (0 \le a < b^2, b > 1)$ \\
\textbf{Output}.  $a \mbox{ (mod }b\mbox{)}$ \\
\hline \\
Let $m$ represent the number of digits in $b$.  \\
1.  Make a copy of $a$ and store it in $q$.  (\textit{mp\_init\_copy}) \\
2.  $q \leftarrow \lfloor q / \beta^{m - 1} \rfloor$ (\textit{mp\_rshd}) \\
\\
Produce the quotient. \\
3.  $q \leftarrow q \cdot \mu$  (\textit{note: only produce digits at or above $m-1$}) \\
4.  $q \leftarrow \lfloor q / \beta^{m + 1} \rfloor$ \\
\\
Subtract the multiple of modulus from the input. \\
5.  $a \leftarrow a \mbox{ (mod }\beta^{m+1}\mbox{)}$ (\textit{mp\_mod\_2d}) \\
6.  $q \leftarrow q \cdot b \mbox{ (mod }\beta^{m+1}\mbox{)}$ (\textit{s\_mp\_mul\_digs}) \\
7.  $a \leftarrow a - q$ (\textit{mp\_sub}) \\
\\
Add $\beta^{m+1}$ if a carry occured. \\
8.  If $a < 0$ then (\textit{mp\_cmp\_d}) \\
\hspace{3mm}8.1  $q \leftarrow 1$ (\textit{mp\_set}) \\
\hspace{3mm}8.2  $q \leftarrow q \cdot \beta^{m+1}$ (\textit{mp\_lshd}) \\
\hspace{3mm}8.3  $a \leftarrow a + q$ \\
\\
Now subtract the modulus if the residue is too large (e.g. quotient too small). \\
9.  While $a \ge b$ do (\textit{mp\_cmp}) \\
\hspace{3mm}9.1  $c \leftarrow a - b$ \\
10.  Clear $q$. \\
11.  Return(\textit{MP\_OKAY}) \\
\hline
\end{tabular}
\end{center}
\end{small}
\caption{Algorithm mp\_reduce}
\end{figure}

\textbf{Algorithm mp\_reduce.}
This algorithm will reduce the input $a$ modulo $b$ in place using the Barrett algorithm.  It is loosely based on algorithm 14.42 of HAC
\cite[pp.  602]{HAC} which is based on the paper from Paul Barrett \cite{BARRETT}.  The algorithm has several restrictions and assumptions which must 
be adhered to for the algorithm to work.

First the modulus $b$ is assumed to be positive and greater than one.  If the modulus were less than or equal to one than subtracting
a multiple of it would either accomplish nothing or actually enlarge the input.  The input $a$ must be in the range $0 \le a < b^2$ in order
for the quotient to have enough precision.  If $a$ is the product of two numbers that were already reduced modulo $b$, this will not be a problem.
Technically the algorithm will still work if $a \ge b^2$ but it will take much longer to finish.  The value of $\mu$ is passed as an argument to this 
algorithm and is assumed to be calculated and stored before the algorithm is used.  

Recall that the multiplication for the quotient on step 3 must only produce digits at or above the $m-1$'th position.  An algorithm called 
$s\_mp\_mul\_high\_digs$ which has not been presented is used to accomplish this task.  The algorithm is based on $s\_mp\_mul\_digs$ except that
instead of stopping at a given level of precision it starts at a given level of precision.  This optimal algorithm can only be used if the number
of digits in $b$ is very much smaller than $\beta$.  

While it is known that 
$a \ge b \cdot \lfloor (q_0 \cdot \mu) / \beta^{m+1} \rfloor$ only the lower $m+1$ digits are being used to compute the residue, so an implied 
``borrow'' from the higher digits might leave a negative result.  After the multiple of the modulus has been subtracted from $a$ the residue must be 
fixed up in case it is negative.  The invariant $\beta^{m+1}$ must be added to the residue to make it positive again.  

The while loop at step 9 will subtract $b$ until the residue is less than $b$.  If the algorithm is performed correctly this step is 
performed at most twice, and on average once. However, if $a \ge b^2$ than it will iterate substantially more times than it should.

\vspace{+3mm}\begin{small}
\hspace{-5.1mm}{\bf File}: bn\_mp\_reduce.c
\vspace{-3mm}
\begin{alltt}
\end{alltt}
\end{small}

The first multiplication that determines the quotient can be performed by only producing the digits from $m - 1$ and up.  This essentially halves
the number of single precision multiplications required.  However, the optimization is only safe if $\beta$ is much larger than the number of digits
in the modulus.  In the source code this is evaluated on lines 36 to 44 where algorithm s\_mp\_mul\_high\_digs is used when it is
safe to do so.  

\subsection{The Barrett Setup Algorithm}
In order to use algorithm mp\_reduce the value of $\mu$ must be calculated in advance.  Ideally this value should be computed once and stored for
future use so that the Barrett algorithm can be used without delay.  

\newpage\begin{figure}[!here]
\begin{small}
\begin{center}
\begin{tabular}{l}
\hline Algorithm \textbf{mp\_reduce\_setup}. \\
\textbf{Input}.   mp\_int $a$ ($a > 1$)  \\
\textbf{Output}.  $\mu \leftarrow \lfloor \beta^{2m}/a \rfloor$ \\
\hline \\
1.  $\mu \leftarrow 2^{2 \cdot lg(\beta) \cdot  m}$ (\textit{mp\_2expt}) \\
2.  $\mu \leftarrow \lfloor \mu / b \rfloor$ (\textit{mp\_div}) \\
3.  Return(\textit{MP\_OKAY}) \\
\hline
\end{tabular}
\end{center}
\end{small}
\caption{Algorithm mp\_reduce\_setup}
\end{figure}

\textbf{Algorithm mp\_reduce\_setup.}
This algorithm computes the reciprocal $\mu$ required for Barrett reduction.  First $\beta^{2m}$ is calculated as $2^{2 \cdot lg(\beta) \cdot  m}$ which
is equivalent and much faster.  The final value is computed by taking the integer quotient of $\lfloor \mu / b \rfloor$.

\vspace{+3mm}\begin{small}
\hspace{-5.1mm}{\bf File}: bn\_mp\_reduce\_setup.c
\vspace{-3mm}
\begin{alltt}
\end{alltt}
\end{small}

This simple routine calculates the reciprocal $\mu$ required by Barrett reduction.  Note the extended usage of algorithm mp\_div where the variable
which would received the remainder is passed as NULL.  As will be discussed in~\ref{sec:division} the division routine allows both the quotient and the 
remainder to be passed as NULL meaning to ignore the value.  

\section{The Montgomery Reduction}
Montgomery reduction\footnote{Thanks to Niels Ferguson for his insightful explanation of the algorithm.} \cite{MONT} is by far the most interesting 
form of reduction in common use.  It computes a modular residue which is not actually equal to the residue of the input yet instead equal to a 
residue times a constant.  However, as perplexing as this may sound the algorithm is relatively simple and very efficient.  

Throughout this entire section the variable $n$ will represent the modulus used to form the residue.  As will be discussed shortly the value of
$n$ must be odd.  The variable $x$ will represent the quantity of which the residue is sought.  Similar to the Barrett algorithm the input
is restricted to $0 \le x < n^2$.  To begin the description some simple number theory facts must be established.

\textbf{Fact 1.}  Adding $n$ to $x$ does not change the residue since in effect it adds one to the quotient $\lfloor x / n \rfloor$.  Another way
to explain this is that $n$ is (\textit{or multiples of $n$ are}) congruent to zero modulo $n$.  Adding zero will not change the value of the residue.  

\textbf{Fact 2.}  If $x$ is even then performing a division by two in $\Z$ is congruent to $x \cdot 2^{-1} \mbox{ (mod }n\mbox{)}$.  Actually
this is an application of the fact that if $x$ is evenly divisible by any $k \in \Z$ then division in $\Z$ will be congruent to 
multiplication by $k^{-1}$ modulo $n$.  

From these two simple facts the following simple algorithm can be derived.

\newpage\begin{figure}[!here]
\begin{small}
\begin{center}
\begin{tabular}{l}
\hline Algorithm \textbf{Montgomery Reduction}. \\
\textbf{Input}.   Integer $x$, $n$ and $k$ \\
\textbf{Output}.  $2^{-k}x \mbox{ (mod }n\mbox{)}$ \\
\hline \\
1.  for $t$ from $1$ to $k$ do \\
\hspace{3mm}1.1  If $x$ is odd then \\
\hspace{6mm}1.1.1  $x \leftarrow x + n$ \\
\hspace{3mm}1.2  $x \leftarrow x/2$ \\
2.  Return $x$. \\
\hline
\end{tabular}
\end{center}
\end{small}
\caption{Algorithm Montgomery Reduction}
\end{figure}

The algorithm reduces the input one bit at a time using the two congruencies stated previously.  Inside the loop $n$, which is odd, is
added to $x$ if $x$ is odd.  This forces $x$ to be even which allows the division by two in $\Z$ to be congruent to a modular division by two.  Since
$x$ is assumed to be initially much larger than $n$ the addition of $n$ will contribute an insignificant magnitude to $x$.  Let $r$ represent the 
final result of the Montgomery algorithm.  If $k > lg(n)$ and $0 \le x < n^2$ then the final result is limited to 
$0 \le r < \lfloor x/2^k \rfloor + n$.  As a result at most a single subtraction is required to get the residue desired.

\begin{figure}[here]
\begin{small}
\begin{center}
\begin{tabular}{|c|l|}
\hline \textbf{Step number ($t$)} & \textbf{Result ($x$)} \\
\hline $1$ & $x + n = 5812$, $x/2 = 2906$ \\
\hline $2$ & $x/2 = 1453$ \\
\hline $3$ & $x + n = 1710$, $x/2 = 855$ \\
\hline $4$ & $x + n = 1112$, $x/2 = 556$ \\
\hline $5$ & $x/2 = 278$ \\
\hline $6$ & $x/2 = 139$ \\
\hline $7$ & $x + n = 396$, $x/2 = 198$ \\
\hline $8$ & $x/2 = 99$ \\
\hline $9$ & $x + n = 356$, $x/2 = 178$ \\
\hline
\end{tabular}
\end{center}
\end{small}
\caption{Example of Montgomery Reduction (I)}
\label{fig:MONT1}
\end{figure}

Consider the example in figure~\ref{fig:MONT1} which reduces $x = 5555$ modulo $n = 257$ when $k = 9$ (note $\beta^k = 512$ which is larger than $n$).  The result of 
the algorithm $r = 178$ is congruent to the value of $2^{-9} \cdot 5555 \mbox{ (mod }257\mbox{)}$.  When $r$ is multiplied by $2^9$ modulo $257$ the correct residue 
$r \equiv 158$ is produced.  

Let $k = \lfloor lg(n) \rfloor + 1$ represent the number of bits in $n$.  The current algorithm requires $2k^2$ single precision shifts
and $k^2$ single precision additions.  At this rate the algorithm is most certainly slower than Barrett reduction and not terribly useful.  
Fortunately there exists an alternative representation of the algorithm.

\begin{figure}[!here]
\begin{small}
\begin{center}
\begin{tabular}{l}
\hline Algorithm \textbf{Montgomery Reduction} (modified I). \\
\textbf{Input}.   Integer $x$, $n$ and $k$ ($2^k > n$) \\
\textbf{Output}.  $2^{-k}x \mbox{ (mod }n\mbox{)}$ \\
\hline \\
1.  for $t$ from $1$ to $k$ do \\
\hspace{3mm}1.1  If the $t$'th bit of $x$ is one then \\
\hspace{6mm}1.1.1  $x \leftarrow x + 2^tn$ \\
2.  Return $x/2^k$. \\
\hline
\end{tabular}
\end{center}
\end{small}
\caption{Algorithm Montgomery Reduction (modified I)}
\end{figure}

This algorithm is equivalent since $2^tn$ is a multiple of $n$ and the lower $k$ bits of $x$ are zero by step 2.  The number of single
precision shifts has now been reduced from $2k^2$ to $k^2 + k$ which is only a small improvement.

\begin{figure}[here]
\begin{small}
\begin{center}
\begin{tabular}{|c|l|r|}
\hline \textbf{Step number ($t$)} & \textbf{Result ($x$)} & \textbf{Result ($x$) in Binary} \\
\hline -- & $5555$ & $1010110110011$ \\
\hline $1$ & $x + 2^{0}n = 5812$ &  $1011010110100$ \\
\hline $2$ & $5812$ & $1011010110100$ \\
\hline $3$ & $x + 2^{2}n = 6840$ & $1101010111000$ \\
\hline $4$ & $x + 2^{3}n = 8896$ & $10001011000000$ \\
\hline $5$ & $8896$ & $10001011000000$ \\
\hline $6$ & $8896$ & $10001011000000$ \\
\hline $7$ & $x + 2^{6}n = 25344$ & $110001100000000$ \\
\hline $8$ & $25344$ & $110001100000000$ \\
\hline $9$ & $x + 2^{7}n = 91136$ & $10110010000000000$ \\
\hline -- & $x/2^k = 178$ & \\
\hline
\end{tabular}
\end{center}
\end{small}
\caption{Example of Montgomery Reduction (II)}
\label{fig:MONT2}
\end{figure}

Figure~\ref{fig:MONT2} demonstrates the modified algorithm reducing $x = 5555$ modulo $n = 257$ with $k = 9$. 
With this algorithm a single shift right at the end is the only right shift required to reduce the input instead of $k$ right shifts inside the 
loop.  Note that for the iterations $t = 2, 5, 6$ and $8$ where the result $x$ is not changed.  In those iterations the $t$'th bit of $x$ is 
zero and the appropriate multiple of $n$ does not need to be added to force the $t$'th bit of the result to zero.  

\subsection{Digit Based Montgomery Reduction}
Instead of computing the reduction on a bit-by-bit basis it is actually much faster to compute it on digit-by-digit basis.  Consider the
previous algorithm re-written to compute the Montgomery reduction in this new fashion.

\begin{figure}[!here]
\begin{small}
\begin{center}
\begin{tabular}{l}
\hline Algorithm \textbf{Montgomery Reduction} (modified II). \\
\textbf{Input}.   Integer $x$, $n$ and $k$ ($\beta^k > n$) \\
\textbf{Output}.  $\beta^{-k}x \mbox{ (mod }n\mbox{)}$ \\
\hline \\
1.  for $t$ from $0$ to $k - 1$ do \\
\hspace{3mm}1.1  $x \leftarrow x + \mu n \beta^t$ \\
2.  Return $x/\beta^k$. \\
\hline
\end{tabular}
\end{center}
\end{small}
\caption{Algorithm Montgomery Reduction (modified II)}
\end{figure}

The value $\mu n \beta^t$ is a multiple of the modulus $n$ meaning that it will not change the residue.  If the first digit of 
the value $\mu n \beta^t$ equals the negative (modulo $\beta$) of the $t$'th digit of $x$ then the addition will result in a zero digit.  This
problem breaks down to solving the following congruency.  

\begin{center}
\begin{tabular}{rcl}
$x_t + \mu n_0$ & $\equiv$ & $0 \mbox{ (mod }\beta\mbox{)}$ \\
$\mu n_0$ & $\equiv$ & $-x_t \mbox{ (mod }\beta\mbox{)}$ \\
$\mu$ & $\equiv$ & $-x_t/n_0 \mbox{ (mod }\beta\mbox{)}$ \\
\end{tabular}
\end{center}

In each iteration of the loop on step 1 a new value of $\mu$ must be calculated.  The value of $-1/n_0 \mbox{ (mod }\beta\mbox{)}$ is used 
extensively in this algorithm and should be precomputed.  Let $\rho$ represent the negative of the modular inverse of $n_0$ modulo $\beta$.  

For example, let $\beta = 10$ represent the radix.  Let $n = 17$ represent the modulus which implies $k = 2$ and $\rho \equiv 7$.  Let $x = 33$ 
represent the value to reduce.

\newpage\begin{figure}
\begin{center}
\begin{tabular}{|c|c|c|}
\hline \textbf{Step ($t$)} & \textbf{Value of $x$} & \textbf{Value of $\mu$} \\
\hline --                 & $33$ & --\\
\hline $0$                 & $33 + \mu n = 50$ & $1$ \\
\hline $1$                 & $50 + \mu n \beta = 900$ & $5$ \\
\hline
\end{tabular}
\end{center}
\caption{Example of Montgomery Reduction}
\end{figure}

The final result $900$ is then divided by $\beta^k$ to produce the final result $9$.  The first observation is that $9 \nequiv x \mbox{ (mod }n\mbox{)}$ 
which implies the result is not the modular residue of $x$ modulo $n$.  However, recall that the residue is actually multiplied by $\beta^{-k}$ in
the algorithm.  To get the true residue the value must be multiplied by $\beta^k$.  In this case $\beta^k \equiv 15 \mbox{ (mod }n\mbox{)}$ and
the correct residue is $9 \cdot 15 \equiv 16 \mbox{ (mod }n\mbox{)}$.  

\subsection{Baseline Montgomery Reduction}
The baseline Montgomery reduction algorithm will produce the residue for any size input.  It is designed to be a catch-all algororithm for 
Montgomery reductions.  

\newpage\begin{figure}[!here]
\begin{small}
\begin{center}
\begin{tabular}{l}
\hline Algorithm \textbf{mp\_montgomery\_reduce}. \\
\textbf{Input}.   mp\_int $x$, mp\_int $n$ and a digit $\rho \equiv -1/n_0 \mbox{ (mod }n\mbox{)}$. \\
\hspace{11.5mm}($0 \le x < n^2, n > 1, (n, \beta) = 1, \beta^k > n$) \\
\textbf{Output}.  $\beta^{-k}x \mbox{ (mod }n\mbox{)}$ \\
\hline \\
1.  $digs \leftarrow 2n.used + 1$ \\
2.  If $digs < MP\_ARRAY$ and $m.used < \delta$ then \\
\hspace{3mm}2.1  Use algorithm fast\_mp\_montgomery\_reduce instead. \\
\\
Setup $x$ for the reduction. \\
3.  If $x.alloc < digs$ then grow $x$ to $digs$ digits. \\
4.  $x.used \leftarrow digs$ \\
\\
Eliminate the lower $k$ digits. \\
5.  For $ix$ from $0$ to $k - 1$ do \\
\hspace{3mm}5.1  $\mu \leftarrow x_{ix} \cdot \rho \mbox{ (mod }\beta\mbox{)}$ \\
\hspace{3mm}5.2  $u \leftarrow 0$ \\
\hspace{3mm}5.3  For $iy$ from $0$ to $k - 1$ do \\
\hspace{6mm}5.3.1  $\hat r \leftarrow \mu n_{iy} + x_{ix + iy} + u$ \\
\hspace{6mm}5.3.2  $x_{ix + iy} \leftarrow \hat r \mbox{ (mod }\beta\mbox{)}$ \\
\hspace{6mm}5.3.3  $u \leftarrow \lfloor \hat r / \beta \rfloor$ \\
\hspace{3mm}5.4  While $u > 0$ do \\
\hspace{6mm}5.4.1  $iy \leftarrow iy + 1$ \\
\hspace{6mm}5.4.2  $x_{ix + iy} \leftarrow x_{ix + iy} + u$ \\
\hspace{6mm}5.4.3  $u \leftarrow \lfloor x_{ix+iy} / \beta \rfloor$ \\
\hspace{6mm}5.4.4  $x_{ix + iy} \leftarrow x_{ix+iy} \mbox{ (mod }\beta\mbox{)}$ \\
\\
Divide by $\beta^k$ and fix up as required. \\
6.  $x \leftarrow \lfloor x / \beta^k \rfloor$ \\
7.  If $x \ge n$ then \\
\hspace{3mm}7.1  $x \leftarrow x - n$ \\
8.  Return(\textit{MP\_OKAY}). \\
\hline
\end{tabular}
\end{center}
\end{small}
\caption{Algorithm mp\_montgomery\_reduce}
\end{figure}

\textbf{Algorithm mp\_montgomery\_reduce.}
This algorithm reduces the input $x$ modulo $n$ in place using the Montgomery reduction algorithm.  The algorithm is loosely based
on algorithm 14.32 of \cite[pp.601]{HAC} except it merges the multiplication of $\mu n \beta^t$ with the addition in the inner loop.  The
restrictions on this algorithm are fairly easy to adapt to.  First $0 \le x < n^2$ bounds the input to numbers in the same range as 
for the Barrett algorithm.  Additionally if $n > 1$ and $n$ is odd there will exist a modular inverse $\rho$.  $\rho$ must be calculated in
advance of this algorithm.  Finally the variable $k$ is fixed and a pseudonym for $n.used$.  

Step 2 decides whether a faster Montgomery algorithm can be used.  It is based on the Comba technique meaning that there are limits on
the size of the input.  This algorithm is discussed in sub-section 6.3.3.

Step 5 is the main reduction loop of the algorithm.  The value of $\mu$ is calculated once per iteration in the outer loop.  The inner loop
calculates $x + \mu n \beta^{ix}$ by multiplying $\mu n$ and adding the result to $x$ shifted by $ix$ digits.  Both the addition and
multiplication are performed in the same loop to save time and memory.  Step 5.4 will handle any additional carries that escape the inner loop.

Using a quick inspection this algorithm requires $n$ single precision multiplications for the outer loop and $n^2$ single precision multiplications 
in the inner loop.  In total $n^2 + n$ single precision multiplications which compares favourably to Barrett at $n^2 + 2n - 1$ single precision
multiplications.  

\vspace{+3mm}\begin{small}
\hspace{-5.1mm}{\bf File}: bn\_mp\_montgomery\_reduce.c
\vspace{-3mm}
\begin{alltt}
\end{alltt}
\end{small}

This is the baseline implementation of the Montgomery reduction algorithm.  Lines 31 to 36 determine if the Comba based
routine can be used instead.  Line 47 computes the value of $\mu$ for that particular iteration of the outer loop.  

The multiplication $\mu n \beta^{ix}$ is performed in one step in the inner loop.  The alias $tmpx$ refers to the $ix$'th digit of $x$ and
the alias $tmpn$ refers to the modulus $n$.  

\subsection{Faster ``Comba'' Montgomery Reduction}

The Montgomery reduction requires fewer single precision multiplications than a Barrett reduction, however it is much slower due to the serial
nature of the inner loop.  The Barrett reduction algorithm requires two slightly modified multipliers which can be implemented with the Comba
technique.  The Montgomery reduction algorithm cannot directly use the Comba technique to any significant advantage since the inner loop calculates
a $k \times 1$ product $k$ times. 

The biggest obstacle is that at the $ix$'th iteration of the outer loop the value of $x_{ix}$ is required to calculate $\mu$.  This means the 
carries from $0$ to $ix - 1$ must have been propagated upwards to form a valid $ix$'th digit.  The solution as it turns out is very simple.  
Perform a Comba like multiplier and inside the outer loop just after the inner loop fix up the $ix + 1$'th digit by forwarding the carry.  

With this change in place the Montgomery reduction algorithm can be performed with a Comba style multiplication loop which substantially increases
the speed of the algorithm.  

\newpage\begin{figure}[!here]
\begin{small}
\begin{center}
\begin{tabular}{l}
\hline Algorithm \textbf{fast\_mp\_montgomery\_reduce}. \\
\textbf{Input}.   mp\_int $x$, mp\_int $n$ and a digit $\rho \equiv -1/n_0 \mbox{ (mod }n\mbox{)}$. \\
\hspace{11.5mm}($0 \le x < n^2, n > 1, (n, \beta) = 1, \beta^k > n$) \\
\textbf{Output}.  $\beta^{-k}x \mbox{ (mod }n\mbox{)}$ \\
\hline \\
Place an array of \textbf{MP\_WARRAY} mp\_word variables called $\hat W$ on the stack. \\
1.  if $x.alloc < n.used + 1$ then grow $x$ to $n.used + 1$ digits. \\
Copy the digits of $x$ into the array $\hat W$ \\
2.  For $ix$ from $0$ to $x.used - 1$ do \\
\hspace{3mm}2.1  $\hat W_{ix} \leftarrow x_{ix}$ \\
3.  For $ix$ from $x.used$ to $2n.used - 1$ do \\
\hspace{3mm}3.1  $\hat W_{ix} \leftarrow 0$ \\
Elimiate the lower $k$ digits. \\
4.  for $ix$ from $0$ to $n.used - 1$ do \\
\hspace{3mm}4.1  $\mu \leftarrow \hat W_{ix} \cdot \rho \mbox{ (mod }\beta\mbox{)}$ \\
\hspace{3mm}4.2  For $iy$ from $0$ to $n.used - 1$ do \\
\hspace{6mm}4.2.1  $\hat W_{iy + ix} \leftarrow \hat W_{iy + ix} + \mu \cdot n_{iy}$ \\
\hspace{3mm}4.3  $\hat W_{ix + 1} \leftarrow \hat W_{ix + 1} + \lfloor \hat W_{ix} / \beta \rfloor$ \\
Propagate carries upwards. \\
5.  for $ix$ from $n.used$ to $2n.used + 1$ do \\
\hspace{3mm}5.1  $\hat W_{ix + 1} \leftarrow \hat W_{ix + 1} + \lfloor \hat W_{ix} / \beta \rfloor$ \\
Shift right and reduce modulo $\beta$ simultaneously. \\
6.  for $ix$ from $0$ to $n.used + 1$ do \\
\hspace{3mm}6.1  $x_{ix} \leftarrow \hat W_{ix + n.used} \mbox{ (mod }\beta\mbox{)}$ \\
Zero excess digits and fixup $x$. \\
7.  if $x.used > n.used + 1$ then do \\
\hspace{3mm}7.1  for $ix$ from $n.used + 1$ to $x.used - 1$ do \\
\hspace{6mm}7.1.1  $x_{ix} \leftarrow 0$ \\
8.  $x.used \leftarrow n.used + 1$ \\
9.  Clamp excessive digits of $x$. \\
10.  If $x \ge n$ then \\
\hspace{3mm}10.1  $x \leftarrow x - n$ \\
11.  Return(\textit{MP\_OKAY}). \\
\hline
\end{tabular}
\end{center}
\end{small}
\caption{Algorithm fast\_mp\_montgomery\_reduce}
\end{figure}

\textbf{Algorithm fast\_mp\_montgomery\_reduce.}
This algorithm will compute the Montgomery reduction of $x$ modulo $n$ using the Comba technique.  It is on most computer platforms significantly
faster than algorithm mp\_montgomery\_reduce and algorithm mp\_reduce (\textit{Barrett reduction}).  The algorithm has the same restrictions
on the input as the baseline reduction algorithm.  An additional two restrictions are imposed on this algorithm.  The number of digits $k$ in the 
the modulus $n$ must not violate $MP\_WARRAY > 2k +1$ and $n < \delta$.   When $\beta = 2^{28}$ this algorithm can be used to reduce modulo
a modulus of at most $3,556$ bits in length.  

As in the other Comba reduction algorithms there is a $\hat W$ array which stores the columns of the product.  It is initially filled with the
contents of $x$ with the excess digits zeroed.  The reduction loop is very similar the to the baseline loop at heart.  The multiplication on step
4.1 can be single precision only since $ab \mbox{ (mod }\beta\mbox{)} \equiv (a \mbox{ mod }\beta)(b \mbox{ mod }\beta)$.  Some multipliers such
as those on the ARM processors take a variable length time to complete depending on the number of bytes of result it must produce.  By performing
a single precision multiplication instead half the amount of time is spent.

Also note that digit $\hat W_{ix}$ must have the carry from the $ix - 1$'th digit propagated upwards in order for this to work.  That is what step
4.3 will do.  In effect over the $n.used$ iterations of the outer loop the $n.used$'th lower columns all have the their carries propagated forwards.  Note
how the upper bits of those same words are not reduced modulo $\beta$.  This is because those values will be discarded shortly and there is no
point.

Step 5 will propagate the remainder of the carries upwards.  On step 6 the columns are reduced modulo $\beta$ and shifted simultaneously as they are
stored in the destination $x$.  

\vspace{+3mm}\begin{small}
\hspace{-5.1mm}{\bf File}: bn\_fast\_mp\_montgomery\_reduce.c
\vspace{-3mm}
\begin{alltt}
\end{alltt}
\end{small}

The $\hat W$ array is first filled with digits of $x$ on line 48 then the rest of the digits are zeroed on line 55.  Both loops share
the same alias variables to make the code easier to read.  

The value of $\mu$ is calculated in an interesting fashion.  First the value $\hat W_{ix}$ is reduced modulo $\beta$ and cast to a mp\_digit.  This
forces the compiler to use a single precision multiplication and prevents any concerns about loss of precision.   Line 110 fixes the carry 
for the next iteration of the loop by propagating the carry from $\hat W_{ix}$ to $\hat W_{ix+1}$.

The for loop on line 109 propagates the rest of the carries upwards through the columns.  The for loop on line 126 reduces the columns
modulo $\beta$ and shifts them $k$ places at the same time.  The alias $\_ \hat W$ actually refers to the array $\hat W$ starting at the $n.used$'th
digit, that is $\_ \hat W_{t} = \hat W_{n.used + t}$.  

\subsection{Montgomery Setup}
To calculate the variable $\rho$ a relatively simple algorithm will be required.  

\begin{figure}[!here]
\begin{small}
\begin{center}
\begin{tabular}{l}
\hline Algorithm \textbf{mp\_montgomery\_setup}. \\
\textbf{Input}.   mp\_int $n$ ($n > 1$ and $(n, 2) = 1$) \\
\textbf{Output}.  $\rho \equiv -1/n_0 \mbox{ (mod }\beta\mbox{)}$ \\
\hline \\
1.  $b \leftarrow n_0$ \\
2.  If $b$ is even return(\textit{MP\_VAL}) \\
3.  $x \leftarrow (((b + 2) \mbox{ AND } 4) << 1) + b$ \\
4.  for $k$ from 0 to $\lceil lg(lg(\beta)) \rceil - 2$ do \\
\hspace{3mm}4.1  $x \leftarrow x \cdot (2 - bx)$ \\
5.  $\rho \leftarrow \beta - x \mbox{ (mod }\beta\mbox{)}$ \\
6.  Return(\textit{MP\_OKAY}). \\
\hline
\end{tabular}
\end{center}
\end{small}
\caption{Algorithm mp\_montgomery\_setup} 
\end{figure}

\textbf{Algorithm mp\_montgomery\_setup.}
This algorithm will calculate the value of $\rho$ required within the Montgomery reduction algorithms.  It uses a very interesting trick 
to calculate $1/n_0$ when $\beta$ is a power of two.  

\vspace{+3mm}\begin{small}
\hspace{-5.1mm}{\bf File}: bn\_mp\_montgomery\_setup.c
\vspace{-3mm}
\begin{alltt}
\end{alltt}
\end{small}

This source code computes the value of $\rho$ required to perform Montgomery reduction.  It has been modified to avoid performing excess
multiplications when $\beta$ is not the default 28-bits.  

\section{The Diminished Radix Algorithm}
The Diminished Radix method of modular reduction \cite{DRMET} is a fairly clever technique which can be more efficient than either the Barrett
or Montgomery methods for certain forms of moduli.  The technique is based on the following simple congruence.

\begin{equation}
(x \mbox{ mod } n) + k \lfloor x / n \rfloor \equiv x \mbox{ (mod }(n - k)\mbox{)}
\end{equation}

This observation was used in the MMB \cite{MMB} block cipher to create a diffusion primitive.  It used the fact that if $n = 2^{31}$ and $k=1$ that 
then a x86 multiplier could produce the 62-bit product and use  the ``shrd'' instruction to perform a double-precision right shift.  The proof
of the above equation is very simple.  First write $x$ in the product form.

\begin{equation}
x = qn + r
\end{equation}

Now reduce both sides modulo $(n - k)$.

\begin{equation}
x \equiv qk + r  \mbox{ (mod }(n-k)\mbox{)}
\end{equation}

The variable $n$ reduces modulo $n - k$ to $k$.  By putting $q = \lfloor x/n \rfloor$ and $r = x \mbox{ mod } n$ 
into the equation the original congruence is reproduced, thus concluding the proof.  The following algorithm is based on this observation.

\begin{figure}[!here]
\begin{small}
\begin{center}
\begin{tabular}{l}
\hline Algorithm \textbf{Diminished Radix Reduction}. \\
\textbf{Input}.   Integer $x$, $n$, $k$ \\
\textbf{Output}.  $x \mbox{ mod } (n - k)$ \\
\hline \\
1.  $q \leftarrow \lfloor x / n \rfloor$ \\
2.  $q \leftarrow k \cdot q$ \\
3.  $x \leftarrow x \mbox{ (mod }n\mbox{)}$ \\
4.  $x \leftarrow x + q$ \\
5.  If $x \ge (n - k)$ then \\
\hspace{3mm}5.1  $x \leftarrow x - (n - k)$ \\
\hspace{3mm}5.2  Goto step 1. \\
6.  Return $x$ \\
\hline
\end{tabular}
\end{center}
\end{small}
\caption{Algorithm Diminished Radix Reduction}
\label{fig:DR}
\end{figure}

This algorithm will reduce $x$ modulo $n - k$ and return the residue.  If $0 \le x < (n - k)^2$ then the algorithm will loop almost always
once or twice and occasionally three times.  For simplicity sake the value of $x$ is bounded by the following simple polynomial.

\begin{equation} 
0 \le x < n^2 + k^2 - 2nk
\end{equation}

The true bound is  $0 \le x < (n - k - 1)^2$ but this has quite a few more terms.  The value of $q$ after step 1 is bounded by the following.

\begin{equation}
q < n - 2k - k^2/n
\end{equation}

Since $k^2$ is going to be considerably smaller than $n$ that term will always be zero.  The value of $x$ after step 3 is bounded trivially as
$0 \le x < n$.  By step four the sum $x + q$ is bounded by 

\begin{equation}
0 \le q + x < (k + 1)n - 2k^2 - 1
\end{equation}

With a second pass $q$ will be loosely bounded by $0 \le q < k^2$ after step 2 while $x$ will still be loosely bounded by $0 \le x < n$ after step 3.  After the second pass it is highly unlike that the
sum in step 4 will exceed $n - k$.  In practice fewer than three passes of the algorithm are required to reduce virtually every input in the 
range $0 \le x < (n - k - 1)^2$.  

\begin{figure}
\begin{small}
\begin{center}
\begin{tabular}{|l|}
\hline
$x = 123456789, n = 256, k = 3$ \\
\hline $q \leftarrow \lfloor x/n \rfloor = 482253$ \\
$q \leftarrow q*k = 1446759$ \\
$x \leftarrow x \mbox{ mod } n = 21$ \\
$x \leftarrow x + q = 1446780$ \\
$x \leftarrow x - (n - k) = 1446527$ \\
\hline 
$q \leftarrow \lfloor x/n \rfloor = 5650$ \\
$q \leftarrow q*k = 16950$ \\
$x \leftarrow x \mbox{ mod } n = 127$ \\
$x \leftarrow x + q = 17077$ \\
$x \leftarrow x - (n - k) = 16824$ \\
\hline 
$q \leftarrow \lfloor x/n \rfloor = 65$ \\
$q \leftarrow q*k = 195$ \\
$x \leftarrow x \mbox{ mod } n = 184$ \\
$x \leftarrow x + q = 379$ \\
$x \leftarrow x - (n - k) = 126$ \\
\hline
\end{tabular}
\end{center}
\end{small}
\caption{Example Diminished Radix Reduction}
\label{fig:EXDR}
\end{figure}

Figure~\ref{fig:EXDR} demonstrates the reduction of $x = 123456789$ modulo $n - k = 253$ when $n = 256$ and $k = 3$.  Note that even while $x$
is considerably larger than $(n - k - 1)^2 = 63504$ the algorithm still converges on the modular residue exceedingly fast.  In this case only
three passes were required to find the residue $x \equiv 126$.


\subsection{Choice of Moduli}
On the surface this algorithm looks like a very expensive algorithm.  It requires a couple of subtractions followed by multiplication and other
modular reductions.  The usefulness of this algorithm becomes exceedingly clear when an appropriate modulus is chosen.

Division in general is a very expensive operation to perform.  The one exception is when the division is by a power of the radix of representation used.  
Division by ten for example is simple for pencil and paper mathematics since it amounts to shifting the decimal place to the right.  Similarly division 
by two (\textit{or powers of two}) is very simple for binary computers to perform.  It would therefore seem logical to choose $n$ of the form $2^p$ 
which would imply that $\lfloor x / n \rfloor$ is a simple shift of $x$ right $p$ bits.  

However, there is one operation related to division of power of twos that is even faster than this.  If $n = \beta^p$ then the division may be 
performed by moving whole digits to the right $p$ places.  In practice division by $\beta^p$ is much faster than division by $2^p$ for any $p$.  
Also with the choice of $n = \beta^p$ reducing $x$ modulo $n$ merely requires zeroing the digits above the $p-1$'th digit of $x$.  

Throughout the next section the term ``restricted modulus'' will refer to a modulus of the form $\beta^p - k$ whereas the term ``unrestricted
modulus'' will refer to a modulus of the form $2^p - k$.  The word ``restricted'' in this case refers to the fact that it is based on the 
$2^p$ logic except $p$ must be a multiple of $lg(\beta)$.  

\subsection{Choice of $k$}
Now that division and reduction (\textit{step 1 and 3 of figure~\ref{fig:DR}}) have been optimized to simple digit operations the multiplication by $k$
in step 2 is the most expensive operation.  Fortunately the choice of $k$ is not terribly limited.  For all intents and purposes it might
as well be a single digit.  The smaller the value of $k$ is the faster the algorithm will be.  

\subsection{Restricted Diminished Radix Reduction}
The restricted Diminished Radix algorithm can quickly reduce an input modulo a modulus of the form $n = \beta^p - k$.  This algorithm can reduce 
an input $x$ within the range $0 \le x < n^2$ using only a couple passes of the algorithm demonstrated in figure~\ref{fig:DR}.  The implementation
of this algorithm has been optimized to avoid additional overhead associated with a division by $\beta^p$, the multiplication by $k$ or the addition 
of $x$ and $q$.  The resulting algorithm is very efficient and can lead to substantial improvements over Barrett and Montgomery reduction when modular 
exponentiations are performed.

\newpage\begin{figure}[!here]
\begin{small}
\begin{center}
\begin{tabular}{l}
\hline Algorithm \textbf{mp\_dr\_reduce}. \\
\textbf{Input}.   mp\_int $x$, $n$ and a mp\_digit $k = \beta - n_0$ \\
\hspace{11.5mm}($0 \le x < n^2$, $n > 1$, $0 < k < \beta$) \\
\textbf{Output}.  $x \mbox{ mod } n$ \\
\hline \\
1.  $m \leftarrow n.used$ \\
2.  If $x.alloc < 2m$ then grow $x$ to $2m$ digits. \\
3.  $\mu \leftarrow 0$ \\
4.  for $i$ from $0$ to $m - 1$ do \\
\hspace{3mm}4.1  $\hat r \leftarrow k \cdot x_{m+i} + x_{i} + \mu$ \\
\hspace{3mm}4.2  $x_{i} \leftarrow \hat r \mbox{ (mod }\beta\mbox{)}$ \\
\hspace{3mm}4.3  $\mu \leftarrow \lfloor \hat r / \beta \rfloor$ \\
5.  $x_{m} \leftarrow \mu$ \\
6.  for $i$ from $m + 1$ to $x.used - 1$ do \\
\hspace{3mm}6.1  $x_{i} \leftarrow 0$ \\
7.  Clamp excess digits of $x$. \\
8.  If $x \ge n$ then \\
\hspace{3mm}8.1  $x \leftarrow x - n$ \\
\hspace{3mm}8.2  Goto step 3. \\
9.  Return(\textit{MP\_OKAY}). \\
\hline
\end{tabular}
\end{center}
\end{small}
\caption{Algorithm mp\_dr\_reduce}
\end{figure}

\textbf{Algorithm mp\_dr\_reduce.}
This algorithm will perform the Dimished Radix reduction of $x$ modulo $n$.  It has similar restrictions to that of the Barrett reduction
with the addition that $n$ must be of the form $n = \beta^m - k$ where $0 < k <\beta$.  

This algorithm essentially implements the pseudo-code in figure~\ref{fig:DR} except with a slight optimization.  The division by $\beta^m$, multiplication by $k$
and addition of $x \mbox{ mod }\beta^m$ are all performed simultaneously inside the loop on step 4.  The division by $\beta^m$ is emulated by accessing
the term at the $m+i$'th position which is subsequently multiplied by $k$ and added to the term at the $i$'th position.  After the loop the $m$'th
digit is set to the carry and the upper digits are zeroed.  Steps 5 and 6 emulate the reduction modulo $\beta^m$ that should have happend to 
$x$ before the addition of the multiple of the upper half.  

At step 8 if $x$ is still larger than $n$ another pass of the algorithm is required.  First $n$ is subtracted from $x$ and then the algorithm resumes
at step 3.  

\vspace{+3mm}\begin{small}
\hspace{-5.1mm}{\bf File}: bn\_mp\_dr\_reduce.c
\vspace{-3mm}
\begin{alltt}
\end{alltt}
\end{small}

The first step is to grow $x$ as required to $2m$ digits since the reduction is performed in place on $x$.  The label on line 52 is where
the algorithm will resume if further reduction passes are required.  In theory it could be placed at the top of the function however, the size of
the modulus and question of whether $x$ is large enough are invariant after the first pass meaning that it would be a waste of time.  

The aliases $tmpx1$ and $tmpx2$ refer to the digits of $x$ where the latter is offset by $m$ digits.  By reading digits from $x$ offset by $m$ digits
a division by $\beta^m$ can be simulated virtually for free.  The loop on line 64 performs the bulk of the work (\textit{corresponds to step 4 of algorithm 7.11})
in this algorithm.

By line 67 the pointer $tmpx1$ points to the $m$'th digit of $x$ which is where the final carry will be placed.  Similarly by line 74 the 
same pointer will point to the $m+1$'th digit where the zeroes will be placed.  

Since the algorithm is only valid if both $x$ and $n$ are greater than zero an unsigned comparison suffices to determine if another pass is required.  
With the same logic at line 81 the value of $x$ is known to be greater than or equal to $n$ meaning that an unsigned subtraction can be used
as well.  Since the destination of the subtraction is the larger of the inputs the call to algorithm s\_mp\_sub cannot fail and the return code
does not need to be checked.

\subsubsection{Setup}
To setup the restricted Diminished Radix algorithm the value $k = \beta - n_0$ is required.  This algorithm is not really complicated but provided for
completeness.

\begin{figure}[!here]
\begin{small}
\begin{center}
\begin{tabular}{l}
\hline Algorithm \textbf{mp\_dr\_setup}. \\
\textbf{Input}.   mp\_int $n$ \\
\textbf{Output}.  $k = \beta - n_0$ \\
\hline \\
1.  $k \leftarrow \beta - n_0$ \\
\hline
\end{tabular}
\end{center}
\end{small}
\caption{Algorithm mp\_dr\_setup}
\end{figure}

\vspace{+3mm}\begin{small}
\hspace{-5.1mm}{\bf File}: bn\_mp\_dr\_setup.c
\vspace{-3mm}
\begin{alltt}
\end{alltt}
\end{small}

\subsubsection{Modulus Detection}
Another algorithm which will be useful is the ability to detect a restricted Diminished Radix modulus.  An integer is said to be
of restricted Diminished Radix form if all of the digits are equal to $\beta - 1$ except the trailing digit which may be any value.

\begin{figure}[!here]
\begin{small}
\begin{center}
\begin{tabular}{l}
\hline Algorithm \textbf{mp\_dr\_is\_modulus}. \\
\textbf{Input}.   mp\_int $n$ \\
\textbf{Output}.  $1$ if $n$ is in D.R form, $0$ otherwise \\
\hline
1.  If $n.used < 2$ then return($0$). \\
2.  for $ix$ from $1$ to $n.used - 1$ do \\
\hspace{3mm}2.1  If $n_{ix} \ne \beta - 1$ return($0$). \\
3.  Return($1$). \\
\hline
\end{tabular}
\end{center}
\end{small}
\caption{Algorithm mp\_dr\_is\_modulus}
\end{figure}

\textbf{Algorithm mp\_dr\_is\_modulus.}
This algorithm determines if a value is in Diminished Radix form.  Step 1 rejects obvious cases where fewer than two digits are
in the mp\_int.  Step 2 tests all but the first digit to see if they are equal to $\beta - 1$.  If the algorithm manages to get to
step 3 then $n$ must be of Diminished Radix form.

\vspace{+3mm}\begin{small}
\hspace{-5.1mm}{\bf File}: bn\_mp\_dr\_is\_modulus.c
\vspace{-3mm}
\begin{alltt}
\end{alltt}
\end{small}

\subsection{Unrestricted Diminished Radix Reduction}
The unrestricted Diminished Radix algorithm allows modular reductions to be performed when the modulus is of the form $2^p - k$.  This algorithm
is a straightforward adaptation of algorithm~\ref{fig:DR}.

In general the restricted Diminished Radix reduction algorithm is much faster since it has considerably lower overhead.  However, this new
algorithm is much faster than either Montgomery or Barrett reduction when the moduli are of the appropriate form.

\begin{figure}[!here]
\begin{small}
\begin{center}
\begin{tabular}{l}
\hline Algorithm \textbf{mp\_reduce\_2k}. \\
\textbf{Input}.   mp\_int $a$ and $n$.  mp\_digit $k$  \\
\hspace{11.5mm}($a \ge 0$, $n > 1$, $0 < k < \beta$, $n + k$ is a power of two) \\
\textbf{Output}.  $a \mbox{ (mod }n\mbox{)}$ \\
\hline
1.  $p \leftarrow \lceil lg(n) \rceil$  (\textit{mp\_count\_bits}) \\
2.  While $a \ge n$ do \\
\hspace{3mm}2.1  $q \leftarrow \lfloor a / 2^p \rfloor$ (\textit{mp\_div\_2d}) \\
\hspace{3mm}2.2  $a \leftarrow a \mbox{ (mod }2^p\mbox{)}$ (\textit{mp\_mod\_2d}) \\
\hspace{3mm}2.3  $q \leftarrow q \cdot k$ (\textit{mp\_mul\_d}) \\
\hspace{3mm}2.4  $a \leftarrow a - q$ (\textit{s\_mp\_sub}) \\
\hspace{3mm}2.5  If $a \ge n$ then do \\
\hspace{6mm}2.5.1  $a \leftarrow a - n$ \\
3.  Return(\textit{MP\_OKAY}). \\
\hline
\end{tabular}
\end{center}
\end{small}
\caption{Algorithm mp\_reduce\_2k}
\end{figure}

\textbf{Algorithm mp\_reduce\_2k.}
This algorithm quickly reduces an input $a$ modulo an unrestricted Diminished Radix modulus $n$.  Division by $2^p$ is emulated with a right
shift which makes the algorithm fairly inexpensive to use.  

\vspace{+3mm}\begin{small}
\hspace{-5.1mm}{\bf File}: bn\_mp\_reduce\_2k.c
\vspace{-3mm}
\begin{alltt}
\end{alltt}
\end{small}

The algorithm mp\_count\_bits calculates the number of bits in an mp\_int which is used to find the initial value of $p$.  The call to mp\_div\_2d
on line 31 calculates both the quotient $q$ and the remainder $a$ required.  By doing both in a single function call the code size
is kept fairly small.  The multiplication by $k$ is only performed if $k > 1$. This allows reductions modulo $2^p - 1$ to be performed without
any multiplications.  

The unsigned s\_mp\_add, mp\_cmp\_mag and s\_mp\_sub are used in place of their full sign counterparts since the inputs are only valid if they are 
positive.  By using the unsigned versions the overhead is kept to a minimum.  

\subsubsection{Unrestricted Setup}
To setup this reduction algorithm the value of $k = 2^p - n$ is required.  

\begin{figure}[!here]
\begin{small}
\begin{center}
\begin{tabular}{l}
\hline Algorithm \textbf{mp\_reduce\_2k\_setup}. \\
\textbf{Input}.   mp\_int $n$   \\
\textbf{Output}.  $k = 2^p - n$ \\
\hline
1.  $p \leftarrow \lceil lg(n) \rceil$  (\textit{mp\_count\_bits}) \\
2.  $x \leftarrow 2^p$ (\textit{mp\_2expt}) \\
3.  $x \leftarrow x - n$ (\textit{mp\_sub}) \\
4.  $k \leftarrow x_0$ \\
5.  Return(\textit{MP\_OKAY}). \\
\hline
\end{tabular}
\end{center}
\end{small}
\caption{Algorithm mp\_reduce\_2k\_setup}
\end{figure}

\textbf{Algorithm mp\_reduce\_2k\_setup.}
This algorithm computes the value of $k$ required for the algorithm mp\_reduce\_2k.  By making a temporary variable $x$ equal to $2^p$ a subtraction
is sufficient to solve for $k$.  Alternatively if $n$ has more than one digit the value of $k$ is simply $\beta - n_0$.  

\vspace{+3mm}\begin{small}
\hspace{-5.1mm}{\bf File}: bn\_mp\_reduce\_2k\_setup.c
\vspace{-3mm}
\begin{alltt}
\end{alltt}
\end{small}

\subsubsection{Unrestricted Detection}
An integer $n$ is a valid unrestricted Diminished Radix modulus if either of the following are true.

\begin{enumerate}
\item  The number has only one digit.
\item  The number has more than one digit and every bit from the $\beta$'th to the most significant is one.
\end{enumerate}

If either condition is true than there is a power of two $2^p$ such that $0 < 2^p - n < \beta$.   If the input is only
one digit than it will always be of the correct form.  Otherwise all of the bits above the first digit must be one.  This arises from the fact
that there will be value of $k$ that when added to the modulus causes a carry in the first digit which propagates all the way to the most
significant bit.  The resulting sum will be a power of two.

\begin{figure}[!here]
\begin{small}
\begin{center}
\begin{tabular}{l}
\hline Algorithm \textbf{mp\_reduce\_is\_2k}. \\
\textbf{Input}.   mp\_int $n$   \\
\textbf{Output}.  $1$ if of proper form, $0$ otherwise \\
\hline
1.  If $n.used = 0$ then return($0$). \\
2.  If $n.used = 1$ then return($1$). \\
3.  $p \leftarrow \lceil lg(n) \rceil$  (\textit{mp\_count\_bits}) \\
4.  for $x$ from $lg(\beta)$ to $p$ do \\
\hspace{3mm}4.1  If the ($x \mbox{ mod }lg(\beta)$)'th bit of the $\lfloor x / lg(\beta) \rfloor$ of $n$ is zero then return($0$). \\
5.  Return($1$). \\
\hline
\end{tabular}
\end{center}
\end{small}
\caption{Algorithm mp\_reduce\_is\_2k}
\end{figure}

\textbf{Algorithm mp\_reduce\_is\_2k.}
This algorithm quickly determines if a modulus is of the form required for algorithm mp\_reduce\_2k to function properly.  

\vspace{+3mm}\begin{small}
\hspace{-5.1mm}{\bf File}: bn\_mp\_reduce\_is\_2k.c
\vspace{-3mm}
\begin{alltt}
\end{alltt}
\end{small}



\section{Algorithm Comparison}
So far three very different algorithms for modular reduction have been discussed.  Each of the algorithms have their own strengths and weaknesses
that makes having such a selection very useful.  The following table sumarizes the three algorithms along with comparisons of work factors.  Since
all three algorithms have the restriction that $0 \le x < n^2$ and $n > 1$ those limitations are not included in the table.  

\begin{center}
\begin{small}
\begin{tabular}{|c|c|c|c|c|c|}
\hline \textbf{Method} & \textbf{Work Required} & \textbf{Limitations} & \textbf{$m = 8$} & \textbf{$m = 32$} & \textbf{$m = 64$} \\
\hline Barrett    & $m^2 + 2m - 1$ & None              & $79$ & $1087$ & $4223$ \\
\hline Montgomery & $m^2 + m$      & $n$ must be odd   & $72$ & $1056$ & $4160$ \\
\hline D.R.       & $2m$           & $n = \beta^m - k$ & $16$ & $64$   & $128$  \\
\hline
\end{tabular}
\end{small}
\end{center}

In theory Montgomery and Barrett reductions would require roughly the same amount of time to complete.  However, in practice since Montgomery
reduction can be written as a single function with the Comba technique it is much faster.  Barrett reduction suffers from the overhead of
calling the half precision multipliers, addition and division by $\beta$ algorithms.

For almost every cryptographic algorithm Montgomery reduction is the algorithm of choice.  The one set of algorithms where Diminished Radix reduction truly
shines are based on the discrete logarithm problem such as Diffie-Hellman \cite{DH} and ElGamal \cite{ELGAMAL}.  In these algorithms
primes of the form $\beta^m - k$ can be found and shared amongst users.  These primes will allow the Diminished Radix algorithm to be used in
modular exponentiation to greatly speed up the operation.



\section*{Exercises}
\begin{tabular}{cl}
$\left [ 3 \right ]$ & Prove that the ``trick'' in algorithm mp\_montgomery\_setup actually \\
                     & calculates the correct value of $\rho$. \\
                     & \\
$\left [ 2 \right ]$ & Devise an algorithm to reduce modulo $n + k$ for small $k$ quickly.  \\
                     & \\
$\left [ 4 \right ]$ & Prove that the pseudo-code algorithm ``Diminished Radix Reduction'' \\
                     & (\textit{figure~\ref{fig:DR}}) terminates.  Also prove the probability that it will \\
                     & terminate within $1 \le k \le 10$ iterations. \\
                     & \\
\end{tabular}                     


\chapter{Exponentiation}
Exponentiation is the operation of raising one variable to the power of another, for example, $a^b$.  A variant of exponentiation, computed
in a finite field or ring, is called modular exponentiation.  This latter style of operation is typically used in public key 
cryptosystems such as RSA and Diffie-Hellman.  The ability to quickly compute modular exponentiations is of great benefit to any
such cryptosystem and many methods have been sought to speed it up.

\section{Exponentiation Basics}
A trivial algorithm would simply multiply $a$ against itself $b - 1$ times to compute the exponentiation desired.  However, as $b$ grows in size
the number of multiplications becomes prohibitive.  Imagine what would happen if $b$ $\approx$ $2^{1024}$ as is the case when computing an RSA signature
with a $1024$-bit key.  Such a calculation could never be completed as it would take simply far too long.

Fortunately there is a very simple algorithm based on the laws of exponents.  Recall that $lg_a(a^b) = b$ and that $lg_a(a^ba^c) = b + c$ which
are two trivial relationships between the base and the exponent.  Let $b_i$ represent the $i$'th bit of $b$ starting from the least 
significant bit.  If $b$ is a $k$-bit integer than the following equation is true.

\begin{equation}
a^b = \prod_{i=0}^{k-1} a^{2^i \cdot b_i}
\end{equation}

By taking the base $a$ logarithm of both sides of the equation the following equation is the result.

\begin{equation}
b = \sum_{i=0}^{k-1}2^i \cdot b_i
\end{equation}

The term $a^{2^i}$ can be found from the $i - 1$'th term by squaring the term since $\left ( a^{2^i} \right )^2$ is equal to
$a^{2^{i+1}}$.  This observation forms the basis of essentially all fast exponentiation algorithms.  It requires $k$ squarings and on average
$k \over 2$ multiplications to compute the result.  This is indeed quite an improvement over simply multiplying by $a$ a total of $b-1$ times.

While this current method is a considerable speed up there are further improvements to be made.  For example, the $a^{2^i}$ term does not need to 
be computed in an auxilary variable.  Consider the following equivalent algorithm.

\begin{figure}[!here]
\begin{small}
\begin{center}
\begin{tabular}{l}
\hline Algorithm \textbf{Left to Right Exponentiation}. \\
\textbf{Input}.   Integer $a$, $b$ and $k$ \\
\textbf{Output}.  $c = a^b$ \\
\hline \\
1.  $c \leftarrow 1$ \\
2.  for $i$ from $k - 1$ to $0$ do \\
\hspace{3mm}2.1  $c \leftarrow c^2$ \\
\hspace{3mm}2.2  $c \leftarrow c \cdot a^{b_i}$ \\
3.  Return $c$. \\
\hline
\end{tabular}
\end{center}
\end{small}
\caption{Left to Right Exponentiation}
\label{fig:LTOR}
\end{figure}

This algorithm starts from the most significant bit and works towards the least significant bit.  When the $i$'th bit of $b$ is set $a$ is
multiplied against the current product.  In each iteration the product is squared which doubles the exponent of the individual terms of the
product.  

For example, let $b = 101100_2 \equiv 44_{10}$.  The following chart demonstrates the actions of the algorithm.

\newpage\begin{figure}
\begin{center}
\begin{tabular}{|c|c|}
\hline \textbf{Value of $i$} & \textbf{Value of $c$} \\
\hline - & $1$ \\
\hline $5$ & $a$ \\
\hline $4$ & $a^2$ \\
\hline $3$ & $a^4 \cdot a$ \\
\hline $2$ & $a^8 \cdot a^2 \cdot a$ \\
\hline $1$ & $a^{16} \cdot a^4 \cdot a^2$ \\
\hline $0$ & $a^{32} \cdot a^8 \cdot a^4$ \\
\hline
\end{tabular}
\end{center}
\caption{Example of Left to Right Exponentiation}
\end{figure}

When the product $a^{32} \cdot a^8 \cdot a^4$ is simplified it is equal $a^{44}$ which is the desired exponentiation.  This particular algorithm is 
called ``Left to Right'' because it reads the exponent in that order.  All of the exponentiation algorithms that will be presented are of this nature.  

\subsection{Single Digit Exponentiation}
The first algorithm in the series of exponentiation algorithms will be an unbounded algorithm where the exponent is a single digit.  It is intended 
to be used when a small power of an input is required (\textit{e.g. $a^5$}).  It is faster than simply multiplying $b - 1$ times for all values of 
$b$ that are greater than three.  

\newpage\begin{figure}[!here]
\begin{small}
\begin{center}
\begin{tabular}{l}
\hline Algorithm \textbf{mp\_expt\_d}. \\
\textbf{Input}.   mp\_int $a$ and mp\_digit $b$ \\
\textbf{Output}.  $c = a^b$ \\
\hline \\
1.  $g \leftarrow a$ (\textit{mp\_init\_copy}) \\
2.  $c \leftarrow 1$ (\textit{mp\_set}) \\
3.  for $x$ from 1 to $lg(\beta)$ do \\
\hspace{3mm}3.1  $c \leftarrow c^2$ (\textit{mp\_sqr}) \\
\hspace{3mm}3.2  If $b$ AND $2^{lg(\beta) - 1} \ne 0$ then \\
\hspace{6mm}3.2.1  $c \leftarrow c \cdot g$ (\textit{mp\_mul}) \\
\hspace{3mm}3.3  $b \leftarrow b << 1$ \\
4.  Clear $g$. \\
5.  Return(\textit{MP\_OKAY}). \\
\hline
\end{tabular}
\end{center}
\end{small}
\caption{Algorithm mp\_expt\_d}
\end{figure}

\textbf{Algorithm mp\_expt\_d.}
This algorithm computes the value of $a$ raised to the power of a single digit $b$.  It uses the left to right exponentiation algorithm to
quickly compute the exponentiation.  It is loosely based on algorithm 14.79 of HAC \cite[pp. 615]{HAC} with the difference that the 
exponent is a fixed width.  

A copy of $a$ is made first to allow destination variable $c$ be the same as the source variable $a$.  The result is set to the initial value of 
$1$ in the subsequent step.

Inside the loop the exponent is read from the most significant bit first down to the least significant bit.  First $c$ is invariably squared
on step 3.1.  In the following step if the most significant bit of $b$ is one the copy of $a$ is multiplied against $c$.  The value
of $b$ is shifted left one bit to make the next bit down from the most signficant bit the new most significant bit.  In effect each
iteration of the loop moves the bits of the exponent $b$ upwards to the most significant location.

\vspace{+3mm}\begin{small}
\hspace{-5.1mm}{\bf File}: bn\_mp\_expt\_d.c
\vspace{-3mm}
\begin{alltt}
\end{alltt}
\end{small}

Line 29 sets the initial value of the result to $1$.  Next the loop on line 31 steps through each bit of the exponent starting from
the most significant down towards the least significant. The invariant squaring operation placed on line 33 is performed first.  After 
the squaring the result $c$ is multiplied by the base $g$ if and only if the most significant bit of the exponent is set.  The shift on line
47 moves all of the bits of the exponent upwards towards the most significant location.  

\section{$k$-ary Exponentiation}
When calculating an exponentiation the most time consuming bottleneck is the multiplications which are in general a small factor
slower than squaring.  Recall from the previous algorithm that $b_{i}$ refers to the $i$'th bit of the exponent $b$.  Suppose instead it referred to
the $i$'th $k$-bit digit of the exponent of $b$.  For $k = 1$ the definitions are synonymous and for $k > 1$ algorithm~\ref{fig:KARY}
computes the same exponentiation.  A group of $k$ bits from the exponent is called a \textit{window}.  That is it is a small window on only a
portion of the entire exponent.  Consider the following modification to the basic left to right exponentiation algorithm.

\begin{figure}[!here]
\begin{small}
\begin{center}
\begin{tabular}{l}
\hline Algorithm \textbf{$k$-ary Exponentiation}. \\
\textbf{Input}.   Integer $a$, $b$, $k$ and $t$ \\
\textbf{Output}.  $c = a^b$ \\
\hline \\
1.  $c \leftarrow 1$ \\
2.  for $i$ from $t - 1$ to $0$ do \\
\hspace{3mm}2.1  $c \leftarrow c^{2^k} $ \\
\hspace{3mm}2.2  Extract the $i$'th $k$-bit word from $b$ and store it in $g$. \\
\hspace{3mm}2.3  $c \leftarrow c \cdot a^g$ \\
3.  Return $c$. \\
\hline
\end{tabular}
\end{center}
\end{small}
\caption{$k$-ary Exponentiation}
\label{fig:KARY}
\end{figure}

The squaring on step 2.1 can be calculated by squaring the value $c$ successively $k$ times.  If the values of $a^g$ for $0 < g < 2^k$ have been
precomputed this algorithm requires only $t$ multiplications and $tk$ squarings.  The table can be generated with $2^{k - 1} - 1$ squarings and
$2^{k - 1} + 1$ multiplications.  This algorithm assumes that the number of bits in the exponent is evenly divisible by $k$.  
However, when it is not the remaining $0 < x \le k - 1$ bits can be handled with algorithm~\ref{fig:LTOR}.

Suppose $k = 4$ and $t = 100$.  This modified algorithm will require $109$ multiplications and $408$ squarings to compute the exponentiation.  The
original algorithm would on average have required $200$ multiplications and $400$ squrings to compute the same value.  The total number of squarings
has increased slightly but the number of multiplications has nearly halved.

\subsection{Optimal Values of $k$}
An optimal value of $k$ will minimize $2^{k} + \lceil n / k \rceil + n - 1$ for a fixed number of bits in the exponent $n$.  The simplest
approach is to brute force search amongst the values $k = 2, 3, \ldots, 8$ for the lowest result.  Table~\ref{fig:OPTK} lists optimal values of $k$
for various exponent sizes and compares the number of multiplication and squarings required against algorithm~\ref{fig:LTOR}.  

\begin{figure}[here]
\begin{center}
\begin{small}
\begin{tabular}{|c|c|c|c|c|c|}
\hline \textbf{Exponent (bits)} & \textbf{Optimal $k$} & \textbf{Work at $k$} & \textbf{Work with ~\ref{fig:LTOR}} \\
\hline $16$ & $2$ & $27$ & $24$ \\
\hline $32$ & $3$ & $49$ & $48$ \\
\hline $64$ & $3$ & $92$ & $96$ \\
\hline $128$ & $4$ & $175$ & $192$ \\
\hline $256$ & $4$ & $335$ & $384$ \\
\hline $512$ & $5$ & $645$ & $768$ \\
\hline $1024$ & $6$ & $1257$ & $1536$ \\
\hline $2048$ & $6$ & $2452$ & $3072$ \\
\hline $4096$ & $7$ & $4808$ & $6144$ \\
\hline
\end{tabular}
\end{small}
\end{center}
\caption{Optimal Values of $k$ for $k$-ary Exponentiation}
\label{fig:OPTK}
\end{figure}

\subsection{Sliding-Window Exponentiation}
A simple modification to the previous algorithm is only generate the upper half of the table in the range $2^{k-1} \le g < 2^k$.  Essentially
this is a table for all values of $g$ where the most significant bit of $g$ is a one.  However, in order for this to be allowed in the 
algorithm values of $g$ in the range $0 \le g < 2^{k-1}$ must be avoided.  

Table~\ref{fig:OPTK2} lists optimal values of $k$ for various exponent sizes and compares the work required against algorithm~\ref{fig:KARY}.  

\begin{figure}[here]
\begin{center}
\begin{small}
\begin{tabular}{|c|c|c|c|c|c|}
\hline \textbf{Exponent (bits)} & \textbf{Optimal $k$} & \textbf{Work at $k$} & \textbf{Work with ~\ref{fig:KARY}} \\
\hline $16$ & $3$ & $24$ & $27$ \\
\hline $32$ & $3$ & $45$ & $49$ \\
\hline $64$ & $4$ & $87$ & $92$ \\
\hline $128$ & $4$ & $167$ & $175$ \\
\hline $256$ & $5$ & $322$ & $335$ \\
\hline $512$ & $6$ & $628$ & $645$ \\
\hline $1024$ & $6$ & $1225$ & $1257$ \\
\hline $2048$ & $7$ & $2403$ & $2452$ \\
\hline $4096$ & $8$ & $4735$ & $4808$ \\
\hline
\end{tabular}
\end{small}
\end{center}
\caption{Optimal Values of $k$ for Sliding Window Exponentiation}
\label{fig:OPTK2}
\end{figure}

\newpage\begin{figure}[!here]
\begin{small}
\begin{center}
\begin{tabular}{l}
\hline Algorithm \textbf{Sliding Window $k$-ary Exponentiation}. \\
\textbf{Input}.   Integer $a$, $b$, $k$ and $t$ \\
\textbf{Output}.  $c = a^b$ \\
\hline \\
1.  $c \leftarrow 1$ \\
2.  for $i$ from $t - 1$ to $0$ do \\
\hspace{3mm}2.1  If the $i$'th bit of $b$ is a zero then \\
\hspace{6mm}2.1.1   $c \leftarrow c^2$ \\
\hspace{3mm}2.2  else do \\
\hspace{6mm}2.2.1  $c \leftarrow c^{2^k}$ \\
\hspace{6mm}2.2.2  Extract the $k$ bits from $(b_{i}b_{i-1}\ldots b_{i-(k-1)})$ and store it in $g$. \\
\hspace{6mm}2.2.3  $c \leftarrow c \cdot a^g$ \\
\hspace{6mm}2.2.4  $i \leftarrow i - k$ \\
3.  Return $c$. \\
\hline
\end{tabular}
\end{center}
\end{small}
\caption{Sliding Window $k$-ary Exponentiation}
\end{figure}

Similar to the previous algorithm this algorithm must have a special handler when fewer than $k$ bits are left in the exponent.  While this
algorithm requires the same number of squarings it can potentially have fewer multiplications.  The pre-computed table $a^g$ is also half
the size as the previous table.  

Consider the exponent $b = 111101011001000_2 \equiv 31432_{10}$ with $k = 3$ using both algorithms.  The first algorithm will divide the exponent up as 
the following five $3$-bit words $b \equiv \left ( 111, 101, 011, 001, 000 \right )_{2}$.  The second algorithm will break the 
exponent as $b \equiv \left ( 111, 101, 0, 110, 0, 100, 0 \right )_{2}$.  The single digit $0$ in the second representation are where
a single squaring took place instead of a squaring and multiplication.  In total the first method requires $10$ multiplications and $18$ 
squarings.  The second method requires $8$ multiplications and $18$ squarings.  

In general the sliding window method is never slower than the generic $k$-ary method and often it is slightly faster.  

\section{Modular Exponentiation}

Modular exponentiation is essentially computing the power of a base within a finite field or ring.  For example, computing 
$d \equiv a^b \mbox{ (mod }c\mbox{)}$ is a modular exponentiation.  Instead of first computing $a^b$ and then reducing it 
modulo $c$ the intermediate result is reduced modulo $c$ after every squaring or multiplication operation.  

This guarantees that any intermediate result is bounded by $0 \le d \le c^2 - 2c + 1$ and can be reduced modulo $c$ quickly using
one of the algorithms presented in chapter six.  

Before the actual modular exponentiation algorithm can be written a wrapper algorithm must be written first.  This algorithm
will allow the exponent $b$ to be negative which is computed as $c \equiv \left (1 / a \right )^{\vert b \vert} \mbox{(mod }d\mbox{)}$. The
value of $(1/a) \mbox{ mod }c$ is computed using the modular inverse (\textit{see \ref{sec;modinv}}).  If no inverse exists the algorithm
terminates with an error.  

\begin{figure}[!here]
\begin{small}
\begin{center}
\begin{tabular}{l}
\hline Algorithm \textbf{mp\_exptmod}. \\
\textbf{Input}.   mp\_int $a$, $b$ and $c$ \\
\textbf{Output}.  $y \equiv g^x \mbox{ (mod }p\mbox{)}$ \\
\hline \\
1.  If $c.sign = MP\_NEG$ return(\textit{MP\_VAL}). \\
2.  If $b.sign = MP\_NEG$ then \\
\hspace{3mm}2.1  $g' \leftarrow g^{-1} \mbox{ (mod }c\mbox{)}$ \\
\hspace{3mm}2.2  $x' \leftarrow \vert x \vert$ \\
\hspace{3mm}2.3  Compute $d \equiv g'^{x'} \mbox{ (mod }c\mbox{)}$ via recursion. \\
3.  if $p$ is odd \textbf{OR} $p$ is a D.R. modulus then \\
\hspace{3mm}3.1  Compute $y \equiv g^{x} \mbox{ (mod }p\mbox{)}$ via algorithm mp\_exptmod\_fast. \\
4.  else \\
\hspace{3mm}4.1  Compute $y \equiv g^{x} \mbox{ (mod }p\mbox{)}$ via algorithm s\_mp\_exptmod. \\
\hline
\end{tabular}
\end{center}
\end{small}
\caption{Algorithm mp\_exptmod}
\end{figure}

\textbf{Algorithm mp\_exptmod.}
The first algorithm which actually performs modular exponentiation is algorithm s\_mp\_exptmod.  It is a sliding window $k$-ary algorithm 
which uses Barrett reduction to reduce the product modulo $p$.  The second algorithm mp\_exptmod\_fast performs the same operation 
except it uses either Montgomery or Diminished Radix reduction.  The two latter reduction algorithms are clumped in the same exponentiation
algorithm since their arguments are essentially the same (\textit{two mp\_ints and one mp\_digit}).  

\vspace{+3mm}\begin{small}
\hspace{-5.1mm}{\bf File}: bn\_mp\_exptmod.c
\vspace{-3mm}
\begin{alltt}
\end{alltt}
\end{small}

In order to keep the algorithms in a known state the first step on line 29 is to reject any negative modulus as input.  If the exponent is
negative the algorithm tries to perform a modular exponentiation with the modular inverse of the base $G$.  The temporary variable $tmpG$ is assigned
the modular inverse of $G$ and $tmpX$ is assigned the absolute value of $X$.  The algorithm will recuse with these new values with a positive
exponent.

If the exponent is positive the algorithm resumes the exponentiation.  Line 77 determines if the modulus is of the restricted Diminished Radix 
form.  If it is not line 70 attempts to determine if it is of a unrestricted Diminished Radix form.  The integer $dr$ will take on one
of three values.

\begin{enumerate}
\item $dr = 0$ means that the modulus is not of either restricted or unrestricted Diminished Radix form.
\item $dr = 1$ means that the modulus is of restricted Diminished Radix form.
\item $dr = 2$ means that the modulus is of unrestricted Diminished Radix form.
\end{enumerate}

Line 69 determines if the fast modular exponentiation algorithm can be used.  It is allowed if $dr \ne 0$ or if the modulus is odd.  Otherwise,
the slower s\_mp\_exptmod algorithm is used which uses Barrett reduction.  

\subsection{Barrett Modular Exponentiation}

\newpage\begin{figure}[!here]
\begin{small}
\begin{center}
\begin{tabular}{l}
\hline Algorithm \textbf{s\_mp\_exptmod}. \\
\textbf{Input}.   mp\_int $a$, $b$ and $c$ \\
\textbf{Output}.  $y \equiv g^x \mbox{ (mod }p\mbox{)}$ \\
\hline \\
1.  $k \leftarrow lg(x)$ \\
2.  $winsize \leftarrow  \left \lbrace \begin{array}{ll}
                              2 &  \mbox{if }k \le 7 \\
                              3 &  \mbox{if }7 < k \le 36 \\
                              4 &  \mbox{if }36 < k \le 140 \\
                              5 &  \mbox{if }140 < k \le 450 \\
                              6 &  \mbox{if }450 < k \le 1303 \\
                              7 &  \mbox{if }1303 < k \le 3529 \\
                              8 &  \mbox{if }3529 < k \\
                              \end{array} \right .$ \\
3.  Initialize $2^{winsize}$ mp\_ints in an array named $M$ and one mp\_int named $\mu$ \\
4.  Calculate the $\mu$ required for Barrett Reduction (\textit{mp\_reduce\_setup}). \\
5.  $M_1 \leftarrow g \mbox{ (mod }p\mbox{)}$ \\
\\
Setup the table of small powers of $g$.  First find $g^{2^{winsize}}$ and then all multiples of it. \\
6.  $k \leftarrow 2^{winsize - 1}$ \\
7.  $M_{k} \leftarrow M_1$ \\
8.  for $ix$ from 0 to $winsize - 2$ do \\
\hspace{3mm}8.1  $M_k \leftarrow \left ( M_k \right )^2$ (\textit{mp\_sqr})  \\
\hspace{3mm}8.2  $M_k \leftarrow M_k \mbox{ (mod }p\mbox{)}$ (\textit{mp\_reduce}) \\
9.  for $ix$ from $2^{winsize - 1} + 1$ to $2^{winsize} - 1$ do \\
\hspace{3mm}9.1  $M_{ix} \leftarrow M_{ix - 1} \cdot M_{1}$ (\textit{mp\_mul}) \\
\hspace{3mm}9.2  $M_{ix} \leftarrow M_{ix} \mbox{ (mod }p\mbox{)}$ (\textit{mp\_reduce}) \\
10.  $res \leftarrow 1$ \\
\\
Start Sliding Window. \\
11.  $mode \leftarrow 0, bitcnt \leftarrow 1, buf \leftarrow 0, digidx \leftarrow x.used - 1, bitcpy \leftarrow 0, bitbuf \leftarrow 0$ \\
12.  Loop \\
\hspace{3mm}12.1  $bitcnt \leftarrow bitcnt - 1$ \\
\hspace{3mm}12.2  If $bitcnt = 0$ then do \\
\hspace{6mm}12.2.1  If $digidx = -1$ goto step 13. \\
\hspace{6mm}12.2.2  $buf \leftarrow x_{digidx}$ \\
\hspace{6mm}12.2.3  $digidx \leftarrow digidx - 1$ \\
\hspace{6mm}12.2.4  $bitcnt \leftarrow lg(\beta)$ \\
Continued on next page. \\
\hline
\end{tabular}
\end{center}
\end{small}
\caption{Algorithm s\_mp\_exptmod}
\end{figure}

\newpage\begin{figure}[!here]
\begin{small}
\begin{center}
\begin{tabular}{l}
\hline Algorithm \textbf{s\_mp\_exptmod} (\textit{continued}). \\
\textbf{Input}.   mp\_int $a$, $b$ and $c$ \\
\textbf{Output}.  $y \equiv g^x \mbox{ (mod }p\mbox{)}$ \\
\hline \\
\hspace{3mm}12.3  $y \leftarrow (buf >> (lg(\beta) - 1))$ AND $1$ \\
\hspace{3mm}12.4  $buf \leftarrow buf << 1$ \\
\hspace{3mm}12.5  if $mode = 0$ and $y = 0$ then goto step 12. \\
\hspace{3mm}12.6  if $mode = 1$ and $y = 0$ then do \\
\hspace{6mm}12.6.1  $res \leftarrow res^2$ \\
\hspace{6mm}12.6.2  $res \leftarrow res \mbox{ (mod }p\mbox{)}$ \\
\hspace{6mm}12.6.3  Goto step 12. \\
\hspace{3mm}12.7  $bitcpy \leftarrow bitcpy + 1$ \\
\hspace{3mm}12.8  $bitbuf \leftarrow bitbuf + (y << (winsize - bitcpy))$ \\
\hspace{3mm}12.9  $mode \leftarrow 2$ \\
\hspace{3mm}12.10  If $bitcpy = winsize$ then do \\
\hspace{6mm}Window is full so perform the squarings and single multiplication. \\
\hspace{6mm}12.10.1  for $ix$ from $0$ to $winsize -1$ do \\
\hspace{9mm}12.10.1.1  $res \leftarrow res^2$ \\
\hspace{9mm}12.10.1.2  $res \leftarrow res \mbox{ (mod }p\mbox{)}$ \\
\hspace{6mm}12.10.2  $res \leftarrow res \cdot M_{bitbuf}$ \\
\hspace{6mm}12.10.3  $res \leftarrow res \mbox{ (mod }p\mbox{)}$ \\
\hspace{6mm}Reset the window. \\
\hspace{6mm}12.10.4  $bitcpy \leftarrow 0, bitbuf \leftarrow 0, mode \leftarrow 1$ \\
\\
No more windows left.  Check for residual bits of exponent. \\
13.  If $mode = 2$ and $bitcpy > 0$ then do \\
\hspace{3mm}13.1  for $ix$ form $0$ to $bitcpy - 1$ do \\
\hspace{6mm}13.1.1  $res \leftarrow res^2$ \\
\hspace{6mm}13.1.2  $res \leftarrow res \mbox{ (mod }p\mbox{)}$ \\
\hspace{6mm}13.1.3  $bitbuf \leftarrow bitbuf << 1$ \\
\hspace{6mm}13.1.4  If $bitbuf$ AND $2^{winsize} \ne 0$ then do \\
\hspace{9mm}13.1.4.1  $res \leftarrow res \cdot M_{1}$ \\
\hspace{9mm}13.1.4.2  $res \leftarrow res \mbox{ (mod }p\mbox{)}$ \\
14.  $y \leftarrow res$ \\
15.  Clear $res$, $mu$ and the $M$ array. \\
16.  Return(\textit{MP\_OKAY}). \\
\hline
\end{tabular}
\end{center}
\end{small}
\caption{Algorithm s\_mp\_exptmod (continued)}
\end{figure}

\textbf{Algorithm s\_mp\_exptmod.}
This algorithm computes the $x$'th power of $g$ modulo $p$ and stores the result in $y$.  It takes advantage of the Barrett reduction
algorithm to keep the product small throughout the algorithm.

The first two steps determine the optimal window size based on the number of bits in the exponent.  The larger the exponent the 
larger the window size becomes.  After a window size $winsize$ has been chosen an array of $2^{winsize}$ mp\_int variables is allocated.  This
table will hold the values of $g^x \mbox{ (mod }p\mbox{)}$ for $2^{winsize - 1} \le x < 2^{winsize}$.  

After the table is allocated the first power of $g$ is found.  Since $g \ge p$ is allowed it must be first reduced modulo $p$ to make
the rest of the algorithm more efficient.  The first element of the table at $2^{winsize - 1}$ is found by squaring $M_1$ successively $winsize - 2$
times.  The rest of the table elements are found by multiplying the previous element by $M_1$ modulo $p$.

Now that the table is available the sliding window may begin.  The following list describes the functions of all the variables in the window.
\begin{enumerate}
\item The variable $mode$ dictates how the bits of the exponent are interpreted.  
\begin{enumerate}
   \item When $mode = 0$ the bits are ignored since no non-zero bit of the exponent has been seen yet.  For example, if the exponent were simply 
         $1$ then there would be $lg(\beta) - 1$ zero bits before the first non-zero bit.  In this case bits are ignored until a non-zero bit is found.  
   \item When $mode = 1$ a non-zero bit has been seen before and a new $winsize$-bit window has not been formed yet.  In this mode leading $0$ bits 
         are read and a single squaring is performed.  If a non-zero bit is read a new window is created.  
   \item When $mode = 2$ the algorithm is in the middle of forming a window and new bits are appended to the window from the most significant bit
         downwards.
\end{enumerate}
\item The variable $bitcnt$ indicates how many bits are left in the current digit of the exponent left to be read.  When it reaches zero a new digit
      is fetched from the exponent.
\item The variable $buf$ holds the currently read digit of the exponent. 
\item The variable $digidx$ is an index into the exponents digits.  It starts at the leading digit $x.used - 1$ and moves towards the trailing digit.
\item The variable $bitcpy$ indicates how many bits are in the currently formed window.  When it reaches $winsize$ the window is flushed and
      the appropriate operations performed.
\item The variable $bitbuf$ holds the current bits of the window being formed.  
\end{enumerate}

All of step 12 is the window processing loop.  It will iterate while there are digits available form the exponent to read.  The first step
inside this loop is to extract a new digit if no more bits are available in the current digit.  If there are no bits left a new digit is
read and if there are no digits left than the loop terminates.  

After a digit is made available step 12.3 will extract the most significant bit of the current digit and move all other bits in the digit
upwards.  In effect the digit is read from most significant bit to least significant bit and since the digits are read from leading to 
trailing edges the entire exponent is read from most significant bit to least significant bit.

At step 12.5 if the $mode$ and currently extracted bit $y$ are both zero the bit is ignored and the next bit is read.  This prevents the 
algorithm from having to perform trivial squaring and reduction operations before the first non-zero bit is read.  Step 12.6 and 12.7-10 handle
the two cases of $mode = 1$ and $mode = 2$ respectively.  

\begin{center}
\begin{figure}[here]
\includegraphics{pics/expt_state.ps}
\caption{Sliding Window State Diagram}
\label{pic:expt_state}
\end{figure}
\end{center}

By step 13 there are no more digits left in the exponent.  However, there may be partial bits in the window left.  If $mode = 2$ then 
a Left-to-Right algorithm is used to process the remaining few bits.  

\vspace{+3mm}\begin{small}
\hspace{-5.1mm}{\bf File}: bn\_s\_mp\_exptmod.c
\vspace{-3mm}
\begin{alltt}
\end{alltt}
\end{small}

Lines 32 through 46 determine the optimal window size based on the length of the exponent in bits.  The window divisions are sorted
from smallest to greatest so that in each \textbf{if} statement only one condition must be tested.  For example, by the \textbf{if} statement 
on line 38 the value of $x$ is already known to be greater than $140$.  

The conditional piece of code beginning on line 48 allows the window size to be restricted to five bits.  This logic is used to ensure
the table of precomputed powers of $G$ remains relatively small.  

The for loop on line 61 initializes the $M$ array while lines 72 and 77 through 86 initialize the reduction
function that will be used for this modulus.

-- More later.

\section{Quick Power of Two}
Calculating $b = 2^a$ can be performed much quicker than with any of the previous algorithms.  Recall that a logical shift left $m << k$ is
equivalent to $m \cdot 2^k$.  By this logic when $m = 1$ a quick power of two can be achieved.

\begin{figure}[!here]
\begin{small}
\begin{center}
\begin{tabular}{l}
\hline Algorithm \textbf{mp\_2expt}. \\
\textbf{Input}.   integer $b$ \\
\textbf{Output}.  $a \leftarrow 2^b$ \\
\hline \\
1.  $a \leftarrow 0$ \\
2.  If $a.alloc < \lfloor b / lg(\beta) \rfloor + 1$ then grow $a$ appropriately. \\
3.  $a.used \leftarrow \lfloor b / lg(\beta) \rfloor + 1$ \\
4.  $a_{\lfloor b / lg(\beta) \rfloor} \leftarrow 1 << (b \mbox{ mod } lg(\beta))$ \\
5.  Return(\textit{MP\_OKAY}). \\
\hline
\end{tabular}
\end{center}
\end{small}
\caption{Algorithm mp\_2expt}
\end{figure}

\textbf{Algorithm mp\_2expt.}

\vspace{+3mm}\begin{small}
\hspace{-5.1mm}{\bf File}: bn\_mp\_2expt.c
\vspace{-3mm}
\begin{alltt}
\end{alltt}
\end{small}

\chapter{Higher Level Algorithms}

This chapter discusses the various higher level algorithms that are required to complete a well rounded multiple precision integer package.  These
routines are less performance oriented than the algorithms of chapters five, six and seven but are no less important.  

The first section describes a method of integer division with remainder that is universally well known.  It provides the signed division logic
for the package.  The subsequent section discusses a set of algorithms which allow a single digit to be the 2nd operand for a variety of operations.  
These algorithms serve mostly to simplify other algorithms where small constants are required.  The last two sections discuss how to manipulate 
various representations of integers.  For example, converting from an mp\_int to a string of character.

\section{Integer Division with Remainder}
\label{sec:division}

Integer division aside from modular exponentiation is the most intensive algorithm to compute.  Like addition, subtraction and multiplication
the basis of this algorithm is the long-hand division algorithm taught to school children.  Throughout this discussion several common variables
will be used.  Let $x$ represent the divisor and $y$ represent the dividend.  Let $q$ represent the integer quotient $\lfloor y / x \rfloor$ and 
let $r$ represent the remainder $r = y - x \lfloor y / x \rfloor$.  The following simple algorithm will be used to start the discussion.

\newpage\begin{figure}[!here]
\begin{small}
\begin{center}
\begin{tabular}{l}
\hline Algorithm \textbf{Radix-$\beta$ Integer Division}. \\
\textbf{Input}.   integer $x$ and $y$ \\
\textbf{Output}.  $q = \lfloor y/x\rfloor, r = y - xq$ \\
\hline \\
1.  $q \leftarrow 0$ \\
2.  $n \leftarrow \vert \vert y \vert \vert - \vert \vert x \vert \vert$ \\
3.  for $t$ from $n$ down to $0$ do \\
\hspace{3mm}3.1  Maximize $k$ such that $kx\beta^t$ is less than or equal to $y$ and $(k + 1)x\beta^t$ is greater. \\
\hspace{3mm}3.2  $q \leftarrow q + k\beta^t$ \\
\hspace{3mm}3.3  $y \leftarrow y - kx\beta^t$ \\
4.  $r \leftarrow y$ \\
5.  Return($q, r$) \\
\hline
\end{tabular}
\end{center}
\end{small}
\caption{Algorithm Radix-$\beta$ Integer Division}
\label{fig:raddiv}
\end{figure}

As children we are taught this very simple algorithm for the case of $\beta = 10$.  Almost instinctively several optimizations are taught for which
their reason of existing are never explained.  For this example let $y = 5471$ represent the dividend and $x = 23$ represent the divisor.

To find the first digit of the quotient the value of $k$ must be maximized such that $kx\beta^t$ is less than or equal to $y$ and 
simultaneously $(k + 1)x\beta^t$ is greater than $y$.  Implicitly $k$ is the maximum value the $t$'th digit of the quotient may have.  The habitual method
used to find the maximum is to ``eyeball'' the two numbers, typically only the leading digits and quickly estimate a quotient.  By only using leading
digits a much simpler division may be used to form an educated guess at what the value must be.  In this case $k = \lfloor 54/23\rfloor = 2$ quickly 
arises as a possible  solution.  Indeed $2x\beta^2 = 4600$ is less than $y = 5471$ and simultaneously $(k + 1)x\beta^2 = 6900$ is larger than $y$.  
As a  result $k\beta^2$ is added to the quotient which now equals $q = 200$ and $4600$ is subtracted from $y$ to give a remainder of $y = 841$.

Again this process is repeated to produce the quotient digit $k = 3$ which makes the quotient $q = 200 + 3\beta = 230$ and the remainder 
$y = 841 - 3x\beta = 181$.  Finally the last iteration of the loop produces $k = 7$ which leads to the quotient $q = 230 + 7 = 237$ and the
remainder $y = 181 - 7x = 20$.  The final quotient and remainder found are $q = 237$ and $r = y = 20$ which are indeed correct since 
$237 \cdot 23 + 20 = 5471$ is true.  

\subsection{Quotient Estimation}
\label{sec:divest}
As alluded to earlier the quotient digit $k$ can be estimated from only the leading digits of both the divisor and dividend.  When $p$ leading
digits are used from both the divisor and dividend to form an estimation the accuracy of the estimation rises as $p$ grows.  Technically
speaking the estimation is based on assuming the lower $\vert \vert y \vert \vert - p$ and $\vert \vert x \vert \vert - p$ lower digits of the
dividend and divisor are zero.  

The value of the estimation may off by a few values in either direction and in general is fairly correct.  A simplification \cite[pp. 271]{TAOCPV2}
of the estimation technique is to use $t + 1$ digits of the dividend and $t$ digits of the divisor, in particularly when $t = 1$.  The estimate 
using this technique is never too small.  For the following proof let $t = \vert \vert y \vert \vert - 1$ and $s = \vert \vert x \vert \vert - 1$ 
represent the most significant digits of the dividend and divisor respectively.

\textbf{Proof.}\textit{  The quotient $\hat k = \lfloor (y_t\beta + y_{t-1}) / x_s \rfloor$ is greater than or equal to 
$k = \lfloor y / (x \cdot \beta^{\vert \vert y \vert \vert - \vert \vert x \vert \vert - 1}) \rfloor$. }
The first obvious case is when $\hat k = \beta - 1$ in which case the proof is concluded since the real quotient cannot be larger.  For all other 
cases $\hat k = \lfloor (y_t\beta + y_{t-1}) / x_s \rfloor$ and $\hat k x_s \ge y_t\beta + y_{t-1} - x_s + 1$.  The latter portion of the inequalility
$-x_s + 1$ arises from the fact that a truncated integer division will give the same quotient for at most $x_s - 1$ values.  Next a series of 
inequalities will prove the hypothesis.

\begin{equation}
y - \hat k x \le y - \hat k x_s\beta^s
\end{equation}

This is trivially true since $x \ge x_s\beta^s$.  Next we replace $\hat kx_s\beta^s$ by the previous inequality for $\hat kx_s$.  

\begin{equation}
y - \hat k x \le y_t\beta^t + \ldots + y_0 - (y_t\beta^t + y_{t-1}\beta^{t-1} - x_s\beta^t + \beta^s)
\end{equation}

By simplifying the previous inequality the following inequality is formed.

\begin{equation}
y - \hat k x \le y_{t-2}\beta^{t-2} + \ldots + y_0 + x_s\beta^s - \beta^s
\end{equation}

Subsequently,

\begin{equation}
y_{t-2}\beta^{t-2} + \ldots +  y_0  + x_s\beta^s - \beta^s < x_s\beta^s \le x
\end{equation}

Which proves that $y - \hat kx \le x$ and by consequence $\hat k \ge k$ which concludes the proof.  \textbf{QED}


\subsection{Normalized Integers}
For the purposes of division a normalized input is when the divisors leading digit $x_n$ is greater than or equal to $\beta / 2$.  By multiplying both
$x$ and $y$ by $j = \lfloor (\beta / 2) / x_n \rfloor$ the quotient remains unchanged and the remainder is simply $j$ times the original
remainder.  The purpose of normalization is to ensure the leading digit of the divisor is sufficiently large such that the estimated quotient will
lie in the domain of a single digit.  Consider the maximum dividend $(\beta - 1) \cdot \beta + (\beta - 1)$ and the minimum divisor $\beta / 2$.  

\begin{equation} 
{{\beta^2 - 1} \over { \beta / 2}} \le 2\beta - {2 \over \beta} 
\end{equation}

At most the quotient approaches $2\beta$, however, in practice this will not occur since that would imply the previous quotient digit was too small.  

\subsection{Radix-$\beta$ Division with Remainder}
\newpage\begin{figure}[!here]
\begin{small}
\begin{center}
\begin{tabular}{l}
\hline Algorithm \textbf{mp\_div}. \\
\textbf{Input}.   mp\_int $a, b$ \\
\textbf{Output}.  $c = \lfloor a/b \rfloor$, $d = a - bc$ \\
\hline \\
1.  If $b = 0$ return(\textit{MP\_VAL}). \\
2.  If $\vert a \vert < \vert b \vert$ then do \\
\hspace{3mm}2.1  $d \leftarrow a$ \\
\hspace{3mm}2.2  $c \leftarrow 0$ \\
\hspace{3mm}2.3  Return(\textit{MP\_OKAY}). \\
\\
Setup the quotient to receive the digits. \\
3.  Grow $q$ to $a.used + 2$ digits. \\
4.  $q \leftarrow 0$ \\
5.  $x \leftarrow \vert a \vert , y \leftarrow \vert b \vert$ \\
6.  $sign \leftarrow  \left \lbrace \begin{array}{ll}
                              MP\_ZPOS &  \mbox{if }a.sign = b.sign \\
                              MP\_NEG  &  \mbox{otherwise} \\
                              \end{array} \right .$ \\
\\
Normalize the inputs such that the leading digit of $y$ is greater than or equal to $\beta / 2$. \\
7.  $norm \leftarrow (lg(\beta) - 1) - (\lceil lg(y) \rceil \mbox{ (mod }lg(\beta)\mbox{)})$ \\
8.  $x \leftarrow x \cdot 2^{norm}, y \leftarrow y \cdot 2^{norm}$ \\
\\
Find the leading digit of the quotient. \\
9.  $n \leftarrow x.used - 1, t \leftarrow y.used - 1$ \\
10.  $y \leftarrow y \cdot \beta^{n - t}$ \\
11.  While ($x \ge y$) do \\
\hspace{3mm}11.1  $q_{n - t} \leftarrow q_{n - t} + 1$ \\
\hspace{3mm}11.2  $x \leftarrow x - y$ \\
12.  $y \leftarrow \lfloor y / \beta^{n-t} \rfloor$ \\
\\
Continued on the next page. \\
\hline
\end{tabular}
\end{center}
\end{small}
\caption{Algorithm mp\_div}
\end{figure}

\newpage\begin{figure}[!here]
\begin{small}
\begin{center}
\begin{tabular}{l}
\hline Algorithm \textbf{mp\_div} (continued). \\
\textbf{Input}.   mp\_int $a, b$ \\
\textbf{Output}.  $c = \lfloor a/b \rfloor$, $d = a - bc$ \\
\hline \\
Now find the remainder fo the digits. \\
13.  for $i$ from $n$ down to $(t + 1)$ do \\
\hspace{3mm}13.1  If $i > x.used$ then jump to the next iteration of this loop. \\
\hspace{3mm}13.2  If $x_{i} = y_{t}$ then \\
\hspace{6mm}13.2.1  $q_{i - t - 1} \leftarrow \beta - 1$ \\
\hspace{3mm}13.3  else \\
\hspace{6mm}13.3.1  $\hat r \leftarrow x_{i} \cdot \beta + x_{i - 1}$ \\
\hspace{6mm}13.3.2  $\hat r \leftarrow \lfloor \hat r / y_{t} \rfloor$ \\
\hspace{6mm}13.3.3  $q_{i - t - 1} \leftarrow \hat r$ \\
\hspace{3mm}13.4  $q_{i - t - 1} \leftarrow q_{i - t - 1} + 1$ \\
\\
Fixup quotient estimation. \\
\hspace{3mm}13.5  Loop \\
\hspace{6mm}13.5.1  $q_{i - t - 1} \leftarrow q_{i - t - 1} - 1$ \\
\hspace{6mm}13.5.2  t$1 \leftarrow 0$ \\
\hspace{6mm}13.5.3  t$1_0 \leftarrow y_{t - 1}, $ t$1_1 \leftarrow y_t,$ t$1.used \leftarrow 2$ \\
\hspace{6mm}13.5.4  $t1 \leftarrow t1 \cdot q_{i - t - 1}$ \\
\hspace{6mm}13.5.5  t$2_0 \leftarrow x_{i - 2}, $ t$2_1 \leftarrow x_{i - 1}, $ t$2_2 \leftarrow x_i, $ t$2.used \leftarrow 3$ \\
\hspace{6mm}13.5.6  If $\vert t1 \vert > \vert t2 \vert$ then goto step 13.5. \\
\hspace{3mm}13.6  t$1 \leftarrow y \cdot q_{i - t - 1}$ \\
\hspace{3mm}13.7  t$1 \leftarrow $ t$1 \cdot \beta^{i - t - 1}$ \\
\hspace{3mm}13.8  $x \leftarrow x - $ t$1$ \\
\hspace{3mm}13.9  If $x.sign = MP\_NEG$ then \\
\hspace{6mm}13.10  t$1 \leftarrow y$ \\
\hspace{6mm}13.11  t$1 \leftarrow $ t$1 \cdot \beta^{i - t - 1}$ \\
\hspace{6mm}13.12  $x \leftarrow x + $ t$1$ \\
\hspace{6mm}13.13  $q_{i - t - 1} \leftarrow q_{i - t - 1} - 1$ \\
\\
Finalize the result. \\
14.  Clamp excess digits of $q$ \\
15.  $c \leftarrow q, c.sign \leftarrow sign$ \\
16.  $x.sign \leftarrow a.sign$ \\
17.  $d \leftarrow \lfloor x / 2^{norm} \rfloor$ \\
18.  Return(\textit{MP\_OKAY}). \\
\hline
\end{tabular}
\end{center}
\end{small}
\caption{Algorithm mp\_div (continued)}
\end{figure}
\textbf{Algorithm mp\_div.}
This algorithm will calculate quotient and remainder from an integer division given a dividend and divisor.  The algorithm is a signed
division and will produce a fully qualified quotient and remainder.

First the divisor $b$ must be non-zero which is enforced in step one.  If the divisor is larger than the dividend than the quotient is implicitly 
zero and the remainder is the dividend.  

After the first two trivial cases of inputs are handled the variable $q$ is setup to receive the digits of the quotient.  Two unsigned copies of the
divisor $y$ and dividend $x$ are made as well.  The core of the division algorithm is an unsigned division and will only work if the values are
positive.  Now the two values $x$ and $y$ must be normalized such that the leading digit of $y$ is greater than or equal to $\beta / 2$.  
This is performed by shifting both to the left by enough bits to get the desired normalization.  

At this point the division algorithm can begin producing digits of the quotient.  Recall that maximum value of the estimation used is 
$2\beta - {2 \over \beta}$ which means that a digit of the quotient must be first produced by another means.  In this case $y$ is shifted
to the left (\textit{step ten}) so that it has the same number of digits as $x$.  The loop on step eleven will subtract multiples of the 
shifted copy of $y$ until $x$ is smaller.  Since the leading digit of $y$ is greater than or equal to $\beta/2$ this loop will iterate at most two
times to produce the desired leading digit of the quotient.  

Now the remainder of the digits can be produced.  The equation $\hat q = \lfloor {{x_i \beta + x_{i-1}}\over y_t} \rfloor$ is used to fairly
accurately approximate the true quotient digit.  The estimation can in theory produce an estimation as high as $2\beta - {2 \over \beta}$ but by
induction the upper quotient digit is correct (\textit{as established on step eleven}) and the estimate must be less than $\beta$.  

Recall from section~\ref{sec:divest} that the estimation is never too low but may be too high.  The next step of the estimation process is
to refine the estimation.  The loop on step 13.5 uses $x_i\beta^2 + x_{i-1}\beta + x_{i-2}$ and $q_{i - t - 1}(y_t\beta + y_{t-1})$ as a higher
order approximation to adjust the quotient digit.

After both phases of estimation the quotient digit may still be off by a value of one\footnote{This is similar to the error introduced
by optimizing Barrett reduction.}.  Steps 13.6 and 13.7 subtract the multiple of the divisor from the dividend (\textit{Similar to step 3.3 of
algorithm~\ref{fig:raddiv}} and then subsequently add a multiple of the divisor if the quotient was too large.  

Now that the quotient has been determine finializing the result is a matter of clamping the quotient, fixing the sizes and de-normalizing the 
remainder.  An important aspect of this algorithm seemingly overlooked in other descriptions such as that of Algorithm 14.20 HAC \cite[pp. 598]{HAC}
is that when the estimations are being made (\textit{inside the loop on step 13.5}) that the digits $y_{t-1}$, $x_{i-2}$ and $x_{i-1}$ may lie 
outside their respective boundaries.  For example, if $t = 0$ or $i \le 1$ then the digits would be undefined.  In those cases the digits should
respectively be replaced with a zero.  

\vspace{+3mm}\begin{small}
\hspace{-5.1mm}{\bf File}: bn\_mp\_div.c
\vspace{-3mm}
\begin{alltt}
\end{alltt}
\end{small}

The implementation of this algorithm differs slightly from the pseudo code presented previously.  In this algorithm either of the quotient $c$ or
remainder $d$ may be passed as a \textbf{NULL} pointer which indicates their value is not desired.  For example, the C code to call the division
algorithm with only the quotient is 

\begin{verbatim}
mp_div(&a, &b, &c, NULL);  /* c = [a/b] */
\end{verbatim}

Lines 109 and 113 handle the two trivial cases of inputs which are division by zero and dividend smaller than the divisor 
respectively.  After the two trivial cases all of the temporary variables are initialized.  Line 148 determines the sign of 
the quotient and line 148 ensures that both $x$ and $y$ are positive.  

The number of bits in the leading digit is calculated on line 151.  Implictly an mp\_int with $r$ digits will require $lg(\beta)(r-1) + k$ bits
of precision which when reduced modulo $lg(\beta)$ produces the value of $k$.  In this case $k$ is the number of bits in the leading digit which is
exactly what is required.  For the algorithm to operate $k$ must equal $lg(\beta) - 1$ and when it does not the inputs must be normalized by shifting
them to the left by $lg(\beta) - 1 - k$ bits.

Throughout the variables $n$ and $t$ will represent the highest digit of $x$ and $y$ respectively.  These are first used to produce the 
leading digit of the quotient.  The loop beginning on line 184 will produce the remainder of the quotient digits.

The conditional ``continue'' on line 187 is used to prevent the algorithm from reading past the leading edge of $x$ which can occur when the
algorithm eliminates multiple non-zero digits in a single iteration.  This ensures that $x_i$ is always non-zero since by definition the digits
above the $i$'th position $x$ must be zero in order for the quotient to be precise\footnote{Precise as far as integer division is concerned.}.  

Lines 214, 216 and 223 through 225 manually construct the high accuracy estimations by setting the digits of the two mp\_int 
variables directly.  

\section{Single Digit Helpers}

This section briefly describes a series of single digit helper algorithms which come in handy when working with small constants.  All of 
the helper functions assume the single digit input is positive and will treat them as such.

\subsection{Single Digit Addition and Subtraction}

Both addition and subtraction are performed by ``cheating'' and using mp\_set followed by the higher level addition or subtraction 
algorithms.   As a result these algorithms are subtantially simpler with a slight cost in performance.

\newpage\begin{figure}[!here]
\begin{small}
\begin{center}
\begin{tabular}{l}
\hline Algorithm \textbf{mp\_add\_d}. \\
\textbf{Input}.   mp\_int $a$ and a mp\_digit $b$ \\
\textbf{Output}.  $c = a + b$ \\
\hline \\
1.  $t \leftarrow b$ (\textit{mp\_set}) \\
2.  $c \leftarrow a + t$ \\
3.  Return(\textit{MP\_OKAY}) \\
\hline
\end{tabular}
\end{center}
\end{small}
\caption{Algorithm mp\_add\_d}
\end{figure}

\textbf{Algorithm mp\_add\_d.}
This algorithm initiates a temporary mp\_int with the value of the single digit and uses algorithm mp\_add to add the two values together.

\vspace{+3mm}\begin{small}
\hspace{-5.1mm}{\bf File}: bn\_mp\_add\_d.c
\vspace{-3mm}
\begin{alltt}
\end{alltt}
\end{small}

Clever use of the letter 't'.

\subsubsection{Subtraction}
The single digit subtraction algorithm mp\_sub\_d is essentially the same except it uses mp\_sub to subtract the digit from the mp\_int.

\subsection{Single Digit Multiplication}
Single digit multiplication arises enough in division and radix conversion that it ought to be implement as a special case of the baseline
multiplication algorithm.  Essentially this algorithm is a modified version of algorithm s\_mp\_mul\_digs where one of the multiplicands
only has one digit.

\begin{figure}[!here]
\begin{small}
\begin{center}
\begin{tabular}{l}
\hline Algorithm \textbf{mp\_mul\_d}. \\
\textbf{Input}.   mp\_int $a$ and a mp\_digit $b$ \\
\textbf{Output}.  $c = ab$ \\
\hline \\
1.  $pa \leftarrow a.used$ \\
2.  Grow $c$ to at least $pa + 1$ digits. \\
3.  $oldused \leftarrow c.used$ \\
4.  $c.used \leftarrow pa + 1$ \\
5.  $c.sign \leftarrow a.sign$ \\
6.  $\mu \leftarrow 0$ \\
7.  for $ix$ from $0$ to $pa - 1$ do \\
\hspace{3mm}7.1  $\hat r \leftarrow \mu + a_{ix}b$ \\
\hspace{3mm}7.2  $c_{ix} \leftarrow \hat r \mbox{ (mod }\beta\mbox{)}$ \\
\hspace{3mm}7.3  $\mu \leftarrow \lfloor \hat r / \beta \rfloor$ \\
8.  $c_{pa} \leftarrow \mu$ \\
9.  for $ix$ from $pa + 1$ to $oldused$ do \\
\hspace{3mm}9.1  $c_{ix} \leftarrow 0$ \\
10.  Clamp excess digits of $c$. \\
11.  Return(\textit{MP\_OKAY}). \\
\hline
\end{tabular}
\end{center}
\end{small}
\caption{Algorithm mp\_mul\_d}
\end{figure}
\textbf{Algorithm mp\_mul\_d.}
This algorithm quickly multiplies an mp\_int by a small single digit value.  It is specially tailored to the job and has a minimal of overhead.  
Unlike the full multiplication algorithms this algorithm does not require any significnat temporary storage or memory allocations.  

\vspace{+3mm}\begin{small}
\hspace{-5.1mm}{\bf File}: bn\_mp\_mul\_d.c
\vspace{-3mm}
\begin{alltt}
\end{alltt}
\end{small}

In this implementation the destination $c$ may point to the same mp\_int as the source $a$ since the result is written after the digit is 
read from the source.  This function uses pointer aliases $tmpa$ and $tmpc$ for the digits of $a$ and $c$ respectively.  

\subsection{Single Digit Division}
Like the single digit multiplication algorithm, single digit division is also a fairly common algorithm used in radix conversion.  Since the
divisor is only a single digit a specialized variant of the division algorithm can be used to compute the quotient.  

\newpage\begin{figure}[!here]
\begin{small}
\begin{center}
\begin{tabular}{l}
\hline Algorithm \textbf{mp\_div\_d}. \\
\textbf{Input}.   mp\_int $a$ and a mp\_digit $b$ \\
\textbf{Output}.  $c = \lfloor a / b \rfloor, d = a - cb$ \\
\hline \\
1.  If $b = 0$ then return(\textit{MP\_VAL}).\\
2.  If $b = 3$ then use algorithm mp\_div\_3 instead. \\
3.  Init $q$ to $a.used$ digits.  \\
4.  $q.used \leftarrow a.used$ \\
5.  $q.sign \leftarrow a.sign$ \\
6.  $\hat w \leftarrow 0$ \\
7.  for $ix$ from $a.used - 1$ down to $0$ do \\
\hspace{3mm}7.1  $\hat w \leftarrow \hat w \beta + a_{ix}$ \\
\hspace{3mm}7.2  If $\hat w \ge b$ then \\
\hspace{6mm}7.2.1  $t \leftarrow \lfloor \hat w / b \rfloor$ \\
\hspace{6mm}7.2.2  $\hat w \leftarrow \hat w \mbox{ (mod }b\mbox{)}$ \\
\hspace{3mm}7.3  else\\
\hspace{6mm}7.3.1  $t \leftarrow 0$ \\
\hspace{3mm}7.4  $q_{ix} \leftarrow t$ \\
8.  $d \leftarrow \hat w$ \\
9.  Clamp excess digits of $q$. \\
10.  $c \leftarrow q$ \\
11.  Return(\textit{MP\_OKAY}). \\
\hline
\end{tabular}
\end{center}
\end{small}
\caption{Algorithm mp\_div\_d}
\end{figure}
\textbf{Algorithm mp\_div\_d.}
This algorithm divides the mp\_int $a$ by the single mp\_digit $b$ using an optimized approach.  Essentially in every iteration of the
algorithm another digit of the dividend is reduced and another digit of quotient produced.  Provided $b < \beta$ the value of $\hat w$
after step 7.1 will be limited such that $0 \le \lfloor \hat w / b \rfloor < \beta$.  

If the divisor $b$ is equal to three a variant of this algorithm is used which is called mp\_div\_3.  It replaces the division by three with
a multiplication by $\lfloor \beta / 3 \rfloor$ and the appropriate shift and residual fixup.  In essence it is much like the Barrett reduction
from chapter seven.  

\vspace{+3mm}\begin{small}
\hspace{-5.1mm}{\bf File}: bn\_mp\_div\_d.c
\vspace{-3mm}
\begin{alltt}
\end{alltt}
\end{small}

Like the implementation of algorithm mp\_div this algorithm allows either of the quotient or remainder to be passed as a \textbf{NULL} pointer to
indicate the respective value is not required.  This allows a trivial single digit modular reduction algorithm, mp\_mod\_d to be created.

The division and remainder on lines 44 and @45,%@ can be replaced often by a single division on most processors.  For example, the 32-bit x86 based 
processors can divide a 64-bit quantity by a 32-bit quantity and produce the quotient and remainder simultaneously.  Unfortunately the GCC 
compiler does not recognize that optimization and will actually produce two function calls to find the quotient and remainder respectively.  

\subsection{Single Digit Root Extraction}

Finding the $n$'th root of an integer is fairly easy as far as numerical analysis is concerned.  Algorithms such as the Newton-Raphson approximation 
(\ref{eqn:newton}) series will converge very quickly to a root for any continuous function $f(x)$.  

\begin{equation}
x_{i+1} = x_i - {f(x_i) \over f'(x_i)}
\label{eqn:newton}
\end{equation}

In this case the $n$'th root is desired and $f(x) = x^n - a$ where $a$ is the integer of which the root is desired.  The derivative of $f(x)$ is 
simply $f'(x) = nx^{n - 1}$.  Of particular importance is that this algorithm will be used over the integers not over the a more continuous domain
such as the real numbers.  As a result the root found can be above the true root by few and must be manually adjusted.  Ideally at the end of the 
algorithm the $n$'th root $b$ of an integer $a$ is desired such that $b^n \le a$.  

\newpage\begin{figure}[!here]
\begin{small}
\begin{center}
\begin{tabular}{l}
\hline Algorithm \textbf{mp\_n\_root}. \\
\textbf{Input}.   mp\_int $a$ and a mp\_digit $b$ \\
\textbf{Output}.  $c^b \le a$ \\
\hline \\
1.  If $b$ is even and $a.sign = MP\_NEG$ return(\textit{MP\_VAL}). \\
2.  $sign \leftarrow a.sign$ \\
3.  $a.sign \leftarrow MP\_ZPOS$ \\
4.  t$2 \leftarrow 2$ \\
5.  Loop \\
\hspace{3mm}5.1  t$1 \leftarrow $ t$2$ \\
\hspace{3mm}5.2  t$3 \leftarrow $ t$1^{b - 1}$ \\
\hspace{3mm}5.3  t$2 \leftarrow $ t$3 $ $\cdot$ t$1$ \\
\hspace{3mm}5.4  t$2 \leftarrow $ t$2 - a$ \\
\hspace{3mm}5.5  t$3 \leftarrow $ t$3 \cdot b$ \\
\hspace{3mm}5.6  t$3 \leftarrow \lfloor $t$2 / $t$3 \rfloor$ \\
\hspace{3mm}5.7  t$2 \leftarrow $ t$1 - $ t$3$ \\
\hspace{3mm}5.8  If t$1 \ne $ t$2$ then goto step 5.  \\
6.  Loop \\
\hspace{3mm}6.1  t$2 \leftarrow $ t$1^b$ \\
\hspace{3mm}6.2  If t$2 > a$ then \\
\hspace{6mm}6.2.1  t$1 \leftarrow $ t$1 - 1$ \\
\hspace{6mm}6.2.2  Goto step 6. \\
7.  $a.sign \leftarrow sign$ \\
8.  $c \leftarrow $ t$1$ \\
9.  $c.sign \leftarrow sign$  \\
10.  Return(\textit{MP\_OKAY}).  \\
\hline
\end{tabular}
\end{center}
\end{small}
\caption{Algorithm mp\_n\_root}
\end{figure}
\textbf{Algorithm mp\_n\_root.}
This algorithm finds the integer $n$'th root of an input using the Newton-Raphson approach.  It is partially optimized based on the observation
that the numerator of ${f(x) \over f'(x)}$ can be derived from a partial denominator.  That is at first the denominator is calculated by finding
$x^{b - 1}$.  This value can then be multiplied by $x$ and have $a$ subtracted from it to find the numerator.  This saves a total of $b - 1$ 
multiplications by t$1$ inside the loop.  

The initial value of the approximation is t$2 = 2$ which allows the algorithm to start with very small values and quickly converge on the
root.  Ideally this algorithm is meant to find the $n$'th root of an input where $n$ is bounded by $2 \le n \le 5$.  

\vspace{+3mm}\begin{small}
\hspace{-5.1mm}{\bf File}: bn\_mp\_n\_root.c
\vspace{-3mm}
\begin{alltt}
\end{alltt}
\end{small}

\section{Random Number Generation}

Random numbers come up in a variety of activities from public key cryptography to simple simulations and various randomized algorithms.  Pollard-Rho 
factoring for example, can make use of random values as starting points to find factors of a composite integer.  In this case the algorithm presented
is solely for simulations and not intended for cryptographic use.  

\newpage\begin{figure}[!here]
\begin{small}
\begin{center}
\begin{tabular}{l}
\hline Algorithm \textbf{mp\_rand}. \\
\textbf{Input}.   An integer $b$ \\
\textbf{Output}.  A pseudo-random number of $b$ digits \\
\hline \\
1.  $a \leftarrow 0$ \\
2.  If $b \le 0$ return(\textit{MP\_OKAY}) \\
3.  Pick a non-zero random digit $d$. \\
4.  $a \leftarrow a + d$ \\
5.  for $ix$ from 1 to $d - 1$ do \\
\hspace{3mm}5.1  $a \leftarrow a \cdot \beta$ \\
\hspace{3mm}5.2  Pick a random digit $d$. \\
\hspace{3mm}5.3  $a \leftarrow a + d$ \\
6.  Return(\textit{MP\_OKAY}). \\
\hline
\end{tabular}
\end{center}
\end{small}
\caption{Algorithm mp\_rand}
\end{figure}
\textbf{Algorithm mp\_rand.}
This algorithm produces a pseudo-random integer of $b$ digits.  By ensuring that the first digit is non-zero the algorithm also guarantees that the
final result has at least $b$ digits.  It relies heavily on a third-part random number generator which should ideally generate uniformly all of
the integers from $0$ to $\beta - 1$.  

\vspace{+3mm}\begin{small}
\hspace{-5.1mm}{\bf File}: bn\_mp\_rand.c
\vspace{-3mm}
\begin{alltt}
\end{alltt}
\end{small}

\section{Formatted Representations}
The ability to emit a radix-$n$ textual representation of an integer is useful for interacting with human parties.  For example, the ability to
be given a string of characters such as ``114585'' and turn it into the radix-$\beta$ equivalent would make it easier to enter numbers
into a program.

\subsection{Reading Radix-n Input}
For the purposes of this text we will assume that a simple lower ASCII map (\ref{fig:ASC}) is used for the values of from $0$ to $63$ to 
printable characters.  For example, when the character ``N'' is read it represents the integer $23$.  The first $16$ characters of the
map are for the common representations up to hexadecimal.  After that they match the ``base64'' encoding scheme which are suitable chosen
such that they are printable.  While outputting as base64 may not be too helpful for human operators it does allow communication via non binary
mediums.

\newpage\begin{figure}[here]
\begin{center}
\begin{tabular}{cc|cc|cc|cc}
\hline \textbf{Value} & \textbf{Char} & \textbf{Value} & \textbf{Char} & \textbf{Value} & \textbf{Char} &  \textbf{Value} & \textbf{Char} \\
\hline 
0 & 0 & 1 & 1 & 2 & 2 & 3 & 3 \\
4 & 4 & 5 & 5 & 6 & 6 & 7 & 7 \\
8 & 8 & 9 & 9 & 10 & A & 11 & B \\
12 & C & 13 & D & 14 & E & 15 & F \\
16 & G & 17 & H & 18 & I & 19 & J \\
20 & K & 21 & L & 22 & M & 23 & N \\
24 & O & 25 & P & 26 & Q & 27 & R \\
28 & S & 29 & T & 30 & U & 31 & V \\
32 & W & 33 & X & 34 & Y & 35 & Z \\
36 & a & 37 & b & 38 & c & 39 & d \\
40 & e & 41 & f & 42 & g & 43 & h \\
44 & i & 45 & j & 46 & k & 47 & l \\
48 & m & 49 & n & 50 & o & 51 & p \\
52 & q & 53 & r & 54 & s & 55 & t \\
56 & u & 57 & v & 58 & w & 59 & x \\
60 & y & 61 & z & 62 & $+$ & 63 & $/$ \\
\hline
\end{tabular}
\end{center}
\caption{Lower ASCII Map}
\label{fig:ASC}
\end{figure}

\newpage\begin{figure}[!here]
\begin{small}
\begin{center}
\begin{tabular}{l}
\hline Algorithm \textbf{mp\_read\_radix}. \\
\textbf{Input}.   A string $str$ of length $sn$ and radix $r$. \\
\textbf{Output}.  The radix-$\beta$ equivalent mp\_int. \\
\hline \\
1.  If $r < 2$ or $r > 64$ return(\textit{MP\_VAL}). \\
2.  $ix \leftarrow 0$ \\
3.  If $str_0 =$ ``-'' then do \\
\hspace{3mm}3.1  $ix \leftarrow ix + 1$ \\
\hspace{3mm}3.2  $sign \leftarrow MP\_NEG$ \\
4.  else \\
\hspace{3mm}4.1  $sign \leftarrow MP\_ZPOS$ \\
5.  $a \leftarrow 0$ \\
6.  for $iy$ from $ix$ to $sn - 1$ do \\
\hspace{3mm}6.1  Let $y$ denote the position in the map of $str_{iy}$. \\
\hspace{3mm}6.2  If $str_{iy}$ is not in the map or $y \ge r$ then goto step 7. \\
\hspace{3mm}6.3  $a \leftarrow a \cdot r$ \\
\hspace{3mm}6.4  $a \leftarrow a + y$ \\
7.  If $a \ne 0$ then $a.sign \leftarrow sign$ \\
8.  Return(\textit{MP\_OKAY}). \\
\hline
\end{tabular}
\end{center}
\end{small}
\caption{Algorithm mp\_read\_radix}
\end{figure}
\textbf{Algorithm mp\_read\_radix.}
This algorithm will read an ASCII string and produce the radix-$\beta$ mp\_int representation of the same integer.  A minus symbol ``-'' may precede the 
string  to indicate the value is negative, otherwise it is assumed to be positive.  The algorithm will read up to $sn$ characters from the input
and will stop when it reads a character it cannot map the algorithm stops reading characters from the string.  This allows numbers to be embedded
as part of larger input without any significant problem.

\vspace{+3mm}\begin{small}
\hspace{-5.1mm}{\bf File}: bn\_mp\_read\_radix.c
\vspace{-3mm}
\begin{alltt}
\end{alltt}
\end{small}

\subsection{Generating Radix-$n$ Output}
Generating radix-$n$ output is fairly trivial with a division and remainder algorithm.  

\newpage\begin{figure}[!here]
\begin{small}
\begin{center}
\begin{tabular}{l}
\hline Algorithm \textbf{mp\_toradix}. \\
\textbf{Input}.   A mp\_int $a$ and an integer $r$\\
\textbf{Output}.  The radix-$r$ representation of $a$ \\
\hline \\
1.  If $r < 2$ or $r > 64$ return(\textit{MP\_VAL}). \\
2.  If $a = 0$ then $str = $ ``$0$'' and return(\textit{MP\_OKAY}).  \\
3.  $t \leftarrow a$ \\
4.  $str \leftarrow$ ``'' \\
5.  if $t.sign = MP\_NEG$ then \\
\hspace{3mm}5.1  $str \leftarrow str + $ ``-'' \\
\hspace{3mm}5.2  $t.sign = MP\_ZPOS$ \\
6.  While ($t \ne 0$) do \\
\hspace{3mm}6.1  $d \leftarrow t \mbox{ (mod }r\mbox{)}$ \\
\hspace{3mm}6.2  $t \leftarrow \lfloor t / r \rfloor$ \\
\hspace{3mm}6.3  Look up $d$ in the map and store the equivalent character in $y$. \\
\hspace{3mm}6.4  $str \leftarrow str + y$ \\
7.  If $str_0 = $``$-$'' then \\
\hspace{3mm}7.1  Reverse the digits $str_1, str_2, \ldots str_n$. \\
8.  Otherwise \\
\hspace{3mm}8.1  Reverse the digits $str_0, str_1, \ldots str_n$. \\
9.  Return(\textit{MP\_OKAY}).\\
\hline
\end{tabular}
\end{center}
\end{small}
\caption{Algorithm mp\_toradix}
\end{figure}
\textbf{Algorithm mp\_toradix.}
This algorithm computes the radix-$r$ representation of an mp\_int $a$.  The ``digits'' of the representation are extracted by reducing 
successive powers of $\lfloor a / r^k \rfloor$ the input modulo $r$ until $r^k > a$.  Note that instead of actually dividing by $r^k$ in
each iteration the quotient $\lfloor a / r \rfloor$ is saved for the next iteration.  As a result a series of trivial $n \times 1$ divisions
are required instead of a series of $n \times k$ divisions.  One design flaw of this approach is that the digits are produced in the reverse order 
(see~\ref{fig:mpradix}).  To remedy this flaw the digits must be swapped or simply ``reversed''.

\begin{figure}
\begin{center}
\begin{tabular}{|c|c|c|}
\hline \textbf{Value of $a$} & \textbf{Value of $d$} & \textbf{Value of $str$} \\
\hline $1234$ & -- & -- \\
\hline $123$  & $4$ & ``4'' \\
\hline $12$   & $3$ & ``43'' \\
\hline $1$    & $2$ & ``432'' \\
\hline $0$    & $1$ & ``4321'' \\
\hline
\end{tabular}
\end{center}
\caption{Example of Algorithm mp\_toradix.}
\label{fig:mpradix}
\end{figure}

\vspace{+3mm}\begin{small}
\hspace{-5.1mm}{\bf File}: bn\_mp\_toradix.c
\vspace{-3mm}
\begin{alltt}
\end{alltt}
\end{small}

\chapter{Number Theoretic Algorithms}
This chapter discusses several fundamental number theoretic algorithms such as the greatest common divisor, least common multiple and Jacobi 
symbol computation.  These algorithms arise as essential components in several key cryptographic algorithms such as the RSA public key algorithm and
various Sieve based factoring algorithms.

\section{Greatest Common Divisor}
The greatest common divisor of two integers $a$ and $b$, often denoted as $(a, b)$ is the largest integer $k$ that is a proper divisor of
both $a$ and $b$.  That is, $k$ is the largest integer such that $0 \equiv a \mbox{ (mod }k\mbox{)}$ and $0 \equiv b \mbox{ (mod }k\mbox{)}$ occur
simultaneously.

The most common approach (cite) is to reduce one input modulo another.  That is if $a$ and $b$ are divisible by some integer $k$ and if $qa + r = b$ then
$r$ is also divisible by $k$.  The reduction pattern follows $\left < a , b \right > \rightarrow \left < b, a \mbox{ mod } b \right >$.  

\newpage\begin{figure}[!here]
\begin{small}
\begin{center}
\begin{tabular}{l}
\hline Algorithm \textbf{Greatest Common Divisor (I)}. \\
\textbf{Input}.   Two positive integers $a$ and $b$ greater than zero. \\
\textbf{Output}.  The greatest common divisor $(a, b)$.  \\
\hline \\
1.  While ($b > 0$) do \\
\hspace{3mm}1.1  $r \leftarrow a \mbox{ (mod }b\mbox{)}$ \\
\hspace{3mm}1.2  $a \leftarrow b$ \\
\hspace{3mm}1.3  $b \leftarrow r$ \\
2.  Return($a$). \\
\hline
\end{tabular}
\end{center}
\end{small}
\caption{Algorithm Greatest Common Divisor (I)}
\label{fig:gcd1}
\end{figure}

This algorithm will quickly converge on the greatest common divisor since the residue $r$ tends diminish rapidly.  However, divisions are
relatively expensive operations to perform and should ideally be avoided.  There is another approach based on a similar relationship of 
greatest common divisors.  The faster approach is based on the observation that if $k$ divides both $a$ and $b$ it will also divide $a - b$.  
In particular, we would like $a - b$ to decrease in magnitude which implies that $b \ge a$.  

\begin{figure}[!here]
\begin{small}
\begin{center}
\begin{tabular}{l}
\hline Algorithm \textbf{Greatest Common Divisor (II)}. \\
\textbf{Input}.   Two positive integers $a$ and $b$ greater than zero. \\
\textbf{Output}.  The greatest common divisor $(a, b)$.  \\
\hline \\
1.  While ($b > 0$) do \\
\hspace{3mm}1.1  Swap $a$ and $b$ such that $a$ is the smallest of the two. \\
\hspace{3mm}1.2  $b \leftarrow b - a$ \\
2.  Return($a$). \\
\hline
\end{tabular}
\end{center}
\end{small}
\caption{Algorithm Greatest Common Divisor (II)}
\label{fig:gcd2}
\end{figure}

\textbf{Proof} \textit{Algorithm~\ref{fig:gcd2} will return the greatest common divisor of $a$ and $b$.}
The algorithm in figure~\ref{fig:gcd2} will eventually terminate since $b \ge a$ the subtraction in step 1.2 will be a value less than $b$.  In other
words in every iteration that tuple $\left < a, b \right >$ decrease in magnitude until eventually $a = b$.  Since both $a$ and $b$ are always 
divisible by the greatest common divisor (\textit{until the last iteration}) and in the last iteration of the algorithm $b = 0$, therefore, in the 
second to last iteration of the algorithm $b = a$ and clearly $(a, a) = a$ which concludes the proof.  \textbf{QED}.

As a matter of practicality algorithm \ref{fig:gcd1} decreases far too slowly to be useful.  Specially if $b$ is much larger than $a$ such that 
$b - a$ is still very much larger than $a$.  A simple addition to the algorithm is to divide $b - a$ by a power of some integer $p$ which does
not divide the greatest common divisor but will divide $b - a$.  In this case ${b - a} \over p$ is also an integer and still divisible by
the greatest common divisor.

However, instead of factoring $b - a$ to find a suitable value of $p$ the powers of $p$ can be removed from $a$ and $b$ that are in common first.  
Then inside the loop whenever $b - a$ is divisible by some power of $p$ it can be safely removed.  

\begin{figure}[!here]
\begin{small}
\begin{center}
\begin{tabular}{l}
\hline Algorithm \textbf{Greatest Common Divisor (III)}. \\
\textbf{Input}.   Two positive integers $a$ and $b$ greater than zero. \\
\textbf{Output}.  The greatest common divisor $(a, b)$.  \\
\hline \\
1.  $k \leftarrow 0$ \\
2.  While $a$ and $b$ are both divisible by $p$ do \\
\hspace{3mm}2.1  $a \leftarrow \lfloor a / p \rfloor$ \\
\hspace{3mm}2.2  $b \leftarrow \lfloor b / p \rfloor$ \\
\hspace{3mm}2.3  $k \leftarrow k + 1$ \\
3.  While $a$ is divisible by $p$ do \\
\hspace{3mm}3.1  $a \leftarrow \lfloor a / p \rfloor$ \\
4.  While $b$ is divisible by $p$ do \\
\hspace{3mm}4.1  $b \leftarrow \lfloor b / p \rfloor$ \\
5.  While ($b > 0$) do \\
\hspace{3mm}5.1  Swap $a$ and $b$ such that $a$ is the smallest of the two. \\
\hspace{3mm}5.2  $b \leftarrow b - a$ \\
\hspace{3mm}5.3  While $b$ is divisible by $p$ do \\
\hspace{6mm}5.3.1  $b \leftarrow \lfloor b / p \rfloor$ \\
6.  Return($a \cdot p^k$). \\
\hline
\end{tabular}
\end{center}
\end{small}
\caption{Algorithm Greatest Common Divisor (III)}
\label{fig:gcd3}
\end{figure}

This algorithm is based on the first except it removes powers of $p$ first and inside the main loop to ensure the tuple $\left < a, b \right >$ 
decreases more rapidly.  The first loop on step two removes powers of $p$ that are in common.  A count, $k$, is kept which will present a common
divisor of $p^k$.  After step two the remaining common divisor of $a$ and $b$ cannot be divisible by $p$.  This means that $p$ can be safely 
divided out of the difference $b - a$ so long as the division leaves no remainder.  

In particular the value of $p$ should be chosen such that the division on step 5.3.1 occur often.  It also helps that division by $p$ be easy
to compute.  The ideal choice of $p$ is two since division by two amounts to a right logical shift.  Another important observation is that by
step five both $a$ and $b$ are odd.  Therefore, the diffrence $b - a$ must be even which means that each iteration removes one bit from the 
largest of the pair.

\subsection{Complete Greatest Common Divisor}
The algorithms presented so far cannot handle inputs which are zero or negative.  The following algorithm can handle all input cases properly
and will produce the greatest common divisor.

\newpage\begin{figure}[!here]
\begin{small}
\begin{center}
\begin{tabular}{l}
\hline Algorithm \textbf{mp\_gcd}. \\
\textbf{Input}.   mp\_int $a$ and $b$ \\
\textbf{Output}.  The greatest common divisor $c = (a, b)$.  \\
\hline \\
1.  If $a = 0$ then \\
\hspace{3mm}1.1  $c \leftarrow \vert b \vert $ \\
\hspace{3mm}1.2  Return(\textit{MP\_OKAY}). \\
2.  If $b = 0$ then \\
\hspace{3mm}2.1  $c \leftarrow \vert a \vert $ \\
\hspace{3mm}2.2  Return(\textit{MP\_OKAY}). \\
3.  $u \leftarrow \vert a \vert, v \leftarrow \vert b \vert$ \\
4.  $k \leftarrow 0$ \\
5.  While $u.used > 0$ and $v.used > 0$ and $u_0 \equiv v_0 \equiv 0 \mbox{ (mod }2\mbox{)}$ \\
\hspace{3mm}5.1  $k \leftarrow k + 1$ \\
\hspace{3mm}5.2  $u \leftarrow \lfloor u / 2 \rfloor$ \\
\hspace{3mm}5.3  $v \leftarrow \lfloor v / 2 \rfloor$ \\
6.  While $u.used > 0$ and $u_0 \equiv 0 \mbox{ (mod }2\mbox{)}$ \\
\hspace{3mm}6.1  $u \leftarrow \lfloor u / 2 \rfloor$ \\
7.  While $v.used > 0$ and $v_0 \equiv 0 \mbox{ (mod }2\mbox{)}$ \\
\hspace{3mm}7.1  $v \leftarrow \lfloor v / 2 \rfloor$ \\
8.  While $v.used > 0$ \\
\hspace{3mm}8.1  If $\vert u \vert > \vert v \vert$ then \\
\hspace{6mm}8.1.1  Swap $u$ and $v$. \\
\hspace{3mm}8.2  $v \leftarrow \vert v \vert - \vert u \vert$ \\
\hspace{3mm}8.3  While $v.used > 0$ and $v_0 \equiv 0 \mbox{ (mod }2\mbox{)}$ \\
\hspace{6mm}8.3.1  $v \leftarrow \lfloor v / 2 \rfloor$ \\
9.  $c \leftarrow u \cdot 2^k$ \\
10.  Return(\textit{MP\_OKAY}). \\
\hline
\end{tabular}
\end{center}
\end{small}
\caption{Algorithm mp\_gcd}
\end{figure}
\textbf{Algorithm mp\_gcd.}
This algorithm will produce the greatest common divisor of two mp\_ints $a$ and $b$.  The algorithm was originally based on Algorithm B of
Knuth \cite[pp. 338]{TAOCPV2} but has been modified to be simpler to explain.  In theory it achieves the same asymptotic working time as
Algorithm B and in practice this appears to be true.  

The first two steps handle the cases where either one of or both inputs are zero.  If either input is zero the greatest common divisor is the 
largest input or zero if they are both zero.  If the inputs are not trivial than $u$ and $v$ are assigned the absolute values of 
$a$ and $b$ respectively and the algorithm will proceed to reduce the pair.

Step five will divide out any common factors of two and keep track of the count in the variable $k$.  After this step, two is no longer a
factor of the remaining greatest common divisor between $u$ and $v$ and can be safely evenly divided out of either whenever they are even.  Step 
six and seven ensure that the $u$ and $v$ respectively have no more factors of two.  At most only one of the while--loops will iterate since 
they cannot both be even.

By step eight both of $u$ and $v$ are odd which is required for the inner logic.  First the pair are swapped such that $v$ is equal to
or greater than $u$.  This ensures that the subtraction on step 8.2 will always produce a positive and even result.  Step 8.3 removes any
factors of two from the difference $u$ to ensure that in the next iteration of the loop both are once again odd.

After $v = 0$ occurs the variable $u$ has the greatest common divisor of the pair $\left < u, v \right >$ just after step six.  The result
must be adjusted by multiplying by the common factors of two ($2^k$) removed earlier.  

\vspace{+3mm}\begin{small}
\hspace{-5.1mm}{\bf File}: bn\_mp\_gcd.c
\vspace{-3mm}
\begin{alltt}
\end{alltt}
\end{small}

This function makes use of the macros mp\_iszero and mp\_iseven.  The former evaluates to $1$ if the input mp\_int is equivalent to the 
integer zero otherwise it evaluates to $0$.  The latter evaluates to $1$ if the input mp\_int represents a non-zero even integer otherwise
it evaluates to $0$.  Note that just because mp\_iseven may evaluate to $0$ does not mean the input is odd, it could also be zero.  The three 
trivial cases of inputs are handled on lines 24 through 30.  After those lines the inputs are assumed to be non-zero.

Lines 32 and 37 make local copies $u$ and $v$ of the inputs $a$ and $b$ respectively.  At this point the common factors of two 
must be divided out of the two inputs.  The block starting at line 44 removes common factors of two by first counting the number of trailing
zero bits in both.  The local integer $k$ is used to keep track of how many factors of $2$ are pulled out of both values.  It is assumed that 
the number of factors will not exceed the maximum value of a C ``int'' data type\footnote{Strictly speaking no array in C may have more than 
entries than are accessible by an ``int'' so this is not a limitation.}.  

At this point there are no more common factors of two in the two values.  The divisions by a power of two on lines 62 and 68 remove 
any independent factors of two such that both $u$ and $v$ are guaranteed to be an odd integer before hitting the main body of the algorithm.  The while loop
on line 73 performs the reduction of the pair until $v$ is equal to zero.  The unsigned comparison and subtraction algorithms are used in
place of the full signed routines since both values are guaranteed to be positive and the result of the subtraction is guaranteed to be non-negative.

\section{Least Common Multiple}
The least common multiple of a pair of integers is their product divided by their greatest common divisor.  For two integers $a$ and $b$ the
least common multiple is normally denoted as $[ a, b ]$ and numerically equivalent to ${ab} \over {(a, b)}$.  For example, if $a = 2 \cdot 2 \cdot 3 = 12$
and $b = 2 \cdot 3 \cdot 3 \cdot 7 = 126$ the least common multiple is ${126 \over {(12, 126)}} = {126 \over 6} = 21$.

The least common multiple arises often in coding theory as well as number theory.  If two functions have periods of $a$ and $b$ respectively they will
collide, that is be in synchronous states, after only $[ a, b ]$ iterations.  This is why, for example, random number generators based on 
Linear Feedback Shift Registers (LFSR) tend to use registers with periods which are co-prime (\textit{e.g. the greatest common divisor is one.}).  
Similarly in number theory if a composite $n$ has two prime factors $p$ and $q$ then maximal order of any unit of $\Z/n\Z$ will be $[ p - 1, q - 1] $.

\begin{figure}[!here]
\begin{small}
\begin{center}
\begin{tabular}{l}
\hline Algorithm \textbf{mp\_lcm}. \\
\textbf{Input}.   mp\_int $a$ and $b$ \\
\textbf{Output}.  The least common multiple $c = [a, b]$.  \\
\hline \\
1.  $c \leftarrow (a, b)$ \\
2.  $t \leftarrow a \cdot b$ \\
3.  $c \leftarrow \lfloor t / c \rfloor$ \\
4.  Return(\textit{MP\_OKAY}). \\
\hline
\end{tabular}
\end{center}
\end{small}
\caption{Algorithm mp\_lcm}
\end{figure}
\textbf{Algorithm mp\_lcm.}
This algorithm computes the least common multiple of two mp\_int inputs $a$ and $b$.  It computes the least common multiple directly by
dividing the product of the two inputs by their greatest common divisor.

\vspace{+3mm}\begin{small}
\hspace{-5.1mm}{\bf File}: bn\_mp\_lcm.c
\vspace{-3mm}
\begin{alltt}
\end{alltt}
\end{small}

\section{Jacobi Symbol Computation}
To explain the Jacobi Symbol we shall first discuss the Legendre function\footnote{Arrg.  What is the name of this?} off which the Jacobi symbol is 
defined.  The Legendre function computes whether or not an integer $a$ is a quadratic residue modulo an odd prime $p$.  Numerically it is
equivalent to equation \ref{eqn:legendre}.

\textit{-- Tom, don't be an ass, cite your source here...!}

\begin{equation}
a^{(p-1)/2} \equiv \begin{array}{rl}
                              -1 &  \mbox{if }a\mbox{ is a quadratic non-residue.} \\
                              0  &  \mbox{if }a\mbox{ divides }p\mbox{.} \\
                              1  &  \mbox{if }a\mbox{ is a quadratic residue}. 
                              \end{array} \mbox{ (mod }p\mbox{)}
\label{eqn:legendre}                              
\end{equation}

\textbf{Proof.} \textit{Equation \ref{eqn:legendre} correctly identifies the residue status of an integer $a$ modulo a prime $p$.}
An integer $a$ is a quadratic residue if the following equation has a solution.

\begin{equation}
x^2 \equiv a \mbox{ (mod }p\mbox{)}
\label{eqn:root}
\end{equation}

Consider the following equation.

\begin{equation}
0 \equiv x^{p-1} - 1 \equiv \left \lbrace \left (x^2 \right )^{(p-1)/2} - a^{(p-1)/2} \right \rbrace + \left ( a^{(p-1)/2} - 1 \right ) \mbox{ (mod }p\mbox{)}
\label{eqn:rooti}
\end{equation}

Whether equation \ref{eqn:root} has a solution or not equation \ref{eqn:rooti} is always true.  If $a^{(p-1)/2} - 1 \equiv 0 \mbox{ (mod }p\mbox{)}$
then the quantity in the braces must be zero.  By reduction,

\begin{eqnarray}
\left (x^2 \right )^{(p-1)/2} - a^{(p-1)/2} \equiv 0  \nonumber \\
\left (x^2 \right )^{(p-1)/2} \equiv a^{(p-1)/2} \nonumber \\
x^2 \equiv a \mbox{ (mod }p\mbox{)} 
\end{eqnarray}

As a result there must be a solution to the quadratic equation and in turn $a$ must be a quadratic residue.  If $a$ does not divide $p$ and $a$
is not a quadratic residue then the only other value $a^{(p-1)/2}$ may be congruent to is $-1$ since
\begin{equation}
0 \equiv a^{p - 1} - 1 \equiv (a^{(p-1)/2} + 1)(a^{(p-1)/2} - 1) \mbox{ (mod }p\mbox{)}
\end{equation}
One of the terms on the right hand side must be zero.  \textbf{QED}

\subsection{Jacobi Symbol}
The Jacobi symbol is a generalization of the Legendre function for any odd non prime moduli $p$ greater than 2.  If $p = \prod_{i=0}^n p_i$ then
the Jacobi symbol $\left ( { a \over p } \right )$ is equal to the following equation.

\begin{equation}
\left ( { a \over p } \right ) = \left ( { a \over p_0} \right ) \left ( { a \over p_1} \right ) \ldots \left ( { a \over p_n} \right )
\end{equation}

By inspection if $p$ is prime the Jacobi symbol is equivalent to the Legendre function.  The following facts\footnote{See HAC \cite[pp. 72-74]{HAC} for
further details.} will be used to derive an efficient Jacobi symbol algorithm.  Where $p$ is an odd integer greater than two and $a, b \in \Z$ the
following are true.  

\begin{enumerate}
\item $\left ( { a \over p} \right )$ equals $-1$, $0$ or $1$. 
\item $\left ( { ab \over p} \right ) = \left ( { a \over p} \right )\left ( { b \over p} \right )$.
\item If $a \equiv b$ then $\left ( { a \over p} \right ) = \left ( { b \over p} \right )$.
\item $\left ( { 2 \over p} \right )$ equals $1$ if $p \equiv 1$ or $7 \mbox{ (mod }8\mbox{)}$.  Otherwise, it equals $-1$.
\item $\left ( { a \over p} \right ) \equiv \left ( { p \over a} \right ) \cdot (-1)^{(p-1)(a-1)/4}$.  More specifically 
$\left ( { a \over p} \right ) = \left ( { p \over a} \right )$ if $p \equiv a \equiv 1 \mbox{ (mod }4\mbox{)}$.  
\end{enumerate}

Using these facts if $a = 2^k \cdot a'$ then

\begin{eqnarray}
\left ( { a \over p } \right ) = \left ( {{2^k} \over p } \right ) \left ( {a' \over p} \right ) \nonumber \\
                               = \left ( {2 \over p } \right )^k \left ( {a' \over p} \right ) 
\label{eqn:jacobi}
\end{eqnarray}

By fact five, 

\begin{equation}
\left ( { a \over p } \right ) = \left ( { p \over a } \right ) \cdot (-1)^{(p-1)(a-1)/4} 
\end{equation}

Subsequently by fact three since $p \equiv (p \mbox{ mod }a) \mbox{ (mod }a\mbox{)}$ then 

\begin{equation}
\left ( { a \over p } \right ) = \left ( { {p \mbox{ mod } a} \over a } \right ) \cdot (-1)^{(p-1)(a-1)/4} 
\end{equation}

By putting both observations into equation \ref{eqn:jacobi} the following simplified equation is formed.

\begin{equation}
\left ( { a \over p } \right ) = \left ( {2 \over p } \right )^k \left ( {{p\mbox{ mod }a'} \over a'} \right )  \cdot (-1)^{(p-1)(a'-1)/4} 
\end{equation}

The value of $\left ( {{p \mbox{ mod }a'} \over a'} \right )$ can be found by using the same equation recursively.  The value of 
$\left ( {2 \over p } \right )^k$ equals $1$ if $k$ is even otherwise it equals $\left ( {2 \over p } \right )$.  Using this approach the 
factors of $p$ do not have to be known.  Furthermore, if $(a, p) = 1$ then the algorithm will terminate when the recursion requests the 
Jacobi symbol computation of $\left ( {1 \over a'} \right )$ which is simply $1$.  

\newpage\begin{figure}[!here]
\begin{small}
\begin{center}
\begin{tabular}{l}
\hline Algorithm \textbf{mp\_jacobi}. \\
\textbf{Input}.   mp\_int $a$ and $p$, $a \ge 0$, $p \ge 3$, $p \equiv 1 \mbox{ (mod }2\mbox{)}$ \\
\textbf{Output}.  The Jacobi symbol $c = \left ( {a \over p } \right )$. \\
\hline \\
1.  If $a = 0$ then \\
\hspace{3mm}1.1  $c \leftarrow 0$ \\
\hspace{3mm}1.2  Return(\textit{MP\_OKAY}). \\
2.  If $a = 1$ then \\
\hspace{3mm}2.1  $c \leftarrow 1$ \\
\hspace{3mm}2.2  Return(\textit{MP\_OKAY}). \\
3.  $a' \leftarrow a$ \\
4.  $k \leftarrow 0$ \\
5.  While $a'.used > 0$ and $a'_0 \equiv 0 \mbox{ (mod }2\mbox{)}$ \\
\hspace{3mm}5.1  $k \leftarrow k + 1$ \\
\hspace{3mm}5.2  $a' \leftarrow \lfloor a' / 2 \rfloor$ \\
6.  If $k \equiv 0 \mbox{ (mod }2\mbox{)}$ then \\
\hspace{3mm}6.1  $s \leftarrow 1$ \\
7.  else \\
\hspace{3mm}7.1  $r \leftarrow p_0 \mbox{ (mod }8\mbox{)}$ \\
\hspace{3mm}7.2  If $r = 1$ or $r = 7$ then \\
\hspace{6mm}7.2.1  $s \leftarrow 1$ \\
\hspace{3mm}7.3  else \\
\hspace{6mm}7.3.1  $s \leftarrow -1$ \\
8.  If $p_0 \equiv a'_0 \equiv 3 \mbox{ (mod }4\mbox{)}$ then \\
\hspace{3mm}8.1  $s \leftarrow -s$ \\
9.  If $a' \ne 1$ then \\
\hspace{3mm}9.1  $p' \leftarrow p \mbox{ (mod }a'\mbox{)}$ \\
\hspace{3mm}9.2  $s \leftarrow s \cdot \mbox{mp\_jacobi}(p', a')$ \\
10.  $c \leftarrow s$ \\
11.  Return(\textit{MP\_OKAY}). \\
\hline
\end{tabular}
\end{center}
\end{small}
\caption{Algorithm mp\_jacobi}
\end{figure}
\textbf{Algorithm mp\_jacobi.}
This algorithm computes the Jacobi symbol for an arbitrary positive integer $a$ with respect to an odd integer $p$ greater than three.  The algorithm
is based on algorithm 2.149 of HAC \cite[pp. 73]{HAC}.  

Step numbers one and two handle the trivial cases of $a = 0$ and $a = 1$ respectively.  Step five determines the number of two factors in the
input $a$.  If $k$ is even than the term $\left ( { 2 \over p } \right )^k$ must always evaluate to one.  If $k$ is odd than the term evaluates to one 
if $p_0$ is congruent to one or seven modulo eight, otherwise it evaluates to $-1$. After the the $\left ( { 2 \over p } \right )^k$ term is handled 
the $(-1)^{(p-1)(a'-1)/4}$ is computed and multiplied against the current product $s$.  The latter term evaluates to one if both $p$ and $a'$ 
are congruent to one modulo four, otherwise it evaluates to negative one.

By step nine if $a'$ does not equal one a recursion is required.  Step 9.1 computes $p' \equiv p \mbox{ (mod }a'\mbox{)}$ and will recurse to compute
$\left ( {p' \over a'} \right )$ which is multiplied against the current Jacobi product.

\vspace{+3mm}\begin{small}
\hspace{-5.1mm}{\bf File}: bn\_mp\_jacobi.c
\vspace{-3mm}
\begin{alltt}
\end{alltt}
\end{small}

As a matter of practicality the variable $a'$ as per the pseudo-code is reprensented by the variable $a1$ since the $'$ symbol is not valid for a C 
variable name character. 

The two simple cases of $a = 0$ and $a = 1$ are handled at the very beginning to simplify the algorithm.  If the input is non-trivial the algorithm
has to proceed compute the Jacobi.  The variable $s$ is used to hold the current Jacobi product.  Note that $s$ is merely a C ``int'' data type since
the values it may obtain are merely $-1$, $0$ and $1$.  

After a local copy of $a$ is made all of the factors of two are divided out and the total stored in $k$.  Technically only the least significant
bit of $k$ is required, however, it makes the algorithm simpler to follow to perform an addition. In practice an exclusive-or and addition have the same 
processor requirements and neither is faster than the other.

Line 58 through 71 determines the value of $\left ( { 2 \over p } \right )^k$.  If the least significant bit of $k$ is zero than
$k$ is even and the value is one.  Otherwise, the value of $s$ depends on which residue class $p$ belongs to modulo eight.  The value of
$(-1)^{(p-1)(a'-1)/4}$ is compute and multiplied against $s$ on lines 71 through 74.  

Finally, if $a1$ does not equal one the algorithm must recurse and compute $\left ( {p' \over a'} \right )$.  

\textit{-- Comment about default $s$ and such...}

\section{Modular Inverse}
\label{sec:modinv}
The modular inverse of a number actually refers to the modular multiplicative inverse.  Essentially for any integer $a$ such that $(a, p) = 1$ there
exist another integer $b$ such that $ab \equiv 1 \mbox{ (mod }p\mbox{)}$.  The integer $b$ is called the multiplicative inverse of $a$ which is
denoted as $b = a^{-1}$.  Technically speaking modular inversion is a well defined operation for any finite ring or field not just for rings and 
fields of integers.  However, the former will be the matter of discussion.

The simplest approach is to compute the algebraic inverse of the input.  That is to compute $b \equiv a^{\Phi(p) - 1}$.  If $\Phi(p)$ is the 
order of the multiplicative subgroup modulo $p$ then $b$ must be the multiplicative inverse of $a$.  The proof of which is trivial.

\begin{equation}
ab \equiv a \left (a^{\Phi(p) - 1} \right ) \equiv a^{\Phi(p)} \equiv a^0 \equiv 1 \mbox{ (mod }p\mbox{)}
\end{equation}

However, as simple as this approach may be it has two serious flaws.  It requires that the value of $\Phi(p)$ be known which if $p$ is composite 
requires all of the prime factors.  This approach also is very slow as the size of $p$ grows.  

A simpler approach is based on the observation that solving for the multiplicative inverse is equivalent to solving the linear 
Diophantine\footnote{See LeVeque \cite[pp. 40-43]{LeVeque} for more information.} equation.

\begin{equation}
ab + pq = 1
\end{equation}

Where $a$, $b$, $p$ and $q$ are all integers.  If such a pair of integers $ \left < b, q \right >$ exist than $b$ is the multiplicative inverse of 
$a$ modulo $p$.  The extended Euclidean algorithm (Knuth \cite[pp. 342]{TAOCPV2}) can be used to solve such equations provided $(a, p) = 1$.  
However, instead of using that algorithm directly a variant known as the binary Extended Euclidean algorithm will be used in its place.  The
binary approach is very similar to the binary greatest common divisor algorithm except it will produce a full solution to the Diophantine 
equation.  

\subsection{General Case}
\newpage\begin{figure}[!here]
\begin{small}
\begin{center}
\begin{tabular}{l}
\hline Algorithm \textbf{mp\_invmod}. \\
\textbf{Input}.   mp\_int $a$ and $b$, $(a, b) = 1$, $p \ge 2$, $0 < a < p$.  \\
\textbf{Output}.  The modular inverse $c \equiv a^{-1} \mbox{ (mod }b\mbox{)}$. \\
\hline \\
1.  If $b \le 0$ then return(\textit{MP\_VAL}). \\
2.  If $b_0 \equiv 1 \mbox{ (mod }2\mbox{)}$ then use algorithm fast\_mp\_invmod. \\
3.  $x \leftarrow \vert a \vert, y \leftarrow b$ \\
4.  If $x_0 \equiv y_0  \equiv 0 \mbox{ (mod }2\mbox{)}$ then return(\textit{MP\_VAL}). \\
5.  $B \leftarrow 0, C \leftarrow 0, A \leftarrow 1, D \leftarrow 1$ \\
6.  While $u.used > 0$ and $u_0 \equiv 0 \mbox{ (mod }2\mbox{)}$ \\
\hspace{3mm}6.1  $u \leftarrow \lfloor u / 2 \rfloor$ \\
\hspace{3mm}6.2  If ($A.used > 0$ and $A_0 \equiv 1 \mbox{ (mod }2\mbox{)}$) or ($B.used > 0$ and $B_0 \equiv 1 \mbox{ (mod }2\mbox{)}$) then \\
\hspace{6mm}6.2.1  $A \leftarrow A + y$ \\
\hspace{6mm}6.2.2  $B \leftarrow B - x$ \\
\hspace{3mm}6.3  $A \leftarrow \lfloor A / 2 \rfloor$ \\
\hspace{3mm}6.4  $B \leftarrow \lfloor B / 2 \rfloor$ \\
7.  While $v.used > 0$ and $v_0 \equiv 0 \mbox{ (mod }2\mbox{)}$ \\
\hspace{3mm}7.1  $v \leftarrow \lfloor v / 2 \rfloor$ \\
\hspace{3mm}7.2  If ($C.used > 0$ and $C_0 \equiv 1 \mbox{ (mod }2\mbox{)}$) or ($D.used > 0$ and $D_0 \equiv 1 \mbox{ (mod }2\mbox{)}$) then \\
\hspace{6mm}7.2.1  $C \leftarrow C + y$ \\
\hspace{6mm}7.2.2  $D \leftarrow D - x$ \\
\hspace{3mm}7.3  $C \leftarrow \lfloor C / 2 \rfloor$ \\
\hspace{3mm}7.4  $D \leftarrow \lfloor D / 2 \rfloor$ \\
8.  If $u \ge v$ then \\
\hspace{3mm}8.1  $u \leftarrow u - v$ \\
\hspace{3mm}8.2  $A \leftarrow A - C$ \\
\hspace{3mm}8.3  $B \leftarrow B - D$ \\
9.  else \\
\hspace{3mm}9.1  $v \leftarrow v - u$ \\
\hspace{3mm}9.2  $C \leftarrow C - A$ \\
\hspace{3mm}9.3  $D \leftarrow D - B$ \\
10.  If $u \ne 0$ goto step 6. \\
11.  If $v \ne 1$ return(\textit{MP\_VAL}). \\
12.  While $C \le 0$ do \\
\hspace{3mm}12.1  $C \leftarrow C + b$ \\
13.  While $C \ge b$ do \\
\hspace{3mm}13.1  $C \leftarrow C - b$ \\
14.  $c \leftarrow C$ \\
15.  Return(\textit{MP\_OKAY}). \\
\hline
\end{tabular}
\end{center}
\end{small}
\end{figure}
\textbf{Algorithm mp\_invmod.}
This algorithm computes the modular multiplicative inverse of an integer $a$ modulo an integer $b$.  This algorithm is a variation of the 
extended binary Euclidean algorithm from HAC \cite[pp. 608]{HAC}.  It has been modified to only compute the modular inverse and not a complete
Diophantine solution.  

If $b \le 0$ than the modulus is invalid and MP\_VAL is returned.  Similarly if both $a$ and $b$ are even then there cannot be a multiplicative
inverse for $a$ and the error is reported.  

The astute reader will observe that steps seven through nine are very similar to the binary greatest common divisor algorithm mp\_gcd.  In this case
the other variables to the Diophantine equation are solved.  The algorithm terminates when $u = 0$ in which case the solution is

\begin{equation}
Ca + Db = v
\end{equation}

If $v$, the greatest common divisor of $a$ and $b$ is not equal to one then the algorithm will report an error as no inverse exists.  Otherwise, $C$
is the modular inverse of $a$.  The actual value of $C$ is congruent to, but not necessarily equal to, the ideal modular inverse which should lie 
within $1 \le a^{-1} < b$.  Step numbers twelve and thirteen adjust the inverse until it is in range.  If the original input $a$ is within $0 < a < p$ 
then only a couple of additions or subtractions will be required to adjust the inverse.

\vspace{+3mm}\begin{small}
\hspace{-5.1mm}{\bf File}: bn\_mp\_invmod.c
\vspace{-3mm}
\begin{alltt}
\end{alltt}
\end{small}

\subsubsection{Odd Moduli}

When the modulus $b$ is odd the variables $A$ and $C$ are fixed and are not required to compute the inverse.  In particular by attempting to solve
the Diophantine $Cb + Da = 1$ only $B$ and $D$ are required to find the inverse of $a$.  

The algorithm fast\_mp\_invmod is a direct adaptation of algorithm mp\_invmod with all all steps involving either $A$ or $C$ removed.  This 
optimization will halve the time required to compute the modular inverse.

\section{Primality Tests}

A non-zero integer $a$ is said to be prime if it is not divisible by any other integer excluding one and itself.  For example, $a = 7$ is prime 
since the integers $2 \ldots 6$ do not evenly divide $a$.  By contrast, $a = 6$ is not prime since $a = 6 = 2 \cdot 3$. 

Prime numbers arise in cryptography considerably as they allow finite fields to be formed.  The ability to determine whether an integer is prime or
not quickly has been a viable subject in cryptography and number theory for considerable time.  The algorithms that will be presented are all
probablistic algorithms in that when they report an integer is composite it must be composite.  However, when the algorithms report an integer is
prime the algorithm may be incorrect.  

As will be discussed it is possible to limit the probability of error so well that for practical purposes the probablity of error might as 
well be zero.  For the purposes of these discussions let $n$ represent the candidate integer of which the primality is in question.

\subsection{Trial Division}

Trial division means to attempt to evenly divide a candidate integer by small prime integers.  If the candidate can be evenly divided it obviously
cannot be prime.  By dividing by all primes $1 < p \le \sqrt{n}$ this test can actually prove whether an integer is prime.  However, such a test
would require a prohibitive amount of time as $n$ grows.

Instead of dividing by every prime, a smaller, more mangeable set of primes may be used instead.  By performing trial division with only a subset
of the primes less than $\sqrt{n} + 1$ the algorithm cannot prove if a candidate is prime.  However, often it can prove a candidate is not prime.

The benefit of this test is that trial division by small values is fairly efficient.  Specially compared to the other algorithms that will be
discussed shortly.  The probability that this approach correctly identifies a composite candidate when tested with all primes upto $q$ is given by
$1 - {1.12 \over ln(q)}$.  The graph (\ref{pic:primality}, will be added later) demonstrates the probability of success for the range 
$3 \le q \le 100$.  

At approximately $q = 30$ the gain of performing further tests diminishes fairly quickly.  At $q = 90$ further testing is generally not going to 
be of any practical use.  In the case of LibTomMath the default limit $q = 256$ was chosen since it is not too high and will eliminate 
approximately $80\%$ of all candidate integers.  The constant \textbf{PRIME\_SIZE} is equal to the number of primes in the test base.  The 
array \_\_prime\_tab is an array of the first \textbf{PRIME\_SIZE} prime numbers.  

\begin{figure}[!here]
\begin{small}
\begin{center}
\begin{tabular}{l}
\hline Algorithm \textbf{mp\_prime\_is\_divisible}. \\
\textbf{Input}.   mp\_int $a$ \\
\textbf{Output}.  $c = 1$ if $n$ is divisible by a small prime, otherwise $c = 0$.  \\
\hline \\
1.  for $ix$ from $0$ to $PRIME\_SIZE$ do \\
\hspace{3mm}1.1  $d \leftarrow n \mbox{ (mod }\_\_prime\_tab_{ix}\mbox{)}$ \\
\hspace{3mm}1.2  If $d = 0$ then \\
\hspace{6mm}1.2.1  $c \leftarrow 1$ \\
\hspace{6mm}1.2.2  Return(\textit{MP\_OKAY}). \\
2.  $c \leftarrow 0$ \\
3.  Return(\textit{MP\_OKAY}). \\
\hline
\end{tabular}
\end{center}
\end{small}
\caption{Algorithm mp\_prime\_is\_divisible}
\end{figure}
\textbf{Algorithm mp\_prime\_is\_divisible.}
This algorithm attempts to determine if a candidate integer $n$ is composite by performing trial divisions.  

\vspace{+3mm}\begin{small}
\hspace{-5.1mm}{\bf File}: bn\_mp\_prime\_is\_divisible.c
\vspace{-3mm}
\begin{alltt}
\end{alltt}
\end{small}

The algorithm defaults to a return of $0$ in case an error occurs.  The values in the prime table are all specified to be in the range of a 
mp\_digit.  The table \_\_prime\_tab is defined in the following file.

\vspace{+3mm}\begin{small}
\hspace{-5.1mm}{\bf File}: bn\_prime\_tab.c
\vspace{-3mm}
\begin{alltt}
\end{alltt}
\end{small}

Note that there are two possible tables.  When an mp\_digit is 7-bits long only the primes upto $127$ may be included, otherwise the primes
upto $1619$ are used.  Note that the value of \textbf{PRIME\_SIZE} is a constant dependent on the size of a mp\_digit. 

\subsection{The Fermat Test}
The Fermat test is probably one the oldest tests to have a non-trivial probability of success.  It is based on the fact that if $n$ is in 
fact prime then $a^{n} \equiv a \mbox{ (mod }n\mbox{)}$ for all $0 < a < n$.  The reason being that if $n$ is prime than the order of
the multiplicative sub group is $n - 1$.  Any base $a$ must have an order which divides $n - 1$ and as such $a^n$ is equivalent to 
$a^1 = a$.  

If $n$ is composite then any given base $a$ does not have to have a period which divides $n - 1$.  In which case 
it is possible that $a^n \nequiv a \mbox{ (mod }n\mbox{)}$.  However, this test is not absolute as it is possible that the order
of a base will divide $n - 1$ which would then be reported as prime.  Such a base yields what is known as a Fermat pseudo-prime.  Several 
integers known as Carmichael numbers will be a pseudo-prime to all valid bases.  Fortunately such numbers are extremely rare as $n$ grows
in size.

\begin{figure}[!here]
\begin{small}
\begin{center}
\begin{tabular}{l}
\hline Algorithm \textbf{mp\_prime\_fermat}. \\
\textbf{Input}.   mp\_int $a$ and $b$, $a \ge 2$, $0 < b < a$.  \\
\textbf{Output}.  $c = 1$ if $b^a \equiv b \mbox{ (mod }a\mbox{)}$, otherwise $c = 0$.  \\
\hline \\
1.  $t \leftarrow b^a \mbox{ (mod }a\mbox{)}$ \\
2.  If $t = b$ then \\
\hspace{3mm}2.1  $c = 1$ \\
3.  else \\
\hspace{3mm}3.1  $c = 0$ \\
4.  Return(\textit{MP\_OKAY}). \\
\hline
\end{tabular}
\end{center}
\end{small}
\caption{Algorithm mp\_prime\_fermat}
\end{figure}
\textbf{Algorithm mp\_prime\_fermat.}
This algorithm determines whether an mp\_int $a$ is a Fermat prime to the base $b$ or not.  It uses a single modular exponentiation to
determine the result.  

\vspace{+3mm}\begin{small}
\hspace{-5.1mm}{\bf File}: bn\_mp\_prime\_fermat.c
\vspace{-3mm}
\begin{alltt}
\end{alltt}
\end{small}

\subsection{The Miller-Rabin Test}
The Miller-Rabin (citation) test is another primality test which has tighter error bounds than the Fermat test specifically with sequentially chosen 
candidate  integers.  The algorithm is based on the observation that if $n - 1 = 2^kr$ and if $b^r \nequiv \pm 1$ then after upto $k - 1$ squarings the 
value must be equal to $-1$.  The squarings are stopped as soon as $-1$ is observed.  If the value of $1$ is observed first it means that
some value not congruent to $\pm 1$ when squared equals one which cannot occur if $n$ is prime.

\begin{figure}[!here]
\begin{small}
\begin{center}
\begin{tabular}{l}
\hline Algorithm \textbf{mp\_prime\_miller\_rabin}. \\
\textbf{Input}.   mp\_int $a$ and $b$, $a \ge 2$, $0 < b < a$.  \\
\textbf{Output}.  $c = 1$ if $a$ is a Miller-Rabin prime to the base $a$, otherwise $c = 0$.  \\
\hline
1.  $a' \leftarrow a - 1$ \\
2.  $r  \leftarrow n1$    \\
3.  $c \leftarrow 0, s  \leftarrow 0$ \\
4.  While $r.used > 0$ and $r_0 \equiv 0 \mbox{ (mod }2\mbox{)}$ \\
\hspace{3mm}4.1  $s \leftarrow s + 1$ \\
\hspace{3mm}4.2  $r \leftarrow \lfloor r / 2 \rfloor$ \\
5.  $y \leftarrow b^r \mbox{ (mod }a\mbox{)}$ \\
6.  If $y \nequiv \pm 1$ then \\
\hspace{3mm}6.1  $j \leftarrow 1$ \\
\hspace{3mm}6.2  While $j \le (s - 1)$ and $y \nequiv a'$ \\
\hspace{6mm}6.2.1  $y \leftarrow y^2 \mbox{ (mod }a\mbox{)}$ \\
\hspace{6mm}6.2.2  If $y = 1$ then goto step 8. \\
\hspace{6mm}6.2.3  $j \leftarrow j + 1$ \\
\hspace{3mm}6.3  If $y \nequiv a'$ goto step 8. \\
7.  $c \leftarrow 1$\\
8.  Return(\textit{MP\_OKAY}). \\
\hline
\end{tabular}
\end{center}
\end{small}
\caption{Algorithm mp\_prime\_miller\_rabin}
\end{figure}
\textbf{Algorithm mp\_prime\_miller\_rabin.}
This algorithm performs one trial round of the Miller-Rabin algorithm to the base $b$.  It will set $c = 1$ if the algorithm cannot determine
if $b$ is composite or $c = 0$ if $b$ is provably composite.  The values of $s$ and $r$ are computed such that $a' = a - 1 = 2^sr$.  

If the value $y \equiv b^r$ is congruent to $\pm 1$ then the algorithm cannot prove if $a$ is composite or not.  Otherwise, the algorithm will
square $y$ upto $s - 1$ times stopping only when $y \equiv -1$.  If $y^2 \equiv 1$ and $y \nequiv \pm 1$ then the algorithm can report that $a$
is provably composite.  If the algorithm performs $s - 1$ squarings and $y \nequiv -1$ then $a$ is provably composite.  If $a$ is not provably 
composite then it is \textit{probably} prime.

\vspace{+3mm}\begin{small}
\hspace{-5.1mm}{\bf File}: bn\_mp\_prime\_miller\_rabin.c
\vspace{-3mm}
\begin{alltt}
\end{alltt}
\end{small}




\backmatter
\appendix
\begin{thebibliography}{ABCDEF}
\bibitem[1]{TAOCPV2}
Donald Knuth, \textit{The Art of Computer Programming}, Third Edition, Volume Two, Seminumerical Algorithms, Addison-Wesley, 1998

\bibitem[2]{HAC}
A. Menezes, P. van Oorschot, S. Vanstone, \textit{Handbook of Applied Cryptography}, CRC Press, 1996

\bibitem[3]{ROSE}
Michael Rosing, \textit{Implementing Elliptic Curve Cryptography}, Manning Publications, 1999

\bibitem[4]{COMBA}
Paul G. Comba, \textit{Exponentiation Cryptosystems on the IBM PC}. IBM Systems Journal 29(4): 526-538 (1990)

\bibitem[5]{KARA}
A. Karatsuba, Doklay Akad. Nauk SSSR 145 (1962), pp.293-294

\bibitem[6]{KARAP}
Andre Weimerskirch and Christof Paar, \textit{Generalizations of the Karatsuba Algorithm for Polynomial Multiplication}, Submitted to Design, Codes and Cryptography, March 2002

\bibitem[7]{BARRETT}
Paul Barrett, \textit{Implementing the Rivest Shamir and Adleman Public Key Encryption Algorithm on a Standard Digital Signal Processor}, Advances in Cryptology, Crypto '86, Springer-Verlag.

\bibitem[8]{MONT}
P.L.Montgomery. \textit{Modular multiplication without trial division}. Mathematics of Computation, 44(170):519-521, April 1985.

\bibitem[9]{DRMET}
Chae Hoon Lim and Pil Joong Lee, \textit{Generating Efficient Primes for Discrete Log Cryptosystems}, POSTECH Information Research Laboratories

\bibitem[10]{MMB}
J. Daemen and R. Govaerts and J. Vandewalle, \textit{Block ciphers based on Modular Arithmetic}, State and {P}rogress in the {R}esearch of {C}ryptography, 1993, pp. 80-89

\bibitem[11]{RSAREF}
R.L. Rivest, A. Shamir, L. Adleman, \textit{A Method for Obtaining Digital Signatures and Public-Key Cryptosystems}

\bibitem[12]{DHREF}
Whitfield Diffie, Martin E. Hellman, \textit{New Directions in Cryptography}, IEEE Transactions on Information Theory, 1976

\bibitem[13]{IEEE}
IEEE Standard for Binary Floating-Point Arithmetic (ANSI/IEEE Std 754-1985)

\bibitem[14]{GMP}
GNU Multiple Precision (GMP), \url{http://www.swox.com/gmp/}

\bibitem[15]{MPI}
Multiple Precision Integer Library (MPI), Michael Fromberger, \url{http://thayer.dartmouth.edu/~sting/mpi/}

\bibitem[16]{OPENSSL}
OpenSSL Cryptographic Toolkit, \url{http://openssl.org}

\bibitem[17]{LIP}
Large Integer Package, \url{http://home.hetnet.nl/~ecstr/LIP.zip}

\bibitem[18]{ISOC}
JTC1/SC22/WG14, ISO/IEC 9899:1999, ``A draft rationale for the C99 standard.''

\bibitem[19]{JAVA}
The Sun Java Website, \url{http://java.sun.com/}

\end{thebibliography}

\input{tommath.ind}

\end{document}


\end{document}


\end{document}


\end{document}
